%%%%%%%%%%%%%%%%%%%%%%%%%%%%%%%%%%%%%%%%%
% Inzane Syllabus Template
% LaTeX Template
% Version 1.2 (8.22.2019)
%
% This template has been downloaded from:
% http://www.LaTeXTemplates.com
%
% Original author:
% Carmine Spagnuolo (cspagnuolo@unisa.it) with major modifications by 
% Zane Wolf (zwolf.mlxvi@gmail.com)
%
% I (Zane) have left a lot of instructions both in the .tex file and the .cls file that can guide you to customize this document to suite your tastes and requirements. Here is a brief guide: 
%  - Changing the Main Color: .cls line 39
%  - Adding more FAQs: .cls line 126 and .tex line 99
%  - Adding TA emails: uncomment .cls lines 220 & 224 and .tex lines 85 and 89
%  - Deleting the FAQ sidebar entirely: .tex line 188
%  - Removing the Lab/TA Info and placing a brief Overview/About section in their place:        uncomment .tex line 91 and .cls line 227, and comment .cls lines for the LAB/TA info        that you no longer want (c. lines 184-227)

% I am also happy to help with crafting/designing modifications to this template to help suite your personal needs in a syllabus. Feel free to reach out! 
%
% License:
% The MIT License (see included LICENSE file)
%
%%%%%%%%%%%%%%%%%%%%%%%%%%%%%%%%%%%%%%%%%

%----------------------------------------------------------------------------------------
%	PACKAGES AND OTHER DOCUMENT CONFIGURATIONS
%----------------------------------------------------------------------------------------

\documentclass[letterpaper]{inzane_syllabus} % a4paper for A4

\usepackage{booktabs, colortbl, xcolor}
\usepackage{tabularx}
\usepackage{enumitem}
\usepackage{ltablex} 
\usepackage{multirow}

\setlist{nolistsep}

\usepackage{lscape}
\newcolumntype{r}{>{\hsize=0.9\hsize}X}
\newcolumntype{w}{>{\hsize=0.6\hsize}X}
\newcolumntype{m}{>{\hsize=.9\hsize}X}

\renewcommand{\familydefault}{\sfdefault}
\renewcommand{\arraystretch}{2.0}
%----------------------------------------------------------------------------------------
%	 PERSONAL INFORMATION
%----------------------------------------------------------------------------------------

\profilepic{fish.jpg} % Profile picture, if the height of the picture is less than that of the cirle, it will have a flat bottom. 

% To remove any of the following, you need to comment/delete the lines in the .cls file (c. line 186). Commenting/deleting the lines below will produce an error. 

%To add different lines, you will need to create the new command, e.g. \profPhone, in the .cls file c. line 76, and command to create the line in the side bar in the .cls file c. line 186

\classname{Logic I} 
\classnum{24.241, Fall 2024} 

%%%%%%%%%%%%%%% PROF INFO
\profname{Benjamin Brast-McKie}
\officehours{Office Hrs: Mon \& Wed 1-2p} 
\office{32-D962}
\siteA{\href{https://carnap.io}{Carnap Website}} 
\siteB{\href{https://canvas.mit.edu/courses/27891}{Canvas Website}} 
\email{brastmck@mit.edu}

%%%%%%%%%%%%%%% TA INFO
\taAname{Philipp Mayr}
\taAofficehours{Office Hrs: By Appointment}
\taAoffice{32-D927}
\TAemail{philmayr@mit.edu}
% \taBname{James}
% \taBofficehours{Office Hrs: Tues \& Thurs 3-4p}
% \taBoffice{MCZ 104}
% \taBemail{}

%%%%%%%%%%%%%%% COURSE INFO
\prereq{Prereq: None}
\classdays{Tues \& Thurs}
\classhours{9:30am - 11am}
\classloc{32-124}

%%%%%%%%%%%%%%% PROBLEM SET INFO
\labdays{Due Fridays}
\labhours{5pm sharp}
\labloc{Online or scanned}


% \about{Fish make up the largest group of vertebrates on the planet, easily outnumbering mammals, marsupials, birds, and reptiles combined. Not only are they abundant, but they've diversified into an extraordinary array of sizes, shapes, lifestyles, and habitats. You can find them in the coldest, deepest parts of the ocean, and in the hottest freshwater ponds in the desert. This course will explore fish diversity and their biology. } 


%---------------------------------------------------------------------------------------
%	 FAQs
%----------------------------------------------------------------------------------------
%to add more questions or remove this section, go to the .cls file and start with lines comment
%lines 226-250. Also comment out this section as well as line 152(ish), the command \makeSide

\qOne{Is logic math?}
\aOne{We will use formal symbols as in mathematics, but this does not make logic a type of math any more than it makes physics a type of mathematics. Rather, logic has a subject mater all its own, though saying what this is will require some care.}

\qTwo{Is logic philosophy?}
\aTwo{\textit{Philosophical Logic} includes the development and application of logical systems to the problems of philosophy as well as the ambition to provide a philosophical account of logic. \textit{Mathematical Logic} concerns the formal properties that logical systems have and falls considerably closer to mathematics. We will be doing a bit of both.}

\qThree{Is logic a descriptive science?}
\aThree{No! Logic is a \textit{normative} science insofar as it aims to regiment how we ought to reason in an artificial language, not merely how we happen to reason in a natural language like English.}

\qFour{Why learn logic?}
\aFour{Logic seeks to describe an ideal for reasoning. Of course, we are all engaged in reasoning. Learning logic is something akin to upgrading your firmware. It will literally change how you think.}

\qFive{What does logic have to show for itself?}
\aFive{Logic played a critical role in putting mathematics on a solid foundation (ZFC is accepted by most working mathematicians) and gave birth to the modern theory of computation as well as modern linguistics.}



%----------------------------------------------------------------------------------------

\begin{document}

%----------------------------------------------------------------------------------------
%	 DESCRIPTION
%----------------------------------------------------------------------------------------

\makeprofile % Print the sidebar

%----------------------------------------------------------------------------------------
%	 OVERVIEW
%----------------------------------------------------------------------------------------
\section{Overview}

During the first part of this course, we will study \textit{propositional logic} (PL). 
This is the logic of truth-functional connectives including: `not', `and', `or', `if-then', and `if and only if'. 
In the second part, we will study \textit{first-order logic} (FOL). 
This is the logic of `for all', `some', and later `is' together with the connectives from propositional logic. 

In both parts, we will present a syntax for a formal language which we will use to regiment arguments in natural language.
We will also introduce a model theoretic semantics in order to interpret our the formal languages and define logical consequence. 
Lastly, we will introduce a proof system for each of our formal languages, defining what it is to be a proof in that system.
For both parts of the course, we will cover basic metalogic, proving that our proof systems are sound and complete.

%----------------------------------------------------------------------------------------
%	 READING MATERIAL
%----------------------------------------------------------------------------------------
\vspace{0.5cm} %I make liberal use of the \vspace{} command to partition and place sections just how I want them. Alter as you see fit. 
\section{Required Text}

ForAllX: MIT Edition (Fall 2024). \\
Chapters will be released each week on Canvas.

%----------------------------------------------------------------------------------------
%	 GRADING SCHEME
%----------------------------------------------------------------------------------------
\vspace{0.5cm}
\section{Grading Scheme}

%below is the \twentyshort environment - a list with only two inputs. However, there is a \twenty environment, which creates a list with four inputs. You can find/alter details of that table in the .cls file c. lines 320. 
\begin{twentyshort}
	%\twentyitemshort{X\%}{Attendance/Participation}
  \twentyitemshort{50\%}{Problem Sets (10/12 at \%5 each)}
	\twentyitemshort{20\%}{Midterm}
  \twentyitemshort{30\%}{Final Exam}
\end{twentyshort}

Grades will not be curved.
A+ = 97-100; A = 93-96; A- = 90-92; B+ = 87-89;\\ 
B = 83-86; B- = 80-82; C+ = 77-79; C = 73-76; C- = 70-72; D = 60-69; F $<$ 60.

%----------------------------------------------------------------------------------------
%	 EXTRAS
%----------------------------------------------------------------------------------------

\vspace{0.5cm}
\section{Problem Sets}

There will be 12 graded problem sets due on Fridays by 5pm.
Your two lowest score will be dropped.
Late work will not be excepted.
Certain problem sets will make use of the online program called `Carnap'.
All work submitted MUST BE YOUR OWN, instantiating a direct causal relation to your own pen, pencil, or keyboard, written in your own voice without someone there with you or texting you answers.
As you will find, even logic leaves room for creativity, and it will be important to get a sense of that for yourself.
You are welcome to work through the problem set with at most two other students IN PREPARATION, but when it comes time to submit answers, you must be on your own.
If you choose to work with others, please indicate your collaborators' names at the top of each assignment.

\vspace{0.5cm}
\section{Carnap}

Parts of some of the problem sets will be assigned on Carnap.
You will need to \href{https://carnap.io/enroll/Logic%20I%20%282024%29}{\underline{enroll}} which you can find demonstrated \href{https://youtu.be/lmkWcxqxEZk}{\underline{here}}.
Using Carnap will require some syntactic care: it is fussy about how things are entered, so you will have to learn to use the right syntax.
Please don't hesitate to get in touch if you run into any issues.

\vspace{0.5cm}
\section{LaTeX}

There will also be written problem sets.
If your handwriting is very clear, you may scan and upload a PDF with your solutions.
Alternatively, you are encouraged to use \LaTeX\ where I will provide optional templates with examples so as to make it easy to typeset the problem set each week.
If you are new to \LaTeX, this is a great chance to begin to practice.
You can also find links to the configurations that I maintain for using \href{https://github.com/benbrastmckie/VSCodium}{\underline{VSCodium}} and \href{https://github.com/benbrastmckie/.config}{\underline{NeoVim}} to write in \LaTeX\ and Markdown.

%%%%%%%%%%%%%%%%%%%%%%%%%%%%%%%%%%%%%%%%%%%%%%%%%%%%%%%%%%%%%%%%%%%%%%%%%%%%%
%                SECOND PAGE
%%%%%%%%%%%%%%%%%%%%%%%%%%%%%%%%%%%%%%%%%%%%%%%%%%%%%%%%%%%%%%%%%%%%%%%%%%%%%

\newpage % Start a new page

\makeSide % Print the FAQ sidebar; To get rid of, simply comment out and uncomment \makeFullPage

% \makeFullPage

\vspace{0.5cm}
\section{Academic Integrity}

Blindly copying someone else's solution (written or typed) is cheating.
By contrast, you are encouraged to talk through a solution step-by-step with a classmate or two where in doing so, everyone involved comes to understand each part.
However, when it comes time to write up and submit the solutions, it is important that you do this for yourself without consulting others throughout the process.

Learning logic requires practice! 
Compare learning to speak a natural language like English. %, or skills like learning to walk on two legs or ride a bike.
There are some tricks and techniques to get acquainted with where once you gain some familiarity, this course should be fun.
But getting comfortable using these techniques takes time, making this a very difficult course to cram for the night before an exam.
The good news is that with consistent practice, you should be able to master the techniques of this course long in advance of the exam.

Doing problem sets is the best way to practice throughout the course.
Cheating on problem sets will be to your own disadvantage in preparing for the exams.
If you cheat on an exam, the academic consequences will be severe, so please don't consider it.
There is more to life than grades; don't let them distract from learning!

Instead of worrying about your grade, I recommend that you focus on mastering logic, doing your best work and feeling good about it.
Logic is an extremely deep subject, and foundational for this information age that we are all a part of.
This course should provide you with an important tool kit that I hope you enjoy learning to use and that will serve you well beyond the end of this semester.
% This has certainly been true for me!

\vspace{0.5cm}
\section{Make-up Policy}

Make-up exams or problem-sets are only permitted for students in the midst of a medical or family emergency.
Making arrangements IN ADVANCE of the due date is required except in particularly difficult circumstances.

\vspace{0.5cm}
\section{Learning Objectives}

%use \begin{outline} or \begin{outline}[enumerate] to create a list with subitems. 
\begin{itemize}
  \item Regiment natural language arguments in formal languages.
  \item Evaluate arguments for validity, producing minimal countermodels if any.
  \item Write formal proofs within a proof system.
  \item Develop an appreciation for meta-logical proofs about our proof systems and their corresponding semantic theories.
  \item Contemplate the philosophy of logic, exploring such questions as: What is logic? What unites logic as a discipline? What is logic good for, and what are its limits? Do the rules of logic describe something universal or conventional?
\end{itemize}

\vspace{0.5cm}
\section{Diversity and Inclusivity Statement}

In all course-related activities and communications, you will be treated with respect.
I welcome individuals of all ages, backgrounds, cultures, beliefs, ethnicities, gender identities and expressions, national origins, religious affiliations, abilities, sexual orientations, and other visible and non-visible differences.
All members of this class are expected to help create a respectful, welcoming, and inclusive environment for every other member of the class.

\vspace{0.5cm}
\section{Accommodations for Students with Disabilities}

If you are a student with learning needs that require accommodation, please contact Disability and Access Services at \texttt{das-student@mit.edu} (or for assistive technology, \texttt{atic-staff@mit.edu}) as soon as possible, to make an appointment to discuss your needs and to obtain an accommodations letter.
Please also e-mail me as soon as possible to set up a time to discuss your learning needs.
As someone who has used these services in the past, you can assume that you will have my full support.


%%%%%%%%%%%%%%%%%%%%%%%%%%%%%%%%%%%%%%%%%%%%%%%%%%%%%%%%%%%%%%%%%%%%%%%%%%%%%
%                COURSE SCHEDULE
%%%%%%%%%%%%%%%%%%%%%%%%%%%%%%%%%%%%%%%%%%%%%%%%%%%%%%%%%%%%%%%%%%%%%%%%%%%%%
\newpage
\makeFullPage
\section{Class Schedule}

Any changes to this schedule will be announced on Canvas.

\vspace{.25in}

\begin{center}
\begin{tabularx}{\textwidth}{p{2.5cm}p{8cm}p{9cm}} %change the width of the comments by changing these cm measurements. Add/substract columns by adding/deleting p{} sections. 
\arrayrulecolor{myCOLOR}\hline
%%%%%%%%%%%%%%%%%%%%%%%%%%%%%%%%%%%%%%%%%%% MODULE 1
\multicolumn{3}{l}{\textbf{\textcolor{myCOLOR}{\large Part 1: Propositional Logic (PL)}}} \\
\hline
% Week & Topic & Readings \\ \hline 
%%Alternatively, instead of Week #, you can do Class date for meeting

Week 0 & Introduction to Logic & ForAllX Ch. 0 \\
Sep 05 & & Problem Set 0 (meet Carnap --- not graded)\\
\arrayrulecolor{maingray}\hline

Week 1 & Syntax for PL \& Recursive Definitions & ForAllX Ch. 1\\
Sep 10, 12 & Regimentation in PL & Problem Set 1 Due Friday 9/13  \\
\arrayrulecolor{maingray}\hline

Week 2 & Semantics for PL & ForAllX Ch. 2 \\
  Sep 17, 19 & Logical Consequence \& Countermodels & Problem Set 2 Due Friday 9/20 \\
\arrayrulecolor{maingray}\hline

Week 3 & Natural Deduction System (PL) & ForAllX Ch. 3 \\
Sep 24, 26 & Natural Deduction Proofs & Problem Set 3 Due Friday 9/27 \\
\arrayrulecolor{maingray}\hline
~\\

\arrayrulecolor{myCOLOR}\hline
  \multicolumn{2}{l}{\textbf{\textcolor{myCOLOR}{\large Part 2: Metalogic (PL)}}} \\
\hline

Week 4 & Mathematical Induction & ForAllX Ch. 4 \\
Oct 1, 3 & Soundness & Problem Set 4 Due Friday 10/4\\
\arrayrulecolor{maingray}\hline

Week 5 & Completeness & ForAllX Ch. 5 \\
Oct 8, 10 & Midterm Review & Problem Set 5 Due Friday 10/11 \\
\arrayrulecolor{maingray}\hline

Week 6 & MIDTERM & 80 minute in class exam \\
Oct 17 & &  No Problem Set 6 \\
\arrayrulecolor{maingray}\hline
~\\


\arrayrulecolor{myCOLOR}\hline
\multicolumn{2}{l}{\textbf{\textcolor{myCOLOR}{\large Part 3: First-Order Logic (FOL) }}} \\
\hline

Week 7 & Syntax for FOL & ForAllX Ch. 7 \\
Oct 22, 24 & Regimentation in FOL &  Problem Set 7 Due Friday 10/25 \\
\arrayrulecolor{myCOLOR}\hline

Week 8 & Semantics for FOL & ForAllX Ch. 8 \\
Oct 29, 31 & Logical Consequence \& Countermodels & Problem Set 8 Due Friday 11/1 \\
\arrayrulecolor{maingray}\hline
 
Week 9 & Syntax FOL$^=$ & ForAllX Ch. 9 \\
Nov 5, 7 & Semantics for FOL$^=$ & Problem Set 9 Due Friday 11/9 \\
\arrayrulecolor{maingray}\hline

Week 10 & Natural Deduction System (FOL) & ForAllX Ch. 10 \\
Nov 12, 14 & Identity Rules (FOL$^=$) & Problem Set 10 Due Friday 11/15 \\
\arrayrulecolor{maingray}\hline

\newpage

\arrayrulecolor{myCOLOR}\hline
\multicolumn{2}{l}{\textbf{\textcolor{myCOLOR}{\large Part 4: Metalogic (FOL)}}} \\
\hline

Week 11 & Soundness (FOL$^=$) & ForAllX Ch. 11 \\
Nov 19, 21 & & Problem Set 11 Due Friday 11/22 \\
\arrayrulecolor{maingray}\hline

Week 12 & Completeness (FOL$^=$) & ForAllX Ch. 12 \\
Nov 26 & & Problem Set 12 Due Friday 11/29 \\
\arrayrulecolor{maingray}\hline

Week 13 & Completeness (FOL$^=$) & ForAllX Ch. 13 \\
Dec 3, 5 & Compactness & Problem Set 13 Due Friday 12/6 \\
\arrayrulecolor{maingray}\hline

Week 14 & Review for the Final & Handout \\
Dec 10 & & No Problem Set \\

\arrayrulecolor{maingray}\hline\\

\arrayrulecolor{myCOLOR}\hline
Week 15 & FINAL EXAM (3 hours) & Time \& Location TBD \\ 
\hline 
\end{tabularx}
\end{center}


%----------------------------------------------------------------------------------------

\end{document} 



