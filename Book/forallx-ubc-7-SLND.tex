%!TEX root = forallx-ubc.tex
\chapter{Natural Deduction Proofs in SL}
\label{ch.ND.proofs}

%JRH to-do: add in some \ellipsesline commands in the proofs 
%take out stuff in strategy about rules of replacement! not legal in SND! 

This chapter introduces a different proof system for Sentential Logic (SL) than the tree method.
The tree method has advantages and disadvantages.
One advantage of trees is that, for the most part, they can be produced in a purely mechanical way where no flash of insight is necessary.
Another advantage is that producing a complete open tree provides a recipe for constructing an interpretation that satisfies the root.
A disadvantage of the tree method is that trees do not provide an intuitive line of reasoning from the premises to the conclusion. 
Although a closed tree shows that the premises together with the negation of the conclusion leads to a contradiction, we don't learn what it is about the premises which lead to the conclusion.
Remember that logic is the study of logically valid arguments, and it would be nice if our proof system gave us a way to argue from the premises to the conclusion in logically valid ways that nevertheless resemble natural forms of reasoning.

Providing such a system of \define{natural deduction} (SD) for SL is the aim of the present chapter.
By contrast with the tree method, SD is intended to model human reasoning, illustrating the connections between various claims.
Consequently, working through a natural deduction proof requires a bit more insight than a tree proof does.
Although natural deduction proofs can be used to prove that an argument is valid, natural deduction system will not tell you if an argument is invalid, nor will it produce an interpretation that invalidates the argument.
That is, there is no equivalent to a completed open tree in natural deduction.

Despite these disadvantages, there is good reason to care about natural deduction systems. % independent of what metalogical properties they may be shown to have.
Recall that the only reason we provided for caring about SL tree proofs is that the tree proof system was shown to be sound and complete.
Accordingly, tree proofs can be used to determine something that we already care about, i.e., which SL arguments are logically valid.
By contrast, we may argue that the natural deduction rules included in SD are forced upon us by the meanings of the connectives included in SL.
For instance, given that $A\eand B$, one may deduce $A$ since what it is for a conjunction like $A\eand B$ to be true is for both of its conjuncts to be true, and so in particular we may conclude that $A$ is true. 
Given a compelling range of inference rules for each of the connectives in SL, one may ask whether it is possible to argue from the premises to the conclusion.
This is to ask whether an argument is \textit{deductively valid}, where knowing the answer to such a question holds interest independent of what metalogical properties the natural deduction system in question may be shown to have.

It will turn out that there is a natural deduction derivation in SD corresponding to every valid SL argument.
Put otherwise, SD is, like the tree method, complete.
Moreover, there are no derivations in SD of a conclusion that fails to be entailed by its premises, and so SD is also sound.
Nevertheless, when an argument is invalid, there is no way to use SD to see that it is valid.
Instead, we get stuck trying different ways to derive the conclusion from the premises, though in the case of an invalid argument, no such attempt will succeed.

In what follows, we will introduce ten basic derivation rules for our five logical connectives in SL.
Each connective will have an introduction and an elimination rule where, taken together, these rules may be taken to capture a certain dimension of the \textit{meaning} of that connective.
In particular, the introduction and elimination rules describe how to reason with each connective.
There is also a trivial rule which lets you reiterate an earlier line in the proof.
This corresponds to the idea that we can always conclude $A$ if $A$ is true.
However trivial, such an inference rule will turn out to have an important role to play in our proof system.

The following section will present the basic rules of inference included in SD.
Given these rules, we will be in a position to provide a precise definition of a proof in SD.
In contrast to trees, SD proofs are sequences of sentences satisfying certain properties.
Already there is something much more natural about sequences of sentences instead of trees.
After all, natural reasoning takes place in time, and time is linear.



% \section{Natural Deduction}
%
% The basic idea of a natural deduction proof is simple.
% You begin by writing down the premises that you are arguing from, numbering each in turn.
% We will then make use of the eleven rules alluded to above.
% If you can use the rules to derive a sentence from what you have already written down, then you can add this derived sentence on a new numbered line.
% If, following the rules, you manage to derive the conclusion from the premises, then you've shown that the argument from the premises to the conclusion is \textit{deductively valid}.
%
% We will provide a precise definition of a proof in SD once we have introduced the basic rules of inference.
% However, to get a sense of things, consider the following SL arguments:
%
% \begin{multicols}{2}
% \emph{Modus Ponens:}
% \begin{earg}
% \item $P \eif Q$
% \item $P$
% \item[\therefore] $Q$
% \end{earg}
%
% \emph{Disjunctive Syllogism:}
% \begin{earg}
% \item $P \eor Q$
% \item $\enot P$
% \item[\therefore] $Q$
% \end{earg}
%
% \end{multicols}
%
% Both are valid; you could confirm this with a four-line truth table.
% Either would demonstrate that there is no interpretation satisfying both premises, while falsifying the conclusion.
% The truth table method does not distinguish between these argument forms; it simply shows that they are both valid.
% There is, however, an interesting and important difference between these two argument forms, gestured at by their labels.
% The first argument, \emph{Modus Ponens}, has a distinctive syntactic form: its premises are a conditional and the antecedent of that conditional, and the conclusion is the consequent; the second has a different form.
%
% The natural deduction method is based on the recognition of particular kinds of valid forms.
% They also correspond reasonably well to familiar forms of informal reasoning.
% If you know a conditional, and you also know its antecedent, it is easy to infer its consequent.
% Imagine being sure that if I ate the chilli, I'll get sick, and also being sure that I ate the chilli.
% You will surely infer that I will get sick.
% \emph{Modus ponens} is the name of this kind of conditional reasoning, and there is a special rule for it in our natural deduction system.

\section{Premises and Assumptions}
\label{sec:PremiseAssumption}

Before introducing the rules, it will help to get a sense of what SD proofs look like in order to articulate some important constraints on the lines to which a rule may appeal.

A SD proof begins with a (possibly empty) list of premises, where these will be indicated by writing `:PR' on the right.
It is often helpful to also include a note of what we want to derive.
For instance, consider the following list of premises:
\begin{fitchproof}
  \hypo{1}{A \eif (B \eif C)} \pr{}
  \hypo{2}{A} \pr{}
\end{fitchproof}
% A line of a proof is \define{live} just in case it is not \define{dead}.
% It is important to take note of the vertical line which abuts the premises included above.
% Were we to apply one of our rules (in this case conditional elimination discussed below), we can only appeal to sentences which are \define{live} abut this vertical line, or abut a vertical line to the left.
% We will refer to such sentences as \define{live} and falling within the \define{scope} of application of our rules.
% In this case, there is only one vertical line where the sentences in the first two lines abut the vertical line, and so both sentences fall within the scope of the rules.
The horizontal line indicates where the premises end and the rest of the derivation begins.
For instance, we may apply conditional elimination (discussed below) to derive the following:
\begin{fitchproof}
  \hypo{1}{A \eif (B \eif C)} \pr{}
  \hypo{2}{A} \pr{}
  \have{3}{B \eif C} \ce{1,2}
\end{fitchproof}
Note that we appealed to lines $1$ and $2$ in order to derive line $3$, indicating as much in the justification of line $3$. 
If $B \eif C$ is all that we wanted to derive, then we would be done. 
However, suppose that we were to continue by adding a new assumption.
\begin{fitchproof}
  \hypo{1}{A \eif (B \eif C)} \pr{}
  \hypo{2}{A} \pr{}
  \have{3}{B \eif C} \ce{1,2}
  \open
    \hypo{4}{B} \as{}
  \close
\end{fitchproof}
At any point in a proof, we can introduce a new assumption, indenting the line on which the assumption occurs and starting a new vertical line.
We will refer to the process of adding an assumption as one of \define{opening a subproof}.
Subproofs are what they sound like: a proof within a proof, starting from a single assumption rather than the premises with which we began.
For instance, we might add the following lines to the proof above:
\begin{fitchproof}
  \hypo{1}{A \eif (B \eif C)} \pr{}
  \hypo{2}{A} \pr{}
  \have{3}{B \eif C} \ce{1,2}
  \open
    \hypo{4}{B} \as{}
    \have{5}{C} \ce{3,4}
    \have{6}{C \eand A} \ai{2,5}
  \close
  \have{7}{B \eif (C \eand A)} \ci{4-6}
\end{fitchproof}
Line $5$ applies conditional elimination (discussed below) on lines $3$ and $4$, and then line $6$ applies conjunction introduction (also discussed below) to lines $2$ and $5$.
We then close the subproof, where this may take place at any point, ending the vertical line and stepping back one level of indentation.
Once a subproof closes, the lines of that closed subproof are \define{dead}, and so cannot be appealed to individually.
Nevertheless, we may appeal to the subproof in its entirety, where line $7$ does just this, using conditional introduction (discussed below) which cites lines $4$--$6$ (note the hyphen in place of the comma).

Every line of a proof that is not dead is referred to as \define{live}, where rules that cite individual lines (as opposed to subproofs) can only appeal to lines that are live at that point in the proof.
For instance, were we to continue our proof a little further, we could not appeal to lines $4$, $5$, or $6$ since these lines are dead.
Thus we stipulate the following restriction:
  \factoidbox{
    To cite an individual line when applying a rule, the line must not occur within a subproof that has been closed before the line where the rule is being applied.
  }
Closing a subproof is also called \define{discharging} the assumption of that subproof.
Subproofs allow us to think about what we could show if we made a further assumption.
Accordingly, we have to be careful to keep track of what assumptions we are making and when it is and is not permitted to appeal to an assumption or the sentences that we can derive from that assumption.
Our proof system does this graphically by indenting and drawing vertical lines.
The details for each rule which makes use of this feature of our proof system will be discussed below, but it is important to have some sense of all of this before introducing the rules.





\section{Reiteration}
\label{sec:reiteration}

The first rule was already eluded to above.
Given any sentence on a live line of a proof, the \emph{reiteration rule} (R) allows you to repeat that sentence on a new line.
\begin{fitchproof}
	\have[$\vdots$]{}{\vdots}
	\have[4]{a1}{A \eand B}
	\have[$\vdots$]{}{\vdots}
	\have[10]{a2}{A \eand B} \by{R}{a1}
\end{fitchproof}
Given that we have written `$A \eand B$' on line $4$, we may repeat this sentence at some later line, e.g., line $10$.
We also add a citation which justifies what we have written.
In this case, we write `R', to indicate that we are using the reiteration rule, and we write `$4$', to indicate that we have applied it to line $4$.
Here is the general expression of the rule:
\factoidbox{
\begin{fitchproof}
	\have[m]{a}{\metaA}
	\have[\ ]{c}{\metaA} \by{R}{a}
\end{fitchproof}}
If $\metaA$ occurs on any line $m$ within the scope of application, we can reiterate $\metaA$, justifying this addition by writing `:$m$ R' to indicate that reiteration was applied to line $m$.
Of course, in an actual proof, the lines are numbered, and so $m$ will take on a numerical value.

Here is an example of three legal applications of rule R followed by an illegal application:

\begin{fitchproof}
		\hypo{r}{P} \pr{}
		\open
			\hypo{l}{\enot Q} \as{for funsies (we'll discuss assumptions later)}
			\have{rl}{\enot Q} \by{R (\textsc{legal})}{l}
			\have{n} {P} \r{r}
			\close
      \have{m} {\enot Q \eif P} \ci{l-n}
      \have{k} {P} \by{R (\textsc{legal})}{r}
    \have{con}{\enot Q}\by{R, (\textsc{illegal})}{l}
	\end{fitchproof}

% Once you have derived something from the premise, that new line is available to help justify future lines.
% It is important to observe that every line after the premises includes a justification starting with a colon `:' which is followed by the relevant line numbers and the derivation rule in question, in that order.
On the second line, we begin a subproof by assuming $\enot Q$.
You can reiterate $\enot Q$ within the subproof as in line $3$, but not when you leave the subproof as in line $7$.
On line $4$, we reiterate the sentence $P$ on line $1$, maintaining the same indentation.
We then close the subproof which we cite in line $5$.
At line $6$, we can reiterate line $1$ which is live, but at line $7$ we cannot appeal to line $2$ since this line is now dead.
Even if we were to open another subproof, we still could not appeal to line $2$.
Rather, the lines of a closed subproof are forever dead. 
Even so, this does not stop us from appealing to the subproof as a whole as we do in line $5$.



\section{Conjunction}

Consider the rule for \textit{conjunction introduction} ($\eand$I):

\factoidbox{
\begin{proof}
	\have[m]{a}\metaA{}
	\have[n]{b}\metaB{}
	\have[\ ]{c}{\metaA{}\eand\metaB{}} \ai{a, b}
\end{proof}}

This rule says that if any SL sentences $\metaA$ and $\metaB$ are on live lines, you may derive their conjunction $\metaA \eand \metaB$.
It is worth noting that $m$ and $n$ need not be consecutive lines, nor do they need to appear in the order listed.
We require only that each line has been established somewhere in the proof, and that both lines are live at the line of application.

Whereas conjunction introduction licenses the derivation of a conjunction from any two sentences, conjunction elimination lets us do the opposite.
Given any live conjunction, we may derive either of its conjuncts.
For instance, if $A \eand (P \eor Q)$ is live, we may derive $A$, or we may derive $P \eor Q$, but we must choose which.
If we wish to derive both, then two applications of the rule is required, though the order does not matter.

Here are the left and right \textit{conjunction elimination} ($\eand$E) rules:
\factoidbox{
\begin{proof}
	\have[m]{ab}{\metaA{}\eand\metaB{}}
	\have[\ ]{a}\metaA{} \ae{ab}
	\have[\ ]{b}\metaB{} \ae{ab}
\end{proof}}
These rules allow us to derive either conjunct.
Although we will often end up deriving both conjunts, we need not do so.
For instance, this is a perfectly acceptable proof:

\begin{proof}
	\hypo[1]{ab}{A \eand B} \pr{}
	\have[2]{b}B \ae{ab}
\end{proof}

Note that the {\eand}E rule only requires one sentence, so we write one line number in the justification.
Here is an example illustrating our two conjunction rules working together.
\begin{earg}
\item[] $[(A\eor B)\eif(C\eor D)] \eand [(E \eor F) \eif (G\eor H)]$
\item[\therefore] $[(E \eor F) \eif (G\eor H)] \eand [(A\eor B)\eif(C\eor D)]$
\end{earg}
The main logical operator in both the premise and conclusion is conjunction.
Since conjunction is symmetric, the argument is obviously valid since the two conjunctions have the same two conjuncts in the opposite order.
In order to provide a proof, we begin by writing down the premise on a numbered line indicating that it is a premise with PR.
After the premises, we draw a horizontal line where everything below this line must be justified by a proof rule.
\begin{proof}
	\hypo{ab}{{[}(A\eor B)\eif(C\eor D){]} \eand {[}(E \eor F) \eif (G\eor H){]}} \pr{}
\end{proof}
From the premise, we can separate the conjuncts with {\eand}E.
This yields the following proof.
\begin{proof}
	\hypo{ab}{{[}(A\eor B)\eif(C\eor D){]} \eand {[}(E \eor F) \eif (G\eor H){]}} \pr{}
	\have{a}{{[}(A\eor B)\eif(C\eor D){]}} \ae{ab}
	\have{b}{{[}(E \eor F) \eif (G\eor H){]}} \ae{ab}
\end{proof}
The {\eand}I rule requires that each of the conjuncts is live somewhere in the proof, though their order and distance from each other does not matter.
Applying the {\eand}I rule to lines $3$ and $2$, we arrive at the desired conclusion.

\begin{proof}
	\hypo{ab}{{[}(A\eor B)\eif(C\eor D){]} \eand {[}(E \eor F) \eif (G\eor H){]}} \pr{}

	\have{a}{{[}(A\eor B)\eif(C\eor D){]}} \ae{ab}
	\have{b}{{[}(E \eor F) \eif (G\eor H){]}} \ae{ab}
	\have{ba}{{[}(E \eor F) \eif (G\eor H){]} \eand {[}(A\eor B)\eif(C\eor D){]}} \ai{b,a}
\end{proof}

This proof may not look terribly interesting or surprising, but it shows how we can use the proof rules together to demonstrate the validity of an argument.
Note that using a truth table to show that this argument is valid would have required a staggering 256 lines, since there are eight sentence letters in the argument.
A tree proof would be less unwieldy than that, but would not have been as simple or natural of an argument.




\section{Disjunction}

Suppose Ludwig is reactionary.\footnote{This section has been adapted from the Calgary remix \S16.7.}
Then Ludwig is either reactionary or libertarian.
Trivial as this may seem, it speaks to the logic of disjunction.
Just as we may derive either conjunct from a conjunction, we may derive a disjunction from either of its disjuncts.

Thus the \textit{disjunction introduction} ($\eor$I) rule may be stated as follows:
\factoidbox{\begin{fitchproof}
	\have[m]{a}{\metaA}
	\have[\ ]{ab}{\metaA\eor\metaB}\oi{a}
	\have[\ ]{ba}{\metaB\eor\metaA}\oi{a}
\end{fitchproof}}
As above, the line $m$ must be live, where we cite this line in the justification of the rule application. 
Since $\metaB$ can be \emph{any} sentence, the following is a perfectly acceptable proof:
\begin{fitchproof}
	\hypo{m}{M} \pr{}
	\have{mmm}{M \eor ([(A\eiff B) \eif (C \eand D)] \eiff [E \eand F])}\oi{m}
\end{fitchproof}
Using a truth table to show this would have taken 128 lines.

The disjunction elimination rule is slightly trickier.
Suppose that either Ludwig is reactionary or he is libertarian.
It does not follow that Ludwig is reactionary, for he might be a libertarian.
Equally, we cannot conclude that Ludwig is libertarian, since he might be reactionary.
Given that we don't know which disjunct is true, it is difficult to deduce anything from a disjunction on its own.
The elimination rule for disjunction provides a workaround.

Suppose that we could show that if Ludwig's is reactionary, then he is an Austrian economist.
Suppose that we could also show that if Ludwig's is a libertarian, then he is also an Austrian economist.
Even though we don't know whether Ludwig is reactionary or a libertarian, it doesn't matter: in either case he is an Austrian economist.
This is a natural way to make use of a disjunction even when we don't know which disjunct is true.
Generalizing on this line of reasoning, consider the following \textit{disjunction elimination} ($\eor$E) rule:
\factoidbox{
	\begin{fitchproof}
		\have[m]{ab}{\metaA\eor\metaB}
		\open
			\hypo[i]{a}{\metaA} {} \as{for \eor E}
			\have[\vdots]{d1}{\vdots}
			\have[j]{c1}{\metaC}
		\close
		\open
			\hypo[k]{b}{\metaB}{} \as{for \eor E}
			\have[\vdots]{d2}{\vdots}
			\have[l]{c2}{\metaC}
		\close
		\have[ ]{c}{\metaC}\oe{ab, a-c1,b-c2}
	\end{fitchproof}}
This rule is somewhat clunkier to write down than our previous rules, but the point is fairly simple.
Suppose that we have some disjunction $\metaA \eor \metaB$.
Suppose that we can also construct two subproofs showing that $\metaC$ can be derived from the assumption that $\metaA$, and that $\metaC$ can be derived from the assumption that $\metaB$.
We can then infer $\metaC$ from the original disjunction $\metaA \eor \metaB$ together with our two subproofs.
As usual, there can be as many lines as you like between $i$ and $j$, and as many lines as you like between $k$ and $l$.
Moreover, the subproofs and the disjunction can come in any order, and do not have to be adjacent to each other as above.
Although the lines $i$--$j$ and $k$--$l$ belong to closed subproofs and so dead, line $m$ must be live.

Some examples will help illustrate.
Consider the following argument:
$$(P \eand Q) \eor (P \eand R) \; \therefore \; P$$
A proof might run like this:
	\begin{fitchproof}
		\hypo{prem}{(P \eand Q) \eor (P \eand R) } \pr{}
			\open
				\hypo{pq}{P \eand Q} \as{for \eor E}
				\have{p1}{P}\ae{pq}
			\close
			\open
				\hypo{pr}{P \eand R} \as{for \eor E}
				\have{p2}{P}\ae{pr}
			\close
		\have{con}{P}\oe{prem, pq-p1, pr-p2}
	\end{fitchproof}
Here is a slightly harder example.
Consider the following argument:
	$$ A \eand (B \eor C) \therefore (A \eand B) \eor (A \eand C)$$
We may then construct the following proof:
	\begin{fitchproof}
		\hypo{aboc}{A \eand (B \eor C)} \pr{}
		\have{a}{A}\ae{aboc}
		\have{boc}{B \eor C}\ae{aboc}
		\open
			\hypo{b}{B} \as{for \eor E}
			\have{ab}{A \eand B}\ai{a,b}
			\have{abo}{(A \eand B) \eor (A \eand C)}\oi{ab}
		\close
		\open
			\hypo{c}{C} \as{for \eor E}
			\have{ac}{A \eand C}\ai{a,c}
			\have{aco}{(A \eand B) \eor (A \eand C)}\oi{ac}
		\close
	\have{con}{(A \eand B) \eor (A \eand C)}\oe{boc, b-abo, c-aco}
	\end{fitchproof}
As natural as the rules may seem in isolation, it is not always obvious how to put them together to get from some premises to a conclusion.
The ability to construct SD proofs requires practice like any skill, and we'll cover some strategies for finding proofs at the end of the chapter.
Nevertheless, once a natural deduction proof has been constructed, each step is easy to justify, making the derivation in total impervious to doubts.
Moreover, this certainty does not stem from any semantic considerations.
Rather, the proof rules are directly justified by our intuitive understanding of how to use the logical connectives in our language.




\section{Conditional Introduction}

The rule for conditional introduction has already been deployed in the informal proofs in the previous chapter, and should have felt both compelling an familiar.
Here is an abbreviated version of the reasoning which showed up in the completeness proof:
	\begin{quote}
		Assume $\Gamma \nproves \bot$.
    Given the lemmas we have proven, it follows from this assumption that $\Gamma \nmodels \bot$.
    Although we don't know whether $\Gamma \nmodels \bot$ holds independent of our assumption, we may conclude that \textit{if} $\Gamma \nproves \bot$, \textit{then} $\Gamma \nmodels \bot$. 
	\end{quote}
The idea here is that you help yourself to something that you may not know is true, do some reasoning to arrive at some further claim, then conclude by asserting a conditional claim: if the assumption is true, then the further claim is true.

Here is a somewhat simpler version of the same style of reasoning.
	\begin{quote}
		Ludwig is reactionary. Therefore if Ludwig is libertarian, then Ludwig is both reactionary \emph{and} libertarian.
	\end{quote}
We may regiment this argument as a natural deduction proof by starting with one premise $R$ for `Ludwig is reactionary':
	\begin{fitchproof}
		\hypo{r}{R} \pr{}
	\end{fitchproof}
We may now make an additional assumption $L$ for `Ludwig is libertarian'.
In common parlance, we might say something like, ``suppose for the sake of argument,'' or in writing informal proofs, we might say, ``assume for conditional proof.''
However, in our proof system SD, we will indicate that we are adding an assumption by writing `AS' on the right, where it is often helpful to also include `for $\eif$Intro' as a note to yourself or your reader.
	\begin{fitchproof}
		\hypo{r}{R} \pr{}
		\open
			\hypo{l}{L} \as{for \eif Intro}{}
	\end{fitchproof}

Note that we are not claiming to have proved $L$ from line 1.
Accordingly, we do not write any justification for the additional assumption on line 2.
Rather, we have started a new subproof by indenting the sentence $L$ and starting a new vertical line.
We have also underlined $L$ since it is playing a role analogous to a premise in our new subproof. 

With this extra assumption in place, we are in a position to use $\eand$I from before.
	\begin{fitchproof}
		\hypo{r}{R} \pr{}
		\open
			\hypo{l}{L} \as{for \eif Intro}{}
			\have{rl}{R \eand L}\ai{r, l}
%			\close
%		\have{con}{L \eif (R \eand L)}\ci{l-rl}
	\end{fitchproof}
Given the assumption $L$, we have deduced $R \eand L$.
We may now discharge our assumption, closing the subproof and adding an appropriate conditional on the next line.
	\begin{fitchproof}
		\hypo{r}{R} \pr{}
		\open
			\hypo{l}{L} \as{for \eif Intro}{}
			\have{rl}{R \eand L}\ai{r, l}
			\close
		\have{con}{L \eif (R \eand L)}\ci{l-rl}
	\end{fitchproof}
Whereas the indented subproof carries out reasoning under the assumption of $L$, line $4$ reverts back to our original proof which carries out reasoning under the assumption of our single premise $R$.
Accordingly, we cannot conclude $R\eand L$ merely under the assumption of $R$ by writing $R\eand L$ at the original level of indenting.
Nevertheless, we can assert the conditional $L\eif(R\eand L)$ as given in $4$, justifying this line by referring to the entire subproof rather than to individual lines of our proof.
In this case, there are only two lines in the subproof, but in general there may be many more.
Even in the case where the subproof only consists of two lines, we must use a hyphen to indicate that we are citing a subproof instead of two lines.

Generalising on this pattern, consider the \textit{conditional introduction} rule ($\eif$I):
\factoidbox{
	\begin{fitchproof}
		\open
			\hypo[i]{a}{\metaA} \as{}
			\have[\vdots]{b}{\vdots}
			\have[k]{c}{\metaB}
		\close
		\have[\ ]{ab}{\metaA\eif\metaB}\ci{a-c}
	\end{fitchproof}}
Whereas the individual lines that proof rules appeal to are required to be live, subproofs consist of dead lines, but this does not prevent them from providing justification for a new line assuming that the beginning and end of the subproof contain the appropriate sentences.
In particular, we may justify a conditional claim with a subproof which begins with the antecedent and ends with the consequent, citing the subproof it is entirety.


\section{Conditional Elimination}

Many different arguments demonstrate the classic inference \emph{modus ponens}:

\begin{multicols}{3}
\begin{earg}
\item[] $P \eif \enot Q$
\item[] $P$
\item[\therefore] $\enot Q$
\end{earg}

\begin{earg}
\item[] $\enot P \eif (A \eiff B)$
\item[] $\enot P$
\item[\therefore] $A \eiff B$
\end{earg}

\begin{earg}
\item[] $(P \eor Q) \eif A$
\item[] $P \eor Q$
\item[\therefore] $A$
\end{earg}

\end{multicols}

The natural deduction system of this chapter will include a rule of inference corresponding to \emph{modus ponens} which goes by the name \textit{conditional elimination} ($\eif$E).
Here is the rule:
\factoidbox{
\begin{proof}
	\have[m]{ab}{\metaA{}\eif\metaB{}}
	\have[n]{a}\metaA{}
	\have[\ ]{b}\metaB{} \ce{ab,a}
\end{proof}}
What this rule says is that if you have a conditional $\metaA\eif\metaB$ on a live line number $m$, and you also have the antecedent $\metaA$ of that conditional on a live line $n$, you can write the consequent $\metaB$ on a new line.
In order to justify this inference, we will list the line numbers $m$ and $n$ in that order as well as `$\eif$E' to specify the rule.
Given the conditional elimination rule, we can prove that the arguments given above are valid.
Here are proofs of two of them:


\begin{multicols}{2}

\begin{proof}
	\hypo{if}{P \eif \enot Q} \pr{}
	\hypo{a}P \pr{}
	\have{c}{\enot Q} \ce{if,a}
\end{proof}


\begin{proof}
	\hypo{if}{(P \eor Q) \eif A} \pr{}
	\hypo{a}{P \eor Q} \pr{}
	\have{c}A \ce{if,a}
\end{proof}

\end{multicols}

Notice that these proofs share the same structure.
We start by listing the premises followed by a horizontal line, where subsequent lines will need to be derived with the rules.
We then apply the conditional elimination rule to get the conclusion, citing the appropriate lines.
One can produce more complicated proofs with the same rule.

\begin{earg}

\item[] $A$ 
\item[] $A \eif B$ 
\item[] $B \eif C$ 
\item[] $C \eif [\enot P \eiff (Q \eor R)]$ 
\item[\therefore] $\enot P \eiff (Q \eor R)$
\end{earg}

We begin by writing our four premises on numbered lines:

\begin{proof}
	\hypo{4}{A} \pr{}
	\hypo{1}{A \eif B} \pr{}
	\hypo{2}{B\eif C} \pr{}
	\hypo{3}{C \eif [\enot P \eiff (Q \eor R)]} \pr{(Want \enot $P \eiff (Q \eor R)$)}
%	\have{b} {B} \ce{1,4}
%	\have{c}{C} \ce{2,b}
%	\have{}{\enot P \eiff (Q \eor R))} \ce{3,c}
\end{proof}

The parenthetical off to the right is optional, but can help to keep track of the conclusion that we are attempting to establish.
The proof will be complete once we derive $\enot P \eiff (Q \eor R)$ by applying the rules to the premises or lines that result from doing so.
Since we cannot use conditional elimination to get to our desired conclusion directly from our premises, it is worth considering what we can do.
For instance, we can use conditional elimination on lines $1$ and $2$ to get $B$ on a new line, and then repeat using our new line together with line $3$ to get $C$ on yet another new line. 
Continuing in this manner gives us the following proof:

\begin{proof}
	\hypo{4}{A} \pr{}
	\hypo{1}{A \eif B} \pr{}
	\hypo{2}{B\eif C} \pr{}
	\hypo{3}{C \eif [\enot P \eiff (Q \eor R)]} \pr{want \enot P \eiff (Q \eor R)}
	\have{b} {B} \ce{4,1}
	\have{c}{C} \ce{2,b}
	\have{}{\enot P \eiff (Q \eor R))} \ce{3,c}
\end{proof}

Having derive line $5$ from lines $1$ and $2$, we may derive $6$ from $3$ and $5$, and then conclude by deriving $7$ from $4$ and $6$.
In general, each time that we appeal to earlier lines in a proof in order to apply a rule, we must check to see if those lines are live.
However, in this case, we have not introduced any assumptions, and so there is no risk that any lines fail to be live.

In order to see the conditional introduction and elimination rules work together, consider:
	$$P \eif Q,\ Q \eif R\ \therefore \ P \eif R$$
We start by listing the premises--- this much is automatic requiring no thinking whatsoever.
But now we have to think about where we are going, i.e., we want to conclude with the conditional $P\eif R$.
A great way to do this is by conditional introduction and so, to use this rule, we must begin by assuming the antecedent $P$ of the conditional we want to conclude.
\begin{fitchproof}
	\hypo{pq}{P \eif Q} \pr{}
	\hypo{qr}{Q \eif R} \pr{}
	\open
		\hypo{p}{P} \as{for \eif I}{}
	\close
\end{fitchproof}
Note that we may appeal to $P$ in the course of our subproof since so far it is still live.
Thus we may reason as follows:
\label{HSproof}
\begin{fitchproof}
	\hypo{pq}{P \eif Q} \pr{}
	\hypo{qr}{Q \eif R} \pr{}
	\open
		\hypo{p}{P} \as{for \eif I}
		\have{q}{Q}\ce{pq,p}
		\have{r}{R}\ce{qr,q}
	\close
	\have{pr}{P \eif R}\ci{p-r}
\end{fitchproof}
Whereas line $4$ derives $Q$ from lines $1$ and $3$ by conditional elimination, we may apply the same rule to derive $R$ on line $5$ from the lines $2$ and $4$.
Finally, we may close our subproof, and conclude $P \eif R$ on line $6$ while citing the subproof on lines $3$--$5$. 






\section{The Biconditional}

The rules for the biconditional will be like double-barrelled versions of the rules for the conditional.
In order to prove $F \eiff G$  you must be able to prove $G$ on the assumption $F$, and separately, prove $F$ on the assumption $G$.
The \textit{biconditional introduction} rule ({\eiff}I) therefore requires two subproofs.
Schematically, the rule looks like this: 
\factoidbox{
\begin{fitchproof}
	\open
		\hypo[i]{a1}{\metaA} \as{for \eiff I}
		\have[\vdots]{c1}{\vdots}
		\have[j]{b1}{\metaB}
	\close
	\open
		\hypo[k]{b2}{\metaB} \as{for \eiff I}
		\have[\vdots]{c2}{\vdots}
		\have[l]{a2}{\metaA}
	\close
	\have[\ ]{ab}{\metaA\eiff\metaB}\bi{a1-b1,b2-a2}
\end{fitchproof}}
There can be as many lines as you like between $i$ and $j$, and as many lines as you like between $k$ and $l$.
Moreover, the subproofs can come in any order, and the second subproof does not need to come immediately after the first.

The biconditional elimination rule ({\eiff}E) lets you do a bit more than the conditional rule.
If you have the left-hand subsentence of the biconditional, you can obtain the right-hand subsentence.
If you have the right-hand subsentence, you can obtain the left-hand subsentence.
\factoidbox{
\begin{fitchproof}
	\have[m]{ab}{\metaA\eiff\metaB}
	\have[n]{a}{\metaA}
	\have[\ ]{b}{\metaB} \be{ab,a}
\end{fitchproof}}
Equally, we may work in the reverse direction:
\factoidbox{\begin{fitchproof}
	\have[m]{ab}{\metaA\eiff\metaB}
	\have[n]{a}{\metaB}
	\have[\ ]{b}{\metaA} \be{ab,a}
\end{fitchproof}}
Note that in the citation for $\eiff$E, we always cite the biconditional first and either the left or right argument depending as the second argument.



\section{Negation}
Here is a simple mathematical argument in English:
\begin{earg}
\item[] Assume there is some greatest natural number, call it $n$.
\item[] Now consider its successor $n+1$ which is also a natural number.
\item[] Since $n+1 > n$, we may conclude that $n$ is not the greatest natural number. 
\item[] But this contradicts our assumption. 
\item[\therefore] Thus there is no greatest natural number.
\end{earg}

This kind of argument form is traditionally called a \emph{reductio} argument, or to use its full Latin name, \emph{reductio ad absurdum} which means ``reduction to absurdity.''
Proofs of this form are also sometimes called \textit{indirect proofs}.
A \emph{reductio} argument assumes something which we would like to show is false, or in common parlance, it assumes something \textit{for the sake of argument}, e.g., that there is a greatest natural number.
Then we show that the assumption leads to a contradiction of some kind.
For instance, we might end up reaching the negation of the \textit{reductio} assumption, or just reach two sentences of the form $\metaB$ and $\enot\metaB$.
Given such a contradiction, we may conclude that the original assumption must have been false.

In mathematics, \textit{reductio} arguments often lead to conclusions like $0=1$ that contradict something that is already known more generally though the negation $0\neq 1$ might not show up anywhere in the proof.
Whether stated or not, what is going on here is that we really have two contradictory claims: $0=1$ and $0\neq 1$, or to be even more explicit, $\enot(0=1)$.
Mathematical proofs typically suppress many of the obvious details, and so do not take the form of fully explicit valid arguments of the kind with which we will be concerned.

The basic rules for negation will allow for \textit{reductio} style arguments of the above form.
Like the conditional introduction rule (\eif I), our negation rules require us to make a new assumption on an indented line, drawing a new vertical line.
If this assumption can be shown to lead to contradictory sentences in a subproof, then we may write the negation of the assumption of this subproof on a new.
Schematically, this is the \textit{negation introduction} ({\enot}I) rule:

%try the  \ellipsesline command line to insert dots more easily!

\factoidbox{
\begin{proof}
\open
	\hypo[m]{na}\metaA{} \as{for \enot I}
	\have[n]{b}\metaB{}
	\have[o]{nb}{\enot\metaB{}}
\close
\have[\ ]{a}[\ ]{\enot\metaA{}}\ni{na-nb} %note that UBC has a more complex citation convention: {na-b, na-nb}
\end{proof}}

On line $m$, we assume $\metaA$ for \emph{reductio}.
Our goal is to derive a contradiction, represented by two sentences $\metaB$ and $\enot\metaB$ on separate lines in any order.
Accordingly, it is often convenient to include a note to ourselves and our readers that we are trying to introduce a negation by reaching a contradiction.
Observe that $\metaB$ could be the same sentence as $\metaA$, e.g. both could be P, but this need not always be the case. 
Once we have derived a contradictory pair of sentences, we may close the subproof, moving to the left one level of indentation.
We may then write the negation of the assumption in the subproof $\enot\metaA$ on a new line, citing the whole subproof by using a hyphen and indicating the negation introduction rule $\enot$I.

To see how the rule works, suppose that we want to derive the law of non-contradiction: $\enot(G \eand \enot G)$.
In general, if you want to conclude a negated sentence, it is natural to assume the negand and see if you can reach a contradiction, though this may not always be the best strategy.
However, in the case of $\enot (G \eand \enot G)$, this is just what we will do, starting a subproof by adding the assumption $G\eand \enot G$ to a proof without any premises.

\begin{proof}
	\open 
		\hypo{gng}{G\eand \enot G}\as{for $\enot$I}
		\have{g}{G}\ae{gng}
		\have{ng}{\enot G}\ae{gng}
	\close
	\have{ngng}{\enot(G \eand \enot G)}\ni{gng-ng}
\end{proof}

Although some proofs require some real creativity, this one is pretty obvious once we make the right assumption.
After all, the only rule we could apply to our assumption is {\eand}E, where two applications give us a contradiction.
By applying $\enot$I, we may conclude the proof.

The \textit{negation elimination} ($\enot$E) rule works in much the same way.
If we assume $\enot\metaA$ and show that it leads to a sentence and its negation, we may conclude $\metaA$.
So the rule looks like this:

\factoidbox{
\begin{proof}
\open
	\hypo[m]{na}{\enot\metaA{}}\as{for \enot E}{}
	\have[n]{b}\metaB{}
	\have[o]{nb}{\enot\metaB{}}
\close
\have[\ ]{a}[\ ]\metaA{}\ne{na-nb}
\end{proof}}

As in the case of negation introduction, it is important that the justification of an application of negation elimination cite an entire subproof, indicating as much with a hyphen.
Additionally, it is important that the contradictory pair of sentences occur in the subproof itself rather than elsewhere in the proof, even if live.
Below is an example which makes an essential appeal to the reiteration rule in order to make use of a negation elimination:

\begin{proof}
	\hypo{p}{P} \pr{}
	\hypo{qnp}{\enot Q \eif \enot P} \pr{want $Q$}
	\open
		\hypo{q}{\enot Q} \as{for $\enot$E}{}
		\have{np}{\enot P}\ce{qnp,q}
		\have{nnp}{P}\by{R}{p}
	\close
	\have{nq}{Q}\ne{q-nnp}
\end{proof}

Negation elimination requires that one show that some sentence and its negation are derivable given the assumption of a negated sentence.
In this case, we establish that $\enot P$ follows from the assumption that $\enot Q$ by conditional elimination.
Even though $P$ occurs on a live line, we must use the reiteration rule in order to include $P$ in our subproof.
Only then may we draw $Q$ as a conclusion by way of negation elimination. 

This concludes our discussion of the rules of SD.
In the following section we will move to discuss some of the mistakes that are easy to make in first learning to write SD proofs.



% \section{Exact Matches}
%
% Conditional elimination, as well as all of our other natural deduction rules, are syntactically defined.
% That is to say, the application of the rules depends on the exact shape of the SL sentences in question.
% Here, again, is the formal statement of the rule:
%
% \begin{proof}
% 	\have[m]{ab}{\metaA{}\eif\metaB{}}
% 	\have[n]{a}\metaA{}
% 	\have[\ ]{b}\metaB{} \ce{ab,a}
% \end{proof}
%
% It says that any time one has, on one line, a sentence made up of some sentence \metaA{}, followed by the `\eif' symbol, followed by some sentence \metaB{}, where one also has \metaA{} on another line, one may derive \metaB{}. This is the only pattern of inference that this rule permits. \metaA{} and \metaB{} can be any sentences of SL, but a line justified by Conditional Elimination must fit this pattern exactly. It is not enough that one can `just see' that a given sentence follows via a similar pattern of inference.
%
% For example, this is \emph{not} a legal derivation in our system:
%
% \begin{proof}
% 	\hypo{ab}{P \eif (A \eand B)} \pr{}
% 	\hypo{a}{P} \pr{}
% 	\have{b}{B}\by{ILLEGAL \eif E}{ab,a}
% \end{proof}
%
% The Conditional Elimination rule requires that the new sentence derived be the consequent of the conditional cited. But in this example, $B$ is not the consequent of $P\eif (A \eand B)$ --- $A \eand B$ is. It is true that $B$ obviously follows from $A \eand B$, but the Conditional Elimination rule doesn't allow you to derive things just because they obviously follow. (Neither does any other rule in our formal system.) To derive $B$ from these premises we'll need to use another rule. (In particular, we will want to use the Conjunction Elimination rule, given below.)
%
% To check to make sure you are applying the rules correctly, one good heuristic is to think about whether you are relying on the rule itself, or on your intuitive understanding of the meanings of the symbols we use in SL. Your intuitive understanding is a good way to think about which rules to use, but to check to make sure you're using the rules properly, think about whether the rules' exact formulations could explain why it is permissible to extend the derivation in the exact way you're working with. Pretend, for instance, that you have no idea what the `\eif' symbol means, but you do know that if you have two sentences joined by it on one line, and the first of those two sentences on another line, then you are allowed to copy down the second sentence exactly on a new line. This --- and no more --- is what the Conditional Elimination rule permits you to do. (This is what we mean when we say the rule is syntactically defined.) It would be pretty trivial to write a computer program to check to see whether a line is properly derived via Conditional Elimination (unlike us, computers are VERY good at following rules, since computers are defined via rules).


\section{Subproofs}

We have already made use of a number of subproofs.
However, using subproofs requires some care, and so this section will describe some of the risks involved.
Consider the following:
\begin{fitchproof}
	\hypo{a}{A} \pr{}
	\open
		\hypo{b1}{B} \as{for \eif I}
		\have{b2}{B} \by{R}{b1}
	\close
	\have{con}{B \eif B}\ci{b1-b2}
\end{fitchproof}
This is perfectly in keeping with the rules that we have laid down above, and it should not seem particularly surprising.
After all, $B \eif B$ is a tautology, and so follows from no premises.
Thus it is just as easy to derive $B \eif B$ from some starting premises.

But now suppose that tried to continue the proof as follows:
\begin{fitchproof}
	\hypo{a}{A} \pr{}
	\open
		\hypo{b1}{B} \as{for \eif I}
		\have{b2}{B} \by{R}{b1}
	\close
	\have{con}{B \eif B}\ci{b1-b2}
	\have{b}{B} \by{$\eif$E \textsc{(illegal)}}{b2,con}
	%\have [\ ]{x}{} 
\end{fitchproof}
%\by{to invoke $\eif$E}{con, b2}
If we were allowed to do this, we could derive any sentence from any other sentence.
However, if you tell me that Anne is fast (symbolized by $A$), we shouldn't be able to conclude that Queen Boudica stood twenty-feet tall (symbolized by $B$).
Thankfully we are prohibited from making this move since our rules only permit us to draw on live lines.
Once a subproof closes, the sentences in that proof are dead, and so we cannot appeal to them at a later point.
This does not mean that we can't appeal to their results.
For instance, we could appeal to $B \eif B$ on a later line. 
Indenting helps to keep track of what we can and can't appeal to in writing proofs since it is easy to see which subproofs are closed.
In particular, once you step back one level of indentation, the indented lines of the subproof above are dead, and so can only be cited by certain rules which appeal to an entire subproof of an appropriate form.

Once we have started thinking about what we can derive from additional assumptions, nothing stops us from asking what we can derive from adding even more assumptions.
Instead of doing this all at once the way that we may begin with many premises, we will do so by opening subproofs within subproofs.
For instance, here is a proof of contraposition, an inference that we relied on in proving soundness and completeness in the previous chapter:

\begin{proof}
  \hypo{qnp}{Q \eif P} \pr{want $\enot P\eif \enot Q$}  
  \open
    \hypo{a}{\enot P} \as{for $\eif$I}
    \open
      \hypo{q}{Q} \as{for $\enot$I}{}
      \have{np}{P}\ce{qnp,q}
      \have{nnp}{\enot P}\by{R}{a}
    \close
    \have{d}{\enot Q} \ni{q-nnp}
  \close
	\have{nq}{\enot P \eif \enot Q}\ci{a-d}
\end{proof}

Since we can't do anything with a conditional by itself, line $2$ introduces the assumption $\enot P$.
This is a natural choice given that we want to conclude $\enot P \eif \enot Q$.
Even so, there is not much more that we can do than before, and so we are forced to introduce yet another assumption $Q$ on line $3$. 
This is also a natural choice given that we would like to conclude $\enot Q$, and we know that we can use $\enot$I to do so if we reach a contradictory pair of sentences from assuming $Q$.
Given our assumptions, we may then derive $P$ in line $4$ by appealing to lines $1$ and $3$, both of which are live. 
Since line $2$ is also live, we may derive $\enot P$ on line $5$ by reiteration.
Closing the second subproof, we may justify $\enot Q$ on line $6$ by citing the lines $3$--$5$.
Now can close the first subproof, using these lines to justify $\enot P \eif \enot Q$ on line $7$. 


For contrast, here is a proof where things go awry:
\begin{fitchproof}
\hypo{a}{A} \pr{}
\open
	\hypo{b}{B} \as{for \eif I}
	\open
		\hypo{c}{C} \as{for \eif I}
		\have{ab}{A \eand B}\ai{a,b}
	\close
	\have{cab}{C \eif (A \eand B)}\ci{c-ab}
\close
\have{bcab}{B \eif(C \eif (A \eand B))}\ci{b-cab}
\have{bcab}{C \eif (A \eand B)}\by{$\eif$I \textsc{(illegal)}}{c-ab}
%\have [\ ]{x}{} \by{to invoke $\eif$I}{c-ab}
\end{fitchproof}
The problem is that the subproof that began with the assumption $C$ was under the assumption of $B$ on line $2$.
By lines $6$ and $7$, we have discharged the assumption $B$, and so are no longer asking what we could show if we assumed $B$.
Although it was perfectly legitimate to draw this same inference on line $5$, by the time we are at line $7$ we cannot appeal to lines $2$--$5$.

Here is one further mistake worth watching out for:
\begin{fitchproof}
\hypo{a}{A} \pr{}
\open
	\hypo{b}{B} \as{for \eif I}
	\open
		\hypo{c}{C} \as{for \eif I}
	\have{bc}{B \eand C}\ai{b,c}
	% \have{c2}{C}\ae{bc}
	\close
\close
  \have{bcab}{B \eif (B \eand C)}\by{$\eif$I (\textsc{illegal})}{b-bc}
%\have [\ ]{x}{} \by{}{b-c2}
\end{fitchproof}
Here we are trying to cite a subproof that begins on line $2$ and ends on line $4$, but the sentence on line $4$ depends not only on the assumption on line $2$, but also on one another assumption (line $3$) which we have not discharged at the end of the subproof.
Put otherwise, the subproof which starts by assuming $B$ does not end with a sentence at all, but rather ends with a subproof. 
Although we can close both subproofs at once, doing so wouldn't be strategic since line $5$ cannot then cite lines $2$--$4$ to justify $B \eif (B \eand C)$.

It is also worth stressing the difference between citing a single line, and citing a subproof with a further example.
In particular, when a rule requires you to cite a subproof, you cannot cite an individual line instead, nor \textit{vice versa}.
So for instance, this is incorrect:
\begin{fitchproof}
\hypo{a}{A} \pr{}
\open
	\hypo{b}{B} \as{for \eif I}
	\open
		\hypo{c}{C} \as{for \eif I}
	\have{bc}{B \eand C}\ai{b,c}
	\have{c2}{C}\ae{bc}
	\close
  \have{c3}{C}\by{R (\textsc{illegal})}{c-c2}
%\have [\ ]{x}{} \by{to invoke R}{c-c2}
\close
\have[7]{bcab}{B \eif C}\ci{b-c3}
\end{fitchproof}
Here, we have tried to justify $C$ on line $6$ by the reiteration rule, but we have done so by citing the subproof on lines $3$--$5$.
Although that subproof could in principle be cited on line $6$, the reiteration rule does not permit us to do so.
Rather, we could have used $\eif$I to derive $C \eif C$ while citing that subproof.
By contrast, the reiteration rule R requires you to cite an individual line that is live, so citing the entire subproof is not permissible.

However obvious these mistakes may seem, it can be tempting to bend the rules when writing natural deduction proofs.
This is like bending the rules while playing chess: you simply are no longer playing chess, but rather moving chess pieces around a boards in a manner that is no longer constrained by the rules of chess, or any other game for that matter.
So in writing your own proofs in SD, keep these rules in mind, sticking to them precisely.




\section{Proof Strategy}
\label{sec.SL.ND.strategy}

So far the examples have been simple, but perhaps you can already get a sense of the strategy that natural deduction proofs sometimes require.
For instance, although it is always permissible to open a subproof with any assumption, there is some strategy involved in picking a useful assumption.
Starting a subproof with an arbitrary, wacky assumption would just waste lines of the proof.
In order to obtain a conditional by {\eif}I, for example, it makes sense to assume the antecedent of the conditional in a subproof and see if you can derive the consequent.
This is an example of a good proof strategy.

It is also always permissible to close a subproof, discharging its assumptions.
However, it will not be helpful to do so until you have reached something useful.
Once the subproof is closed, you can only cite the entire subproof in a justification for a line following that subproof.
Those rules that call for a subproof, or multiple subproofs, require that the last line of the subproof is a sentence of some form or other.
For instance, you are only allowed to cite a subproof for $\eif$I if the line you are justifying is of the form $\metaA \eif \metaB$, $\metaA$ is the assumption of your subproof, and $\metaB$ is the last line of your subproof.
This constrains the strategies that one might hope to employ in attempting to construct proof in SD.

Getting good at natural deduction will take some practice.
The good news is that natural deduction proofs are a lot more interesting to construct than tree proofs, and a much more beneficial skill: practising natural deduction will streamline your reasoning well beyond the scope of this course, where the same cannot be said for constructing truth-tables or tree proofs.
Although there are no fail-safe methods, and certainly no substitute for practice, there are some general rules of thumb and strategies that are worth keeping in mind.


\paragraph{Work backwards from what you want.}
The ultimate goal is to derive the conclusion. Look at the conclusion and ask what the introduction rule is for its main logical operator. This gives you an idea of what should happen \emph{just before} the last line of the proof. Then you can treat this line as if it were your goal. Ask what you could do to derive this new goal.

For example: If your conclusion is a conditional $\metaA{}\eif\metaB{}$, plan to use the {\eif}I rule. This requires starting a subproof in which you assume \metaA{}. In the subproof, you want to derive \metaB{}.

\paragraph{Work forwards from what you have.}
When you are starting a proof, look at the premises and consider what implications they might have, or what you would need to derive in order to make use of the premises.
It can help to think about the elimination rules for the main operators of the premises, or the sentences that you have derived so far.

For example: If you have a conditional $\metaA\eif\metaB$, and you also have $\metaA$ on a line, $\eif$E is a pretty natural move to make.

% \paragraph{Change what you are looking at.}
% Replacement rules can often make your life easier. If a proof seems impossible, try out some different substitutions.
%
% For example: It is often difficult to prove a disjunction using the basic rules. If you want to show $\metaA{}\eor\metaB{}$, it is often easier to show $\enot\metaA{}\eif\metaB{}$ and use the MC rule.
%
% Some replacement rules should become second nature. If you see a negated disjunction, for instance, you should immediately think of DeMorgan's rule.

\paragraph{Repeat as necessary.}
A long proof is just a number of short proofs linked together, so you can fill the gap by alternately working back from the conclusion and forward from the premises.
Once you have decided how you might be able to get to the conclusion, ask what you might be able to do with the premises.
Then consider the target sentences again and ask how you might reach them.

\paragraph{Try a reductio when nothing else works.}
If you cannot find a way to show something directly, try assuming its negation and see where this leads.
Sometimes this can help unlock a proof, perhaps even leading you to a direct line of argument.

\paragraph{Persist.}
Try different things.
If one approach fails, then try something else.
In general, there are typically many ways to construct a proof.




\section{Proofs and Provability}

Given the natural deduction rules specified above, we may present the following definition:

\factoidbox{
  A natural deduction \define{proof} (or \define{derivation}) of a conclusion $\metaA$ from a set of premises $\Gamma$ in SD is any sequence of lines ending with $\metaA$ on a live line where every line in the sequence is either: (1) a premise in $\Gamma$; (2) a discharged assumption; or (3) follows from previous lines by one of the natural deduction rules for SD. 
}

An SL sentence $\metaA$ is \define{provable} (or \define{derivable}) from $\Gamma$ in SD just in case there is a natural deduction proof (derivation) of $\metaA$ from $\Gamma$ in SD, where we may write this: $\Gamma \proves \metaA$. 
% Whereas `$\Gamma \proves_{T} \metaA$' was shorthand for `$\Gamma, \enot \metaA \proves_{T} \bot$', 

Two sentences $\metaA$ and $\metaB$ are \define{provably equivalent} (or \define{interderivable}) if and only if both $\metaA\proves\metaB$ and $\metaB\proves\metaA$.
% Additionally, we will take $\Gamma\proves\bot$ to mean that $\Gamma\proves\metaB\eand\enot\metaB$ for some SL sentence $\metaB$.
% It is relatively easy to show that two sentences are provably equivalent since this requires a pair of proofs.
% Showing that sentences are \emph{not} provably equivalent would be much harder.
% It would be just as hard as showing that a sentence is not a theorem.
% (In fact, these problems are interchangeable. Can you think of a sentence that would be a theorem if and only if \metaA{} and \metaB{} were provably equivalent?)
A set of sentences $\Gamma$ is \define{provably inconsistent} if and only if $\Gamma\proves\bot$ where $\bot$ is our arbitrarily chosen contradiction, e.g., $A\eand\enot A$.


Provability is relative to a proof system, so the meaning of the `$\proves$' symbol featured in this chapter should be distinguished from the one we used for trees.
When necessary, we can specify which turnstile we intend by including a reference to the proof system in question, letting `$\proves_{T}$' stand for provability in the tree system, and `$\proves_{SD}$' stand for provability in this natural deduction system.
Since this chapter is concerned with natural deduction, you can take `$\proves$' to mean `$\proves_{SD}$' in this chapter unless specified otherwise.
% For instance, it is important to disambiguate the turnstiles in making the following observation: whereas $\Gamma \proves_{T} \metaA$ was defined as $\Gamma, \enot \metaA \proves_{T} \bot$ for the tree proof system, here we have defined $\Gamma \proves_{SD} \bot$ in terms of $\Gamma \proves_{SD} \metaB\eand\enot\metaB$ for some $\metaB$. 

% So $\Gamma\proves\metaA$ can be read as: $\metaA$ is derivable from $\Gamma$.
As above, it is often convenient to write `$\metaA_1,\metaA_2,\ldots\proves\metaB$' as a shorthand for $\set{\metaA_1,\metaA_2,\ldots}\proves\metaB$.
We will refer to an SL sentence $\metaA$ as a \define{theorem} of SD just in case $\proves\metaA$, i.e., $\varnothing\proves\metaA$.
It is important to note that the sentences of SL are only theorems \textit{relative to a proof system}, so there is no sense in which $A\eor\enot A$ is a theorem full stop.
Nevertheless, it is natural to expect $A\eor\enot A$ to be a theorem of any respectable (i.e., complete) proof system for SL.

In order to show that something is a theorem we have to derive it from no premises.
But how could we show that something is \emph{not} a theorem?
More generally, how could we show that $\Gamma \nproves \metaA$?
Showing that there is no proof of $\metaA$ from $\Gamma$ would seem to require searching the space of all natural deduction proofs, and this is not bound to be easy. 
For instance, even if you (or a computer program) failed to derive $\metaA$ from $\Gamma$ in a thousand different ways, perhaps their is a proof that has not yet been considered.
As we've emphasized already, this brings out an important difference between our natural deduction system and the tree proof system.
Whereas the tree method will determine whether $\Gamma \proves_{T} \metaA$ or not, the natural deduction system does not provide a straightforward way to demonstrate that an argument is invalid.

% It is easy to show that a set is provably inconsistent: You just need to assume the sentences in the set and prove a contradiction. Showing that a set is \emph{not} provably inconsistent will be much harder. It would require more than just providing a proof or two; it would require showing that proofs of a certain kind are \emph{impossible}.





\section{Provability and Entailment}

The double turnstile `$\models$' for logical entailment remains unchanged, and so no subscript is required.
% Put otherwise, entailment is not relative to a proof system, but rather to a class of interpretations.
% In particular, we were concerned with interpretations which assigned the sentence letters of SL to exactly one truth-value.
% Since this is the only definition of an interpretation that we will consider for SL, there is only one corresponding notion of entailment $\models$.
Although the soundness and completeness proofs showed that $\Gamma \proves_{T} \metaA$ if and only if $\Gamma \models \metaA$, a question remains as to whether a similar result may be established for SD.
% A natural strategy for establishing this connection is to show that $\Gamma \proves_{SD} \metaA$ if and only if $\Gamma \proves_{T} \metaA$.

Although the soundness proof for natural deduction is easy, the completeness proof is considerably more difficult.
Rather, we will simply state that SD is both sound and complete:

\factoidbox{
  SD is both sound and complete, i.e., $\Gamma \proves_{SD} \metaA$ if and only if $\Gamma \models \metaA$.
}

Given the soundness and completeness of the tree method, it follows that our two proof systems are equivalent in the sense that anything provable in one is also provable in the other: $\Gamma\proves_{T}\metaA$ if and only if $\Gamma\proves_{SD}\metaA$.
In practice, this is a good thing.
For instance, suppose you are not sure if $\Gamma \proves_{SD} \metaA$, but if it does, you would like to provide a derivation.
Using the tree method, you could check whether $\Gamma \proves_{T} \metaA$, and if the tree you construct closes, then go on to look for a natural deduction proof of $\metaA$ from $\Gamma$. 
If we can show that $\Gamma \nproves_{T} \metaA$ by constructing a complete open tree, then we can conclude that there is no natural deduction derivation of $\metaA$ from $\Gamma$, and so no use looking for one. 
Nevertheless, a direct proof of $\metaA$ from $\Gamma$ in SD is more compelling than a closed tree.
Whereas it is easy to justify every move in a natural deduction derivation, it is a lot harder to use a closed tree to create a compelling argument.

Given that SD is sound and complete, the \emph{theorems} of SD are coextensive with the \emph{tautologies} of SL.
Whereas showing that a sentence is a tautology involves checking a truth-table or constructing a semantic argument, showing that a sentence is a theorem involves constructing a proof.
Some proofs are easier to construct than others, but even for very long proofs, it is easy to check whether each line follows from the previous lines.
% Moreover, the resulting argument is the kind of argument that you could present an analogue of in English in order to be persuasive.
Although there is no formal way of showing that a sentence is \emph{not} a theorem, there is a method for showing that a sentence is not a tautology: construct an SL tree and read off the interpretation.
% We need only construct a model in which the sentence is false. Given a choice between showing that a sentence is not a theorem and showing that it is not a tautology, it would be easier to show that it is not a tautology.


\section{Logical Analysis}

If we can translate an argument into SL, we now have a number of tools that we can use to investigate that argument's logical properties.
Consider the following options:

\begin{table}[h]
\begin{center}
\begin{tabular*}{\textwidth}{p{12em}|p{10em}|p{12em}|}
\cline{2-3}

 & \multicolumn{1}{|c|}{YES} & \multicolumn{1}{|c|}{NO}\\
\cline{2-3}

Is \metaA{} a tautology? & Prove $\proves\metaA{}$ & Give a model in which \metaA{} is false\\
\cline{2-3}

Is \metaA{} a contradiction? &  Prove $\proves\enot\metaA{}$ & Give a model in which \metaA{} is true\\
\cline{2-3}

Is \metaA{} contingent? & Give a model in which \metaA{} is true and another in which \metaA{} is false & Prove $\proves\metaA{}$ or $\proves\enot\metaA{}$\\
\cline{2-3}

Are \metaA{} and \metaB{} equivalent? & Prove \mbox{$\metaA{}\proves\metaB{}$} and \mbox{$\metaB{}\proves\metaA{}$}  & Give a model in which \metaA{} and \metaB{} have different truth values\\
\cline{2-3}

Is the set $\Gamma$ consistent? & Give a model in which all the sentences in $\Gamma$ are true & Taking the sentences in $\Gamma$, prove $\metaB$ and $\enot\metaB$\\
\cline{2-3}

Is `$\metaA, \metaB, \ldots\ \therefore\ \metaC$' valid? & Prove $\metaA{}, \metaB{}, \ldots \proves\metaC{}$ & Give a model that satisfies $\set{\metaA, \metaB, \ldots,\ \enot \metaC}$\\
\cline{2-3}
\end{tabular*}
\end{center}
% \caption{Sometimes it is easier to show something by providing proofs than it is by providing models. Sometimes it is the other way round.  It depends on what you are trying to show.}
\label{table.ProofOrModel}
\end{table}
% \FloatBarrier

Answering the questions on the left may sometimes be easy, and other times far from obvious.
Whereas the tree proof system provides a fail-safe way to determine the answer, the SD proof system provides a way to construct compelling arguments.





% \iffalse

\section{Derived Rules}
\label{sec:basic}

A \define{derived rule} is a rule of inference that has not been included as a basic rule of the proof system but may be derived using the basic rules included in that system.
Accordingly, anything that can be proven with a derived rule can be proven without it.

You can think of derived rules as subroutines which can be used to shorten proofs, making some proofs easier to write and more intuitive to read.
Like the basic rule, derived rules will use meta-variables since they hold for any sentences whatsoever.


%\section{Basic and derived rules}


%We have so far introduced five rules: Conditional Elimination, \emph{modus tollens}, Disjunction Elimination, Conjunction Elimination, and Conjunction Introduction. There are still more rules still to learn, but it is helpful first to pause and draw a distinction between different kinds of rules.

%Many of our rules, we have seen, carry the name `Elimination' or `Introduction', along with the name of one of our SL connectives. In fact, every rule we've seen so far except \emph{modus tollens} has had such a name. Such rules are the \emph{basic} rules in our natural deduction system. The basic rules comprise an Introduction and an Elimination rule for each connective, plus one more rule. \emph{Modus tollens} is NOT a basic rule; we will call it a \emph{derived} rule.

%A derived rule is a non-basic rule whose validity we can derive using basic rules only. 

%We have already seen the Introduction and Elimination rules for conjunction, and the elimination rules for disjunction and conditionals. In the next several sections, we'll finish canvassing the basic rules, then say a bit more about \emph{modus tollens} and other derived rules.

\subsection{\textit{Modus Tollens}}

Modus tollens is an extremely important and common inference rule in ordinary reasoning.
Here is the derived rule for \textit{modus tollens} (MT):

\begin{proof}
	\have[m]{ab}{\metaA{}\eif\metaB{}}
	\have[n]{a}{\enot\metaB{}}
	\have[\ ]{b}{\enot\metaA{}} \by{MT}{ab,a}
\end{proof}

If you have a conditional on one numbered line and the negation of its consequent on another line, you may derive the negation of its antecedent on a new line.
We abbreviate the justification for this rule as `MT' for \emph{modus tollens}.
For instance, if you know that if Sue found the treasure, then she is happy, and you also know that Sue isn't happy, then you can infer that Sue didn't find the treasure.
Inferences of this form are extremely common.

In order to derive MT from our basic rules we will construct a derivation in the manner above while using meta-variables instead of sentences of SL to provide the following proof schema:

\begin{proof}
	\hypo{p}{\metaA{}}
	\hypo{qnp}{\metaB{} \eif \enot \metaA{}} \want{\enot \metaB{}}
	\open
		\hypo{q}{\metaB{}} \by{:AS for $\enot$I}{}
		\have{np}{\enot \metaA{}}\ce{qnp,q}
		\have{nnp}{\metaA{}}\by{R}{p}
	\close
	\have{nq}{\enot \metaB{}}\ni{q-nnp}
\end{proof}

Since \metaA{} and \metaB{} are meta-variables, the lines above do not constitute an SD proof.
Rather, the lines above are a \define{proof schema} which is a kind of recipe for constructing proofs.
Given any values for $\metaA$ and $\metaB$, the proof schema for MT will yield an SD proof as an instance.
Accordingly, applications of MT can always be replaced with an appropriate instance of the proof schema for MT which only refers to the basic rules included in SD.
Nevertheless, MT is a convenient shortcut and so we will add it to our list of derived rules.

Here is an example employing \textit{modus tollens} several times over.

\begin{proof}
	\hypo{ab}{A \eif B}
	\hypo{bc}{B \eif C}
	\hypo{cd}{C \eif D}
	\hypo{nd}{\enot D} \want{\enot A}
	\have{c}{\enot C} \by{MT}{cd,nd}
	\have{b}{\enot B} \by{MT}{bc,c}
	\have{a}{\enot A} \by{MT}{ab,b}
\end{proof}

At each of lines 5--7, we cite a conditional and the negation of its consequent to infer the negation of its antecedent.
Without citing MT, this proof would be much more cumbersome.


\subsection{Dilemma}

One of the most difficult proof rules to apply is disjunction elimination, and so it will be convenient to derive proof rules that streamline arguments from disjunctive sentences.
Consider the \textit{dilemma rule} (DL):

\begin{proof}
	\have[m]{ab}{\metaA{}\eor\metaB{}} 
	\have[n]{ac}{\metaA{}\eif\metaC{}}
	\have[o]{bc}{\metaB{}\eif\metaC{}}
	\have[\ ]{c}{\metaC{}} \by{DL}{ab,ac,bc}
\end{proof}

If you know that two conditionals are true, and they have the same consequent, and you also know that one of the two antecedents is true, then conclusion is true no matter which antecedent is true.
We may derive this rule as follows:

\begin{proof}
	\hypo{ab}{\metaA{}\eor\metaB{}}
	\hypo{ac}{\metaA{}\eif\metaC{}}
	\hypo{bc}{\metaB{}\eif\metaC{}}\by{want \metaC{}}{}
	\open
		\hypo{nc}{\metaA}\as{}
		\have{na}{\metaC}\ce{ac,nc}
  \close
  \open
    \hypo{b2}\metaB\as{}
    \have{c2}{\metaC}\ce{bc,b2}
  \close
	\have{c}{\metaC} \oe{ab,nc-na,b2-c2}
\end{proof}

Whereas $\eor$E cites subproofs, DL only appeals to live lines in a proof, and so may be easier to apply in certain contexts.
For example, suppose you know all of the following:

\begin{earg}
\item[] If it is raining, the car is wet.
\item[] If it is snowing, the car is wet.
\item[] It is raining or it is snowing.
\end{earg}

From these premises, you can definitely establish that the car is wet.
This is the form that DL captures, nicely resonating with a common form of reasoning.

As in the case of MT, the DL rule doesn't allow us to prove anything we couldn't prove via basic rules.
Anytime you wanted to use the DL rule, you could always include a few extra steps to prove the same result without DL.
Nevertheless, DL captures an natural form of reasoning in its own right, and so is well worth including in our stock of derived rules.




\subsection{Disjunctive Syllogism}

Although DL is occasionally useful, there other common forms of reasoning from a disjunction which DL does not capture.
In particular, consider the following argument.

\begin{earg}
\item[] $P \eor Q$
\item[] $\enot P$
\item[\therefore] $Q$
\end{earg}

Even small children and non-human animals can engage in reasoning of the form given above.
For instance, if a ball is under one of two cups but you don't know which, and then it is revealed that it is not under one of the cups, it is natural to conclude that the ball must be under the other cup.
This inference is called \textit{disjunctive syllogism} (DS):

\begin{multicols}{2}

\begin{proof}
	\have[m]{ab}{\metaA{}\eor\metaB{}}
	\have[n]{nb}{\enot\metaB{}}
	\have[\ ]{a}\metaA{} \by{DS}{ab,nb}
\end{proof}

\begin{proof}
	\have[m]{ab}{\metaA{}\eor\metaB{}}
	\have[n]{na}{\enot\metaA{}}
	\have[\ ]{b}\metaB{} \by{DS}{ab,nb}
\end{proof}

\end{multicols}

We represent two different inference patterns here, because the rule allows you to conclude \emph{either} disjunct from the negation of the other.
Nevertheless, both go by the same name as is the case for other symmetrical rules like $\eand$E.

Consider the following argument:

\begin{earg}
\item[] $\enot L \eif (J \eor L)$
\item[] $\enot L$
\item[\therefore] $J$
\end{earg}

It is easy to see that $J\eor L$ follows by $\eif$E from the two premises, but it is difficult to see how the proof will go next were we constrained to our basic rules.
However, given DS, it is plain to see that $J$ follows immediately from $J\eor L$ and $\enot L$. 
So the proof is easy:

\begin{proof}
	\hypo{c}{\enot L \eif (J \eor L)}
	\hypo{a}{\enot L}  \want {J}
	\have{3}{J \eor L} \ce{c,a}
	\have{4}{J} \by{DS}{a,3}
\end{proof}

Like DL, the derived rule DS makes it easier to derive proofs.




\subsection{Hypothetical Syllogism}

We also add \textit{hypothetical syllogism} (HS) as a derived rule:

\begin{proof}
	\have[m]{ab}{\metaA\eif\metaB}
	\have[n]{bc}{\metaB\eif\metaC}
	\have[\ ]{ac}{\metaA\eif\metaC}\by{HS}{ab,bc}
\end{proof}

Note that HS does not cite any subproofs, and so makes for elegant proofs that are easy to read.
The same cannot be said for the proof schema for HS:

\begin{proof}
	\hypo[1]{ab}{\metaA\eif\metaB}
	\hypo[2]{bc}{\metaB\eif\metaC}
  \open 
    \hypo[3]{a}{\metaA} \as{}
    \have[4]{b}{\metaB} \ce{ab,a}
    \have[5]{c}{\metaC} \ce{bc,b}
  \close
  \have[6]{ac}{\metaA\eif\metaC} \ci{a-c}
\end{proof}


\subsection{Contraposition}

We also add \textit{contraposition} (CP) as a derived rule:

\begin{proof}
	\have[m]{ab}{\metaA\eif\metaB}
	\have[\ ]{ba}{\enot\metaA\eif\enot\metaB}\by{CP}{ab}
\end{proof}

Not only is this inference natural, it is extremely useful.
We have had various occasions to use CP in the informal proofs in the previous chapter.
The derivation goes as follows:

\begin{proof}
	\hypo[1]{ab}{\metaA\eif\metaB}
  \open 
    \hypo[2]{nb}{\enot\metaB} \as{}
    \open
    \hypo[3]{a}{\metaA} \as{}
    \have[4]{b}{\metaB} \ce{ab,a}
    \have[5]{xb}{\enot\metaB} \r{nb}
    \close
  \have[6]{na}{\enot\metaA} \ni{a-xb}
  \close
  \have[7]{ba}{\enot\metaB\eif\enot\metaA} \ci{nb-na}
\end{proof}

Whereas the proof above involves two subproofs, one embedded in the other, applications of CP directly cite live lines of a proof, greatly simplifying the resulting argument.





\subsection{Negative Biconditionals}

Biconditional elimination only works when we have a biconditional together with one of the arguments of the biconditional on live lines.
However, it in cases where we have the negation of one of the arguments of a biconditional, it is convenient to make use of the following derived rule for \textit{negative biconditionals} (NB):

\begin{multicols}{2}

\begin{proof}
	\have[m]{ab}{\metaA\eiff\metaB}
	\have[n]{na}{\enot\metaA}
	\have[\ ]{nb}{\enot\metaB}\by{NB}{ab,na}
\end{proof}

\begin{proof}
	\have[m]{ab}{\metaA\eiff\metaB}
	\have[n]{nb}{\enot\metaB}
	\have[\ ]{na}{\enot\metaA}\by{NB}{ab,nb}
\end{proof}

\end{multicols}

The derivations for NB go as follows:

\begin{multicols}{2}

\begin{proof}
	\hypo[1]{ab}{\metaA\eiff\metaB}
	\hypo[2]{na}{\enot\metaA}
  \open 
    \hypo[3]{b}{\metaB} \as{}
    \have[4]{a}{\metaA} \be{ab,b}
    \have[5]{xa}{\enot\metaA} \r{na}
  \close
  \have[6]{nb}{\enot\metaB} \ni{b-xa}
\end{proof}

\begin{proof}
	\hypo[1]{ab}{\metaA\eiff\metaB}
	\hypo[2]{nb}{\enot\metaB}
  \open 
    \hypo[3]{a}{\metaA} \as{}
    \have[4]{b}{\metaB} \be{ab,a}
    \have[5]{xb}{\enot\metaB} \r{nb}
  \close
  \have[6]{na}{\enot\metaA} \ni{a-xb}
\end{proof}

\end{multicols}



\subsection{Double Negation}

Whereas we have included two similar rules for negation introduction and elimination, some texts only include negation introduction together with the following rule for \textit{double negation elimination} (DN):

\begin{proof}
	\have[m]{dna}{\enot\enot\metaA}
	\have[\ ]{a}{\metaA}\by{DN}{dna}
\end{proof}

Although some philosophers of logic contest DN, arguing instead for \textit{intuitionistic logics} in which DN is neither basic nor derivable, most take DN to be a useful and extremely natural inference to draw.
After all, what is meant by saying that it is not the case that the ball is not round, and yet it fails to be the case that the ball is round?
Or to take the converse, what is meant by saying that the ball is round, but it fails to be the case that the ball is not not round.
Classical logic avoids this mess by accepting DN even if it is derived instead of being included as a basic rule.
Here is the derivation of DN in the present system SD:

\begin{proof}
	\hypo[1]{dna}{\enot\enot\metaA}
  \open 
    \hypo[3]{da}{\enot\metaA} \as{}
    \have[4]{xa}{\enot\enot\metaA} \r{dna}
  \close
  \have[5]{a}{\metaA} \ne{da-xa}
\end{proof}

As in the other case, our derived rule DN allows us to draw natural inferences with minimal complexity, avoiding the need to open any subproofs.




\subsection{Ex Falso Quodlibet}

From a falsehood anything follows, or in Latin, \textit{ex falso quodlibet}.
For instance, if $A$ is false, then $\enot A$ is true, and so if we were to take $A$ to also be true, then together we may derive $B$ from this contradiction. 
More generally, we have the following rule (EFQ):

\begin{proof}
	\have[m]{a}{\metaA}
	\have[n]{na}{\enot\metaA}
	\have[\ ]{b}{\metaB}\by{EFQ}{a,na}
\end{proof}

This inference is occasionally convenient since, given $\metaA$ and $\enot\metaA$ on live lines we may draw any conclusion that might happen to want on the next line. 
Here is the derivation of EFQ.

\begin{proof}
	\hypo[1]{a}{\metaA}
	\hypo[2]{na}{\enot\metaA}
  \open 
    \hypo[3]{nb}{\enot\metaB} \as{}
    \have[4]{xa}{\metaA} \r{a}
    \have[5]{xna}{\enot\metaA} \r{na}
  \close
  \have[6]{b}{\metaB} \ne{nb-xna}
\end{proof}

This puts a syntactic spin on a semantic idea that we considered before: just as unsatisfiable sets of sentences entail everything, everything can be proven from any $\metaA$ and $\enot\metaA$. 


\section{Schemata}

You might be beginning to wonder just how many derived rules there are.
Indeed, this way of thinking inspires an important approach to logic: instead of focusing of theorems and proofs in SL, perhaps we should really be focusing on the \textit{rule schemata} themselves.
As a special case we have \textit{axiom schemata} which amount to rule schemata which do not cite any lines.
For instance, here is an axiom schemata called the \textit{Law of Excluded Middle} (LEM): $\metaA\eor\enot\metaA$.
Once we have derived LEM--- a worthwhile exercise--- we may write any instance of $\metaA\eor\enot\metaA$ on any line of a proof without citing any previous lines whatsoever.

Instead of thinking about valid SL arguments and their corresponding proofs, one might think about a logic as a set of rule schemata.
This approach to logic has some advantages since it cuts to the chase, focusing on what we are ultimately concerned with, namely \textit{logical forms}.
It is also easy to prove things about such systems since they are easy to characterise abstractly.
The disadvantage of such an approach is that these proof systems are somewhat harder to think about.
By contrast, it is nice to have concrete instances of SL sentences and arguments instead of logical forms all the way down.
This more concrete approach will be maintained throughout this text, however, it is nevertheless worth drawing the connection given all of our derived rules above.
Indeed, we may think of the system of natural deduction SD as the smallest set to include the basic rule schemata in SD together with all rule schemata that are derivable from these basic rules.
This set of logical forms of reasoning may be taken to provide an abstract answer to the question, ``What is logic about?''


% \iffalse

\subsection{Replacement}

Consider how you would prove this argument form valid: $F\eif(G\eand H)$ \therefore\ $F\eif G$

Perhaps it is tempting to write down the premise and apply the {\eand}E rule to the conjunction $(G \eand H)$. This is impermissible, however, because the basic rules of proof can only be applied to whole sentences. In order to use {\eand}E, we need to get the conjunction $(G \eand H)$ on a line by itself. Here is a proof:

\begin{proof}
	\hypo{fgh}{F\eif(G\eand H)} \pr{}
	\open
		\hypo{f}{F}\as{want $G$}
		\have{gh}{G \eand H}\ce{fgh,f}
		\have{g}{G}\ae{gh}
	\close
	\have{fg}{F \eif G}\ci{f-g}
\end{proof}

The rules we have seen so far must apply to wffs that are on a proof line by themselves. We will now introduce some derived rules that may be applied to wffs that are parts of more complex sentences. These are called \define{rules of replacement}, because they can be used to replace part of a sentence with a logically equivalent expression. One simple rule of replacement is Commutativity (abbreviated Comm), which says that we can swap the order of conjuncts in a conjunction or the order of disjuncts in a disjunction. We define the rule thus:

\begin{center}
\begin{tabular}{rl}
$(\metaA{}\eand\metaB{}) \Longleftrightarrow (\metaB{}\eand\metaA{})$\\
$(\metaA{}\eor\metaB{}) \Longleftrightarrow (\metaB{}\eor\metaA{})$\\
$(\metaA{}\eiff\metaB{}) \Longleftrightarrow (\metaB{}\eiff\metaA{})$
& Comm
\end{tabular}
\end{center}

The double arrow means that you can take a subformula on one side of the arrow and replace it with the subformula on the other side. The arrow is double-headed because rules of replacement work in both directions. And replacement rules --- unlike all the rules we've seen so far --- can be applied to wffs that are part of more complex sentences. They don't need to be on their own line.

Consider this argument: $(M \eor P) \eif (P \eand M)$ \therefore\ $(P \eor M) \eif (M \eand P)$

It is possible to give a proof of this using only the basic rules, but it will be somewhat tedious. With the Comm rule, we can provide a proof easily:

\begin{proof}
	\hypo{1}{(M \eor P) \eif (P \eand M)} \pr{}
	\have{2}{(P \eor M) \eif (P \eand M)}\by{Comm}{1}
	\have{n}{(P \eor M) \eif (M \eand P)}\by{Comm}{2}
\end{proof}

(We need to apply the rule twice, because each application allows one transformation. We transformed the antecedent first, then the consequent. The opposite order would also have been fine.)

Another rule of replacement is Double Negation (DN). With the DN rule, you can remove or insert a pair of negations for any wff in a line, even if it isn't the whole line. This is the rule:

\begin{center}
\begin{tabular}{rl}
$\enot\enot\metaA{} \Longleftrightarrow \metaA{}$ & DN
\end{tabular}
\end{center}

Two more replacement rules  are called De Morgan's Laws, named for the 19th-century British logician August De Morgan. (Although De Morgan did formalize and publish these laws, many others discussed them before him.) The rules capture useful relations between negation, conjunction, and disjunction. Here are the rules, which we abbreviate DeM:

\begin{center}
\begin{tabular}{rl}
$\enot(\metaA{}\eor\metaB{}) \Longleftrightarrow (\enot\metaA{}\eand\enot\metaB{})$\\
$\enot(\metaA{}\eand\metaB{}) \Longleftrightarrow (\enot\metaA{}\eor\enot\metaB{})$
& DeM
\end{tabular}
\end{center}

As we have seen, $\metaA{}\eif\metaB{}$ is equivalent to $\enot\metaA{}\eor\metaB{}$. A further replacement rule captures this equivalence. We abbreviate the rule MC, for `material conditional.' It takes two forms:

\begin{center}
\begin{tabular}{rl}
$(\metaA{}\eif\metaB{}) \Longleftrightarrow (\enot\metaA{}\eor\metaB{})$ &\\
$(\metaA{}\eor\metaB{}) \Longleftrightarrow (\enot\metaA{}\eif\metaB{})$ & MC
\end{tabular}
\end{center}

Now consider this argument: $\enot(P \eif Q)$ \therefore\ $P \eand \enot Q$

As always, we could prove this argument valid using only the basic rules. With rules of replacement, though, the proof is much simpler:

\begin{proof}
	\hypo{1}{\enot(P \eif Q)} \pr{}
	\have{2}{\enot(\enot P \eor Q)}\by{MC}{1}
	\have{3}{\enot\enot P \eand \enot Q}\by{DeM}{2}
	\have{4}{P \eand \enot Q}\by{DN}{3}
\end{proof}

A final replacement rule captures the relation between conditionals and biconditionals. We will call this rule biconditional exchange and abbreviate it {\eiff}{ex}.

\begin{center}
\begin{tabular}{rl}
$[(\metaA{}\eif\metaB{})\eand(\metaB{}\eif\metaA{})] \Longleftrightarrow (\metaA{}\eiff\metaB{})$
& {\eiff}{ex}
\end{tabular}
\end{center}



%
%Although they don't do it in the book, I've been in the habit of writing $(\metaA{}\eand\metaB{}\eand\metaC{})$ and dropping the inner pair of parentheses. This is fine. If we'd wanted to, we could have defined the basic rules in a more general way:
%
%\begin{proof}
%	\have[n]{a1}{\metaA{}_1}
%	\have{2}{\metaA{}_2}
%	\have[\vdots]{1}{\vdots}
%	\have[n]{an}{\metaA{}_n}
%	\have[\ ]{aaa}{\metaA{}_1~\eand\ldots\eand~\metaA{}_n} \ai{}
%\end{proof}
%
%\bigskip
%\begin{proof}
%	\have{3}{\metaA{}_1~\eand\ldots\eand~\metaA{}_n}
%	\have{1}{\metaA{}_i} \ae{}
%\end{proof}
%
%\bigskip
%\begin{proof}
%	\have{1}\metaA{}
%	\have{3}{\metaA{}\eor\metaB{}_1\eor\metaB{}_2\ldots\eor\metaB{}_n} \ai{}
%\end{proof}
%
%We don't need these extended versions, since for any given n we could prove them as a derived rule.
%
%
%The basic rules for conjunction can be valuable in a proof even if there are no conjunctions in any of the assumptions; the basic rules for disjunction can be used even if there are no disjunctions in any assumptions; and similarly for the other basic rules. The rules for identity are different, in that there must be an identity claim in some assumption in order for the rules to do any work. Other than the trivial identity that we can introduce with the {=}I rule
%
%
%do not apply we can now prove that identity is \emph{transitive}: If $a=b$ and $b=c$, then $a=c$. The proof proceeds in this way:
%\begin{proof}
%	\open
%		\hypo{p}{a=b \eand b=c}\by{want $a=c$}{}
%		\have{ab}{a=b}\ae{p}
%		\have{bc}{b=c}\ae{p}
%		\have{ac}{a=c}\by{{=}E}{ab,bc}
%	\close
%	\have{conc}{(a=b \eand b=c)\eif a=c} \ci{p-ac}
%\end{proof}
%
%
%As an example, consider this argument:
%\begin{quote}
%There is only one button in my pocket. There is a blue button in my pocket. Therefore, there is no button in my pocket that is not blue.
%\end{quote}
%We begin by defining a symbolization key:
%\begin{ekey}
%\item{UD:} buttons in my pocket
%\item{Bx:} $x$ is blue.
%\end{ekey}
%\begin{proof}
%	\hypo{one}{\forall x\forall y\ x=y}
%	\hypo{eb}{\exists x Bx} \by{want $\enot\exists x \enot Bx$}{}
%	\open
%		\hypo{be1}{Be}
%		\have{ef1}{e=f}\Ae{one}
%		\have{bf1}{Bf}\by{{=}E}{ef1,be1}
%	\close
%	\have{bf}{Bf}\Ee{eb,be1-bf1}
%	\have{ab}{\forall x Bx}\Ai{bf}
%	\have{nnab}{\enot\enot\forall x Bx}\by{DN}{ab}
%	\have{nenb}{\enot\exists x\enot Bx}\by{QN}{nnab}
%\end{proof}

% \iffalse
 
\clearpage



\practiceproblems

\solutions
\problempart
\label{pr.justifySLproof}
Provide a justification (rule and line numbers) for each line of proof that requires one {\color{black}(which for \textit{Carnap} means ALL of them!)}
\begin{multicols}{2}
\begin{proof}
\hypo{1}{W \eif \enot B}
\hypo{2}{A \eand W}
\hypo{2b}{B \eor (J \eand K)}
\have{3}{W}{}
\have{4}{\enot B} {}
\have{5}{J \eand K} {}
\have{6}{K}{}
\end{proof}

\begin{proof}
\hypo{1}{L \eiff \enot O}
\hypo{2}{L \eor \enot O}
\open
	\hypo{a1}{\enot L}
	\have{a2}{\enot O}{}
	\have{a3}{L}{}
	\have{a4}{\enot L}{}
\close
\have{3}{L}{}
\end{proof}

\begin{proof}
\hypo{1}{Z \eif (C \eand \enot N)}
\hypo{2}{\enot Z \eif (N \eand \enot C)}
\open
	\hypo{a1}{\enot(N \eor  C)}
	\have{a2}{\enot N \eand \enot C} {}
	\open
		\hypo{b1}{Z}
		\have{b2}{C \eand \enot N}{}
		\have{b3}{C}{}
		\have{b4}{\enot C}{}
	\close
	\have{a3}{\enot Z}{}
	\have{a4}{N \eand \enot C}{}
	\have{a5}{N}{}
	\have{a6}{\enot N}{}
\close
\have{3}{N \eor C}{}
\end{proof}
\end{multicols}

\solutions
\problempart
\label{pr.solvedSLproofs}
Give a proof for each argument in SL.
\begin{earg}
\item $K\eand L$, \therefore $K\eiff L$
\item $A\eif (B\eif C)$, \therefore $(A\eand B)\eif C$
\item $P \eand (Q\eor R)$, $P\eif \enot R$, \therefore $Q\eor E$
\item $(C\eand D)\eor E$, \therefore $E\eor D$
\item $\enot F\eif G$, $F\eif H$, \therefore $G\eor H$
\item $(X\eand Y)\eor(X\eand Z)$, $\enot(X\eand D)$, $D\eor M$ \therefore $M$
\end{earg}

\problempart
Give a proof for each argument in SL.
\begin{earg}
\item $Q\eif(Q\eand\enot Q)$, \therefore\ $\enot Q$
\item $J\eif\enot J$, \therefore\ $\enot J$
\item $E\eor F$, $F\eor G$, $\enot F$, \therefore\ $E \eand G$
\item $A\eiff B$, $B\eiff C$, \therefore\ $A\eiff C$
\item $M\eor(N\eif M)$, \therefore\ $\enot M \eif \enot N$
\item $S\eiff T$, \therefore\ $S\eiff (T\eor S)$
\item $(M \eor N) \eand (O \eor P)$, $N \eif P$, $\enot P$, \therefore\ $M\eand O$
\item $(Z\eand K) \eor (K\eand M)$, $K \eif D$, \therefore\ $D$
\end{earg}


\solutions
\problempart
\label{pr.SLND.theorems}
Show that each of the following sentences is a theorem in SL.
\begin{earg}
\item $O \eif O$
\item $N \eor \enot N$
\item $\enot(P\eand \enot P)$
\item $\enot(A \eif \enot C) \eif (A \eif C)$
\item $J \eiff [J\eor (L\eand\enot L)]$
\end{earg}

\problempart
Show that each of the following pairs of sentences are provably equivalent in SL.
\begin{earg}
\item $\enot\enot\enot\enot G$, $G$
\item $T\eif S$, $\enot S \eif \enot T$
\item $R \eiff E$, $E \eiff R$
\item $\enot G \eiff H$, $\enot(G \eiff H)$
\item $U \eif I$, $\enot(U \eand \enot I)$
\end{earg}

\solutions
\problempart
\label{pr.solvedSLproofs2}
Provide proofs to show each of the following.
\begin{earg}
\item $M \eand (\enot N \eif \enot M) \proves (N \eand M) \eor \enot M$
\item \{$C\eif(E\eand G)$, $\enot C \eif G$\} $\proves$ $G$
\item \{$(Z\eand K)\eiff(Y\eand M)$, $D\eand(D\eif M)$\} $\proves$ $Y\eif Z$
\item \{$(W \eor X) \eor (Y \eor Z)$, $X\eif Y$, $\enot Z$\} $\proves$ $W\eor Y$
\end{earg}



\problempart
For the following, provide proofs using only the basic rules. The proofs will be longer than proofs of the same claims would be using the derived rules.
\begin{earg}
\item Show that MT is a legitimate derived rule. Using only the basic rules, prove the following: \metaA{}\eif\metaB{}, \enot\metaB{}, \therefore\ \enot\metaA{}
\item Show that Comm is a legitimate rule for the biconditional. Using only the basic rules, prove that $\metaA{}\eiff\metaB{}$ and $\metaB{}\eiff\metaA{}$ are equivalent.
\item Using only the basic rules, prove the following instance of DeMorgan's Laws: $(\enot A \eand \enot B)$, \therefore\ $\enot(A \eor B)$
\item Show that {\eiff}{ex} is a legitimate derived rule. Using only the basic rules, prove that $D\eiff E$ and $(D\eif E)\eand(E\eif D)$ are equivalent.
\end{earg}




\problempart
\begin{earg}
\item If you know that $\metaA{}\proves\metaB{}$, what can you say about $(\metaA{}\eand\metaC{})\proves\metaB{}$? Explain your answer.
\item If you know that $\metaA{}\proves\metaB{}$, what can you say about $(\metaA{}\eor\metaC{})\proves\metaB{}$? Explain your answer.
\end{earg}

\fi


%****************STUFF JH CUT FROM UBC EDITION*******************************


\iffalse 

\section{Ichikawa: The remaining basic rules}

All of the rules introduced in this chapter are summarized in the Quick Reference guide at the end of this book.

\subsection{Disjunction Introduction}
If $M$ is true, then $M \eor N$ must also be true. In general, the Disjunction Introduction rule ({\eor}I) allows us to derive a disjunction if we already have one of its two disjuncts:

\begin{proof}
	\have[m]{a}\metaA{}
	\have[\ ]{ab}{\metaA{}\eor\metaB{}}\oi{a}
	\have[\ ]{ba}{\metaB{}\eor\metaA{}}\oi{a}
\end{proof}

One can introduce a disjunct in either position --- it can be the first disjunct or the second disjunct. Accordingly, both options are listed here. (One is not required to do both; you can just take whichever version you find most helpful.)

As always, \metaB{} can be \emph{any} sentence whatsoever. So the following is a legitimate proof:

\begin{proof}
	\hypo{m}{M}
	\have{mmm}{M \eor ([(A\eiff B) \eif (C \eand D)] \eiff [E \eand F])}\oi{m}
\end{proof}

It may seem odd that just by knowing $M$ we can derive a conclusion that includes sentences like $A$, $B$, and the rest --- sentences that have nothing to do with $M$. Yet the conclusion follows immediately by {\eor}I. This is as it should be: The truth conditions for the disjunction mean that, if \metaA{} is true, then $\metaA{}\eor \metaB{}$ is true regardless of what \metaB{} is. So the conclusion could not be false if the premise were true; the argument form is valid.



\subsection{Conditional Introduction}

Consider this argument:
\begin{earg}
\item[] $R \eor F$
\item[\therefore] $\enot R \eif F$
\end{earg}
The argument form seems like it should be valid --- either $R$ or $F$ is true, so if $R$ isn't true, $F$ must be. (You can confirm that it is valid by examining the truth tables.) The Conditional Introduction rule can demonstrate that this is so.

We begin the proof by writing down the premise of the argument, making a note of our intended conclusion, and drawing a horizontal line, like this:

\begin{proof}
	\hypo{rf}{R \eor F} \want{\enot R \eif F}
\end{proof}

If we had $\enot R$ as a further premise, we could derive $F$ by the {\eor}E rule. But we do not have $\enot R$ as a premise of this argument, nor can we derive it directly from the premise we do have --- so we cannot simply prove $F$. What we will do instead is start a \emph{subproof}, a proof within the main proof. When we start a subproof, we draw another vertical line to indicate that we are no longer in the main proof. Then we write in an assumption for the subproof. This can be anything we want. Here, it will be helpful to assume $\enot R$. We want to show that, if we did assume that, we would be able to derive $F$. So we make a new assumption that $\enot R$, and give ourselves a note that we wish to derive $F$. Our proof now looks like this:

\begin{proof}
	\hypo{rf}{R \eor F}  \want{\enot R \eif F}
	\open
		\hypo{nr}{\enot R} \want {F}
		\have{}{}
	\close
\end{proof}

It is important to emphasize that we are not claiming to have \emph{proven} $\enot R$ from the premise on line 1. We do not need to write in any justification for the assumption line of a subproof. (The `want' is a note to ourself, not a justification.) The new horizontal line indicates that a new \emph{assumption} is being made; we also include a vertical line that extend to future lines, to indicate that this assumption remains in effect. You can think of the subproof as posing the question: What could we show \emph{if} $\enot R$ were true? We are trying to show that we could derive $F$. And indeed, we can:

\begin{proof}
	\hypo{rf}{R \eor F}  \want{\enot R \eif F}
	\open
		\hypo{nr}{\enot R} \want {F}
		\have{f}{F}\by{DS}{rf, nr}
	\close
\end{proof}

This has shown that \emph{if} we had $\enot R$ as a premise, \emph{then} we could prove $F$. In effect, we have proven that $F$ follows from $\enot R$. This is indeed very close to demonstrating the conditional, $\enot R \eif F$. This last step is just what the Conditional Introduction rule will allow us to perform. We close the subproof and derive $\enot R \eif F$ in the main proof. Our final proof looks like this:

\begin{proof}
	\hypo{rf}{R \eor F}
	\open
		\hypo{nr}{\enot R}\want {F}
		\have{f}{F}\by{DS}{rf, nr}
	\close
	\have{nrf}{\enot R \eif F}\ci{nr-f}
\end{proof}

The {\eif}I rule lets us \define{discharge} the assumption we'd been making, ending that vertical line. We also stop indenting --- the difference in placement of lines 3 and 4 emphasizes that they are importantly different: during lines 2 and 3, we were \emph{assuming} that \enot $R$. By the time we get to line 4, we are no longer making that assumption.

Notice that the justification for applying the {\eif}I rule is the entire subproof. That's why we justify it by reference to a range of lines, instead of a comma-separated list. Usually that will be more than just two lines.

It may seem as if the ability to assume anything at all in a subproof would lead to chaos: Does it allow you to prove any conclusion from any premises? The answer is no, it does not. Consider this proof schema:

\begin{proof}
	\hypo{a}\metaA{}
	\open
		\hypo{b1}\metaB{}
		\have{b2}{\metaB{}\eand \metaA{}} \ai{a,b1}
	\close
\end{proof}

Does this show that one can conjoin any arbitrary sentence \metaB{} with premise \metaA{}? After all, we've written \metaB{}\eand\metaA{} on a line of a proof that began with \metaA{}, without violating any of the rules of our system. The reason this doesn't have that implication is the vertical line that still extends into line 3. That line indicates that the assumption made at line 2 is still in effect. When the vertical line for the subproof ends, the subproof is \emph{closed}. In order to complete a proof, you must close all of the subproofs. The conclusion to be proved must not be `blocked off' by a vertical line; it should be aligned with the premises.

In this example, there is no way to close the subproof and show that the conjunction follows from line 1 alone. One can only close a subproof via particular rules that allow you to do so; {\eif}I is one such rule; {\eand}I does not close subproofs. One can't just close a subproof willy-nilly. Closing a subproof is called \emph{discharging} the assumptions of that subproof. So we can put the point this way: You cannot complete a proof until you have discharged all of the assumptions other than the original premises of the argument.

Of course, it is legitimate to do this:

\begin{proof}
	\hypo{a}\metaA{}
	\open
		\hypo{b1}\metaB{}
		\have{b2}{\metaB{}\eand \metaA{}} \ai{a,b1}
	\close
	\have{c}{\metaB{} \eif (\metaB{}\eand\metaA{})} \ci{b1-b2}
\end{proof}

This should not seem so strange, though. The conclusion on line 4 really does follow from line 1. (Draw a truth table if you need convincing of this.) 

Once an assumption has been discharged, any lines that have been shown to follow from that assumption --- i.e., those lines inside the box indicated by the vertical line of that assumption --- cannot be cited as justification on further lines. So this development of the proof above, for instance, is not permitted:

\begin{proof}
	\hypo{rf}{R \eor F}
	\open
		\hypo{nr}{\enot R}\want {F}
		\have{f}{F}\by{DS}{rf, nr}
	\close
	\have{nrf}{\enot R \eif F}\ci{nr-f}
	\have{bad}{F \eor A}\oi{f}
\end{proof}

Once the assumption made at line 2 has been discharged at line 4, the lines within that assumption --- 2 and 3 --- are unavailable for further justification. So one cannot perform Disjunctive Syllogism  on line 3 at line 5. Line 3 was not demonstrated to follow from the premise on line 1 --- it follows only from this combined with the \emph{assumption} on line 2. And by the time we get to line 5, we are no longer making that assumption.

Put in its general form, the {\eif}I rule looks like this:

\begin{proof}
	\open
		\hypo[m]{a}\metaA{} \by{want \metaB{}}{}
		\have[n]{b}\metaB{}
	\close
	\have[\ ]{ab}{\metaA{}\eif\metaB{}}\ci{a-b}
\end{proof}

When we introduce a subproof, we typically write what we want to derive off to the right. This is just so that we do not forget why we started the subproof if it goes on for five or ten lines. There is no `want' rule. It is a note to ourselves, and not formally part of the proof.

Although it is consistent with the natural deduction rules to open a subproof with any assumption you please, there is some strategy involved in picking a useful assumption. Starting a subproof with a random assumption is a terrible strategy. It will just waste lines of the proof. In order to derive a conditional by the {\eif}I rule, for instance, you must assume the antecedent of that conditional.

The {\eif}I rule also requires that the consequent of the conditional be the last line of the subproof. It is always permissible to close a subproof and discharge its assumptions, but it will not be helpful to do so until you get what you want. This is an illustration of the observation made above, that unlike the tree method, the natural deduction method requires some strategy and thinking ahead.

You should \emph{never} make an assumption without a plan for how to discharge it.





\subsection{Biconditional Introduction}
Biconditionals indicate that the two sides have the same truth value. One establishes a biconditional by establishing each direction of it as conditionals. To derive $W \eiff X$, for instance, you must establish both $W \eif X$ and $X \eif W$. (You might derive those conditionals via {\eif}I, or you might get them some other way. They might even simply be premises.) Those conditionals may occur in either order; they need not be on consecutive lines. (Compare the shape of the {\eand}I rule.) Schematically, the Biconditional Introduction rule works like this:

\begin{proof}
	\have[m]{ab}{\metaA{}\eif\metaB{}}
	\have[n]{ba}{\metaB{}\eif\metaA{}}
	\have[\ ]{c}{\metaA{}\eiff\metaB{}} \bi{ab, ba}
\end{proof}


\subsection{Biconditional Elimination}

The Biconditional Elimination rule ({\eiff}E) is a generalized version of \emph{modus ponens} ({\eif}E). If you have the left-hand subsentence of the biconditional, you can derive the right-hand subsentence. If you have the right-hand subsentence, you can derive the left-hand subsentence. This is the rule:



\begin{multicols}{2}
\begin{proof}
	\have[m]{ab}{\metaA{}\eiff\metaB{}}
	\have[n]{a}\metaA{}
	\have[\ ]{b}\metaB{} \be{ab,a}
\end{proof}
\begin{proof}
	\have[m]{ab}{\metaA{}\eiff\metaB{}}
	\have[n]{a}\metaB{}
	\have[\ ]{b}\metaA{} \be{ab,a}
\end{proof}
\end{multicols}

As in the case of Disjunction Elimination, we include both versions under the same name, so that you don't need to worry about whether the side you already have is the left-hand side of the biconditional or the right-hand side. Whichever side it is, you may derive the other via Biconditional Elimination.

\fi 

