%!TEX root = forallx-ubc.tex
\chapter{Quantifier Logic}
\label{ch.QL}

This chapter introduces a logical language called QL which includes quantifiers like `for all' ($\qt{\forall}{}$) and `there is' ($\qt{\exists}{}$).
Quantifier logic is also sometimes called \emph{predicate logic} because the basic units of the language are predicates and singular terms.
Whereas a one-place predicate is a term for property, two-place predicates are terms for relations, and $n$-place predicates are similar but relate more objects.
Quantifier logic is also often called \textit{first-order logic} because the quantifiers range over objects and not over properties or relations as in \textit{second-order logic}.
For instance, QL has the expressive power to quantify over people, animals, or things, but not over their various properties such as \textit{being blue}, or relations like \textit{being taller than}, etc.




\section{The Expressive Limitations of SL}

Consider the following argument:

\begin{earg}
  \item[] Every human is mortal.
  \item[] Socrates is human.
  \item[\therefore] Socrates is mortal.
\end{earg}

In order to symbolize this argument in SL, we will need a symbolization key. 

\begin{ekey}
  \item[E:] Every human is mortal.
  \item[H:] Socrates is human.
  \item[M:] Socrates is mortal.
\end{ekey}

Notice that there is no way to break down the sentences above into smaller parts.
However, consider the resulting regimentation of the argument in SL:

\begin{earg}
  \item[] $E$
  \item[] $H$
  \item[\therefore] $M$
\end{earg}

It is easy to show that this argument is invalid.
However, the argument in English is intuitively valid.
Notice that although the sentences given above include the same terms, this is not captured by the regimentation in SL.
Moreover, there is no better regimentation in SL.

Here's another common case:

\begin{earg}
  \item[] All humans are mammals.
  \item[] All mammals are multi-celled organisms.
  \item[\therefore] All humans are multi-celled organisms.
\end{earg}

This argument is clearly valid.
However, were we to symbolize this argument in SL, the best we could do looks like this:

\begin{ekey}
  \item[H:] All humans are mammals.
  \item[M:] All mammals are multi-celled organisms.
  \item[O:] All humans are multi-celled organisms.
\end{ekey}

Again, there is no way to further decompose the sentences given above into smaller parts.
However, the resulting argument in SL is invalid for the same reason:

\begin{earg}
  \item[] $H$
  \item[] $M$
  \item[\therefore] $O$
\end{earg}

Since there is nothing wrong with our regimentation when restricted to SL, we may take issue with the language SL itself, concluding that we have simply reached the limit of what can be captured with the expressive resources which SL provides.
Instead of carving arguments up into \textit{sentences}, we have to look deeper into the sentences themselves in order to capture the validity of the English arguments considered above.
In particular, we need to be able to regiment predicates like `is human', `is mortal', `is a mammal', etc., as well as the quantifiers `every' and `all', and also names like `Socrates'.
Providing the expressive resources to do so will be the ambition of this chapter, where the following chapters will develop a semantics, theory of entailment, and proof system for the resulting language QL.

As we will see, QL is extremely powerful.
Indeed, most of mathematics is developed in a first-order language.
Understanding this language and its limits will prove to be an invaluable resource, shedding light on a wide range of theoretical applications as well as extending your powers of logical thinking in day-to-day reasoning. 
The development of first-order logics and their applications constitute some of the most important and influential theories that human beings have developed, and modern logic is still at its very beginning.





\section{Primitive Expressions in QL}

Our first key idea will be to include \textit{predicates} in QL.
An easy way to think about predicates is as terms for properties: the predicate `is red' stands for the property of being red.
We use predicates to ascribe properties to objects.
On their own, predicates are neither true nor false, though they are meaningful nevertheless.
For instance, English speakers will know what the predicate `is red' means and how to use it to make claims about the world.
Nevertheless, it doesn't make sense to assign a truth-value to `is red', `loves', `is between Sam and Mary in birth order', `is careful and quite', or any other predicate, however complex.
Rather predicates are used to ascribe properties and relations to objects, where those objects may or may not have the properties and relations which are ascribed to them.

In order to refer to objects, we will need another basic type of expression for \textit{names}.
For instance, the name `Christoph' refers to a particular object, in this case a person.
Accordingly, we may construct the sentence `Christoph is Italian', where this may be true or false depending on whether the object that `Christoph' names has the property expressed by the predicate `is Italian'.
Although names like `Christoph' are meaningful on their own, they do not have truth-values any more than predicates have truth-values.
Rather, it is by combining predicates with singular terms that we will construct \textit{atomic sentences} which do have truth-values.
Given the same sentential operators included in SL, we may then construct complex sentences which inherit their truth-values from their parts.
Despite introducing a range of new symbols to the language, QL will maintain our previous convention that only sentences have truth-values.

Given that we can construct sentences with nothing more than names and predicates, you might be wondering where the quantifiers fit in.
After all, there is no grammatical way to append `every', `all', or `some' to sentences like `Christoph is Italian'.
What is missing are the \define{variables}.
In order to bring this out, consider the following version of \textbf{H}:

\begin{ekey}
\item[H$'$:] Everything is such that if it is human, then it is a mammal.
\end{ekey}

However stilted, this claim is perfectly intelligible and express exactly what \textbf{H} expressed before.
Three observations are in order.
Whereas \textbf{H} includes the plural terms `humans' and `mammals', we now have the clearly identifiable predicates `is human' and `is a mammal'.
For instance, whereas we may combine `is human' with a name like `Christoph' in order to produce the perfectly intelligible declarative sentence `Christoph is human', the same cannot be said for the plural term `humans' since `Christoph humans' is nonsense. 

Second, we may consider the role that `it' plays in \textbf{H$'$}.
Whereas a name like `Christoph' refers to an object, the term `it' also appears to indicate an object.
Rather, `it' is playing the role of a variable.
Since both constants and variables refer to objects, we will refer to both as \textit{singular terms}.
% Although it is not always clear which object a variable may refer to, variables are also singular terms. % just like names such as `Christoph'.
% Although it is not clear what `it' refers to, `it' is a singular term insofar as it belongs to the right grammatical category to refer to an object. 
Accordingly, `it' may be combined with a predicate to form a sentence.
In particular, `it' may be combined with predicates to produce `it is human' and `it is a mammal'.
Considered on their own, both `it is human' and `it is a mammal' are sentences though they include the \textit{free variable} `it', and so are said to be \textit{open sentences}.
Nevertheless, these sentences may be combined with a conditional operator to form `if it is human, then it is mammal' which is also an open sentence.
Appending `Everything is such that' to the conditional sentence `if it is human, then it is mammal' produces \textbf{H$'$}. %, where the former is said to \textit{bind} the unbound variables in the latter.
Instead of speaking about a particular, \textbf{H$'$} says of each object that \textit{it} is a mammal if \textit{it} is human.
Given any object, both occurrences of `it' refer to that object, at least until we move on to the next object since we are making a general claim about everything.
% If you like, you could imagine that such a claim is equivalent to a very long conjunction were each conjunct says of a different thing that it is a mammal if human.
Put otherwise, both occurrences of `it' are \textit{bound} by the same quantifier `Everything is such that'.

Third, we may observe that the perfectly natural occurrence of `All' in \textbf{H} has been replaced with the somewhat cumbersome `Everything is such that'.
Whereas `All' is a \textit{generalised quantifier}, `Everything is such that' is an attempt to express the \textit{universal quantifier} $\qt{\forall}{x}$ with the resources of English.
These quantifiers differ in a number of important respects.
In particular, generalised quantifiers like `every', `some', `all', `most', `many', etc., combine with two descriptive terms in order to produce a grammatical sentence.
By contrast, we may observe that `Everything is such that' combines with the open sentence `if it is human, then it is mammal' to produce \textbf{H$'$}.
Although `it' is unbound in `if it is human, then it is mammal' when considered on its own, both occurrences `it' are bound by `Everything is such that' in \textbf{H$'$}.
Accordingly, \textbf{H$'$} is a \textit{closed sentence} since it does not include any unbound variables.

Although universal quantifiers can be approximated within English, generalised quantifiers are much more common and natural to use in English.
Despite this disparity, generalised quantifiers are more complicated than universal quantifiers, making them much less useful for systematic theorising.
Moreover, we may use universal quantifiers to regiment claims in English which are expressed with generalised quantifiers without too much trouble. 
For these reasons, we will not include formal analogues of generalised quantifiers in QL.

Given this overview of the new expressive resources that we will include in QL, the following sections will provide a much more details introduction of each of the elements that we touched on above.
We will begin with names and variables.


\subsection{Singular Terms}

In English, a proper name is a term that refers to a person, place, or thing.
For instance, `Christoph' refers to Christoph.
In QL, we will take lower-case letters `$a$', `$b$', `$c$',\ldots from the beginning of the alphabet with optional subscripts to be \define{constants} which we will make use of to regiment names.
Constants are the formal analogue of the names in English. 
% Constants are singular terms since they refer to a unique individual.

In English, the same name may refer to different objects in different contexts.
Whereas `Willard' might pick out different people in different contexts who have the same name, we will avoid this ambiguity in QL by insisting that constants refer to at most one individual on any given interpretation.
It is for this reason that we have included infinitely many constants (with the help of the subscripts) so that we don't run out of names.

% TODO: save definite descriptions if we include an operator to make definite descriptions singular terms
% There are other ways to refer to individuals besides using names.
% In particular, we often use descriptions.
% Whereas it is indeterminate to say `the person standing next to the podium' if more than one person is standing next to the podium or nobody is, this expression succeeds in picking out a unique individual if there is just one person standing next to the podium.
% Or to take another case, we may use `the tallest person in the room' to refer to a unique individual assuming that there are at least some people in the room and that one of them is taller than all the others.
% Such terms are referred to as \define{definite descriptions}.
% Although the definite descriptions that succeed in referring to unique objects are singular terms, we will not need to include primitive terms in our language to express definite descriptions.
% Instead, we will construct definite descriptions with the resources of QL.

Another important type of singular term are the \define{variables} that we considered above.
In English, terms such as `it', `her', etc., often play this role.
In QL, we will use lower-case letters `$x$', `$y$', `$z$',\ldots from the end of the alphabet with optional subscripts as variables.
We will use variables in combination with the quantifiers in order to make claims about every object, or about some object, specifying what properties those objects have.
However, in order to make claims about objects, we will need more than singular terms.





\subsection{Predicates}

A \textsc{one-place} predicate is used to express a property of individuals.
For instance, `is hungry' is a one-place predicate which, in combination with a singular term, forms an atomic sentence such as `Kate is hungry' and `She is hungry'.
In claiming that a predicate is one-place, we are indicating that just one singular term is needed to construct an atomic sentence.
This corresponds to the idea that the predicate expresses a property that a single object may have or fail to have.
For instance, either Kate is hungry or she isn't.
Accordingly, the sentence `Kate is hungry' is true or false, but the truth-value of this sentence only depends on Kate and whether she is hungry or not, and so no other singular terms need to be provided.

A \textsc{two-place} predicate is used to express a relation between individuals.
For instance, the two-place predicate `is taller than' expresses a height relation between objects.
Given a two-place predicate together with two singular terms, we may make a claim which is either true or false.
For instance, `Sam is taller than John' is true just in case Sam is taller than John.
It is worth noting that it is possible to use a two-place predicate together with two instances of the same singular term.
Just as we may say that `Jay loves Cary', we may say, `Jay loves Jay', or somewhat more naturally, `Jay loves himself'.
Although there is just one object at issue in such cases, the predicate `knows' is two-place all the same. 

The singular terms that a predicate requires to form a formula are referred to as the \define{arguments} of that predicate.
Whereas one-place predicates require one argument and two-place predicates require two arguments, in general we may consider $n$-place predicates which require $n$ arguments. 
Although it is easy to generalise, it is uncommon to consider more than three-place predicates in English.
For instance, we may say that Sue is between Henry and Tom in birth order, where `between' has three arguments.
Even if it is uncommon to do so in English, nothing stops us from including predicates with more than three arguments in QL.
It is convenient to refer to the number of arguments that a predicate has as the \define{adicity} of that predicate, where this comes from the convention of referring to one-place predicates as \define{monadic}, two-place predicates as \define{dyadic}, three-place predicates as \define{triadic}, etc.
Predicates with more than one place are called \define{polyadic}.
% You can make an $n$-adic predicate by replacing $n$ names in any sentence with blanks.
We will also include $0$-place predicates which correspond to the sentence letters with which we are familiar from before.


% In general, you can think about predicates as schematic sentences that need to be filled out with some number of terms. Conversely, you can start with sentences and make predicates out of them by removing terms. Consider the sentence, `Buchanan Tower is North of Barber but South of Buchanan E.' By removing a singular term, we can recognize this sentence as using any of three different monadic predicates:
% \begin{earg}
% \item[] \blank is North of Barber but South of Buchanan E.\\
% \item[] Buchanan Tower is North of \blank but South of Buchanan E.\\
% \item[] Buchanan Tower is North of Barber but South of \blank.
% \end{earg}
%
% By removing two singular terms, we can recognize three different dyadic predicates:
% \begin{earg}
% \item[] Buchanan Tower is North of \blank but South of \blank.\\
% \item[] \blank is North of Barber but South of \blank.\\
% \item[] \blank is North of \blank but South of Buchanan E.\\
% \end{earg}
%
% By removing all three singular terms, we can recognize one \define{three-place} or \define{triadic} predicate:
% \begin{center}
% \item[] \blank is North of \blank but South of \blank.\\
% \end{center}
%
% If we are translating this sentence into QL, should we translate it with a one-, two-, or three-place predicate? It depends on what we want to be able to say. If we're only interested in discussing the location of buildings relative to Barber, then the generality of the three-place predicate is unnecessary. If we only want to discuss whether buildings are North of Barber and South of Buchanan E, a one-place predicate will be enough.
%
% In general, we can have predicates with as many places as we need {\color{black}(you might get squeamish at infinity though!)}. 

In order to symbolize  $n$-place predicates in QL, we will use capital letters $A^n$ through $Z^n$, with or without subscripts, where the superscripts indicate the arity of the predicate.
Although officially the predicates include superscripts, we will almost always drop the superscripts.
After all, in most contexts it is clear that `$A$' is intended to be a tryadic predicate in `$Axcy$', and a dyadic predicate in `$Aaz$'.
Thus in order to indicate both the arity and interpretation of predicates, symbolization keys will use variables.
For instance, consider the following:

\begin{ekey}
\item[Ax:] $x$ is angry.
\item[Hx:] $x$ is happy.
\item[Sxy:] $x$ is shorter than $y$.
\item[T$_1$xy:] $x$ is at least as tall as $y$.
\item[T$_2$xy:] $x$ is at least as tough as $y$.
\item[Bxyz:] $y$ is between $x$ and $z$.
% \item[d:] Donald
% \item[g:] Gregor
% \item[m:] Marybeth
\end{ekey}

It is worth emphasising that the variables above do no further work than to specify the order of the arguments in a given predicate. 
Since monadic predicates only have one argument, the variable merely indicates its adicity.
By contrast, the order of the variables does matter for dyadic predicates like $T_1$.
For instance, we could have introduced the predicate:

\begin{ekey}
\item[T$_3$xy:] $y$ is at least as tall as $x$.
\end{ekey}

Having reversed the order of the variables above, `$T_3$' expresses the converse of the relation expressed by `$T_1$'.
This makes a big difference.
However, it would not have mattered if we were to have replaced `$x$' with some other variable such as `$z$' or `$y_{13}$'.

In order to put these predicates to work, we may now turn to combine the primitive symbols that we have included in QL in order to produce atomic sentences.





\subsection{Atomic Sentences}

Consider the following sentences:

\begin{earg}
\nix{I am inclined to change these to Cordelia, Hamlet, and Macbeth}
\item[\ex{terms1}] Donald is angry.
\item[\ex{terms2}] Donald is shorter than Gregor.
\item[\ex{terms3}] If Donald is angry, then so are Gregor and Marybeth.
\item[\ex{terms4}] Marybeth is at least as tall and as tough as Gregor.
% \item[\ex{terms5}] Gregor is between Donald and Marybeth.
\end{earg}

We have already provided the predicates that we will need to regiment the sentences above, but we have not introduced any constants.
Consider the following:

\begin{groupitems}
\begin{ekey}
\item[d:] Donald
\item[g:] Gregor
\item[m:] Marybeth
\end{ekey}
\end{groupitems}

Given these resources, we may regiment sentence \ref{terms1} as: $Ad$.
Since the `$x$' in the symbolization key entry for `$Ax$' is just a placeholder, we have replaced the variable with the constant which names Donald, resulting in what we will refer to as an atomic formula.

More precisely, an \define{atomic sentence} of QL is an $n$-place predicate followed by $n$ constants where $n$ is any non-negative integer.
Accordingly, $0$-place predicates are atomic formulas of QL where we will refer to $0$-place predicates as \define{sentence letters}.


Whereas `$Ad$' includes the monadic predicate `$A$', atomic sentences may include predicates of any adicity.
For instance, sentence \ref{terms2} can be regimented with the dyadic predicate `$S$' as: $Sdg$.
In general, atomic sentences will include an $n$-place predicate followed by $n$ constants. 

By contrast with sentences \ref{terms1} and \ref{terms2}, sentence \ref{terms3} is not atomic on account of including sentential operators.
Nevertheless, we may regiment this sentence by identifying its atomic parts: $Ad$, $Ag$, and $Am$.
These atomic sentences will play a similar role to the sentence letters included in SL.
In particular, we may regiment sentence \ref{terms3} by: $Ad \eif (Ag \eand Am)$.

Sentence \ref{terms4} is similar though the atomic sentences are a little trickier to identify.
We begin by observing that two things are said of Marybeth and Gregor: first, Marybeth is at least as tall as Gregor; and second, Marybeth is at least as tough as Gregor.
Thus we can build the atomic sentences: $T_1mg$ and $T_2mg$. 
The last step is to put these together: $T_1mg \eand T_2mg$.





\subsection{Open Sentences}

Consider the following sentence:

\begin{earg}
\item[\ex{open1}] Either it is crimson or ruby since she wants to impress him.
\end{earg}

In considering the predicates we will need, we may introduce the following:

\begin{ekey}
% \item[Fx:] $x$ is frustrated.
% \item[Ux:] $x$ is upset.
% \item[Hx:] $x$ is human.
% \item[Mx:] $x$ is mammal.
\item[Cx:] $x$ is crimson.
\item[Rx:] $x$ is ruby.
\item[Wxy:] $x$ wants to impress $y$.
\end{ekey}

Since there are no names in the sentence above, we do need to include constants in our symbolization key.
Instead, we can get by with variables alone.
In particular, we may regiment sentence \ref{open1} by first identifying its atomic parts.
To do so, we may observe that the object in question is said to be one of two colors, and that the person in question wants to impress someone else.
Since the same object is either crimson or ruby, we will use the same variable for each: $Cx$, $Rx$.
We use different variables for `she' and `him', where this yields: $Wyz$.
Putting these pieces together, we get: $(Cx \eor Rx) \eand Wyz$.

Although $Cx$, $Rx$, and $Wyz$ are atomic, these expressions are not sentences since it is hard to assign truth-values to such expressions.
Rather, these expressions include \textit{free variables} that are not bound by quantifiers, and so will be referred to as \textit{open sentences} where the precise definitions will be given shortly.
Nevertheless, it is important to get a sense of open sentences in order to appreciate the role that the quantifiers play.

In order to get a better sense of open sentences as well as some of the subtleties of regimenting sentences in QL, consider the closely related sentence:

\begin{earg}
\item[\ex{open2}] Either it is crimson or ruby since she likes those colors.
\end{earg}

Now we will need the following symbolization key:

\begin{ekey}
\item[Hxy:] $x$ has the color $y$.
\item[Lxy:] $x$ likes $y$.
\item[c:] Crimson
\item[r:] Ruby
\end{ekey}

Instead of taking `is crimson' and `is ruby' to be monadic predicates, we have included constants for each together with a new dyadic predicate for `has the color'.
This is useful since now we can say that she likes each color by using its name.
Thus we have the atomic open sentences: $Hxc$, $Hxr$, $Lyc$, and $Lyr$.
We can then build the following regimentation: $(Hxc \eor Hxr) \eand (Lyc \eand Lyr)$.
Despite how similar the sentences \ref{open1} and \ref{open2} are to each other, their regimentations are very different, and by no means exhaustive.

We may conclude this subsection with a final example:

\begin{earg}
\item[\ex{open3}] If it is a dolphin, then it is a mammal and he won't eat it.
\end{earg}

Here we only need the following predicates:

\begin{ekey}
\item[Dx:] $x$ is dolphin.
\item[Mx:] $x$ is mammal.
\item[Eyx:] $y$ will eat $x$.
\end{ekey}

As before, we may build the atomic open sentences $Hx$, $Mx$, and $\enot Eyx$, drawing on these to construct: $Hx \eif (Mx \eand \enot Eyx)$.
By itself, this open sentence appears to refer to some unnamed individual, however, this hardly captures its power.
Rather, it is by combining open sentences with quantifiers that we may make general claims of considerable interest.
Indeed, such claims are common throughout mathematics, the sciences, and day-to-day life.



\subsection{Quantifiers}

We are now ready to introduce quantifiers.
Unlike singular terms and predicates, quantifiers are a new kind of monadic logical connective similar to negation.
Instead of relying on an interpretation of our language to specify the meaning of the quantifiers, we will provide semantic clauses for the quantifiers in an analogous manner to the semantic clauses that we provided for the sentential connectives $\enot$, $\eand$, $\eor$, $\eif$, and $\eiff$.

Consider the sentences:

\begin{earg}
\item[\ex{qu1}] Someone is happy.
\item[\ex{qu2}] Nobody is as guilty as Donald.
\item[\ex{qu3}] Everybody loves somebody.
\end{earg}

We will need the following symbolization key:

\begin{ekey}
\item[Hx:] $x$ is happy.
\item[Gxy:] $x$ is as guilty as $y$.
\item[Lxy:] $x$ loves $y$.
\item[d:] Donald.
\end{ekey}

It might be tempting to regiment sentence \ref{qu1} as: $Hx$.
After all, if `It is happy' is true, then surely someone--- whoever `It' refers to--- must be happy.
Tempting as this may be, these are very different claims.
In order to see how these claims differ, we may consider who these claims \textit{concern} or \textit{are about}.
Although it may not be clear who exactly, the sentence `It is happy' is presumable about whoever `It' refers to where this is some individual.
By contrast, claiming that someone is happy does not appear to be about any particular individual at all.
For instance, suppose there are many happy people but `It' happens to be used to refer to a tables.
Although sentence \ref{qu1} may be true, `It is happy' is false.

Instead of asserting anything about a particular, quantifiers make claims about a \textit{domain} of individuals which we will introduce in the following chapter.
Depending on the context of assertion, the domain may differ.
Nevertheless, quantified claims may be understood to make claims about all of the individuals in the domain in question.
By contrast, open sentences make claims about at most one individual.
These are important differences, and we will have a lot more to say about them when we introduce the semantics for QL.

Although the sentence `It is happy' does not say the same thing as sentence \ref{qu1}, we may nevertheless begin by regimenting `It is happy'.
Given the symbolization key above, this yields: $Hx$.
What is missing is to add that \textit{there is something such that} it is happy.
We will accomplish this by appending an \define{existential quantifier} $\qt{\exists}{x}$, where this yields: $\qt{\exists}{x} Hx$.
% On its own, `$\qt{\exists}{x}$' may be read `there is something such that', and similarly, `$\qt{\forall}{x}$' may be read `everything is such that'.
As defined below, the quantifier \textit{binds} the variable in $Hx$ which is said to be \textit{bound}.
Since the result does not contain any free variables, $\qt{\exists}{x} Hx$ is a sentence of QL. 
% Here the quantifier `$\qt{\exists}{x}$' \define{binds} all instances of the variable `$x$' in the open sentence `$Hx$'.
% We may refer to the variable following the quantifier as the \define{binding variable} and the variable `$x$' in `$Hx$' as a \define{bound variable}.
% Whereas `$x$' is \define{unbound} in the open sentence `$Hx$', the variable `$x$' is \define{bound} in `$\qt{\exists}{x} Hx$'.
% Since there are no unbound variables in `$\qt{\exists}{x} Hx$', this sentence is \define{closed}.
% As we will see in the following chapter, the closed sentences are the primary targets for interpretation since they may be assigned truth-values independent of an assignment of variables to elements of the domain.
% In the following chapter, we will consider \define{assignments} of variables to individuals where this will allow us to interpret quantified sentences.

In order to regiment sentence \ref{qu2}, we may ask what this sentence says.
One way to read this sentence is that: it is not the case that someone is as guilty as Donald.
Unpacking this claim even further: it is not the case that something is such that it is as guilty as Donald.
However cumbersome, this reading makes the following open sentence plain to see: $Gxd$.
Although $Gxd$ contains the constant $d$, it also contains the free variable $x$ which we may bind with an existential quantifier as before: $\qt{\exists}{x} Gxd$.
We may then negate the result: $\enot \qt{\exists}{x} Gxd$.

Although this provides a fine regimentation of sentence \ref{qu2}, there is another natural reading.
Instead of beginning with negation, we may take sentence \ref{qu2} to express that everyone is not as guilty as Donald.
Put otherwise, everything is such that it is not as guilty as Donald.
Here we begin as before with $Gxd$ but then immediately add negation: $\enot Gxd$.
Note that this sentence is open since it includes the free variable $x$. 
Finally, we add the \define{universal quantifier} $\qt{\forall}{x}$, where this yields: $\qt{\forall}{x} \enot Gxd$.
It turns out that this reading is logically equivalent to the first regimentation of sentence \ref{qu2} considered above.
% Nevertheless, in many cases sentences will admit of multiple regimentations which are not all logically equivalent.

% In these quantified sentences, the variable $x$ serves as a kind of placeholder. The expression $\qt{\forall}{x}$ means that you can pick anything from the domain of discourse and put it in as $x$. There is no special reason to use $x$ rather than some other variable. The sentence $\qt{\forall}{x} Hx$ means exactly the same thing as $\qt{\forall}{y} Hy$, $\qt{\forall}{z} Hz$, and $(\qt{\forall}{x_5}) Hx_5$.
%
% To translate sentence \ref{q.e}, we introduce another new symbol: the \define{existential quantifier}, $\exists$. Like the universal quantifier, the existential quantifier requires a variable. Sentence \ref{q.e} can be translated as $\qt{\exists}{x} Ax$. This means that there is some $x$ which is angry. {\color{black}More precisely, it means that at least one thing in our domain of discourse is angry. If our domain of discourse is people, then} it means that there is \emph{at least one} angry person. Once again, the variable is a kind of placeholder; we could just as easily have translated sentence \ref{q.e} as $\qt{\exists}{z} Az$.
%
% Consider these further sentences:
% \begin{earg}
% \item[\ex{q.ne}] No one is angry.
% \item[\ex{q.en}] There is someone who is not happy.
% \item[\ex{q.na}] Not everyone is happy.
% \end{earg}
%
% Sentence \ref{q.ne} can be paraphrased as, `It is not the case that someone is angry.' This can be translated using negation and an existential quantifier: $\enot \qt{\exists}{x} Ax$. Yet sentence \ref{q.ne} could also be paraphrased as, `Everyone is not angry.' With this in mind, it can be translated using negation and a universal quantifier: $\qt{\forall}{x} \enot Ax$. Both of these are acceptable translations, because they are logically equivalent. The critical thing is whether the negation comes before or after the quantifier.

% The sentence following the quantifier is said to be in the \define{scope} of the quantifier.
% We will provide a formal definition of scope later on, but intuitively it is the part of the sentence that the quantifier is able to .
In general, $\qt{\forall}{x} \metaA$ is logically equivalent to $\enot\qt{\exists}{x}\enot\metaA$, and similarly, $\qt{\exists}{x} \metaA$ is logically equivalent to $\enot\qt{\forall}{x}\enot\metaA$.
This means that any sentence which can be symbolized with a universal quantifier can be symbolized with an existential quantifier together with an appropriate number of negation signs, and \textit{vice versa}.
Even though there is no logical difference in translating with one quantifier rather than the other, sometimes one translation might seem more natural than the other.
Other times, the difference between regimenting with an existential as opposed to a universal quantifier will simply be a matter of taste.

It remains to regiment sentence \ref{qu3}.
This says of everything that it loves something.
Focusing on `it loves something', we may unpack this to mean that there is something where it loves that thing.
Thus we have: everything is such that there is something where it loves that thing.
Notice that without the help of variables, it is easy to lose track of which quantifier is binding which variable.
It is for this reason that the method that we have employed above of unpacking the sentence in stilted English is of limited use.
Nevertheless, we do need to make sure we know what the claim is saying, and sometimes it helps to write it out a few different ways in English.
Often the best method is to proceed in stages.

We may begin by taking sentence \ref{qu3} to say that something is such that it loves somebody.
Thus we have: $\qt{\forall}{x} LS x$ where we have temporarily introduced `$LSx$' for `$x$ loves somebody'.
Next we may take `$LS x$' to mean `there is some $y$ where $x$ loves $y$' which we may regiment: $\qt{\exists}{y} Lxy$.
Replacing `$LS x$' in our first pass regimentation with `$\qt{\exists}{y} Lxy$' yields the final product: $\qt{\forall}{x} \qt{\exists}{y} Lxy$.
In general, sentences with mixed quantifiers are tricky and require a lot of care to get right.
Even if some cases you can immediately intuit a correct regimentation, it is important to learn how to systematically decompose complex sentences into simpler parts before putting the pieces back together.
This will take some practice.

Having introduced each of the new primitives included in QL, we may proceed to define the well formed formulas as well as the sentences of QL a little more carefully.





\section{The Well-Formed Formulas of QL}

We have already seen each of the primitive symbols included in QL.
It remains to draw on these elements in order to define the well-formed formulae of QL, as well as the sentences of QL.
In many respects this will resemble the treatment of the well-formed formulas of SL, however, in other respects these two languages will differ.
In particular, we will not identify the well-formed formulas of QL with the sentences of QL.

Collecting the elements that we have introduced so far, the primitive symbols of QL include:


\begin{center}
\begin{tabular}{|c|c|}
\hline
$n$-place predicates for $n\geq 0$ & $A^n,B^n,C^n,\ldots,Z^n$\\
with subscripts, as needed & $A_1^n, B_1^n, Z_1^n, A_2^n, A_{25}^n, J_{375}^n,\ldots$\\
\hline
constants & $a,b,c,\ldots,v$\\ % %Josh: Ichikawa language goes up to `w'
with subscripts, as needed & $a_1, w_4, h_7, m_{32},\ldots$\\
\hline
variables & $w, x,y,z$\\ % % JH: Ichikawa language excludes `w' as a variable 
with subscripts, as needed & $x_1, y_1, z_1, x_2,\ldots$\\
\hline
sentential connectives & \enot, \eand, \eor, \eif, \eiff\\
\hline
quantifiers& $\forall, \exists$\\
\hline
parentheses&( , )\\
\hline
\end{tabular}
\end{center}

% \begin{enumerate}[leftmargin=1.5in]
%   \item[\it Constants:] $a, b, c,\ldots$. 
%   \item[\it Variables:] $x, y, z,\ldots$.
%   \item[\it Predicates:] $A_m^n,\ldots, Z_m^n$ for all $n,m\geq 0$.
%   \item[\it Quantifiers:] $\forall,\exists$.
%   \item[\it Connectives:] $\enot, \eand, \eor, \eif, \eiff$.
%   \item[\it Parentheses:] $(,)$.
% \end{enumerate}

Just as we did for SL, we will define the \define{well-formed formulas} (wffs) of QL recursively:

\begin{enumerate}
  \item $\F^n\alpha_1,\ldots,\alpha_n$ is a wff if $\F^n$ is an $n$-place predicate and $\alpha_1,\ldots,\alpha_n$ are singular terms.
\item If \metaA{} and \metaB{} are wffs and $\alpha$ is a variable, then:
	\begin{enumerate}
    % \begin{multicols}{2}
      \item $\qt{\exists}{\alpha}\metaA{}$ is a wff;
      \item $\qt{\forall}{\alpha}\metaA{}$ is a wff;
      \item $\enot\metaA{}$ is a wff;
      % \item[] ~
      \item $(\metaA{}\eand\metaB{})$ is a wff;
      \item $(\metaA{}\eor\metaB{})$ is a wff;
      \item $(\metaA{}\eif\metaB{})$ is a wff; and
      \item $(\metaA{}\eiff\metaB{})$ is a wff.
    % \end{multicols}
	\end{enumerate}
\item Nothing else is a wff.
\end{enumerate}

As brought out in Ch.~\ref{ch.SL}, the definitions above are strictly speaking nonsense.
This is because the clauses above \textit{use} rather than \textit{mention} the meta-variables $\metaA$ and $\metaB$ for wffs, as well as new meta-variables for $n$-place predicates $\F^n$ and singular terms $\alpha_1,\ldots,\alpha_n$.
Moreover, it would not help to simply add quotes since the meta-variables that we have used above do not belong to QL.
This is what motivated the introduction of corner quotes before, and we may reproduce a similar tactic here.
However, instead of doing so, we will follow the common convention of relying on the reader to know where the corner quotes are supposed to go.
If you do not remember how to do this, look back to $\S\ref{sec:quotation}$ for review. 

It is common to refer to the clauses included in the definition of the wffs of QL as the \define{composition rules} for the wffs of QL since they tell us how to \textit{compose} wffs from the primitive symbols included in QL.
In particular, the first clauses allows us to construct \define{atomic wffs} where the other clauses allow us to construct \define{complex wffs}.
Since we have already seen a number of paradigm cases of wffs in QL, it will help to consider some examples which do not follow the composition rules above.
For the sake of readability, we will drop subscripts and superscripts below and throughout much of what follows:

\begin{earg}
  \item[\ex{sen1}] $aFbx$.
  \item[\ex{sen2}] $\qt{\forall}{a}Fbx$.
  \item[\ex{sen3}] $\qt{\forall}{y}Fbx$.
  \item[\ex{sen4}] $\qt{\forall}{x}Fbx$.
  \item[\ex{sen5}] $\qt{\exists}{y} \qt{\forall}{x}Fbx$.
  \item[\ex{sen6}] $GFbx$.
  \item[\ex{sen7}] $G \eand Fbx$.
\end{earg}

The expression in \ref{sen1} is complete nonsense, or to use the definition given above, \ref{sen1} is \textit{not} a wff of QL.
The reason is that a constant $a$ occurs before the predicate $F$, but there is no way to achieve this given the composition rules provided above.
Were we to remove this misplaced constant, the result would be $Fbx$ which is a wff of QL. 

The expression in \ref{sen2} is not a wff of QL since the constant $a$ follows the quantifier $\forall$.
There is no way to achieve this given our composition rules, and for good reason.
For instance, consider the English analogue `for every Sally, Jim loves it'.
Although we might be able to contrive a way to make sense of this in English--- something natural languages are very good at--- our initial point remains: there is no way to construct \ref{sen2} given the composition rules for QL, and so \ref{sen2} is not a wff of QL.

The expression in \ref{sen3} is a wff of QL.
Perhaps this comes as a surprise.
After all, the binding variable $y$ differs from what we might expect to be the bound variable $x$.
Accordingly, the quantifier does not make any substantive contribution to the sentence.
Be this as it may, we may construct $\ref{sen3}$ all the same.
We may simulate this effect in English by saying: everything is such that Jim loves Raha.
Here, the quantifier `everything is such that' does no substantive work, where something similar may be said for $\qt{\forall}{y}$ in \ref{sen3}.

The expression in \ref{sen4} is also a wff of QL, but this time the quantifier succeeds in binding the variable $x$.
For instance, we might use this wff to regiment the claim `Everything is such that Jim loves it', or more naturally, `Jim loves everything'.

The expression in \ref{sen5} is a wff of QL, but includes an extra quantifier that does no work.
This is because the binding variable $y$ does not bind any variables. 
Perhaps surprisingly, the very same may be said were we to replace $y$ with $x$.
We will have more to say about such cases in the following subsection where we will introduce the notion of \textit{scope}.

The expression in \ref{sen6} is not a wff of QL since it includes two concatenated predicates $F$ and $G$.
There is no more latitude for this in QL than there is for concatenating two sentence letters in SL.
After all, this would be like saying in English: is red Jim loves it. 

The expression in \ref{sen7} is not a wff of QL, but only for the pedantic reason that we have not included parentheses.
Leaving such pedantry to the side, we will go on dropping outermost parentheses when no ambiguity results.
For instance, were to wish to bind the variable $x$ with a quantifier, parentheses would be required.
Although both $\qt{\forall}{x} G \eand Fbx$ and $\qt{\forall}{x}(G \eand Fbx)$ are wffs of QL, the quantifier only succeeds in binding the variable $x$ in the latter.






\section{Quantifier Scope}

Given a quantified wff of QL with the form $\qt{\exists}{\alpha}\metaA$ or $\qt{\forall}{\alpha}\metaA$, we may define $\metaA$ to be the \define{scope} of each of these quantifiers.
% existential quantifier $\qt{\exists}{\alpha}$ in $\qt{\exists}{\alpha} \metaA$, and similarly, $\metaA$ is the \define{scope} of the universal quantifier $\qt{\forall}{\alpha}$ in $\qt{\forall}{\alpha} \metaA$.
In considering the composition rules above, the scope of a quantifier is the wff to which that quantifier is applied.
Although it is easy to identify the scope of a quantifier when the quantifier is the main connective, this is not always the case.

In the sentence $\qt{\exists}{x} Gx \eif Gl$, the scope of the existential quantifier is the expression $Gx$.
Would it make a difference if the scope of the quantifier were the whole sentence as in $\qt{\exists}{x}(Gx \eif Gl)$?
In order to answer this question, consider the following symbolization key:

\begin{ekey}
\item[Gx:] $x$ is a guitarist.
\item[l:] Lenny.
\end{ekey}

Given this key above, $\qt{\exists}{x} Gx \eif Gl$ reads: if there is some guitarist, then Lenny is a guitarist.
By contrast, $\qt{\exists}{x} (Gx \eif Gl)$ reads: there is some person such that if that person were a guitarist, then Lenny would be a guitarist.
Recall that the conditional here is a material conditional, and so true any time the antecedent is false.
Let the constant $j$ denote Jack who we may assume is not a guitarist.
It follows that the sentence $Gj \eif Gl$ is true because $Gj$ is false.
Since Jack is such that if he is a guitarist then Lenny is a guitarist, it follows that $\qt{\exists}{x} (Gx \eif Gl)$ is true.
The sentence is true because there is a non-guitarist, regardless of whether Lenny is a guitarist.
This may strike you as strange.

% Something strange happened when we changed the scope of the quantifier to range over the conditional.
It turns out that $\qt{\exists}{x} Gx \eif Gl$ is logically equivalent to $\qt{\forall}{x} (Gx \eif Gl)$, and $\qt{\exists}{x} (Gx \eif Gl)$ is logically equivalent to $\qt{\forall}{x} Gx \eif Gl$.
This oddity does not arise with other connectives, nor does it arise if the variable only occurs in the consequent.
For example, $\qt{\exists}{x} Gx \eand Gl$ is logically equivalent to $\qt{\exists}{x} (Gx \eand Gl)$, and $Gl \eif \qt{\exists}{x} Gx$ is logically equivalent to $\qt{\exists}{x}(Gl \eif Gx)$.
What this brings out is yet another unusual features of the material conditional.

% Note that quantifiers count as logical connectives, so one can sensibly ask whether the main connective of a given sentence is a quantifier or something else. (Calling it a `connective' can be slightly confusing, since, unlike connectives like conjunction and disjunction, it doesn't literally \emph{connect} two sentences. Quantifiers are like negations in this respect --- each does count as a connective.) If the scope of the quantifier is the entire sentence, then that quantifier is the main connective, as in $\qt{\forall}{x} (Gx \eif Gl)$. If the scope of the quantifier is limited to a subsentence, then that quantifier is not the main connective. For example, in $\qt{\exists}{x} Gx \eif Gl$, the main connective is the conditional; the existential is part of the antecedent.






\subsection{QL Sentences}

Recall that a declarative sentence (or just sentence for short) is an expression that can either be true or false.
In SL, every wff was a sentence insofar as it made sense to interpret every wff of SL.
But this is not true for many of the wffs which QL includes.
For instance, consider the following symbolization key:

\begin{ekey}
\item[\bf Lxy:] $x$ loves $y$
\item[\bf b:] Boris
\end{ekey}

The expression $Lzz$ is an atomic wff since a two-place predicate is followed by two singular terms, in this case both of which are instances of the variable $z$.
We may then ask what it would mean for $Lzz$ to be true. 
For instance, perhaps it means that $z$ is self loving in the same way that $Lbb$ means that Boris loves himself.
However, since $z$ is a variable, it does not name something the way that a constant like $b$ does.
Whereas there is a clear way to interpret constants, there is no equally determinate way to interpret variables.
% It is for this reason that we will introduce \textit{assignments} in the following chapter where an assignment fixes the reference of variables since we cannot rely on the interpretation of QL to do so.

% In order to make sense of a variable, we need a quantifier to tell us how to interpret that variable.
In order to be able to say which wffs of QL can be interpreted and which cannot, it will help to provide precise definitions of some of the notions that we made intuitive use of above.
To begin with, we may provide the following recursive definition of \define{free variables}:

\begin{enumerate}
  \item $\alpha$ is free in $\F^n\alpha_1,\ldots,\alpha_n$ if $\alpha=\alpha_i$ for some $1\leq i\leq n$ where $\alpha$ is a variable, $\F^n$ is an $n$-place predicate, and $\alpha_1,\ldots,\alpha_n$ are singular terms.
  \item If $\metaA$ and $\metaB$ are wffs and $\alpha$ and $\beta$ are variables, then:
    \begin{enumerate}
        \item $\alpha$ is free in $\qt{\exists}{\beta}\metaA$ if $\alpha$ is free in $\metaA$ and $\alpha\neq\beta$;
        \item $\alpha$ is free in $\qt{\forall}{\beta}\metaA$ if $\alpha$ is free in $\metaA$ and $\alpha\neq\beta$;
        \item $\alpha$ is free in $\enot\metaA$ if $\alpha$ is free in $\metaA$;
        \item $\alpha$ is free in $(\metaA\eand\metaB)$ if $\alpha$ is free in $\metaA$ or $\alpha$ is free in $\metaB$;
        \item $\alpha$ is free in $(\metaA\eor\metaB)$ if $\alpha$ is free in $\metaA$ or $\alpha$ is free in $\metaB$;
        \item $\alpha$ is free in $(\metaA\eif\metaB)$ if $\alpha$ is free in $\metaA$ or $\alpha$ is free in $\metaB$;
        \item $\alpha$ is free in $(\metaA\eiff\metaB)$ if $\alpha$ is free in $\metaA$ or $\alpha$ is free in $\metaB$;
    \end{enumerate}
  \item Nothing else is a free variable. 
\end{enumerate}

Observe the manner in which the definition above follows the same recursive structure as the composition rules for QL.
Given any wffs of the form $\qt{\exists}{\alpha} \metaA$ or $\qt{\forall}{\alpha} \metaA$, we may refer to $\alpha$ as the \define{binding variable} of these quantifiers where every free occurrence of $\alpha$ in $\metaA$ is \define{bound} by the quantifiers $\qt{\exists}{\alpha}$ and $\qt{\forall}{\alpha}$ in $\qt{\exists}{\alpha} \metaA$ and $\qt{\forall}{\alpha} \metaA$, respectively. 
Accordingly, quantifiers \define{bind} all free occurrences of their binding variable which occur within their scope.

An \define{open sentence} of QL is any wff of QL that includes free variables.
A \define{sentence} of QL is any wff of QL which does not include any free variables.
% Letting the \define{degree} of a wff of QL be the number of distinct variables in that wff (excluding multiple instances of the same variable), sentences are of degree zero.
% We may also observe that 
Since there are wffs of QL that include free variables, not all wffs are sentences of QL.
Consider the examples:

\begin{earg}
  \item[\ex{sen8}] $\qt{\forall}{x}\qt{\forall}{x}(Ex \eor Dy) \eif \qt{\exists}{z}(Rzx \eif Lzx).$ 
  \item[\ex{sen9}] $\qt{\forall}{x}(\qt{\forall}{x}(Ex \eor Dy) \eif \qt{\exists}{z}(Rzx \eif Lzx)).$ 
  \item[\ex{sen10}] $\qt{\exists}{y}\qt{\forall}{x}(\qt{\forall}{x}(Ex \eor Dy) \eif \qt{\exists}{z}(Rzx \eif Lzx)).$ 
\end{earg}

The scope of the first universal quantifier on the left in \ref{sen8} is $\qt{\forall}{x}(Ex \eor Dy)$.
Although $x$ occurs in $\qt{\forall}{x}(Ex \eor Dy)$, it is a bound occurrence. 
Rather, $y$ is the only free variable in $\qt{\forall}{x}(Ex \eor Dy)$.
Since $y\neq x$, we may observe that $y$ remains free in $\qt{\forall}{x}\qt{\forall}{x}(Ex \eor Dy)$. 
Moving to the consequent of \ref{sen8}, both occurrences of $z$ are bound, and neither occurrence of $x$ are bound.
Accordingly, \ref{sen8} is a wff of QL, but not a sentence of QL.

We may change the scope of the universal quantifier on the far left in \ref{sen8} by adding an additional pair of parentheses as given by \ref{sen9}.
Now the scope of the left most universal quantifier is $\qt{\forall}{x}(Ex \eor Dy) \eif \qt{\exists}{z}(Rzx \eif Lzx)$.
Whereas $x$ is bound by the universal quantifier in the antecedent, $x$ is free in the consequent, and so only these latter occurrences of $x$ are bound by the outermost universal quantifier in \ref{sen9}. 
Nevertheless, $y$ remains free throughout, and so \ref{sen9} is not a sentence of QL though it is a wff of QL.

In order to bind the free occurrence of $y$ in \ref{sen9}, we may add an additional quantifier whose binding variable is $y$ as given by \ref{sen10}.
Since \ref{sen10} does not include any free variables, \ref{sen10} is a sentence of QL and so amenable to interpretation.







\section{Regimentation in QL}

We now have the basic pieces of QL.
Translating more complicated sentences will only be a matter of knowing the right way to combine predicates, constants, quantifiers, variables, and sentential connectives.
Consider these sentences:

\begin{earg}
\item[\ex{quan1}] Every coin in my pocket is a loonie.
\item[\ex{quan2}] Some coin on the table is a dime.
\item[\ex{quan3}] Not all the coins on the table are loonies.
\item[\ex{quan4}] None of the coins in my pocket are dimes.
\end{earg}

In order to regiment these sentences, we may provide the following symbolization key:

\begin{ekey}
\item[Cx:] $x$ is a coin.
\item[Px:] $x$ is in my pocket.
\item[Tx:] $x$ is on the table.
\item[Lx:] $x$ is a loonie.
\item[Dx:] $x$ is a dime.
\end{ekey}

Sentence \ref{quan1} is most naturally translated with a universal quantifier.
The universal quantifier says something about everything, not just about coins, or the coins in my pocket.
Accordingly, we may take \ref{quan1} to say that, for anything, \textit{if} it is a coin and in my pocket, \textit{then} it is a loonie.
So we can translate it as $\qt{\forall}{x}((Cx \eand Px) \eif Lx)$.

Since sentence \ref{quan1} is about coins that are both in my pocket \emph{and} that are loonies, it might be tempting to translate it using a conjunction.
However, the sentence $\qt{\forall}{x}((Cx \eand Px) \eand Lx)$ would mean that everything is a coin in my pocket and a loonie.
It would also be wrong to regiment sentence \ref{quan1} as $\qt{\forall}{x}(Cx \eif (Px \eand Lx))$ since this say that everything that is a coin is in my pocket and a loonie. 
This is a very strong claim, and is unlikely to be true since there are a lot of coins out there which are neither in my pocket or loonies.

These examples bring out the idea of \define{restricting} a universal quantifier.
Since saying something about everything in an unrestricted way is only very rarely something that we intend to do, universal claims almost always take the following form:

\begin{earg}
\item[\ex{quan5}] $\forall\alpha(\metaA(\alpha) \eif \metaB).$ 
\end{earg}

Here $\metaA(\alpha)$ is a wff of QL in which $\alpha$ occurs as a free variable, and so $\alpha$ as it occurs in $\metaA(\alpha)$ is bound by the quantifier $\forall\alpha$.
Accordingly, the sentence \ref{quan5} says that $\metaB$ holds of everything that satisfies the condition $\metaA$.
Although we have not required $\metaB$ to also include $\alpha$ as a free variable, it is typical that $\metaB$ would include free occurrences of $\alpha$.
For instance, in the case above, we compared taking $Cx \eand Px$ to restrict the quantifier with taking just $Cx$ to restrict the quantifier. 
Whereas the former allows us to make universal claims about just the coins in my pocket, the latter allows us to make claims about all coins.

Sentence \ref{quan2} is most naturally translated with an existential quantifier.
It says that there is some coin which is both on the table and which is a dime.
So we can translate it as $\qt{\exists}{x}(Cx \eand (Tx \eand Dx))$.
Notice that we did not use a conditional to restrict the existential quantifier in the same way as we did with the universal quantifier.
Instead, we used conjunction to say that there is at least one thing which is a coin and moreover it is on the table and is a dime.
Using conjunction with an existential quantifier is common pattern.

What would it mean to write $\qt{\exists}{x}(Cx \eif (Tx \eand Dx))$?
This says that there is something which is a dime on the table \textit{if it is a coin}.
Suppose that there are no coins that are dimes on the table, but there is at least one thing which is not a coin, say the planet Jupiter which we will symbolize by the constant $j$.
Since the planet Jupiter is not a coin, it follows that $Cj \eif (Tj \eand Dj)$ is true since the antecedent $Cj$ is false. 
As a result, $\qt{\exists}{x}(Cx \eif (Tx \eand Dx))$ is true even though we have assumed that there are no coins that are dimes on the table.
More generally, whenever there is something that is not a coin, $\qt{\exists}{x}(Cx \eif (Tx \eand Dx))$ will be true, making this an extremely weak claim.
Although the conditional is often used to restrict a universal quantifier, a conditional within the scope of an existential quantifier results in extremely week claims which we almost never intend to assert.
Accordingly, it's a good rule of thumb that you shouldn't be putting conditionals in the scope of existential quantifiers.

Sentence \ref{quan3} can be paraphrased as, `It is not the case that every coin on the table is a loonie'.
So we can translate it as $\enot \qt{\forall}{x}((Cx \eand Tx) \eif Lx)$.
Alternatively, we paraphrase sentence \ref{quan3} as, `Some coin on the table is not a loonie'.
We would then translate sentence \ref{quan3} as $\qt{\exists}{x}((Cx \eand Tx) \eand \enot Lx)$.
These two translations are logically equivalent.
More generally, $\enot\qt{\forall}{x}\metaA{}$ and $\qt{\exists}{x}\enot\metaA{}$ are logically equivalent, as are $\enot(\metaA{}\eif\metaB{})$ and $\metaA{}\eand\enot\metaB{}$.
This is something that we will return to in the following chapter.
% Whereas we already have resources to establish the latter equivalence, the next chapter will allow us to also prove the former equivalence.

Sentence \ref{quan4} can be paraphrased as, `It is not the case that there is a coin in my pocket that is a dime'.
This can be regimented by $\enot\qt{\exists}{x}(Cx \eand (Px \eand Dx))$.
It might also be paraphrased as `Every coin in my pocket is not a dime' which we may regiment by $\qt{\forall}{x}((Cx \eand Px) \eif \enot Dx)$.
These two translations are logically equivalent, and so both correctly regiment sentence \ref{quan4}.






\section{Paraphrasing Pronouns}

When regimenting English sentences in QL, it is often helpful to paraphrase the sentence in English in a manner that exposes the logical features of that sentence.
We have already provided a number of examples of this above.
When paraphrasing, it is important that you do not accidentally make changes to the logical structure of the sentence since mistakes at this stage will end up resulting in the wrong regimentation.
% Sometimes it will help to move through multiple stages paraphrasing a single sentence.
% Each successive paraphrase should move from the original sentence closer to something that you can directly regiment in QL.

Paraphrasing often requires departing from the superficial structure that the sentence has.
For instance, consider the following symbolization key:

\begin{ekey}
  \item[Sx:] $x$ is in session.
  \item[Ax:] $x$ is full of activity.
  \item[m:] MIT
\end{ekey}

Now consider these sentences:

\begin{earg}
  \item[\ex{pronoun1}] If MIT is in session, then it is full of activity.
  \item[\ex{pronoun2}] If some institute is in session, then it is full of activity.
\end{earg}

Sentence \ref{pronoun1} and sentence \ref{pronoun2} have the same words in the consequent, but they cannot be regimented in the same way.
This is because the `it' in sentence \ref{pronoun1} is not bound by a quantifier but rather indicates the same subject while avoiding repetition.
By contrast, both occurrences of `it' in sentence \ref{pronoun2} are bound by the outermost quantifier.
Whereas there is nothing to change about sentence \ref{pronoun2}, we may paraphrase sentence \ref{pronoun1} as follows:

\begin{earg}
  \item[\ex{pronoun3}] If MIT is in session, then MIT is full of activity.
\end{earg}

Compare the following regimentations of sentences \ref{pronoun1} and \ref{pronoun3} respectively:

\begin{earg}
  \item[\ex{pronoun4}] $Sm \eif Am$.
  \item[\ex{pronoun5}] $Sm \eif Ax$.
\end{earg}

Although it might be tempting to try to regiment the `it' in the original sentence \ref{pronoun1} by a variable, this would result in the open sentence \ref{pronoun5} in place of the sentence \ref{pronoun4}.
Since the original sentence says something entirely about MIT and nothing about some yet to be bound variable the way `It is red' does, sentence \ref{pronoun4} provides a better regimentation than \ref{pronoun5} does.




\section{Ambiguous Predicates}

Suppose we want to regiment this sentence:

\begin{earg}
  \item[\ex{amb1}] Adina is a skilled surgeon.
\end{earg}

Consider the following symbolization key:

\begin{ekey}
  \item[Kx:] $x$ is skilled
  \item[Rx:] $x$ is a surgeon
  \item[a:] Adina
\end{ekey}

This yields the following:

\begin{earg}
  \item[\ex{amb2}] $Ka \eand Ra$.
\end{earg}

In English, this reads: Adina is skilled and Adina is a surgeon.
Here one may object that being skilled and also being a surgeon is not the same thing as being a skilled surgeon.
For instance, perhaps it is possible to be both skilled and a surgeon without being a skilled surgeon.
Accordingly, we could have specified the following predicate instead:

\begin{ekey}
  \item[Sx:] $x$ is a skilled surgeon
  \item[a:] Adina
\end{ekey}

We may then provide the following regimentation:

\begin{earg}
  \item[\ex{amb3}] $Sa$.
\end{earg}

Considering sentence \ref{amb1} on its own, it may be unclear whether to go with regimentation \ref{amb2} or \ref{amb3}.
However, in the context of an argument, there may be good reason to go one way rather than the other.
For instance, suppose that we want to regiment this argument:

\begin{earg}
  \item[] The hospital will only hire a skilled surgeon. 
  \item[] Adina is a skilled.
  \item[] Billy is a surgeon, but is not skilled.
  \item[\therefore] Therefore, the hospital will hire Adina and not Billy.
\end{earg}

Here we need to distinguish being a \textit{skilled surgeon} from merely being a \textit{surgeon}.
Consider this symbolization key:

\begin{ekey}
  \item[Gx:] $x$ is greedy
  \item[Hx:] The hospital will hire $x$
  \item[Kx:] $x$ is skilled
  \item[Rx:] $x$ is a surgeon
  \item[a:] Adina
  \item[b:] Billy
\end{ekey}

Now the argument can be regimented this way:

\begin{earg}
\label{surgeon2}
  \item[] $\qt{\forall}{x}(Hx \eif (Kx \eand Rx))$
  \item[] $Ka$
  \item[] $Rb \eand \enot Kb$
  \item[\therefore] $\enot Hb \eand (Ha \eif Ra)$
\end{earg}

Next suppose that we want to translate this argument:

\begin{earg}
\label{surgeon3}
  \item[] Carol is a skilled surgeon and a tennis player.
  \item[\therefore] Therefore, Carol is a skilled tennis player.
\end{earg}

Suppose that we were to use the symbolization key given above along with the following additions:

\begin{ekey}
  \item[Tx:] $x$ is a tennis player
  \item[c:] Carol
\end{ekey}

Then the argument becomes:

\begin{earg}
\item[] $(Rc \eand Kc) \eand Tc$
\item[\therefore] $Tc \eand Kc$
\end{earg}

This argument in QL is valid, but the original argument in English is not.
Something has gone wrong with our regimentation.
The problem is that there is a difference between being \emph{skilled as a surgeon} and \emph{skilled as a tennis player}.
Regimenting this argument correctly requires two separate predicates, one for each type of skill.
Thus we may add the following:

\begin{ekey}
  \item[K$_1$x:] $x$ is skilled as a tennis player
  \item[K$_2$x:] $x$ is skilled as a surgeon
\end{ekey}

We may now regiment the argument in this way:

\begin{earg}
\label{surgeon3correct}
\item[] $(K_2c \eand Rc) \eand Tc$
\item[\therefore] $K_1c \eand Tc$
\end{earg}

Like the English argument it regiments, this QL argument is invalid. 
% Notice that there is no logical connection between $K_2c$ and $Rc$.
% As symbols of QL, they might be any one-place predicates.
% In English there is a connection between being a \textit{surgeon} and being a \textit{skilled surgeon} since every skilled surgeon is a surgeon.
% In order to capture this connection, we symbolize `Carol is a skilled surgeon' as $Rc \eand K_1c$. This means: `Carol is a surgeon and is skilled as a surgeon.'

Similar problems can arise with predicates like `is good', `is big', `is tall', etc.
Just as skilled surgeons and skilled tennis players have different skills, big dogs, big mice, and big problems are all big in different ways.
Must we always distinguish between different ways of being skilled, good, big, or tall? 
No.
As the argument about Billy shows, sometimes we only need to talk about one kind of skill.
If you are translating an argument that is just about dogs, it is fine to use the predicate `$x$ is big'.
However, if the argument is also about mice, it might be important to let use the predicate `$x$ is big for a dog' instead.
In general, we try to introduce as few predicates as possible while nevertheless capturing the intended meaning of the sentence or argument in question.






\section{Multiple Quantifiers}

Consider this following symbolization key and the sentences that follow it:

\begin{ekey}
\item{Dx:} $x$ is a dog.
\item{Fxy:} $x$ is a friend of $y$.
\item{Oxy:} $x$ owns $y$.
\item{f:} Fifi
\item{g:} Gerald
\end{ekey}

\begin{earg}
\item[\ex{dog1}] Fifi is a dog.
\item[\ex{dog2}] Gerald is a dog owner.
\item[\ex{dog3}] Someone is a dog owner.
\item[\ex{dog4}] All of Gerald's friends are dog owners.
\item[\ex{dog5}] Every dog owner is the friend of a dog owner.
\end{earg}

Sentence \ref{dog1} is easy: $Df$.

Sentence \ref{dog2} can be paraphrased as, `There is a dog that Gerald owns.
This can be translated as $\qt{\exists}{x}(Dx \eand Ogx)$.

Sentence \ref{dog3} can be paraphrased as, `There is someone such that that person is a dog owner.'
The subsentence `that person is a dog owner' is just like sentence \ref{dog2}, except that it is about \textit{that person} rather than being about Gerald.
We can save ourselves some writing by paraphrasing \ref{dog3} as `There is some $y$ such that $y$ is a dog owner'.
So we can translate sentence \ref{dog3} as $\qt{\exists}{y} \qt{\exists}{x}(Dx \eand Oyx)$, replacing `is a dog owner' with our previous regimentation. 
%(Although we could swap the $x$s and $y$s, it is important that we use two different variables here.)

Sentence \ref{dog4} can be paraphrased as, `Every friend of Gerald is a dog owner.
We can expand this as, `Everything is such that, if it is a friend of Gerald, then it is a dog owner'.
Translating part of this sentence, we get $\qt{\forall}{x}(Fxg \eif\mbox{$x$ is a dog owner})$.
Again, it is important to recognize that `$x$ is a dog owner' is structurally just like sentence \ref{dog2}.
Since we already have a quantifier binding $x$, we will need a different variable for the existential quantifier.
Any other variable will do.
Using $z$, sentence \ref{dog4} can be translated as $\qt{\forall}{x}(Fxg \eif\qt{\exists}{z}(Dz \eand Oxz))$.

Sentence \ref{dog5} can be paraphrased as `For any $x$ that is a dog owner, there is a dog owner who is $x$'s friend'.
Partially translated, this becomes: 
  $$\qt{\forall}{x}\bigl[\mbox{$x$ is a dog owner}\eif\qt{\exists}{y}(\mbox{$y$ is a dog owner}\eand Fxy)\bigr].$$
Completing the translation, sentence \ref{dog5} becomes: 
  $$\qt{\forall}{x}\bigl[\qt{\exists}{z}(Dz \eand Oxz)\eif\qt{\exists}{y}\bigl(\qt{\exists}{z}(Dz \eand Oyz)\eand Fxy\bigr)\bigr].$$

Regimenting English sentences in QL will take some practice, and often there will be more than one way to go.
When you come up with a regimentation, it is often worth considering if there are any other regimentations that you could have provided.
Often there are, and they are not always logically equivalent.
Sometimes what this means is that the original claim is ambiguous.
Other times, some regimentations will be more natural than that others.

% \iffalse

Consider the following symbolization key and sentences:
\begin{ekey}
\item[Lxy:] $x$ likes $y$.
\item[i:] Imre.
\item[k:] Karl.
\end{ekey}
\begin{earg}
\item[\ex{likes1}]Imre likes everyone that Karl likes.
\item[\ex{likes2}]There is someone who likes everyone who likes everyone that he likes.
\end{earg}

Sentence \ref{likes1} can be partially translated as $\qt{\forall}{x}(\mbox{Karl likes $x$}\eif\mbox{Imre likes $x$})$.
This becomes $\qt{\forall}{x}(Lkx\eif Lix)$.

Sentence \ref{likes2} is complex.
There is little hope of writing down the whole translation immediately, but we can proceed in stages.
An initial, partial translation might look like this: $$\qt{\exists}{x}\ \mbox{everyone who likes everyone that $x$ likes is liked by $x$}$$
The part that remains in English is a universal sentence, so we translate further: $$\qt{\exists}{x}\qt{\forall}{y}(\mbox{$y$ likes everyone that $x$ likes}\eif\mbox{$x$ likes $y$}).$$
The antecedent of the conditional is structurally just like sentence \ref{likes1}, with $y$ and $x$ in place of Imre and Karl.
So sentence \ref{likes2} can be completely translated in this way $$\qt{\exists}{x}\qt{\forall}{y}\bigl[\qt{\forall}{z}(Lxz \eif Lyz) \eif Lxy\bigr]$$

When symbolizing sentences with multiple quantifiers, it is best to proceed by small steps.
Paraphrase the English sentence so that the logical structure is readily regimented in QL.
Then translate piecemeal, replacing the daunting task of translating a long sentence with the simpler task of translating shorter formulae.




\iffalse

\section{Common student errors}

A sentence that says everything with one property also has another property should be translated as a universal governing a conditional. Using the obvious interpretation key:

\begin{itemize}
\item `Every student is working': $\qt{\forall}{x} (Sx \eif Wx)$
\item `Every student has a friend': $\qt{\forall}{x} (Sx \eif \qt{\exists}{y} Fxy)$
\item `Only students with friends are working': $\qt{\forall}{x} [(Sx \eand Wx) \eif \qt{\exists}{y} Fxy]$
\end{itemize}

One common error is to translate sentences of this form with a different kind of shape --- for example, as a universal governing a conjunction, or as an existential governing a conditional. These are very inaccurate translations of these English sentences:

\begin{itemize}
\item $\qt{\forall}{x} (Sx \eand Wx)$
\item $\qt{\forall}{x} (Sx \eand \qt{\exists}{y} Fxy)$
\item $\qt{\forall}{x} [(Sx \eand Wx) \eand \qt{\exists}{y} Fxy]$
\end{itemize}

These say that \emph{every} object in the UD is a student with the properties in question. Everyone is a student that is working; everyone is a student with a friend; everyone is a working student who has a friend. Any time you have a universal governing a conjunction, you are making a very strong claim --- you're not just talking about objects with a particular property, you're saying that multiple things are true about every single object in the domain. Be very careful if you find yourself offering a universal over a conjunction, and make sure you don't mean to use a conditional instead.

It is also a serious mistake to use an existential instead of a universal for sentences like these:

\begin{itemize}
\item $\qt{\exists}{x} (Sx \eif Wx)$
\item $\qt{\exists}{x} (Sx \eif \qt{\exists}{y} Fxy)$
\item $\qt{\exists}{x} [(Sx \eand Wx) \eif \qt{\exists}{y} Fxy]$
\end{itemize}

These are very weak claims. They say that there is some object in the domain that satisfies a certain conditional. For example, $\qt{\exists}{x} (Sx \eif Wx)$ says there is something in the domain such that, if it is a student, it is working. Given the truth conditions for the material conditional, this will be true if there is even one object in the domain that is not a student, regardless of who is and isn't working; it will also be true if there is even one object in the domain that is working, regardless of who is and isn't a student.

If you find yourself offering, as a translation of some English sentence, an existential governing a conditional, you are almost certainly making a mistake. This is not a reasonable translation of any ordinary English sentence. You probably want either a universal over a conditional (everything with one property has another property) or an existential over a conjunction (there is something with the following properties). 




% \iffalse



\practiceproblems

\problempart
\label{pr.QLbojackall}
Using the symbolization key given, translate each English-language sentence into QL. Hint: all of these sentences are well-translated as universals governing conditionals.
\begin{ekey}
\item[UD:] all humans and animals in the world of \emph{Bojack Horseman}
\item[Dx:] $x$ is a dog.
\item[Cx:] $x$ is a cat.
\item[Hx:] $x$ is a horse.
\item[Bx:] $x$ is a human being.
\item[Mx:] $x$ is a movie star.
\item[Lxy:] $x$ lives with $y$.
\item[Rxy:] $x$ represents $y$ (as $y$'s agent).
\item[Wxy:] $x$ worked on a movie with $y$.
\item[b:] Bojack
\item[c:] Princess Caroline
\item[d:] Dianne
\item[p:] Mr.\ Peanutbutter
\end{ekey}
\begin{earg}
\item Every movie star is a dog.
\item Every movie star is a dog or a cat.
\item All dog movie stars live with a human being.
\item Everyone who lives with Dianne is a movie star.
\item Princess Caroline represents every dog movie star.
\item Anyone who worked on a movie with Bojack lives with a movie star.
\item Only humans live with Mr.\ Peanutbutter.
\item Everyone who has ever worked on a movie with Bojack is either a dog, a cat, or someone who lives with a movie star.
\end{earg}

\solutions
\problempart
\label{pr.QLbojacksome}
Using the same symbolization key, translate each English-language sentence into QL. Hint: all of these sentences are well-translated as existentials governing conjunctions.
\begin{earg}
\item Mr.\ Peanutbutter lives with a human.
\item Dianne lives with a dog who worked on a movie with Bojack.
\item Princess Caroline represents a horse who lives with a human being.
\item Some human being who worked on a movie with Mr.\ Peanutbutter lives with a dog or a cat.
\item Bojack worked on a movie with a human movie star.
\item Bojack worked on a movie with a nonhuman movie star who lives with Dianne.
\end{earg}

\problempart
\label{pr.QLbojackother}
Using the same symbolization key, translate each English-language sentence into QL. Hint: these should have different forms than the cases above.
\begin{earg}
\item If Mr.\ Peanutbutter is a movie star, then all dogs are movie stars.
\item A dog who lives with Dianne and Princess Caroline represents a horse.
\item Princess Caroline represents a horse and a dog.
\item Princess Caroline represents everyone.
\item Dianne doesn't live with anyone.
\item No movie star has ever both worked on a movie with Bojack, and worked on a movie with any cat.
\item Princess Caroline is a cat, but she doesn't represent any cats.
\end{earg}



\solutions
\problempart
\label{pr.QLalligators}
Using the symbolization key given, translate each English-language sentence into QL.
\begin{ekey}
\item[UD:] all animals
\item[Ax:] $x$ is an alligator.
\item[Mx:] $x$ is a monkey.
\item[Rx:] $x$ is a reptile.
\item[Zx:] $x$ lives at the zoo.
\item[Lxy:] $x$ loves $y$.
\item[a:] Amos
\item[b:] Bouncer
\item[c:] Cleo
\end{ekey}
\begin{earg}
\item Amos, Bouncer, and Cleo all live at the zoo. 
\item Bouncer is a reptile, but not an alligator. 
\item If Cleo loves Bouncer, then Bouncer is a monkey. 
\item If both Bouncer and Cleo are alligators, then Amos loves them both.
\item Some reptile lives at the zoo. 
\item Every alligator is a reptile. 
\item Any animal that lives at the zoo is either a monkey or an alligator. 
\item There are reptiles which are not alligators.
\item Cleo loves a reptile.
\item Bouncer loves all the monkeys that live at the zoo.
\item All the monkeys that Amos loves love him back.
\item If any animal is a reptile, then Amos is.
\item If any animal is an alligator, then it is a reptile.
\item Every monkey that Cleo loves is also loved by Amos.
\item There is a monkey that loves Bouncer, but Bouncer does not reciprocate this love.
\end{earg}



\problempart
\label{pr.BarbaraEtc}
These are syllogistic figures identified by Aristotle and his successors, along with their medieval names. Translate each argument into QL.
\begin{description}
\item[Barbara] All $B$s are $C$s. All $A$s are $B$s.
	\therefore\  All $A$s are $C$s.
\item[Baroco] All $C$s are $B$s. Some $A$ is not $B$.
	\therefore\  Some $A$ is not $C$.
\item[Bocardo] Some $B$ is not $C$. All $A$s are $B$s.
	\therefore\  Some $A$ is not $C$.
\item[Celantes] No $B$s are $C$s. All $A$s are $B$s.
	\therefore\  No $C$s are $A$s.
\item[Celarent] No $B$s are $C$s. All $A$s are $B$s.
	\therefore\  No $A$s are $C$s.
\item[Cemestres] No $C$s are $B$s. No $A$s are $B$s.
	\therefore\  No $A$s are $C$s.
\item[Cesare] No $C$s are $B$s. All $A$s are $B$s.
	\therefore\  No $A$s are $C$s.
\item[Dabitis] All $B$s are $C$s. Some $A$ is $B$.
	\therefore\  Some $C$ is $A$.
\item[Darii] All $B$s are $C$s. Some $A$ is $B$.
	\therefore\  Some $A$ is $C$.
\item[Datisi] All $B$s are $C$s. Some $A$ is $B$.
	\therefore\  Some $A$ is $C$.
\item[Disamis] Some $B$ is $C$. All $A$s are $B$s.
	\therefore\  Some $A$ is $C$.
\item[Ferison] No $B$s are $C$s. Some $A$ is $B$.
	\therefore\  Some $A$ is not $C$.
\item[Ferio] No $B$s are $C$s. Some $A$ is $B$.
	\therefore\  Some $A$ is not $C$.
\item[Festino] No $C$s are $B$s. Some $A$ is $B$.
	\therefore\  Some $A$ is not $C$.
\item[Baralipton] All $B$s are $C$s. All $A$s are $B$s.
	\therefore\  Some $C$ is $A$.
\item[Frisesomorum] Some $B$ is $C$. No $A$s are $B$s.
	\therefore\  Some $C$ is not $A$.
\end{description}


\solutions
\problempart Using the symbolization key given, translate each English-language sentence into QL.
\label{pr.QLdogtrans}
\begin{ekey}
\item[UD:] all animals
\item[Dx:] $x$ is a dog.
\item[Sx:] $x$ likes samurai movies.
\item[Lxy:] $x$ is larger than $y$.
\item[b:] Bertie
\item[e:] Emerson
\item[f:] Fergis
\end{ekey}
\begin{earg}
\item Bertie is a dog who likes samurai movies.
\item Bertie, Emerson, and Fergis are all dogs.
\item Emerson is larger than Bertie, and Fergis is larger than Emerson.
\item All dogs like samurai movies.
\item Only dogs like samurai movies.
\item There is a dog that is larger than Emerson.
\item If there is a dog larger than Fergis, then there is a dog larger than Emerson.
\item No animal that likes samurai movies is larger than Emerson.
\item No dog is larger than Fergis.
\item Any animal that dislikes samurai movies is larger than Bertie.
\item There is an animal that is between Bertie and Emerson in size.
\item There is no dog that is between Bertie and Emerson in size.
\item No dog is larger than itself.
\item For every dog, there is some dog larger than it.
\item There is an animal that is smaller than every dog.
\end{earg}


\problempart
\label{pr.QLarguments}
For each argument, write a symbolization key and translate the argument into QL.
\begin{earg}
\item Nothing on my desk escapes my attention. There is a computer on my desk. As such, there is a computer that does not escape my attention.
\item All my dreams are black and white. Old TV shows are in black and white. Therefore, some of my dreams are old TV shows.
\item Neither Holmes nor Watson has been to Australia. A person could see a kangaroo only if they had been to Australia or to a zoo. Although Watson has not seen a kangaroo, Holmes has. Therefore, Holmes has been to a zoo.
\item No one expects the Spanish Inquisition. No one knows the troubles I've seen. Therefore, anyone who expects the Spanish Inquisition knows the troubles I've seen.
\item An antelope is bigger than a bread box. I am thinking of something that is no bigger than a bread box, and it is either an antelope or a cantaloupe. As such, I am thinking of a cantaloupe.
\item All babies are illogical. Nobody who is illogical can manage a crocodile. Berthold is a baby. Therefore, Berthold is unable to manage a crocodile.
\end{earg}

\solutions
\problempart
\label{pr.QLcandies}
Using the symbolization key given, translate each English-language sentence into QL.
\begin{ekey}
\item[UD:] candies
\item[Cx:] $x$ has chocolate in it.
\item[Mx:] $x$ has marzipan in it.
\item[Sx:] $x$ has sugar in it.
\item[Tx:] Boris has tried $x$.
\item[Bxy:] $x$ is better than $y$.
\end{ekey}
\begin{earg}
\item Boris has never tried any candy.
\item Marzipan is always made with sugar.
\item Some candy is sugar-free.
\item The very best candy is chocolate.
\item No candy is better than itself.
\item Boris has never tried sugar-free chocolate.
\item Boris has tried marzipan and chocolate, but never together.
\item Any candy with chocolate is better than any candy without it.
\item Any candy with chocolate and marzipan is better than any candy that lacks both.
\end{earg}


\solutions
\problempart
\label{pr.QLpotluck}
Using the symbolization key given, translate each English-language sentence into QL.
\begin{ekey}
\item[UD:] people and dishes at a potluck
\item[Rx:] $x$ has run out.
\item[Tx:] $x$ is on the table.
\item[Fx:] $x$ is food.
\item[Px:] $x$ is a person.
\item[Lxy:] $x$ likes $y$.
\item[e:] Eli
\item[f:] Francesca
\item[g:] the guacamole
\end{ekey}
\begin{earg}
\item All the food is on the table.
\item If the guacamole has not run out, then it is on the table.
\item Everyone likes the guacamole.
\item If anyone likes the guacamole, then Eli does.
\item Francesca only likes the dishes that have run out.
\item Francesca likes no one, and no one likes Francesca.
\item Eli likes anyone who likes the guacamole.
\item Eli likes everyone who likes anyone that he likes.
\item If there is a person on the table already, then all of the food must have run out.
\end{earg}


\solutions
\problempart
\label{pr.QLballet}
Using the symbolization key given, translate each English-language sentence into QL.
\begin{ekey}
\item[UD:] people
\item[Dx:] $x$ dances ballet.
\item[Fx:] $x$ is female.
\item[Mx:] $x$ is male.
\item[Cxy:] $x$ is a child of $y$.
\item[Sxy:] $x$ is a sibling of $y$.
\item[e:] Elmer
\item[j:] Jane
\item[p:] Patrick
\end{ekey}
\begin{earg}
\item All of Patrick's children are ballet dancers.
\item Jane is Patrick's daughter.
\item Patrick has a daughter.
\item Jane is an only child.
\item All of Patrick's daughters dance ballet.
\item Patrick has no sons.
\item Jane is Elmer's niece.
\item Patrick is Elmer's brother.
\item Patrick's brothers have no children.
\item Jane is an aunt.
\item Everyone who dances ballet has a sister who also dances ballet.
\item Every man who dances ballet is the child of someone who dances ballet.
\end{earg}

\problempart
\label{pr.freeQL}
Identify which variables are bound and which are free.
\begin{earg}
\item $\qt{\exists}{x} Lxy \eand \qt{\forall}{y} Lyx$
\item $\qt{\forall}{x} Ax \eand Bx$
\item $\qt{\forall}{x} (Ax \eand Bx) \eand \qt{\forall}{y}(Cx \eand Dy)$
\item $\qt{\forall}{x}\qt{\exists}{y}[Rxy \eif (Jz \eand Kx)] \eor Ryx$
\item $\qt{\forall}{x_1}(Mx_2 \eiff Lx_2x_1) \eand \qt{\exists}{x_2} Lx_3x_2$
\end{earg}

\solutions
\problempart
\label{pr.subinstanceQL}
\begin{earg}
\item Identify which of the following are substitution instances of $\qt{\forall}{x} Rcx$: $Rac$, $Rca$, $Raa$, $Rcb$, $Rbc$, $Rcc$, $Rcd$, $Rcx$
\item Identify which of the following are substitution instances of $\qt{\exists}{x}\qt{\forall}{y} Lxy$:
$\qt{\forall}{y} Lby$, $\qt{\forall}{x} Lbx$, $Lab$, $\qt{\exists}{x} Lxa$
\end{earg}


\fi
