%!TEX root = forallx-ubc.tex
%\chapter{Truth tables}
%\label{ch.TruthTables}


\practiceproblemsA{ch.TruthTables}
If you want additional practice, you can construct truth tables for any of the sentences and arguments in the exercises for the previous chapter.



\problempart
\label{HW2.E}
Provide the truth table for the complex formula:
$$((P \eif (( P \eif Q) \eiff (P \eand \enot R))) \eor R)$$
Indicate whether the formula is tautological, contradictory, or contingent. If it is contingent, provide a model that satisfies it and one that falsifies it.


\begin{tabular}{@{ }c@{ }@{ }c@{ }@{ }c | c@{ }@{}c@{}@{ }c@{ }@{ }c@{ }@{}c@{}@{}c@{}@{ }c@{ }@{ }c@{ }@{ }c@{ }@{}c@{}@{ }c@{ }@{}c@{}@{ }c@{ }@{ }c@{ }@{ }c@{ }@{ }c@{ }@{}c@{}@{}c@{}@{}c@{}@{ }c@{ }@{ }c@{ }@{ }c}
$P$ & $Q$ & $R$ &  & ( & $P$ & $\eif $ & ( & ( & $P$ & $\eif $ & $Q$ & ) & $\eiff $ & ( & $P$ & $\&$ & $\enot$ & $R$ & ) & ) & ) & $\lor$ & $R$ & \\
\hline 
 &  &  &  &  &  &  &  &  &  &  &  &  &  &  &  &  &  &  &  &  &  &  & & \\
 &  &  &  &  &  &  &  &  &  &  &  &  &  &  &  &  &  &  &  &  &  &  & & \\
  &  &  &  &  &  &  &  &  &  &  &  &  &  &  &  &  &  &  &  &  &  &  & & \\
 &  &  &  &  &  &  &  &  &  &  &  &  &  &  &  &  &  &  &  &  &  &  & & \\
 &  &  &  &  &  &  &  &  &  &  &  &  &  &  &  &  &  &  &  &  &  &  & & \\
  &  &  &  &  &  &  &  &  &  &  &  &  &  &  &  &  &  &  &  &  &  &  & & \\
 &  &  &  &  &  &  &  &  &  &  &  &  &  &  &  &  &  &  &  &  &  &  & & \\
  &  &  &  &  &  &  &  &  &  &  &  &  &  &  &  &  &  &  &  &  &  &  & & \\
\end{tabular}




\problempart
\label{HW3.A}
Provide the complete truth table for this SL sentence:
$$((P \eor Q) \eand (\enot P \eor \enot Q)) \eiff R$$
Indicate whether it is tautological, contradictory, or contingent. If it is contingent, provide an assignment of truth values that satisfies it and one that falsifies it.



\solutions
\problempart
\label{pr.TT.TTorC}
Determine whether each sentence is a tautology, a contradiction, or a contingent sentence. Justify your answer with a complete or partial truth table where appropriate.
\begin{earg}
\item $A \eif A$ %taut
\item $\enot B \eand B$ %contra
\item $C \eif\enot C$ %contingent
\item $\enot D \eor D$ %taut
\item $(A \eiff B) \eiff \enot(A\eiff \enot B)$ %tautology
\item $(A\eand B) \eor (B\eand A)$ %contingent
\item $(A \eif B) \eor (B \eif A)$ % taut
\item $\enot[A \eif (B \eif A)]$ %contra
\item $(A \eand B) \eif (B \eor A)$  %taut
\item $A \eiff [A \eif (B \eand \enot B)]$ %contra
\item $\enot(A \eor B) \eiff (\enot A \eand \enot B)$ %taut
\item $\enot(A\eand B) \eiff A$ %contingent
\item $\bigl[(A\eand B) \eand\enot(A\eand B)\bigr] \eand C$ %contradiction
\item $A\eif(B\eor C)$ %contingent
\item $[(A \eand B) \eand C] \eif B$ %taut
\item $(A \eand\enot A) \eif (B \eor C)$ %tautology
\item $\enot\bigl[(C\eor A) \eor B\bigr]$ %contingent
\item $(B\eand D) \eiff [A \eiff(A \eor C)]$%contingent
\end{earg}


% Chapter 3 Part D
\solutions
\problempart
\label{pr.TT.equiv}
Determine whether each pair of sentences is logically equivalent. Justify your answer with a complete or partial truth table where appropriate.
\begin{earg}
\item $A$, $\enot A$ %No
\item $A$, $A \eor A$ %Yes
\item $A\eif A$, $A \eiff A$ %No
\item $A \eor \enot B$, $A\eif B$ %No
\item $A \eand \enot A$, $\enot B \eiff B$ %Yes
\item $\enot(A \eand B)$, $\enot A \eor \enot B$ %Yes
\item $\enot(A \eif B)$, $\enot A \eif \enot B$ %No
\item $(A \eif B)$, $(\enot B \eif \enot A)$ %Yes
\item $[(A \eor B) \eor C]$, $[A \eor (B \eor C)]$ %Yes
\item $[(A \eor B) \eand C]$, $[A \eor (B \eand C)]$ %No
\end{earg}

% Chapter 3 Part E
\solutions
\problempart
\label{pr.TT.consistent}
Determine whether each set of sentences is consistent or inconsistent. Justify your answer with a complete or partial truth table where appropriate.
\begin{earg}
\item $A\eif A$, $\enot A \eif \enot A$, $A\eand A$, $A\eor A$ %consistent
\item $A \eand B$, $C\eif \enot B$, $C$ %inconsistent
\item $A\eor B$, $A\eif C$, $B\eif C$ %consistent
\item $A\eif B$, $B\eif C$, $A$, $\enot C$ %inconsistent
\item $B\eand(C\eor A)$, $A\eif B$, $\enot(B\eor C)$  %inconsistent
\item $A \eor B$, $B\eor C$, $C\eif \enot A$ %consistent
\item $A\eiff(B\eor C)$, $C\eif \enot A$, $A\eif \enot B$ %consistent
\item $A$, $B$, $C$, $\enot D$, $\enot E$, $F$ %consistent
\end{earg}




\problempart
\label{HW3.B}

Use a complete truth table to evaluate this argument form for validity:

\begin{earg}
\item[] $(P \eif Q) \eor (Q \eif P)$
\item[] $P$
\item[\therefore] $P\eiff Q$
\end{earg}

Indicate whether it valid or invalid. If it is invalid, provide an interpretation that satisfies the premises and falsifies the conclusion. 




\solutions
\problempart
\label{pr.TT.valid}
Determine whether each argument form is valid or invalid. Justify your answer with a complete or partial truth table where appropriate.
\begin{earg}
\item $A\eif A$ \therefore\ $A$ %invalid
\item $A\eor\bigl[A\eif(A\eiff A)\bigr]$ \therefore\ A %invalid
\item $A\eif(A\eand\enot A)$ \therefore\ $\enot A$ %valid
\item $A\eiff\enot(B\eiff A)$ \therefore\ $A$ %invalid
\item $A\eor(B\eif A)$ \therefore\ $\enot A \eif \enot B$ %valid
\item $A\eif B$, $B$ \therefore\ $A$ %invalid
\item $A\eor B$, $B\eor C$, $\enot A$ \therefore\ $B \eand C$ %invalid
\item $A\eor B$, $B\eor C$, $\enot B$ \therefore\ $A \eand C$ %valid
\item $(B\eand A)\eif C$, $(C\eand A)\eif B$ \therefore\ $(C\eand B)\eif A$ %invalid
\item $A\eiff B$, $B\eiff C$ \therefore\ $A\eiff C$ %valid
\end{earg}

\solutions
\problempart
\label{pr.TT.concepts}
Answer each of the questions below and justify your answer.
\begin{earg}
\item Suppose that \metaA{} and \metaB{} are logically equivalent. What can you say about $\metaA{}\eiff\metaB{}$?
%\metaA{} and \metaB{} have the same truth value on every line of a complete truth table, so $\metaA{}\eiff\metaB{}$ is true on every line. It is a tautology.
\item Suppose that $(\metaA{}\eand\metaB{})\eif\metaC{}$ is contingent. What can you say about the argument ``\metaA{}, \metaB{}, \therefore\metaC{}''?
%The sentence is false on some line of a complete truth table. On that line, \metaA{} and \metaB{} are true and \metaC{} is false. So the argument is invalid.
\item Suppose that $\{\metaA{},\metaB{}, \metaC{}\}$ is inconsistent. What can you say about $(\metaA{}\eand\metaB{}\eand\metaC{})$?
%Since there is no line of a complete truth table on which all three sentences are true, the conjunction is false on every line. So it is a contradiction.
\item Suppose that \metaA{} is a contradiction. What can you say about the argument ``\metaA{}, \metaB{}, \therefore\metaC{}''?
%Since \metaA{} is false on every line of a complete truth table, there is no line on which \metaA{} and \metaB{} are true and \metaC{} is false. So the argument is valid.
\item Suppose that \metaC{} is a tautology. What can you say about the argument ``\metaA{}, \metaB{}, \therefore\metaC{}''?
%Since \metaC{} is true on every line of a complete truth table, there is no line on which \metaA{} and \metaB{} are true and \metaC{} is false. So the argument is valid.
\item Suppose that \metaA{} and \metaB{} are logically equivalent. What can you say about $(\metaA{}\eor\metaB{})$?
%Not much. $(\metaA{}\eor\metaB{})$ is a tautology if \metaA{} and \metaB{} are tautologies; it is a contradiction if they are contradictions; it is contingent if they are contingent.
\item Suppose that \metaA{} and \metaB{} are \emph{not} logically equivalent. What can you say about $(\metaA{}\eor\metaB{})$?
%\metaA{} and \metaB{} have different truth values on at least one line of a complete truth table, and $(\metaA{}\eor\metaB{})$ will be true on that line. On other lines, it might be true or false. So $(\metaA{}\eor\metaB{})$ is either a tautology or it is contingent; it is \emph{not} a contradiction.
\end{earg}

\problempart
\phantomsection\label{pr.altConnectives}
We could leave the biconditional (\eiff) out of the language. If we did that, we could still write `$A\eiff B$' so as to make sentences easier to read, but that would be shorthand for $(A\eif B) \eand (B\eif A)$. The resulting language would be formally equivalent to SL, since $A\eiff B$ and $(A\eif B) \eand (B\eif A)$ are logically equivalent in SL. If we valued formal simplicity over convenience, we could replace more of the connectives with notational conventions and still have a language equivalent to SL. 
%Ichikawa has `formal simplicity over expressive richness' but there's actually no difference in expressive power in these cases! just in convenience/ease

There are a number of equivalent languages with only two connectives. It would be enough to have only negation and the material conditional. Show this by writing sentences that are logically equivalent to each of the following using only parentheses, sentence letters, negation (\enot), and the material conditional (\eif).
\begin{earg}
\item\leftsolutions\ $A\eor B$
%$\enot A \eif B$
\item\leftsolutions\ $A\eand B$
%$\enot(A \eif \enot B)$
\item\leftsolutions\ $A\eiff B$
%$\enot [(A\eif B) \eif \enot(B\eif A)]$
\end{earg}
%...
% Break out of the {earg} environment to give new instructions. 

We could have a language that is equivalent to SL with only negation and disjunction as connectives. Show this: Using only parentheses, sentence letters, negation (\enot), and disjunction (\eor), write sentences that are logically equivalent to each of the following.
% Resume the {earg} environment and restore the counter.
%...
\begin{earg}
\setcounter{eargnum}{\arabic{OLDeargnum}}
\item $A \eand B$
%$\enot(\enot A \eor \enot B)$
\item $A \eif B$
%$\enot A \eor B$
\item $A \eiff B$
%$\enot(\enot A \eor \enot B) \eor \enot(A \eor B)$
\end{earg}
%...
The \emph{Sheffer stroke} is a logical connective with the following characteristic truthtable:
\begin{center}
\begin{tabular}{c|c|c}
\metaA{} & \metaB{} & \metaA{}$|$\metaB{}\\
\hline
1 & 1 & 0\\
1 & 0 & 1\\
0 & 1 & 1\\
0 & 0 & 1
\end{tabular}
\end{center}
%...
\begin{earg}
\setcounter{eargnum}{\arabic{OLDeargnum}}
\item Write a sentence using the connectives of SL that is logically equivalent to $(A|B)$.
\end{earg}
%...
Every sentence written using a connective of SL can be rewritten as a logically equivalent sentence using one or more Sheffer strokes. Using no connectives other than the Sheffer stroke, write sentences that are equivalent to each of the following. 
%...
\begin{earg}
\setcounter{eargnum}{\arabic{OLDeargnum}}
\item $\enot A$
\item $(A\eand B)$
\item $(A\eor B)$
\item $(A\eif B)$
\item $(A\eiff B)$
\end{earg}


\problempart
\label{HW2.F}


The connective `\eor' indicates \emph{inclusive disjunction}; it is true if either \emph{or both} of the disjuncts is true. One might be interested in \emph{exclusive disjunction}, which requires that exactly one of the disjuncts be true. Let us temporarily extend SL to include a connective for exclusive disjunction, allowing sentences of the form $(\metaA{} \oplus \metaB{})$, where \metaA{} and \metaB{} are sentences, meaning that exactly one of \metaA{} and \metaB{} are true.
	\begin{earg}
		\item Provide the truth table for $(\metaA{} \oplus \metaB{})$.
		
		
		\begin{tabular}{@{ }c@{ }@{ }c | c@{ }@{ }c@{ }@{ }c@{ }@{ }c@{ }@{ }c}
$\metaA{}$ & $\metaB{}$ &  & $\metaA{}$ & $\oplus$ & $\metaB{}$ & \\
\hline 
 &  &  &  &  & & \\
 &  &  &  &  &  & \\
 &  &  &  &  &  & \\
 &  &  &  &  &  & \\
\end{tabular}
		
		
%				\begin{tabular}{@{ }c@{ }@{ }c | c@{ }@{ }c@{ }@{ }c@{ }@{ }c@{ }@{ }c}
%$P$ & $Q$ &  & $P$ & $\oplus$ & $Q$ & \\
%\hline 
%1 & 1 &  &  & \textcolor{red}{1} & & \\
%1 & 0 &  &  & \textcolor{red}{1} &  & \\
%0 & 1 &  &  & \textcolor{red}{1} &  & \\
%0 & 0 &  &  & \textcolor{red}{0} &  & \\
%\end{tabular}
		
		
		\item $\oplus$ is \emph{definable} in terms of \eor, \eand, and \enot. This means that there is a formula using only these latter connectives that is equivalent to --- true in all the same models as --- $(P \oplus Q)$. Provide such a formula.
		\item Using truth tables, prove that the formula provided in the last question is equivalent to $(P \oplus Q)$.
	\end{earg}


