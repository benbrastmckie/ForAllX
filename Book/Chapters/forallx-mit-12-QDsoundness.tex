%!TEX root = forallx-mit.tex
\chapter{The Soundness of QD$^=$}
  \label{ch.SoundQD}

In Ch~\ref{ch.SLsoundcomplete}, we showed that the tree method was sound and complete with respect to the semantics for SL.
In particular, soundness showed that whenever a tree with root $\Gamma\cup\set{\enot\metaA}$ closed, we could conclude that $\Gamma\cup\set{\enot\metaA}$ is unsatisfiable, and so $\Gamma \models \metaA$.
As a result, we could rely on the tree method to evaluate whether the premises of an argument entailed its conclusion.
The completeness result concerned the converse, showing that whenever some premises entail a conclusion, the tree that results from taking those premises together with the negation of the conclusion will close.
Accordingly, the tree method can always be relied upon to evaluate whether some premises entail their conclusion.

Although these are important results if we are to make use of the tree method, proof trees are of little independent interest.
Moreover, when it comes to extending the tree method to accommodate quantifiers, there is no guarantee that the tree will always complete.
This makes the tree method considerably less useful.
Additionally, proving soundness and completeness for the tree method does not tell us anything about our natural deduction systems SD or QD, and this is something we would like to understand.
Even if SD and QD do a good job of modelling natural reasoning in SL and QL$^=$ respectively, we should like to know whether the reasoning that we can conduct in the proof systems tells us anything about entailment and, subsequently, the validity of arguments.
Specifically, we should like to know the following: 
  $$\Gamma \proves_{\textsc{qd}} \metaA \text{ just in case } \Gamma \models \metaA.$$
Whereas the right-to-left direction is completeness, the left to right direction is soundness.
If QD were to fail to be sound, we have little reason to care about QD for we could not rely on it to establish valid arguments.
Showing that QD is sound will be the focus of this chapter where it will follow as a result that SD is also sound.

As in the soundness proof for the tree method, the soundness proof for QD will go by induction on the length of proof.
In particular, we will be measuring proofs by how many lines they include.
If we can show that soundness holds for proofs of length $1$ and, moreover, that soundness holds for proofs of length $n+1$ if soundness holds for proofs of length $n$, we may conclude by induction that soundness holds for proofs of any length.
% This, of course, is no different from the statement of the soundness of QD.




\section{Induction on Length}%
  \label{sec:InductionLength}

Assume $\Gamma \proves_{\textsc{qd}} \metaA$.
It follows that there is some proof in QD of $\metaA$ from the premises $\Gamma$ where we may call this proof $X$. 
Recall that we may evaluate which lines are live at any point in the course of a natural deduction proof in QD.
For ease of exposition, it will help to introduce some notation that we will use throughout.
In particular, we may take $\metaA_i$ to be the sentence in the $i$-th line of the proof $X$ and $\Gamma_i$ to include all and only the undischarged assumptions (including premises) at the $i$-th line in $X$.
We will then aim to show the following:

\begin{enumerate}[leftmargin=1.5in]
  \item[\it Base:] $\Gamma_1 \models \metaA_1$.
  \item[\it Induction:] $\Gamma_{n} \models \metaA_{n}$ if $\Gamma_k \models \metaA_k$ for every $k\leq n$.
\end{enumerate}

Given that we can establish these two claims, it follows by induction that $\Gamma_n \models \metaA_n$ for all $n$.
Since every proof is finite in length, there is some last line $m$ of the proof $X$ where $\Gamma_m=\Gamma$ is the set of premises and $\metaA_m=\metaA$ is the conclusion, and so $\Gamma \models \metaA$. 
Thus we may conclude:
  $$\text{If } \Gamma \proves_{\textsc{qd}} \metaA, \text{ then } \Gamma \models \metaA.$$
This provides the basic outline of the soundness proof.
It remains to establish the \textit{Base} and \textit{Induction} steps above.
In order to do so, we will begin by proving a number of lemmas in the following section before proceeding to show that the proof rules for QD preserve validity.



\section{Semantic Lemmas}%
  \label{sec:CL}

Consider the following claims.

\label{box:Lemma1}
\factoidbox{
  \textbf{Lemma \ref{sec:CL}.1}: If $\Gamma \cup \set{\metaA} \models \metaB$, then $\Gamma \models \metaA \supset \metaB$.
}

\begin{quote} 
  \textit{Proof:} Assume $\Gamma \cup \set{\metaA} \models \metaB$ and let $\M=\tuple{\D,\I}$ be any model that satisfies $\Gamma$.
  Either $\M$ satisfies $\metaA$ or not, and so their are two cases to consider. 

  \textit{Case 1:}
  If $\M$ satisfies $\metaA$, then $\M$ satisfies $\Gamma \cup \set{\metaA}$.
  By assumption $\M$ satisfies $\metaB$, and so $\VV{\I}{\va{a}}(\metaB)=1$ for some variable assignment $\va{a}$ over $\D$.
  Thus $\VV{\I}{\va{a}}(\metaA \eif \metaB)=1$ by the semantics for the conditional, and so $\M$ satisfies $\metaA \supset \metaB$.

  \textit{Case 2:}
  If $\M$ does not satisfy $\metaA$, then $\VV{\I}{\va{a}}(\metaB)\neq 1$ for every variable assignment $\va{a}$ over $\D$. 
  By the semantics for the conditional, $\VV{\I}{\va{a}}(\metaA \eif \metaB)=1$ for all $\va{a}$, and so for some $\va{a}$ in particular.
  Thus $\M$ satisfies $\metaA \supset \metaB$.

  Since $\M$ satisfies $\metaA \supset \metaB$ in both cases, we may conclude that $\Gamma \models \metaA \supset \metaB$. 
\end{quote}




\label{box:Lemma2}
\factoidbox{
  \textbf{Lemma \ref{sec:CL}.2}: If $\Gamma \models \metaA$ and $\Gamma \models \enot\metaA$, then $\Gamma$ is unsatisfiable.
}

\begin{quote} 
  \textit{Proof:} Assume $\Gamma \models \metaA$ and $\Gamma \models \enot\metaA$.
  Assume for contradiction that $\Gamma$ is satisfiable, and so there is some model $\M$ which satisfies $\Gamma$. 
  By assumption, $\M$ satisfies $\metaA$ and $\enot\metaA$, and so there is some variable assignment $\va{a}$ where $\VV{\I}{\va{a}}(\enot\metaA)=1$.
  By the semantics for negation, $\VV{\I}{\va{a}}(\metaA)\neq 1$ contradicting the above.
  Thus $\Gamma$ is unsatisfiable. 
\end{quote}





\label{box:Lemma3}
\factoidbox{
  \textbf{Lemma \ref{sec:CL}.3}: If $\Gamma \models \metaA$ and $\Gamma \subseteq \Gamma'$, then $\Gamma' \models \metaA$.
}

\begin{quote} 
  \textit{Proof:} Assume $\Gamma \models \metaA$ and $\Gamma \supseteq \Gamma'$.
  Letting $\M=\tuple{\D,\I}$ be any model that satisfies $\Gamma'$, it follows that $\VV{\I}{}(\metaB)=1$ for all $\metaB \in \Gamma'$.
  Since $\Gamma\subseteq \Gamma'$, it follows that $\VV{\I}{}(\metaB)=1$ for all $\metaB \in \Gamma$, and so $\M$ satisfies $\Gamma$. 
  By assumption, $\M$ satisfies $\metaA$, and so $\Gamma \models \metaA$.
\end{quote}





\label{box:Lemma4}
\factoidbox{
  \textbf{Lemma \ref{sec:CL}.4}: 
    If $\Gamma \cup \set{\metaA}$ is unsatisfiable, then $\Gamma \models \enot\metaA$.
    % Similarly, if $\Gamma \cup \set{\enot\metaA}$ is unsatisfiable, then $\Gamma \models \metaA$.
}

\begin{quote} 
  \textit{Proof:} Assume $\Gamma \cup \set{\metaA}$ is unsatisfiable and let $\M$ be any model that satisfies $\Gamma$. 
  If $\M$ satisfied $\metaA$, then $\M$ would satisfy $\Gamma \cup \set{\metaA}$ contrary to assumption, and so $\M$ does not satisfy $\metaA$.
  Thus $\VV{\I}{}(\metaA)\neq 1$, and so $\VV{\I}{\va{a}}(\metaA)\neq 1$ for all variable assignments $\va{a}$ over $\D$.
  It follows that $\VV{\I}{\va{a}}(\enot\metaA)= 1$ for some $\va{a}$ in particular, and so $\VV{\I}{}(\enot\metaA)= 1$.
  We may then conclude that $\M$ satisfies $\enot\metaA$, and so $\Gamma \models \enot\metaA$.
\end{quote}





\label{box:Lemma5} % TODO: move into semantics chapter in later version
\factoidbox{
  \textbf{Lemma \ref{sec:CL}.5}: $\forall\alpha\metaA \models \metaA\unisub{\beta}{\alpha}$ where $\beta$ is a constant and $\alpha$ is a variable. 
}

\begin{quote} 
  \textit{Proof:} Let $\M=\tuple{\D,\I}$ be any model that satisfies $\qt{\forall}{\alpha}\metaA$, and so $\VV{\I}{\va{a}}(\qt{\forall}{\alpha}\metaA)=1$ for some $\va{a}$.
  Thus $\VV{\I}{\va{c}}(\metaA)=1$ for every $\alpha$-variant $\va{c}$ of $\va{a}$ by the semantics for the universal quantifier.
  We will show by induction on complexity that $\VV{\I}{\va{c}}(\metaA\unisub{\beta}{\alpha})=\VV{\I}{\va{c}}(\metaA)$ for every $\alpha$-variant $\va{c}$ of $\va{a}$, and so for some $\alpha$-variant in particular.

  \textit{Base:} Assume $\comp(\metaA)=0$ and let $\va{e}$ be an any $\alpha$-variant of $\va{a}$ where $\va{e}(\alpha)=\I(\beta)$.
  If $\metaA=\F^n\alpha_1,\ldots\alpha_n$ where $\gamma_i=\beta$ if $\alpha_i=\alpha$ and otherwise $\gamma_i=\alpha_i$, the we have:

  \vspace{-.2in}
  \begin{align*}
    \VV{\I}{\va{e}}(\metaA)=1 &\textit{ ~iff~ } \VV{\I}{\va{e}}(\F^n\alpha_1,\ldots,\alpha_n)=1\\
      &\textit{ ~iff~ } \tuple{\VV{\I}{\va{e}}(\alpha_1),\ldots,\VV{\I}{\va{e}}(\alpha_n)}\in\I(\F^n)\\
      (\star) &\textit{ ~iff~ } \tuple{\VV{\I}{\va{e}}(\gamma_1),\ldots,\VV{\I}{\va{e}}(\gamma_n)}\in\I(\F^n)\\
      &\textit{ ~iff~ } \VV{\I}{\va{e}}(\F^n\gamma_1,\ldots,\gamma_n)=1\\
      &\textit{ ~iff~ } \VV{\I}{\va{e}}(\metaA\unisub{\beta}{\alpha})=1.
  \end{align*}

  % then $\VV{\I}{\va{e}}(\metaA)=1$ just in case $\tuple{\VV{\I}{\va{e}}(\alpha_1),\ldots,\VV{\I}{\va{e}}(\alpha_n)}\in\I(\F^n)$.
  For any $1\leq i\leq n$ where $\alpha_i=\alpha$, it follows that $\va{e}(\alpha_i)=\I(\beta)$ by the definition of $\va{e}$, and so $\VV{\I}{\va{e}}(\alpha_i)=\VV{\I}{\va{e}}(\beta)=\VV{\I}{\va{e}}(\gamma_i)$. 
  For any $1\leq i\leq n$ where $\alpha_i\neq\alpha$, it follows that $\VV{\I}{\va{e}}(\alpha_i)=\VV{\I}{\va{e}}(\gamma_i)$ by the definition of $\gamma_i$. 
  Thus the biconditional $(\star)$ holds where the other biconditionals follow from the definition of $\metaA$, the semantics for atomic wffs, or the definition of the substitution $\metaA\unisub{\beta}{\alpha}$, and so $\VV{\I}{\va{e}}(\metaA)=\VV{\I}{\va{e}}(\metaA\unisub{\beta}{\alpha})$.
  % As a result, $\tuple{\VV{\I}{\va{e}}(\alpha_1),\ldots,\VV{\I}{\va{e}}(\alpha_n)}\in\I(\F^n)$ just in case $\tuple{\VV{\I}{\va{e}}(\gamma_1),\ldots,\VV{\I}{\va{e}}(\gamma_n)}\in\I(\F^n)$ 
  % Since $\F^n\gamma_1,\ldots,\gamma_n=\metaA\unisub{\beta}{\alpha}$, it follows that $\VV{\I}{\va{e}}(\metaA)=1$ just in case $\VV{\I}{\va{e}}(\metaA\unisub{\beta}{\alpha})=1$, and so $\VV{\I}{\va{e}}(\metaA)=\VV{\I}{\va{e}}(\metaA\unisub{\beta}{\alpha})$.
  
  If instead $\metaA$ is $\alpha_1=\alpha_2$, then assuming as before that $\gamma_i=\beta$ if $\alpha_i=\alpha$ and otherwise $\gamma_i=\alpha_i$, we have the following biconditionals:

  \vspace{-.2in}
  \begin{align*}
    \VV{\I}{\va{e}}(\metaA)=1 &\textit{ ~iff~ } \VV{\I}{\va{e}}(\alpha_1=\alpha_2)=1\\
      &\textit{ ~iff~ } \VV{\I}{\va{e}}(\alpha_1)=\VV{\I}{\va{e}}(\alpha_n)\\
      (\ast) &\textit{ ~iff~ } \VV{\I}{\va{e}}(\gamma_1)=\VV{\I}{\va{e}}(\gamma_2)\\
      &\textit{ ~iff~ } \VV{\I}{\va{e}}(\gamma_1=\gamma_2)=1\\
      &\textit{ ~iff~ } \VV{\I}{\va{e}}(\metaA\unisub{\beta}{\alpha})=1.
  \end{align*}

  We may justify $(\ast)$ in the same way as above, where the justifications of the other biconditionals are also analogous. 
  % since $\VV{\I}{\va{e}}(\metaA)=1$, we know that $\VV{\I}{\va{e}}(\alpha_1)=\VV{\I}{\va{e}}(\alpha_2)$.
  % By similar reasoning to that given above, $\VV{\I}{\va{e}}(\gamma_1)=\VV{\I}{\va{e}}(\gamma_2)$ where $\gamma_i=\beta$ if $\alpha_i=\alpha$ and $\gamma_i=\alpha_i$ otherwise.
  Thus $\VV{\I}{\va{e}}(\metaA)=\VV{\I}{\va{e}}(\metaA\unisub{\beta}{\alpha})$ in both of the atomic cases considered above. 
  Since $\va{e}$ was an arbitrary $\alpha$-variant of $\va{a}$, it follows that $\VV{\I}{\va{c}}(\metaA\unisub{\beta}{\alpha})=\VV{\I}{\va{c}}(\metaA)$ for every $\alpha$-variant $\va{c}$ of $\va{a}$.

  \textit{Induction:} Assume for induction that $\VV{\I}{\va{c}}(\metaA)=\VV{\I}{\va{c}}(\metaA\unisub{\beta}{\alpha})$ for all $\alpha$-variants $\va{c}$ of $\va{a}$ if $\comp(\metaA)\leq n$. 
  Letting $\comp(\metaA)=n+1$, there are seven cases to consider corresponding to the operators $\enot,\eand,\eor,\eif,\eiff,\qt{\forall}{\gamma},$ and $\qt{\exists}{\gamma}$.
  Consider the following.

  \textit{Case 1:} Assume $\metaA=\enot\metaB$.
  Since $\comp(\metaA)=n+1=\comp(\enot\metaB)=\comp(\metaB)+1$, it follows that $\comp(\metaB)\leq n$, and so $\VV{\I}{\va{c}}(\metaB)=\VV{\I}{\va{c}}(\metaB\unisub{\beta}{\alpha})$ for any $\alpha$-variant $\va{c}$ of $\va{a}$ by hypothesis.
  Thus $\VV{\I}{\va{c}}(\enot\metaB)=\VV{\I}{\va{c}}(\enot\metaB\unisub{\beta}{\alpha})$ for any $\alpha$-variant $\va{c}$ of $\va{a}$ by the semantics for negation.
  The cases for $\eand,\eor,\eif,$ and $\eiff$ are similar.

  \textit{Case 6:} Assume $\metaA=\qt{\forall}{\gamma}\metaB$.
  Since $\comp(\metaA)=n+1=\comp(\qt{\forall}{\gamma}\metaB)=\comp(\metaB)+1$, it follows that $\comp(\metaB)\leq n$, and so $\VV{\I}{\va{c}}(\metaB)=\VV{\I}{\va{c}}(\metaB\unisub{\beta}{\alpha})$ for any $\alpha$-variant $\va{c}$ of $\va{a}$ by hypothesis.
  % If $\gamma=\alpha$, then $\VV{\I}{\va{e}}(\metaB)=\VV{\I}{\va{c}}(\metaB\unisub{\beta}{\alpha})$ for any $\gamma$-variant $\va{e}$ of $\va{a}$ by hypothesis.
  Consider the following biconditionals:

  \vspace{-.2in}
  \begin{align*}
    \VV{\I}{\va{c}}(\metaA)=1 &\textit{ ~iff~ } \VV{\I}{\va{c}}(\qt{\forall}{\gamma}\metaB)=1\\
      &\textit{ ~iff~ } \VV{\I}{\va{e}}(\metaB)=1 \text{ for all } \gamma\text{-variants } \va{e} \text{ of } \va{c}\\ 
      (\dagger) &\textit{ ~iff~ } \VV{\I}{\va{e}}(\metaB\unisub{\beta}{\alpha})=1 \text{ for all } \gamma\text{-variants } \va{e} \text{ of } \va{c}\\  
      &\textit{ ~iff~ } \VV{\I}{\va{c}}(\qt{\forall}{\gamma}\metaB\unisub{\beta}{\alpha})=1\\ 
      &\textit{ ~iff~ } \VV{\I}{\va{e}}(\metaA\unisub{\beta}{\alpha})=1.
  \end{align*}

  If $\gamma=\alpha$, then every $\gamma$-variant of $\va{c}$ is an $\alpha$-variant of $\va{a}$, and so $(\dagger)$ follows by hypothesis. 
  If $\gamma\neq\alpha$, then $\va{e}(\alpha)=\va{c}(\alpha)$ for every $\gamma$-variant $\va{e}$ of $\va{c}$, and so every $\gamma$-variant $\va{e}$ of $\va{a}$ is an $\alpha$

  does not occur in either $\metaB$ or $\metaB\unisub{\beta}{\alpha}$

  Thus $\VV{\I}{\va{e}}(\qt{\forall}{\gamma}\metaB)=\VV{\I}{\va{e}}(\qt{\forall}{\gamma}\metaB\unisub{\beta}{\alpha})$ follows by the semantics for negation. 

\end{quote}


\label{box:Lemma6} % TODO: move into semantics chapter in later version
\factoidbox{
  \textbf{Lemma \ref{sec:CL}.6}: $\forall\alpha\metaA \models \metaA\unisub{\beta}{\alpha}$ where $\beta$ is a constant and $\alpha$ is a variable. 
}

\label{box:Lemma7} % TODO: move into semantics chapter in later version
\factoidbox{
  \textbf{Lemma \ref{sec:CL}.7}: $\metaA\unisub{\beta}{\alpha} \models \exists\alpha\metaA$ where $\beta$ is a constant and $\alpha$ is a variable. 
}


\label{box:Lemma8}
\factoidbox{
  \textbf{Lemma \ref{sec:CL}.8}: For any constant $\beta$ that does not occur in $\forall\alpha\metaA$ or in any sentence $\metaC\in\Gamma$, if $\Gamma \models \metaA\unisub{\beta}{\alpha}$, then $\Gamma \models \forall\alpha\metaA$. 
}


\label{box:Lemma9}
\factoidbox{
  \textbf{Lemma \ref{sec:CL}.9}: For any constant $\beta$ that does not occur in $\exists\alpha\metaA$, $\metaB$, or in any sentence $\metaC\in\Gamma$, if $\Gamma \models \exists\alpha\metaA$ and $\Gamma \cup \set{\metaA\unisub{\beta}{\alpha}} \models \metaB$, then $\Gamma \models \metaB$.
  }


\label{box:Lemma10}
\factoidbox{
  \textbf{Lemma \ref{sec:CL}.10}: If $\alpha$ and $\beta$ are constants, then both $\alpha = \beta,\ \metaA\unisub{\alpha}{\gamma} \models \metaA\unisub{\beta}{\gamma}$ and $\alpha = \beta,\ \metaA\unisub{\beta}{\gamma} \models \metaA\unisub{\alpha}{\gamma}$.
  }


\section{Proof Rules}%
  \label{sec:ProofRules}





\subsection{Reiteration Rule}%
  \label{sub:ReiterationRule}
  




\subsection{Negation Rules}%
  \label{sub:NegationRules}
  




\subsection{Conjunction Rules}%
  \label{sub:ConjunctionRules}
  



\subsection{Disjunction Rules}%
  \label{sub:DisjunctionRules}
 



\subsection{Conditional Rules}%
  \label{sub:ConditionalRules}
  




\subsection{Biconditional Rules}%
  \label{sub:BiconditionalRules}
  




\subsection{Universal Quantifier Rules}%
  \label{sub:UniversalRules}
  




\subsection{Existential Quantifier Rules}%
  \label{sub:ExistentialRules}
 




\subsection{Identity Rules}%
  \label{sub:IdentityRules}
  




\section{Soundness Theorem}%
  \label{sec:SoundnessTheorem}
  



\section{Conclusion}%
  \label{sec:Conclusion}
  
% TODO: thus we can trust our natural deduction system

% TODO: entails that SD is also sound


\iffalse

\practiceproblems

\solutions
\problempart
\label{pr.QL.trees.tautology}
Use a tree to test whether the following sentences are tautologies. If they are not tautologies, describe a model on which they are false.
\begin{earg}
\item $\qt{\forall}{x} \qt{\forall}{y} (Gxy \eif \qt{\exists}{z} Gxz)$
\item $\qt{\forall}{x} Fx \eor \qt{\forall}{x} (Fx \eif Gx)$
\item $\qt{\forall}{x} (Fx \eif (\enot Fx \eif \qt{\forall}{y} Gy))$
\item $\qt{\exists}{x} (Fx \eor \enot Fx)$
\item $\qt{\exists}{x} Jx \eiff \enot \qt{\forall}{x} \enot Jx$
\item $\qt{\forall}{x} (Fx \eor Gx) \eif (\qt{\forall}{y} Fy \eor \qt{\exists}{x} Gx)$
\end{earg}

\solutions
\problempart
\label{pr.QL.trees.validity}
Use a tree to test whether the following argument forms are valid. If they are not, give a model as a counterexample.
\begin{earg}
\item $Fa$, $Ga$, \therefore\ $\qt{\forall}{x} (Fx \eif Gx)$
\item $Fa$, $Ga$, \therefore\ $\qt{\exists}{x} (Fx \eand Gx)$
\item $\qt{\forall}{x} \qt{\exists}{y} Lxy$, \therefore\ $\qt{\exists}{x} \qt{\forall}{y} Lxy$
\item $\qt{\exists}{x} (Fx \eand Gx)$, $Fb \eiff Fa$, $Fc \eif Fa$, \therefore\ $Fa$
\item $\qt{\forall}{x} \qt{\exists}{y} Gyx$, \therefore\ $\qt{\forall}{x} \qt{\exists}{y} (Gxy \eor Gyx)$
\end{earg}

\problempart
\label{pr.QL.trees.translation.and.validity}
Translate each argument into QL, specifying a UD, then use a tree to evaluate the resulting form for validity. If it is invalid, give a model as a counterexample.
\begin{earg}
\item Every logic student is studying. Deborah is not studying. Therefore, Deborah is not a logic student.
\item Kirk is a white male Captain. Therefore, some Captains are white.
\item The Red Sox are going to win the game. Every team who wins the game will be celebrated. Therefore, the Red Sox will be celebrated.
\item The Red Sox are going to win the game. Therefore, the Yankees are not going to win the game.
\item All cats make Frank sneeze, unless they are hairless. Some hairless cats are cuddly. Therefore, some cuddly things make Frank sneeze.
\end{earg}

\fi
