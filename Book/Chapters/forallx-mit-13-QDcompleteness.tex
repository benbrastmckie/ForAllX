%!TEX root = ../forallx-mit.tex
\chapter{The Completeness of QD}
\label{ch.QDcomplete}

\section{Introduction}
  \label{sec:Introduction}

% TODO: introduction
  % equivalence with: consistent sets are satisfiable
  % basic outline of the proof
    % expand language to include infinitely more constants
    % saturate \Gamma by adding witness wffs 
    % expand \Sigma to the maximal \Delta
    % show that \Delta is consistent and deductively closed
    % build Henkin model out of \Delta
    % restrict to original language
    % show that the Henkin model satisfies \Delta






\section{Extensions}%
  \label{sub:Extensions}

Assume $\Gamma$ is a consistent set of QL$^=$ sentences where we will maintain this assumption throughout the proof in order to show that $\Gamma$ is satisfiable.
It follows that there is no QL$^=$ sentence $\metaA$ where $\Gamma\proves \metaA\wedge\neg \metaA$.
This section will construct a superset $\Delta\supseteq\Gamma$ which may be shown to be saturated, maximal, and consistent.
  % TODO: change definition of consistency
We will begin by adding a range of new constants to $QL^=$ which will serve as witnesses for the quantified sentences in $\Gamma$. 





\subsection{Witnesses}%
  \label{sub:Witnesses}
  

Let QL$^=_{\N}$ be a language like QL$^=$ except for including the natural numbers $\N$ as an additional set of constants.
Even though $\Gamma$ is consistent in QL$^=$, it does not immediately follow that $\Gamma$ is consistent in QL$^=_{\N}$.
In general, adding expressive resources to a language can provide the grounds for new derivations and so we need to check that a contradiction cannot be derived in QL$^=_{\N}$.
In order to rule this possibility out, we will begin with the following lemma:

\begin{Lthm} \label{lemma:gencon}
  If $\Gamma\proves\metaA$ and $\beta$ is a constant that does not occur in any $\metaB\in\Gamma$, then there is a variable $\alpha$ which does not occur in $\metaA$ such that $\Gamma\proves \qt{\forall}{\alpha}\metaC$ where $\metaC=\metaA\unisub{\alpha}{\beta}$ and the derivation of $\qt{\forall}{\alpha}\metaC$ from $\Gamma$ does not include $\beta$.
\end{Lthm}

\begin{quote} 
  \textit{Proof:} Assume $\Gamma\proves\metaA$ where $\beta$ is a constant that does not occur in any $\metaB\in\Gamma$.
\end{quote}





\begin{Lthm} \label{lemma:const}
  If $\Gamma$ is a consistent in QL$^=$, then $\Gamma$ is also a consistent in QL$^=_{\N}$.
\end{Lthm}

\begin{quote} 
  \textit{Proof:} Assume $\Gamma$ is a consistent set of QL$^=$ sentences.
\end{quote}

Although the proof follows easily given \textbf{\ref{lemma:gencon}}, the proof is not immediate.
Nevertheless, it is hardly surprising that merely adding new constants would enable the derivation of a contradiction from $\Gamma$ when no contradiction is derivable from $\Gamma$ without those additional constants.
Given $\Gamma$ is consistent in QL$^=$, we may conclude that $\Gamma$ is consistent in QL$^=_{\N}$.





\subsection{Saturation}%
  \label{sub:Saturation}

We will now move to extend $\Gamma$ to a saturated set of sentences, where a set of sentences $\Sigma$ is \define{saturated} just in case $\metaA\unisub{\beta}{\alpha}\in\Sigma$ for some constant $\beta$ whenever $\qt{\exists}{\alpha}\metaA\in\Sigma$.
Consider the fixed enumeration of existentially qu

Letting $\set{\metaA_i:\qt{\exists}{\alpha}\metaA_i}$
\begin{align*}
  \label{name}
  \Sigma_0     &= \Gamma\\
  \Sigma_{n+1} &= \Sigma_n\cup\set{\metaA_{n+1}\unisub{n+1}{\alpha}}
\end{align*}



  
\section{Henkin Model}%
  \label{sub:HenkinModel}





\section{Satisfiability}%
  \label{sub:Satisfiability}
 




\section{Conclusion}%
  \label{sub:Conclusion}
  



\iffalse

\practiceproblems

\solutions
\problempart
\label{pr.QLalttrees-sound}
Following are possible modifications to our QL tree system. For each, imagine a system that is like the system laid out in this chapter, except for the indicated change. Would the modified tree system be sound? If so, explain how the proof given in this chapter would extend to a system with this rule; if not, give a tree that is a counterexample to the soundness of the modified system.
\begin{earg}
\item Change the rule for existentials to this rule:
	\factoidbox{
	\begin{center}
	\begin{prooftree}
	{not line numbering}
	[\qt{\exists}{\script{x}}\metaA{}, checked={\script{a}}
		[\metaA{}\substitute{\script{x}}{\script{a}}, just=for \emph{any} \script{a}
		]
	]
	\end{prooftree}
	\end{center}
	}
	
\item Change the rule for existentials to this rule:
	\factoidbox{
	\begin{center}
	\begin{prooftree}
	{not line numbering}
	[\qt{\exists}{\script{x}}\metaA{}, checked=d
		[\metaA{}\substitute{\script{x}}{d}, just=(whether or not $d$ is new)
		]
	]
	\end{prooftree}
	\end{center}
	}

\item Change the rule for existentials to this rule:
	\factoidbox{
	\begin{center}
	\begin{prooftree}
	{not line numbering}
	[\qt{\exists}{\script{x}}\metaA{}, checked
		[\metaA{}\substitute{\script{x}}{\script{a}}, just={for 3 different names, old or new}
		[ , grouped
		[\metaA{}\substitute{\script{x}}{\script{b}}, grouped
		[ , grouped
		[\metaA{}\substitute{\script{x}}{\script{c}}, grouped
		]
		]
		]
		]
		]
	]
	\end{prooftree}
	\end{center}
	}

\item Change the rule for universals to this rule:
	\factoidbox{
	\begin{center}
	\begin{prooftree}
	{not line numbering}
	[\qt{\forall}{\script{x}}\metaA{}, checked
		[\metaA{}\substitute{\script{x}}{\script{a}}, just={for 3 different names, old or new}
		[ , grouped
		[\metaA{}\substitute{\script{x}}{\script{b}}, grouped
		[ , grouped
		[\metaA{}\substitute{\script{x}}{\script{c}}, grouped
		]
		]
		]
		]
		]
	]
	\end{prooftree}
	\end{center}
	}

\item Change the rule for existentials to this rule:
	\factoidbox{
	\begin{center}
	\begin{prooftree}
	{not line numbering}
	[\qt{\exists}{\script{x}}\metaA{}, checked
		[\metaA{}\substitute{\script{x}}{\script{a}}, just={for 3 new names}
		[ , grouped
		[\metaA{}\substitute{\script{x}}{\script{b}}, grouped
		[ , grouped
		[\metaA{}\substitute{\script{x}}{\script{c}}, grouped
		]
		]
		]
		]
		]
	]
	\end{prooftree}
	\end{center}
	}

\item Change the rule for universals to this rule:
	\factoidbox{
            	\begin{center}
            \begin{prooftree}
            {not line numbering}
            [\qt{\forall}{\script{x}}\metaA{}, checked={\script{a}}
            	[\metaA{}\substitute{\script{x}}{\script{a}}, just=where \script{a} is \emph{new}
            	]
            ]
            \end{prooftree}
            \end{center}
	}

\item Change the rule for conjunction to this rule:
	\factoidbox{
            	\begin{center}
            \begin{prooftree}
            {not line numbering}
            	[\metaA{} \eand \metaB{}, checked
            		[\qt{\exists}{\script{x}} \metaA{}, just=where \script{x} does not occur in \metaA{}
			[\metaB{}, grouped
            		]
            		]
		]
            \end{prooftree}
            \end{center}
	}


\item Change this requirement (given on page \pageref{branchcompletion.defined})...
	\factoidbox{A branch is \define{complete} if and only if either (i) it is closed, or (ii) every resolvable sentence in every branch has been resolved, and for every general sentence and every name \script{a} in the branch, the \script{a} instance of that general sentence has been taken.}
	...to this one:
	\factoidbox{A branch is \define{complete} if and only if either (i) it is closed, or (ii) every resolvable sentence in every branch has been resolved, and for every general sentence, \emph{at least one instance of} that general sentence has been taken.}

\item Change the branch completion requirement to:
	\factoidbox{\ldots and for every general sentence and every name \script{a} \emph{that is above that general sentence in the branch}, the \script{a} instance of that general sentence has been taken.}

\item Change the branch completion requirement to:
	\factoidbox{\ldots and for every general sentence and every name \script{a} in the branch, the \script{a} instance of that general sentence has been taken, \emph{and at least one additional new instance of that general sentence has also been taken}.}
	
	\end{earg}
	
	
	
	
\solutions
\problempart
\label{pr.QLalttrees-complete}
For each of the rule modifications given in Part \ref{pr.QLalttrees-sound}, would the modified tree system be complete? If so, explain how the proof given in this chapter would extend to a system with this rule; if not, give a tree that is a counterexample to the completeness of the modified system.

\fi

