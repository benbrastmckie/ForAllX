%!TEX root = forallx-ubc.tex
\chapter{Identity}
  \label{ch.identity}

The last two chapters introduced the syntax and semantics for QL.
In this chapter, we will extend both the syntax and semantics of QL to accommodate the \define{identity} predicate `$=$', referring to this extended language as QL$^=$.

It is important to emphasises that to say that $\v{x}$ and $\v{y}$ are identical is different from saying that $\v{x}$ and $\v{y}$ are \textit{duplicates} though this is a common way of using the word `identity' in English.
For instance, consider the use of the word `identical' in the following case:

  \begin{quote}
    \texttt{Spheres}: Consider a possible world in which there is nothing but two identical spheres, similar in every way to each other, separated by just one meter.
    Although there is no property that they do not share, the two spheres are distinct.
    After all, there are two spheres, not just one.
  \end{quote}

Insofar as the spheres are distinct--- there are two of them, not one--- we will say that they are not \textit{numerically identical}, or just \textit{non-identical} for short.
If `$a$' names one sphere and `$b$' names the other, we may express this with the sentence `$\neg{=}ab$' where `$=$' is a two place predicate, or `$a\neq b$' for the sake of readability and familiarity.
In this sense of identity, it is not true that the two spheres are identical as claimed in \texttt{Spheres}.
Indeed, no two things whatsoever are identical in our sense since if they were, then there would not be two of them but rather just one thing perhaps with different names.

Before showing how to include a designated predicate for identity in the syntax and semantics for QL$^=$, it will help to guide our ambitions by considering some of what motivates this addition.
After all, QL is very powerful language, at least by comparison to SL.
Why should we need to further extend QL?
Can't we get by without including identity in the language? % by instead taking any $2$-place predicate in the language (e.g., $I$) to symbolize identity as we would for any other $2$-place predicate?



\section{Identity and Logic}
  \label{sec:IdentityLogic}

It turns out that there is a lot that cannot be said without an identity predicate.
You might be wondering why we can't just declare that a certain predicate be used to express identity the way that we do in regimenting other predicates in QL.
For instance, suppose we were to regiment `Hesperus is Phosphorus' as `$Ihp$' given the following symbolization key:

\begin{ekey}
% \begin{multicols}{2}
\item[Ixy:] $x$ is $y$
\item[h:] Hesperus
\item[p:] Phosphorus
\item[v:] Venus
% \end{multicols}
\end{ekey}

One might take the regimentation given above to do as good a job as any of our regimentations in QL.
Why does identity deserve special treatment?

Consider the following English argument regimented with the symbolization given above:

\begin{multicols}{2}
  
\begin{earg}
  \item[] Hesperus is Phosphorus.
  \item[] Phosphorus is Venus.
  \item[\therefore] Hesperus is Venus.
\end{earg}

\begin{earg}
  \item[] $Ihp$
  \item[] $Ipv$
  \item[\therefore] $Ihv$
\end{earg}

\end{multicols}

This argument is invalid.
For instance, consider the following countermodel:

\begin{partialmodel}
	$\D$		& $\set{h,p,v}$\\
  $\I(I)$ & $\set{\tuple{h,p},\tuple{p,v}}$\\
	$\I(h)$	& $h$\\
	$\I(p)$	& $p$\\
	$\I(v)$	& $v$
\end{partialmodel}

Since $\tuple{h,p},\tuple{p,v}\in\I(I)$ but $\tuple{h,v}\notin\I(I)$, it follows that $\tuple{\I(h),\I(p)},\tuple{\I(p),\I(v)}\in\I(I)$ and $\tuple{\I(h),\I(v)}\notin\I(I)$, and so by definition $\tuple{\VV{\I}{\va{a}}(h),\VV{\I}{\va{a}}(p)},\tuple{\VV{\I}{\va{a}}(p),\VV{\I}{\va{a}}(v)}\in\I(I)$ and $\tuple{\VV{\I}{\va{a}}(h),\VV{\I}{\va{a}}(v)}\notin\I(I)$ where $\va{a}$ is any variable assignment that does no substantive work here.
It follows that $\VV{\I}{\va{a}}(Ihp)=\VV{\I}{\va{a}}(Ipv)=1$ and yet $\VV{\I}{\va{a}}(Ihv)\neq 1$.
Since there is such a v.a. as $\hat{a}$ and $Ihp$, $Ipv$, and $Ihv$ are all sentences, we may conclude that $\VV{\I}{}(Ihp)=\VV{\I}{}(Ipv)=1$ and yet $\VV{\I}{}(Ihv)\neq 1$.
Having produced a model which makes the premises true and the conclusion false, it follows that the argument is not valid.

An analogous argument shows that the following argument is also invalid:

\begin{multicols}{2}

\begin{ekey}
  \item[Txy:] $x$ is taller than $y$
  \item[k:] Kate
  \item[s:] Sam
  \item[l:] Lu
\end{ekey}

\begin{earg}
  \item[] Kate is taller than Sam.
  \item[] Sam is taller than Lu.
  \item[\therefore] Kate is taller than Lu.
  \item[] ~
\end{earg}

\end{multicols}

\begin{multicols}{2}

\begin{partialmodel}
	$\D$		& $\set{k,s,l}$\\
  $\I(T)$ & $\set{\tuple{k,s},\tuple{s,l}}$\\
	$\I(k)$	& $k$\\
	$\I(s)$	& $s$\\
	$\I(l)$	& $l$
\end{partialmodel}

\begin{earg}
  \item[] $Tks$
  \item[] $Tsl$
  \item[\therefore] $Tkl$
\end{earg}
\vfill
\strut

\end{multicols}

Replacing `$I$' with `$T$' and similarly replacing `$h$' with `$k$' and so on for the other constants, a very similar semantic proof could be repurposed to show that the premises in the argument above do not entail the conclusion.
Here we may ask if this is right, and if so, why we shouldn't say the very same thing about the identity argument.

Certainly it should be admitted that the taller-than argument is a very strong argument in ordinary contexts.
After all, given the intended interpretation of English, any \textit{possibility} in which Kate is taller than Sam and in which Sam is taller than Lu is also a possibility in which Kate is taller than Lu. 
The reason the argument is invalid is that nothing forces us to interpret the predicate `is taller than' as meaning what it usually means.
Put otherwise, the dyadic predicate `is taller than' is a \define{non-logical} term of our language, and is to be regimented by a non-logical dyadic predicate in QL which we may interpret by any subset of $\D^2$ given any domain $\D$ whatsoever. 
This makes the argument easy to invalidate.

Were to want to make the taller-than argument valid, we would have to add an additional premise as in the follow arguments:

\begin{multicols}{2}
  
\begin{earg}
  \item[] $(Tks \eand Tsl) \eif Tkl$
  \item[] $Tks$
  \item[] $Tsl$
  \item[\therefore] $Tkl$
\end{earg}

\begin{earg}
  \item[] $\forall x\forall y \forall z((Txy \eand Tyz) \eif Txz)$
  \item[] $Tks$
  \item[] $Tsl$
  \item[\therefore] $Tkl$
\end{earg}

\end{multicols}

Both of the arguments above are valid.
Whereas the argument on the right starts off by asserting that the taller-than relation is transitive, the argument on the left appeals to a particular instances of the transitivity of the taller-than relation.

It is natural to assume that on the intended interpretation of `is taller than' in English, we mean to express a transitive relation since this is how heights behave.
In reasoning from $Tks$ and $Tsl$ to $Tkl$, we are implicitly relying on our intuitive grasp of a particular interpretation rather than general logical features of the sentences involved. 
That is, the argument is convincing not because of its logical form, but because of the particular interpretation that we are assuming, i.e., where `is taller than' expresses a transitive relation.
When we add a premises which requires $T$ to be transitive (or else add the relevant instance), we are making our assumptions explicit in a way that avoids reliance on a particular intended interpretation of our language.
Instead, the amended arguments given above are valid since any model which makes these premises true makes their conclusions true. 
Put otherwise, the conclusion follows by virtue of the logical forms of the sentences and not a particular interpretation.

What about the identity argument?
Certainly we could reproduce a similar story, claiming that as it stands, the identity argument we started off with is not valid but could be made valid by adding a premise that requires identity to be transitive.
The question is whether this would be appropriate in the case of an identity predicate.
More specifically, is it permissible to interpret identity as \textit{any} subset of $\D^2$ over any domain $\D$?

The answer is certainly `Yes' since this is exactly how we would interpret the identity argument in QL.
Nevertheless, there is good reason not to go this way, choosing instead to include a designated predicate for identity in QL$^=$.
Recast in QL$^=$, the identity argument becomes:

\begin{multicols}{2}
  
\begin{earg}
  \item[] Hesperus is Phosphorus.
  \item[] Phosphorus is Venus.
  \item[\therefore] Hesperus is Venus.
\end{earg}

\begin{earg}
  \item[] $h=p$
  \item[] $p=v$
  \item[\therefore] $h=v$
\end{earg}

\end{multicols}

Instead of taking this argument to only be convincing when we restrict consideration to an intended interpretation where `$=$' means identity, we are taking identity to be a logical notion akin to negation, conjunction, and the quantifiers.
Rather than relying on the intended interpretation of our language to tell us what identity means, we are going to provide semantic clauses in just the same way that we did for the other logical terms of our language.
% Roughly, `$\alpha=\beta$' is true in a model $\M=\tuple{\D,\I}$ given a variable assignment $\va{a}$ just in case $\alpha$ and $\beta$ refer to the same individual in the model.
% Note that we have used `the same individual' to provide the semantics for identity.
As a result, the argument above will turn out to be valid as it stands.

You might be wondering why we don't do something similar for the `is taller than' predicate, and so on for other notions like `between', or `is older than', etc.
There are two reasons worth considering.
The first is that there is no clear stopping point.
Were we to start expanding the range of logical predicates whose interpretation we hold fixed by providing semantic clauses, we could go and go forever.
This in itself does not require that we do so--- we could just choose to include certain predicates in the logical vocabulary of our language and not others given our purposes.
The second reason is more forceful: in order to provide a semantic clause for the taller-than predicate `$T$', we would have to provide a theory of what it is for one thing to be taller than another.
Without providing such a theory, nothing guarantees that $T$ is transitive, and so the taller-than argument would remain invalid. 

% Suppose one were to attempt to provide a theory of the taller-than predicate by taking $T$ to be transitive and irreflexive, using QL to state these assumptions.
% Here we may object that there are lots of relations that are transitive and irreflexive, and so given these constraints alone, we have no reason to think that $T$ picks out the taller-than relation rather than some other relation.
% For instance `is older than' is also transitive and irreflexive, but means something completely different than `is taller than'.
%
% Although one might attempt to improve on our initial theory of being taller than, doing so reaches beyond the subject-matter of logic.
% Rather, logic is concerned with the so called \define{logical terms} like negation, conjunction, quantification, and identity, where these notions do not concern any particular ways for things to be (e.g., one thing being taller than another), but rather provide very general conceptual resources for articulating theories in the first place.
% For instance, one may point out that we might use a language like QL in order to provide a theory constraining the interpretation of $T$.
% In particular, consider the following constraint:
%   $$(\alpha=\beta) \supset (T\alpha\gamma\equiv T\beta\gamma).$$
% This says that if $\alpha=\beta$, then $\alpha$ is taller than $\gamma$ just in case $\beta$ is taller than $\gamma$. 
% Certainly this seems true, and yet it is unclear how we would state such a principle for our theory of the taller-than relation without recourse to the identity predicate.
% This gives us reason to think that the theory of the taller-than relation should be presented in QL$^=$ instead of QL. 
% By contrast, we do not need to appeal to $T$ in order to provide a theory of identity.


Although one might attempt to provide a theory of the taller-than relation, doing so reaches beyond the subject-matter of logic.
Moreover, it would be natural to use a language like QL$^=$ in order to develop such a theory. 
The same cannot so easily be said for identity.
Instead of falling outside the subject-matter of logic, identity is taken to fit squarely within our present aim to develop the conceptual resources that we need to articulate theories in the first place.
Instead of requiring that we develop an independent theory of identity, the semantics for identity will rely on our understanding of identity in the metalanguage in much the same way that the semantics for negation relied on an understanding of negation in the metalanguage.


Before pressing on, it is worth considering three more cases involving identity.
To begin with, consider the following example originally presented by Gottlob Frege:

\begin{multicols}{2}

\begin{ekey}
  % \begin{multicols}{2}
    \item[Rx:] $x$ is rising.
    \item[h:] Hesperus
    \item[p:] Phosphorus
  % \end{multicols}
\end{ekey}

\begin{earg}
  \item[] Hesperus is rising.
  \item[] Hesperus is Phosphorus.
  \item[\therefore] Phosphorus is rising.
\end{earg}

\end{multicols}

As specified below, identity is a primitive symbol of QL$^=$.
Accordingly, we do not need to include identity in the symbolization key given above to provide the following regimentation.

\begin{earg}
  \item[] $Rh$
  \item[] $h=p$
  \item[\therefore] $Rp$
\end{earg}

This is a valid argument.
Instead of restricting consideration to an intended interpretation, or else adding some further assumptions, the conclusion is entailed by the premises given the semantics that we will provide for QL$^=$.

% However, insofar as different models may have different domains, the interpretation of the identity predicate will have to change accordingly.
% Instead of assigning the identity predicate to some subset of $\D^2$ for some particular domain $\D$ once and for all, we will provide a semantic clause for identity in much the same way that we did for the other logical terms included in QL. 
% We will provide these details in due course, but for now we will focus on the motivations for doing so.

Next consider the regimentation of the following argument:

\begin{multicols}{2}

\begin{ekey}
  \item[Lxy:] $x$ loves $y$
  \item[Dx:] $x$ is a DJ
  \item[c:] Cara
  \item[p:] Pedro
  \item[d:] DJ Faro
\end{ekey}

\vfill
\strut
\columnbreak

\begin{earg}
  \item[] Only Cara loves Pedro.
  \item[] DJ Faro loves Pedro.
  \item[\therefore] Cara is DJ Faro.
\end{earg}

\begin{earg}
  \item[] $\forall x(Lxp \eiff x=c)$
  \item[] $Ldp$
  \item[\therefore] $c=d$
\end{earg}

\end{multicols}

This is a valid argument.
Although we could say that Cara loves Pedro in QL, we could not say that \textit{only} Cara loves Pedro in QL since to do so we would need to say that anything that loves Pedro is identical to Cara in addition to saying that Cara loves Pedro.
Here we may accomplish both claims at once by saying that for anything, it loves Pedro just in case it is identical to Cara.
Since Cara is identical to herself, she must love Pedro, and moreover, for anything that loves Pedro, it must be identical to Cara.
Since DJ Faro loves Pedro, we may conclude that DJ Faro must be identical to Cara.
Reasoning in this way requires that we extend the expressive power of QL by including identity in the language.

Here is a third example. 
Consider the following regimentations of sentence $\ref{M1}$:

\begin{earg}
  \item[\ex{M1}] Mozart composed at least two things.
\end{earg}

\begin{multicols}{2}

\begin{ekey}
  \item[Cxy:] $x$ composed $y$
  % \item[Sx:] $x$ is a thing. 
  \item[m:] Mozart
  \item[] ~
\end{ekey}

\begin{earg}
  \item[\ex{M2}] $\qt{\exists}{x} Cmx \eand \qt{\exists}{y} Cmy$.
  \item[\ex{M3}] $(\qt{\exists}{x} Cmx \eand \qt{\exists}{y} Cmy) \eand x\neq y$.
  \item[\ex{M4}] $\qt{\exists}{x}\qt{\exists}{y}((Cmx \eand Cmy) \eand x\neq y)$.
\end{earg}

\end{multicols}

Although sentence $\ref{M2}$ can be stated in QL, this regimentation does not require that there are at least two things that Mozart composed.
This is because both conjuncts could be satisfied by the same thing, and so the sentence would be true if there was just one thing that Mozart composed.
Sentence $\ref{M3}$ is worse since this is not even a sentence. 
Rather, it includes free variables which fall outside of the scope of both quantifiers.
By contrast, sentence $\ref{M4}$ provides an adequate regimentation, though does so by making the quantifiers have scope over all instances of $x$ and $y$. 
The success of this regimentation has profound consequences for it means that we can regiment `at least two' in QL$^=$.
As we will soon see, we may also regiment `at most two', where `exactly two' will be regimented by their conjunction.

What all of these cases show us is that if we want to reflect the logical relationships having to do with identity, we need special logical vocabulary to do so.
Just as we introduced the `$\forall$' and `$\exists$' to regiment quantified claims, we also need a special symbol `$=$' for identity.




\section{The Syntax for QL$^=$}
  \label{sec:SyntaxQL=}

The primitive symbols included in QL$^=$ are exactly the same as those included in QL with the single addition of the identity predicate `$=$'.
Thus we have the following:

\vspace{.2in}
\begin{center}
  \begin{tabular}{|c|c|}
    \hline
      $n$-place predicates for $n\geq 0$ & $A^n,B^n,C^n,\ldots,Z^n$\\
      with subscripts, as needed & $A_1^n, B_1^n, Z_1^n, A_2^n, A_{25}^n, J_{375}^n,\ldots$\\
    \hline
      constants & $a,b,c,\ldots,v$\\
      with subscripts, as needed & $a_1, w_4, h_7, m_{32},\ldots$\\
    \hline
      variables & $w, x,y,z$\\
      with subscripts, as needed & $x_1, y_1, z_1, x_2,\ldots$\\
    \hline
      sentential connectives & \enot, \eand, \eor, \eif, \eiff\\
    \hline
      identity & $=$\\
    \hline
      quantifiers& $\forall, \exists$\\
    \hline
      parentheses&( , )\\
    \hline
  \end{tabular}
\end{center}
\vspace{.2in}

Here we have included `$=$' in our alphabet of primitive symbols.
This may seem like a small change and given the examples above, it may be obvious to you how to build up sentences in QL$^=$.
Nevertheless, we need to define the wffs of QL$^=$ afresh, where something similar will be repeated for our other recursive definitions that we provided before. 
Although much will be as it was before, it is important to attend to the differences that occur throughout the definitions given in the following two sections. %, and it never hurts to review these essential definitions which give shape to our language.


Whereas in QL there was just one way to form atomic wffs, now we will have two separate ways to form wffs.
Consider the following recursive definition of the wffs in QL$^=$:

\begin{enumerate}
  \item $\F^n\alpha_1,\ldots,\alpha_n$ is a wff if $\F^n$ is an $n$-place predicate and $\alpha_1,\ldots,\alpha_n$ are singular terms.
  \item $\alpha=\beta$ is a wff if $\alpha$ and $\beta$ are singular terms.
\item If $\metaA$ and $\metaB$ are wffs and $\alpha$ is a variable, then:
	\begin{enumerate}
    % \begin{multicols}{2}
      \item $\qt{\exists}{\alpha}\metaA$ is a wff;
      \item $\qt{\forall}{\alpha}\metaA$ is a wff;
      \item $\enot\metaA$ is a wff;
      % \item[] ~
      \item $(\metaA\eand\metaB)$ is a wff;
      \item $(\metaA\eor\metaB)$ is a wff;
      \item $(\metaA\eif\metaB)$ is a wff; and
      \item $(\metaA\eiff\metaB)$ is a wff.
    % \end{multicols}
	\end{enumerate}
\item Nothing else is a wff.
\end{enumerate}

Officially, the clauses above are non-sense, and can only be made sense of by adding corner quotes in appropriate places.
Having explained how to do this above, we will rely on the reader to know where these corner quotes are implicitly intended.

We may either form atomic wffs as we did in QL, or we may form wffs with the identity predicate together with two singular terms.
Nevertheless, nothing requires identity wffs to be sentences since they may include free variables in just the same way that $2$-place predicates may combine with free variables.
This means that there is a new way for free variables to occur in wffs and so we will have to extend our definition of free variables accordingly:

\begin{enumerate}
  \item $\alpha$ is free in $\F^n\alpha_1,\ldots,\alpha_n$ if $\alpha=\alpha_i$ for some $1\leq i\leq n$ where $\alpha$ is a variable, $\F^n$ is an $n$-place predicate, and $\alpha_1,\ldots,\alpha_n$ are singular terms.
  \item $\alpha$ is free in $\beta=\gamma$ if $\alpha=\beta$ or $\alpha=\gamma$ where $\alpha$ is a variable.
  \item If $\metaA$ and $\metaB$ are wffs and $\alpha$ and $\beta$ are variables, then:
    \begin{enumerate}
        \item $\alpha$ is free in $\qt{\exists}{\beta}\metaA$ if $\alpha$ is free in $\metaA$ and $\alpha\neq\beta$;
        \item $\alpha$ is free in $\qt{\forall}{\beta}\metaA$ if $\alpha$ is free in $\metaA$ and $\alpha\neq\beta$;
        \item $\alpha$ is free in $\enot\metaA$ if $\alpha$ is free in $\metaA$;
        \item $\alpha$ is free in $(\metaA\eand\metaB)$ if $\alpha$ is free in $\metaA$ or $\alpha$ is free in $\metaB$;
        \item $\alpha$ is free in $(\metaA\eor\metaB)$ if $\alpha$ is free in $\metaA$ or $\alpha$ is free in $\metaB$;
        \item $\alpha$ is free in $(\metaA\eif\metaB)$ if $\alpha$ is free in $\metaA$ or $\alpha$ is free in $\metaB$;
        \item $\alpha$ is free in $(\metaA\eiff\metaB)$ if $\alpha$ is free in $\metaA$ or $\alpha$ is free in $\metaB$;
    \end{enumerate}
  \item Nothing else is a free variable. 
\end{enumerate}

Given the definition of free variables in QL$^=$, we may define an \define{open sentence} of QL$^=$ to be any wff of QL$^=$ which includes free variables.
In just the same way as before, a \define{sentence} of QL$^=$ is any wffs of QL$^=$ which does not include any free variables. 
Given these definitions, the syntax for QL$^=$ is complete.
We may now turn to interpret the sentences of QL$^=$. 

% TODO: add regimentation examples



\section{The Semantics for QL$^=$}%
  \label{sec:SemanticsQL=}
  
In $\S\ref{ch9.ModelsQL}$, we defined a \define{model} of QL to be any ordered pair $\M=\tuple{\D,\I}$ where the \define{domain} $\D$ is any nonempty set and the \define{interpretation} $\I$ over $\D$ satisfies the following conditions: 

\begin{enumerate}[leftmargin=1.5in]
  \item[\sc Constants:] $\I(\alpha)\in\D$ for every constant $\alpha$ of QL. 
  \item[\sc Predicates:] $\I(\F^n)\subseteq\D^n$ for every $n$-place predicate $\F^n$ of QL where $n\geq 0$.
\end{enumerate}

Since adding identity to the list or primitive symbols does not effect the manner in which the constants or predicates are interpreted, no change is required to the definition of a model for QL$^=$.
Accordingly, QL$^=$ and QL have precisely the same models. 

Recall the manner in which \define{variable assignments} were defined over a domain $\D$ to be any function $\va{a}$ from the variables in QL to elements of $\D$.
Again, no change is required since neither the variables nor the domains that we might consider are effected by the addition of the identity predicate to the language.
For a similar reason, we may also preserve the definition of a \define{$\alpha$-variant} of $\va{a}$ as any variable assignment $\va{c}$ where $\va{c}(\beta)=\va{a}(\beta)$ for all $\beta\neq\alpha$.
Lastly, we include for the sake of completeness the aggregation function from before:

\vspace{-.2in}
\begin{align*}
  \VV{\I}{\va{a}}{(\alpha)}=
    \begin{cases}
      \I(\alpha) & \text{if } \alpha \text{ is a constant} \\
      \va{a}(\alpha) & \text{if } \alpha \text{ is a variable.}
    \end{cases}
\end{align*}

So far, all of the semantic definitions have remained the same as they were in QL.
Nevertheless, the semantics for QL$^=$ will differ insofar as it includes an extra clause for identity, mirroring the changes we made to the definition of the wffs of QL$^=$ given above. 

\begin{enumerate}[labelsep=.15in]
  \item[($A$)] $\VV{\I}{\va{a}}(\F^n\alpha_1,\ldots,\alpha_n)=1$ just in case $\tuple{\VV{\I}{\va{a}}{(\alpha_1)},\ldots,\VV{\I}{\va{a}}{(\alpha_n)}}\in\I(\F^n)$.
  \item[($=$)] $\VV{\I}{\va{a}}(\alpha=\beta)=1$ just in case $\VV{\I}{\va{a}}(\alpha)=\VV{\I}{\va{a}}(\beta)$.
  \item[(\hspace{1pt}$\forall$\hspace{1pt})] $\VV{\I}{\va{a}}(\qt{\forall}{\alpha}\metaA)=1$ just in case $\VV{\I}{\va{c}}(\metaA)=1$ for every $\alpha$-variant $\va{c}$ of $\va{a}$.
  \item[(\hspace{1pt}$\exists$\hspace{1pt})] $\VV{\I}{\va{a}}(\qt{\exists}{\alpha}\metaA)=1$ just in case $\VV{\I}{\va{c}}(\metaA)=1$ for some $\alpha$-variant $\va{c}$ of $\va{a}$.
  \item[(\enot)] $\VV{\I}{\va{a}}(\enot\metaA{})=1$ just in case $\VV{\I}{\va{a}}(\metaA{})\neq 1$.
  \item[(\eor)] $\VV{\I}{\va{a}}(\metaA{} \eor \metaB{})=1$ just in case $\VV{\I}{\va{a}}(\metaA{})=1$ or $\VV{\I}{\va{a}}(\metaB{})=1$ (or both).
  \item[(\eand)] $\VV{\I}{\va{a}}(\metaA{} \eand \metaB{})=1$ just in case $\VV{\I}{\va{a}}(\metaA{})=1$ and $\VV{\I}{\va{a}}(\metaB{})=1$.
  \item[(\eif)] $\VV{\I}{\va{a}}(\metaA{} \eif \metaB{})=1$ just in case $\VV{\I}{\va{a}}(\metaA{})\neq 1$ or $\VV{\I}{\va{a}}(\metaB{})=1$ (or both).
  \item[(\eiff)] $\VV{\I}{\va{a}}(\metaA{} \eiff \metaB{})=1$ just in case $\VV{\I}{\va{a}}(\metaA{})=\VV{\I}{\va{a}}(\metaB{})$.
\end{enumerate}

Having added the clause for identity, the other clauses continue to apply.
It is worth comparing the semantic clause for identity to the clause for negation and considering the following worry.

\begin{quote}
  \texttt{Gripe}:
  The semantic clause for identity doesn't tell us anything because we have used identity--- indeed the same symbol--- on both sides of the semantic clause.
  So in order to know about whether an identity sentence such as $\alpha=\beta$ is true in a model on an assignment, we already need to know what is identical to what.
  Thus the semantics does not tell us anything we didn't already know.
\end{quote}

If we were attempting to understand what `$=$' means without drawing on any previous understanding, then certainly we should agree that the semantic clauses given above get us nowhere.
However, the very same thing may be said for at least negation, conjunction, disjunction, and the quantifiers.
In each of these cases, analogues of the terms with which we are concerned appear in the metalanguage and play a critical role in stating the semantic clauses.
Thus we cannot lean on our semantics to learn what these terms mean without any prior understanding of at least their analogues in the metalanguage.

All of this we must learn to accept.
Where the grip goes wrong is in thinking that there is any other way in which semantics might proceed.
Instead of constructing something out nothing, the semantic clauses allow us to use the meanings we already grasp in order to interpret a simplified formal language in a systematic way.
Identity is no exception, though perhaps even more poignant given that we have used the same symbol in the metalanguage for identity.

Having defined truth relative to a model and assignment function, we are now in a position to specify what it is for a sentence of QL$^=$ to be true in a model (independent of assignments):

\begin{enumerate}[labelsep=.15in]
  \item[($\metaA$)] $\VV{\I}{}(\metaA)=1$ just in case $\VV{\I}{\va{a}}(\metaA)=1$ for some $\va{a}$ where $\metaA$ is a sentence of QL$^=$.
\end{enumerate}

This definition is just as it was before save that `QL' has been replaced by `QL$^=$'.
For completeness, we will copy over the definitions of satisfaction and entailment given their importance, where the other semantic notions are the same as they were before.

\begin{enumerate}[leftmargin=1.5in]
  \item[\sc Satisfaction:] $\M=\tuple{\D,\I}$ satisfies $\Gamma$ just in case $\VV{\I}{}(\metaA)=1$ for every $\metaA\in\Gamma$.
  \item[\sc Entailment:] $\Gamma\models\metaA$ just in case every model that satisfies $\Gamma$ also satisfies $\metaA$.
\end{enumerate}

Although there is a lot of redundancy with the syntax and semantics that we provided for QL, hopefully this gives you a good overview of all of the working piece that make up these theories.
In the remaining sections of this chapter, we will put these theories to work in order to evaluate sentences and arguments in QL$^=$ that we could not adequate regiment in QL.
As we will see, this language is very powerful and perhaps for this reason has become the \textit{lingua franca} in which a wide range of theories have been developed.
One prominent example is set theory where the dyadic predicate `$\in$' for set-membership may be axiomatized in QL$^=$.







\section{Uniqueness}

Recall the following argument from before:

\begin{multicols}{2}

\begin{earg}
  \item[] Only Cara loves Pedro.
  \item[] DJ Faro loves Pedro.
  \item[\therefore] DJ Faro is Cara.
\end{earg}

\begin{earg}
  \item[] $\qt{\forall}{x}(Lxp \eiff x=c)$
  \item[] $Ldp$
  \item[\therefore] $d=c$
\end{earg}

\end{multicols}

We are now in a position to show that this argument is valid.

\begin{quote}
\label{unique1}
  \textit{Proof:}
  Let $\M=\tuple{\D,\I}$ be a QL model where $\VV{\I}{}(\forall x(Lxp \eiff x=c))=1$ and $\VV{\I}{}(Ldp)=1$.
  Thus $\VV{\I}{\va{a}}(\forall x(Lxp \eiff x=c))=1$ and $\VV{\I}{\va{c}}(Ldp)=1$ for some $\va{a}$ and $\va{c}$, and so $\tuple{\VV{\I}{\va{c}}(d),\VV{\I}{\va{c}}(p)}\in\I(L)$.
  It follows that $\tuple{\I(d),\I(p)}\in\I(L)$.

  Let $\va{e}$ be an $x$-variant of $\va{a}$ where $\va{e}(x)=\I(d)$.
  Given the universal claim above, $\VV{\I}{\va{e}}(Lxp \eiff x=c)=1$, and so $\VV{\I}{\va{e}}(Lxp)=\VV{\I}{\va{e}}(x=c)$.
  Thus $\tuple{\VV{\I}{\va{e}}(x),\VV{\I}{\va{e}}(p)}\in\I(L)$ just in case $\VV{\I}{\va{e}}(x)=\VV{\I}{\va{e}}(c)$, and so $\tuple{\va{e}(x),\I(p)}\in\I(L)$ just in case $\va{e}(x)=\I(c)$.
  Since $\va{e}(x)=\I(d)$, we know that $\tuple{\I(d),\I(p)}\in\I(L)$ just in case $\I(d)=\I(c)$.

  Given the above, we may conclude that $\I(d)=\I(c)$.
  Thus $\VV{\I}{\va{g}}(d)=\VV{\I}{\va{g}}(c)$ where $\va{g}$ is any v.a., and so $\VV{\I}{\va{g}}(d=c)=1$.
  Since $d=c$ is a sentence, we may conclude that $\VV{\I}{}(d=c)=1$. 
  Hence $\forall x(Lxp \eiff x=c), Ldp \models d=c$ as desired. 
\end{quote}

The proof above only requires one critical choice.
Given that $\VV{\I}{\va{a}}(\forall x(Lxp \eiff x=c))=1$, we know that $\VV{\I}{\va{b}}(Lxp \eiff x=c)=1$ for any $x$-variant $\va{b}$ of $\va{a}$ whatsoever.
We were careful to choose the $x$-variant $\va{e}$ of $\va{a}$ where $\va{e}(x)=\I(d)$.
This is akin to instantiating $x$ by $d$, resulting in the sentence $Ldp \eiff d=c$ which, together with $Ldp$ entails $d=c$.
Instead of replacing `$x$' with `$d$' as we did just now to illustrate, we chose the $x$-variant $\va{e}$ where $\va{e}(x)=\I(d)$.

Given that only Cara loves Pedro, we may think of her as the unique-Pedro-lover.
That is, not only is there something out there that loves Pedro, Cara is \textit{the} Pedro-lover.
Suppose we forget Cara's name, but remember this prominent fact about her.
We might then ask: is the Pedro-lover the same as DJ Faro?
Instead of using her name, we are using this distinguishing feature to refer to her.
This is a common practice since we don't have names for everything in English, and even when we do, we don't always know the name of the relevant individual in question.
Nevertheless, if we know that there is some particular way that only one thing happens to be, then we may use that way of being to pick out that thing.

Whereas Cara's distinguishing feature was loving Pedro, in general we may appeal to any condition however complex.
For instance, perhaps many people love Pedro, but Cara is the only DJ to love Pedro.
We may express this with $\forall x((Dx \eand Lxp) \eiff x=c)$.
By replacing the constant `$c$' with a variable as in $\forall x((Dx \eand Lxp) \eiff x=y)$, we may consider the condition of being the only DJ to love Pedro.
This open sentence is akin to the complex property of \textit{being the only DJ to love Pedro} where we may appeal to this distinguishing feature to refer to an individual, in this case Cara.
This brings us to the topic of definite descriptions.





\section{Definite Descriptions}
  \label{sec.DefiniteDescription}

In 1905, Bertrand Russell famously characterized \emph{definite descriptions} in terms of identity.
% A definite description is a description that implies that only one object satisfies it.
In the paradigm cases, definite descriptions use the definite article `the'.
Suppose one were to hear a crying in the next room, saying `the baby is hungry'.
This is to claim that the \emph{one and only baby} (in the vicinity) is hungry.
Russell was motivated in part by the apparent fact that one can use this sort of language in a meaningful way even if one is wrong about whether there's any baby around.
If there is no baby--- the crying is a recording --- the statement is false, but it's still meaningful.
For this reason, Russell was reluctant to suppose that we should understand `the baby' as a name.
Remember, in QL, all names have to refer to objects in the domain.
Instead, the sentence can be understood to be an existentially quantified claim about a unique baby.
Saying `the baby is hungry', according to Russell, is to say three things: there is a baby, there's no other baby than that one, and that baby is hungry.
That second element, the uniqueness claim, can be expressed in QL with identity. 

In order to bring out the contrast, consider the following symbolization key:

\begin{ekey}
  \item[Bx:] $x$ is a baby
  \item[Hx:] $x$ is hungry
  \item[j:] Jonathan
\end{ekey}

A sentence like `Jonathan is hungry' is straightforwardly translated as $Hj$.
According to Russell's theory of definite descriptions, `the baby is hungry' has a much more complex logical form which we may regiment in either of the follows ways:

\begin{earg}
  \item[\ex{Def1}] $\qt{\exists}{x} ((Bx \eand \qt{\forall}{y} (By \eif y = x)) \eand Hx)$.
  \item[\ex{Def2}] $\qt{\exists}{x} (\qt{\forall}{y} (By \eiff y = x) \eand Hx)$.
\end{earg}

Whereas the first regimentation says that there is some baby which is the only baby and is hungry, the second regimentation collapses the first two parts of the first regimentation, claiming that the unique baby is hungry.
In order to express the uniqueness of being a baby, we have used the biconditional as in the previous section.
That is, $x$ is the unique baby insofar as for any $y$ whatsoever, $y$ is a baby if and only if $y$ is identical to $x$.

As brought out in the previous section, the distinguishing feature need not be expressed by a single predicate.
For instance, suppose there is a baby who is sleeping right in front of us, but we hear crying from the other room.
One may then be a little more specific by saying `the crying baby is hungry'.
We may expand our symbolization key to regiment this claim.

\begin{multicols}{2}

\begin{ekey}
  \item[Cx:] $x$ is crying
\end{ekey}

\begin{earg}
  \item[\ex{Def3}] $\qt{\exists}{x} (\qt{\forall}{y} ((Cy \eand By) \eiff y = x) \eand Hx)$.
\end{earg}

\end{multicols}

Instead of the single predicate `$B$', we have used `$C$' together with `$B$' in order to form the open sentence `$Cy \eand By$' which describes the individual to which we intend to refer.

In order to speak generally about the means by which we may refer to some unique individual satisfying a certain condition, let $\metaA(\alpha)$ be any wff of QL$^=$ in which the variable $\alpha$ is free. 
If $\alpha$ is the only free variable in $\metaA(\alpha)$, we may take $\metaA(\alpha)$ to be a \define{description}.
Moreover, $\metaA(\alpha)$ provides a \define{definite description} just in case $\metaA(\alpha)$ is a description which just one thing satisfies, i.e., $\qt{\exists}{\beta}\qt{\forall}{\alpha}(\metaA(\alpha)\eiff \alpha = \beta)$ where $\alpha$ and $\beta$ are distinct variables.
Given a definite description $\metaA(\alpha)$, we may make claims about the object satisfying that description by conjoining another description $\metaB(\beta)$ within the scope of the existential quantifier as follows: 
  $$\qt{\exists}{\beta}(\qt{\forall}{\alpha}(\metaA(\alpha)\eiff \alpha = \beta) \eand \metaB(\beta)).$$
This reads: the unique thing for which $\metaA$ is such that $\metaB$.
Sentence $\ref{Def3}$ is an instance of this general recipe, and reads: the unique thing for which it is a crying baby, is hungry. 
Russell's idea is that this is what is going on when we use the definite article `the' since we may say the same thing much more naturally with: \textit{the} crying baby is hungry.

One of the interesting features of Russell's theory is that `the baby is not hungry' is not the negation of `the baby is hungry'.
Instead, the negation applies only to the last conjunct:

\begin{earg}
  \item[\ex{Def4}] $\qt{\exists}{x} (\qt{\forall}{y} ((Cy \eand By) \eiff y = x) \eand \enot Hx)$.
\end{earg}

The reason Russell designed his theory this way was that he thought that both of these sentences equally implied that there is a baby.
If there is no baby, then you'd be mistaken in saying either `the baby is hungry' or `the baby is not hungry'.
Consequently, one can't be the negation of the other, but rather requires the analysis given above.

As a treatment of the truth conditions of English sentences, Russell's theory is controversial.
For instance, consider the following case:

\begin{multicols}{2}

\begin{ekey}
  \item[Kxy:] $x$ is king of $y$
  \item[Bx:] $x$ is Bald
  \item[f:] France
\end{ekey}

\begin{earg}
  \item[\ex{Def4}] The king of France is bald.
  \item[\ex{Def5}] $\qt{\exists}{x} (\qt{\forall}{y} (Kyf \eiff y = x) \eand Bx)$.
  \item[] ~
\end{earg}

\end{multicols}

Some philosophers of language think that sentences that seem to presuppose the existence of something that isn't there aren't straightforwardly false, but are rather defective in some other way--- perhaps they fail to be meaningful, or perhaps they take on some truth value other than true or false.
These matters are beyond the scope of this book, and so we will remain neutral on whether Russell's theory is an accurate treatment of English.
Nevertheless, without including identity in the language, this question would not even arise.
This helps to bring out what is distinctive about the expressive power of QL$^=$ in contrast to QL. 







\section{Quantities}
  \label{sec:Quantities}

Including identity in QL$^=$ permits us to express claims about quantities that we couldn't in QL.
In \S\ref{sec:IdentityLogic} we considered the sentence `Mozart composed at least two things' where identity was found to play a critical role.
In particular, we provided the following regimentation:

\begin{earg}
  \item[\ref{M4}.] $\qt{\exists}{x}\qt{\exists}{y}((Cmx \eand Cmy) \eand x\neq y)$.
\end{earg}

Given that we were able to regiment `at least two', you might suspect that we can also regiment `at least three', and so on for the other natural numbers.
Consider the following:

\begin{earg}
  \item[\ex{Q1}] $\qt{\exists}{x}\qt{\exists}{y}\qt{\exists}{z}(((((Cmx \eand Cmy) \eand Cmz) \eand x\neq y) \eand x\neq z) \eand y\neq z)$.
\end{earg}

Though it is a lot longer than sentence $\ref{M4}$, the sentence above says that there are at least three things that Mozart composed.
Given that conjunction is associative and commutative, all but the outermost parentheses are more trouble than they are worth.
In general, we will indulge in the convention of dropping the parentheses that occur in long conjunctions and long disjunctions.
Thus we may rewrite sentence $\ref{Q1}$ as follows:

\begin{earg}
  \item[\ex{Q2}] $\qt{\exists}{x}\qt{\exists}{y}\qt{\exists}{z}(Cmx \eand Cmy \eand Cmz \eand x\neq y \eand x\neq z \eand y\neq z)$.
\end{earg}

This is a lot easier to read and nothing significant is lost.
It is important to stress that we can only drop parentheses in sentences which \textit{only} include conjunction, or \textit{only} include disjunction.
Even so, these sentences are bound to get very long for large values of $n$.

In order to characterize quantities in a more general way, it can be useful to introduce some abbreviations for what we will refer to as the \define{inequality quantifiers}.
Instead of adding new primitive symbols to QL$^=$, we are only providing conventions for abbreviating long expression with much shorter expressions for the sake of readability.

In order to state these abbreviations in a general way, we will take $\beta$ to be \define{free for} $\alpha$ in $\metaA$ just in case there is no free occurrence of $\alpha$ in $\metaA$ in the scope of a quantifier that binds $\beta$. 
For instance, $y$ is not free for $x$ in `$\forall y Fxy$' since replacing `$x$' with `$y$' would yield `$\forall yFyy$' where the quantifier `$\forall y$' would end up binding an extra variable.
Roughly speaking, you can take `$\beta$ is free for $\alpha$' to mean `$\beta$ can replace $\alpha$ without leading to extra binding'.

We may then take $\metaA\unisub{\beta}{\alpha}$ to be the result of replacing all free occurrences of $\alpha$ in $\metaA$ with $\beta$ where $\beta$ is required to be free for $\alpha$ in $\metaA$. 
For instance, $\forall yFxy\unisub{z}{x}$ is the wff $\forall yFzy$, and $\forall yFxy\unisub{y}{x}$ is undefined since $y$ is not free for $x$.
Given this notation, we may define the following abbreviations for quantifiers of the form `at least $n$ things are such that $\metaA$':

\vspace{-.2in}
\begin{align*}
  \qt{\exists_{\geq 1}}{\alpha}\metaA &\colonequals \qt{\exists}{\alpha}\metaA\\ 
  \qt{\exists_{\geq n+1}}{\alpha}\metaA &\colonequals \qt{\exists}{\alpha}(\metaA \eand \qt{\exists_{\geq n}}{\beta}(\alpha \neq \beta \eand \metaA\unisub{\beta}{\alpha})) \text{ where } \beta \text{ is free for } \alpha \text{ in } \metaA.
\end{align*}
\vspace{-.2in}

Here `$\colonequals$' represents that the left side is merely an abbreviation for the right side. 
These abbreviations have a recursive structure which defines $\qt{\exists_{\geq n}}{\alpha}$ for all $n$.
We can put these quantifiers to work to consider sets of sentences like:

\vspace{-.2in}
\begin{align*}
  \Gamma_{\infty} &\colonequals \set{\qt{\exists_{\geq n}}{x}(x=x): n\in\mathbb{N}}.
\end{align*}
\vspace{-.2in}

For any natural number $n$, the set $\Gamma_\infty$ includes a sentence that says at least $n$ things that are self-identical.
Not only may we show that there are models which satisfy $\Gamma$, these models must have infinite domains.
That we can begin to express claims about the infinite further demonstrates just how much more expressive power QL$^=$ has than QL. 

In addition to being able to say that there are at least $n$ things that satisfy a certain condition, QL$^=$ permits us to say that that there are at most $n$ things that satisfy a certain condition. 
Consider the following sentence and its regimentation:

\begin{earg}
  \item[\ex{Q3}] Mozart composed at most two things.
  \item[\ex{Q4}] $\qt{\forall}{x}\qt{\forall}{y}\qt{\forall}{z}((Cmx \eand Cmy \eand Cmz) \supset (x = y \eor x = z \eor y = z))$.
  % \item[\ex{Q4}] $\qt{\exists}{x}\qt{\exists}{y}\qt{\forall}{z}(Cmz \eif (z= x \eor z= y))$.
\end{earg}

This says that for any $x$, $y$, and $z$ which Mozart composed, at least two of them are identical. 
So far, nothing prevents all of them from being identical or requires there to be something which Mozart composed. 
% We may also that there at most three things that Mozart composed.
%
% \begin{earg}
%   \item[\ex{Q5}] Mozart composed at most three things.
%   \item[\ex{Q6}] $\qt{\forall}{x}\qt{\forall}{y}\qt{\forall}{z}\qt{\forall}{w}((Cmx \eand Cmy \eand Cmz \eand Cmw) \supset (x = y \eor x = z \eor y = z \eor y = z))$.
%   % \item[\ex{Q6}] $\qt{\exists}{x}\qt{\exists}{y}\qt{\exists}{z}\qt{\forall}{w}(Cmw \eif (w = x \eor w = y \eor w = z))$.
% \end{earg}
%
% Both sentences $\ref{Q4}$ and $\ref{Q6}$ are true if there are less than three things that Mozart composed, including the case where Mozart didn't compose anything.
More generally, we may say that there are at most $n$ things which satisfy a given condition $\metaA$.
Although we could define this recursively in a similar fashion to what was given above, we may avoid doing so by adopting the following convention for all $n$.

\vspace{-.2in}
\begin{align*}
  \qt{\exists_{\leq n}}{\alpha}\metaA &\colonequals \neg\qt{\exists_{\geq n+1}}{\alpha}\metaA.
\end{align*}
\vspace{-.2in}

This says that there are not at least $n+1$ things that are $\metaA$, and so no more than $n$ things that are $\metaA$. 
We may combine these two types of quantifiers to say that there are between $n$ and $m$ things that are $\metaA$ as given by the following abbreviation: 

\vspace{-.2in}
\begin{align*}
  \qt{\exists_{(n,m)}}{\alpha}\metaA &\colonequals \qt{\exists_{\geq n}}{\alpha}\metaA \eand \qt{\exists_{\leq m}}{\alpha}\metaA \text{ where } n\leq m.
\end{align*}
\vspace{-.2in}

In the special case where $n=m$, the statement $\qt{\exists_{(n,n)}}{\alpha}\metaA$ says that exactly $n$ things are $\metaA$. 
We have already seen instances of this above with uniqueness.
After all, saying that \textit{only} Cara loves Pedro is like saying there is \textit{exactly one} thing that loves Pedro where this entails both that there is \textit{at least} one thing that loves Pedro and that there is \textit{at most} one thing that loves Pedro, namely Cara.
We may refer to such quantifiers as \define{cardinality quantifiers}.
% Although this is one way to define cardinality quantifiers, there is another method.

Suppose we want to say that there are exactly two things that Mozart composed.
As above, one way to do this is to conjoin sentences $\ref{M4}$ and $\ref{Q4}$ since this amounts to saying that there is at least two things that Mozart composed and at most two things that Mozart composed. 
However, we can simplify the regimentation by means of the following:

\begin{earg}
  \item[\ex{Q7}] Mozart composed exactly two things.
  \item[\ex{Q8}] $\qt{\exists}{x}\qt{\exists}{y}(x \neq y \eand \qt{\forall}{z}(Cmz \eiff (z = x \eor z = y)))$.
  % \item[\ex{Q8}] $\qt{\exists}{x}\qt{\exists}{y}(Cmx \eand Cmy \eand x \neq y \eand \qt{\forall}{z}(Cmz \eif (z = x \eor z = y)))$.
\end{earg}

More generally, we may define the cardinality quantifiers as follows:

\vspace{-.2in}
\begin{align*}
  % \qt{\exists_{1}}{\alpha}\metaA(\alpha) &\colonequals \qt{\exists}{\alpha}(\metaA(\alpha) \eand \qt{\exists_0}{\beta}(\alpha \neq \beta \eand \metaA(\beta/\alpha))) \text{ for } \beta \text{ not free in } \metaA(\alpha)\\
  %   &\colonequals \qt{\exists}{\alpha}(\metaA(\alpha) \eand \qt{\forall}{\beta}\enot(\alpha \neq \beta \eand \metaA(\beta/\alpha))) \text{ for } \beta \text{ not free in } \metaA(\alpha)\\
  %   &\colonequals \qt{\exists}{\alpha}(
  %     \metaA(\alpha) \eand 
  %       \qt{\forall}{\beta}(
  %         \metaA(\beta/\alpha) \eif \alpha = \beta
  %       )
  %     ) \text{ for } \beta \text{ not free in } \metaA(\alpha)\\
  % \qt{\exists_{2}}{\alpha}\metaA(\alpha) &\colonequals \qt{\exists}{\alpha}(
  %   \metaA(\alpha) \eand 
  %     \qt{\exists_1}{\beta}(
  %       \alpha \neq \beta \eand \metaA(\beta/\alpha) 
  %     )
  %   ) \text{ for } \beta \text{ not free in } \metaA(\alpha)\\
  % &\colonequals \qt{\exists}{\alpha}(
  %   \metaA(\alpha) \eand \qt{\exists}{\beta}(
  %     \alpha \neq \beta \eand \metaA(\beta/\alpha) \eand 
  %       \qt{\forall}{\gamma}(
  %         (\alpha \neq \gamma \eand \metaA(\gamma/\alpha)) \eif \beta = \gamma
  %       )
  %     ) 
  %   ) \text{ for } \gamma \text{ not free in } \metaA(\beta)\\ 
  \qt{\exists_0}{\alpha}\metaA &\colonequals \qt{\forall}{\alpha}\enot\metaA\\
  \qt{\exists_{n+1}}{\alpha}\metaA &\colonequals \qt{\exists}{\alpha}(\metaA \eand \qt{\exists_n}{\beta}(\alpha \neq \beta \eand \metaA(\beta/\alpha))) \text{ for } \beta \text{ not free in } \metaA
  % \qt{\exists_1}{\alpha}\metaA &\colonequals \qt{\exists}{\alpha}(\metaA \eand \qt{\exists_0}{\beta}(\alpha\neq\beta \eand \metaA(\beta/\alpha))) \text{ for } \beta \text{ not free in } \metaA\\
  % \qt{\exists_1}{\beta}\metaA &\colonequals \qt{\exists}{\beta}(\metaA \eand \qt{\exists_0}{\gamma}(\beta\neq\gamma \eand \metaA(\gamma/\beta))) \text{ for } \gamma \text{ not free in } \metaA\\
  % \qt{\exists_2}{\alpha}\metaA &\colonequals \qt{\exists}{\alpha}(\metaA \eand \qt{\exists_1}{\beta}(\alpha\neq\beta \eand \metaA(\beta/\alpha))) \text{ for } \beta \text{ not free in } \metaA\\
  % \qt{\exists_{n+1}}{\alpha}\metaA &\colonequals \qt{\exists}{\alpha}(\metaA \eand \qt{\exists_n}{\beta}(\beta\neq\alpha \eand \metaA(\beta/\alpha)) \text{ for } \beta \text{ not free in } \metaA\\
\end{align*}

We know what it is for no $\alpha$ to be $\metaA$--- that much is easy.
What it is for $n+1$ things to be such that $\metaA$ is for something to be $\metaA$ and $n$ other things to be such that $\metaA$. 
Given these recursive definitions, we may work out the following quantifiers:

\vspace{-.2in}
\begin{align*}
  \qt{\exists_0}{\alpha}\metaA &\colonequals \qt{\forall}{\alpha}\enot\metaA\\
  \qt{\exists_1}{\alpha}\metaA &\colonequals \qt{\exists}{\alpha}\qt{\forall}{\beta}(\metaA(\beta/\alpha) \eiff \beta = \alpha)\\ 
  % \text{ for distinct } \alpha \text{ and } \beta\\
  \qt{\exists_2}{\alpha}\metaA &\colonequals \qt{\exists}{\alpha}\qt{\exists}{\beta}(\alpha\neq\beta \eand \qt{\forall}{\gamma}(\metaA(\gamma/\alpha) \eiff (\gamma = \alpha \eor \gamma = \beta)))\\
  % \text{ for distinct } \alpha, \beta, \text{ and } \gamma\\ 
  \qt{\exists_3}{\alpha}\metaA &\colonequals \qt{\exists}{\alpha}
      \qt{\exists}{\beta}
        \qt{\exists}{\gamma}(
          \alpha\neq\beta \eand \alpha\neq\gamma \eand \beta\neq\gamma \eand \qt{\forall}{\delta}(
            \metaA(\delta/\alpha) \eiff (
              \delta = \alpha \eor \delta = \beta \eor \delta = \gamma
            )
          )
        )\\
  % \text{ for distinct } \alpha, \beta, \text{ and } \gamma\\ 
  & ~\vdots
\end{align*}

Although the results differ in logical form from those that we may derive from $\qt{\exists_{(n,n)}}$ for different values of $n$, they are logically equivalent.
% The fact that these cardinality quantifiers are definable in QL$^=$ demonstrates the expressive power of this language in contrast to QL. 
Whereas the inequality operators are interesting in their own right, the cardinality operators express something very specific, further demonstrating the expressive power of QL$^=$.
% For instance, we may consider a set of sentence $\Gamma=\set{}$
% Given these definitions, we may consider the following set of sentences.




\section{Leibniz's Law}

In $\S\ref{sec:IdentityLogic}$, we saw that treating identity as any other predicate invalidated the argument:

\begin{multicols}{2}
  
\begin{earg}
  \item[] Hesperus is rising.
  \item[] Hesperus is Phosphorus.
  \item[\therefore] Phosphorus is rising.
\end{earg}

\begin{earg}
  \item[] $Rh$
  \item[] $Ihp$
  \item[\therefore] $Rp$
\end{earg}

\end{multicols}

The problem was that if the extension of the identity predicate can be assigned to any subset of $\D^2$, there was no guarantee that $h$ and $p$ refer to the same object.
The problem was avoided by including identity among the primitive symbols of the language and providing its semantics as above.
We may replace `$I$' with `$=$' in the argument above where the result is valid.
This is because if `$h=p$' is true in a model, then `$h$' and `$p$' refer to the very same object in the domain, and so `$Rh$' and `$Rp$' will have the same truth-value.
% Consequently, given any sentence $\metaA(\alpha)$ in which the constant $\alpha$ occurs, the sentence $\metaA(\beta/\alpha)$  that results from replacing each occurrence of $\alpha$ in $\metaA(\alpha)$ with $\beta$ will have the same truth value as $\metaA(\alpha)$ if $\alpha=\beta$.
% This is because identity is transitive and identical objects have the same properties.
More generally, we may show that all instances of the following schema are valid where $\alpha$ and $\beta$ are constants and $\metaA$ is a sentence, referring to this as a version of \textit{Leibniz's Law}:
  $$\alpha=\beta \vDash \metaA \equiv \metaA(\beta/\alpha).$$
The idea behind this schema is that if we are given two names for the same object, then whatever we can say about that object with one name will equally apply if we use the other name.
This might seem natural in the case above.
After all, if Hesperus is Phosphorus, how could it be that Hesperus is rising without Phosphorus rising?

However compelling this particular instance may be, Leibniz's Law admits of a wide range of exceptions.
For instance, consider the following argument:

\begin{multicols}{2}
  
\begin{ekey}
    \item[Bxy:] $x$ believes that $y$ is rising.
    \item[t:] Thales
    \item[h:] Hesperus
    \item[p:] Phosphorus
\end{ekey}

\begin{earg}
  \item[] $Bth$
  \item[] $h=p$
  \item[\therefore] $Btp$
  \item[] ~
\end{earg}

\end{multicols}

Since the identity of Hesperus and Phosphorus was not always known, we might imagine a time when Thales believed that Hesperus is rising without also believing that Phosphorus is rising.
As a result, the argument above may have true premises and a false conclusion.
To take another case, we may imagine that Lois loves Clark Kent and Lois does not love Superman despite the fact that Clark Kent is Superman.
Nevertheless, both of these arguments are valid given the semantics for QL$^=$.
Has something gone wrong?
% Do we need to somehow modify the semantics we provided for identity?

The constants $\alpha$ and $\beta$ may be said to \textit{co-refer} just in case they name the same thing, i.e., $\alpha=\beta$.
One common response holds that some claims are \textit{opaque} insofar as we cannot freely substitute co-referring constants.\footnote{Alternatively, one may attribute the opacity of `Thales believes that Hesperus is rising' to the opacity of the `Thales believes that' operator. Such suggestions are developed in epistemic logics.}
Belief claims are paradigmatic of opacity: just because one believes that $\metaA$ doesn't mean that one must believe that $\metaA(\beta/\alpha)$ whenever $\alpha=\beta$.
For instance, suppose that Kaya is learning arithmetic and has come to believe that $2$ is even.
Nevertheless, she hasn't learned anything about prime numbers and so does not believe that the first prime number is even despite the fact that $2$ is the first prime number. 

% In QL$^=$, we will take co-referring terms to always be substitutable for each other as claimed by the schema above.
Given that Leibniz's Law is valid over the semantics for QL$^=$, restricting Leibniz's Law requires significant revisions to the present semantics.
For our purposes here, we will assume that co-referring terms can always be substituted for each other as asserted by the schema above.
This amounts to the assumption that none of the sentences with which we will be concerned are opaque.
This is a significant limitation since it is easy to introduce predicates like $B$ given above. 
% One way to remedy the situation is to introduce a set of \textit{opaque predicates} $O^n_1,O^n_2,\ldots$ for each $n\geq 0$ which do not admit of free substitution. 
% Whereas the arguments above are valid when regimented with predicates that are not opaque, they are invalid when regimented with opaque predicates.
Nevertheless, overcoming this limitation is far from straightforward and lies outside the scope of our present concern.
Although QL$^=$ is a flexible and expressive powerful language, every language has its limits.
Nevertheless, QL$^=$ is perfectly adequate for a wide range of applications. 
In particular, it is natural to assume that mathematics is \textit{transparent} insofar as it does not include any opaque claims.
% Although QL$^=$ will remain useful for taking about the way things are, QL$^=$ will not permit us to speak  









\iffalse

\practiceproblems

\solutions
\problempart
\label{pr.QL-ID-spies}
Using the symbolization key given, translate each English-language sentence into QL with identity. For sentences containing definite descriptions, assume Russell's theory.
\begin{ekey}
\item[UD:] people
\item[Kx:] $x$ knows the combination to the safe.
\item[Sx:] $x$ is a spy.
\item[Vx:] $x$ is a vegetarian.
\item[Txy:] $x$ trusts $y$.
\item[h:] Hofthor
\item[i:] Ingmar
\end{ekey}
\begin{earg}
\item Hofthor is a spy, but no vegetarian is a spy.
\item No one knows the combination to the safe unless Ingmar does.
\item No spy knows the combination to the safe.
\item Neither Hofthor nor Ingmar is a vegetarian.
\item Hofthor trusts a vegetarian.
\item Everyone who trusts Ingmar trusts a vegetarian.
\item Everyone who trusts Ingmar trusts someone who trusts a vegetarian.
\item Only Ingmar knows the combination to the safe.
\item Ingmar trusts Hofthor, but no one else.
\item The person who knows the combination to the safe is a vegetarian.
\item The person who knows the combination to the safe is not a spy.
\end{earg}

\solutions
\problempart
\label{pr.QL-ID-cards}
Using the symbolization key given, translate each English-language sentence into QL with identity. For sentences containing definite descriptions, assume Russell's theory.
\begin{ekey}
\item[UD:] cards in a standard deck
\item[Bx:] $x$ is black.
\item[Cx:] $x$ is a club.
\item[Dx:] $x$ is a deuce.
\item[Jx:] $x$ is a jack.
\item[Mx:] $x$ is a man with an axe.
\item[Ox:] $x$ is one-eyed.
\item[Wx:] $x$ is wild.
\end{ekey}
\begin{earg}
\item All clubs are black cards.
\item There are no wild cards.
\item There are at least two clubs.
\item There is more than one one-eyed jack.
\item There are at most two one-eyed jacks.
\item There are exactly two black jacks.
\item There are exactly four deuces.
\item The deuce of clubs is a black card.
\item One-eyed jacks and the man with the axe are wild.
\item If the deuce of clubs is wild, then there is exactly one wild card.
\item The man with the axe is not a jack.
\item The deuce of clubs is not the man with the axe.
\end{earg}

\solutions
\problempart
\label{pr.QLbuffy}
Using the symbolization key given, translate each English-language sentence into QL with identity.
\begin{ekey}
\item[UD:] people, generations, and monsters
\item[Gx:] $x$ is a generation.
\item[Hx:] $x$ is human.
\item[Sx:] $x$ is a slayer.
\item[Vx:] $x$ is a vampire.
\item[Dx:] $x$ is a demon.
\item[Wx:] $x$ is a werewolf.
\item[Fx:] $x$ is a force of darkness.
\item[Axy:] $x$ will stand against $y$.
\item[Bxy:] $x$ is born in generation $y$.
\item[Kxy:] $x$ will kick $y$.
\item[b:] Buffy
\item[f:] Faith
\item[w:] Willow
\end{ekey}
\begin{earg}
\item Buffy and Willow were born unto the same generation.
\item There is no more than one slayer born in each generation.
\item A slayer other than Buffy is one of the forces of darkness.
\item Willow will stand against any force of darkness other than a werewolf.
\item Faith will kick everyone except herself.
\item Buffy will kick anyone who stands against a slayer, unless they are also kicking vampires or demons.
\item In every generation a slayer is born.
\item In every generation a slayer is born. She will stand against the vampires, demons, and forces of darkness.
\item In every generation a slayer is born. She alone will stand against the vampires, demons, and forces of darkness.
\end{earg}



\problempart Using the symbolization key given, translate each English-language sentence into QL with identity. For sentences containing definite descriptions, assume Russell's theory.
\begin{ekey}
\item[UD:] animals in the world
\item[Bx:] $x$ is in Farmer Brown's field.
\item[Hx:] $x$ is a horse.
\item[Px:] $x$ is a Pegasus.
\item[Wx:] $x$ has wings.
\end{ekey}
\begin{earg}
\item There are at least three horses in the world.
\item There are at least three animals in the world.
\item There is more than one horse in Farmer Brown's field.
\item There are three horses in Farmer Brown's field.
\item There is a single winged creature in Farmer Brown's field; any other creatures in the field must be wingless.
\item The Pegasus is a winged horse.
\item The animal in Farmer Brown's field is not a horse.
\item The horse in Farmer Brown's field does not have wings.
\end{earg}

\solutions
\problempart Demonstrate each of the following, either by constructing a model, or by explaining why it's impossible to do so. If you wish, you can draw a tree to help you answer these questions; however, it is good conceptual practice to tackle some of these questions directly by thinking about just what you'd need to put in your model.
\label{pr.IdentityModels}
\begin{earg}
\item Show that $\{{\enot}Raa, \qt{\forall}{x} (x{=}a \eor Rxa)\}$
is consistent.
%There are many possible answers. Here is one:
%\begin{partialmodel}
%UD & \{Harry, Sally\}\\
%\extension{R} &\{\ntuple{Sally, Harry}\}\\
%\referent{a} & Harry
%\end{partialmodel}
\item\ Show that $\{\qt{\forall}{x}\qt{\forall}{y}\qt{\forall}{z}(x{=}y \eor y{=}z \eor x{=}z),
\qt{\exists}{x}\qt{\exists}{y}\ x{\neq} y\}$ is consistent.
%There are no predicates or constants, so we only need to give a UD.
%Any UD with 2 members will do.
\item\ Show that $\{\qt{\forall}{x}\qt{\forall}{y}\ x{=}y, \qt{\exists}{x}\ x {\neq} a\}$ is inconsistent.
%We need to show that it is impossible to construct a model in which these are both true. Suppose $\qt{\exists}{x}\ x {\neq} a\$ is true in a model. There is something in the universe of discourse that is \emph{not} the referent of $a$. So there are at least two things in the universe of discourse: \referent{a} and this other thing. Call this other thing \metaB{}--- we know $a {\neq} \metaB{}$. But if $a {\neq} \metaB{}$, then $\qt{\forall}{x}\qt{\forall}{y}\ x=y$ is false. So the first sentence must be false if the second sentence is true is true. As such, there is no model in which they are both true. Therefore, they are inconsistent.
\item Show that $\qt{\exists}{x} (x {=} h \eand x {=} i)$ is contingent.
\item Show that \{$\qt{\exists}{x}\qt{\exists}{y}(Zx \eand Zy \eand x{=}y)$, $\enot Zd$, $d{=}s$\} is consistent.
\item Show that `$\qt{\forall}{x}(Dx \eif \qt{\exists}{y} Tyx)$ \therefore\ $\qt{\exists}{y} \qt{\exists}{z}\ y{\neq} z$' is invalid.
\end{earg}

\solutions
\problempart Construct a tree to test the following entailment claims. If they are false, provide a model that demonstrates this.
\label{pr.IdentityTrees}
\begin{earg}
\item\  $\models \qt{\forall}{x} \qt{\forall}{y} (x{=}y \eif y{=}x)$
\item\ $\models \qt{\forall}{x} \qt{\exists}{y} \: x{=}y$
\item\  $\models \qt{\exists}{x} \qt{\forall}{y} \: x{=}y$
\item   $\qt{\exists}{x} \qt{\forall}{y} \: x{=}y \models \qt{\forall}{x} \qt{\forall}{y} (Rxy \eiff Ryx)$
\item   $\models \enot \qt{\forall}{x} \qt{\forall}{y} \qt{\forall}{z} [(Axy \eand Azx \eand y{=}z) \eif Axx] $
\item  $\qt{\forall}{x} \qt{\forall}{y} \: x{=}y \models \qt{\exists}{x} Fx \eiff \qt{\forall}{x} Fx$
\item $\qt{\forall}{x} (x{=}a \eor x{=}b), Ga \eiff \enot Gb \models \enot \qt{\exists}{x} \qt{\exists}{y} \qt{\exists}{z} (Gx \eand Gy \eand \enot Gz)$
\item $\qt{\forall}{x} (Fx \eif x{=}f), \qt{\exists}{x} (Fx \eor \qt{\forall}{y} \: y{=}x) \models Ff$
\item $\qt{\exists}{x} \qt{\exists}{y} Dxy \models \qt{\forall}{x_{1}} \! \qt{\forall}{x_{2}}\! \qt{\forall}{x_{3}}\! \qt{\forall}{x_{4}}\! [(Dx_{1}x_{2} \eand Dx_{3}x_{4}) \eif (x_{2}{\neq}x_{3} \eor Dx_{1}x_{4})]$
\end{earg}

\problempart In \S \ref{sec.quantity} we looked at two different translations of `Mozart composed exactly two things'. Use trees to prove that they are equivalent.

\problempart Translate these arguments into QL with identity, and evaluate them for validity with a tree. (Don't be surprised or discouraged if some of these trees end up very complex.)
\label{pr.IdentityArguments}
\begin{earg}
\item Dudley will threaten anyone who threatens anyone. Therefore, Dudley will threaten himself.
\item The exam is easy. Therefore every exam Sheila took was easy. (Use Russell's theory of definite descriptions.)
\item Three wise men visited Jesus. Every wise man who visited Jesus gave Jesus a gift. Therefore, Jesus received more than one gift.
\item Worf is the only Klingon in Starfleet. Everyone in Starfleet is brave. All brave Klingons are warriors. Therefore, there is at least one brave warrior in Starfleet.
\item Worf is the only Klingon in Starfleet. Everyone in Starfleet is brave. All brave Klingons are warriors. Therefore, there is exactly one brave warrior in Starfleet.
\item Every person likes every kind of sandwich that is tasty. Jack is a person. Jack likes exactly one kind of sandwich. Therefore, no more than one kind of sandwich is tasty.
\end{earg}


\fi 
