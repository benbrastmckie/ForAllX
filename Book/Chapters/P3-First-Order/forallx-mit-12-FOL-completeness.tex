%!TEX root = ../forallx-mit.tex
\chapter{The Completeness of FOL$^=$}
\label{ch.FOL-completeness}

\section{Introduction}
  \label{sec:Introduction}

Consider the calculator from before that can compute basic arithmetical operations.
If that calculator sometimes gave false answers we probably shouldn't call it a calculator at all.
Although it may often given the right results, there would be no way to know if it was giving us the right result or not, and so could not be relied upon.
Put otherwise, the calculator is not \textit{sound} with respect to the truths of arithmetic since some of its answers are false.
For an analogous reason, it was important to show that FOL$^=$ was sound over the semantics for $\FI$ so that we could rely on FOL$^=$ to conduct valid reasoning.
If $\metaA$ is derivable from $\MetaG$, we know by soundness that $\metaA$ is also a logical consequence of $\MetaG$.
Put formally: if $\MetaG\vdash\metaA$, then $\MetaG\vDash\metaA$.

Completeness asserts the converse so that we may conclude that $\metaA$ is derivable from $\MetaG$ whenever $\metaA$ is a logical consequence of $\MetaG$, or more compactly: if $\MetaG\vDash\metaA$, then $\MetaG\vdash\metaA$.
You might recall that our calculator from before is not complete.
In fact, no calculator is complete for purely material reasons: no matter how much memory a calculator may have, there are numbers big enough that will exhaust its memory.
For instance, raising one large number to another large number will quickly use up the memory.
As you may have observed in elementary school, there are some arithmetical operations that the calculator simply cannot compute, yielding `ERROR' as a result.
So long as the calculator doesn't spit out any false answers--- i.e., it is sound--- it is still be of considerable use despite its incompleteness.

A separate question is whether, in principle, there could be an effective procedure which yields the right answers to any arithmetical operations.
By `effective procedure' we do not mean a material computing device but rather an abstract method which could be fully specified with precise rules which one could in principle follow to compute the result of any arithmetical operations.
It turns out that there is no such effective procedure for arithmetic.
Put otherwise, arithmetic is incomplete.
FOL$^=$ does not share this same fate.
Rather, we will show in the following section that whenever a conclusion is a logical consequence of some premises, that conclusion is derivable from those premises in FOL$^=$.
% This is an impressive result.

Much will be as before where instead of beginning with $\MetaG\vDash\metaA$ as an assumption and arguing to the conclusion $\MetaG\vdash\metaA$, we will focus on establishing a closely related result: 
\begin{enumerate}[leftmargin=1.5in]
  \item[\textbf{\ref{thm:FOL-conscomp}}] Every consistent set of $\FI$ sentences $\MetaG$ is satisfiable. 
\end{enumerate}
Recall that a set of $\FI$ sentences $\MetaG$ is inconsistent if $\bot$ is derivable from $\MetaG$, and consistent otherwise.
Assuming $\MetaG\vDash\metaA$, we know that $\MetaG\cup\set{\enot\metaA}$ is unsatisfiable by \textbf{\ref{lemma:PL-unsat_consequence}}, and so inconsistent by \textbf{\ref{thm:FOL-conscomp}}.
It follows that $\MetaG\cup\set{\enot\metaA}\vdash\bot$, and so we may derive a contradiction from $\MetaG\cup\set{\enot\metaA}$ in FOL$^=$.
Given $\MetaG$ as premises and $\enot\metaA$ as an assumption, it follows by negation elimination that $\MetaG \vdash \metaA$, establishing completeness (see \textbf{\ref{cor:FOL-completeness}}).

It remains to establish \textbf{\ref{thm:FOL-conscomp}}.
The proof will proceed in a number of stages.
Whereas PL only concerned the wfs of $\PL$ which build up complex sentences from sentence letters and the truth-functional operators, we must now take into consideration the wffs that can be constructed from predicates, constants, and variables, building up more complex wffs with the truth-functional operators and quantifiers which bind the variables that fall within their scope.
Accordingly, we will extend our language to include a countably infinite number of new constants.
For simplicity, we will use the set of natural numbers $\N$, calling our new language $\FIN$. 
Just as we added a countably infinite number of constants $\zeta_0, \zeta_1, \ldots$ for each lower case letter $\zeta \in \set{a, b, \ldots, t}$, we have effectively added one more countably infinite stock of constants to use to build sentences.
Assuming that $\MetaG$ is consistent in $\FI$, we will show in $\S\ref{sub:Witnesses}$ that $\MetaG$ is also consistent in $\FIN$.
This completes the first stage of the proof.

You might be wondering what difference a few extra constants will make to our language given how many constants we had to begin with.
The reason for this addition is that we would like to extend $\MetaG$ to a bigger set of sentences which is guaranteed to include instances of every existential claim. 
We will refer to these instances as \textit{witnesses}.
That is, if $\qt{\exists}{x}Fxa$ is a sentence in $\MetaG$, then we would like to include an instance such as $F1a$ where our instantiating constant is guaranteed not to conflict with anything else in $\MetaG$.
An easy way to do this is to draw on our set of new constants.
Since $\MetaG$ is still the same set of wfss from $\FI$, none of our new constants occur in any sentence in $\MetaG$.
For instance $1$ does not occur in any sentence in $\MetaG$, and so our instance $F1a$ will not conflict with what belongs to $\MetaG$. 
Extending $\MetaG$ to include witnesses for all existential claims constitutes the second stage of the proof where we will refer to this larger set $\MetaS_\MetaG$ as \textit{saturated}. 
As we will show in $\S\ref{sub:Saturation}$, $\MetaS_\MetaG$ is also consistent. 

The next stage will be familiar from before where we will extend $\MetaS_\MetaG$ to includes every sentence or its negation but not both, calling this \textit{maximal} set $\MetaD_{\MetaS_\MetaG}$, or just $\MetaD$ for short. 
We will show in $\S\ref{sub:Maximization}$ that $\MetaD$ is consistent, where it follows that $\MetaD$ is \textit{deductively closed} insofar as it contains every sentence that is derivable from $\MetaD$.
Deductive closure is a very important and convenient property which will play a critical role in the later stages of the proof.

Having extended $\MetaG$ to a much bigger set of sentences $\MetaD$ that is saturated, maximal, consistent, and deductively closed, we will proceed to use this set to construct a model that shows $\MetaD$ is satisfiable, and so $\MetaG$ is satisfiable as a result. 
Accordingly, it will be convenient to say that a model $\M$ \textsc{satisfies} a set $\Lambda$ of wfss just in case $\VV{\I}{}(\metaC) = 1$ for all $\metaC \in \Lambda$. 

As in the proof of \textsc{PL Completeness}, we will use the set of wfss $\MetaG$ to build a model that satisfies $\MetaG$.
However strange it may be to use the symbols that make up the syntax of a language to provide a model of that language, there is no circularity.
Rather, to interpret a first-order language like $\FI$ and $\FIN$, all we need is a domain where any nonempty set will suffice.
In what follows, we will interpret $\FIN$ over the domain $\D_\MetaD$ whose members are sets of constants in our extended language $\FIN$ which we will define below.
Although these might be obscure objects to think about, nothing prevents us from using them to interpret $\FIN$.
% So long as we have a nonempty set of them, we can proceed to interpret our predicates over this domain.

Given $\D_{\MetaD}$, we will specify referents and extensions for the constants and predicates of our language in such a way that the resulting model $\M_\MetaD$ satisfies all and only the sentences that belong to $\MetaD$.
As a result, $\M_\MetaD$ satisfies $\MetaD$, and since $\MetaG\subseteq\MetaD$, we may conclude that $\MetaG$ is satisfiable. 
As before, we will refer to this cleverly constructed model $\M_\MetaD$ as a Henkin model after Leon Henkin who developed this proof strategy in 1949.

This provides a rough overview of the proof strategy that will be deployed below.
If you find that you get lost along the way, it can help to return to this overview to regain your bearings and keep track of what is happening and why.
Slogging on in the dark is rarely advisable.
% Rather, it is better to keep zooming out so that you can keep track of where you are and where you are headed to next.







\section{Extensions}%
  \label{sec:Extensions}

Assume $\MetaG$ is a consistent set of $\FI$ sentences.
We will maintain this assumption throughout the following sections in order to show that $\MetaG$ is satisfiable. 
The following section begins by constructing an extension of $\MetaG$ called $\MetaD$--- i.e., where $\MetaG \subseteq \MetaD$---  which we will show is saturated, maximal, and consistent.
These properties will enable us to show that $\MetaD$ is \define{deductively closed} insofar as $\metaA\in\MetaD$ whenever $\MetaD\vdash\metaA$. 
Deductive closure will play a critical role in constructing a Henkin model $\M_\MetaD$ which satisfies $\MetaD$, and so satisfies $\MetaG$ as a consequence. 
% The following section will begin by defining $\FIN$.





\subsection{Witnesses}%
  \label{sub:Witnesses}
  

Let $\FIN$ be a language like $\FI$ except for including the natural numbers $\N$ as an additional set of constants.
Even though $\MetaG$ is consistent in $\FI$, it does not immediately follow that $\MetaG$ is consistent in $\FIN$.
In general, adding expressive resources to a language can provide the grounds for new derivations and so we need to check that a contradiction cannot be derived in $\FIN$ from $\MetaG$.
In order to rule out this possibility out, we will prove the following lemma.
Assuming that $\beta$ is free for $\alpha$ in every line of a proof $X$, it will be convenient to take $X\unisub{\beta}{\alpha}$ to be the result of replacing $\beta$ with $\alpha$ in every line of $X$.

% \begin{Lthm} \label{lemma:gencon}
%   If $\MetaG\vdash\metaA$ and $\beta$ is a constant that does not occur in any $\metaB\in\MetaG$, then there is a variable $\alpha$ which does not occur in $\metaA$ such that $\MetaG\vdash \qt{\forall}{\alpha}\metaC$ where $\metaC=\metaA\unisub{\alpha}{\beta}$ and the derivation of $\qt{\forall}{\alpha}\metaC$ from $\MetaG$ does not include $\beta$.
% \end{Lthm}
%
% \begin{quote} 
%   \textit{Proof:} Assume $\MetaG\vdash\metaA$ where $\beta$ is a constant that does not occur in any $\metaB\in\MetaG$.
%   Thus there is some FOL$^=$ proof $X$ of $\metaA$ from $\MetaG$ where $\beta$ does not occur in any line of $X$. 
%   Let $\alpha$ be the first variable that does not occur in any line of $X$, and let $X\unisub{\alpha}{\beta}$ be the result of substituting $\alpha$ for $\beta$ in ever line of $X$.
%   Since $\beta$ does not occur in any $\metaB\in\MetaG$, it follows that $X\unisub{\alpha}{\beta}$ is a proof $\metaA\unisub{\alpha}$ from $\MetaG$ 
% \end{quote}

\begin{Lthm} \label{lemma:FOL-prsub}
  If $\alpha$ is a constant and $X$ is an FOL$^=$ derivation in which the constant $\beta$ does not occur, then $X\unisub{\beta}{\alpha}$ is also an FOL$^=$ derivation.
\end{Lthm}

\begin{quote} 
  \textit{Proof:}
  This proof is left as an exercise for the reader.
  % Assume that $\alpha$ is a constant and $X$ is a proof in which the constant $\beta$ does not occur.
  % The proof goes by a routine induction on the length of $X$, and so the details will be omitted.
  \qed
  %   For easy of exposition, we will take $\metaA_i$ to be the $i$\textsuperscript{th} line of $X$ and $\MetaG_i$ to be the undischarged assumptions at line $i$.
  %   We then aim to show that $\MetaG_i\vDash\metaA_i$ for ever line $i$ in $X$. 
  %
  %   \textit{Base:} We know that $\metaA_1$ is either a premise, assumption, or instance of identity introduction $\alpha=\alpha$ for a constant $\alpha$. 
  %   In each case, $\MetaG_1\vdash$
\end{quote}


Whereas the rules of FOL$^=$ were defined for the wfss of $\FI$, they may just as easily be defined for $\FIN$, referring to the resulting proof system as FOL$^=_\N$. 
Recalling the definition of a FOL$^=$ \define{derivation} from Chapter \ref{ch.FOL-deduction}, we may similarly define the analogue for FOL$^=_\N$ to be just like it was before while drawing on the wider range of wfss of $\FIN$.

\factoidbox{
  A \define{derivation} (or \define{proof}) of $\metaA$ from $\MetaG$ in FOL$^=_\N$ is any finite sequence of wfss of $\FIN$ ending in $\metaA$ where every wfs in the sequence is either: (1) a premise in $\MetaG$; (2) an assumption which is eventually discharged; or (3) follows from previous lines by a natural deduction rule for FOL$^=_\N$ besides AS. 
}

Recall that consistency was relative to a proof system.
Whereas a set of wfss $\MetaG$ of $\FI$ is \define{consistent} in FOL$^=$ just in case $\MetaG \nvdash \bot$ in FOL$^=$, we may also say that a set of wfss $\MetaG$ of $\FIN$ is \define{consistent} in FOL$^=_\N$ just in case $\MetaG \nvdash \bot$ in FOL$^=_\N$.
We may then prove the following.

\begin{Lthm} \label{lemma:FOL-const}
  If $\MetaG$ is consistent in FOL$^=$, then $\MetaG$ is also consistent in FOL$^=_\N$.
\end{Lthm}

\begin{quote} 
  \textit{Proof:} Assume that $\MetaG$ is a consistent in FOL$^=$.
  Assume for contradiction that $\MetaG$ is inconsistent in FOL$^=_\N$, and so $\MetaG\vdash A\eand\enot A$ in FOL$^=_\N$.
  Thus there is a derivation $X$ of $A\eand\enot A$ from $\MetaG$ where every line of $X$ is a wfs of $\FIN$.
  Since every proof is finite, there are at most finitely many constants that occur in $X$, and so at most finitely many constants in $X$ that belong to $\N$.

  Letting $\vec{n}=\tuple{n_1,\ldots,n_m}$ include all constants in $\N$ that occur in $X$, we may take $\vec{\alpha}=\tuple{\alpha_1,\ldots,\alpha_m}$ to be a sequence of $\FI$ constants where $\alpha_i$ is the $i$\textsuperscript{th} $\FI$ constant not to occur in $X$.
  By defining $X\unisub{\vec{\alpha}}{\vec{n}}\colonequals X\unisub{\alpha_1}{n_1}\ldots\unisub{\alpha_m}{n_m}$ to be the result of substituting $\alpha_i$ for $n_i$ in for all $1\leq i\leq n$ in every line of $X$, it follows by $m$ applications of \textbf{\ref{lemma:FOL-prsub}} that $X\unisub{\vec{\alpha}}{\vec{n}}$ is a proof of $(A\eand\enot A)\unisub{\vec{\alpha}}{\vec{n}}$ from $\MetaG\unisub{\vec{\alpha}}{\vec{n}}$ in FOL$^=$ since every line in $X\unisub{\vec{\alpha}}{\vec{n}}$ is a $\FI$ wfs.
  % where  $\MetaG\unisub{\vec{\alpha}}{\vec{n}}\colonequals\set{\metaB\unisub{\vec{\alpha}}{\vec{n}}:\metaB\in\MetaG}$.
  % Observe that $X\unisub{\vec{\alpha}}{\vec{n}}=X\unisub{\alpha_1}{n_1}\ldots\unisub{\alpha_m}{n_m}$.

  Since $\MetaG$ is consistent in FOL$^=$, we know that $\metaB$ is a $\FI$ wfs for every $\metaB\in\MetaG$, and so $\MetaG\unisub{\vec{\alpha}}{\vec{n}}=\MetaG$.
  Similarly, $(A\eand\enot A)\unisub{\vec{\alpha}}{\vec{n}}=(A\eand\enot A)$ which is a wfs of $\FI$. 
  As a result, $X\unisub{\vec{\alpha}}{\vec{n}}$ is a proof of $A\eand\enot A$ from $\MetaG$ in FOL$^=$, and so $\MetaG$ is not consistent in FOL$^=$, contradicting the above.
  Thus $\MetaG$ is consistent in FOL$^=_\N$.
  \qed
\end{quote}

Although the proof of \textbf{\ref{lemma:FOL-const}} is not immediate, it is hardly surprising that merely adding new constants would enable the derivation of a contradiction from $\MetaG$ when no contradiction is derivable from $\MetaG$ without those additional constants.
Since $\MetaG$ is consistent in $\FI$, we may conclude that $\MetaG$ is consistent in $\FIN$.
We may now proceed to extend $\MetaG$ further, using all of the expressive resources of the extended language $\FIN$. 
In particular, the following section will begin by finding a constant from $\N$ to witness each existential claim that may occur in $\MetaG$ so that we never end up in a situation where an existential claim is true, but no particular instance is true. 
Nevertheless, this is one of the more opaque portions of the proof where it will only become clear later on why the definitions given here are needed.





\subsection{Saturation}%
  \label{sub:Saturation}

We will now move to extend $\MetaG$ to a saturated set of wfss.
Letting $\metaA(\alpha)$ be a wff of $\FIN$ with at most one free variable $\alpha$, we may take a set of wfss $\MetaS$ to be \define{saturated} in $\FIN$ just in case for each wff $\metaA(\alpha)$ of $\FIN$, there is a constant $\beta$ in $\FIN$ where $(\qt{\exists}{\alpha}\metaA \eif \metaA\unisub{\beta}{\alpha})\in\MetaS$.
In order to extend $\MetaG$ to a saturated set $\MetaS$, fix an enumeration $\metaA_1(\alpha_1),\metaA_2(\alpha_2),\metaA_3(\alpha_3),\ldots$ of all wffs of $\FIN$ with at most one free variable. %, and let $n_0$ be the first natural number that does not occur in $\metaA_0(\alpha_0)$.
We may then provide the following recursive definition:
\begin{enumerate}[leftmargin=1in]
  \item[\it $\theta$-Base:] $\theta_1=(\qt{\exists}{\alpha_1}\metaA_1\eif \metaA_1\unisub{n_1}{\alpha_1})$ where $n_1\in\N$ is the first constant not in $\metaA_1$.
  \item[\it $\theta$-Recursion:] $\theta_{k+1}=(\qt{\exists}{\alpha_{k+1}}\metaA_{k+1}\eif \metaA_{k+1}\unisub{n_{k+1}}{\alpha_{k+1}})$  where $n_{k+1}\in\N$ is the first constant not in $\metaA_{k+1}$ or $\theta_j$ for any $j\leq k$.
\end{enumerate}
% TODO: show that there is a denumerable number of wffs of $\FI$
% TODO: recall previous result here
Given the infinite supply of new constants $\N$, we may always find an unused constant at each stage $k$ in the process of constructing $\theta_{k+1}$.
We may then extend $\MetaG$ to the saturated set $\MetaS_\MetaG$:
  % $$ \MetaS_\MetaG=\MetaG\cup\set{\theta_i:i\in\N}. $$
\begin{align*}
  \label{name}
  \MetaS_0     &= \MetaG\\
  \MetaS_{n+1} &= \MetaS_n\cup\set{\theta_n}\\
  \MetaS_\MetaG &= \bigcup_{i\in\N}\MetaS_n.
\end{align*}
Equivalently, $\MetaS_\MetaG=\MetaG\cup\set{\theta_i:i\in\N}$.
The reason we did not define $\MetaS_\MetaG$ in this way is to ease the exposition of the proof that $\MetaS_\MetaG$ is consistent which goes by induction on the stages of the construction of $\MetaS_\MetaG$. 
% Not only is $\MetaS_\MetaG$ a saturated superset of $\MetaG$, we may also show that $\MetaS_\MetaG$ is consistent. 
Accordingly, you can think of the recursive definition of $\MetaS_\MetaG$ as little more than a convenient notation reminiscent of the constructions used Chapter \ref{ch.PL-completeness}.

A similar sort of recursive construction will come up again when we introduce a maximal consistent extension of $\MetaS_\MetaG$.
The reason these constructions are used both here and while proving \textsc{PL Completeness} before is that in each case they help us to establish the consistency of a set of wfss.
Whereas in Chapter \ref{ch.PL-completeness} this happened just once, now we will have two stages of consistent extension.
Since we will continue to speak of consistency throughout may of the results that follow, it will become cumbersome to specify that we have consistency in FOL$^=_\N$ each time. 
Accordingly, we often speak of sets of wfss as being consistent full stop, where it is to be understood that we have FOL$^=_\N$ in mind. 

Before showing that $\MetaS_\MetaG$ is consistent, we will begin by proving the following lemma.


% \begin{Lthm} \label{lemma:prunigen}
%   $\MetaG\vdash \qt{\forall}{\alpha}\metaA$ if $\MetaG\vdash\metaA\unisub{\beta}{\alpha}$ where $\beta$ is a constant not occurring in $\MetaG$ or $\metaA$.
% \end{Lthm}
%
% \begin{quote} 
%   \textit{Proof:} Assume $\MetaG\vdash\metaA\unisub{\beta}{\alpha}$ where $\beta$ is a constant not occurring in $\MetaG$ or $\metaA$. 
% \end{quote}
%
%
%

\begin{Lthm} \label{lemma:FOL-incon}
  If $\Lambda\cup\set{\metaA}$ is inconsistent, then $\Lambda\vdash\enot\metaA$. 
\end{Lthm}

\begin{quote} 
  \textit{Proof:}
  Identical to \textbf{\ref{lemma:PL-incon}}.
  % Assume $\Lambda\cup\set{\metaA}$ is inconsistent.
  % Thus $\Lambda\cup\set{\metaA}\vdash A\eand\enot A$, and so there is some proof $X$ of $A\eand\enot A$ from $\Lambda\cup\set{\metaA}$. 
  % Let $X'$ be the result of replacing the premise $\metaA$ with the assumption of $\metaA$, appending lines for $A$ and $\enot A$ by conjunction elimination $\eand$E. 
  % We may then apply negation introduction $\enot$I in order to discharge the assumption of $\metaA$.
  % The result is a proof of $\enot\metaA$ from $\Lambda$, and so $\Lambda\vdash\enot\metaA$. 
  \qed
\end{quote}



It is worth reviewing the proof of \textbf{\ref{lemma:PL-incon}} if you cannot remember the details.
As with so many proofs, a small technique can make the difference between being able to explain why something is true even if you have convinced yourself of its truth.



% TODO: move up CH3 and then repeat above
\begin{Lthm} \label{lemma:FOL-prcut}
  If $\Lambda \vdash \metaA$ and $\Pi\cup\set{\metaA} \vdash \metaB$, then $\Lambda\cup\Pi \vdash \metaB$. 
\end{Lthm}

\begin{quote} 
  \textit{Proof:} Assume that $\Lambda \vdash \metaA$ and $\Pi\cup{\metaA} \vdash \metaB$.
  It follows that there is a proof of $\metaA$ from $\Lambda$ as well as a proof $Y$ of $\metaB$ from $\Pi$. 
  Let $Z$ be the result of replacing the line in which $\metaA$ occurs as a premise with $X$.
  Since $Z$ proves $\metaB$ from the premises $\Lambda$ in $X$ together with the premises $\Pi\cup\set{\metaA}$ in $Y$ with the exception of $\metaA$, we may conclude that $Z$ proves $\metaA$ from $\Lambda\cup\Pi$, and so $\Lambda\cup\Pi\vdash\metaB$.
  \qed
\end{quote}



% % TODO: SAVE
% In the proof given above we made explicit appeal to \textbf{\ref{lemma:FOL-prcut}}.
% This rule is referred to as a \define{structural metarules} given the fundamental role that it plays in manipulating and multiplying derivations.
% We will see one more structural rule in due course.
% Since such metarules are both intuitive and commonplace, it is typical to suppress explicit reference to structural rules like \textbf{\ref{lemma:FOL-prcut}}.
% We will attempt to make all applications of the metarules explicit here, but this is something to look out for in logic more generally.






\begin{Lthm} \label{lemma:FOL-cont}
  If $\Lambda\vdash\metaA$ and $\Lambda\vdash\enot\metaA$, then $\Lambda$ is inconsistent. 
\end{Lthm}

\begin{quote} 
  \textit{Proof:} 
  Identical to \textbf{\ref{lemma:PL-cont}}.
  \qed
  % Assume $\Lambda\vdash\metaA$ and $\Lambda\vdash\enot\metaA$.
  % Thus there is a FOL$^=$ proof $X$ of $\metaA$ from $\Lambda$ as well as a FOL$^=$ proof $Y$ of $\enot\metaA$ from $\Lambda$. 
  % Letting $Z$ be the result of concatenating $X$ and $Y$ and using EFQ from $\S\ref{EFQ}$ to derive $A\eand\enot A$, we may observe that $Z$ is a proof of $A\eand\enot A$ from $\Lambda$, and so $\Lambda$ is inconsistent.
\end{quote}
  





\begin{Lthm} \label{lemma:FOL-sat}
  If $\MetaG$ is consistent, then $\MetaS_\MetaG$ is consistent and saturated in $\FIN$. 
\end{Lthm}

\begin{quote} 
  \textit{Proof:} 
  Recall the enumeration of wffs of $\FIN$ with one free variable from before. 
  Letting $\metaA(\alpha)$ be any wff of $\FIN$ with one free variable, $\metaA(\alpha)=\metaA_i(\alpha_i)$ for some $i\in\N$.
  By construction $\theta_i\in\MetaS_\MetaG$ where $\theta_i=(\qt{\exists}{\alpha_i}\metaA_i \eif \metaA_i\unisub{n_i}{\alpha_i})$.
  Thus there is some constant $n_i$ where $(\qt{\exists}{\alpha}\metaA \eif \metaA\unisub{n_i}{\alpha})\in\MetaS_\MetaG$, and so $\MetaS_\MetaG$ is saturated. 

  The proof that $\MetaS_\MetaG$ is consistent goes by induction on its construction where the consistency of $\MetaS_0$ follows from its definition given the starting assumption.
  Assume $\MetaS_m$ is consistent.
  To show that $\MetaS_{m+1}$ is consistent, assume for contradiction that $\MetaS_{m+1}$ is not consistent.
  Since $\MetaS_{m+1}=\MetaS_m\cup\set{\theta_{m+1}}$, we know that $\MetaS_m\vdash \enot\theta_{m+1}$ by \textbf{\ref{lemma:FOL-incon}}, and so $\MetaS_m\vdash\enot(\qt{\exists}{\alpha_{m+1}}\metaA_{m+1}\eif \metaA_{m+1}\unisub{n_{m+1}}{\alpha_{m+1}})$.

  Given that the derived rules for PL are also derived rules in FOL$^=$, it follows by \textbf{\ref{lemma:FOL-prcut}} that $\MetaS_m\vdash \qt{\exists}{\alpha_{m+1}}\metaA_{m+1}$ and $\MetaS_m\vdash\enot\metaA_{m+1}\unisub{n_{m+1}}{\alpha_{m+1}}$.
  In particular, we may derive the rules $\enot(\metaA \eif \metaB) \vdash \metaA$ and $\enot(\metaA \eif \metaB) \vdash \enot\metaB$, where applications of these rules together with \textbf{\ref{lemma:FOL-prcut}} justifies the claims indicated in the previous sentence.

    % TODO: cite derived rule; add this rule to chapter 6
  Since $n_{m+1}$ does not occur in $\metaA_{m+1}$ or in $\MetaS_m$, it follows by universal introduction $\forall$I that $\MetaS_m\vdash \qt{\forall}{\alpha_{m+1}}\enot\metaA_{m+1}$.
  Given that $\qt{\forall}{\alpha_{m+1}}\enot\metaA_{m+1} \vdash \enot\qt{\exists}{\alpha_{m+1}}\metaA_{m+1}$ by the quantifier exchange rule $(\forall\enot)$ derived in $\S\ref{QER}$, we know that $\MetaS_m \vdash \enot\qt{\exists}{\alpha_{m+1}}\metaA_{m+1}$ by \textbf{\ref{lemma:FOL-prcut}}, and so $\MetaS_m$ is inconsistent by \textbf{\ref{lemma:FOL-cont}}, contradicting our assumption. 
  Thus we may conclude that $\MetaS_{m+1}$ is consistent, and so it follows by induction that $\MetaS_k$ is consistent for all $k\in\N$ as desired.

  Assume for contradiction that $\MetaS_\MetaG$ is inconsistent. 
  By definition $\MetaS_\MetaG\vdash A\wedge\neg A$, and so there is a proof $X$ of $A\wedge\neg A$ from the premises $\MetaS_\MetaG$.
  Since every proof is finite, at most finitely many premises in $\MetaS_\MetaG$ are cited in $X$, and so there is some $m\in\N$ where every premise cited in $X$ occurs in $\MetaS_m$.
  As a result, $\MetaS_m\vdash A\wedge\neg A$, and so $\MetaS_m$ is inconsistent, contradicting the above. 
  Thus $\MetaS_\MetaG$ is consistent. 
  \qed
\end{quote}





\subsection{Maximization}%
  \label{sub:Maximization}
  
A set of wfss $\MetaD$ is \define{maximal} in $\FIN$ just in case either $\metaB\in\MetaD$ or $\enot\metaB\in\MetaD$ for every sentence $\metaB$ of $\FIN$.
Having shown that $\MetaS_\MetaG$ is consistent if $\MetaG$ is consistent, we may now maximize $\MetaS_\MetaG$ by adding every sentence that we can consistently. 
Whereas before we enumerated all wffs which contain a single free variable, we will now enumerate all wfss $\metaB_0,\metaB_1,\metaB_2,\ldots$ of $\FIN$ whatsoever in order to present the following recursive construction:
\begin{align*}
  % \label{name}
  \MetaD_0     &= \MetaS \\
  \MetaD_{n+1} &= 
    \begin{cases}
      \MetaD_n\cup\set{\metaB_n} &\text{if } \MetaG_n\cup\set{\metaB_n} \text{ is consistent}\\
      \MetaD_n\cup\set{\enot\metaB_n} &\text{otherwise}.
    \end{cases}\\
  \MetaD_\MetaS &= \bigcup_{i\in\N}\MetaD_n. 
\end{align*}
If $\MetaS_\MetaG$ is consistent, we may show that $\MetaD_{\MetaS_\MetaG}$ is both consistent and maximal where it follows as a result that $\MetaD_{\MetaS_\MetaG}$ is deductively closed.
Moreover, we may show $\MetaG\subseteq\MetaS_\MetaG\subseteq\MetaD_{\MetaS_\MetaG}$ where $\MetaD_{\MetaS_\MetaG}$ is saturated on account of including $\MetaS_\MetaG$.
These properties will form the basis upon which the Henkin model is constructed in section $\S\ref{sub:HenkinModel}$ and then shown to satisfy $\MetaG$.
In order to establish these results, we will begin by relabeling the following supporting lemma.




\begin{Lthm} \label{lemma:FOL-conin}
  If $\Lambda \cup \set{\metaA}$ and $\Lambda\cup \set{\enot\metaA}$ are both inconsistent, then $\Lambda$ is inconsistent. 
\end{Lthm}

\begin{quote} 
  \textit{Proof:}
  Identical to \textbf{\ref{lemma:PL-prweak}}.
  % Assume that $\Lambda \cup \set{\metaA}$ and $\Lambda\cup \set{\enot\metaA}$ are both inconsistent.
  % It follows that $\Lambda\vdash \enot\metaA$ and $\Lambda\vdash\enot\enot\metaA$ by \textbf{\ref{lemma:FOL-incon}}, and so there is some proof $X$ of $\enot\metaA$ from $\Lambda$, and some proof $Y$ of $\enot\enot\metaA$ from $\Lambda$. 
  % Let $Z$ be the result of concatenating $X$ and $Y$ and using EFQ from $\S\ref{EFQ}$ on the last lines of $X$ and $Y$ to derive $A\eand\enot A$. 
  % Since the only premises in $Z$ are the premises in $X$ and $Y$, we may conclude that $\Lambda\vdash A\eand\enot A$, and so $\Lambda$ is inconsistent. 
\end{quote}





\begin{Lthm} \label{lemma:FOL-max}
  If $\MetaG$ is consistent, then $\MetaD_{\MetaS_\MetaG}$ is maximal in $\FIN$ and consistent. 
\end{Lthm}

\begin{quote} 
  \textit{Proof:} 
  Let $\metaA$ be any wfs of $\FIN$.
  Thus $\metaA=\metaB_i$ for some $i\in\N$ given the enumeration above where either $\metaB_i\in\MetaD_{i+1}$ or $\enot\metaB_i\in\MetaD_{i+1}$.
  Since $\MetaD_{i+1}\subseteq\MetaD_{\MetaS_\MetaG}$, either $\metaA\in\MetaD_{\MetaS_\MetaG}$ or $\enot\metaA\in\MetaD_{\MetaS_\MetaG}$, and so $\MetaD_{\MetaS_\MetaG}$ is maximal in $\FIN$.

  The proof that $\MetaD_{\MetaS_\MetaG}$ is consistent goes by induction on the construction of $\MetaD_{\MetaS_\MetaG}$, where we know by \textbf{\ref{lemma:FOL-sat}} that $\MetaS_\MetaG=\MetaD_0$ is consistent. 
  Assume for weak induction that $\MetaD_n$ is consistent. 
  There are two cases to consider.

  \textit{Case 1:} $\MetaD_n\cup\set{\metaB_n}$ is consistent, and so $\MetaD_{n+1}=\MetaD_n\cup\set{\metaB_n}$ is consistent. 

  \textit{Case 2:} $\MetaD_n\cup\set{\metaB_n}$ is not consistent, and so $\MetaD_{n+1}=\MetaD_n\cup\set{\enot\metaB_n}$. 
  Assume for contradiction that $\MetaD_n\cup\set{\enot\metaB_n}$ is not consistent. 
  By \textbf{\ref{lemma:FOL-conin}}, $\MetaD_n$ is inconsistent, contradicting the hypothesis. 
  % It follows that $\MetaD_n\vdash\enot\enot\metaB_n$ by \textbf{\ref{lemma:FOL-incon}}, and so $\MetaD_n\vdash\metaB_n$ by DN.
  % Since $\MetaD_n\cup\set{\metaB_n}$ is also not consistent, $\MetaD_n\cup\set{\metaB_n} \vdash A\eand\enot A$.
  % By \textbf{\ref{lemma:FOL-prcut}}, $\MetaD_n \vdash A\eand\enot A$, and so $\MetaD_n$ is not consistent, contradicting the hypothesis. 
  Thus $\MetaG_{n+1}$ is consistent. 

  Since $\MetaG_{n+1}$ is consistent, it follows by induction that $\MetaG_n$ is consistent for all $n\in\N$.
  Assume for contradiction that $\MetaD_{\MetaS_\MetaG}$ is inconsistent.
  Thus $\MetaD_{\MetaS_\MetaG}\vdash \bot$, and so there is a derivation $Y$ of $\bot$ from $\MetaD_{\MetaS_\MetaG}$ in FOL$^=_\N$. 
  Since $Y$ is finite, there is a finite number of premises cited in $Y$, and so there is some $k\in\N$ where every premise cited in $Y$ belongs to $\MetaD_k$.
  As a result, $Y$ is also a proof of $\bot$ from $\MetaD_k$, and so $\MetaD_k$ is inconsistent, contradicting the above. 
  Thus $\MetaD_{\MetaS_\MetaG}$ is consistent. 
  \qed
\end{quote}




\begin{Lthm} \label{lemma:FOL-include}
  $\MetaG\subseteq\MetaS_\MetaG\subseteq\MetaD_{\MetaS_\MetaG}$ where $\MetaD_{\MetaS_\MetaG}$ is saturated.
\end{Lthm}
 
\begin{quote} 
  \textit{Proof:} 
  By definition, $\MetaG=\MetaS_0$ where $\MetaS_0\subseteq\MetaS_\MetaG$, and $\MetaS_\MetaG=\MetaD_0$ where $\MetaD_0\subseteq\MetaD_{\MetaS_\MetaG}$.
  Thus $\MetaG\subseteq\MetaD_{\MetaS_\MetaG}$.
  Moreover, $\MetaD_{\MetaS_\MetaG}$ is saturated since otherwise there would be some wff $\metaA(\alpha)$ of $\FIN$ with one free variable but no constant $n$ where $(\qt{\exists}{\alpha}\metaA \eif \metaA\unisub{n}{\alpha})\in\MetaD_{\MetaS_\MetaG}$. 
  Since $\MetaS_\MetaG\subseteq\MetaD_{\MetaS_\MetaG}$, there would be no constant $n$ where $(\qt{\exists}{\alpha}\metaA \eif \metaA\unisub{n}{\alpha})\in\MetaS_\MetaG$, and so $\MetaS_\MetaG$ would not be saturated, contradicting \textbf{\ref{lemma:FOL-sat}}.
  % Thus $\MetaD_{\MetaS_\MetaG}$ is saturated. 
  \qed
\end{quote}

As brought out in the proof of \textsc{PL Completeness} before, maximal consistent sets of wfss have some nice properties where principle among these is deductive closure.



\subsection{Deductive Closure}%
  \label{sub:DeductiveClosure}

Maximal consistent sets of wfss are deductively closed insofar as they contain every wfs derivable from that set as a member.
Put formally, $\MetaD$ is \define{deductively closed} just in case $\metaA\in\MetaD$ whenever $\MetaD\vdash\metaA$.
Since $\MetaD\vdash\metaA$ whenever $\metaA\in\MetaD$, deductively closed sets of wfss are identical to the set of wfss which they derive.
In order to show that $\MetaD_{\MetaS_\MetaG}$ is deductively closed, we will begin by relabeling the following general purpose lemma from before. 


\begin{Lthm} \label{lemma:FOL-deductive}
  If $\MetaD$ is maximal in $\FIN$ and consistent, then $\metaA\in\MetaD$ whenever $\MetaD\vdash\metaA$.
\end{Lthm}

\begin{quote} 
  \textit{Proof:} Assume $\MetaD$ is maximal in $\FIN$ and consistent where $\MetaD\vdash\metaA$.
  If $\MetaD\vdash\enot\metaA$, then $\MetaD$ is inconsistent by \textbf{\ref{lemma:FOL-cont}}, contradicting the assumption.
  Thus $\MetaD\nproves\enot\metaA$, and so $\enot\metaA\notin\MetaD$ since otherwise $\MetaD\vdash\enot\metaA$. 
  Since $\MetaD$ is maximal, $\metaA\in\MetaD$. 
  \qed
\end{quote}



This completes the setup for the construction of the Henkin Model presented in the following section.
Whereas saturation was required to make sure every existential claim has a witness, maximal consistency is familiar from Chapter \ref{ch.PL-completeness}.

Although certain similarities will persist, the Henkin model that we will define over $\MetaD_{\MetaS_\MetaG}$ will differ considerably from the Henkin interpretation that we introduced before. 
In particular, constructing the domain will require some care so as to ensure that any two constants that name the same individual are assigned to the same element of the domain of our Henkin model.
None of these considerations occurred before, and indeed would be considerably simpler in a language without identity.
These are the costs incurred by expanding the expressive power of our language to include not just predicates and the quantifiers but also a designated identity predicate for which we provided a semantics.





\section{Henkin Model}%
  \label{sub:HenkinModel}

Having extended the consistent set of sentences $\MetaG$ in $\FI$ to a saturated maximal consistent set of sentences $\MetaD_{\MetaS_\MetaG}$ in $\FIN$ which was shown to be deductively closed, we may proceed to use $\MetaD_{\MetaS_\MetaG}$ to construct a Henkin model that satisfies $\MetaD_{\MetaS_\MetaG}$, and so satisfies $\MetaG$ as a result.
For ease of exposition, we will often drop the subscripts, assuming $\MetaD=\MetaD_{\MetaS_\MetaG}$ throughout.

We begin by letting $\C$ be the set of all constants in $\FIN$ where this includes all the typical constants included in $\FI$ together with the natural numbers $\N$ which we added. 
Since more than one constant can refer to the same element in a domain, we will model the elements of the domain as equivalence classes of co-referring constants.
Consider the following:
  \begin{enumerate}[leftmargin=1.5in]
    \item[\it Element:] $[\alpha]_\MetaD=\set{\beta\in\C : \alpha=\beta\in\MetaD}$.
    \item[\it Domain:] $\D_\MetaD=\set{[\alpha]_\MetaD \subseteq \C : \alpha\in\C}$.
  \end{enumerate}
The equivalence class $[\alpha]_\MetaD$ is the set of constants in $\C$ which includes $\beta$ just in case $\alpha=\beta$ belongs to $\MetaD$.
In order to show that $[\alpha]_\MetaD\neq\varnothing$ for any constant $\alpha\in\C$, we begin by relabeling the weakening principle from before which will be of general utility throughout the proof:





% TODO: move weakening to FOL$^=$ chapter?
\begin{Lthm} \label{lemma:FOL-prweak}
  If $\Lambda\vdash\metaA$, then $\Lambda\cup\Pi\vdash\metaA$.
\end{Lthm}

\begin{quote} 
  \textit{Proof:} 
  Identical to \textbf{\ref{lemma:PL-prweak}}.
  % Assuming that $\Lambda\vdash\metaA$, there is a proof $X$ of $\metaA$ from $\Lambda$ in FOL$^=$.
  % Since $\Lambda\subseteq\Lambda\cup\Pi$, it follows that $X$ is a proof of $\metaA$ from $\Lambda\cup\Pi$ in FOL$^=$, and so $\Lambda\cup\Pi\vdash\metaA$. 
  \qed
\end{quote}



% It follows that every element $[\alpha]_\MetaD$ of the domain is nonempty.
% Since $\MetaD$ is maximal, all such identity sentences or their negations will belong to $\MetaD$ where we are gathering together all of the co-referring constants into sets which are taking to be the elements of the domain $\D_\MetaD$. 
Every constant $\alpha\in\C$ generates an element $[\alpha]_\MetaD\in\D_\MetaD$ which includes $\alpha$ as a member: 


\begin{Lthm} \label{lemma:FOL-nonempty}
  $\alpha\in[\alpha]_\MetaD$ for any constant $\alpha\in\C$.
\end{Lthm}

\begin{quote} 
  \textit{Proof:} 
  Let $\alpha\in\C$ be an arbitrary constant. 
  Since $\vdash \alpha=\alpha$ by identity introduction $=$I, it follows that $\MetaD\vdash\alpha=\alpha$ by \textbf{\ref{lemma:FOL-prweak}}, and so $\alpha=\alpha\in\MetaD$ by \textbf{\ref{lemma:FOL-deductive}}.
  Thus $\alpha\in[\alpha]_\MetaD$ for any constant $\alpha\in\C$.
  \qed
\end{quote}

Next we may show that the elements of $\D_\MetaD$ are well defined with the following: 

\begin{Lthm} \label{lemma:FOL-define}
  If $\alpha=\beta\in\MetaD$, then $[\alpha]_\MetaD=[\beta]_\MetaD$.
\end{Lthm}

\begin{quote} 
  \textit{Proof:}
  Assume $\alpha=\beta\in\MetaD$.
  Letting $\gamma\in[\alpha]_\MetaD$, it follows that $\alpha=\gamma\in\MetaD$.
  Since $\alpha=\beta,\alpha=\gamma\vdash\beta=\gamma$ by identity elimination $=$E, we know that $\MetaD\vdash\beta=\gamma$ by \textbf{\ref{lemma:FOL-prweak}}, and so $\beta=\gamma\in\MetaD$ by \textbf{\ref{lemma:FOL-deductive}}. 
  Thus $\gamma\in[\beta]_\MetaD$, and so generalising on $\gamma$, it follows that $[\alpha]_\MetaD\subseteq[\beta]_\MetaD$.
  By parity of reasoning, we may conclude that $[\beta]_\MetaD\subseteq[\alpha]_\MetaD$, and so $[\alpha]_\MetaD=[\beta]_\MetaD$ as desired.
  \qed
\end{quote}

This shows that it does not matter which element $\alpha\in[\alpha]_\MetaD$ we choose to represent the element $[\alpha]_\MetaD$.
For instance, if $\beta\in[\alpha]_\MetaD$, then $\alpha=\beta\in\MetaD$ and so we could have written `$[\beta]_\MetaD$' in place of `$[\alpha]_\MetaD$' since $[\alpha]_\MetaD=[\beta]_\MetaD$.
As a result, the elements in $\D_\MetaD$ are well-defined. 

Having constructed the domain, we may proceed to specify an interpretation of the constants and predicates included in $\FIN$.
Rather than specifying any interpretation at all, we will make a number of especially convenient choices in order to guarantee that the resulting model satisfies all of the wfss in $\MetaD$.
In particular, consider the following definitions:
  \begin{enumerate}[leftmargin=1.5in]
    \item[\it Constants:] $\I_\MetaD(\alpha)=[\alpha]_\MetaD$ for all constants $\alpha\in\C$. 
    \item[\it Predicates:] $\I_\MetaD(\F^n)=\set{\tuple{[\alpha_1]_\MetaD,\ldots,[\alpha_n]_\MetaD}\in\D_\MetaD^n:\F^n\alpha_1,\ldots,\alpha_n\in\MetaD}$.
  \end{enumerate}
Whereas every constant $\alpha$ is assigned to the element $[\alpha]_\MetaD$ it generates, the extension of any $n$-place predicate $\F^n$ includes all and only the ordered tuples $\tuple{[\alpha_1]_\MetaD,\ldots,[\alpha_n]_\MetaD}$ for which $\F^n\alpha_1,\ldots,\alpha_n\in\MetaD$.
% Put otherwise, each constant $\alpha$ is assigned to the element $[\alpha]_\MetaD$ that it represents where the atomic sentences of the form $\F^n\alpha_1,\ldots,\alpha_n\in\MetaD$ specify which $n$-tuples of elements belong to the extension of $\F^n$.
Given that $[\alpha]_\MetaD=[\beta]_\MetaD$ may hold for distinct constants $\alpha$ and $\beta$, we must check that there is no ensuing conflict among the atomic sentences included in $\MetaD$.
Put otherwise, we must show that the extensions of predicates are well-defined as follows:





\begin{Lthm} \label{lemma:FOL-preddef}
  If $\alpha_i=\beta_i\in\MetaD$, then $\F^n\alpha_1,\ldots,\alpha_n\in\MetaD$ just in case $\F^n\alpha_1,\ldots,\alpha_n\unisub{\beta_i}{\alpha_i}\in\MetaD$.
\end{Lthm}

\begin{quote} 
  \textit{Proof:} 
  Assume that $\alpha_i=\beta_i\in\MetaD$ for some $\alpha_i,\beta_i\in\C$ where $\F^n\alpha_1,\ldots,\alpha_n\in\MetaD$.
  It follows that $\MetaD\vdash\F^n\alpha_1,\ldots,\alpha_n\unisub{\beta_i}{\alpha_i}$ by identity elimination $=$E, and so $\F^n\alpha_1,\ldots,\alpha_n\unisub{\beta_i}{\alpha_i}\in\MetaD$ by \textbf{\ref{lemma:FOL-deductive}}.
  By parity of reasoning, we may conclude that $\F^n\alpha_1,\ldots,\alpha_n\in\MetaD$ just in case $\F^n\alpha_1,\ldots,\alpha_n\unisub{\beta_i}{\alpha_i}\in\MetaD$.
  \qed
\end{quote}


Focusing on the first index, suppose that $\alpha_1=\beta_1\in\MetaD$ where $\F^n\alpha_1,\ldots,\alpha_n\in\MetaD$. 
We know by \textit{Predicates} that $\tuple{[\alpha_1]_\MetaD,\ldots,[\alpha_n]_\MetaD}\in\I_\MetaD(\F^n)$, and so $\tuple{[\beta_1]_\MetaD,\ldots,[\alpha_n]_\MetaD}\in\I_\MetaD(\F^n)$ since $[\alpha_1]_\MetaD=[\beta_1]_\MetaD$ by \textbf{\ref{lemma:FOL-define}}.
Thus $\F^n\beta_1,\ldots,\alpha_n\in\MetaD$ again by \textit{Predicates}. 
More generally, \textbf{\ref{lemma:FOL-preddef}} shows that the same considerations apply no matte the index. 
It follows that the extension $\I_\MetaD(\F^n)$ for any predicate $\F^n$ is well-defined since their is no possibility of disagreement about whether $\tuple{[\alpha_1]_\MetaD,\ldots,[\alpha_n]_\MetaD}\in\I_\MetaD(\F^n)$ by merely changing the representative $\alpha_i$ for the element $[\alpha_i]_\MetaD$ to $\beta_i$ whenever $\alpha_i = \beta_i$.

Observe that $\M_\MetaD=\tuple{\D_\MetaD,\I_\MetaD}$ satisfies the definition of a $\FIN$ model.
Since this construction is due to Leon Henkin (1949), we will refer to $\M_\MetaD$ as the \define{henkin model} for $\MetaG$, recalling that $\MetaD=\MetaD_{\MetaS_\MetaG}$.
It remains to show that $\M_\MetaD$ satisfies $\MetaD$, and so satisfies $\MetaG$ as a result.

Much will turn on the details that we have presented so far. %, and so it is worth gathering a few more intuitions before pressing on.
In order to gather a few more intuitions for how $\MetaD$ determines what $\D_\MetaD$ includes before pressing on, let $\C^3 = \set{a, b, c}$ where $\MetaG_3 = \set{a \neq b, a \neq c, a \neq b}$ and $\MetaD^3$ is defined from $\MetaG_3$ in a similar manner as above.
It is easy to show that $\D_{\Delta^3} = \set{\set{a},\set{b},\set{c}}$.
Were we to let $\MetaG_2 = \set{a = b, a \neq c, a \neq b}$ where $\MetaD^2$ is defined from $\MetaG_2$, then $\D_{\Delta^2} = \set{\set{a, b},\set{c}}$ since $[a]_{\MetaD^2} = [b]_{\MetaD^2} = \set{a, b}$.
More generally, the more identities included in $\MetaD^2$, the fewer elements in the domain for the constants to name.






\section{Satisfiability}%
  \label{sec:Satisfiability}

We turn now to present some of the more substantial lemmas upon which the completeness of FOL$^=$ will ultimately depend. 
To being with, consider the following quantifier lemmas.

\begin{Lthm} \label{lemma:FOL-exists}
  $\VV{\I_\MetaD}{\va{a}}(\qt{\exists}{\alpha}\metaA)=1$ just in case $\VV{\I_\MetaD}{\va{a}}(\metaA\unisub{\beta}{\alpha})=1$ for some constant $\beta\in\C$.
  % $\M_\MetaD$ satisfies $\qt{\exists}{\alpha}\metaA$ just in case $\M_\MetaD$ satisfies $\metaA\unisub{\beta}{\alpha}$ for some constant $\beta$.
\end{Lthm}


\begin{quote} 
  \textit{Proof:}
  Let $\va{a}$ be a variable assignment defined over $\D_\MetaD$ where $\VV{\I_\MetaD}{\va{a}}(\qt{\exists}{\alpha}\metaA)=1$.
  It follows that $\VV{\I_\MetaD}{\va{c}}(\metaA)=1$ for some $\alpha$-variant $\va{c}$ of $\va{a}$ by the semantics for the existential quantifier.
  Given that $\va{c}(\alpha)\in\D_\MetaD$, we know that $\va{c}(\alpha)=[\beta]_\MetaD$ for some constant $\beta\in\C$.
  Moreover, we know that $\I_\MetaD(\beta)=[\beta]_\MetaD$ and so $\va{c}(\alpha)=\I_\MetaD(\beta)$.
  Thus $\val{\I}{\va{c}}(\alpha)=\val{\I}{\va{c}}(\beta)$, and so $\VV{\I_\MetaD}{\va{c}}(\metaA)=\VV{\I_\MetaD}{\va{c}}(\metaA\unisub{\beta}{\alpha})$ by \textbf{\ref{lemma:sub}}.

  Since $\alpha$ does not occur in $\metaA\unisub{\beta}{\alpha}$ and $\va{c}$ is a $\alpha$-variant of $\va{a}$, we know $\va{c}(\gamma) = \va{a}(\gamma)$ for all variables $\gamma$ that occur in $\metaA\unisub{\beta}{\alpha}$.
  Thus $\VV{\I_\MetaD}{\va{c}}(\metaA\unisub{\beta}{\alpha})=\VV{\I_\MetaD}{\va{a}}(\metaA\unisub{\beta}{\alpha})$ by \textbf{\ref{lemma:VarAgree}}, and so $\VV{\I_\MetaD}{\va{a}}(\metaA\unisub{\beta}{\alpha}) = 1$ for some $\beta\in\C$ given the identities above.

  Assume instead that $\VV{\I_\MetaD}{\va{a}}(\metaA\unisub{\beta}{\alpha})=1$ for some constant $\beta\in\C$.
  Letting $\va{c}$ be the $\alpha$-variant of $\va{a}$ where $\va{c}(\alpha)=\I_\MetaD(\beta)$, it follows that $\val{\I}{\va{c}}(\alpha)=\val{\I}{\va{c}}(\beta)$, and so $\VV{\I_\MetaD}{\va{c}}(\metaA)=\VV{\I_\MetaD}{\va{c}}(\metaA\unisub{\beta}{\alpha})$ by \textbf{\ref{lemma:sub}}.
  Thus $\VV{\I_\MetaD}{\va{c}}(\metaA)=1$, and so $\VV{\I_\MetaD}{\va{a}}(\qt{\exists}{\alpha}\metaA)=1$ follows by the semantics for the existential quantifier. 
  \qed
\end{quote}






\begin{Lthm} \label{lemma:FOL-forall}
  $\VV{\I_\MetaD}{\va{a}}(\qt{\forall}{\alpha}\metaA)=1$ just in case $\VV{\I_\MetaD}{\va{a}}(\metaA\unisub{\beta}{\alpha})=1$ for all constants $\beta\in\C$.
\end{Lthm}

\begin{quote} 
  \textit{Proof:}
  Let $\va{a}$ be defined over $\D_\MetaD$ where $\VV{\I_\MetaD}{\va{a}}(\qt{\forall}{\alpha}\metaA)=1$.
  It follows that $\VV{\I_\MetaD}{\va{c}}(\metaA)=1$ for every $\alpha$-variant $\va{c}$ of $\va{a}$ by the semantics for the universal quantifier.
  Let $\beta\in\C$ be any constant. 
  Since $[\beta]_\MetaD\in\D_\MetaD$, we may let $\va{c}$ be the $\alpha$-variant of $\va{a}$ where $\va{c}(\alpha)=[\beta]_\MetaD$.
  Given that $\I_\MetaD(\beta)=[\beta]_\MetaD$, it follows that $\va{c}(\alpha)=\I_\MetaD(\beta)$.
  As a result, $\val{\I}{\va{c}}(\alpha)=\val{\I}{\va{c}}(\beta)$, and so $\VV{\I_\MetaD}{\va{c}}(\metaA)=\VV{\I_\MetaD}{\va{c}}(\metaA\unisub{\beta}{\alpha})$ by \textbf{\ref{lemma:sub}}.

  Since $\alpha$ does not occur in $\metaA\unisub{\beta}{\alpha}$ and $\va{c}$ is a $\alpha$-variant of $\va{a}$, we know $\va{c}(\gamma) = \va{a}(\gamma)$ for all variables $\gamma$ that occur in $\metaA\unisub{\beta}{\alpha}$.
  Thus $\VV{\I_\MetaD}{\va{c}}(\metaA\unisub{\beta}{\alpha})=\VV{\I_\MetaD}{\va{a}}(\metaA\unisub{\beta}{\alpha})$ by \textbf{\ref{lemma:VarAgree}}, and so $\VV{\I_\MetaD}{\va{a}}(\metaA\unisub{\beta}{\alpha}) = 1$ by the identities above.
  Generalizing on $\beta\in\C$, it follows that $\VV{\I_\MetaD}{\va{a}}(\metaA\unisub{\beta}{\alpha})=1$ for all constants $\beta\in\C$.

  Assume instead that $\VV{\I_\MetaD}{\va{a}}(\metaA\unisub{\beta}{\alpha})=1$ for every constant $\beta\in\C$.
  Letting $\va{c}$ be any $\alpha$-variant of $\va{a}$, it follows that $\va{c}(\alpha)\in\D_\MetaD$, and so $\va{c}(\alpha)=[\beta]_\MetaD$ for some constant $\beta\in\C$.
  Thus $\val{\I}{\va{c}}(\alpha)=\val{\I}{\va{c}}(\beta)$, and so $\VV{\I_\MetaD}{\va{c}}(\metaA)=\VV{\I_\MetaD}{\va{c}}(\metaA\unisub{\beta}{\alpha})$ by \textbf{\ref{lemma:sub}}.
  Since $\alpha$ does not occur in $\metaA\unisub{\beta}{\alpha}$ and $\va{c}$ is a $\alpha$-variant of $\va{a}$, we know $\va{c}(\gamma) = \va{a}(\gamma)$ for all variables $\gamma$ that occur in $\metaA\unisub{\beta}{\alpha}$.
  Thus $\VV{\I_\MetaD}{\va{c}}(\metaA\unisub{\beta}{\alpha})=\VV{\I_\MetaD}{\va{a}}(\metaA\unisub{\beta}{\alpha})$ by \textbf{\ref{lemma:VarAgree}}.
  By assumption, $\VV{\I_\MetaD}{\va{a}}(\metaA\unisub{\beta}{\alpha}) = 1$, and so $\VV{\I_\MetaD}{\va{c}}(\metaA) = 1$ follows from the identities established above.
  Generalizing on $\va{c}$, we may conclude that $\VV{\I_\MetaD}{\va{a}}(\qt{\forall}{\alpha}\metaA)=1$ by the semantics for the universal quantifier. 
  \qed
\end{quote}




Although the lemmas above are of limited significance on their own, they play a critical role in the proof of the following lemma which establishes that the Henkin Model $\M_\MetaD$ has the desired property of satisfying $\MetaD$. 
Instead of proving this claim directly, it will be easier to establish a slightly stronger claim that $\MetaD$ satisfies all and only the wfss in $\MetaD$. 
That $\MetaD$ and so $\MetaG$ are satisfiable will then follow as an immediate result.

Whereas the lemmas so far have been relatively easy to establish, all the pieces that we have developed will come together in the following lemma.
Since the lemma goes by induction on complexity, there are a number of cases to check, resulting in a much longer proof.
As always, take care to refer back to the beginning of the proof if you get lost and need to regain your bearings.
Given the importance of this lemma, all but one case has been included in full.

\begin{Lthm} \label{lemma:FOL-truth}
  If $\MetaD$ is a saturated maximal consistent set of $\FIN$ sentences and $\metaA$ is any $\FIN$ sentence, then $\M_\MetaD$ satisfies $\metaA$ just in case $\metaA\in\MetaD$.  
\end{Lthm}

\begin{quote} 
  \textit{Proof:} 
  Let $\MetaD$ be a saturated maximal consistent set of $\FIN$ sentences and $\metaA$ is any $\FIN$ sentence.
  Letting $\va{a}$ be an arbitrary variable assignment defined over $\D_\MetaD$, we will show that $\VV{\I_\MetaD}{\va{a}}(\metaA)=1$ just in case $\metaA\in\MetaD$ by induction on the complexity of the wfss of $\FIN$.
  There are two base cases and seven induction cases.

  \textit{Base:} 
  Assume $\comp(\metaA)=0$ and so either $\metaA$ is $\F^n\alpha_1,\ldots,\alpha_n$ or $\alpha_1=\alpha_n$ for some constants $\alpha_1,\ldots,\alpha_n\in\C$.
  % Letting $\va{a}$ be an arbitrary variable assignment defined over $\D_\MetaD$, 
  Consider the following biconditionals:

  \vspace{-.2in}
  \begin{align*}
    \VV{\I_\MetaD}{\va{a}}(\F^n\alpha_1,\ldots,\alpha_n)=1 % &\textit{ ~iff~ } \VV{\I_\MetaD}{\va{a}}(\F^n\alpha_1,\ldots,\alpha_n)=1 \text{ for some v.a. } \va{a}\\
      &\textit{ ~iff~ } \tuple{\val{\I_\MetaD}{\va{a}}(\alpha_1),\ldots,\val{\I_\MetaD}{\va{a}}(\alpha_n)}\in\I_\MetaD(\F^n)\\
      &\textit{ ~iff~ } \tuple{\I_\MetaD(\alpha_1),\ldots,\I_\MetaD(\alpha_n)}\in\I_\MetaD(\F^n)\\
      &\textit{ ~iff~ } \tuple{[\alpha_1]_\MetaD,\ldots,[\alpha_n]_\MetaD}\in\I_\MetaD(\F^n)\\
      &\textit{ ~iff~ } \F^n\alpha_1,\ldots,\alpha_n\in\MetaD.
  \end{align*}

  Whereas the final biconditional follows by the definition of $\I_\MetaD$, all of the other biconditionals are immediate from the definitions together with the assumptions.
  Something similar may be observed for identity sentences in $\FIN$:

  \vspace{-.2in}
  \begin{align*}
    \VV{\I_\MetaD}{\va{a}}(\alpha_1=\alpha_n)=1 %&\textit{ ~iff~ } \VV{\I_\MetaD}{\va{a}}(\alpha_1=\alpha_n)=1 \text{ for some v.a. } \va{a}\\
      &\textit{ ~iff~ } \val{\I_\MetaD}{\va{a}}(\alpha_1)=\val{\I_\MetaD}{\va{a}}(\alpha_n)\\
      &\textit{ ~iff~ } \I_\MetaD(\alpha_1)=\I_\MetaD(\alpha_n)\\
      &\textit{ ~iff~ } [\alpha_1]_\MetaD=[\alpha_n]_\MetaD\\
      (\ast) &\textit{ ~iff~ } \alpha_1=\alpha_n\in\MetaD.
  \end{align*}

  In support of the final biconditional, assume $[\alpha_1]_\MetaD=[\alpha_n]_\MetaD$.
  By \textbf{\ref{lemma:FOL-nonempty}}, we know that $\alpha_n\in[\alpha_n]_\MetaD$, and so $\alpha_n\in[\alpha_1]_\MetaD$.
  By definition, $\alpha_1=\alpha_n\in\MetaD$.
  Together with \textbf{\ref{lemma:FOL-define}}, we may conclude that $(\ast)$ holds where the other biconditionals follow from the definitions and the assumption that $\alpha_1,\alpha_n\in\C$.
  It follows that $\VV{\I_\MetaD}{\va{a}}(\metaA)=1$ just in case $\metaA\in\MetaD$ whenever $\comp(\metaA)=0$.

  \textit{Induction:}
  Assume for induction that $\VV{\I_\MetaD}{\va{a}}(\metaA)=1$ just in case $\metaA\in\MetaD$ whenever $\comp(\metaA)\leq n$. 
  Let $\metaA$ be a sentence of $\FIN$ where $\comp(\metaA)=n+1$.

  \textit{Case 1:}
  Assume $\metaA=\enot\metaB$.
  Since $\comp(\enot\metaB)=\comp(\metaB)+1$ and $\comp(\metaA)=n+1$, it follows that $\comp(\metaB)=n$.
  We may then reason as follows:

  \vspace{-.2in}
  \begin{align*}
    \VV{\I_\MetaD}{\va{a}}(\metaA)=1 &\textit{ ~iff~ } \VV{\I_\MetaD}{\va{a}}(\enot\metaB)=1\\
      &\textit{ ~iff~ } \VV{\I_\MetaD}{\va{a}}(\metaB)\neq 1 \\
      (\hspace{1.6pt}\star\hspace{1.6pt}) &\textit{ ~iff~ } \metaB\notin\MetaD \\
      (\neg) &\textit{ ~iff~ } \enot\metaB\in\MetaD \\
      &\textit{ ~iff~ } \metaA\in\MetaD.
  \end{align*}

  Assuming $\metaB \notin \MetaD$, it follows by the maximality assumed of $\MetaD$ that $\enot\metaB \in \MetaD$. 
  Conversely, if $\enot\metaB \in \MetaD$, then $\metaB \notin \MetaD$ since otherwise $\MetaD \vdash \bot$ by EFQ in $\S\ref{EFQ}$, making $\MetaD$ inconsistent contrary to assumption.
  This establishes $(\neg)$ where $(\hspace{1pt}\star\hspace{1pt})$ holds by hypothesis and the other biconditionals follow from the semantics for negation together and the case assumption.

  \textit{Case 2:}
  Assume $\metaA=\metaB\eand\metaC$.
  Since $\comp(\metaB\eand\metaC)=\comp(\metaB)+\comp(\metaC)+1$ and  $\comp(\metaA)=n+1$, it follows that $\comp(\metaB), \comp(\metaC)\leq n$.
  Thus we have:

  \vspace{-.2in}
  \begin{align*}
    \VV{\I_\MetaD}{\va{a}}(\metaA)=1 &\textit{ ~iff~ } \VV{\I_\MetaD}{\va{a}}(\metaB\eand\metaC)=1\\
      &\textit{ ~iff~ } \VV{\I_\MetaD}{\va{a}}(\metaB)=\VV{\I_\MetaD}{\va{a}}(\metaC)=1 \\
      (\hspace{1.6pt}\star\hspace{1.6pt}) &\textit{ ~iff~ } \metaB,\metaC\in\MetaD \\
      (\eand) &\textit{ ~iff~ } \metaB\eand\metaC\in\MetaD \\
      &\textit{ ~iff~ } \metaA\in\MetaD.
  \end{align*}

  Assuming that $\metaB,\metaC\in\MetaD$, we know that $\MetaD\vdash\metaB\eand\metaC$ by conjunction introduction $\eand$I, and so $\metaB\eand\metaC\in\MetaD$ by \textbf{\ref{lemma:FOL-deductive}}.
  Assuming instead that $\metaB\eand\metaC\in\MetaD$, it follows that $\MetaD\vdash\metaB$ and $\MetaD\vdash\metaC$ by conjunction elimination $\eand$E, and so $\metaB,\metaC\in\MetaD$ by \textbf{\ref{lemma:FOL-deductive}}.
  This establishes $(\eand)$.

  Additionally, $(\hspace{1.6pt}\star\hspace{1.6pt})$ holds by hypothesis, and the other biconditionals follow from the semantics for conjunction along with the case assumption.


  \textit{Case 3:}
  Assume $\metaA=\metaB\eor\metaC$.
  Since $\comp(\metaB\eor\metaC)=\comp(\metaB)+\comp(\metaC)+1$ and  $\comp(\metaA)=n+1$, it follows that $\comp(\metaB),\comp(\metaC)\leq n$.
  Thus we have:

  \vspace{-.2in}
  \begin{align*}
    \VV{\I_\MetaD}{\va{a}}(\metaA)=1 &\textit{ ~iff~ } \VV{\I_\MetaD}{\va{a}}(\metaB\eor\metaC)=1\\
      &\textit{ ~iff~ } \VV{\I_\MetaD}{\va{a}}(\metaB)=1 \text{ or } \VV{\I_\MetaD}{\va{a}}(\metaC)=1 \\
      (\hspace{1.6pt}\star\hspace{1.6pt}) &\textit{ ~iff~ } \metaB\in\MetaD \text{ or } \metaC\in\MetaD \\
      (\eor) &\textit{ ~iff~ } \metaB\eor\metaC\in\MetaD \\
      &\textit{ ~iff~ } \metaA\in\MetaD.
  \end{align*}

  Assuming that $\metaB\in\MetaD$, we know that $\MetaD\vdash\metaB\eor\metaC$ by disjunction introduction $\eor$I, and so $\metaB\eor\metaC\in\MetaD$ by \textbf{\ref{lemma:FOL-deductive}}.
  Analogous reasoning shows that $\metaB\eor\metaC\in\MetaD$ if $\metaC\in\MetaD$, and so $\metaB\eor\metaC\in\MetaD$ if either $\metaB\in\MetaD$ or $\metaC\in\MetaD$. 

  Assume instead that $\metaB\eor\metaC\in\MetaD$.
  If $\metaB\in\MetaD$, then either $\metaB\in\MetaD$ or $\metaC\in\MetaD$.
  If $\metaB\notin\MetaD$, then $\enot\metaB\in\MetaD$ by the maximality assumed of $\MetaD$, and so $\MetaD\vdash\metaC$ by DS from $\S\ref{DS}$.
  Thus $\metaC\in\MetaD$ by \textbf{\ref{lemma:FOL-deductive}}, and so either $\metaB\in\MetaD$ or $\metaC\in\MetaD$. 
  It follows that $\metaB\in\MetaD$ or $\metaC\in\MetaD$ if $\metaB\eor\metaC\in\MetaD$ which together with the above establishes $(\eor)$.  

  Additionally, $(\hspace{1.6pt}\star\hspace{1.6pt})$ holds by hypothesis, and the other biconditionals follow from the semantics for disjunction along with the case assumption.

  \textit{Case 4:}
  Assume $\metaA=\metaB\eif\metaC$.
  Since $\comp(\metaB\eif\metaC)=\comp(\metaB)+\comp(\metaC)+1$ and  $\comp(\metaA)=n+1$, it follows that $\comp(\metaB),\comp(\metaC)\leq n$.
  Thus we have:

  \vspace{-.2in}
  \begin{align*}
    \VV{\I_\MetaD}{\va{a}}(\metaA)=1 &\textit{ ~iff~ } \VV{\I_\MetaD}{\va{a}}(\metaB\eif\metaC)=1\\
      &\textit{ ~iff~ } \VV{\I_\MetaD}{\va{a}}(\metaB)\neq 1 \text{ or } \VV{\I_\MetaD}{\va{a}}(\metaC)=1 \\
      (\hspace{2.9pt}\star\hspace{2.9pt}) &\textit{ ~iff~ } \metaB\notin\MetaD \text{ or } \metaC\in\MetaD \\
      (\eif) &\textit{ ~iff~ } \metaB\eif\metaC\in\MetaD \\
      &\textit{ ~iff~ } \metaA\in\MetaD.
  \end{align*}

  Assuming that $\metaB\notin\MetaD$, we know that $\enot\metaB\in\MetaD$ by the maximality of $\MetaD$.
  Moreover, it is easy to derive $\enot\metaB\vdash\metaB\eif\metaC$ since given $\enot\metaB$ as a premise, we may use the assumption rule AS to write $\metaB$ on a second line, deriving $\metaC$ by EFQ from $\S\ref{EFQ}$ and using conditional introduction $\eif$I to discharge the assumption.
  It follows that $\MetaD\vdash\metaB\eif\metaC$ by \textbf{\ref{lemma:FOL-prweak}}, and so $\metaB\eif\metaC\in\MetaD$ by \textbf{\ref{lemma:FOL-deductive}}.

  Assuming instead that $\metaC\in\MetaD$, we may derive $\metaC\vdash\metaB\eif\metaC$ since given $\metaC$ as a premise, we may use the assumption rule AS to write $\metaB$ on a second line.
  By then using the reiteration rule R, we may rewrite the premise $\metaC$, discharging our assumption with conditional introduction $\eif$I in order to derive $\metaB\eif\metaC$ from $\metaC$. 
  Thus $\MetaD\vdash\metaB\eif\metaC$ by \textbf{\ref{lemma:FOL-prweak}}, and so $\metaB\eif\metaC\in\MetaD$ by \textbf{\ref{lemma:FOL-deductive}}.
  We may then conclude that $\metaB\eif\metaC\in\MetaD$ if either $\metaB\notin\MetaD$ or $\metaC\in\MetaD$.

  Assume instead that $\metaB\eif\metaC\in\MetaD$.
  If $\metaB\notin\MetaD$, then $\metaB\notin\MetaD$ or $\metaC\in\MetaD$.
  If $\metaB\in\MetaD$, then $\MetaD\vdash\metaC$ by conditional elimination $\eif$E, and so $\metaC\in\MetaD$ by \textbf{\ref{lemma:FOL-deductive}}.
  Thus $\metaB\notin\MetaD$ or $\metaC\in\MetaD$ if $\metaB\eif\metaC\in\MetaD$ which, given the above, establishes $(\eif)$.  

  Additionally, $(\hspace{1.6pt}\star\hspace{1.6pt})$ holds by hypothesis, and the other biconditionals follow from the semantics for the conditional along with the case assumption.

  \textit{Case 5:}
  Assume $\metaA=\metaB\eiff\metaC$. (Exercise for the reader.)

  \textit{Case 6:}
  Assume $\metaA=\qt{\exists}{\alpha}\metaB$.
  Since $\comp(\qt{\exists}{\alpha}\metaB)=\comp(\metaB)+1$ and $\comp(\metaA)=n+1$, it follows that $\comp(\metaB)=n$.
  We may then reason as follows:

  \vspace{-.2in}
  \begin{align*}
    \VV{\I_\MetaD}{\va{a}}(\metaA)=1 &\textit{ ~iff~ } \VV{\I_\MetaD}{\va{a}}(\qt{\exists}{\alpha}\metaB)=1\\
      % &\textit{ ~iff~ } \VV{\I_\MetaD}{\va{c}}(\metaB)=1 \text{ for some } \alpha\text{-variant } \va{c} \text{ of } \va{a}\\
      (\hspace{1pt}\ast\hspace{1pt}) &\textit{ ~iff~ } \VV{\I_\MetaD}{\va{a}}(\metaB\unisub{\beta}{\alpha})=1 \text{ for some constant } \beta\in\C\\
      (\hspace{.7pt}\star\hspace{.7pt}) &\textit{ ~iff~ } \metaB\unisub{\beta}{\alpha}\in\MetaD \text{ for some constant } \beta\in\C\\ 
      (\hspace{.7pt}\exists\hspace{.7pt}) &\textit{ ~iff~ } \qt{\exists}{\alpha}\metaB\in\MetaD \\
      &\textit{ ~iff~ } \metaA\in\MetaD.
  \end{align*}

  Assume $\metaB\unisub{\beta}{\alpha}\in\MetaD$ for some constant $\beta\in\C$.
  Thus $\MetaD\vdash \qt{\exists}{\alpha}\metaB$ by existential introduction $\exists$I, and so $\qt{\exists}{\alpha}\metaB\in\MetaD$ by \textbf{\ref{lemma:FOL-deductive}}.
  % TODO: consider case with no free variables
  Assuming $\qt{\exists}{\alpha}\metaB\in\MetaD$ instead, we know that $\metaB$ has at most one free variable $\alpha$, and so $\metaB=\metaA_i(\alpha_i)$ for some $i\in\N$ where $\alpha_i=\alpha$ by the enumeration given in $\S\ref{sub:Saturation}$.
  Thus $\qt{\exists}{\alpha_i}\metaA_i\eif \metaA_i\unisub{n_i}{\alpha_i}\in\MetaD$ by the saturation assumed of $\MetaD$.
  Since $n_i\in\C$, it follows that $\qt{\exists}{\alpha}\metaB\eif\metaB\unisub{\beta}{\alpha}\in\MetaD$ for some $\beta\in\C$, and so $\MetaD\vdash\metaB\unisub{\beta}{\alpha}$ by conditional elimination $\eif$E. 
  We may then conclude by \textbf{\ref{lemma:FOL-deductive}} that $\metaB\unisub{\beta}{\alpha}\in\MetaD$, thereby establishing $(\hspace{.7pt}\exists\hspace{.7pt})$.

  Additionally, $(\hspace{.7pt}\star\hspace{.7pt})$ holds by hypothesis, $(\hspace{.7pt}\ast\hspace{.7pt})$ is given by \textbf{\ref{lemma:FOL-exists}}, and the other biconditionals follow from the case assumption.

  \textit{Case 7:}
  Assume $\metaA=\qt{\forall}{\alpha}\metaB$.
  Since $\comp(\qt{\forall}{\alpha}\metaB)=\comp(\metaB)+1$ and $\comp(\metaA)=n+1$, it follows that $\comp(\metaB)=n$.
  We may then reason as follows:

  \vspace{-.2in}
  \begin{align*}
    \VV{\I_\MetaD}{\va{a}}(\metaA)=1 &\textit{ ~iff~ } \VV{\I_\MetaD}{\va{a}}(\qt{\forall}{\alpha}\metaB)=1\\
      % &\textit{ ~iff~ } \VV{\I_\MetaD}{\va{c}}(\metaB)=1 \text{ for some } \alpha\text{-variant } \va{c} \text{ of } \va{a}\\
      (\hspace{1pt}\ast\hspace{1pt}) &\textit{ ~iff~ } \VV{\I_\MetaD}{\va{a}}(\metaB\unisub{\beta}{\alpha})=1 \text{ for every constant } \beta\in\C\\
      (\hspace{.7pt}\star\hspace{.7pt}) &\textit{ ~iff~ } \metaB\unisub{\beta}{\alpha}\in\MetaD \text{ for every constant } \beta\in\C\\ 
      (\hspace{.3pt}\forall\hspace{.3pt}) &\textit{ ~iff~ } \qt{\forall}{\alpha}\metaB\in\MetaD \\
      &\textit{ ~iff~ } \metaA\in\MetaD.
  \end{align*}

  Assuming $\qt{\forall}{\alpha}\metaB \in \MetaD$ and letting $\beta \in \C$ be arbitrary, it follows that $\MetaD \vdash \metaB\unisub{\beta}{\alpha}$ by universal elimination $\forall$E, and so $\metaB\unisub{\beta}{\alpha} \in \MetaD$ by \textbf{\ref{lemma:FOL-deductive}}. 

  Assume instead that $\qt{\forall}{\alpha}\metaB \notin \MetaD$.
  Since $\MetaD$ is maximal, $\enot\qt{\forall}{\alpha}\metaB \in \MetaD$.
  By ($\enot\forall$) from $\S\ref{QER}$, we know that $\enot\qt{\forall}{\alpha}\metaB \vdash \qt{\exists}{\alpha}\enot\metaB$, and so $\MetaD \vdash \qt{\exists}{\alpha}\enot\metaB$ by \textbf{\ref{lemma:FOL-prweak}}.
  Thus $\qt{\exists}{\alpha}\enot\metaB \in \MetaD$ by \textbf{\ref{lemma:FOL-deductive}}.

  Given that $\enot\metaB$ has at most one free variable $\alpha$, it follows that $\enot\metaB=\metaA_i(\alpha_i)$ for some $i\in\N$ where $\alpha_i=\alpha$ by the enumeration given in $\S\ref{sub:Saturation}$.
  Since $\MetaD$ is saturated, it follows that $\qt{\exists}{\alpha_i} \metaA_i \eif \metaA_i\unisub{n_i}{\alpha_i} \in \MetaD$ where $n_i\in\C$, and so $\qt{\exists}{\alpha}\enot\metaB \eif \enot\metaB\unisub{\beta}{\alpha} \in  \MetaD$ for some $\beta \in \C$.
  By conditional elimination $\eif$E, we know $\MetaD \vdash \enot\metaB\unisub{\beta}{\alpha}$.

  If $\metaB\unisub{\beta}{\alpha} \in \MetaD$, then $\MetaD \vdash \metaB\unisub{\beta}{\alpha}$, and so it would follow by \textbf{\ref{lemma:FOL-cont}} that $\MetaD$ is inconsistent contrary to assumption. 
  Thus $\metaB\unisub{\beta}{\alpha} \notin \MetaD$ for some $\beta \in \C$.
  By contraposition, we may conclude that if $\metaB\unisub{\beta}{\alpha} \in \MetaD$ for all $\beta \in \C$, then $\qt{\forall}{\alpha}\metaB \in \MetaD$.
  Together with the above, this establishes $(\forall)$. 

  Additionally, $(\hspace{.7pt}\star\hspace{.7pt})$ holds by hypothesis, $(\hspace{.7pt}\ast\hspace{.7pt})$ is given by \textbf{\ref{lemma:FOL-forall}}, and the other biconditionals follow from the case assumption.

  \textit{Conclusion:}
  It follows by induction that $\VV{\I_\MetaD}{\va{a}}(\metaA)=1$ just in case $\metaA\in\MetaD$ where the variable assignment $\va{a}$ and sentence $\metaA$ in $\FIN$ where both arbitrary. 
  Thus we have:
  \begin{align*}
    \M_\MetaD \text{ satisfies } \metaA &\textit{ ~iff~ } \VV{\I_\MetaD}{}(\metaA)=1\\
      &\textit{ ~iff~ } \VV{\I_\MetaD}{\va{a}}(\metaA)=1 \text{ for all v.a. } \va{a}\\
      (\star) &\textit{ ~iff~ } \metaA\in\MetaD.
  \end{align*}
  Whereas the induction argument presented above was required to establish $(\star)$, the other biconditionals follow from the definitions.
  This completes the proof.
  \qed
\end{quote}

Most of the work is done.
All that remains is to connect the dots in order to establish \textsc{FOL$^=$ Completeness}.
In particular, the following section extend the results that we have established for $\FIN$ to apply to our original language $\FI$.
Since the details of the lemmas given above are independent of the global structure of the proof, it is worth reviewing the way that the proof was set up in order to regain perspective of the whole before continuing.




\section{Restriction}%
  \label{sub:Restriction}
 
It follows immediately from \textbf{\ref{lemma:FOL-truth}} that $\M_\MetaD$ satisfies $\MetaD$ and so $\M_\MetaD$ satisfies $\MetaG$ since $\MetaG\subseteq\MetaD$ by \textbf{\ref{lemma:FOL-include}}. 
Since $\M_\MetaD$ is a model of $\FIN$ and not $\FI$, it remains to show that $\MetaG$ is satisfied by a model of $\FI$.
Thus we will restrict $\M_\MetaD$ to $\FI$ as follows: 
  \begin{enumerate}[leftmargin=1.5in]
    \item[\it Restriction:] $\I'_\MetaD(\alpha)=[\alpha]_\MetaD$ for every constant $\alpha$ in $\FI$.
    \item[~] $\I'_\MetaD(\F^n)=\I_\MetaD(\F^n)$ for all $n$-place predicates $\F^n$ and $n\in\N$. 
  \end{enumerate}
Since the predicates in $\FIN$ are the same as those in $\FI$, no change to the predicate extensions is required.
By contrast, the set of constants in $\FI$ is a proper subset of the set of constants in $\FIN$. 
Given that our aim is to restrict consideration to the expressions in $\FI$, it doesn't matter that the elements in $\D_\MetaD$ over which we are interpreting $\FI$ may contain constants that do not belong to $\FI$.
Rather, we simply need some nonempty set of elements over which to interpret $\FI$.
Since nothing requires the domain by which we interpret $\FI$ to only include constants that belong to $\FI$, interpreting $\FI$ over the domain $\D_\MetaD$ will suffice.

Given these considerations, we may take $\D'_\MetaD=\D_\MetaD$ as before, letting $\M'_\MetaD=\tuple{\D'_\MetaD,\I'_\MetaD}$ be the restriction of $\M$ to $\FI$ as defined above. 
Since every wfs of $\FIN$ is also a wfs of $\FI$, it is easy to show that $\M'_\MetaD$ and $\M_\MetaD$ satisfy the same $\FI$ sentences, and so $\M'_\MetaD$ satisfies $\MetaG$ where $\M'_\MetaD$ is a model of $\FI$. 
Thus we may conclude that $\MetaG$ is satisfiable as desired. 

In support of this conclusion, we begin by drawing on \textbf{\ref{lemma:model}} in order to establish the following result without having to indulge in another induction proof on the complexity.

\begin{Lthm} \label{lemma:FOL-satrest}
  For all $\FI$ sentences $\metaA$, $\M'_\MetaD$ satisfies $\metaA$ just in case $\M_\MetaD$ satisfies $\metaA$.
\end{Lthm}

\begin{quote} 
  \textit{Proof:}
  Since $\M'=\tuple{\D_\MetaD,\I_\MetaD'}$ and $\M=\tuple{\D_\MetaD,\I_\MetaD}$ share the same domain $\D_\MetaD$ where $\I_\MetaD$ and $\I'_\MetaD$ agree about every $n$-place predicate $\F^n$ and constant $\alpha$ that occurs in any wfs $\metaA$ of $\FI$, we know that $\VV{\I'_\MetaD}{\va{a}}(\metaA) = \VV{\I_\MetaD}{\va{a}}(\metaA)$ for any variable assignment $\va{a}$ defined over $\D_\MetaD$ by \textbf{\ref{lemma:model}}.
  We may then reason as follows:
  \begin{align*}
    \M'_\MetaD \text{ satisfies } \metaA &\textit{ ~iff~ } \VV{\I'_\MetaD}{}(\metaA)=1\\
    &\textit{ ~iff~ } \VV{\I'_\MetaD}{\va{a}}(\metaA)=1 \text{ for all v.a. } \va{a}\\
    (\star) &\textit{ ~iff~ } \VV{\I_\MetaD}{\va{a}}(\metaA)=1 \text{ for all v.a. } \va{a}\\
    &\textit{ ~iff~ } \VV{\I_\MetaD}{}(\metaA)=1\\
    &\textit{ ~iff~ } \M_\MetaD \text{ satisfies } \metaA.
  \end{align*}
  Whereas $(\star)$ follows from \textbf{\ref{lemma:model}}, the other biconditionals follow from the definitions.
  This completes the proof.
  \qed
\end{quote}

Since $\M_\MetaD$ satisfies $\MetaG$ where every $\metaA\in\MetaG$ is a sentence of $\FI$, it follows from \textbf{\ref{lemma:FOL-satrest}} that $\M'_\MetaD$ satisfies $\MetaG$ where $\M'_\MetaD$ is a $\FI$ model. 
Thus $\MetaG$ is satisfiable with respect to the models of $\FI$.
Since $\MetaG$ was any consistent set, we may draw the following conclusion:

% \factoidbox{
  \begin{Tthm} \label{thm:FOL-conscomp}
    Every consistent set of $\FI$ sentences $\MetaG$ is satisfiable. 
  \end{Tthm}
% }

\begin{quote} 
  \textit{Proof:} 
  Let $\MetaG$ be a consistent set of $\FI$ sentences in FOL$^=$.
  By \textbf{\ref{lemma:FOL-const}}, $\MetaG$ is a set of $\FIN$ sentences that is consistent in FOL$^=_\N$, and so $\MetaS_\MetaG$ is consistent and saturated in $\FIN$ by \textbf{\ref{lemma:FOL-sat}}. 
  Given \textbf{\ref{lemma:FOL-max}} and \textbf{\ref{lemma:FOL-include}}, $\MetaD_{\MetaS_\MetaG}$ is a saturated maximal consistent set of sentences in $\FIN$ where $\MetaG\subseteq\MetaD_{\MetaS_\MetaG}$.
  Letting $\MetaD=\MetaD_{\MetaS_\MetaG}$, \textbf{\ref{lemma:FOL-truth}} shows that the Henkin model $\M_\MetaD$ satisfies $\metaA$ just in case $\metaA\in\MetaD$, and so $\M_\MetaD$ satisfies $\MetaD$.
  Having shown that $\MetaG\subseteq\MetaD$, we know that $\M_\MetaD$ satisfies $\MetaG$.
  Since $\MetaG$ is a set of $\FI$ sentences, it follows by \textbf{\ref{lemma:FOL-satrest}} that there is a model $\M'_\MetaD$ of $\FI$ that satisfies $\MetaG$.
  Thus $\MetaG$ is satisfiable.
  \qed
\end{quote}

Whereas we began with the assumption that $\MetaG$ is consistent in FOL$^=$ which is made up of rules defined for the language $\FI$, \textbf{\ref{lemma:FOL-const}} established the consistency of $\MetaG$ in FOL$^=_\N$ since $\MetaG$ is also a set of wfss of $\FIN$.
Throughout the remainder of the lemma, consistency is defined with respect to FOL$^=_\N$ rather than FOL$^=$ in order to show that $\MetaG$ is satisfied by a model of $\FIN$. 
It is not until \textbf{\ref{lemma:FOL-satrest}} that consideration is returned to $\FI$, showing that $\MetaG$ is also satisfied by a model of $\FI$, and so is satisfiable in the desired sense.

Given the previous result, the completeness of FOL$^=$ over the semantics for $\FI$ follows as a corollary. 
The proof below formalizes the reasoning that we provided early on in order to motivate our approach to ultimately establishing the following result.



\begin{Cthm}[\sc FOL$^=$ Completeness] \label{cor:FOL-completeness}
  If $\MetaG\vDash\metaA$, then $\MetaG\vdash\metaA$.
\end{Cthm}

\begin{quote} 
  \textit{Proof:} Assume $\MetaG\vDash\metaA$ and let $\M=\tuple{\D,\I}$ be a model that satisfies $\MetaG$.
  It follows that $\M$ satisfies $\metaA$, and so $\VV{\I}{\va{a}}(\metaA)=1$ for every variable assignment $\va{a}$.
  Given the semantics for negation, $\VV{\I}{\va{a}}(\enot\metaA)\neq 1$ for every variable assignment $\va{a}$, and so $\VV{\I}{}(\enot\metaA)\neq 1$.
  Thus $\M$ does not satisfy $\enot\metaA$.
  By generalising on $\M$, no model that satisfies $\MetaG$ also satisfies $\enot\metaA$, and so $\MetaG\cup\set{\metaA}$ is unsatisfiable. 
  By \textbf{\ref{thm:FOL-conscomp}}, $\MetaG\cup\set{\enot\metaA}$ is inconsistent, and so $\MetaG\vdash\enot\enot\metaA$ by \textbf{\ref{lemma:FOL-incon}}.
  Since $\enot\enot\metaA\vdash\metaA$ by DN from $\S\ref{EFQ}$, we may conclude that $\MetaG\vdash\metaA$ by \textbf{\ref{lemma:FOL-prcut}}.
  \qed
\end{quote}

Completeness may seems like a good property for any proof system to have.
In analogy, if we could build a complete calculator that could compute any arithmetical operation, that would seem like a good thing.
With respect to the semantics we provided for $\FI$, the completeness of FOL$^=$ shows that there is no (extensionally) better proof system which allows us to derive a valid inference that FOL$^=$ leaves out.
However, there is another perspective which takes completeness to describe a certain limitation on what sorts of logical consequences hold between the sentences in $\FI$, calling the notion of logical consequence that we provided for $\FI$ into question.
We will close by briefly reflecting on the status of FOL$^=$. % and its corresponding theory of logical consequence, discouraging any temptation to consider FOL$^=$.
% Rather than calling for any specific revision to the semantics that we provided, the






\section{Compactness}%
  \label{sec:Compactness}

In order to discipline the following reflections we may begin with an important consequence of the completeness of FOL$^=$, mirroring a related result that we established for PL.

\begin{Cthm} \label{cor:compact}
  If $\MetaG\vDash\metaA$, then there is a finite subset $\Lambda\subseteq\MetaG$ where $\Lambda\vDash\metaA$.
\end{Cthm}

\begin{quote} 
  \textit{Proof:} 
  Assume $\MetaG\vDash\metaA$.  
  Thus $\MetaG\vdash\metaA$ by \textsc{FOL$^=$ Completeness}, and so there is a derivation $X$ of $\metaA$ from $\MetaG$.
  Letting $\MetaG_X$ be the set of premises which appear in $X$, it follows that $\MetaG_X\vdash\metaA$, and so $\MetaG_X\vDash\metaA$ by \textsc{FOL$^=$ Soundness}.
  Since $X$ is finite, $\MetaG_X$ is also finite, and so there is a finite subset $\Lambda\subseteq\MetaG$ where $\Lambda\vDash\metaA$.
  \qed
\end{quote}



\begin{Cthm}[Compactness] \label{cor:compact2}
  $\MetaG$ is satisfiable if every finite subset $\Lambda\subseteq\MetaG$ is satisfiable.
\end{Cthm}

\begin{quote} 
  \textit{Proof:} 
  Assume for contraposition that $\MetaG$ is unsatisfiable. 
  It follows vacuously that $\MetaG\vDash\bot$, and so $\Lambda\vDash\bot$ by \textbf{\ref{cor:compact}} for some finite subset $\Lambda\subseteq\MetaG$.
  Thus there is some finite subset $\Lambda\subseteq\MetaG$ that is unsatisfiable. 
  By contraposition, if every finite subset $\Lambda\subseteq\MetaG$ is satisfiable, then $\MetaG$ is satisfiable. 
  \qed
\end{quote}


This property is referred to as \define{compactness}.
Although compactness may seems like a nice property, it demonstrates that there cannot be wfs in $\FI$ which are only logical consequences of infinite sets of wfss of $\FI$.
However, there would seems to be some natural examples.
For instance, let $\MetaG_\infty=\set{\exists_{\geq n}xFx:n\in\N}$ be the set of wfs which say that at least $n$ things are $F$ for every natural number $n$. 
Although it would seem that it is a logical consequence of $\MetaG_\infty$ that infinitely many things are $F$, this logical consequence cannot hold by compactness.
More precisely, there is no wfs of $\FI$ which asserts that infinitely many things are $F$.

In order to see this, suppose that there were a wfs $A_\infty$ of $\FI$ that is satisfied by all and only the models in which infinitely many things are $F$.
It follows that $\MetaG_\infty\vDash A_\infty$, and so $\Lambda\vDash A_\infty$ for some finite subset $\Lambda\subseteq\MetaG_\infty$ by compactness.
However, since every finite subset $\Lambda\subseteq\MetaG_\infty$ will have a finite model, $A_\infty$ must have a finite model given that $\Lambda\vDash A_\infty$. 
But this contradicts the assumption that $A_\infty$ is only satisfied by models in which infinitely many things are $F$.
As a result, there are no wfs of $\FI$ such as $A_\infty$ that only have infinite models. %, and so there is much that we cannot say in our present language.
% Although this does not mean that $\FI$ along with its semantics and the definition of logical consequence is of no use, it does means that we should not take 

These conclusions do not tell against infinity, but rather expose a limitation of our present semantics.
Although this is a limitation that we can accept, it suggests that there are stronger notions of logical consequence that we may wish to consider.
These semantic theories will not be compact, and so will not admit of complete logics since otherwise we could construct a similar argument to what was given above.
From this perspective, completeness describes a limitation of our semantics for $\FI$ rather than a virtue.
Even though $\FI$ along with its semantics and proof system FOL$^=$ is extremely useful for a wide range of applications, logic does not end here. 
Rather, the systems that we have covered are just the beginning.


%%% TODO: add final section on decidability

% Although attempting to show that $\MetaG \plvdash \metaA$ was not guaranteed to lead to a proof or to the conclusion that no proof exists, truth-tables provided an effective procedure for determining whether $\MetaG\vDash\metaA$ or not. 
% Put otherwise, logical consequence is \define{decidable} in PL insofar as there is an effective procedure for determining in a finite number of steps whether $\MetaG \vDash \metaA$.
% By soundness and completeness... % TODO continue
%
% By contrast, logical consequence in $\FI$ is \define{undecidable} insofar as there is no such method for determining whether or not $\MetaG \vDash \metaA$.
% Even though we may construct a tree method for $\FI$, this method is not guaranteed to lead to either a positive or negative determination in the way that it did for PL.
% This makes the tree method a lot less interesting for $\FI$.
% As a result, the following chapters will establish soundness and completeness for FOL$^=$ rather than for an extension of the tree method for $\FI$.


\iffalse

\practiceproblems

\solutions
\problempart
\label{pr.QLalttrees-sound}
Following are possible modifications to our QL tree system. For each, imagine a system that is like the system laid out in this chapter, except for the indicated change. Would the modified tree system be sound? If so, explain how the proof given in this chapter would extend to a system with this rule; if not, give a tree that is a counterexample to the soundness of the modified system.
\begin{earg}
\item Change the rule for existentials to this rule:
	\factoidbox{
	\begin{center}
	\begin{prooftree}
	{not line numbering}
	[\qt{\exists}{\script{x}}\metaA{}, checked={\script{a}}
		[\metaA{}\substitute{\script{x}}{\script{a}}, just=for \emph{any} \script{a}
		]
	]
	\end{prooftree}
	\end{center}
	}
	
\item Change the rule for existentials to this rule:
	\factoidbox{
	\begin{center}
	\begin{prooftree}
	{not line numbering}
	[\qt{\exists}{\script{x}}\metaA{}, checked=d
		[\metaA{}\substitute{\script{x}}{d}, just=(whether or not $d$ is new)
		]
	]
	\end{prooftree}
	\end{center}
	}

\item Change the rule for existentials to this rule:
	\factoidbox{
	\begin{center}
	\begin{prooftree}
	{not line numbering}
	[\qt{\exists}{\script{x}}\metaA{}, checked
		[\metaA{}\substitute{\script{x}}{\script{a}}, just={for 3 different names, old or new}
		[ , grouped
		[\metaA{}\substitute{\script{x}}{\script{b}}, grouped
		[ , grouped
		[\metaA{}\substitute{\script{x}}{\script{c}}, grouped
		]
		]
		]
		]
		]
	]
	\end{prooftree}
	\end{center}
	}

\item Change the rule for universals to this rule:
	\factoidbox{
	\begin{center}
	\begin{prooftree}
	{not line numbering}
	[\qt{\forall}{\script{x}}\metaA{}, checked
		[\metaA{}\substitute{\script{x}}{\script{a}}, just={for 3 different names, old or new}
		[ , grouped
		[\metaA{}\substitute{\script{x}}{\script{b}}, grouped
		[ , grouped
		[\metaA{}\substitute{\script{x}}{\script{c}}, grouped
		]
		]
		]
		]
		]
	]
	\end{prooftree}
	\end{center}
	}

\item Change the rule for existentials to this rule:
	\factoidbox{
	\begin{center}
	\begin{prooftree}
	{not line numbering}
	[\qt{\exists}{\script{x}}\metaA{}, checked
		[\metaA{}\substitute{\script{x}}{\script{a}}, just={for 3 new names}
		[ , grouped
		[\metaA{}\substitute{\script{x}}{\script{b}}, grouped
		[ , grouped
		[\metaA{}\substitute{\script{x}}{\script{c}}, grouped
		]
		]
		]
		]
		]
	]
	\end{prooftree}
	\end{center}
	}

\item Change the rule for universals to this rule:
	\factoidbox{
            	\begin{center}
            \begin{prooftree}
            {not line numbering}
            [\qt{\forall}{\script{x}}\metaA{}, checked={\script{a}}
            	[\metaA{}\substitute{\script{x}}{\script{a}}, just=where \script{a} is \emph{new}
            	]
            ]
            \end{prooftree}
            \end{center}
	}

\item Change the rule for conjunction to this rule:
	\factoidbox{
            	\begin{center}
            \begin{prooftree}
            {not line numbering}
            	[\metaA{} \eand \metaB{}, checked
            		[\qt{\exists}{\script{x}} \metaA{}, just=where \script{x} does not occur in \metaA{}
			[\metaB{}, grouped
            		]
            		]
		]
            \end{prooftree}
            \end{center}
	}


\item Change this requirement (given on page \pageref{branchcompletion.defined})...
	\factoidbox{A branch is \define{complete} if and only if either (i) it is closed, or (ii) every resolvable sentence in every branch has been resolved, and for every general sentence and every name \script{a} in the branch, the \script{a} instance of that general sentence has been taken.}
	...to this one:
	\factoidbox{A branch is \define{complete} if and only if either (i) it is closed, or (ii) every resolvable sentence in every branch has been resolved, and for every general sentence, \emph{at least one instance of} that general sentence has been taken.}

\item Change the branch completion requirement to:
	\factoidbox{\ldots and for every general sentence and every name \script{a} \emph{that is above that general sentence in the branch}, the \script{a} instance of that general sentence has been taken.}

\item Change the branch completion requirement to:
	\factoidbox{\ldots and for every general sentence and every name \script{a} in the branch, the \script{a} instance of that general sentence has been taken, \emph{and at least one additional new instance of that general sentence has also been taken}.}
	
	\end{earg}
	
	
	
	
\solutions
\problempart
\label{pr.QLalttrees-complete}
For each of the rule modifications given in Part \ref{pr.QLalttrees-sound}, would the modified tree system be complete? If so, explain how the proof given in this chapter would extend to a system with this rule; if not, give a tree that is a counterexample to the completeness of the modified system.

\fi

