%!TEX root = ../../forallx-mit.tex
\chapter{Logical Consequence}
\label{ch.PL-semantics}

Whereas the previous chapter introduced truth tables, this chapter will present the truth table method which provides a decidable procedure for evaluating the validity of arguments in $\PL$.
Given the (albeit limited) expressive power of $\PL$, this amounts to a mechanical way of evaluating the natural language arguments that admit of reasonably faithful regimentations in $\PL$.
After reviewing both the advantages and disadvantages of this procedure, the second half of the chapter will present a more versatile alternative.




\section{Truth-Functional Operators}

Any wfs of $\PL$ that is not a sentence letter is composed of sentence letters together with the sentential operators.
In Chapter \ref{ch.PL-syntax}, we offered truth tables for each operator.
The fact that it is possible to give truth tables like this is very significant.
It means that our operators are \define{truth-functional}.
That is to say, the only thing that matters for determining the truth-value of a given wfs of $\PL$ is the truth-values of its constituent.
For instance, to determine the truth-value of a sentence $\enot A$, the only thing that matters is the truth-value of $A$.
% You don't have to know what $A$ means, or where it came from, or what evidence there is for it and what that evidence might depend on.
Given any interpretation, the truth-value of a negation on that interpretation is a \textit{function} of the truth-value of its negand on that interpretation, and likewise for the other operators.

We are using the same notion of a function that you have probably encountered in mathematics.
First we may define the \define{cartesian product} $X \times Y$ of the sets $X$ and $Y$ to be the set of all ordered pairs $\tuple{x, y}$ where $x \in X$ and $y \in Y$ which we may write in set-builder notation as $X \times Y \coloneq \set{\tuple{x, y} : x \in X,\ y \in Y}$.\footnote{The `$\coloneq$' symbol signifies that a definition is being provided.}
A \define{relation} from $X$ to $Y$ is any subset $A \subseteq X \times Y$.
% Given a set $D$ and a set called the $R$, 
A \define{function} $f: D \to R$ from the \define{domain} $D$ to the \define{range} $R$ is any relation $f \subseteq D \times R$ from $D$ to $R$ where $f(x) = f(y)$ for any $x, y \in D$ where $x = y$. 
Intuitively, a function from one set to another associates each member of the first set (the domain) with exactly one member of the second set (the range).
Once the first element is fixed, the function uniquely selects an element of the second set.
Instead of always writing $f(x, y)$, it makes sense to write $f(x) = y$ for the unique element in the range $y$ to which the element $x$ in the domain is mapped.
For instance, given any numerical value of $x$, we may unambiguously determine the value of $x^{2}$, and so $f(x)=x^{2}$ is a function.
In the same way, the truth-value of $A$ on an interpretation will unambiguously determine the truth-value of $\enot A$ on that interpretation.

Truth-functionality is not inevitable.
The syntax of English, for example, permits one to make a new declarative English sentence by prefixing the phrase `Ted Cruz doesn't care whether' in front of any declarative English sentence.
In this respect, `Ted Cruz doesn't care whether' is syntactically similar to `$\enot$' in $\PL$: it is a sentential operator, producing new sentences from old. 
Nevertheless, it is impossible to give a truth-functional characterization of the operator `Ted Cruz doesn't care whether' in English that respects its intuitive meaning in English.
If you want to know whether Ted Cruz cares about fixing Texas's electrical grid, it's not enough to know whether anyone is fixing Texas's electrical grid.
If it is being fixed, he might care or he might not.
If it is not be fixed, again he might care or might not. 
Thus `Ted Cruz doesn't care whether' is not truth-functional: it operates on more than just the truth-value of its argument. % \footnote{Recall the other meaning that `argument' has from $\S\ref{sec.Bicon}$.}
By contrast, the sentential operators included in $\PL$ are truth-functional, and it is for this reason that we are able to construct truth tables.



\section{Complete Truth Tables}

A truth table for a sentence may be constructed by writing the sentence in question at the top right of a table, and each of the distinct sentence letters immediately to the left on the top row.
We then add $2^n$ rows below the top row where $n$ is the number of distinct sentence letters. 
For instance, if there are only two sentence letters, we will need four rows of truth-values.
Beginning with the sentence letter furthest to the left, we fill out the column with $2^{(n-1)}$ copies of 1 followed by $2^{(n-1)}$ copies of 0.
Moving to the next sentence letter, we fill out the column with $2^{(n-2)}$ copies of 1 followed by $2^{(n-2)}$ copies of 0.
We then proceed to the next sentence letter (if there is one), following the same pattern as before but with $2^{(n-3)}$ copies of 1 and 0, respectively.
Continue this process until all sentence letters in the table have truth-values below them.
This completes the truth table setup.

In order to assign truth-values to complex sentences, consider the following \define{characteristic truth tables} where instead of particular sentence letters, we will use schematic variables:

\begin{table}[htb]
\begin{center}
\begin{tabular}{c|c}
\metaA & \enot\metaA\\
\hline
1 & 0\\
0 & 1\\
~\\
~
\end{tabular}
\ \ \ \ 
\begin{tabular}{c|c|c|c|c|c}
\metaA & \metaB & $\metaA\eand\metaB$ & $\metaA\eor\metaB$ & $\metaA\eif\metaB$ & $\metaA\eiff\metaB$\\
\hline
1 & 1 & 1 & 1 & 1 & 1\\
1 & 0 & 0 & 1 & 0 & 0\\
0 & 1 & 0 & 1 & 1 & 0\\
0 & 0 & 0 & 0 & 1 & 1
\end{tabular}
\end{center}
\caption{The characteristic truth tables for the operators of $\PL$.}
\label{table.CharacteristicTTs}
\end{table}

The table above provides a general recipe for calculating truth-values for any sentences $\metaA$ and $\metaB,$ however complex.
The characteristic truth table for conjunction, for example, gives the truth conditions for any sentence of the form $\metaA\eand\metaB$.
Even if the conjuncts $\metaA$ and $\metaB$ are complex sentences, the conjunction is true if and only if both $\metaA$ and $\metaB$ are true.

Let's construct a truth table for the complex sentence $(H\eand I)\eif H$.
We consider all possible combinations of $1$ and $0$ for $H$ and $I$, which gives us four rows.
We then copy the truth-values for the sentence letters and write them underneath the letters in the sentence.

\begin{center}
\begin{tabular}{c|c|@{\TTon}*{5}{c}@{\TToff}}
$H$&$I$&$(H$&\eand&$I)$&\eif&$H$\\
\hline
 1 & 1 & \TTbf{1} && \TTbf{1} && \TTbf{1}\\
 1 & 0 & \TTbf{1} && \TTbf{0} && \TTbf{1}\\
 0 & 1 & \TTbf{0} && \TTbf{1} && \TTbf{0}\\
 0 & 0 & \TTbf{0} && \TTbf{0} && \TTbf{0}
\end{tabular}
\end{center}
So far, all we have done is duplicate the first two columns.
We have written the $H$ column twice--- once under each $H$--- and the $I$ column once under the $I$.

Now consider the subsentence $H \eand I$ which is a conjunction. 
Since $H$ and $I$ are both true on the first row and a conjunction is true when both conjuncts are true, we write a 1 underneath the conjunction symbol.
In the other three rows, at least one of the conjuncts is false, so the conjunction $H \eand I$ is false.
So we write 0s under the conjunction symbol on those rows:
\begin{center}
\begin{tabular}{c|c|@{\TTon}*{5}{c}@{\TToff}}
$H$&$I$&$(H$&\eand&$I)$&\eif&$H$\\
\hline
 1 & 1 & 1 & \TTbf{1} & 1 & & 1\\
 1 & 0 & 1 & \TTbf{0} & 0 & & 1\\
 0 & 1 & 0 & \TTbf{0} & 1 & & 0\\
 0 & 0 & 0 & \TTbf{0} & 0 & & 0
\end{tabular}
\end{center}
The entire sentence $(H \eand I) \eif H$ is a conditional.
On the second row, for example, $H\eand I$ is false and $H$ is true.
Since a conditional is true when the antecedent is false, we write a 1 in the second row underneath the conditional symbol.
Using the truth table for the material conditional where $H \eand I$ is the antecedent and $H$ as the consequent, we may derive the following values for the material conditional claim as a whole:
\begin{center}
\begin{tabular}{c|c|@{\TTon}*{5}{c}@{\TToff}}
$H$&$I$&$(H$&\eand&$I)$&\eif&$H$\\
\hline
 1 & 1 & & 1 & & \TTbf{1} & 1\\
 1 & 0 & & 0 & & \TTbf{1} & 1\\
 0 & 1 & & 0 & & \TTbf{1} & 0\\
 0 & 0 & & 0 & & \TTbf{1} & 0
\end{tabular}
\end{center}
In computing the value for the conditional (the column under $\eif$), it is only important to look at the values for its antecedent (the column under $\eand$) and the value of its consequent (the column under $H$ on the right).
The column of 1s underneath the conditional tells us that the sentence $(H \eand I)\eif H$ is true regardless of the truth-values of $H$ and $I$.
They can be true or false in any combination, and the compound sentence still comes out true.
% Although the truth-values under the conjuncts $H$ and $I$ are typically included, we have left them out above for the sake of clarity in while introducing this process. 

It is crucial that we have considered all of the possible combinations.
If we only had a two-line truth table, we could not be sure that the sentence was not false for some other combination of truth-values.
Since each row of the truth table represents a different way of interpreting the relevant sentence letters $H$ and $I$, each row corresponds to a distinct interpretation of the sentence letters in question. 
Moreover, every possible combination of truth-values for $H$ and $I$ have been included.
Since the truth-values of all other sentence letters do not effect the truth-value of the sentence in question, we may conclude that $(H \eand I)\eif H$ is \textit{true in all interpretations} whatsoever.
In other words, $(H \eand I)\eif H$ is a \textit{tautology}.

In this example, we have not repeated all of the entries in every successive table, so that it's easier for you to see which parts are new. When actually writing truth tables on paper, however, it is impractical to erase whole columns or rewrite the whole table for every step. Although it is more crowded, the truth table can be written in this way:

\begin{center}
\begin{tabular}{c|c|@{\TTon}*{5}{c}@{\TToff}}
$H$&$I$&$(H$&\eand&$I)$&\eif&$H$\\
\hline
 1 & 1 & 1 & {1} & 1 &\TTbf{1} & 1\\
 1 & 0 & 1 & {0} & 0 &\TTbf{1} & 1\\
 0 & 1 & 0 & {0} & 1 &\TTbf{1} & 0\\
 0 & 0 & 0 & {0} & 0 &\TTbf{1} & 0
\end{tabular}
\end{center}

Most of the columns underneath the sentence are only there for bookkeeping purposes.
When you become more adept with truth tables, you will probably no longer need to copy over the columns for each of the sentence letters.
In any case, the truth-values for the original sentence is given by the column underneath the main logical operator of the sentence which in this case is the column underneath the conditional, marked in \textbf{bold} for clarity.

A \define{complete truth table} has a row for all possible combinations of $1$ and $0$ for the sentence letters and the characteristic truth tables have been used to write truth-values below all the operators.
The size of the complete truth table depends on the number of different sentence letters in the table.
A sentence that contains only one sentence letter requires only two rows, as in the characteristic truth table for negation.
This is true even if the same letter is repeated many times, as in the sentence $[(C\eiff C) \eif C] \eand \enot(C \eif C)$.
The complete truth table requires only two lines because there are only two possible interpretations of $C$: either it is true or it is false.
A single sentence letter can never be marked both 1 and 0 on the same row.
The truth table for this sentence looks like this:

\begin{center}
\begin{tabular}{c|@{\TTon}*{15}{c}@{\TToff}}
$C$ & [( & $C$ & \eiff & $C$ & ) & \eif & $C$ & ]  & \eand & \enot & ( & $C$ &\eif & $C$ & )\\
\hline
 1 &    & 1 &  1  & 1 &   & 1  & 1 & &\TTbf{0}&  0& &   1 &  1  & 1 &   \\
 0 &    & 0 &  1  & 0 &   & 0  & 0 & &\TTbf{0}&  0& &   0 &  1  & 0 &   \\
\end{tabular}
\end{center}

Looking at the column underneath the main operator, we see that the sentence is false on both rows of the table, and so it is false regardless of whether $C$ is true or false.
Since the rows of the truth table correspond to the different possible interpretations of the relevant sentence letters, the sentence above is false on all interpretations, and so it is a \textit{contradiction}.

A sentence that contains two sentence letters requires $2^2$ lines for a complete truth table as in the other characteristic truth tables and the table for $(H \eand I)\eif I$ above.
A sentence that contains three sentence letters requires $2^3$ lines.
For example:
  
\begin{center}
\begin{tabular}{c|c|c|@{\TTon}*{5}{c}@{\TToff}}
$M$&$N$&$P$&$M$&\eand&$(N$&\eor&$P)$\\
\hline
%           M        &     N   v   P
1 & 1 & 1 & 1 & \TTbf{1} & 1 & 1 & 1\\
1 & 1 & 0 & 1 & \TTbf{1} & 1 & 1 & 0\\
1 & 0 & 1 & 1 & \TTbf{1} & 0 & 1 & 1\\
1 & 0 & 0 & 1 & \TTbf{0} & 0 & 0 & 0\\
0 & 1 & 1 & 0 & \TTbf{0} & 1 & 1 & 1\\
0 & 1 & 0 & 0 & \TTbf{0} & 1 & 1 & 0\\
0 & 0 & 1 & 0 & \TTbf{0} & 0 & 1 & 1\\
0 & 0 & 0 & 0 & \TTbf{0} & 0 & 0 & 0
\end{tabular}
\end{center}

This table shows that $M\eand(N\eor P)$ is true on some interpretations and false on others depending on the truth-values of $M$, $N$, and $P$, and so it is \textit{contingent}.

% Recall that a complete truth table with $n$ sentence letters will have $2^n$ rows.
A complete truth table for a sentence that contains four different sentence letters requires $2^4$ lines.
This means that the truth table method becomes syntactically unmanageable very quickly.
This is a significant limitation of this method.
% By contrast, the semantic clauses can much more easily accommodate sentences and arguments with many sentence letters.





\section{Truth Table Definitions}
\label{sec.usingtruthtables}

Recall from \S\ref{sec-tautologydef} that an English sentence was said to be a tautology just in case it is true on all interpretations, a contradiction just in case it is false on all interpretations, and logically contingent just in case it is true on some interpretations and false on others.
Similarly, an English sentence logically entails another just in case the latter is true on any interpretation on which the former is true, and two English sentences are logically equivalent just in case they logically entail each other and so are true on exactly the same interpretations.
Even in restricting consideration to truth-values in interpreting the sentences of English, there is no well-defined set of English sentences, and so no corresponding definition of an interpretation as an assignment of every English sentence to a unique truth-value.
It is for this reason that we introduced $\PL$ in Chapter $\ref{ch.PL-syntax}$, carefully defining the wfss of $\PL$.

Given the definition of the wfs of $\PL$ together with the characteristic truth tables for the sentential operators of $\PL$, we are now in a position to define the interpretations of $\PL$ in a way that we could not do so for English.
In particular, an \define{interpretation} of $\PL$ is any function from the set of wfs of $\PL$ (the domain) to the set of truth-values $\set{0,1}$ (the range) which satisfies the characteristic truth tables given above.
For instance, any interpretation that assigns $A$ to $0$ will assign $\enot A$ to $1$. 
More generally, an interpretation assigns $\enot \metaA$ to $1$ just in case it assigns $\metaA$ to $0$ since this is what is required to satisfy the characteristic truth table for negation. 
Similarly, an interpretation assigns $\metaA \eand \metaB$ to $1$ just in case it assigns both $\metaA$ and $\metaB$ to $1$ as specified by the characteristic truth table for conjunction. 
% The assignments of truth-values to sentences in which the other sentential operators occur are specified by their
Given any wfs of $\PL$, the characteristic truth table for the main operator of that sentence determines the truth-value of that sentence as a function of the truth-value(s) for its arguments.
% As brought out above, the characteristic truth tables specify the truth-values of the wfs of $\PL$ as a function of the truth-values of in which that operator   assign truth-values to in corresponding ways.

Rather than worrying about the truth-value of all wfs of $\PL$ whatsoever, constructing a complete truth tables provides a way to interpret a sentence while limiting consideration to its parts.
This approach depends on the truth-functionality of the sentential operators.
% We may then put this method to work to define a \define{tautology} to be any wfs of $\PL$ which is true in every row of its complete truth table, and a \define{contradiction} to be any wfs of $\PL$ which is false in every row of its complete truth table. 
% Similarly, a wfs of $\PL$ is \define{logically contingent} just in case it is true in at least one row of its complete truth table and false in another row. 
% Given a complete truth table for a wfs of $\PL$, ther
We may then put this method to work to define a \define{tautology} of $\PL$ to be any sentence of $\PL$ whose truth table only has $1$s under its main operator.
Accordingly, a tautology is a sentence of $\PL$ whose truth does not depend on the particular truth-values of the sentence letters from which it was constructed, but rather follows from the \textit{logical form} of that sentence.
Similarly, a sentence is a \define{contradiction} in $\PL$ just in case the column under its main operator is $0$ on every row of its complete truth table.
Instead of being true in virtue of its logical form, a contradiction is false in virtue of its logical form.
A sentence is \define{logically contingent} in $\PL$ just in case the column under its main operator includes both $1$s and $0$s. %, and so is neither a tautology nor a contradiction.

% TODO this equivalence only holds given that we can prove that a sentences truth-value only depends on the sentence letters which it contains


\subsection{Logical Entailment and Equivalence}

A sentence in English was said to logically entail another just in case the latter is true on any interpretation on which the former is true.
Two sentences were then said to be logically equivalent in English just in case they logically entail each other, and so have the same truth-value on all interpretations.
We can now say with greater precision that a sentence of $\PL$ \define{logically entails} another just in case the complete truth table for these two sentences is such that on any row, the latter has a $1$ under its main operator whenever the former has a $1$ on that row under its main operator. 
Consider the following example:

\begin{center}
\begin{tabular}{c|c|@{\TTon}*{2}{c}@{\TToff}|@{\TTon}*{4}{c}@{\TToff}}
  $A$ & $B$ & $\enot$ & $A$ & $\enot$ & $(A$ & $\eand$ & $B)$ \\
\hline
 1 & 1 & \TTbf{0} & 1 & \TTbf{0} & 1 & 1 & 1 \\
 1 & 0 & \TTbf{0} & 1 & \TTbf{1} & 1 & 0 & 0 \\
 0 & 1 & \TTbf{1} & 0 & \TTbf{1} & 0 & 0 & 1 \\
 0 & 0 & \TTbf{1} & 0 & \TTbf{1} & 0 & 0 & 0 
\end{tabular}
\end{center}

Whereas before we only considered truth tables for one wfs at a time, the truth table above interprets two wfs of $\PL$ at once.
By including a column for all sentence letters contained in either wfs, the complete truth table exhausts all possible combinations of truth-values for the sentence letters upon which the truth of the wfss depend.
We may the observe that every row in which $\enot A$ has a $1$ under its main operator is also a row in which $\enot (A \eand B)$ has a $1$ under its main operator, and so $\enot A$ logically entails $\enot (A \eand B)$.
By contrast, given the second row of truth-values, $\enot (A \eand B)$ does not logically entail $\enot A$.
% Since the rows correspond to the interpretations of these two sentences, this captures the idea that 

In the special case where two wfs of $\PL$ logically entail each other, we may say that those sentences are \define{logically equivalent}.
It follows that any two wfs of $\PL$ are logically equivalent just in case the truth-values under their main operators are the same on every row of their complete truth table.
For instance, consider the sentences $\enot(A \eor B)$ and $\enot A \eand \enot B$.
In order to find out if they are logically equivalent we may construct their complete truth table:

\begin{center}
\begin{tabular}{c|c|@{\TTon}*{4}{c}@{\TToff}|@{\TTon}*{5}{c}@{\TToff}}
$A$ & $B$ & $\enot$ & $(A$ & $\eor$ & $B)$ & $\enot$ & $A$ & \eand & \enot & $B$ \\
\hline
 1 & 1 & \TTbf{0} & 1 & 1 & 1 & 0 & 1 & \TTbf{0} & 0 & 1\\
 1 & 0 & \TTbf{0} & 1 & 1 & 0 & 0 & 1 & \TTbf{0} & 1 & 0\\
 0 & 1 & \TTbf{0} & 0 & 1 & 1 & 1 & 0 & \TTbf{0} & 0 & 1\\
 0 & 0 & \TTbf{1} & 0 & 0 & 0 & 1 & 0 & \TTbf{1} & 1 & 0
\end{tabular}
\end{center}

The columns under the main operators for the two wfs on the right are identical.
Since the rows of a complete truth table exhaust the different interpretations of the relevant sentence letters, this amounts to requiring the two wfs of $\PL$ to have the same truth-value on every interpretation.
It is for this reason that the two wfs of $\PL$ are logically equivalent.
% TODO this also turns on the equivalence mentioned in the TODO above





\subsection{Satisfiability}
\label{sub:PL-Satisfiability}

In the previous example, a truth table was constructed for two wfss of $\PL$ at once.
More generally, we may take a \define{complete truth table} for a set $\Gamma$ of wfss of $\PL$ to be the result of listing each wfs in $\Gamma$ side-by-side on the top right of a table and then completing the table for each wfs in a similar manner to what was described above.

Recall that a set of sentences in English is satisfiable just in case there is an interpretation which makes them all true.
Analogously, we may wish to say that a set $\Gamma$ of wfss in $\PL$ is \define{satisfiable} just in case there is a row of a complete truth table including every wfs in the set where the main operator under every sentence is $1$, and \define{unsatisfiable} otherwise.
For instance, look again at the truth table above.
We see that $\set{\enot(A \eor B), \enot A \eand \enot B}$ is satisfiable, because there is at least one row where both sentences have $1$ under their main operators.

To take another example, we may ask if the set $\set{A, \enot (A \eor \enot B)}$ is satisfiable.
Here the answer is `No' as the following truth table shows:

\begin{center}
\begin{tabular}{c|c|@{\TTon}*{5}{c}@{\TToff}}
$A$ & $B$ & $\enot$ & $(A$ & $\eor$ & $\enot$ & $B)$ \\
\hline
\TTbf{1} & 1 & \TTbf{0} & 1 & 1 & 0 & 1 \\
\TTbf{1} & 0 & \TTbf{0} & 1 & 1 & 1 & 0 \\
\TTbf{0} & 1 & \TTbf{1} & 0 & 0 & 0 & 1 \\
\TTbf{0} & 0 & \TTbf{0} & 0 & 1 & 1 & 0 
\end{tabular}
\end{center}

Since there is no row in which both of the sentences in the set $\set{A, \enot (A \eor \enot B)}$ have a $1$ under their main operator, we may conclude that the set is unsatisfiable.




\subsection{Logical Consequence and Validity}
\label{sub:Consequence}

% TODO say that logical consequences that hold may nevertheless be said to be valid though this is loose

% TODO introduce truth table notion of validity in contrast to informal def as with the other concepts above

Whereas Chapter $\ref{ch.introduction}$ provided an intuitive definition of logical consequence for the sentences of English, we may now appeal to complete truth tables in order to provide a correlate for $\PL$.
In particular, a wfs $\metaA$ of $\PL$ is a \define{logical consequence} of a set of wfss $\Gamma$ of $\PL$ just in case every row of a complete truth table for the wfs in $\Gamma$ together with $\metaA$ is such that $\metaA$ has a $1$ under its main operator whenever every sentence in $\Gamma$ has a $1$ under its main operator. 
We may then say that an argument in $\PL$ with premises $\Gamma$ and conclusion $\metaA$ is \define{valid} just in case the conclusion is a logical consequence of the premises. 

Instead of saying that an English argument is valid by appealing to the interpretations of English, we may say that an argument in English has a $\PL$ regimentation that is valid. 
Regimenting an English argument in $\PL$ and constructing its complete truth table provides a way to identify the logical features which explain why the argument in English is valid.
Although there is a precise mathematical definition of validity in $\PL$, the same cannot be said for what does and does not count as a faithful regimentation of the argument.
Rather, this much remains intuitive, relying on and individual's best judgment.
In many cases, there may not be a uniquely best regimentation, or indeed any good regimentation at all.

% just in case there is no interpretation $\I$ of $\PL$ in which the premises are true and the conclusion is false.\footnote{Or more completely, it follows that an argument in $\PL$ is valid just in case there is no interpretation $\I$ where $\V{\I}(\metaA)=1$ for every premise $\metaA$ and yet $\V{\I}(\metaB)=1$ for the conclusion $\metaB$.}
Given any argument in $\PL$ whose premises are unsatisfiable, it follows that the argument is \textit{vacuously valid} since there is no row in a complete truth table for the premises with a $1$ under the main operator of every premise, and so vacuously, every row in a complete truth table in which there is a $1$ under the main operator of every premises also has a $1$ under the main operator for the conclusion.
For instance, consider the following argument:
    
\begin{earg}
  \eitem{$\neg A \eand B$}
  \uitem{$\enot B$ \quad}
  \eitem{$B$}
\end{earg}

Given the truth table method presented above, it is straightforward to show that the argument is indeed valid.
In particular, consider the following complete truth table:

\begin{center}
\begin{tabular}{c|c|@{\TTon}*{4}{c}@{\TToff}|@{\TTon}*{2}{c}@{\TToff}|c}
  $A$&$B$&\enot&$A$&\eand&$B$&\enot&B&B\\
\hline
% A   B   -   A      &       B         -    B     B     
  1 & 1 & 0 & 1 & \TTbf{0} & 1 & \TTbf{0} & 1 & \TTbf{1}\\
  1 & 0 & 0 & 1 & \TTbf{0} & 0 & \TTbf{1} & 0 & \TTbf{0}\\
  0 & 1 & 1 & 0 & \TTbf{1} & 1 & \TTbf{0} & 1 & \TTbf{1}\\
  0 & 0 & 1 & 0 & \TTbf{0} & 0 & \TTbf{1} & 0 & \TTbf{0}
\end{tabular}
\end{center}

It is easy to see that there is no row in which the premises all have $1$ under their main operators, and so trivially, there is no rows in which the premises all have a $1$ under their main operators and the conclusion has a $0$ under its main operator.
Put otherwise, every row in which the premises all have $1$ under their main operators--- all zero of them--- is such that the conclusion has a $1$ under its main operator.
Thus the argument is valid.

% Instead of appealing to the truth table for this argument, we may prove that the argument is valid by assuming that there is an interpretation $\I$ which makes the premises true and the conclusion false. 
% Accordingly $\V{\I}(\enot A \eand B)=\V{\I}(\enot B)=1$ and $\V{\I}(B)=0$.
% It follows by the semantic clause for conjunction that $\V{\I}(B)=1$, and from the semantic clause for negation that $\V{\I}(B)=0$, and so both $\I(B)=1$ and $\I(B)=0$ by the semantic clause for sentence letters.
% This is a contradiction, since $B$ cannot be both true and not true on any given interpretation.
% Thus there is no such interpretation $\I$, and so the argument is valid.
% More generally, any argument with unsatisfiable premises is valid.

It is worth contrasting the example above with the following argument:

\begin{earg}
  \eitem{$\enot L \eif (J \eor L)$}
  \uitem{$\enot L$ \quad}
  \eitem{$J$}
\end{earg}

It is good practice to consult your intuitions about whether an argument is valid before beginning to complete its truth table.
For instance, without looking any further, consider whether the argument above is valid.
Once you have a guess we may construct the following:

\begin{center}
\begin{tabular}{c|c|@{\TTon}*{6}{c}@{\TToff}|@{\TTon}*{2}{c}@{\TToff}|@{\TTon}c@{\TToff}}
$J$&$L$&\enot&$L$&\eif&$(J$&\eor&$L)$&\enot&L&J\\
\hline
%J   L   -   L      ->     (J   v   L)
 1 & 1 & 0 & 1 & \TTbf{1} & 1 & 1 & 1 & \TTbf{0} & 1 & \TTbf{1}\\
 1 & 0 & 1 & 0 & \TTbf{1} & 1 & 1 & 0 & \TTbf{1} & 0 & \TTbf{1}\\
 0 & 1 & 0 & 1 & \TTbf{1} & 0 & 1 & 1 & \TTbf{0} & 1 & \TTbf{0}\\
 0 & 0 & 1 & 0 & \TTbf{0} & 0 & 0 & 0 & \TTbf{1} & 0 & \TTbf{0}
\end{tabular}
\end{center}

To determine whether the argument is valid, check to see whether there are any rows on which both premises have a $1$ under their main operators, but the conclusion has a $0$ under its main operator.
Notice that unlike the previous argument, there is a row in which both premises have a $1$ under their main operators.
Nevertheless, the only row in which both the premises have a $1$ under their main operators is the second row, and in that row the conclusion also has a $1$ under its main operator.
It follows that the argument is valid in $\PL$, though it is not vacuously valid as before.
Rather, vacuous validity is a special case.

Here is another example.
Is the following argument valid?
Try to check intuitively first.

\begin{earg}
  \eitem{$P \eif Q$}
  \uitem{$\enot P$ \quad}
  % \eitem{$\enot Q \eif (Q \eif Q)$}
  \eitem{$\enot Q$}
\end{earg}

% We may now evaluate the validity of the argument, by constructing a complete truth table: % and check whether there is a row that assigns 1 to the premises and 0 to the conclusion:

\begin{center}
\begin{tabular}{@{ }c@{ }@{ }c | c@{ }@{ }c@{ }@{ }c@{ }@{ }c@{ }@{ }c | c@{ }@{ }c | c@{ }@{ }c}
$P$ & $Q$ &  & $P$ & $\eif$ & $Q$ &  & $\enot$ & $P$ & $\enot$ & $Q$\\
\hline 
1 & 1 &  & 1 & \TTbf{1} & 1 &  & \TTbf{0} & 1 & \TTbf{0} & 1\\
1 & 0 &  & 1 & \TTbf{0} & 0 &  & \TTbf{0} & 1 & \TTbf{1} & 0\\
0 & 1 &  & 0 & \TTbf{1} & 1 &  & \TTbf{1} & 0 & \TTbf{0} & 1\\
0 & 0 &  & 0 & \TTbf{1} & 0 &  & \TTbf{1} & 0 & \TTbf{1} & 0\\
\end{tabular}
\end{center}

On the third row, each premise has a $1$ under its main operator but the conclusion does not, and so the argument is invalid: the conclusion is not a logical consequence of its premises.

% We may establish much the same by taking $\I$ to be an interpretation of $\PL$ which makes the premises true and the conclusion false. 
% Thus $\V{\I}(P \eif Q)=\V{\I}(\enot P)=1$ and $\V{\I}(\enot Q)=0$. 
% By the semantic clause for negation we know that $\V{\I}(P)=0$ and $\V{\I}(Q)=1$, and so $\I(P)=0$ and $\I(Q)=1$ by the semantic clause for sentence letters.
% Moreover, this is perfectly satisfiable with $\V{\I}(P \eif Q)=1$.
% Accordingly, any interpretation $\I$ in which $\I(P)=0$ and $\I(Q)=1$ where the truth-values of all other sentence letters are arbitrary counts as a counterexample to the validity of the argument above. 

% Taking the truth-values of the sentence letters that do not occur in a given argument to be arbitrary is akin to simply ignoring all other sentence letters as we do in constructing complete truth tables for arguments.
% Although the functional definition of an interpretation and valuation provides an elegant account of validity in $\PL$, 


\section{Decidability}%
  \label{sec:Decidability}
  
Evaluating wfss in $\PL$ by constructing complete truth tables provides a simple mechanical procedure that is straightforward to systematically employ.
Moreover, since every wfs of $\PL$ is of finite length and contains a finite number of sentence letters, constructing a complete truth table for a wfs of $\PL$ provides a finite procedure which determines whether that sentence is a tautology, contradiction, or logical contingent.
Put otherwise, constructing a complete truth table for a wfs of $\PL$ provides an \define{effective method} for determining whether that wfs is a tautology or not, and similarly for the other logical properties that sentence may or may not have.
Since there is an effective method for determining whether a wfs of $\PL$ is a tautology, we may say that it is \define{decidable} whether a wfs of $\PL$ is a tautology.
It is similarly decidable whether a wfs of $\PL$ is a contradiction or logically contingent.

Given a finite set $\Gamma$ of wfss of $\PL$, constructing a complete truth table for those wfss provides an effective method for determining whether $\Gamma$ is satisfiable or not, and so the question of whether $\Gamma$ is satisfiable is decidable. 
However, the same cannot be said for infinite sets of wfss of $\PL$.
Even though each wfs of $\PL$ is of finite length with finitely many sentence letters, an infinite set of wfss of $\PL$ may contain infinitely many sentence letters, requiring an infinitely large truth table.
Although there is nothing to prevent us from \textit{defining} infinitely large truth tables mathematically, there is of course little hope of \textit{using} such an infinite truth table to determine whether an infinite set of sentences is satisfiable.
It is for this reason that the truth table method does not provide an effective procedure for deciding whether a set of wfs of $\PL$ is satisfiable. 
Of course, there could be another effective method for determining whether a set of $\PL$ sentences is satisfiable, and so we cannot claim that it is \define{undecidable} whether a set of sentences is satisfiable just because one method cannot be used.

Although it is beyond the scope of this course, it is in fact undecidable whether an infinite set of wfs of $\PL$ is satisfiable or not.
That is, there is no effective procedure that we could hope to use to determine whether any infinite set of wfs of $\PL$ is satisfiable. %, though there may be particular infinite sets that we may are or are not satisfiable.
However, even in restricting consideration to finite sets, we may observe that it is often infeasible to construct truth tables for a wfs or set of wfss of $\PL$ which include too many sentence letters.
For instance, a wfs or set of wfss of $\PL$ which includes $5$ sentence letters would require $32$ lines, and double that again for $6$ sentence letters. 
At least for humans using pen and paper, this is at about the limit for what it is possible to use without making mistakes.

Here one might be tempted to respond by appealing to computers.
Instead of attempting to write out truth tables by hand, perhaps the method is best developed with computational assistance to avoid making mistakes.
This leads into the \textit{Boolean satisfiability problem} in computer science, and also falls outside the scope of this course.
Rather, we will be concerned with methods for working out reasoning on paper in a finite amount of time.
In later chapters, we will have reason to require these methods to simulate certain natural patterns of reasoning.
However, before then it will be important to clean up a few loose ends in order to set the stage for these developments.
We will begin by presenting a different procedure.




\section{Partial Truth Tables}

To show that a wfs of $\PL$ is a tautology, we need to show that $1$ occurs below its main operator on every row of its complete truth table.
So we need a complete truth table.
However, to show that a wfs is \textit{not} a tautology we only need to complete a row in which $0$ is beneath its main operator.
Therefore, in order to show that a wfs is not a tautology, it is enough to provide a \textit{partial truth table} regardless of how many sentence letters the wfs might include.

For example, consider $(U \eand T) \eif (S \eand W)$.
We want to show that it is \textit{not} a tautology by providing a partial truth table.
To do so, we begin by writing $0$ under the main operator which is a material conditional.
In order for the conditional to be false, there must be a $1$ under the antecedent and a $0$ under the consequent.
We fill these in as follows:

\begin{center}
\begin{tabular}{c|c|c|c|@{\TTon}*{7}{c}@{\TToff}}
$S$&$T$&$U$&$W$&$(U$&\eand&$T)$&\eif    &$(S$&\eand&$W)$\\
\hline
   &   &   &   &    &  1  &    &\TTbf{0}&    &   0 &   
\end{tabular}
\end{center}

In order for a $1$ to occur under $U\eand T$, a $1$ must also occur under both $U$ and $T$ as follows:

\begin{center}
\begin{tabular}{c|c|c|c|@{\TTon}*{7}{c}@{\TToff}}
$S$&$T$&$U$&$W$&$(U$&\eand&$T)$&\eif    &$(S$&\eand&$W)$\\
\hline
   & 1 & 1 &   &  1 &  1  & 1  &\TTbf{0}&    &   0 &   
\end{tabular}
\end{center}

Remember that each instance of a given sentence letter must have the same truth-value in a given row of a truth table.
You can't have $1$ occur under one instance of $U$ and $0$ occur under another instance of $U$ in the same row.
Thus we put a $1$ under each instance of $U$ and $T$.

Now we just need to work out what follows from the $0$ under $S\eand W$.
In particular, a $0$ must occur under either $S$ or $W$, or both.
Making an arbitrary decision, we may finish the table:

\begin{center}
\begin{tabular}{c|c|c|c|@{\TTon}*{7}{c}@{\TToff}}
$S$&$T$&$U$&$W$&$(U$&\eand&$T)$&\eif    &$(S$&\eand&$W)$\\
\hline
 0 & 1 & 1 & 0 &  1 &  1  & 1  &\TTbf{0}&  0 &   0 & 0  
\end{tabular}
\end{center}

Although showing that a wfs of $\PL$ is a tautology requires a complete truth table, showing that a wfs of $\PL$ is not a tautology only requires a partial truth table with a single row where $0$ occurs below the main operator of that wfs.
That's what we've just done.
In just the same way, to show that a wfs of $\PL$ is not a contradiction, you only need to construct a single row of a truth table where $1$ occurs below the main operator of that wfs.
By contrast, to show that a wfs of $\PL$ is a contradiction, you must show that a $0$ occurs below the main operator on \textit{every} row of a complete truth table, and so you need a complete truth table.


A wfs of $\PL$ is contingent just in case its complete truth table has a row in which $1$ occurs below the main operator and another row in which $0$ occurs below the main operator.
Thus to show that a wfs of $\PL$ is contingent requires a partial truth table with just two rows.
For example, we can show that the sentence above is contingent as follows:

\begin{center}
\begin{tabular}{c|c|c|c|@{\TTon}*{7}{c}@{\TToff}}
$S$&$T$&$U$&$W$&$(U$&\eand&$T)$&\eif    &$(S$&\eand&$W)$\\
\hline
 0 & 1 & 1 & 0 &  1 &  1  & 1  &\TTbf{0}&  0 &   0 & 0 \\
 0 & 1 & 0 & 0 &  0 &  0  & 1  &\TTbf{1}&  0 &   0 & 0
\end{tabular}
\end{center}

Just as there happens to be more than one combination of truth-values which makes the sentence false, there are even more ways to make the sentence true.
However, this is not always the case.
For instance, given a sentence letter $A$, there is exactly two lines in its complete truth table, one in which it is true, and the other in which it is false.

Showing that a wfs of $\PL$ is not contingent requires providing a complete truth table. %, because it requires showing that the sentence is a tautology or that it is a contradiction.
In particular, one must either show that the complete truth table for the wfs has a $1$ under its main operator on all rows, or show that the wfs has a $0$ under its main operator on all rows.
If you do not know whether a particular sentence is contingent or not, then you do not know whether you will need a complete or partial truth table.
One way to proceed is to start working on a complete truth table, stopping as soon as you complete rows that show the sentence is contingent.
If not, then you must complete the truth table in full.
% Even though two carefully selected rows will show that a contingent sentence is contingent, there is nothing wrong with filling in more rows.

Showing that two wfss of $\PL$ are logically equivalent requires providing a complete truth table.
By contrast, showing that two wfss of $\PL$ are not logically equivalent only requires a partial truth table with one row in which a $1$ occurs below the main operator of one of the wfss and a $0$ occurs below the main operator of the other wfs. 

Showing that a set of wfss of $\PL$ is satisfiable requires a single row of a truth table in which a $1$ occurs below the main operator of every wfs in the set.
% The rest of the table is irrelevant, so a one-line partial truth table will do.
However, to show that a set of wfss of $\PL$ is unsatisfiable requires a complete truth table since you must show that on every row of a complete truth table for the set, there is a $0$ below the main operator of at least one of the wfss in the set.
Of course, we could only hope to succeed if the set of wfss is finite.

Showing that an argument is valid requires a complete truth table.
Showing that an argument is invalid only requires providing a single row of a partial truth table in which a $1$ occurs below the main operator of every premise and a $0$ occurs below the conclusion.
Thus we have:

\begin{table}[ht]
\begin{center}
\begin{tabular}{|c|c|c|}
\cline{1-3}
& YES & NO\\
\cline{1-3}
Tautology & complete truth table & one-line partial truth table\\
Contradiction &  complete truth table  & one-line partial truth table\\
Contingent & two-line partial truth table & complete truth table\\
Entailment & complete truth table & one-line partial truth table\\
Equivalent & complete truth table & one-line partial truth table\\
Satisfiable & one-line partial truth table & complete truth table\\
Valid & complete truth table & one-line partial truth table\\
\cline{1-3}
\end{tabular}
\end{center}
% \caption{When to use a partial truth table}
% \label{table.CompleteVsPartial}
\end{table}

The table above summarizes when a complete truth table is required and when a partial truth table will suffice.
If you are trying to remember whether you need a complete truth table or not, the general rule is, if you're looking to establish a claim about \emph{every} interpretation, you need a complete table.
Otherwise, a one-line or perhaps two-line truth table may do instead.







\section{Semantics}%
  \label{sec:Semantics}

Although partial truth tables might help to avoid doing some amount of work, invariably you will end up needing to construct complete truth tables for sometimes long wfs of $\PL$ or else for arguments or sets which include many wfss of $\PL$.
In addition to being extremely tedious and time consuming, constructing large truth tables is also highly prone to human error.
These provide some initial reasons that one might hope to devise an alternative.

Complete truth tables also played an important role in defining what it is for a wfs of $\PL$ to be a tautology, where something similar may be said for the definitions of a contradiction as well as a logically contingent wfs of $\PL$. 
Similarly, the definitions for logically entailment, logical equivalence, satisfiability, logical consequence, and the validity of arguments given above all appealed to complete truth tables.
Whereas the corresponding definitions for the sentences of English could not be made precise given that there is no well-defined sense of what counts as a grammatical sentence of English, the sections above appealed to complete truth tables for the wfss of $\PL$ in order to avoid this problem. % for the logical properties which a wfs or set of wfss of $\PL$ may be said to h 
Nevertheless, the truth table definitions of the logical notions considered above still leave something to be desired.

So far we have identified the rows of a truth table for a relevant wfs or set of wfss of $\PL$ with the relevant range of interpretations for that wfs or set of wfss.
Of course, the truth tables that we can write down are are finite in size and so cannot specify truth-values for every sentence letter of $\PL$.
Rather, the rows of a truth table provide \define{partial interpretations} of $\PL$ by specifying all combinations of truth-values for some finite set of sentence letters.
Insofar as a truth table includes all of the sentence letters that occur in the wfs(s) in question, the truth-values for all other sentence letters do not make a difference, and so may be safely ignored.
Nevertheless, we cannot define the interpretations of $\PL$ as the rows of any finite truth table since no finite truth table specifies truth-values for every sentence letter.
This provides further motivation to present a more general approach.

Next we may consider the definition of a complete truth table itself.
Rather than providing a formal definition, we appealed to the table that results from writing the wfs(s) of $\PL$ in question at the top right of the table with all of the sentence letters it contains at the top left. 
We then provided a procedure for adding the appropriate number of rows depending on how many sentence letters were involved and distributing truth-values accordingly.
The rest of the values in the table were then to be added by appealing to the characteristic truth tables as a rubric.
% Although this does not interpret the language in full, we claimed that the resulting truth table exhausts the different distributions of truth-values that are relevant to the wfs(s) in question.
This left open certain ambiguities like the order of the sentence letters as well as the order of the wfss in a complete truth table for a set of wfss of $\PL$.
Additionally, at least given what we have said so far, one must rely on an intuitive grasp of the main operator for each subsentence, as well as how to correctly apply the characteristic truth tables.

Although with some ingenuity we could tighten up all of these details by either eliminating ambiguities or else establishing that the ambiguities do not make a difference, the definitions themselves are bound to become even more cumbersome to state precisely.
Rather, the truth table definitions given above are best understood as intuitive approximations of the precise definitions to which we will soon turn.
Despite their imprecision, the truth table definitions provide a natural and accessible account of the logical notions that we are after, and so may be preserved as helpful heuristics in contemplating the abstract definitions to follow.

% TODO mention that operators express truth functions in connection with above

Rather than relying on diagrams, officially we will take an \define{interpretation} of $\PL$ to be any function $\I$ from the set of sentence letters for $\PL$ to the set of truth-values $\set{1, 0}$, thereby assigning every sentence letter to exactly one truth-value.
Although interpretations only specify the truth-values of sentence letters, we may draw on any interpretation $\I$ of $\PL$ in order to define another function which assigns every wfs of $\PL$ to exactly one truth-value in accordance with the characteristic truth tables but without appealing to the characteristic truth tables.
More precisely, given any interpretation $\I$ of $\PL$, we may recursively define the \define{valuation function} $\V{\I}$ over the domain of wfss for $\PL$ by way of the following semantics: % for each of the sentential operators included in $\PL$:

% \factoidbox{
%   \textsc{Valuation Function:} $\V{\I}(\metaA)=1$ \textit{iff} either:
%   \begin{itemize}[leftmargin=.25in]
%     \item $\I(\metaA)=1$ (and so $\metaA$ is a sentence letter of $\PL$);
%     \item $\metaA = \enot\metaB$ where $\V{\I}(\metaB)=0$;
%     \item $\metaA = (\metaB \eor \metaC)$ where either $\V{\I}(\metaB)=1$ or $\V{\I}(\metaC)=1$ (or both);
%     \item $\metaA = (\metaB \eand \metaC)$ where both $\V{\I}(\metaB)=1$ and $\V{\I}(\metaC)=1$;
%     \item $\metaA = (\metaB \eif \metaC)$ where either $\V{\I}(\metaB)=0$ or $\V{\I}(\metaC)=1$ (or both);
%     \item $\metaA = (\metaB \eiff \metaC)$ where $\V{\I}(\metaB)=\V{\I}(\metaC)$;
%   \end{itemize}
% }

\factoidbox{
  \textsc{Valuation Function:} For any wfss $\metaA$ and $\metaB$ of $\PL$: 
  \begin{enumerate}[leftmargin=.25in]
    \item[] $\V{\I}(\metaA)=\I(\metaA)$ \textit{if} $\metaA$ is a sentence letter of $\PL$.
    \item[] $\V{\I}(\enot\metaA)=1$ \textit{iff} $\V{\I}(\metaA)=0$.
    \item[] $\V{\I}(\metaA \eor \metaB)=1$ \textit{iff} $\V{\I}(\metaA)=1$ or $\V{\I}(\metaB)=1$ (or both).
    \item[] $\V{\I}(\metaA \eand \metaB)=1$ \textit{iff} $\V{\I}(\metaA)=1$ and $\V{\I}(\metaB)=1$.
    \item[] $\V{\I}(\metaA \eif \metaB)=1$ \textit{iff} $\V{\I}(\metaA)=0$ or $\V{\I}(\metaB)=1$ (or both).
    \item[] $\V{\I}(\metaA \eiff \metaB)=1$ \textit{iff} $\V{\I}(\metaA)=\V{\I}(\metaB)$.
  \end{enumerate}
}

% As brought out above, a function is a set of ordered pairs.
% Accordingly, a function $g$ may be said to be \define{contained in} $f$ just in case $g \subseteq f$.
% The valuation function $\V{\I}$ is the \define{smallest function} to satisfy the constraints above insofar as it is contained in every function to 
% By requiring $\V{\I}$ to be the \textit{smallest} function to satisfy the above, we guarantee that there is `$\V{\I}$' refers to a unique 

% TODO 

The clauses above hold for all sentences $\metaA$ and $\metaB$ of $\PL$, thereby extending any interpretation $\I$ of $\PL$ to $\V{\I}$ in order to specify a unique truth-value for every wfs of $\PL$.
Whereas the characteristic truth tables for the operators specify the truth-values for complex sentences visually, the semantic clauses above specify the same information functionally.

In Chapter $\ref{ch.PL-syntax}$, we specified the primitive symbols for $\PL$.
These included the sentence letters, punctuation, and sentential operators. 
% Whereas the sentence letters make up the non-logical vocabulary of $\PL$ and are interpreted by truth-values, the sentential operators make up the logical vocabulary of $\PL$ and are assigned to truth-functions once and for all.\footnote{If you like, the punctuation can be understood as built into the operators, i.e., $\corner{(\underline{~~} \star \underline{~~})}$ where $\star$ is any binary sentential operator of $\PL$ and the blanks are where the arguments go.}
When interpreting $\PL$, you are not allowed to change the meaning of the sentential operators.
For instance, you cannot take the `$\enot$' symbol to mean what `$\eand$' usually does.
Rather, the `$\enot$' symbol will always have the same semantic clause, and so will always express the same truth-function for negation.
Since the meanings for the sentential operators are fixed by the semantic clauses given in the definition of the valuation function, it is common to refer to the sentential operators as \define{logical constants}.

The sentence letters are sometimes referred to as the \define{non-logical vocabulary} and are interpreted by assigning them to either $1$ or $0$, where it is the combination of assignments which may differ between interpretations of $\PL$.
% Formal languages are built from three kinds of elements: \define{logical symbols}, \define{non-logical symbols}, and \define{punctuation} i.e., the left and right parentheses.
% operators like `\eand' and `\eif' are logical symbols (also called \define{logical constants}) because their meaning is specified by semantic clauses which hold their meanings fixed.
% The sentence letters in $\PL$ are non-logical symbols because their meaning is not specified by the semantics for $\PL$.
Accordingly, when we translate an argument from English into $\PL$, the sentence letter `$M$' does not have its meaning fixed as a result.
Rather, we rely on interpretations to assign truth-values to sentence letter such as `$M$', where the truth-values provide a maximally course-grained way to model what those sentence letters mean, i.e., whether they express a proposition that obtains or does not obtain.


% TODO turn next two sections into a formal revision of the truth table definitions


\section{Formal Definitions}
  \label{sec:Definitions}

% Instead of defining what it is for a wfs of $\PL$ to be a tautology, contradiction, or contingent in terms 
Having provided a definition of the interpretations of $\PL$ that is both mathematically precise and simple to state, we may now put this definition to work to redefine the logical properties and relations discussed above.
Letting $\metaA$ and $\metaB$ be wfs of $\PL$, $\MetaG$ be a set of wfss of $\PL$, and $\I$ and $\J$ be interpretations of $\PL$, we may present the following official definitions: 
% In particular, consider the following:

\factoidbox{
  \begin{itemize}[leftmargin=0in]
    \item[] \define{tautology}: $\metaA$ is a \textit{tautology iff} $\V{\I}(\metaA) = 1$ for all $\I$.
    \item[] \define{contradiction}: $\metaA$ is a \textit{contradiction iff} $\V{\I}(\metaA) = 0$ for all $\I$.
    \item[] \define{contingent}: $\metaA$ is \textit{logically contingent iff} $\V{\I}(\metaA) \neq \V{\J}(\metaA)$ for some $\I$ and $\J$.
    \item[] \define{entailment}: $\metaA$ \textit{logically entails} $\metaB$ \textit{iff} $\V{\I}(\metaA) \leq \V{\I}(\metaB)$ for all $\I$.
    \item[] \define{equivalence}: $\metaA$ is \textit{logically equivalent} to $\metaB$ \textit{iff} $\V{\I}(\metaA) = \V{\I}(\metaB)$ for all $\I$.
    \item[] \define{satisfiable}: $\Gamma$ is \textit{satisfiable iff} there is some $\I$ where $\V{\I}(\metaG) = 1$ for all $\metaG \in \Gamma$.
    \item[] \define{consequence}: $\MetaG \vDash \metaA$ \textit{iff} $\V{\I}(\metaA) = 1$ for all $\I$ where $\V{\I}(\metaG) = 1$ for all $\metaG \in \Gamma$.
  \end{itemize}
}

% An argument in $\PL$ is \define{logically valid} just in case for every interpretation $\I$ of $\PL$, if the valuation $\V{\I}$ makes the premises true (i.e., $\V{\I}(\metaA)=1$ for every premise $\metaA$), then $\V{\I}$ also makes the conclusion true (i.e., $\V{\I}(\metaB)=1$ where $\metaB$ is the conclusion).
% Alternatively, we could take an argument in $\PL$ to be \define{logically valid} just in case every line in a complete truth table which includes every sentence in the argument is such that if the premises are true on that line, the conclusion is also true on that line.
% Even though the lines of a truth table do not specify truth-values for all sentence letters of $\PL$, only the sentence letters that occur in the argument are relevant, and so the truth-values of all other sentence letters may be safely ignored.
% Accordingly, these two definitions are equivalent.
% Nevertheless, it is definitionally simpler to quantify over all interpretations $\I$ of $\PL$ rather than the lines of an appropriately constructed complete truth table for the argument in question.

% A set of $\PL$ sentences $\Gamma$ \define{logically entails} $\metaA$ just in case every interpretation that satisfies $\Gamma$ also satisfies $\metaA$.
% Logical entailment is also called the \define{logical consequence} relation, or \define{semantic consequence}, and is written: $\Gamma \vDash \metaA$. 
% An entire chapter is named after this relation for good reason: 

Note that the double turnstile `$\vDash$' has been introduced for the logical consequence relation which--- like the schematic variable `$\metaA$'--- is part of the \emph{metalanguage} that we are using to discuss $\PL$, and not a part of $\PL$ itself.
Although officially $\vDash$ takes a set of $\PL$ wfss on the left together with a single $\PL$ wfs on the right, it is both common and convenient to drop the set notation, writing `$\metaA_1,\ldots,\metaA_n \vDash \metaB$' instead of `$\set{\metaA_1,\ldots,\metaA_n} \vDash \metaB$'.
% Even though entailment is connected in important ways to the validity of arguments, it is important to remember to apply the definitions rigorously and precisely.

As before, we may say that an argument in $\PL$ is \define{valid} just in case it's conclusion is a logical consequence of its set of premises.
Recall that an argument in $\PL$ is a sequence of $\PL$ wfs, and so a completely different type of thing than a set of wfs of $\PL$.
After all, sets do not specify the order of their members.
Accordingly, we cannot say that for any set of $\PL$ wfss $\Gamma$ and wfs $\metaA$, if $\Gamma \vDash \metaA$, then \textit{the} argument whose premises are the wfss in $\Gamma$ and whose conclusion is $\metaA$ is valid.
This is because there may fail to be a unique argument that we can construct from a set of $\PL$ wfss $\Gamma$ and a further $\PL$ wfs $\metaA$.
For instance, assuming $\Gamma$ includes at least two wfss, we can construct different arguments by reversing the order of the premises.
Instead, we may claim something more general: $\Gamma \vDash \metaA$ just in case \textit{every} argument where the wfss in $\Gamma$ are the premises (in some order or other) and $\metaA$ is the conclusion is valid.

% It will help to consider a simple example.
% For instance, take the following claim: 
%   $$P, Q \vDash P\eor Q.$$ 
% The logical consequence above is true since for every interpretation $\I$ of $\PL$, if $\V{\I}(P) = 1$ and $\V{\I}(Q) = 1$, then $\V{\I}(P \eor Q) = 1$.
% Accordingly, we may say that the argument $P, Q\ \therefore P \eor Q$ which has $P$ as its first premise, $Q$ as its second premise, and $P\eand Q$ as its conclusion is logically valid. 
% Although closely related, this is different claim from the logical consequence stated above. 
% % Rather, it is a claim about an argument.
% Moreover, there is a distinct argument which takes $Q$ to be its first premise, $P$ as its second premise, and $P\eand Q$ as its conclusion--- i.e., $Q, P\ \therefore P \eor Q$--- though this second argument is also valid.
% By contrast, the $P$ and $Q$ in the logical consequence claim do not have an order, though we happened to write $P$ before $Q$.
% % This is because sets do not preserve order: $\set{P,Q}=\set{Q,P}$.
% On the other hand, permuting $P$ and $Q$ in $P \eor Q$ results in a sentence that does not occur above, namely $Q \eor P$, and so this would be a distinct logical consequence from the claim above.

Logical consequence is the most important semantic concept that we will study in this course.
In the following chapter, we will introduce a proof theoretic analogue for derivability which we will represent with the single turnstile `$\vdash$'.
As we will then go on to show in the metalogical portions of this book, these two relations have the same extension, providing two radically different perspectives on one and the same thing, i.e., formal reasoning.
We will then repeat this same methodology for languages with greater expressive power.
For the time being, we only pause to indicate the central role that logical consequence will play throughout this text.



% \subsection{Satisfiability}
% \label{sub:PL-Satisfiability}
%
% % TODO define satisfaction and characterize logical consequence in these terms
%
% Recall that an \define{interpretation} is any function which assigns a truth-value to every sentence letter of $\PL$.
% We will also refer to interpretations of $\PL$ as \define{models} of $\PL$ since interpretations provide an extensional way to represent the meanings of the sentence letters in $\PL$.\footnote{Intensional and hyperintensional semantic frameworks provide much more fine-grained representations of meanings, but this doesn't make extensional frameworks irrelevant. After all, first-order set theory is an extensional theory which many take to provide a foundation for mathematics.}
%
% Whereas a sentence $\metaA$ may be either true of false on a given interpretation $\I$ (i.e., the valuation function $\V{\I}$ induced by $\I$ assigns that sentence to $1$), sets of sentences do not have truth-values.
% Accordingly, we may say that an $\PL$ interpretation $\I$ \define{satisfies} a set of $\PL$ sentences $\Gamma=\set{\metaA_{1},\metaA_{2},\ldots}$ just in case $\V{\I}(\metaA)=1$ for all $\metaA \in \Gamma$.
% Derivatively, it is convenient to say that an interpretation $\I$ of $\PL$ \define{satisfies} a single sentence $\metaA$ just in case it satisfies the singleton set $\set{\metaA}$, i.e., the set which includes $\metaA$ as its only member. 
%
% % TODO define tautology and contradiction again in terms of satisfiability




\section{Semantic Proofs}
  \label{sec:SemanticProofs}

The formal definitions given above are certainly much more elegant than the truth table definitions for the corresponding terms, avoiding all the ambiguities that we mentioned but did not take the time to fully resolve above.
Although these new definitions do not provide effective methods for deciding logical questions in the same way as the truth table definitions, they will nevertheless allow us to prove things about the wfss of $\PL$ and their various logical properties and relationships.
For instance, consider the following set of wfss:
$$ \MetaG = \set{A \eand  C,\ B \eiff \enot A,\ C \eif (B \eand D),\ E} $$
It turns out that this set is unsatisfiable.
However, to show this using a truth table would requiring constructing a complete truth table with five sentence letters and thirty-two rows.
In addition to the tedium, the chances of making a mistake cannot be overlooked.
Instead of attempting this, we can prove that $\MetaG$ is unsatisfiable by assuming that it is satisfiable and appealing to the formal definitions in order to derive a contradiction: 

\begin{quote} 
  \textit{Proof:} Assume for contradiction that the set $\MetaG$ is satisfiable.
  By the definition of satisfiability, $\V{\I}(A \eand  C) = \V{\I}(B \eiff \enot A) = \V{\I}(C \eif (B \eand D)) = \V{\I}(E) = 1$ for some interpretation $\I$ of $\PL$. 
  It follows that $\V{\I}(A) = \V{\I}(C) = 1$ by the semantics for conjunction, and so $\V{\I}(\enot A) = 0$ given the semantics for negation. 
  Since $\V{\I}(B) = \V{\I}(\enot A)$ by the semantics for the biconditional, $\V{\I}(B) = 0$. 

  Since $\V{\I}(C \eif (B \eand D)) = 1$, we know by the semantics for the conditional that either $\V{\I}(C) = 0$ or $\V{\I}(B \eand D) = 1$.
  Having already shown that $\V{\I}(C) \neq 0$, we may conclude that $\V{\I}(B \eand D) = 1$, and so $\V{\I}(B) = 1$ and $\V{\I}(D) = 1$ by the semantics for conjunction.
  Thus $\V{\I}(B) \neq 0$, contradicting the above.
  \qed
\end{quote}

This proof has been made fully explicit so as to make it easy to follow.
A proof that is hard to follow is not a very good proof.
However, using too many words is not a good thing either, cluttering what might be easier to see otherwise.
Writing clear and readable proofs is a skill that requires judgment and practice.
If you are new to proof writing, it is best to begin by making everything explicit before tightening things up to write with concision.

Note that the proof stopped once we got a contradiction.
Although we might have continued by saying that our assumption must be false given the contradiction that we derived, this much is automatic given the introductory clause ``Assume for contradiction\ldots'' which sets up this expectation.
This is a good example of the kinds of writing strategies that you can use to write \define{informal proofs}.
By contrast to the formal proofs that we will go on to write in the following chapter, informal proofs are written in the metalanguage mathematical English and are often about wfss of our object language $\PL$.
In the case above, we established that $\MetaG$ is unsatisfiable. 
% In the following chapter we will introduce a proof system in which to write formal proofs in $\PL$ itself. 
In order to give you a sense of some of the other methods and phrasing for writing clear and concise informal proofs, it will help to consider a few more examples.


\begin{earg}
  \eitem{Either the butler is the murderer or the gardener isn't who he says he is. }
  \uitem{The gardener is who he says he is.}
  \eitem{The butler is the murderer.}
\end{earg}

In order to determine its validity, let's translate this argument into $\PL$ and evaluate the resulting formal argument for validity with a truth table.
In particular:

\begin{ekey}
  \item[B:] The butler is the murderer.
  \item[G:] The gardener is who he says he is.
\end{ekey}

\begin{earg}
  \eitem{$B \eor \enot G$}
  \uitem{$G$ \quad\quad }
  \eitem{$B$}
\end{earg}

It should be pretty clear that this is a valid argument, but to show this we may now write a semantic proof which establishes its validity.
Whereas the proof given above proceeded by \textit{reductio ad absurdum}, deriving a contradiction from the negation of what we wanted to prove, it is perhaps easiest to write a direct proof that the argument above is valid.

\begin{quote} 
  \textit{Proof:}
  Let $\I$ be an arbitrary interpretation of $\PL$ for which both of the premises are true, i.e., $\V{\I}(B \eor \enot G) = \V{\I}(G) = 1$. 
  We know by the semantics for negation that $\V{\I}(\enot G) = 0$, where either $\V{\I}(B) = 1$ or $\V{\I}(\enot G) = 1$ by the semantics for disjunction, and so may conclude that $\V{\I}(B) = 1$.
  Since $\I$ was arbitrary, we may conclude more generally that $\V{\I}(B) = 1$ for any $\I$ where $\V{\I}(B \eor \enot G) = \V{\I}(G) = 1$, and so the argument is indeed valid.
  \qed
\end{quote}

It is typical to leave this final sentence off assuming that it is sufficiently clear what you are setting out to prove and how you are intended to do so.
Nevertheless, it doesn't hurt to include this extra line to make it especially clear in case you are uncertain whether your proof is easy to interpret.
In the case above, the key signpost that we used was the construction `\textit{Let} $\I$ be an \textit{arbitrary} interpretation of $\PL$ \ldots' since this signals that we will establish something general about $\PL$ interpretations.
We then restrict consideration to those interpretations $\I$ of $\PL$ which make the premises true with `for which both of the premises are true', checking to see that such an arbitrary interpretation makes the conclusion true.

Whereas the first semantic argument that we gave allowed us to show that $\MetaG$ is unsatisfiable without having to draw a truth table with thirty-two rows, the winnings in the case above are somewhat less substantial. 
In particular, the truth table only requires that we include four rows since there are just two sentence letters.
% We'll set up a table with the two sentence letters included above, and check to see whether there is a row that assigns `1' to both premises and assigns `0' to the conclusion.
Thus we have:

\begin{center}
  \begin{tabular}{c|c|@{\TTon}*{4}{c}@{\TToff}|@{\TTon}c@{\TToff}|@{\TTon}c@{\TToff}}
    $B$&$G$&$(B$&\eor&\enot&$G$)&$G$&$B$\\
    \hline
    1 & 1 & 1 & \TTbf{1} & 0 & 1 & \TTbf{1} & \TTbf{1}\\
    1 & 0 & 1 & \TTbf{1} & 1 & 0 & \TTbf{0} & \TTbf{1}\\
    0 & 1 & 0 & \TTbf{0} & 0 & 1 & \TTbf{1} & \TTbf{0}\\
    0 & 0 & 0 & \TTbf{1} & 1 & 0 & \TTbf{0} & \TTbf{0}\\
  \end{tabular}
\end{center}

If the argument were invalid, there would be a row on which the first two bold values are $1$ but the third is $0$.
There is no such row, so the argument is valid.
Put otherwise, every row in which the premises have a $1$ under their main operators is a row in which the conclusion has a $1$ under its main operator. 
% Just as in the case of our semantic proof, we may 
Thus the argument in $\PL$ is valid. 

In this case, both the semantic proof and truth table methods require a similar amount of work.
Given either method, we may conclude that the argument in $\PL$ is valid. 
We may then go on to explain that the English argument is valid by appealing to the fact that the English argument can be regimented by a valid argument in $\PL$.

Before pressing on, it will be important to consider one more type of semantic proof which appeals to cases.
Consider the following argument:

\begin{earg}
  \eitem{Either Kat or Lu will with the race.}
  \eitem{If Kat wins, then she will celebrate.}
  \uitem{If Lu wins, then she will celebrate.}
  \eitem{Either Kat or Lu will celebrate.}
\end{earg}

This argument can be regimented as follows:

\begin{ekey}
  \item[$K$:] Kat will win.
  \item[$L$:] Lu will win.
  \item[$C_1$:] Kat will celebrate.
  \item[$C_2$:] Lu will celebrate.
\end{ekey}

\begin{earg}
  \eitem{$K \eor L$}
  \eitem{$K \eif C_1$}
  \uitem{$L \eif C_2$}
  \eitem{$C_1 \eor C_2$}
\end{earg}

This argument may seem plain enough since this is just the kind of day-to-day reasoning that we are all accustomed to doing.
Nevertheless, providing a truth table would require sixteen rows.
Although the semantic proof is easier than attempting to fill out so many rows without making any mistakes, there is no way to avoid a \textit{proof by cases}, at least insofar as we are to provide a direct proof that the $\PL$ argument is valid.

Writing with slightly more concision than before, we may reason as follows:

\begin{quote} 
  \textit{Proof:}
  Let $\I$ be an arbitrary interpretation of $\PL$ where:
    $$\V{\I}(K \eor L) = \V{\I}(K \eif C_1) = \V{\I}(L \eif C_2) = 1.$$ 
  By the semantics for disjunction, either $\V{\I}(K) = 1$ or $\V{\I}(L) = 1$.
  Consider:
  
  \textit{Case 1:}
  Assume $\V{\I}(K) = 1$.
  By the semantics for the conditional, it follows from the above that either $\V{\I}(K) = 0$ or $\V{\I}(C_1) = 1$, and so $\V{\I}(C_1) = 1$.
  Thus $\V{\I}(C_1 \eor C_2) = 1$ follows from the semantics for disjunction.

  \textit{Case 2:}
  Assume $\V{\I}(L) = 1$.
  By the semantics for the conditional, it follows from the above that either $\V{\I}(L) = 0$ or $\V{\I}(C_2) = 1$, and so $\V{\I}(C_1) = 1$.
  Thus $\V{\I}(C_1 \eor C_2) = 1$ follows from the semantics for disjunction.

  Thus $\V{\I}(C_1 \eor C_2) = 1$ in both cases. 
  Since $\I$ was arbitrary, the conclusion is true on any interpretation where the premises true, and so the argument is valid. 
  \qed
\end{quote}

Although the method of proof by cases is an essential tool to have in your toolkit, these proofs are often harder to read and can also be harder to write clearly.
Accordingly, proofs by cases are to be avoided whenever possible.
In the case above, we could have avoided introducing cases by using a \textit{reductio} argument (short for \textit{reductio ad absurdum}) instead.

\begin{quote} 
  \textit{Proof:}
  Assume for contradiction that the argument is invalid.
  Thus the conclusion is not a logical consequence of the premises, and so there is some interpretation $\I$ of $\PL$ where $\V{\I}(K \eor L) = \V{\I}(K \eif C_1) = \V{\I}(L \eif C_2) = 1$ and $\V{\I}(C_1 \eor C_2) = 0$. 
  By the semantics for disjunction, both $\V{\I}(C_1) = \V{\I}(C_2) = 0$.  

  By the semantics for the material conditional, either $\V{\I}(K) = 0$ or $\V{\I}(C_1) = 1$, and similarly, either $\V{\I}(L) = 0$ or $\V{\I}(C_2) = 1$.
  Given the above, $\V{\I}(K) = \V{\I}(L) = 0$, and so $\V{\I}(K \eor L) = 0$ by the semantics for disjunction, contradicting the above. 
  \qed
\end{quote}

This proof doesn't worry about cases and in that way is a bit easier to follow.
Nevertheless, direct proofs are often preferable to \textit{reductio} style proofs since deriving a contradiction is not quite as explanatory as what a direct proof shows.
Which proof strategy to use is a judgment call and it can take some practice to know which approach is best to use when.
Even once you have chosen a basic strategy, the way that you order the steps of your proof can also have a significant effect on the clarity of your resulting proof.
These are points that are worth considering as you practice writing clear and concise proofs in this course.

So far, we have only provided semantic proofs for claims that would have required a complete truth table.
This is no accident.
For instance, in the case where a set of $\PL$ wfss is satisfiable, or an argument $\PL$ is valid, all we need to do is find a particular interpretation of the relevant sentence letters of $\PL$ in order to draw the desired conclusion. 
Here we may observe that providing a partial truth table does just that, and so is often to be preferred.
We may then go on to record a particular row of a truth table by using the notation $\I(\metaA)$ to specify a truth value for each sentence letter $\metaA$ that occurs in the truth table.

% % TODO: SAVE?
% There are at least two advantages to evaluating a natural language argument for validity by translating it into $\PL$.
% For one thing, using truth tables to evaluate arguments is a formal method that does not require any particular rational insight or intuition.
% Although it is sometimes easy to tell whether an argument in English is valid, there are many complex arguments where it is next to impossible.
% It is an advantage to have a mechanical procedure by which to check an argument for validity that is easy to carry out without running into errors.
% A second advantage, alluded to in Chapter \ref{ch.PL-syntax}, is that regimenting English arguments in $\PL$ can help to identify a common logical form.
%
% As we have already seen, sentences and arguments in English can be regimented in many different ways where some regimentations are better than others.
% Accordingly, we cannot conclude that an English argument is invalid just because it has an invalid regimentation in $\PL$ since there could be a better regimentation that is valid.
% Even when an English argument does not have any valid regimentation in $\PL$, this does not mean that it could not have a valid regimentation in some expressively more powerful language like $\FOL$ which we will study later on in this course.
% For the time being, we will focus on regimenting and evaluating the validity of arguments in English with the resources of $\PL$.
%
% Arguments in English have an intended interpretation insofar as they are meaningful to us in the form they are presented purely in virtue of our understanding of English.
% The same cannot be said for $\PL$ which includes such sentences as $A$, $\enot (B \eor \enot C)$, $A \eif \enot (B \eif C)$, etc., which do not \textit{already} mean something to us.
% This gap is filled by: (1) providing a symbolization key which specifies the English sentences that some sentences in $\PL$ are intended to regiment; and (2) interpreting the relevant sentences of $\PL$ by specifying truth-values for the sentence letters that they contain.
% If an argument in English can be naturally regimented by an $\PL$ argument that is valid, we have good reason to take the argument in English to be valid.

% Any argument which has this logical form is valid.
% More generally, we may say that:
%
% \factoidbox{
% Any English argument that can be regimented by a valid $\PL$ argument is valid.
% }
%
% What we have not commented on is what does and does not count as a proper regimentation, where one might hope to specify this as well so as to remove all indeterminacy.
% To do so, one would have to be able to specify the logical form of English sentences and the arguments that they form.
% However, we cannot even manage to provide an account of which expressions count as English sentences, much less agree about their true logical form.
% Despite this shortcoming, it is often possible to provide natural regimentations of English arguments in $\PL$, where doing so is a matter of practice and judgement rather than the mechanical application of some broader theory.
% Even for complex arguments, it is sometimes easy to provide a valid regimentation in $\PL$ or some other formal language.
% For instance, consider the following case:
%
% \begin{earg}
%   \item Either Canada is a democracy if and only if Poland is neither part of the European Union nor majority Catholic, or my dog is not not lazy.
%   \item My dog is not lazy.
%   \item[\therefore] Canada is a democracy if and only if Poland is neither part of the European Union nor majority Catholic.
% \end{earg}
%
% One could translate this argument by identifying atomic sentences, symbolising them by sentence letters, and building up the complex sentences in these terms.
% In this instance, however, that extra structure isn't needed to capture the validity of the argument.
% Instead, we can use this symbolization key:
%
% \begin{ekey}
% \item[A:] Canada is a democracy if and only if Poland is neither part of the European Union nor majority Catholic.
% \item[L:] My dog is not lazy.
% \end{ekey}
%
% For many purposes, this would be a poor choice of symbolization keys.
% It ignores the internal structure of two central sentences.
% But if, as here, that structure is irrelevant, you can save yourself some work by engaging at a higher level of abstraction.
%

%%%

% \iffalse
%
% \practiceproblems
% If you want additional practice, you can construct truth tables for any of the sentences and arguments in the exercises for the previous chapter.
%
%
%
% \problempart
% \label{HW2.E}
% Provide the truth table for the complex formula:
% $$((P \eif (( P \eif Q) \eiff (P \eand \enot R))) \eor R)$$
% Indicate whether the formula is tautological, contradictory, or contingent.
% If it is contingent, provide an interpretation in which it is true as well as an interpretation in which it is false.
%
%
% \begin{tabular}{@{ }c@{ }@{ }c@{ }@{ }c | c@{ }@{}c@{}@{ }c@{ }@{ }c@{ }@{}c@{}@{}c@{}@{ }c@{ }@{ }c@{ }@{ }c@{ }@{}c@{}@{ }c@{ }@{}c@{}@{ }c@{ }@{ }c@{ }@{ }c@{ }@{ }c@{ }@{}c@{}@{}c@{}@{}c@{}@{ }c@{ }@{ }c@{ }@{ }c}
% $P$ & $Q$ & $R$ &  & ( & $P$ & $\eif $ & ( & ( & $P$ & $\eif $ & $Q$ & ) & $\eiff $ & ( & $P$ & $\&$ & $\enot$ & $R$ & ) & ) & ) & $\lor$ & $R$ & \\
% \hline 
%  &  &  &  &  &  &  &  &  &  &  &  &  &  &  &  &  &  &  &  &  &  &  & & \\
%  &  &  &  &  &  &  &  &  &  &  &  &  &  &  &  &  &  &  &  &  &  &  & & \\
%   &  &  &  &  &  &  &  &  &  &  &  &  &  &  &  &  &  &  &  &  &  &  & & \\
%  &  &  &  &  &  &  &  &  &  &  &  &  &  &  &  &  &  &  &  &  &  &  & & \\
%  &  &  &  &  &  &  &  &  &  &  &  &  &  &  &  &  &  &  &  &  &  &  & & \\
%   &  &  &  &  &  &  &  &  &  &  &  &  &  &  &  &  &  &  &  &  &  &  & & \\
%  &  &  &  &  &  &  &  &  &  &  &  &  &  &  &  &  &  &  &  &  &  &  & & \\
%   &  &  &  &  &  &  &  &  &  &  &  &  &  &  &  &  &  &  &  &  &  &  & & \\
% \end{tabular}
%
%
%
%
% \problempart
% \label{HW3.A}
% Provide the complete truth table for this $\PL$ sentence:
% $$((P \eor Q) \eand (\enot P \eor \enot Q)) \eiff R$$
% Indicate whether it is tautological, contradictory, or contingent. 
% If it is contingent, provide an interpretation in which it is true as well as an interpretation in which it is false.
%
%
%
% \solutions
% \problempart
% \label{pr.TT.TTorC}
% Determine whether each sentence is a tautology, a contradiction, or a contingent sentence. Justify your answer with a complete or partial truth table where appropriate.
% \begin{earg}
% \item $A \eif A$ %taut
% \item $\enot B \eand B$ %contra
% \item $C \eif\enot C$ %contingent
% \item $\enot D \eor D$ %taut
% \item $(A \eiff B) \eiff \enot(A\eiff \enot B)$ %tautology
% \item $(A\eand B) \eor (B\eand A)$ %contingent
% \item $(A \eif B) \eor (B \eif A)$ % taut
% \item $\enot[A \eif (B \eif A)]$ %contra
% \item $(A \eand B) \eif (B \eor A)$  %taut
% \item $A \eiff [A \eif (B \eand \enot B)]$ %contra
% \item $\enot(A \eor B) \eiff (\enot A \eand \enot B)$ %taut
% \item $\enot(A\eand B) \eiff A$ %contingent
% \item $\bigl[(A\eand B) \eand\enot(A\eand B)\bigr] \eand C$ %contradiction
% \item $A\eif(B\eor C)$ %contingent
% \item $[(A \eand B) \eand C] \eif B$ %taut
% \item $(A \eand\enot A) \eif (B \eor C)$ %tautology
% \item $\enot\bigl[(C\eor A) \eor B\bigr]$ %contingent
% \item $(B\eand D) \eiff [A \eiff(A \eor C)]$%contingent
% \end{earg}
%
%
% % Chapter 3 Part D
% \solutions
% \problempart
% \label{pr.TT.equiv}
% Determine whether each pair of sentences is logically equivalent. Justify your answer with a complete or partial truth table where appropriate.
% \begin{earg}
% \item $A$, $\enot A$ %No
% \item $A$, $A \eor A$ %Yes
% \item $A\eif A$, $A \eiff A$ %No
% \item $A \eor \enot B$, $A\eif B$ %No
% \item $A \eand \enot A$, $\enot B \eiff B$ %Yes
% \item $\enot(A \eand B)$, $\enot A \eor \enot B$ %Yes
% \item $\enot(A \eif B)$, $\enot A \eif \enot B$ %No
% \item $(A \eif B)$, $(\enot B \eif \enot A)$ %Yes
% \item $[(A \eor B) \eor C]$, $[A \eor (B \eor C)]$ %Yes
% \item $[(A \eor B) \eand C]$, $[A \eor (B \eand C)]$ %No
% \end{earg}
%
% % Chapter 3 Part E
% \solutions
% \problempart
% \label{pr.TT.satisfiable}
% Determine whether each set of sentences is satisfiable or unsatisfiable. Justify your answer with a complete or partial truth table where appropriate.
% \begin{earg}
% \item $A\eif A$, $\enot A \eif \enot A$, $A\eand A$, $A\eor A$ %satisfiable
% \item $A \eand B$, $C\eif \enot B$, $C$ %unsatisfiable
% \item $A\eor B$, $A\eif C$, $B\eif C$ %satisfiable
% \item $A\eif B$, $B\eif C$, $A$, $\enot C$ %unsatisfiable
% \item $B\eand(C\eor A)$, $A\eif B$, $\enot(B\eor C)$  %unsatisfiable
% \item $A \eor B$, $B\eor C$, $C\eif \enot A$ %satisfiable
% \item $A\eiff(B\eor C)$, $C\eif \enot A$, $A\eif \enot B$ %satisfiable
% \item $A$, $B$, $C$, $\enot D$, $\enot E$, $F$ %satisfiable
% \end{earg}
%
%
%
%
% \problempart
% \label{HW3.B}
%
% Use a complete truth table to evaluate this argument form for validity:
%
% \begin{earg}
% \item[] $(P \eif Q) \eor (Q \eif P)$
% \item[] $P$
% \item[\therefore] $P\eiff Q$
% \end{earg}
%
% Indicate whether it valid or invalid.
% If it is invalid, provide an interpretation in which the premises are true and the conclusion is false.
%
%
%
%
% \solutions
% \problempart
% \label{pr.TT.valid}
% Determine whether each argument form is valid or invalid. Justify your answer with a complete or partial truth table where appropriate.
% \begin{earg}
% \item $A\eif A$ \therefore\ $A$ %invalid
% \item $A\eor\bigl[A\eif(A\eiff A)\bigr]$ \therefore\ A %invalid
% \item $A\eif(A\eand\enot A)$ \therefore\ $\enot A$ %valid
% \item $A\eiff\enot(B\eiff A)$ \therefore\ $A$ %invalid
% \item $A\eor(B\eif A)$ \therefore\ $\enot A \eif \enot B$ %valid
% \item $A\eif B$, $B$ \therefore\ $A$ %invalid
% \item $A\eor B$, $B\eor C$, $\enot A$ \therefore\ $B \eand C$ %invalid
% \item $A\eor B$, $B\eor C$, $\enot B$ \therefore\ $A \eand C$ %valid
% \item $(B\eand A)\eif C$, $(C\eand A)\eif B$ \therefore\ $(C\eand B)\eif A$ %invalid
% \item $A\eiff B$, $B\eiff C$ \therefore\ $A\eiff C$ %valid
% \end{earg}
%
% \solutions
% \problempart
% \label{pr.TT.concepts}
% Answer each of the questions below and justify your answer.
% \begin{earg}
% \item Suppose that \metaA and \metaB are logically equivalent for some sentences of $\PL$ $\metaA$ and $\metaB$. What can you say about $\metaA\eiff\metaB$?
% %\metaA and \metaB have the same truth-value on every line of a complete truth table, so $\metaA\eiff\metaB$ is true on every line. It is a tautology.
% \item Suppose that $(\metaA\eand\metaB)\eif\metaC$ is contingent for some sentences of $\PL$ $\metaA, \metaB,$ and $\metaC$. What can you say about the argument ``\metaA, \metaB, \therefore\metaC''?
% %The sentence is false on some line of a complete truth table. On that line, \metaA and \metaB are true and \metaC is false. So the argument is invalid.
% \item Suppose that $\{\metaA,\metaB, \metaC\}$ is unsatisfiable for some sentence of $\PL$ $\metaA, \metaB,$ and $\metaC$. What can you say about $(\metaA\eand\metaB\eand\metaC)$?
% %Since there is no line of a complete truth table on which all three sentences are true, the conjunction is false on every line. So it is a contradiction.
% \item Suppose that \metaA is a contradiction for some sentence of $\PL$ $\metaA$. What can you say about the argument ``\metaA, \metaB, \therefore\metaC''?
% %Since \metaA is false on every line of a complete truth table, there is no line on which \metaA and \metaB are true and \metaC is false. So the argument is valid.
% \item Suppose that \metaC is a tautology for some sentence of $\PL$ $\metaC$. What can you say about the argument ``\metaA, \metaB, \therefore\metaC''?
% %Since \metaC is true on every line of a complete truth table, there is no line on which \metaA and \metaB are true and \metaC is false. So the argument is valid.
%   \item Suppose that \metaA and \metaB are logically equivalent for some sentences of $\PL$ $\metaA$ and $\metaB$. What can you say about $(\metaA\eor\metaB)$?
% %Not much. $(\metaA\eor\metaB)$ is a tautology if \metaA and \metaB are tautologies; it is a contradiction if they are contradictions; it is contingent if they are contingent.
% \item Suppose that \metaA and \metaB are \emph{not} logically equivalent for some sentences of $\PL$ $\metaA$ and $\metaB$. What can you say about $(\metaA\eor\metaB)$?
% %\metaA and \metaB have different truth-values on at least one line of a complete truth table, and $(\metaA\eor\metaB)$ will be true on that line. On other lines, it might be true or false. So $(\metaA\eor\metaB)$ is either a tautology or it is contingent; it is \emph{not} a contradiction.
% \end{earg}
%
% \problempart
% \phantomsection\label{pr.altoperators}
% We could leave the biconditional (\eiff) out of the language. If we did that, we could still write `$A\eiff B$' so as to make sentences easier to read, but that would be shorthand for $(A\eif B) \eand (B\eif A)$. The resulting language would be formally equivalent to $\PL$, since $A\eiff B$ and $(A\eif B) \eand (B\eif A)$ are logically equivalent in $\PL$.
%
% There are a number of languages that are expressively equivalent to $\PL$ that only include two operators.
% It would be enough to have only negation and the material conditional.
% Show this by writing sentences that are logically equivalent to each of the following using only parentheses, sentence letters, negation (\enot), and the material conditional (\eif):
% \begin{earg}
% \item\leftsolutions\ $A\eor B$
% %$\enot A \eif B$
% \item\leftsolutions\ $A\eand B$
% %$\enot(A \eif \enot B)$
% \item\leftsolutions\ $A\eiff B$
% %$\enot [(A\eif B) \eif \enot(B\eif A)]$
% \end{earg}
% %...
% % Break out of the {earg} environment to give new instructions. 
%
% We could have a language that is expressively equivalent to $\PL$ with only negation and disjunction as operators.
% Show this by using only parentheses, sentence letters, negation (\enot), and disjunction (\eor) to write sentences that are logically equivalent to each of the following:
% % Resume the {earg} environment and restore the counter.
% %...
% \begin{earg}
% \setcounter{eargnum}{\arabic{OLDeargnum}}
% \item $A \eand B$
% %$\enot(\enot A \eor \enot B)$
% \item $A \eif B$
% %$\enot A \eor B$
% \item $A \eiff B$
% %$\enot(\enot A \eor \enot B) \eor \enot(A \eor B)$
% \end{earg}
% %...
% The \emph{Sheffer stroke} is a logical operator with the following characteristic truthtable:
% \begin{center}
% \begin{tabular}{c|c|c}
% \metaA & \metaB & \metaA$|$\metaB\\
% \hline
% 1 & 1 & 0\\
% 1 & 0 & 1\\
% 0 & 1 & 1\\
% 0 & 0 & 1
% \end{tabular}
% \end{center}
% %...
% \begin{earg}
% \setcounter{eargnum}{\arabic{OLDeargnum}}
% \item Write a sentence using the operators of $\PL$ that is logically equivalent to $(A|B)$.
% \end{earg}
% %...
% Every sentence written using a operator of $\PL$ can be rewritten as a logically equivalent sentence using one or more Sheffer strokes. Using no operators other than the Sheffer stroke, write sentences that are equivalent to each of the following. 
% %...
% \begin{earg}
% \setcounter{eargnum}{\arabic{OLDeargnum}}
% \item $\enot A$
% \item $(A\eand B)$
% \item $(A\eor B)$
% \item $(A\eif B)$
% \item $(A\eiff B)$
% \end{earg}
%
%
% \problempart
% \label{HW2.F}
%
%
% The operator `\eor' indicates \emph{inclusive disjunction}; it is true if either \emph{or both} of the disjuncts is true. One might be interested in \emph{exclusive disjunction}, which requires that exactly one of the disjuncts be true. Let us temporarily extend $\PL$ to include a operator for exclusive disjunction, allowing sentences of the form $(\metaA \oplus \metaB)$, where \metaA and \metaB are sentences, meaning that exactly one of \metaA and \metaB are true.
% \begin{earg}
%   \item Provide the truth table for $(\metaA \oplus \metaB)$.
%
%   \begin{tabular}{@{ }c@{ }@{ }c | c@{ }@{ }c@{ }@{ }c@{ }@{ }c@{ }@{ }c}
%     $\metaA$ & $\metaB$ &  & $\metaA$ & $\oplus$ & $\metaB$ & \\
%     \hline 
%      &  &  &  &  & & \\
%      &  &  &  &  &  & \\
%      &  &  &  &  &  & \\
%      &  &  &  &  &  & \\
%   \end{tabular}\\
%
%   \item $\oplus$ is \emph{definable} in terms of \eor, \eand, and \enot. This means that there is a formula using only these latter operators that has the same truth table. Provide such a formula.
% \end{earg}
%
%
% \fi




\section{Tautologies and Weakening}
  \label{sec:TautologyWeakening}

What should we make of the following claim:
  $$P\eand Q \vDash A\eiff\enot\enot A.$$
Notice that the sentence letters on the left-hand-side are unrelated to the sentence letters on the right-hand-side.
So there is a straightforward sense in which the two sides of the logical consequence above have \emph{nothing to do with one another}.
Nevertheless, the claim above is true: every interpretation for which the wfs on the left is true is also an interpretation on which the wfs on the right is true for the simple reason that $A\eiff\enot\enot A$ is true on every interpretation whatsoever.
As a result, we could have dropped the wfs on the left entirely, writing:
  $$\vDash A\eiff\enot\enot A.$$
The claim above is short for $\varnothing\vDash A\eiff\enot\enot A$ where $\varnothing$ is a convenient notation for the empty set $\set{}$, i.e., the set of no wfs of $\PL$.
Since we said above that we are often going to drop the set notation when stating logical consequences, we do not need to write the empty set at all when a wfs is a logical consequence of the empty set of wfss as above.

It is easy to show that a wfs $\metaA$ of $\PL$ is a tautology just in case $\vDash \metaA$. 
Since there are no wfss in $\varnothing$, every interpretation $\I$ of $\PL$ vacuously makes all of the wfs in $\varnothing$ true, and so $\vDash \metaA$ just in case $\V{\I}(\metaA) = 1$ for every interpretation $\I$ of $\PL$, i.e., $\metaA$ is a tautology. 
Given this equivalence, it is common to define the $\PL$ tautologies in terms of logical consequence.
In particular, we could have said that a wfs $\metaA$ of $\PL$ is a \define{tautology} just in case $\vDash \metaA$.
% Dropping set notation, we may leave off the empty set entirely, writing $\vDash \metaA$ for brevity.

In order to get another perspective on why $\metaA$ is a tautology just in case $\vDash \metaA$, it will help to consider the set of interpretations that make a wfs $\metaA$ of $\PL$ true. 
For brevity, we may define the \define{interpretation set} $\vert{\metaA} \coloneq \set{\I : \V{\I}(\metaA) = 1}$ to be the set of all and only those $\PL$ interpretations $\I$ for which $\V{\I}(\metaA) = 1$.
That is, given the set of all interpretations, each wfs $\metaA$ of $\PL$ corresponds to a unique subset of interpretations in which $\metaA$ is true, i.e., $\V{\I}(\metaA) = 1$.
Or to put it another way, each wfs $\metaA$ of $\PL$ amounts to a \textit{constraint} on the interpretations of $\PL$ where only those interpretations $\I$ for which $\V{\I}(\metaA) = 1$ satisfy the constraint. 

Given the definition of an interpretation sets for the wfss of $\PL$, we may provide a slightly more abstract characterization of logical consequence.
By the official definition, $\MetaG \vDash \metaA$ asserts that every $\PL$ interpretation $\I$ which makes every $\metaG \in \MetaG$ true also makes $\metaA$ true. 
Another way to say this is that the intersection of all of the interpretation sets $\vert{\metaG}$ for $\metaG \in \MetaG$ is a subset of the interpretation set for the conclusion $\vert{\metaA}$.
Formally, we may write this as follows:
  $$ \bigcap \set{ \vert{\metaG} : \metaG \in \MetaG} \subseteq \vert{\metaA}. $$
What this says is that every interpretation $\I$ which belongs to every interpretation set $\vert{\metaG}$ for $\metaG \in \MetaG$ is also in the interpretation set $\vert{\metaA}$ for the conclusion.
Since an interpretation $\I$ belongs to an interpretation set $\vert{\metaB}$ just in case $\V{\I}(\metaB) = 1$, this is equivalent to requiring every interpretation $\I$ where $\V{\I}(\metaG) = 1$ for all $\metaG \in \MetaG$ to be such that $\V{\I}(\metaA) = 1$.
But this is exactly what is asserted by $\MetaG \vDash \metaA$, only in the language of set theory.

Despite the equivalence, we will maintain our notation and definition for logical consequence $\MetaG \vDash \metaA$ rather than the set theoretic analogue given above. 
Nevertheless, it can help to take more than one perspective on the same thing, especially for a concept that is as important as logical consequence.
The set theoretic analogue given above provides a vivid account of the way that each $\metaG \in \MetaG$ may constrain the interpretations of $\PL$ in which $\metaA$ is said to be true. 
If there are no wfss in $\MetaG$ at all, i.e., $\MetaG = \varnothing$, this corresponds to imposing no constraints on the interpretations of $\PL$ in which $\metaA$ is said to be true. 
Put otherwise, all interpretations belong to $\vert{\metaA}$, and so $\V{\I}(\metaA) = 1$ for all interpretations $\I$ of $\PL$, i.e., $\metaA$ is a tautology. 

% In order to see why this definition makes sense, we may observe that every $\PL$ interpretation satisfies the empty set $\varnothing$.
% After all, an $\PL$ interpretation $\I$ could only fail to satisfy a set $\Gamma$ of $\PL$ sentences if there were some sentence in $\Gamma$ that fails to be true on $\I$.
% Since there are no sentences in $\varnothing$, the empty set $\varnothing$ is vacuously satisfied by every $\PL$ interpretation whatsoever.
% By definition, an $\PL$ sentence $\metaA$ is entailed by the empty set (i.e., $\vDash \metaA$) just in case every interpretation to satisfy the empty set $\varnothing$ also satisfies $\metaA$.
% Since we have shown that every interpretation satisfies the empty set, this is equivalent to claiming that every $\PL$ interpretation satisfies $\metaA$. 
% Thus, $\vDash \metaA$ if and only if $\V{\I}(A)=1$ for all $\PL$ interpretations $\I$.   

In order to make this more concrete, consider the following claims:

\begin{earg} \label{tauts}
  \eitem{$\vDash P \eor \enot P$.}
  \eitem{$\vDash P \eiff (P \eor (P \eand Q))$.}
  \eitem{$\vDash (P \eand\enot P)\eif (A \eor B)$.}
\end{earg}

Although for different reasons, each of the claims above is true, and so the wfss on the right are all tautologies.
It follows that the following claims are also true:

\begin{earg} \label{weaker}
  \eitem{$A, B\ \vDash P \eor \enot P$.}
  \eitem{$\enot P\ \vDash P \eiff (P \eor (P \eand Q))$.}
  \eitem{$C \eor \enot C\ \vDash (P \eand\enot P)\eif (A \eor B)$.}
\end{earg}

Given that the wfss on the right are all tautologies, we can add whatever wfss that we like on the left.
More generally, it is easy to prove the following principle:

\begin{Lthm}[\it Weakening] \label{lemma:weakening}
  If $\MetaG \vDash \metaA$, then $\MetaG \cup \MetaS \vDash \metaA$.
\end{Lthm}

% \factoidbox{
%   \textsc{Weakening:} If $\MetaG \vDash \metaA$, then $\MetaG \cup \MetaS \vDash \metaA$.
% }

This principle says that whenever $\metaA$ is a logical consequence of $\MetaG$, then $\metaA$ is also a logical consequence of $\MetaG \cup \MetaS$ for any set of wfss $\MetaS$ of $\PL$ whatsoever where $\MetaG \cup \MetaS$ is the union set including all of the wfss in either $\MetaG$ or $\MetaS$ and nothing besides. 
In the special case where $\MetaG = \varnothing$ is empty, we may conclude that if $\metaA$ is a tautology (i.e., $\vDash \metaA$), then $\metaA$ is also a logical consequence of every set $\MetaS$ of wfss of $\PL$ (i.e., $\MetaS \vDash \metaA$).
It is an inference of this kind which is what justified drawing \ref{weaker}\ref{1} -- \ref{weaker}\ref{3} as conclusions from \ref{tauts}\ref{1} -- \ref{tauts}\ref{3} above.

The reason that weakening holds is easy to see given the set theoretic perspective on logical consequence presented before.
Given that a logical consequence $\MetaG \vDash \metaA$ holds, adding further conditions beyond just those included in $\MetaG$ will only further restrict the interpretations of $\PL$ for which $\metaA$ is said to be true. 
Since we already know that all interpretations that make every $\metaG \in \MetaG$ true also make $\metaA$ true given that $\MetaG \vDash \metaA$, then any subset of the interpretations that make every $\metaG \in \MetaG$ true is sure to also make $\metaA$ true, and so $\MetaG \cup \MetaS \vDash \metaA$ for any set of wfss $\MetaS$.
% Replacing $\MetaG$ with the potentially larger set of wfss $\MetaG \cup \MetaS$ amounts to imposing further constraints that the interpretations in question must satisfy. 
To put the point set theoretically, $\bigcap \set{ \vert{\metaG} : \metaG \in \MetaG \cup \MetaS} \subseteq \bigcap \set{ \vert{\metaG} : \metaG \in \MetaG}$ since $\MetaG \subseteq \MetaG \cup \MetaS$. 
% That is, $\MetaG \cup \MetaS$ imposes at least as many constraints as $\MetaG$ on its own. 







\section{Contradictions and Unsatisfiability}
  \label{sec:ContradictionUnsat}

What if a given set $\MetaG$ of wfss $\PL$ imposes so many constraints that there is no interpretation $\I$ of $\PL$ which makes all the wfss $\metaG \in \MetaG$ true, i.e., $\V{\I}(\metaG) = 1$ for all $\metaG \in \MetaG$? 
Put otherwise, what if $\MetaG$ is unsatisfiable? 
In such a case, $\bigcap \set{ \vert{\metaG} : \metaG \in \MetaG} = \varnothing$, and since the empty set $\varnothing$ is a subset of every set, we know that $\varnothing \subseteq \vert{\metaA}$ for any wfs $\metaA$ of $\PL$. 
That is if there are no interpretations that make all of the wfss $\metaG \in \MetaG$ true, then it follows vacuously that all zero of those interpretations are guaranteed to make $\metaA$ true.
More succinctly, every wfs $\metaA$ of $\PL$ is a logical consequence of any unsatisfiable set $\MetaG$ of wfss of $\PL$ whatsoever.

Consider the following examples:

\begin{earg} \label{contra}
  \eitem{$Q, \enot Q\ \vDash P$.}
  \eitem{$R \eif (Q \eand \enot Q), R\ \vDash \enot R$.}
  \eitem{$\enot (A \eif A) \vDash (P \eand\enot P)\eif (A \eor B)$.}
\end{earg}

Some of the above are more obvious than others.
For instance, it is easy to see that there is no interpretation of $\PL$ in which both $\V{\I}(Q) = 1$ and $\V{\I}(\enot Q) = 1$ since a contradiction follows immediately from the semantics for negation were we to assume otherwise.
% Put otherwise, we may show that the intersection $\bigcap\set{\vert{Q}, \vert{\enot Q}} = \varnothing$.
As a result, \ref{contra}\ref{1} is valid for the vacuous reason that no interpretation satisfies $\set{Q, \enot Q}$.
Although slightly more work is required, \ref{contra}\ref{2} is also valid since $\set{R \eif (Q \eand \enot Q), R}$ is also unsatisfiable.

By similar reasoning, we can show that \ref{contra}\ref{3} is valid since $\set{\enot (A \eif A)}$ is also unsatisfiable. 
However, unlike the previous cases, this set has only one member, namely the wfs $\enot (A \eif A)$.
Since $\enot (A \eif A)$ can be shown to be a contradiction, it follows that any set of wfss to which it belongs is unsatisfiable.
In particular, the singleton set $\set{\enot (A \eif A)}$ is unsatisfiable, and so every wfs $\metaA$ of $\PL$ is a logical consequence of $\set{\enot (A \eif A)}$. 
More generally, any set containing a contradiction is unsatisfiable for the simple reason that there is no interpretation which makes a contradiction true.
As a result, every wfs of $\PL$ is a logical consequence of a set of wfss of $\PL$ which includes a contradiction. 

% TODO: transition

Having defined what it is for a wfs of $\PL$ to be a tautology in terms of logical consequence, it is natural to consider how to use the notion of logical consequence to define what it is for a wfs of $\PL$ to be a contradiction.
It will help to consider the following claim:
  $$P \eand \enot P \vDash Q.$$
This statement is true.
It says that $Q$ is true in every $\PL$ interpretation in which $P \eand \enot P$ is true.
This follows vacuously since $P \eand \enot P$ is not true in any $\PL$ interpretations, and so $Q$ is true in every interpretation in which $P \eand \enot P$ is true (all zero of them).
Or to take another approach, think about what it would take for the logical consequence above to be false: there would have to be a $\PL$ interpretation in which $P \eand \enot P$ is true and $Q$ is false.
But there are no $\PL$ interpretations in which $P \eand \enot P$ is true, and so $Q$ is a logical consequence.

Observe that the reason that the logical consequence above is true has nothing to do with the wfs $Q$.
That is, we do not need to appeal to anything about the logical form of $Q$ in order to explain why $P \eand \enot P \vDash Q$. 
Thus the same considerations would demonstrate that $P \eand \enot P \vDash \metaA$ for any wfs $\metaA$ of $\PL$.
Moreover, the same conclusion holds were we to replace $P \eand \enot P$ with any other $\PL$ wfs that is not true on any interpretation.
For instance, consider:

\begin{earg}
  \eitem{$A \eand \enot A \vDash \enot Q \eif R$.}
  \eitem{$\enot (P \eor \enot P) \vDash (A_1 \eor A_2) \eif \enot (A_3 \eiff (A_4 \eand \enot A_2))$.}
\end{earg}

Since there are no $\PL$ interpretations in which the wfss on the left are true, you do not need to examine the wfss on the right to confirm that the logical consequences above are true.
Given that the wfss on the right do not matter, it would be nice to have a way to represent that any wfs $\metaA$ of $\PL$ is a logical consequence of the wfss on the left.
There are two common ways to do this, but both make use of the same notation: $\Gamma \vDash \bot$.
One way to interpret the logical consequence above is as a universal claim that $\Gamma$ entails every wfs $\metaA$ of $\PL$ whatsoever. 
Another way to interpret this logical consequence is to take `$\bot$' to abbreviate some particular contradiction of $\PL$, though it doesn't matter which.
These conventions turn out to amount to the very same thing.
For simplicity, we will assume the latter convention where `$\bot$' abbreviates the contradiction `$A \eand \enot A$' for definiteness.

We have already observed that every wfs $\metaA$ of $\PL$ is a logical consequence of an unsatisfiable set $\MetaG$ of wfss of $\PL$.
Given any unsatisfiable set $\MetaG$ of wfss of $\PL$, it follows that $\bot$ in particular is a logical consequence of $\MetaG$, i.e., $\MetaG \vDash \bot$. 
Conversely, we may prove the following: 

\begin{Lthm} \label{lemma:unsat_bot}
  If $\MetaG \vDash \bot$, then $\MetaG$ is unsatisfiable. 
\end{Lthm} \vspace{-.2in}

\begin{quote} 
  \textit{Proof:}
  Let $\Gamma$ be an arbitrary set of wfss of $\PL$ where $\Gamma \vDash \bot$.
  Assume that $\Gamma$ is satisfiable for contradiction.
  Thus there is a $\PL$ interpretation $\I$ where $\V{\I}(\metaG) = 1$ for all $\metaG \in \MetaG$, and so $\V{\I}(\bot) = 1$ given the assumption that $\Gamma \vDash \bot$.
  Given the definition of $\bot$, we may conclude that $\V{\I}(A \eand \enot A) = 1$.
  By the semantics for conjunction $\V{\I}(A) = \V{\I}(\enot A) = 1$, and so $\V{\I}(A) = 0$ by the semantics for negation, resulting in a direct contradiction.
  Thus $\Gamma$ is unsatisfiable. 
  \qed
\end{quote}

This shows that a set $\MetaG$ of wfss of $\LP$ is unsatisfiable if and only if $\MetaG \vDash \bot$.
In the special case where $\MetaG = \set{\metaA}$, we may say that $\metaA$ is a contradiction just in case $\metaA \vDash \bot$. 
This provides a way to characterize contradictions in terms of logical consequence.

We may close by mentioning a connection between logical consequence and unsatisfiability that we will have occasion to return to later in the metalogical portions of this text.

\begin{Lthm} \label{lemma:unsat_consequence}
  $\MetaG \vDash \metaA$ just in case $\MetaG \cup \set{\enot \metaA}$ is unsatisfiable. 
\end{Lthm}

We will make considerable use of this principle in proving things about the proof system for $\PL$ that we will now turn to introduce in the following chapter.






% Consider the following examples:
%
% \begin{earg}
% \item[] $A \eand \enot A \vDash \bot$.
% \item[] $P, \enot P \vDash \bot$.
% \item[] $P \eiff Q, Q \eiff\enot P \vDash \bot$.
% \end{earg}
%
% The sets of sentences on the left of each logical consequence above are unsatisfiable.\footnote{Recall that we have omitted set notation for ease of exposition.}
% As a special case, we may say that an $\PL$ sentence $\metaA$ is a \define{contradiction} just in case its singleton set $\set{\metaA}$ is unsatisfiable, i.e., $\metaA \vDash \bot$.
% Whereas contradictions are sentences of $\PL$, only \textit{sets} of sentences of $\PL$ may be said to be unsatisfiable.
% Nevertheless, it follows by the weakening principle given above that any set of $\PL$ sentences which contains a contradiction is unsatisfiable.
% By contrast, there are lots of unsatisfiable sets of sentences that do not contain contradictions.
% For instance, neither $P \eiff Q$ nor $Q \eiff \neg P$ are contradictions. 
% Nevertheless, the set $\set{P \eiff Q, Q \eiff \neg P}$ is unsatisfiable. 
% Compare the following examples to those given above:
%
% \begin{earg}
% \item[] $(P \eor \enot P) \vDash\bot$
% \item[] $P, \enot Q, (R \eor Q) \vDash\bot$
% \item[] $\enot P, \enot Q, (P \eif \enot\enot Q) \vDash\bot$
% \end{earg}
%
% Since the sentences on the left can be satisfied, the three entailments given above are false.
%
% % TODO transition
%
% Recall that the definition of satisfiability given in $\S\ref{sub:PL-Satisfiability}$ appealed to complete truth tables: a set of $\PL$ sentences $\Gamma$ is \define{satisfiable} just in case there is at least one line of a complete truth table including every sentence in $\Gamma$ on which all of the sentences in $\Gamma$ are true.
% Alternatively, we may avoid appealing to complete truth tables by saying that a set of $\PL$ sentences $\Gamma$ is \define{satisfiable} just in case there is an interpretation $\I$ of $\PL$ where every sentence in $\Gamma$ is true, i.e., $\V{\I}(\metaA)$ for all $\metaA$ in $\Gamma$.\label{def.PL-satisfiability}
% Accordingly, we may show that any set of $\PL$ sentences $\Gamma$ is satisfiable just in case it is satisfiable. 
% Put otherwise, satisfiability and satisfiablility have the same \textit{extension} insofar as they apply to the very same sets of $\PL$ sentences.
% As a result, if you ever need help seeing whether a given set of sentences can be satisfied, you can always draw out its truth table to check if any row assigns a `1' every sentence in the set.
% If there is no row which assigns a `1` to every sentence in the set, then the set is unsatisfiable








% \section{Unsatisfiability and Logical Consequence}
%
% We have begun to see that there are some important connections between satisfiability and logical consequence which helped us to get a better sense of the logical properties that tautologies and contradictions each enjoy.
% Rather than merely appealing to examples, it will help to say something more systematic about some of the connections that we have already drawn as well as some which we have not.
% This section will provide such an account.



% \section{Defining truth-conditions in $\PL$}
% \label{sec:truth$\PL$}
% \label{sec.semantics$\PL$}
%
% % %JH: this section seems to have some conceptual problems. Try to find an analog in the Logic Book or Calgary book and rewrite this! 
% %Could honestly make more sense to just get right to notions of truth-functional entailment, etc. It's not like we're really explaining `truth' in a first-semester logic class. Would need to get into disquotation theory. 
% %so maybe move this and the next section to the end of the chapter! Ultimately, I'd like to combine this stuff w/ the truth tables anyway, like in the logic book! 
% %alternatively, I could combine this section and the next w/ recursion/induction in a chapter on that. 
%
% {\color{black}This section formally characterizes the \emph{truth-conditions} of well-formed formulae (wffs) in $\PL$. These are the logical conditions under which a sentence is true or false.} We build on what we already know from constructing truth tables. In Chapter~\ref{ch.TruthTables}, we used truth tables to reliably test whether a sentence is a tautology in $\PL$, whether two sentences are equivalent, whether an argument is valid, and so on. For instance: \metaA is a tautology in $\PL$ if and only if it is assigned `1' on every line of a complete truth table.
%
% Truth tables work because each line of a truth table corresponds to a way the world might be, at least conceptually or logically. We consider all the logically possible combinations of 1s and 0s for the sentence letters that compose the sentences we care about. A truth table allows us to determine what would happen given these different combinations. 
%
% {\color{black}To formally define truth-conditions in $\PL$, we introduce a function \textit{\textbf{a}} that assigns---for each interpretation---a 1 or 0 to each of the atomic sentences of $\PL$. We call this function a \textit{truth-value assignment}. We proceed to use the function \textit{\textbf{a}} to construct a more general function \textit{\textbf{v}} called a `valuation function'. Unlike \textit{\textbf{a}}, \textit{\textbf{v}} is defined on arbitrarily complex wffs in $\PL$. We show it is possible to construct the valuation function \textit{\textbf{v}} such that it has the following desirable property:} for any wff \metaA, \textit{\textbf{v}}(\metaA)$=1$ if \metaA is true and \textit{\textbf{v}}(\metaA)$=0$ if \metaA is false. 
%
% %It does this by means of a valuation function $\mathbf{v}$, that acts on atomic sentences 
% %We can interpret this function as accurately defining the truth-conditions for $\PL$ if it assigns 1 to all of the true sentences of $\PL$ and 0 to all of the false sentences of $\PL$. 
%
% Recall that the recursive definition of a wff of $\PL$ has two stages: The first step says that atomic sentences (solitary sentence letters) are wffs. The second stage considers wffs that are constructed out of more basic wffs. There is a clause in the definition for each of the sentential operators. For example, if \metaA is a wff, then \enot\metaA is a wff.

% \subsection{Truth for Atomic Sentences}
%
% Our strategy for defining the valuation \textit{\textbf{v}} will also be in two steps. The first step will handle truth-conditions for atomic sentences; the second step will handle truth-conditions for compound sentences.
%
% How can we define the truth-conditions for an atomic sentence of $\PL$? Consider, for example, the sentence $M$. Without an interpretation, we cannot say whether $M$ is true or false. It might mean anything. If we use `$M$' to symbolize `The moon orbits the Earth', then $M$ is true. If we use `$M$' to symbolize `The moon is a giant turnip', then $M$ is false.
%
% When we give a symbolization key for $\PL$, we provide a translation into English of the sentence letters that we use. In this way, the interpretation specifies what each of the sentence letters \emph{means}. However, this is not enough to determine whether or not that sentence is true. Assessing the sentences about the moon, for instance, requires that you know some rudimentary astronomy. Imagine a small child who became convinced that the moon is a giant turnip. She could understand what the sentence `The moon is a giant turnip' means, but mistakenly think that it were true.
%
% So a symbolization key alone does not determine whether a sentence is true or false. Truth or falsity depends also on what the world is like. If `$M$' meant `The moon is a giant turnip' and the real moon were a giant turnip, then $M$ would be true. To determine a truth-value via the symbolization key, one has to first translate the sentence into English, and then rely on one's knowledge of what the world is like.
%
% We want a logical system that can proceed without astronomical investigation. Moreover, we want to abstract away from the specific commitments of a given symbolization key. So our characterization of truth-conditions proceeds in a different way. We ignore any proffered symbolization key, and take, from a given interpretation, a \emph{truth-value assignment}. Formally, this is just a function that tells us the truth-value of all the atomic sentences. Call this function `\textit{\textbf{a}}' (for `assignment'). We define \textit{\textbf{a}} for all atomic sentence letters \script{P}, such that
% \begin{displaymath}
% a(\script{P}) =
% \left\{
% 	\begin{array}{ll}
% 	1 & \mbox{if \script{P} is true},\\
% 	0 & \mbox{otherwise.}
% 	\end{array}
% \right.
% \end{displaymath}
% This means that \textit{\textbf{a}} takes any atomic sentence of $\PL$ and assigns it either a one or a zero; one if the sentence is true, zero if the sentence is false. 
%
% You can think of \textit{\textbf{a}} as being like a row of a truth table. Whereas a truth table row assigns a truth-value to a few atomic sentences, the truth-value assignment assigns a value to every atomic sentence of $\PL$. There are infinitely many sentence letters, and the truth-value assignment gives a value to each of them {\color{black}(if you're worried about infinity, this might trouble you!)}. When constructing a truth table, we only care about sentence letters that affect the truth-value of sentences that interest us. As such, we ignore the rest.
%
% It is important to note that the truth-value assignment, \textit{\textbf{a}}, is not part of the language $\PL$. Rather, it is part of the mathematical machinery that we are using to describe $\PL$. It encodes which atomic sentences are true and which are false.
%
% \subsection{Truth for arbitrary $\PL$ sentences}
%
% We now define the valuation function, \textit{\textbf{v}}, using the same recursive structure that we used to define a wff of $\PL$.
%
% \begin{enumerate}
% \item If \metaA is a sentence letter, then \textit{\textbf{v}}(\metaA) $=$ \textit{\textbf{a}}(\metaA).
% %\setcounter{Example}{\arabic{enumi}}\end{enumerate}
% %...
% % Break out of the {enumerate} environment to say something about what is
% % going on. Using \setcounter in this way preserves the numbering, so
% % that the list can resume after the comments.
%
% %This is a mathematical equals sign, not the identity predicate we defined for QL.
%
% % Resume the {enumerate} environment and restore the counter.
% %...
% %\begin{enumerate}\setcounter{enumi}{\arabic{Example}}
%
% \item If \metaA is ${\enot}\metaB$ for some sentence \metaB, then
% \begin{displaymath}\textit{\textbf{v}}(\metaA) =
% 	\left\{\begin{array}{ll}
% 	1 & \mbox{if \textit{\textbf{v}}(\metaB) $= 0$},\\
% 	0 & \mbox{otherwise.}
% 	\end{array}\right.
% \end{displaymath}
%
% \item If \metaA is $(\metaB\eand\metaC)$ for some sentences \metaB, \metaC, then
% \begin{displaymath}\textit{\textbf{v}}(\metaA) =
% 	\left\{\begin{array}{ll}
% 	1 & \mbox{if \textit{\textbf{v}}(\metaB) $= 1$ and \textit{\textbf{v}}(\metaC) $= 1$,}\\
% 	0 & \mbox{otherwise.}
% 	\end{array}\right.
% \end{displaymath}
% \setcounter{Example}{\arabic{enumi}}\end{enumerate}
% %...
% \label{truthdefinition}
% You may be tempted to worry that this definition is circular, because it uses the word `and' in trying to define `and.' But remember, we are not attempting to give a definition of the English word `and'; we are giving a definition of truth-conditions for sentences of $\PL$ containing the logical symbol `\eand.' We define truth-conditions for object language sentences containing the symbol `\eand' using the metalanguage word `and.' There is nothing circular about that.
%
% %...
% \begin{enumerate}\setcounter{enumi}{\arabic{Example}}
%
% \item If \metaA is $(\metaB\eor\metaC)$ for some sentences \metaB, \metaC, then
% \begin{displaymath}\textit{\textbf{v}}(\metaA) =
% 	\left\{\begin{array}{ll}
% 	0 & \mbox{if \textit{\textbf{v}}(\metaB) $= 0$ and \textit{\textbf{v}}(\metaC) $= 0$,}\\
% 	1 & \mbox{otherwise.}
% 	\end{array}\right.
% \end{displaymath}
% %\setcounter{Example}{\arabic{enumi}}\end{enumerate}
% %...
% %Notice that this defines truth for sentences containing the symbol `\eor' using the word `and.'
% %...
% %\begin{enumerate}\setcounter{enumi}{\arabic{Example}}
%
% \item If \metaA is $(\metaB\eif\metaC)$ for some sentences \metaB, \metaC, then
% \begin{displaymath}\textit{\textbf{v}}(\metaA) =
% 	\left\{\begin{array}{ll}
% 	0 & \mbox{if \textit{\textbf{v}}(\metaB) $= 1$ and \textit{\textbf{v}}(\metaC) $= 0$,}\\
% 	1 & \mbox{otherwise.}
% 	\end{array}\right.
% \end{displaymath}
%
% \item If \metaA is $(\metaB\eiff\metaC)$ for some sentences \metaB, \metaC, then
% \begin{displaymath}\textit{\textbf{v}}(\metaA) =
% 	\left\{\begin{array}{ll}
% 	1 & \mbox{if \textit{\textbf{v}}(\metaB) $=$ \textit{\textbf{v}}(\metaC)},\\
% 	0 & \mbox{otherwise.}
% 	\end{array}\right.
% \end{displaymath}
% \end{enumerate}
%
% Since the definition of \textit{\textbf{v}} has the same structure as the definition of a wff, we know that \textit{\textbf{v}} assigns a value to \emph{every} wff of $\PL$. Since the sentences of $\PL$ and the wffs of $\PL$ are the same, this means that \textit{\textbf{v}} returns the truth-value of every sentence of $\PL$.
%
% Setting aside tautologies and contradictions, the truth or falsity of a sentence in $\PL$ is always \emph{relative to} some interpretation. This is because the sentence's truth-conditions do not say whether it is true or false. Rather, the truth-conditions specify how the truth of that sentence relates to a truth-value assignment \textit{\textbf{a}} to atomic sentence letters.




%************************JH******************************
%stuff I cut from Ichikawa's version or additional concerns I had about it: 

%JH: following seems unnecessary
%(Technical side note: one might be tempted to \emph{identify} interpretations with assignments of truth-values to atomic sentences. (In past versions of this book, that's actually what I did.) For a variety of reasons, this is not (any longer) my preference. One reason is that there can be different ways to assign truth-values to atomic sentences, corresponding to one and the same row of the truth table, if one includes values for atoms that are not mentioned in the truth table. In a truth table for a sentence that doesn't include an $R$, for instance, these two assignments of truth-values to atoms are effectively equivalent, and correspond to the same row of the truth table: \{$P=1$, $Q=1$, $R=1$\}, \{$P=1$, $Q=1$, $R=0$\}. So there could be different interpretations corresponding to the same row of the truth table, but an interpretation \emph{determines} a row of a truth table.

%A more complex motivation for my terminological choice here has to do with the relationship between $\PL$ and QL, a more complex language we'll learn later in the textbook. I'll return to this connection in \S\ref{sec.0PlaceModels}.)

%JH: I don't understand the following paragraph or the point of it...
%Once we construct a truth table, the symbols `1' and `0' are divorced from their metalinguistic meaning of `true' and `false'. We interpret `1' as meaning `true', but the formal properties of 1 are defined by the characteristic truth tables for the various operators.  The symbols in a truth table have a formal meaning that we can specify entirely in terms of how the operators operate. For example, if $A$ is value 1, then $\enot A$ is value 0.

%JH: part of the following `defN of truth' seems circular, since we appeal to a pretheoretic notion of `true sentences of $\PL$' when we say ``We can interpret this function as a definition of truth for $\PL$ if it assigns 1 to all of the true sentences of $\PL$ and 0 to all of the false sentences of $\PL$". Indeed, we could flip the valuation and we would syntactically have just as good a characterization of truth conditions. We just wouldn't have a `match' with the world. But this modeling principle/norm is separate from the abstract project of characterizing truth-conditions recursively, i.e. showing the compositionality of truth-conditions in $\PL$. 

%JH: it seems a bit misleading to describe this construction as giving a definition of `truth' in sentential logic (indeed as the last paragraph of this section itself seems to indicate!). Rather, we are showing that sentential logic has a compositionality principle: the truth-value of a sentence is determined by the truth-values of component parts. 

%JH: following seems to indicate that we're not really defining `truth' in $\PL$. we're showing that truth in $\PL$ is compositional. Nice illustration of recursive reasoning, but that's about it! so could be a good warmup for the week on induction/recursion. but not essential for the HW problems for ch. 2 and 3 on truth tables, entailment, validity. 


\iffalse

\practiceproblems



\problempart
\label{HW3.C}
Each of the following claims can be evaluated with truth tables. For each, what would you look for in a completed truth table to evaluate it? The Greek letters can stand for any arbitrary sentence of $\PL$. The first claim has the answer filled out for you.

\begin{earg}
		\item[0.] $\Phi$ is a tautology.
		To evaluate this claim, check to see whether the main operator of $\Phi$ has a 1 under it in every row. If so, it is true.
		\item $\Phi \vDash \Psi$. 
		To evaluate this claim, check to see whether... 
		
		\item $\Phi$ is contingent.
\item $\Phi \vDash \bot$
		\item $\emptyset \vDash \Phi$
	\end{earg}


\problempart
Determine whether each entailment claim is true. You may construct a truth table to test it if you like, but these examples are simple enough so that you may be able to just think it through and get the right answer.
\begin{earg}
\item $Q \vDash (P \eor Q)$
\item $P, Q \vDash (P \eor P)$
\item $P \eiff Q, P \vDash Q$ %  \textcolor{red}{True.}
\item $S \vDash (Q \eif Q)$   %\textcolor{red}{True.}
\item $P \eand \enot P \vDash (Q \eor \enot Q)$%   \textcolor{red}{True.}
\item $(P \eand \enot P) \vDash (Q \eand \enot Q)$
\item $(P \eor \enot P) \vDash \bot$
\item $(P \eor \enot P) \vDash Q$%   \textcolor{red}{False: $\{P=1, Q=0\}$.}
\item $\vDash (P \eor \enot P)$
\item $\vDash (P \eand \enot P)$
\item $(A \eor B) \eif \enot P \vDash (P \eor \enot P)$
\item $(P \eiff Q) \vDash ((P \eand Q) \eor \enot (P \eor Q))$
\end{earg}

\fi 

