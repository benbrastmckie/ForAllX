%!TEX root = forallx-ubc.tex

%\chapter{What is logic?}
%\label{ch.intro}

\chapter[Practice Problems]{Exercises}
\label{app.exercises}

\practiceproblemsA{ch.intro}
At the end of each chapter, you will find a series of practice problems that review and explore the material covered in the chapter. There is no substitute for actually working through some problems, because logic is more about a way of thinking than it is about memorizing facts. The answers to some of the problems are provided at the end of the book in appendix \ref{app.solutions}; the problems that are solved in the appendix are marked with a \solutions.

\solutions
\problempart
\label{pr.Sentences1}
Which of the following are `sentences' in the logical sense?
\begin{earg}
\item England is smaller than China.
\item Greenland is south of Jerusalem.
\item Is New Jersey east of Wisconsin?
\item The atomic number of helium is 2.
\item The atomic number of helium is $\pi$.
\item I hate overcooked noodles.
\item Blech! Overcooked noodles!
\item Overcooked noodles are disgusting.
\item Take your time.
\item This is the last question.
\end{earg}

\problempart
\label{hw1.B}
Which of the following are `sentences' in the logical sense?
	\begin{earg}
		\item I would like a double cheeseburger with no onions.
		\item Thank you very much for that gracious reception.
		\item If you strike me down, I shall become more powerful than you could possibly imagine.
		\item There are more trees at UBC than there are flowers in my office and my Uncle Jack really seems to like drinking apple juice, or if that's not apple juice, then he really seems to like whatever it is that he's drinking, but anyway, what I'm really trying to say is, I'm hungry and I could really go for a burger or a bag of scorpions right about now.
		\item I did it
		\item No invalid arguments have impossible premises.
	\end{earg}

\problempart
\label{pr.EnglishTautology}
For each of the following: Is it a tautology, a contradiction, or a contingent sentence?
\begin{earg}
\item Caesar crossed the Rubicon.
\item Someone once crossed the Rubicon.
\item No one has ever crossed the Rubicon.
\item If Caesar crossed the Rubicon, then someone has.
\item Even though Caesar crossed the Rubicon, no one has ever crossed the Rubicon.
\item If anyone has ever crossed the Rubicon, it was Caesar.
\end{earg}

\solutions
\problempart
\label{pr.MartianGiraffes}
Look back at the sentences G1--G4, and consider each of the following sets of sentences. Which are consistent? Which are inconsistent? % on p.~\pageref{MartianGiraffes}
\begin{earg}
\item G2, G3, and G4
\item G1, G3, and G4
\item G1, G2, and G4
\item G1, G2, and G3
\end{earg}

\solutions
\problempart
\label{pr.EnglishCombinations}
Which of the following is possible? If it is possible, give an example. If it is not possible, explain why.
\begin{earg}
\item A valid argument that has one false premise and one true premise
\item A valid argument that has a false conclusion
\item A valid argument, the conclusion of which is a contradiction
\item An invalid argument, the conclusion of which is a tautology
\item A tautology that is contingent
\item Two logically equivalent sentences, both of which are tautologies
\item Two logically equivalent sentences, one of which is a tautology and one of which is contingent
\item Two logically equivalent sentences that together are an inconsistent set
\item A consistent set of sentences that contains a contradiction
\item An inconsistent set of sentences that contains a tautology
\end{earg}


\problempart
\label{hw1.C}
For each, give an argument with the indicated features, or explain why it is impossible to do so:
	\begin{earg}
		\item Valid, but not sound.
		\item Valid, with an impossible conclusion.
		\item Sound, with an impossible premise.
		\item Sound, and an instance of this form:
			\begin{earg}
				\item[] if $P$ then Q
				\item[] R
				\item[\therefore] $Q$
			\end{earg}
	\end{earg}


\problempart
\label{pr.ImpossiblePremises}
Is this argument valid? Why or why not? (Hint: here and elsewhere in logic, read the definitions of our formal terms literally.)
\begin{earg}
\item[(1)] PHIL 220 is a course with a final exam.
\item[(2)] No courses ever have final exams.
\item[\therefore] Everyone is going to get an A+ in PHIL 220.
\end{earg}

\problempart
\label{hw1.A}
For each, indicate whether it is true or false.
	\begin{earg}
		\item All arguments with true premises and true conclusions are sound.
		\item Only valid arguments are sound.
		\item If an argument to the conclusion $A$ is sound, then an argument to the conclusion not $A$ is not sound.
		\item All arguments with at least one impossible premise are valid.
		\item All invalid arguments are instances of invalid argument forms.
		\item No invalid arguments have impossible premises.
	\end{earg}
