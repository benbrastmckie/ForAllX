%!TEX root = forallx-ubc.tex
%\pagestyle{plain}
\pagestyle{appendixstyle}
% change margins so that all the rules will fit
\setlength{\topmargin}{0 in}
\setlength{\headheight}{0 in}
\setlength{\headsep}{0 in}
\setlength{\textheight}{9 in}
\setlength{\evensidemargin}{0.25 in}
\setlength{\oddsidemargin}{0.25 in}
\setlength{\textwidth}{6 in}

\pagenumbering{gobble}

%\chapter[Quick Reference]{Quick Reference}
%\label{app.quickreference}

\section*{\hfill \normalsize \bf Satisfaction Semantics for QL: Key concepts and results \hfill} 

A QL-model $\mathfrak{M} := (D, I)$ consists of a non-empty set $D$ of objects---called the domain of $\mathfrak{M}$---and a map $I$ (the \textit{interpretation} of $\mathfrak{M}$), which maps the vocabulary of QL to objects and ordered pairs from $D$ such that (1) For each constant $c$, $I(c)$ is an element of $D$, called the \textit{referent} or denotation of $c$, and (2) For each k-place predicate $P$, $I(P)$ is a set of ordered $k$-tuples of objects in $D$, called the \textit{extension} of $P$. ($I$ maps atomic sentences to true or false)

A \textbf{variable assignment} for $I$ is a function $d_I$ that maps each variable to an object in the domain $D$. e.g. $d_I (y)$ might equal the object $5 \in \mathbb{N}$ 
%see p. 337 of logic book, for examples 



\textbf{Satisfaction for Atomic wffs}: $\metav{Q}$ of the form $\metav{P}t_1\dots t_k$ where each $t_i$ is a term. $d_I$ satisfies $\metav{Q}$ provided that the k-tuple $\langle t_1^D, \dots , t_k^D \rangle$ lies in the extension of $\metav{Q}$, i.e. in $I(\metav{Q})$

\textbf{Satisfaction for conjunctions}: $d_I$ satisfies $\metav{Q} \eand \metav{R}$ iff it satisfies both $\metav{Q}$ and $\metav{R}$

\textit{Shorthand}: if $Pc$ is true in $\mathfrak{M}$, then $I(c)=r$ \textbf{satisfies} $Px$ in $\mathfrak{M}$. We can write $\mathfrak{M}_{\mathbf{d}_I} \entails Px$

A \textbf{variant} of a variable assignment  $d_I$ is a modified function $d_I[r/x]$ that assigns object $r \in D$ to $x$ and otherwise assigns all other non-$x$ variables the same objects as $d_I$. \\ e.g. $d_I[r/x](x) = r$ and $d_I[r/x] (y) = d_I (y) $

Satisfaction conditions for \textbf{Existentially Quantified} wff $\qt{\exists}{x} \metav{Q}$: a $d_I$ satisfies $\qt{\exists}{x} \metav{Q}$ provided there is SOME object $r \in D$ such that $d_I [r/x]$ satisfies $\metav{Q}$ (i.e. some x-variant of $d_I$)\\ Intuition: provided there's at least one thing you can plug in for $x$ s.t. $\metav{Q}$ comes out true

Satisfaction conditions for \textbf{Universally Quantified} wff $\qt{\forall}{x} \metav{Q}$: a $d_I$ satisfies $\qt{\forall}{x} \metav{Q}$ provided $d_I [r/x]$ satisfies $\metav{Q}$ for EACH object $r \in D$ (i.e. ALL the x-variants of $d_I$). \\ Intuition: provided no matter what you plug in for $x$, $\metav{Q}$ comes out true

\textbf{All-or-nothing Lemma} (11.1.3): given a model $\mathfrak{M} = (D, I)$ and a QL \textit{sentence} $\metav{P}$ (i.e. a wff with no free variables), either all variable assignments $d_I$ satisfy $\metav{P}$ or none do.

\textbf{Truth-in-QL}: A sentence $\metav{P}$ of QL is \emph{true} on model $\mathfrak{M}$ iff some variable assignment $d_I$ satisfies $\metav{P}$ in $\mathfrak{M}$. In this case, we write $\mathfrak{M} \entails \metav{P}$. Otherwise, a sentence $\metav{P}$ of QL is \emph{false} on model $\mathfrak{M}$, i.e. if no variable assignment $d_I$ satisfies $\metav{P}$ in $\mathfrak{M}$.

\textbf{Substitution Lemma} (11.1.1): let $\metav{Q}$ be a wff of QL. The variable assignment $d_I$ satisfies $\metav{Q}\unisub{\script{x}}{\script{c}}$ if and only if $d_I [I(\script{c})/\script{x}]$ satisfies $\metav{Q}$ (intuition: $\script{x}$ refers to the same thing as $\script{c}$) \\ e.g. let $\metav{Q}=Fxx$, then $\metav{Q}\unisub{x}{c} = Fcc$. Let $I(c)$ be some object $r\in D$ s.t. $\langle (r,r) \rangle \in Ext(F)$

\textbf{Locality Lemma} (11.1.7): consider a QL-sentence $\metav{P}$ and two QL-models $\mathfrak{M}^1 := (D, I_1)$ and $\mathfrak{M}^2 := (D, I_2)$ with the same domain $D$, whose interpretation functions $I_1$ and $I_2$ give the same interpretations for any constants or predicates appearing in QL-sentence $\metav{P}$. \\ Then  $\mathfrak{M}^1 \entails \metav{P}$ if and only if $\mathfrak{M}^2 \entails \metav{P}$. (Note that any differences between $\mathfrak{M}^1$ and $\mathfrak{M}^2$ arise from how they interpret QL-symbols NOT appearing in $\metav{P}$).  


\iffalse
\begin{itemize}

\item A QL-model $\mathfrak{M} := (D, I)$ consists of a non-empty set $D$ of objects, called the domain of $\mathfrak{M}$, and a map I (the \textit{interpretation} of $\mathfrak{M}$), which maps the vocabulary of $\mathfrak{L}$ to objects and ordered pairs from $D$ such that (1) For each constant $c \in \mathfrak{L}$, $I(c)$ is an element of $D$, called the \textit{referent} or denotation of $c$ and (2) For each k-place predicate $P$ of $\mathfrak{L}$, $I(P)$ is a set of ordered $k$-tuples of objects in $D$, called the \textit{extension} of $P$. ($I$ maps atomic sentences to true or false)

\begin{itemize}

\item For each constant $c \in \mathfrak{L}$, $I(c)$ is an element of $D$, called the \textit{referent} or denotation of $c$

\item For each k-place predicate $P$ of $\mathfrak{L}$, $I(P)$ is a set of ordered $k$-tuples of objects in $D$, called the \textit{extension} of $P$

\end{itemize}
\end{itemize}
\fi 




\newpage

\section*{\hfill \normalsize \bf Derivation Schemas for Proving Soundness of QND \hfill} 
\label{QND-soundness}
				
				\vspace{-1em}
				
All the rules of SND, plus the following quantifier rules. The rules of SND govern inferences where the main logical operator is one of the connectives from {\it{SL}}. Reiteration also allowed.

%JH preferred substitution notation, from Logic Book: $\metaA{}[\script{c} / \script{x}]$
%Note that you can modify \unisub command in the style file to toggle styles. 

\textbf{Substitution instance}: ``$\metav{Q}\unisub{\script{x}}{\script{c}}$'' is the sentence you get from $\qt{\forall}{\script{x}}\metav{Q}$ or $\qt{\exists}{\script{x}}\metav{Q}$ by dropping the quantifier and putting $\script{c}$ in place of every $\script{x}$ in $\metav{Q}$. The other variables are untouched! \\ Read ``$\unisub{x}{c}$'' as saying ``unisub $c$ for every $x$''\\[1ex]%Equivalent notation: \metav{Q}\hspace{.15em}\raisebox{.3ex}{\fbox{$\script{x}\Rightarrow\script{c}$}}. \\[1ex]
\textbf{Partial Substitution instance}:``$\metav{Q}\freesub{\script{c}}{\script{x}}$'' is the sentence you get by replacing some but not necessarily all instances of the constant $\script{c}$ in $\metav{Q}$ with the variable $\script{x}$. \\ We may write ``$\metav{Q}[\script{c}]$'' to indicate that the constant $\script{c}$ appears in $\metav{Q}$
%But we'll write things out long-hand in the schemata below! 

% (${\bm{P[a/x]}}$  is the sentence you get from $(\qt{\forall}{x}) {\bm{P}}$ by dropping the $(\qt{\forall}{x}) $ quantifier and putting ${\bf{a}}$ in the place of every $x$ in ${\bm{P}}$)



\begin{multicols}{2}

\textit{Universal Elimination} ($\forall$E) 

\begin{proof}
	\have[h]{a}{\qt{\forall}{\script{x}}\metav{Q}}
	\have [\vdots] {n} {\hspace{2em} \vdots}
	\have[k+1]{c}{\metav{Q}\unisub{\script{x}}{\script{c}}} \Ae{a}
	%\have[\ ]{c}{\metaA{}\unisub{\script{x}}{\script{c}}} \Ae{a}
\end{proof}

\phantom{We replace every $\script{x}$ variable  in \metav{Q} with the same constant $\script{c}$. I.e. we go from $\qt{\forall}{\script{x}}\metav{Q}$ to $\metav{Q}[\script{c} / \script{x}]$. Other variables are untouched.} 

\vspace{2.5em}

\textit{Universal Introduction} ($\forall$I) 

\begin{proof}
	\have[h]{a}{\metav{Q}}
	\have [\vdots] {n} {\hspace{2em} \vdots}
	\have[k+1]{c}{\qt{\forall}{\script{x}}\metav{Q}\unisub{\script{c}}{\script{x}}} \Ai{a}
\end{proof}

\textbf{Provided that both} \\
\textbf{(i)} $\script{c}$ does not occur in any undischarged assumptions that \metav{Q} is in the scope of. \\
\textbf{(ii)} $\script{x}$ does not occur already in $\metav{Q}$ \\ (auto-enforced by needing to replace EVERY instance of $\script{c}$ with $\script{x}$  in \metav{Q}\unisub{\script{c}}{\script{x}})
%I think the following is a mistake: $\script{c}$ does not occur already in $\forall \script{x}\metaA{}$. 



\vfill\null
\columnbreak

\textit{Existential Introduction} ($\exists$I)

\begin{proof}
\have[h]{a}{\metav{Q}}
	\have [\vdots] {n} {\hspace{2em} \vdots}
\have[k+1]{c}{\qt{\exists}{\script{x}}\metav{Q}\freesub{\script{c}}{\script{x}}} \Ei{a}
\end{proof}

Provided that $\script{x}$ does not occur in $\metav{Q}[\script{c}]$ (auto-enforced by (a) needing $\metav{Q}$ to be a sentence and (b) our recursion clause for QL-wffs) %\\ Note that $\script{x}$ may replace some or all occurrences of $\script{c}$.





\textit{Existential Elimination} ($\exists$E)

\begin{fitchproof}
	\have[h]{a}{\qt{\exists}{\script{x}} \metav{Q}}
	\have [\ ] {n} {\hspace{2em} \vdots}
	\open	
		\hypo[j]{b}{\metav{Q}\unisub{\script{x}}{\script{c}}} \as{for $\exists$E}
		\have [\ ] {n2} {\hspace{2em} \vdots}
		\have[m]{c}{\metav{P}_{k+1}}
	\close
	%\have [\ ] {n3} {\hspace{2em} \vdots}
	\have[k+1]{d}{\metav{P}_{k+1}} \Ee{a,b-c}
\end{fitchproof}

\textbf{Provided that} \\
(i)  $\script{c}$ does not occur in any other undischarged assumptions that \metav{Q}\unisub{\script{x}}{\script{c}} is in the scope of \\
(ii) $\script{c}$ does not occur already in $\qt{\exists}{\script{x}} \metav{Q}$ \\
(iii) $\script{c}$ does not occur in $\metav{P}_{k+1}$ \\

%\footnotesize{Simplified restriction (easier to remember): \textbf{provided $\script{c}$ doesn't occur anywhere else outside the subproof}. (Moral: always use a distinct constant for existential elimination, and you'll satisfy the three restrictions above automatically!)}
%would be nice to shorten this to `provided c does not occur outside the subproof, but not sure if this is actually true! e.g. we could instantiate some other universal with c, and so have Fc outside the subproof. 

\end{multicols} %ends multicol started for QND rules 
