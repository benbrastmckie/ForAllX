%!TEX root = forallx-ubc.tex

\chapter*{Preface to the Fall 2023 MIT Edition}
\label{ch.preface2}
\addcontentsline{toc}{chapter}{Preface to the MIT edition}

The goal of this modified version of Ichikawa's \textit{ForAllX} is to bring to the masses some of the best features of the \textit{Logic Book}, for free!\footnote{McGraw--Hill: catch me if you can!} Of the existing free texts, Ichikawa's version seemed the closest (I hope that I was not wrong about this!). Why the \textit{Logic Book} system? It seems to me to strike the best balance of intellectual and practical virtues: no special symbols or special rules, a pleasing symmetry between introduction and elimination rules that makes things easy to remember, and the philosophical advantages of using proof-by-cases for disjunction elimination (along with the opportunity this affords to test if students are \textit{really} paying attention to derivation syntax). 

Below is an ongoing list of modifications I have made, starting with some of the bigger ones:

\begin{itemize}

\item Natural Deduction in Sentential Logic: I have modified Ichikawa's system to be in complete accordance with that in the \textit{Logic Book}. This mainly involves relegating Disjunctive Syllogism to a derived rule, and modifying the justification syntax for indirect proofs.  I have also inserted expository material from the Calgary edition of ForAllX, since I preferred their organization. 

\item To help integrate the text better with the online \textit{Carnap} proof checker, I have modified the syntax in the Selinger Fitch system to mirror what students must enter into \textit{Carnap} for Natural Deduction proofs in system LogicBookSD. One simply changes the \verb|\by{}{}| command to print the arguments in the opposite order, along with adding a colon first. I also added \verb|\pr{}| and \verb|\as{}| commands for premises and subproof assumptions. Unfortunately, the \textit{Carnap} systems are very sensitive to syntax, and it seems cruel to ask students to learn a separate syntax for homework problems than the one they are reading in the text. (I have also made the style file compatible with Kl{\"u}wer's Fitch system, but I don't recommend it, since it cannot be easily reformatted). 

\item I have systematically replaced the quantifier/bound variable syntax to match that of the \textit{Logic Book} (this involved over 700 replacements!). Who would have thought that \textit{Carnap} would care so much about the difference between `$\exists x$' vs. `$(\exists x$)', but hey that's syntax for you! (System LogicBookPD accepts only the latter!). To avoid this tedium in the future, all quantifier/bound variable instances are written with a command \verb|\qt{}{}| whose first argument takes the quantifier and second argument takes the bound variable. This way, one can easily modify the formatting in the style file (e.g. removing or adding parentheses). Again, it would seem cruel to ask students to remember that while in \textit{Carnap} using system LogicBookPD, they MUST put parentheses around $\exists x$ and $\forall x$, if that's not the format they're seeing in their text. 

\item I have reworked the recursive definition of truth-conditions in Quantifier Logic (QL) (\S\ref{sec:satisfactionQL}). Hopefully this makes it both (1) easier to understand and (2) slightly more precise (although these may be competing virtues!). Whereas Ichikawa refers to this aspect of formal semantics as giving a definition of `truth', I prefer to say we are defining `truth-conditions' (I don't know of any intro logic books that provide a theory or definition of `truth'!). I have reworked the definition of truth-conditions for SL (\S\ref{sec:truthSL}) in a similar vein. 

\item I've created a more detailed list of rule schemas for SL tree rules, in the appendix! 
%have also replaced `recursive' with `inductive' or `proof by induction' in Chapter on soundness/completeness for trees. 

\item I have input the Use--Mention section from Calgary ForallX.

\item The UBC edition tends to use the past-tense in a lot of places where the present-tense sounds better. It is part of the conceptual grammar of logic (and mathematics) that it is timeless; Logic is forever! (I've also tried to reduce the number of semi-colons). 

\item I have rearranged and re-numbered chapters so as to match the week numbers at MIT, starting with a half-week `0.' This way, most chapter numbers, week numbers, and problem set numbers coincide. Hopefully, this makes it easier to remember what we have to do each week!

\item I've tried to enclose most of my additions within \verb| {\color{black} } | so that they can be easily found and changed (e.g. deleted!). This might be useful for some of the more playful asides, which instructors more staid than myself might find distasteful. 
%could ruffle the feathers of more staid instructors than myself. 

\item \textit{Philosophical Enrichment}: last but not least, I have incorporated philosophical asides at various places where the formal discussion runs headlong into philosophical waters. Intro logic books typically suppress any and all philosophy, making the whole subject seem cut and dry. Far from it! Really what goes on is that these books lead students down a narrow, carefully constructed path, where---unbeknownst to them---cliffs drop off on either side. Of course our glass houses will remain pristine if we never take a hammer to them. Obviously, I make no attempt to resolve any of these philosophical quandaries in these asides. I hope that they will engender a sense of wonder (or worry!) in the reader. First order logic provides an ideal setting for illustrating how a subject matter can be settled technically but not philosophically. If you're teaching logic or taking it in a Philosophy department, then let's do some philosophy! 



\end{itemize}

To the students especially: there are going to be bugs, so please pack your bugspray. 

\begin{flushright}
Josh Hunt \\
August 2022 \\
joshhunt@mit.edu
\end{flushright}


\chapter*{Preface to the UBC Edition}
\label{ch.preface}
\addcontentsline{toc}{chapter}{Preface to the UBC edition}

\textit{Hunt: The following first person pronouns refer to Ichikawa, not me! This is true for many of the first person pronouns in the book. Although, if you come across one in the context of a mildly ridiculous-sounding claim, that's probably me!}

This preface outlines my approach to teaching logic, and explains the way this version of \emph{forall x} differs from Magnus's original. The preface is intended more for instructors than for students. 

I have been teaching logic at the University of British Columbia since 2011; starting in 2017, I decided to prepare this textbook, based on and incorporating much of P. D. Magnus's \emph{forall x}, which has been freely available for use and modification since 2005. Preparing this text had two main advantages for me: it allowed me to tailor the text precisely to my teaching preferences and emphasis, and, because it is available for free, it is saving money for students. (I encourage instructors to take this latter consideration pretty seriously. If you have a hundred students a year, requiring them each to buy a \$50 textbook takes \$5,000 out of students' pockets each year. If you teach with this or another free book instead, you'll save your students \$50,000 over ten years. It can be sort of annoying to switch textbooks if you're used to something already. But is staying the course worth \$50,000 of your students' money?)

This text was designed for a one-semester, thirteen-week course with no prerequisites. At UBC, the course has quite a mix of students with diverse academic backgrounds. For many it is their first philosophy course. As I teach Introduction to Formal Logic, the course has three central aims: (1) to help students think more clearly about arguments and argumentative structure, in a way applicable to informal arguments in philosophy and elsewhere; (2) to provide some familiarity and comfort with formal proof systems, including practice setting out formal proofs with each step justified by a syntactically-defined rule; and (3) to provide the conceptual groundwork for metatheoretical proofs, introducing the ideas of rigorous informal proofs about formal systems, preparing students for possible future courses focusing on metalogic and computability. I try to give those three elements roughly equal focus in my course, and in this book.

The book introduces two different kinds of formal proof systems --- analytic tableaux (`trees') and Fitch-style natural deduction. Unlike many logic texts, it puts its greater emphasis on trees. There are two reasons I have found this to be useful. One is that the algorithmic nature of tree proofs means that one can be assured to achieve successful proofs on the basis of patience and careful diligence, as opposed to requiring a difficult-to-quantify (and difficult-to-teach) `flash of insight'. The other is that the soundness and completeness theorems for tree methods are simpler and more intuitive than they are for natural deduction systems, and I find it valuable to expose students to proofs of significant metatheoretical results early in their logical studies. (I prove soundness and completeness for a sentential logic tree system in the fifth week of the semester.) As presented here, the soundness and completeness proofs emphasize contrasting the systems students learn with hypothetical alternative systems that modify the rules in various ways. A rule like this would undermine the soundness of the system, but not its completeness. If we changed the rules in this way, it would still be both sound and complete. Etc. This helps give intuitive substance to these theorems.

I also include a Fitch-style natural deduction system, both for sentential and quantified logic, both because its premise-conclusion form is particularly helpful for thinking about informal arguments, and because it is important to recognize and follow proofs laid out in that kind of format, for example in more advanced philosophical material. While students do learn to do Fitch-style proofs, I emphasize less of that puzzle-solving kind of skill here than in many textbooks.

The book begins with a systematic engagement with sentential logic in conventional ways: translations, sentential connectives, models, truth tables, and both proof systems, including soundness and completeness for the tree system. Students are thereby able to familiarize themselves with all the central metalogical ideas, incorporating relatively simple logical symbolism, before introducing predicates, quantifiers, and identity. Once we enrich the language, we go through those previous ideas again, using our more complex vocabulary.

The first book I used for teaching was Greg Restall's \emph{Logic} (McGill--Queen's University Press, 2006), which I used for several years. My approach to teaching logic is heavily informed by that book; its influence in this text is particularly clear in the discussion of trees. (The natural deduction system I use is rather different from Restall's.)

In preparing this text, I began with Magnus's original and edited freely. There are sections where Magnus's prose has been retained entirely, and many of the exercises I have taken unchanged from the original. But I have also restructured many things and added quite a bit of new material. Unlike my version, which focuses on sentential logic before introducing predicates and quantification, Magnus's version integrated the discussion of sentential and quantificational systems, e.g.\ covering translation for both before discussing models and proofs for either. The original also did not include trees or soundness and completeness proofs. The two chapters on trees (\ref{ch.sl.trees} and \ref{ch.QLTrees}) and soundness and completeness (\ref{ch.SLsoundcomplete} and \ref{ch.QLsoundcomplete}) were written from scratch; my chapter on identity (\ref{ch.identity}) is also original. The other material in this edition incorporates Magnus's original material, some parts more heavily edited than others. I have slightly modified Magnus's natural deduction rules.

After a couple of years working with `beta' versions of the text online, I released the 1.0 version, along with the source code, in December 2018. The 2.0 version is new in summer 2020. The biggest changes in the latest round of revisions are in Chapter \ref{ch.ND.proofs}, where the order of presentation of the natural deduction rules has changed, and more examples have been added within the text. The rationale of the change was to start illustrating proofs earlier in the presentation of the rules. I've also put a bit more emphasis on the importance of exact matching of rule forms, and written a bit more precisely about the difference between SL proofs and proof schemas, when discussing derived rules. The other slightly substantive change I've made is to attend more precisely to how I'm using the term `interpretation' in the formal semantics for SL and QL. One of my aims is to emphasize the continuity between the two languages --- in my system, QL is literally a generalization of SL, and definitions of truth, entailment, etc., can be preserved. Various other smaller changes have been made as well, mostly stylistic changes and typo corrections. In summer 2021, I standardized and slightly modified the notation for assumptions in natural deduction proofs. I frequently make quite small corrections; the latest version is always on Github.

Many thanks, first and foremost, to P.D.\ Magnus for providing this wonderful resource under a Creative Commons license, which made it freely available and gave me the right to modify and distribute it under the same licensing agreement. I hope other instructors will also feel free to either teach directly from this version, or to modify it to develop their own. The typesetting for trees is via Clea F.\ Rees's prooftrees package; thanks to her for making it available.

I'm grateful to the students in my 2017--20 PHIL 220 courses at UBC, who had an in-progress version of this book as their course textbook. They patiently and helpfully found and pointed out mistakes as I wrote them (incentivized, perhaps, by an offer of extra credit); this version has many fewer errors than it otherwise would have had. Thanks also to Cavell Chan and Joey Deeth, who did careful proofreading, and generated many solutions to exercises for the answer key, and to Laura Greenstreet for LaTeX and other technical help. These three assistants were supported by a UBC Library Open Access Grant in 2018--19.

I am maintaining a list of known issues and errors for this book, to be corrected in future editions, under `issues' at \url{https://github.com/jonathanichikawa/for-all-x}. If you see any mistakes, please feel free to add them there directly, or to email me with them. The most recent version of the book is also always available for download there too.

\begin{flushright}
Jonathan Ichikawa \\
University of British Columbia \\
October 2021 \\
ichikawa@gmail.com
\end{flushright}
