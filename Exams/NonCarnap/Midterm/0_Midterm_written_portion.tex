 \documentclass[12pt]{article}
 
 \usepackage{geometry}
\geometry{verbose,tmargin=1in,bmargin=1in,lmargin=1in,rmargin=1in}

\usepackage{enumitem} 
\usepackage{multicol}
 
 \iffalse
 \textheight     11truein
 \vsize     10.0truein
 \topmargin      .15truein
 \textwidth 6.5truein \columnwidth \textwidth 
 \setlength{\oddsidemargin}{0truein} 
 %\footheight     0.0truein
 \footskip       0.75truein
 \headheight     .25truein
 \headsep        0.25truein
 \fi 

\usepackage{amsmath} %for align* environment and gather*
\usepackage{xref}
 %    \textheight     10.0truein
 \usepackage{graphics}
 \usepackage{pstricks}
% \usepackage{pst-tree}
% \usepackage{pst-node,pst-tree}
 \usepackage{makeidx}
 
\newcommand{\set}[1]{\lbrace#1\rbrace} %%Set brackets
% \newcommand{\I}{\mathcal{I}}
% \newcommand{\J}{\mathcal{J}}
\newcommand{\B}{\mathcal{B}}
% \newcommand{\even}{\texttt{Even}}
% \newcommand{\comp}{\texttt{Comp}}
\newcommand{\res}{\texttt{Res}}
% \newcommand{\simp}{\texttt{Simple}}
% \newcommand{\leng}{\texttt{Length}}
% \newcommand{\V}[1]{\mathcal{V}_{#1}} %%Corner quotes

 
 %\usepackage{forallx-ubc-Hunt} %calls local modified style file, but could lead to conflicts. i ought to make my own `common.sty' file for the problem sets. maybe using the same `common' file as the lecture notes? i really ought to just have a single consistent set of style files for the book, problem sets, and lecture notes! 
 
 %\def\therefore{\ensuremath{\ldotp\dot{}\,\ldotp}}
% disjunction
\def\eor{\ensuremath{\vee}}
% conjunction: 
% {\,^{_{_{_{_{\mbox{\footnotesize\textbullet}}}}}}} gives the dot
\def\eand{\ensuremath{\,\&\,}}
% conditional: \rightarrow gives the right arrow
\def\eif{\ensuremath{\supset}}
% biconditional: \leftrightarrow gives the left and right arrow
\def\eiff{\ensuremath{\equiv}}
% negation: {\sim} gives the swung dash 
%\def\enot{\ensuremath{\neg}}
%\def\enot{\ensuremath{\sim}} %note that \sim is defined as a relation, which leads to spacing issues. adding a \! leads to more spacing issues (piled up double negations).  

\def\enot{\ensuremath{{\sim}}} %redefining as {\sim} treats the tilde as a unary operator, rather than a relation, solving a lot of the spacing issues. 

\let\oldsim\sim %renames any \sim commands as \oldsim. 
\renewcommand{\sim}{{\oldsim}} %redefines \sim as unary operator version of \sim, in case there are any straggling \sim commands in the wild

% metalanguage variables: change greek to A and B if you prefer
\def\metaA{\ensuremath{\varPhi}}
\def\metaB{\ensuremath{\varPsi}}
\def\metaC{\ensuremath{\varOmega}}
\def\metaD{\ensuremath{\varDelta}}
\def\metaSetX{\ensuremath{\mathcal{X}}}
\def\metaSetY{\ensuremath{\mathcal{Y}}}
\def\metaSetZ{\ensuremath{\mathcal{Z}}}

%Calgary script and metav commands: 
\newcommand*{\script}[1]{\ensuremath{\mathcal{#1}}}
\newcommand*{\metav}[1]{\ensuremath{\mathcal{#1}}}

\def\proves{\ensuremath{\vdash}}
\def\entails{\ensuremath{\vDash}}
\def\nproves{\ensuremath{\nvdash}}
\def\nentails{\ensuremath{\nvDash}}


% \pagestyle{empty}
 
 % Tree stuff
 
  \usepackage{prooftrees} %i copied over prooftrees file from Ichikawa source files, which I think is pre-2019 version
  
 %Note that I probably ought to just update my prooftrees package, since I downloaded the zip file and can just paste over the older version! (but who knows what else this could change...)
%JRH: adding in definition of line no override, local option included in 2019 revision. since my prooftrees package is not up to date! 
% see code here: https://tex.stackexchange.com/questions/415976/manually-set-line-numbers-if-prooftrees-sty
% see p. 24 of prooftrees manual for directions on using this. works w/ {}, e.g. line no override={n+1}

%the command `vdotsline' lets you put anything in number column, without a period appearing afterwards. so it's like `line no override' without \linenumberstyle
%e.g. for vertical dots vertically aligned, use: vdotsline={\\[-0.55em] \vdots}

\forestset{
  line no override/.style={
    before drawing tree={
      for name/.process={Ow}{proof tree proof line no}{line no ##1}{
        content=\linenumberstyle{#1},
        typeset node,
      },
    },
  },
  no line no/.style={
    before drawing tree={
      for name/.process={Ow}{proof tree proof line no}{line no ##1}{
        content=,
        typeset node,
      },
    },
  },
  vdotsline/.style={
    before drawing tree={
      for name/.process={Ow}{proof tree proof line no}{line no ##1}{
        content=#1,
        typeset node,
      },
    },
  },
  default preamble={
	single branches,
	close with=\ensuremath{\times},
	just sep=1.75em,
	line no sep=1.75em
	}
}

\begin{document}

\input macs
%\input fitch
\newcommand{\detritus}[1]{}


\thispagestyle{empty}



\iffalse
\parindent = 0pt
\hspace*{0.0in}\parbox[t]{2.5in}{
Philosophy 24.241\\[3pt]
Symbolic Logic\\[3pt]
Fall, 2022
}
\fi 

%\bigskip %\bigskip

\iffalse 
\begin{center}
\Large\bf Problem Set 5\\[1ex] 
 Due Fri. {\bf{October 14th}} by 5pm Eastern\\[3ex]
\end{center}
\fi

\begin{center}
\large Midterm for Logic I: 24.241 \\[1ex] 
\normalsize (4 problems in 80 minutes)\\
  \vspace{.1in}
\normalsize --- 100 points total ---\\
  \vspace{.25in}
\normalsize (Please give yourself lots of room to work, and write clearly.)
  \vspace{.5in}
%\large Solutions: Please Never Share or Upload to Internets or I'll have to make many new problems in the future and that would be a BIG BUMMER \\[3ex] 
% \textbf{Answer FOUR questions total: 1 and 2, 3 Xor 4, 5 Xor 6 (exclusive or's!)}
\end{center}

%Question 0: if you worked with up to two classmates, please list their names! 
%Some of these problems draw from the posted Induction and Recursion notes.\\

%For questions 1 and 2, provide good translations of the following arguments into the language of sentential logic. Then, investigate their validity using the tree method (STD)

\noindent
Below is the symbolization key for problems \textbf{(P1)} and \textbf{(P2)}: 

\begin{itemize}
  % \item[$J$:] John believes the cookie is tasty.
  % \item[$M$:] Mary believes the cookie is tasty.
  % \item[$C$:] The cookie really is tasty.
  % \item[$A$:] Adrian is at the party.
  % \item[$B$:] Beth goes to the party.
  % \item[$C$:] Chuck goes to the party.
  % \item[$D$:] Adrian and Beth get into a debate.
  % \item[$E$:] Chuck entertains Beth.
  % \item[$H$:] The party is horrible.
  % \item[$K$:] Adrian knows that Beth is considering going to the party. 
  \begin{multicols}{2}
    \item[$F$:] The fair coin is flipped.
    \item[$H$:] The fair coin landed heads.
    \item[$T$:] The fair coin landed tails.
    
    \item[$D$:] Dor plays at the party.
    \item[$E$:] Eyal plays at the party.
    \item[$P$:] The party will be a success.
  \end{multicols}
\end{itemize}

\bigskip

\noindent
For the arguments given in \textbf{(P1)} and \textbf{(P2)}, complete the following tasks [5pt per task]:

\begin{itemize}
  \item[(I)] Use the symbolization key to regiment the argument in SL.
  \item[(II)] Use the tree method to check the validity of the SL argument.
    You do not need to label each step, but do not skip any steps, and mark branches as either complete and open or closed.
    Please give yourself lots of room so that your tree is easy to read.
  \item[(III)] If the argument is valid, provide an natural deduction proof in SD (you may use any of the derived rules provided on the last page but make sure to justify each line), and otherwise provide an interpretation that invalidates the argument.
  \item[(IV)] Does your regimentation of the argument in SL accurately capture the intuitive validity/invalidity of the argument in English? Answer with `YES' or `NO'.
  \item[(V)] If your answer to (IV) was `NO', what has been missed? 
    If `YES', does your answer to (III) succeed in explaining why the argument is valid/invalid?
\end{itemize}

\bigskip

\begin{enumerate}
  % \item[\bf (P1)] John believes the cookie is tasty just in case the cookie really isn't tasty.
  %   Mary only believes the cookie is tasty if John does and the cookie really is tasty.
  %   So Mary doesn't believe the cookie is tasty.
  %   [20pt]
  % \item[\bf (P2)]  Adrian is at the party and knows that Beth is considering going. So long as Beth and Adrian don't get into a debate, the party won't be horrible, but it will be horrible if they do. However, if Beth goes to the party, she is sure to debate Adrian. But if Chuck goes to the party with Beth, then he will keep her entertained her, and so Beth and Adrian won't get into a debate. As a result, it's not true that the party will be horrible if Beth goes. [20pt]
  \item[\bf (P1)] If the fair coin had been flipped, it would have either landed heads or tails.
    But it's not true that if it had been flipped, it would have landed heads.
    Therefore if it had been flipped, it would have landed tails.
  \item[\bf (P2)] If either Dor or Eyal play, the party will be a success.
    Thus the party will be a success if Dor plays, and equally, the party will be a success if Eyal plays.
\end{enumerate}

\newpage


Recall the following lemmas from the proofs of soundness and completeness:

\begin{enumerate}
  \item[\tt Lemma 1:] Every satisfiable branch $\B$ in an SL tree $X$ is open. 
  \item[\tt Lemma 2:] If $X$ is an SL tree with a satisfiable branch $\B$, then any tree $X'$ which is the result of resolving a sentence in $\B$ has a satisfiable branch $\B'$.  
  \item[\tt Lemma 3:] Every SL tree with a satisfiable root has a satisfiable branch.
  \item[\tt Lemma 4:] Every SL tree $X$ has a finite number of branches.  
  \item[\tt Lemma 5:] For any SL tree $X$ with root $\Gamma$ and $\varphi\in[X]$, there is an SL tree $Y$ with root $\Gamma$ where $\res(Y)<\res(X)$. 
  \item[\tt Lemma 6:] For any tree $X$ with root $\Gamma$, there is a complete tree $X'$ with root $\Gamma$. 
  \item[\tt Lemma 7:] Every complete open branch in an SL tree is satisfiable.
\end{enumerate}

\bigskip

\noindent
Recall the following meta-logical theorems where `$\vdash$' is defined in terms of SL trees proofs.\\

\bigskip

  \begin{itemize}[leftmargin=1.5in]
    \begin{multicols}{2}
    \item[\sc (A)] If $\Gamma \vdash \bot$, then $\Gamma \vDash \bot$.
    \item[\sc (B)] If $\Gamma \vDash \bot$, then $\Gamma \vdash \bot$.
    \end{multicols}
  \end{itemize}

\bigskip

\begin{enumerate}

  \item[\bf (P3)] Complete the following tasks for your choice of either \textsc{(A)} or \textsc{(B)}.
    \begin{itemize}
      \item[(I)] Draw on the relevant lemmas to prove the theorem, stating it's name. [10pt]
      \item[(II)] Briefly discuss what this theorem teaches us about the tree proof system. [10pt]
      \item[(III)] State whether the claim given below is true or false. [5pt]
    \end{itemize}
  \item[\bf Claim:] If the theorem had been false, the tree proof system would still have been useful.
  % \begin{itemize}
  %   \item[(I)] Draw on the lemmas above (you won't need all of them!) to provide a proof of this metalogical claim.
  %   \item[(II)] 
  %   \item[(III)] 
  %   \item[(IV)] 
  % \end{itemize}

\bigskip

% \item[\bf (P4)] Consider a tree proof system exactly like the one we provided in this class except the rule for negated biconditionals has been replaced with the following alternative:
%
% \bigskip
%
% \begin{center}
% \textit{Next Negated Biconditional} (N\enot \eiff) \vspace{0.5em}
%
% \begin{prooftree}
% {line numbering, single branches}
% [\enot(\metaA{}\eiff\metaB{}), line no override={m}
% [\vdots, vdotsline={\\[-0.55em] \vdots}, grouped
% 	[\metaA{} \eand \enot\metaB{}, line no override={j}, just={m N\enot \eiff}
% 	[N, grouped, line no override={j+1}
% 	]
% 	]	
% 	[\enot\metaA{} \eand \metaB{}
% 	[\enot N, grouped
% 	]
% 	]
% ]
% ]
% \end{prooftree}
% \end{center}
%
%   \bigskip
%
% \textbf{(a) Is the modified tree system sound?} If so, prove that the modified system is sound; if not, give a tree that is a counterexample to soundness. [10pt]
%
% \textbf{(b) Is the modified tree system complete?} If so, prove that the modified system is complete; if not, give a tree that is a counterexample to completeness. [10pt]
%
% \textit{Hint}: You only need to add one case to one lemma to prove soundness/completeness, but do give the basic proof setup for the lemma in question.
%
%
% \newpage

\item[\bf (P4)] Complete the following tasks:
  \begin{itemize}
    \item[(I)] Derive $\varphi\vee(\varphi\wedge\psi) \vdash \varphi$ from basic rules. [5pt]
    % \item[(II)] $\varphi \supset \psi\ \vdash \neg\psi \supset \neg\varphi$. [5pt]
    \item[(II)] Derive $\varphi \equiv \psi,\ \neg\varphi\ \vdash \neg\psi$ from basic rules. [5pt]
    \item[(III)] Show that $\Gamma = \set{ A\vee(A\wedge B),\ D \equiv \neg(A\wedge C),\ \neg (C \wedge D) \supset \neg(C \supset C) }$ is provably inconsistent in SD by using any of the derived rules provided on the following page. 
      Remember to number every line and include a justification for each.
      [15pt]
  \end{itemize}
\end{enumerate}

\newpage

\noindent
You may use any of the following derived rules, though you will not need them all!

\bigskip

\begin{enumerate}[leftmargin=2.75in]
  \item[\it Disjunctive Syllogism \textsc{(DS):}] $\varphi \vee \psi,\ \neg \varphi \vdash \psi$.
  \item[\it Modus Tollens \textsc{(MT):}] $\varphi \supset \psi,\ \neg\psi\ \vdash \neg\varphi$.
  \item[\it Contraposition \textsc{(CP):}] $\varphi \supset \psi\ \vdash \neg\psi \supset \neg\varphi$.
  \item[\it $\neg$-Modus Tollens \textsc{(BMT):}] $\varphi \equiv \psi,\ \neg\varphi\ \vdash \neg\psi$.
  \item[\it $\equiv$-Contraposition \textsc{(BCP):}] $\varphi \equiv \psi\ \vdash \neg\psi \equiv \neg\varphi$.
  \item[\it Hypothetical Syllogism \textsc{(HS):}] $\varphi \supset \psi,\ \psi \supset \chi\ \vdash \varphi \supset \chi$.
  \item[\it Dilemma \textsc{(DL):}] $\varphi \vee \psi,\ \varphi \supset \chi,\ \psi \supset \chi\ \vdash \chi$.
  \item[\it Ex Falso Quodlibet \textsc{(EFQ):}] $\varphi,\ \neg\varphi\ \vdash \psi$.
  \item[\it Law of Excluded Middle \textsc{(LEM):}] $\vdash \varphi\vee\neg\varphi$.
  \item[\it Law of Non-Contradiction \textsc{(LNC):}] $\vdash \neg(\varphi\wedge\neg\varphi)$.
  \item[\it $\vee$-Commutativity \textsc{($\vee$CM):}] $\varphi \vee \psi\ \vdash \psi \vee \varphi$.
  \item[\it $\wedge$-Commutativity \textsc{($\wedge$CM):}] $\varphi \wedge \psi\ \vdash \psi \wedge \varphi$.
  \item[\it $\equiv$-Commutativity \textsc{($\equiv$CM):}] $\varphi \equiv \psi\ \vdash \psi \equiv \varphi$.
  \item[\it Double Negation \textsc{(DN):}] $\neg\neg\varphi\ \dashv\vdash \varphi$.
  \item[\it $\vee$-Conditional \textsc{($\vee$C):}] $\varphi \supset \psi\ \dashv\vdash \neg\varphi \vee \psi$.
  \item[\it $\neg$-Conditional \textsc{($\neg$C):}] $\neg(\varphi \supset \psi)\ \dashv\vdash \varphi \wedge \neg \psi$.
  \item[\it $\wedge$-De Morgan's \textsc{($\wedge$DM):}] $\neg(\varphi\wedge\psi)\dashv\vdash\neg\varphi\vee\neg\psi$.
  \item[\it $\vee$-De Morgan's \textsc{($\vee$DM):}] $\neg(\varphi\vee\psi)\dashv\vdash\neg\varphi\wedge\neg\psi$.
  \item[\it ${\vee}{\wedge}$-Distribution \textsc{($\vee$D):}] $\varphi\vee(\psi\wedge\chi) \dashv\vdash (\varphi\vee\psi)\wedge(\varphi\vee\chi)$.
  \item[\it ${\wedge}{\vee}$-Distribution \textsc{($\wedge$D):}] $\varphi\wedge(\psi\vee\chi) \dashv\vdash (\varphi\wedge\psi)\vee(\varphi\wedge\chi)$.
  \item[\it ${\wedge}{\vee}$-Absorption \textsc{($\wedge$AB):}] $\varphi\wedge(\varphi\vee\psi) \dashv\vdash \varphi$.
  \item[\it ${\vee}{\wedge}$-Absorption \textsc{($\vee$AB):}] $\varphi\vee(\varphi\wedge\psi) \dashv\vdash \varphi$.
  \item[\it $\wedge$-Associativity \textsc{($\wedge$AS):}] $\varphi\wedge(\psi\wedge\chi) \dashv\vdash (\varphi\wedge\psi)\wedge\chi$.
  \item[\it $\vee$-Associativity \textsc{($\vee$AS):}] $\varphi\vee(\psi\vee\chi) \dashv\vdash (\varphi\vee\psi)\vee\chi$.
\end{enumerate}


    








\iffalse

% \item[\bf (P5)] Let $\bot=A\wedge\neg A$ and assume $\varphi\vdash_{SD}\bot$ and $\psi\vdash_{SD}\bot$ where SD is our system of natural deduction. Complete the following tasks:
%   \begin{itemize}
%     \item[(I)] Show that $\varphi\dashv\vdash\psi$. [10pt]


 \textit{Using induction}, show that you cannot build a formula of propositional logic that is a contradiction using horseshoe (\eif) as your \textit{only} connective (i.e. every contradiction in SL contains at least one other connective besides horseshoe). [$20$ points] \\

-- Hint: call a formula ``\textbf{calm}'' if: (a) it is not a contradiction or (b) it
contains a ``\textit{stressed-connective}'' tilde (\enot), a vel (\eor), an ampersand $(\eand)$, or a biconditional (\eiff). For shorthand, you may call these four non-horseshoe connectives the ``stressed-connectives''. Show by induction that every formula of propositional logic is ``calm'', and explain how this establishes what is to be proven. \\

-- You may use our ``lazy induction schema'' for SL. \textit{Keep calm and logic on!} \\

\makebox[\textwidth]{\textbf{REMINDERS for when (you think) you're done}:}

\item[] If time remains, check your work for silly mistakes!!!

\item[] DON'T LEAVE ANY QUESTION BLANK!!!! Plz WRITE SOMETHING, so that you can be awarded partial credit. 

\item[] Make sure that you have answered EACH part of EACH question

\item[] Make sure you actually clicked `Submit' on Carnap for EACH problem!

\item[] Make sure you've written \textbf{YOUR NAME} on any looseleaf \\ \makebox[\textwidth]{(and username if your first name is `Daniel')}

\item[] Make sure you actually finished the full proof above after proving that every SL sentence is calm! (or at least explain HOW you \textit{would} finish the proof if you had successfully shown that every SL sentence is calm)

%\item Prove that in a sound tree system, no argument G is both a tree-tautology and a tree-contradiction. Let `G' be an arbitrary argument with finite premise set $\Gamma$ and conclusion $\Theta$. 

%(\textit{Hints}: Show that if an argument is tree-invalid, then it is not tree-valid. To do this, apply a key fact coming from our proof of completeness, and then apply the soundness result. Alternatively, you could do a proof by contradiction, using the same results.) 
% (but you'll still need soundness and the same key fact from our completeness proof. We stand here face to face with the hardness of the logical must). 

%\makebox[\textwidth]{\textbf{Solution}:}

%Suppose that argument G is tree-invalid. Then by definition, there exists at least one tree with root $\Gamma \cup\{\enot \Theta\} $ that has a complete open branch. Call this complete open branch `Oprah.' From our proof of completeness, we can use Oprah to define a truth value assignment that satisfies the root. This would be a TVA that makes the premises true but the conclusion false. Hence, the argument G must be semantically invalid, i.e. $\Gamma \nentails \Theta$. Since the system is Sound, we can apply the contrapositive of soundness to conclude that \textit{it is not the case that} G is tree-valid, i.e. $\Gamma \nvdash_{STD} \Theta$. 

%Hence, if an argument is tree-invalid, then it is not tree-valid (assuming our system is sound and complete). Hence, no argument is both tree-invalid and tree-valid.

%\makebox[\textwidth]{\textit{Alternative solution: proof by contradiction}:}

%Assume for contradiction that there is some argument G that is both tree-valid and tree-invalid. By soundness, G is semantically valid. Hence, whenever the premises $\Gamma$ are true, so is the conclusion $\Theta$. Hence, $\Gamma \cup\{\enot \Theta\} $ is inconsistent. 

%Yet, if G is tree-invalid, then there exists a tree with root $\Gamma \cup\{\enot \Theta\} $ that has a complete open branch. From our proof of completeness, we saw how to use such a branch to construct a TVA where the root is satisfied. So this TVA would satisfy $\Gamma \cup\{ \enot \Theta\} $. But that contradicts that $\Gamma \cup\{ \enot \Theta\} $ is unsatisfiable! Hence, G cannot also be tree-invalid. 

% \newpage
%
% \item Call a string over $\{b, e\}$ a ``4-beeb palindrome" if it is (i) a palindrome that has ``$beeb$" as the middle four letters and (ii) has a string-length that is divisible by four (so a 4-beeb palindrome can have $4, 8, 12, 16, \dots 4 \times n, \dots$ many letters) [$15+15 = 30$ points]
% % You can assume the language has an empty string, or that it doesn't, but state which option you are going with, as the answer %will be slightly different in each case.\\
%
% \textbf{(i) Give a recursive definition} of the set of ``4-beeb palindromes", labeling each of the relevant clauses. [$15$ points]
%
% \textbf{(ii) prove by induction} that every 4-beeb palindrome has an even number of ``$b$"'s. Please state your \textbf{induction hypothesis} clearly! [$15$ points] \\

%   \item Test the following argument for validity \textit{using the \textbf{tree} method}, labelling all relevant parts of your tree. Briefly justify your answer. [$20$ points]:
% \begin{align*}
% (\enot B \eand \enot D) \eif \enot A \\ 
% %(\enot C \eor H) \eand  (H \eif \enot H)\\
% - - - - - - - - - -\\
% \therefore \; [A \eif (B \eor D)]
% \end{align*}
% \\ (feel free to use symbols `P', `Q', and conclusion `C' if you don't feel like writing Greek letters or worry about your handwriting!)
%Let $\Phi$, \Psi , and \Theta be wffs 
%From GB 303, HW 4

%\makebox[\textwidth]{\textbf{Solution}:}



\item (i) Translate the following argument into the language of sentential logic. (ii) Check its validity using a tree, and state your conclusion. If the argument is invalid, use the tree to find a truth value assignment that makes its premises true and conclusion false.

\begin{quote}
If logic monkeys are hirsute, then logic monkeys are orgulous. And if space dogs are splenetic, then space dogs are bilious. So both if logic monkeys are hirsute then space dogs are bilious, and if space dogs are splenetic then logic monkeys are orgulous. 
\end{quote}

Symbolization Key: H = logic monkeys are hirsute; O = logic monkeys are orgulous; S = space dogs are splenetic; B = space dogs are bilious

\fi 



\end{document}
