%!TEX root = forallx-ubc.tex
%\pagestyle{plain}
\pagestyle{appendixstyle}
% change margins so that all the rules will fit
\setlength{\topmargin}{0 in}
\setlength{\headheight}{0 in}
\setlength{\headsep}{0 in}
\setlength{\textheight}{9 in}
\setlength{\evensidemargin}{0.25 in}
\setlength{\oddsidemargin}{0.25 in}
\setlength{\textwidth}{6 in}

%\pagenumbering{gobble}

\chapter[Quick Reference]{Quick Reference}
\label{app.quickreference}

\section*{Characteristic Truth Tables}
\label{app.CharacteristicTTs}

%\hfill
%\begin{tabular}{c|c}
%\metaA{} & \enot\metaA{}\\
%\hline
%T & F\\
%F & T 
%\end{tabular}
%\hfill
%\begin{tabular}{c|c|c|c|c|c}
%\metaA{} & \metaB{} & \metaA{}\eand\metaB{} & \metaA{}\eor\metaB{} & \metaA{}\eif\metaB{} & \metaA{}\eiff\metaB{}\\
%\hline
%T & T & T & T & T & T\\
%T & F & F & T & F & F\\
%F & T & F & T & T & F\\
%F & F & F & F & T & T
%\end{tabular}
%\hfill

%\vfill
\begin{tabular}{l l}
\begin{tabular}[t]{c|c}
\metaA{} & \enot\metaA{}\\
\hline
1 & 0\\
0 & 1\\
\end{tabular}
&
\begin{tabular}[t]{c|c|c|c|c|c}
\metaA{} & \metaB{} & \metaA{}\eand\metaB{} & \metaA{}\eor\metaB{} & \metaA{}\eif\metaB{} & \metaA{}\eiff\metaB{}\\
\hline
1 & 1 & 1 & 1 & 1 & 1\\
1 & 0 & 0 & 1 & 0 & 0\\
0 & 1 & 0 & 1 & 1 & 0\\
0 & 0 & 0 & 0 & 1 & 1
\end{tabular}\\
\end{tabular}
\hfill

%\vfill

\section*{Symbolization}
%\begin{center}
\label{app.symbolization}
\begin{tabular*}{\textwidth}{rl}
\multicolumn{2}{l}{\textsc{Sentential Connectives} (chapter \ref{ch.SL})}\\ \\
It is not the case that $P$. & $\enot P$\\
Either $P$, or $Q$. & $(P \eor Q)$ (remember that this is INCLUSIVE `or') \\
Neither $P$, nor $Q$. & $\enot(P \eor Q)$\ or \ $(\enot P \eand \enot Q)$\\
Both $P$, and $Q$. & $(P \eand Q)$\\
If $P$, then $Q$. & $(P \eif Q)$\\
$P$ \textbf{only} if $Q$. & $(P \eif Q)$\ or \ $(\enot Q \eif \enot P)$ (i.e. the contrapositive) \\
$P$ if and only if $Q$. & $(P \eiff Q)$\\
$B$ if $A$. & $(A \eif B)$ (thing after solitary `if' is antecedent!) \\
Unless $P$, $Q$. $P$ unless $Q$. & $(P \eor Q)$ (other ones are harder to remember!) \\
\\
\multicolumn{2}{l}{\label{SymbolizingPredicates}\textsc{Predicates} (chapter \ref{ch.QL})}\\ \\
All $F$s are $G$s. & $\qt{\forall}{x}(Fx \eif Gx)$\\
Some $F$s are $G$s. & $\qt{\exists}{x}(Fx \eand Gx)$\\
Not all $F$s are $G$s. & $\enot\qt{\forall}{x}(Fx \eif Gx)$\ or\ $\qt{\exists}{x}(Fx \eand \enot Gx)$\\
No $F$s are $G$s. & $\qt{\forall}{x}(Fx \eif\enot Gx)$\ or\ $\enot\qt{\exists}{x}(Fx \eand Gx)$\\
\\
\multicolumn{2}{l}{\textsc{Identity} (chapter \ref{ch.identity})}\\ \\
Only $j$ is a $G$. & $\qt{\forall}{x}(Gx \eiff (x{=}j))$\\
Everything besides $j$ is a $G$. & $\qt{\forall}{x}(x {\neq} j \eif Gx)$\\
%$j$ is more $R$ than anyone else. & $\qt{\forall}{x}(x\neq j \eif Rjx)$\\
The $F$ is $G$. & $\qt{\exists}{x}(Fx \eand \qt{\forall}{y}(Fy \eif x{=}y) \eand Gx)$\\
The $F$ is not a $G$\scriptsize{(-Unit!)}. & $\qt{\exists}{x}(Fx \eand \qt{\forall}{y}(Fy \eif x{=}y) \eand \enot Gx)$\\
%
%\multicolumn{2}{l}{`The F is not G' can be translated two ways:} \\
%It is not the case that the F is G. (wide)& $\enot\qt{\exists}{x}(Fx \eand \qt{\forall}{y}(Fy \eif x{=}y) \eand Gx)$\\
%The $F$ is non-$G$. (narrow) & $\qt{\exists}{x}(Fx \eand \qt{\forall}{y}(Fy \eif x{=}y) \eand \enot Gx)$
\end{tabular*}
%\end{center}


\newpage

\label{Tree Rules}
%{\hfill \normalsize \bf \phantom{Tree Rules} \hfill}
{\hfill \normalsize \bf Tree Rules for Sentential Logic (System STD) \hfill}

%page on how to override line numbers: 
% https://tex.stackexchange.com/questions/415976/manually-set-line-numbers-if-prooftrees-sty


%\textbf{\large Sentential Rules} (chapter \ref{ch.sl.trees})

%note that could put vdots in line number for vdots row, but prints with a period afterwards, so looks weird and two tight to the m. 

\makebox[\textwidth]{\textit{Double Negation} (\enot)} \vspace{-1em}
\begin{center}
\begin{prooftree}
{line numbering}
[\enot\enot\metaA{}, line no override={m}
[\vdots, vdotsline={\\[-0.55em] \vdots}, grouped
	[\metaA{}, line no override={j}, just={m \enot}
	]
	]
]
\end{prooftree}
\end{center}

%\vphantom{v}\\ \vdots
%\phantom{v} \\

\vspace{1em}

\begin{multicols}{2}

%row1col1
%\begin{quote}\framebox{\parbox[][75px][c]{\linewidth}{
\textit{Conjunction} (\eand) \vspace{-1em}
\begin{center}
%\textit{Conjunction} (\eand) %\vspace{-1em}
\begin{prooftree} % CONJUNCTION 
{line numbering}
[\metaA{}\eand\metaB{}, line no override={m}
[\vdots, vdotsline={\\[-0.55em] \vdots}, grouped
	[\metaA{}, line no override={j}, just={m \eand}
	[\metaB{}, grouped, line no override={j+1}
	]
	]
	]
]
\end{prooftree}
\end{center}
%}}
%\end{quote}

\phantom{vvv}

\textit{Disjunction} (\eor) \vspace{-1em}
\begin{center}
\begin{prooftree}
{line numbering, single branches}
[\metaA{}\eor\metaB{}, line no override={m}
[\vdots, vdotsline={\\[-0.55em] \vdots}, grouped
	[\metaA{}, line no override={j}, just={m \eor}]
	[\metaB{}]
]
]
\end{prooftree}
\end{center}



\phantom{vvv}

\textit{Conditional} (\eif) \vspace{-1em}
\begin{center}
\begin{prooftree}
{line numbering, single branches}
[\metaA{}\eif\metaB{}, line no override={m}
[\vdots, vdotsline={\\[-0.55em] \vdots}, grouped
	[\enot\metaA{}, line no override={j}, just={m \eif}]
	[\metaB{}]
]
]
\end{prooftree}
\end{center}

%\vspace{0.5em}
\phantom{vvv}

\textit{Biconditional} (\eiff) \vspace{-1em}
\begin{center}
\begin{prooftree}
{line numbering, single branches}
[\metaA{}\eiff\metaB{}, line no override={m}
[\vdots, vdotsline={\\[-0.55em] \vdots}, grouped
	[\metaA{}, line no override={j}, just={m \eiff}
		[\metaB{}, grouped, line no override={j+1}]
	]
	[\enot\metaA{}
		[\enot\metaB{}, grouped]
	]
]
]
\end{prooftree}
\end{center}

%\vfill \null


\columnbreak
%***************************NEGATED CONNECTIVES****************************************

\textit{Negated Conjunction} (\enot \eand) \vspace{-1em}
\begin{center}
\begin{prooftree}
{line numbering, single branches}
[\enot(\metaA{}\eand\metaB{}), line no override={m}
[\vdots,vdotsline={\\[-0.55em] \vdots}, grouped
	[\enot\metaA{}, line no override={j}, just={m \enot \eand}]
	[\enot\metaB{}]
]
]
\end{prooftree}
\end{center}

\vspace{3.3em}

\textit{Negated Disjunction} (\enot \eor) \vspace{-1em}
\begin{center}
\begin{prooftree} %NEGATED DISJUNCTION 
{line numbering, single branches}
[\enot(\metaA{}\eor\metaB{}), line no override={m}
[\vdots, vdotsline={\\[-0.55em] \vdots}, grouped
	[\enot\metaA{}, line no override={j}, just={m \enot \eor}
	[\enot\metaB{}, grouped,  line no override={j+1}
	]
	]
]
]
\end{prooftree}
\end{center}

\vspace{1em}
%\phantom{vvv}

%\iffalse
\textit{Negated Conditional} (\enot \eif) \vspace{-1em}
\begin{center}
\begin{prooftree}
{line numbering, single branches}
[\enot(\metaA{}\eif\metaB{}), line no override={m}
[\vdots, vdotsline={\\[-0.55em] \vdots}, grouped
	[\metaA{}, line no override={j}, just={m \enot \eif}
	[\enot\metaB{}, grouped, line no override={j+1}
	]
	]
]
]
\end{prooftree}
\end{center}
%\fi 

\vspace{1em}

\textit{Negated Biconditional} (\enot \eiff) \vspace{-1em}
\begin{center}
\begin{prooftree}
{line numbering, single branches}
[\enot(\metaA{}\eiff\metaB{}), line no override={m}
[\vdots, vdotsline={\\[-0.55em] \vdots}, grouped
	[\metaA{}, line no override={j}, just={m \enot \eiff}
		[\enot\metaB{}, grouped]
	]
	[\enot\metaA{}
		[\metaB{}, grouped, line no override={j+1}]
	]
]
]
\end{prooftree}
\end{center}

%\vfill \null
%\columnbreak 
\end{multicols}





\iffalse
%\begin{groupitems}
	True.

	\begin{prooftree}
	{
	to prove={\{P, P \eif Q, Q \eif \enot P\} \vdash{}\bot},
	}
	[P, name=p1, just=Ass, highlight line no
	[P \eif Q, checked, name=p2, grouped, highlight wff
	[Q \eif \enot P, checked, name=p3, grouped, line no override={n+1}
		[\enot P, close]
		[Q
			[\enot Q, close]
			[\enot P, close]
		]
	]
	]
	]
	\end{prooftree}
%	\end{groupitems}
\fi 

%trying to figure out where to put line no override={n}









\iffalse %Ichikawa versions

\begin{multicols}{5}

%row1col1
%\begin{quote}\framebox{\parbox[][75px][c]{\linewidth}{
\begin{center}
\begin{prooftree}
{not line numbering}
[\metaA{}\eand\metaB{}
	[\metaA{}
	[\metaB{}, grouped
	]
	]
]
\end{prooftree}
\end{center}
%}}
%\end{quote}



%row2col1 - not or
%\begin{quote}\framebox{\parbox[][75px][c]{\linewidth}{
\begin{center}
\begin{prooftree}
{not line numbering,
single branches}
[\enot(\metaA{}\eor\metaB{})
	[\enot\metaA{}
	[\enot\metaB{}, grouped
	]
	]
]
\end{prooftree}
\end{center}
%}}
%\end{quote}

%row3col1 - iff
%\begin{quote}\framebox{\parbox[][75px][c]{\linewidth}{
\begin{center}
\begin{prooftree}
{not line numbering,
single branches}
[\metaA{}\eiff\metaB{}
	[\metaA{}
		[\metaB{}, grouped]
	]
	[\enot\metaA{}
		[\enot\metaB{}, grouped]
	]
]
\end{prooftree}
\end{center}
%}}
%\end{quote}

%row1col2
%\begin{quote}\framebox{\parbox[][75px][c]{\linewidth}{
\begin{center}
\begin{prooftree}
{not line numbering,
single branches}
[\enot(\metaA{}\eand\metaB{})
	[\enot\metaA{}]
	[\enot\metaB{}]
]
\end{prooftree}
\end{center}
%}}
%\end{quote}

%row2col2 - if
%\begin{quote}\framebox{\parbox[][75px][c]{\linewidth}{
\begin{center}
\begin{prooftree}
{not line numbering,
single branches}
[\metaA{}\eif\metaB{}
	[\enot\metaA{}]
	[\metaB{}]
]
\end{prooftree}
\end{center}
%}}
%\end{quote}

%row3col2 - not iff
%\begin{quote}\framebox{\parbox[][75px][c]{\linewidth}{
\begin{center}
\begin{prooftree}
{not line numbering,
single branches}
[\enot(\metaA{}\eiff\metaB{})
	[\metaA{}
		[\enot\metaB{}, grouped]
	]
	[\enot\metaA{}
		[\metaB{}, grouped]
	]
]
\end{prooftree}
\end{center}
%}}
%\end{quote}

%row1col3 - or
%\begin{quote}\framebox{\parbox[][75px][c]{\linewidth}{
\begin{center}
\begin{prooftree}
{not line numbering,
single branches}
[\metaA{}\eor\metaB{}
	[\metaA{}]
	[\metaB{}]
]
\end{prooftree}
\end{center}
%}}
%\end{quote}

%row2col3 - not if
%\begin{quote}\framebox{\parbox[][75px][c]{\linewidth}{
\begin{center}
\begin{prooftree}
{not line numbering,
single branches}
[\enot(\metaA{}\eif\metaB{})
	[\metaA{}
	[\enot\metaB{}, grouped
	]
	]
]
\end{prooftree}
\end{center}
%}}
%\end{quote}

%row3col3 dn
%\begin{quote}\framebox{\parbox[][75px][c]{\linewidth}{
\begin{center}
\begin{prooftree}
{not line numbering}
[\enot\enot\metaA{}
	[\metaA{}]
]
\end{prooftree}
\end{center}
%}}
%\end{quote}



\end{multicols}


\fi 

\newpage

\textbf{\large Quantifier Rules} (chapter \ref{ch.QL})

\begin{multicols}{4}


%\factoidbox{
\begin{center}
\begin{prooftree}
{not line numbering}
[\qt{\exists}{\script{x}}\metaA{}, checked={\script{a}}
	[\metaA{}\substitute{\script{x}}{\script{a}}]
]
\end{prooftree}

*\script{a} must be new
\end{center}
%}


%\factoidbox{
\begin{center}
\begin{prooftree}
{not line numbering}
[\qt{\forall}{\script{x}}\metaA{}, subs=\script{a} 
	[\metaA{}\substitute{\script{x}}{\script{a}}
	]
]
\end{prooftree}

for any \script{a}
\end{center}
%}

%\factoidbox{
\begin{center}
\begin{prooftree}
{not line numbering}
[\enot\qt{\exists}{\script{x}}\metaA{}, subs=\script{a} 
	[\enot\metaA{}\substitute{\script{x}}{\script{a}}
	]
]
\end{prooftree}

for any \script{a}
\end{center}
%}

\label{notall-treerule}
%\factoidbox{
\begin{center}
\begin{prooftree}
{not line numbering}
[\enot\qt{\forall}{\script{x}}\metaA{}, checked={\script{a}}
	[\enot\metaA{}\substitute{\script{x}}{\script{a}}
	]
]
\end{prooftree}

*\script{a} must be new
\end{center}
%}

\end{multicols}

\textbf{\large Identity Rules} (chapter \ref{ch.identity})


If a branch contains an identity claim \script{a}{=}\script{b} and some sentence \metaA{}, then you may add \metaA{}\substitute{\script{a}}{\script{b}} or \metaA{}\substitute{\script{b}}{\script{a}} to the branch.

A branch is \define{complete} if and only if (i) it is closed or (ii) the following conditions are met:
\begin{enumerate}
\item Every resolvable sentence in the branch has been resolved;
\item For every general quantified sentence \metaA{} and every name \script{a} in the branch, the \script{a} instance of \metaA{} is in the branch; and
\item For every identity claim $\script{a}{=}\script{b}$ in the branch, either:
	\begin{enumerate}
	\item for every atomic sentence or negated atomic sentence \metaA{} containing \script{a} in the branch, \metaA{}\substitute{\script{a}}{\script{b}} is in the branch, or 
	\item for every atomic sentence or negated atomic sentence \metaA{} containing \script{b} in the branch, \metaA{}\substitute{\script{b}}{\script{a}} is in the branch.
	\end{enumerate}
\end{enumerate}

A branch closes if and only if either (i) it contains some sentence \metaA{} along with its negation, \enot\metaA{}, or (ii) it contains, for some name \script{a}, the sentence $\script{a}{\neq}\script{a}$.
\newpage

% This starts a new page and skips a page if necessary so as
% to start on an even numbered page.
% That way, the rules of proof will be on facing pages.
% It fills it in with a somewhat gratuitous reference table.
%\ifthenelse{\isodd{\thepage}}{
%	\ \vspace{2 in}\par\centerline{[ This page intentionally left blank. ]}
%\begin{table}
%	Sometimes it is easier to show something by providing proofs than it is by providing models. Sometimes it is the other way round.
%	\begin{center}
%	\begin{tabular*}{\textwidth}{p{10em}|p{10em}|p{10em}|}
%	\cline{2-3}
%	 & {\centerline{YES}} & {\centerline{NO}}\\
%	\cline{2-3}
%	Is \metaA{} a tautology? & prove $\vdash\metaA{}$ & give a model in which \metaA{} is false\\
%	\cline{2-3}
%	Is \metaA{} a contradiction? &  prove $\vdash\enot\metaA{}$ & give a model in which \metaA{} is true\\
%	\cline{2-3}
%	Is \metaA{} contingent? & give a model in which \metaA{} is true and another in which \metaA{} is false & prove $\vdash\metaA{}$ or $\vdash\enot\metaA{}$\\
%	\cline{2-3}
%	Are \metaA{} and \metaB{} equivalent? & prove \mbox{$\metaA{}\vdash\metaB{}$} and \mbox{$\metaB{}\vdash\metaA{}$}  & give a model in which \metaA{} and \metaB{} have different truth values\\
%	\cline{2-3}
%	Is the set \model{A} consistent? & give a model in which all the sentences in \model{A} are true & taking the sentences in \model{A}, prove \metaB{} and \enot\metaB{}\\
%	\cline{2-3}
%	Is the argument \mbox{`\script{P}, \therefore\ \metaC{}'} valid? & prove $\script{P}\vdash\metaC{}$ & give a model in which \script{P} is true and \metaC{} is false\\
%	\cline{2-3}
%	\end{tabular*}
%	\end{center}
%\end{table}
%	\newpage
%}{}
% eliminate page numbers





%\pagestyle{empty} %turns off page numbers!



\section*{\hfill \normalsize \bf Derivation Rules of Natural Deduction (System SND) \hfill} 
\label{nd-proofrules}
				
				\vspace{-1em}
				
				\iffalse %suppressing Rule R to fit others on page
\begin{center}
%\textsc{Reiteration}
\textit{Reiteration} (R) \vspace{-1em}
\begin{proof}
	\have[m]{a}{\metaA{}}
	\have[\ ]{c}{\metaA{}} \by{R}{a}
\end{proof}
\end{center}
\fi 

\begin{multicols}{2}


\textit{Conjunction Introduction} (\eand I) \vspace{-1em}
\begin{proof}
	\have[m]{a}{\metaA{}}
	\have[n]{b}{\metaB{}}
	\have[\ ]{c}{\metaA{}\eand\metaB{}} \ai{a, b}
	\have[\ ]{d}{\metaB{}\eand\metaA{}} \ai{a, b}
\end{proof}

\vspace{1em}

\textit{Conditional Introduction} (\eif I) \vspace{-1em}
%\nopagebreak
\begin{proof}
	\open
		\hypo[m]{a}{\metaA{}} \as{for \eif I}{}%\by{want \metaB{}}{}
		\have[n]{b}{\metaB{}}
	\close
	\have[\ ]{ab}{\metaA{}\eif\metaB{}}\ci{a-b}
\end{proof}

\vspace{0.6em}

\textit{Negation Introduction} (\enot I) \vspace{-1em}
\begin{proof}
\open
	\hypo[m]{na}\metaA{} \as{for \enot I}   %\by{:AS for \enot I}{}
	\have[n]{b}\metaB{}
	\have[o]{nb}{\enot\metaB{}}
\close
\have[\ ]{a}[\ ]{\enot\metaA{}}\ni{na-nb}
\end{proof}

\vspace{0.6em}

\textit{Disjunction Introduction} (\eor I) \vspace{-1em}

\begin{proof}
	\have[m]{a}{\metaA{}}
	\have[\ ]{ab}{\metaA{}\eor\metaB{}}\oi{a}
	\have[\ ]{ba}{\metaB{}\eor\metaA{}}\oi{a}
\end{proof}

%\vspace{1.9em} %3.9 for no extra vspaces
\vspace{0.6em}

\textit{Biconditional Introduction} (\eiff I) \vspace{-1em}

\begin{fitchproof}
	\open
		\hypo[i]{a1}{\metaA{}} \as{for \eiff I}
		\have[j]{b1}{\metaB{}}
	\close
\breakline
	\open
		\hypo[k]{b2}{\metaB{}} \as{for \eiff I}
		\have[l]{a2}{\metaA{}}
	\close
	\have[\ ]{ab}{\metaA{}\eiff\metaB{}}\bi{a1-b1,b2-a2}
\end{fitchproof}

\iffalse %Ichikawa version:
\begin{proof}
	\have[m]{ab}{\metaA{}\eif\metaB{}}
	\have[n]{ba}{\metaB{}\eif\metaA{}}
	\have[\ ]{c}{\metaA{}\eiff\metaB{}} \bi{ab, ba}
\end{proof}
\fi 

\vfill\null
\columnbreak

%\newpage

\textit{Conjunction Elimination} (\eand E) \vspace{-1em}

\begin{proof}
	\have[m]{ab}{\metaA{}\eand\metaB{}}
	\have[\ ]{a}{\metaA{}} \ae{ab}
	\have[\ ]{b}{\metaB{}} \ae{ab}
\end{proof}

%\vspace{1.9em}
%\vspace{2.9em}
\vspace{0.75em}

\textit{Conditional Elimination} (\eif E)  \vspace{-1em}

\begin{proof}
	\have[m]{ab}{\metaA{}\eif\metaB{}}
	\have[n]{a}{\metaA{}}
	\have[\ ]{b}{\metaB{}} \ce{ab,a}
\end{proof}

%\vspace{1em}
\vspace{0.45em}

\textit{Negation Elimination} (\enot E)  \vspace{-1em}
%%note that I think I'm missing some brackets around the sentences on various liens below! works in proof environment but less robust in nd environment. so i ought to fix these here and in negation intro 

\begin{proof}
\open
	\hypo[m]{na}{\enot\metaA{}} \as{for \enot E}
	\ellipsesline
	\have[n]{b}\metaB{}
	\have[o]{nb}{\enot\metaB{}}
\close
\have[\ ]{a}[\ ]\metaA{}\ne{na-nb}
\end{proof}

\vspace{1em}

\textit{Disjunction Elimination} (\eor E)  \vspace{-1em}

\begin{proof}
\have[m]{ab}{\metaA{}\eor\metaB{}}
	\open
		\hypo[i]{a}{\metaA{}} \as{for \eor E}
		\have[j]{c1}{\metaC{}}
	\close
\breakline
	\open
		\hypo[k]{b}{\metaB{}} \as{for \eor E}
		\have[l]{c2}{\metaC{}}
	\close
	\have[\ ]{c}{\metaC{}} \oe{ab,a-c1, b-c2}
\end{proof}


\iffalse %Ichikawa Disjunction Syllogism version:
\begin{proof}
	\have[m]{ab}{\metaA{}\eor\metaB{}}
	\have[n]{nb}{\enot\metaB{}}
	\have[\ ]{a}{\metaA{}} \oe{ab,nb}
\end{proof}

\begin{proof}
	\have[m]{ab}{\metaA{}\eor\metaB{}}
	\have[n]{na}{\enot\metaA{}}
	\have[\ ]{b}{\metaB{}} \oe{ab,nb}
\end{proof}

\fi 


%\pagebreak



\textit{Biconditional Elimination} (\eiff E) \vspace{-1em}

\begin{proof}
	\have[m]{ab}{\metaA{}\eiff\metaB{}}
	\have[n]{a}{\metaB{}}
	\have[\ ]{b}{\metaA{}} \be{ab,a}
\end{proof}
%Can also go from \metaA{} to \metaB{} using \eiff E

\iffalse
\begin{proof}
	\have[m]{ab}{\metaA{}\eiff\metaB{}}
	\have[n]{a}{\metaA{}}
	\have[\ ]{b}{\metaB{}} \be{ab,a}
\end{proof}
\fi 



\end{multicols} %ends multicol started for SD rules 

%\pagestyle{empty}

\section*{\hfill \normalsize \bf Derivation Rules of Natural Deduction System QND \hfill} 
\label{nd-proofrules}
				
				\vspace{-1em}
				
All the rules of SND, plus the following rules for the quantifiers! The rules of SND govern inferences where the main logical operator is one of the connectives from {\it{SL}}. Reiteration is also allowed.

%JH preferred substitution notation, from Logic Book: $\metaA{}[\script{c} / \script{x}]$
%Note that you can modify \substitute command in the style file to toggle styles. 

Recall: $\metaA{}\substitute{\script{x}}{\script{c}}$ is the sentence you get from $\qt{\forall}{\script{x}}\metaA{}$ by dropping the $\qt{\forall}{\script{x}}$ quantifier and putting $\script{c}$ in place of every $\script{x}$ in $\metaA{}$. The other variables are untouched! Equivalent notation: \metaA{}\hspace{.15em}\raisebox{.3ex}{\fbox{$\script{x}\Rightarrow\script{c}$}}. But we'll write things out long-hand in the schemata below! 

% (${\bm{P[a/x]}}$  is the sentence you get from $(\qt{\forall}{x}) {\bm{P}}$ by dropping the $(\qt{\forall}{x}) $ quantifier and putting ${\bf{a}}$ in the place of every $x$ in ${\bm{P}}$)



\begin{multicols}{2}

\textit{Universal Elimination} ($\forall$E) 

\begin{proof}
	\have[m]{a}{\qt{\forall}{\script{x}}\metaA{}(\dots \script{x} \dots \script{x} \dots)}
	\have [\vdots] {n} {\hspace{2em} \vdots}
	\have[s]{c}{\metaA{}(\dots \script{c} \dots \script{c} \dots)} \Ae{a}
	%\have[\ ]{c}{\metaA{}\substitute{\script{x}}{\script{c}}} \Ae{a}
\end{proof}

\phantom{We replace every $\script{x}$ variable  in \metaA{} with the same constant $\script{c}$. I.e. we go from $\qt{\forall}{\script{x}}\metaA{}$ to $\metaA{}[\script{c} / \script{x}]$. Other variables are untouched.} 

\vspace{2.5em}

\textit{Universal Introduction} ($\forall$I) 

\begin{proof}
	\have[m]{a}{\metaA{}(\dots \script{c} \dots \script{c} \dots)}
	\have [\vdots] {n} {\hspace{2em} \vdots}
	\have[s]{c}{\qt{\forall}{\script{x}}\metaA{}(\dots \script{x} \dots \script{x} \dots)} \Ai{a}
\end{proof}

\textbf{Provided that both} \\
\textbf{(i)} $\script{c}$ does not occur in any undischarged assumptions that \metaA{} is in the scope of. \\
\textbf{(ii)} $\script{x}$ does not occur already in $\metaA{}(\dots \script{c} \dots \script{c} \dots)$.
%I think the following is a mistake: $\script{c}$ does not occur already in $\forall \script{x}\metaA{}$. 



\vfill\null
\columnbreak

\textit{Existential Introduction} ($\exists$I)

\begin{proof}
	\have[m]{a}{\metaA{}(\dots \script{c} \dots \script{c} \dots)}
	\have [\vdots] {n} {\hspace{2em} \vdots}
	\have[s]{c}{\qt{\exists}{\script{x}}\metaA{}(\dots \script{x} \dots \script{x} \dots)} \Ei{a}
\end{proof}

\textbf{Provided that} $\script{x}$ does not occur already in $\metaA{}(\dots \script{c} \dots \script{c} \dots)$. Note that $\script{x}$ may replace some or all occurrences of $\script{c}$.





\textit{Existential Elimination} ($\exists$E)

\begin{proof}
	\have[m]{a}{\qt{\exists}{\script{x}} \metaA{}}
	\have [\ ] {n} {\hspace{2em} \vdots}
	\open	
		\hypo[n]{b}{\metaA{}(\dots \script{c} \dots \script{c} \dots)} \as{for $\exists$E}
		\have [\ ] {n2} {\hspace{2em} \vdots}
		\have[s]{c}{\metaB{}}
	\close
	%\have [\ ] {n3} {\hspace{2em} \vdots}
	\have[s+1]{d}{\metaB{}} \Ee{a,b-c}
\end{proof}

\textbf{Provided that} \\
(i)  $\script{c}$ does not occur in any undischarged assumptions that \metaA{} is in the scope of \\
(ii) $\script{c}$ does not occur already in $\qt{\exists}{\script{x}} \metaA{}$ \\
(iii) $\script{c}$ does not occur in \metaB{} \\

\footnotesize{Simplified restriction (easier to remember): \textbf{provided $\script{c}$ doesn't occur anywhere else outside the subproof}. (Moral: always use a distinct constant for existential elimination, and you'll satisfy the three restrictions above automatically!)}
%would be nice to shorten this to `provided c does not occur outside the subproof, but not sure if this is actually true! e.g. we could instantiate some other universal with c, and so have Fc outside the subproof. 

\end{multicols} %ends multicol started for QND rules 




\section*{\hfill \normalsize \bf Derivation Rules of Natural Deduction System QND \hfill} 
\label{nd-proofrules}
				
				\vspace{-1em}
				
All the rules of SND, plus the following rules for the quantifiers! The rules of SND govern inferences where the main logical operator is one of the connectives from {\it{SL}}. Reiteration is also allowed.

%JH preferred substitution notation, from Logic Book: $\metaA{}[\script{c} / \script{x}]$
%Note that you can modify \substitute command in the style file to toggle styles. 

Recall: $\metav{Q}\substitute{\script{x}}{\script{c}}$ is the sentence you get from $\qt{\forall}{\script{x}}\metav{Q}$ by dropping the $\qt{\forall}{\script{x}}$ quantifier and putting $\script{c}$ in place of every $\script{x}$ in $\metav{Q}$. The other variables are untouched! Equivalent notation: \metav{Q}\hspace{.15em}\raisebox{.3ex}{\fbox{$\script{x}\Rightarrow\script{c}$}}. But we'll write things out long-hand in the schemata below! 

% (${\bm{P[a/x]}}$  is the sentence you get from $(\qt{\forall}{x}) {\bm{P}}$ by dropping the $(\qt{\forall}{x}) $ quantifier and putting ${\bf{a}}$ in the place of every $x$ in ${\bm{P}}$)



\begin{multicols}{2}

\textit{Universal Elimination} ($\forall$E) 

\begin{proof}
	\have[m]{a}{\qt{\forall}{\script{x}}\metav{Q}}
	\have [\vdots] {n} {\hspace{2em} \vdots}
	\have[s]{c}{\metav{Q}\substitute{\script{x}}{\script{c}}} \Ae{a}
	%\have[\ ]{c}{\metaA{}\substitute{\script{x}}{\script{c}}} \Ae{a}
\end{proof}

\phantom{We replace every $\script{x}$ variable  in \metav{Q} with the same constant $\script{c}$. I.e. we go from $\qt{\forall}{\script{x}}\metav{Q}$ to $\metav{Q}[\script{c} / \script{x}]$. Other variables are untouched.} 

\vspace{2.5em}

\textit{Universal Introduction} ($\forall$I) 

\begin{proof}
	\have[m]{a}{\metav{Q}}
	\have [\vdots] {n} {\hspace{2em} \vdots}
	\have[s]{c}{\qt{\forall}{\script{x}}\metav{Q}\substitute{\script{c}}{\script{x}}} \Ai{a}
\end{proof}

\textbf{Provided that both} \\
\textbf{(i)} $\script{c}$ does not occur in any undischarged assumptions that \metav{Q} is in the scope of. \\
\textbf{(ii)} $\script{x}$ does not occur already in $\metav{Q}$.
%I think the following is a mistake: $\script{c}$ does not occur already in $\forall \script{x}\metaA{}$. 



\vfill\null
\columnbreak

\textit{Existential Introduction} ($\exists$I)

\begin{proof}
\have[m]{a}{\metav{Q}}
	\have [\vdots] {n} {\hspace{2em} \vdots}
\have[s]{c}{\qt{\exists}{\script{x}}\metav{Q}\substitutesome{c}{\script{x}}} \Ei{a}
\end{proof}

\textbf{Provided that} $\script{x}$ does not occur already in $\metav{Q}(\dots \script{c} \dots \script{c} \dots)$. Note that $\script{x}$ may replace some or all occurrences of $\script{c}$.





\textit{Existential Elimination} ($\exists$E)

\begin{fitchproof}
	\have[m]{a}{\qt{\exists}{\script{x}} \metav{Q}}
	\have [\ ] {n} {\hspace{2em} \vdots}
	\open	
		\hypo[n]{b}{\metav{Q}\substitute{\script{x}}{\script{c}}} \as{for $\exists$E}
		\have [\ ] {n2} {\hspace{2em} \vdots}
		\have[s]{c}{\metaB{}}
	\close
	%\have [\ ] {n3} {\hspace{2em} \vdots}
	\have[s+1]{d}{\metaB{}} \Ee{a,b-c}
\end{fitchproof}

\textbf{Provided that} \\
(i)  $\script{c}$ does not occur in any undischarged assumptions that \metav{Q} is in the scope of \\
(ii) $\script{c}$ does not occur already in $\qt{\exists}{\script{x}} \metav{Q}$ \\
(iii) $\script{c}$ does not occur in \metaB{} \\

\footnotesize{Simplified restriction (easier to remember): \textbf{provided $\script{c}$ doesn't occur anywhere else outside the subproof}. (Moral: always use a distinct constant for existential elimination, and you'll satisfy the three restrictions above automatically!)}
%would be nice to shorten this to `provided c does not occur outside the subproof, but not sure if this is actually true! e.g. we could instantiate some other universal with c, and so have Fc outside the subproof. 

\end{multicols} %ends multicol started for QND rules 







































%\newpage
\twocolumn %carries down to end of the document 

%{\large \bf Additional Rules}

{\large \bf Identity Rules}

\begin{proof}
	\have[\ \,\,\,]{x}{\script{c}{=}\script{c}} \by{=I}{}
\end{proof}

\begin{proof}
	\have[m]{e}{\script{c}{=}\script{d}}
	\have[n]{a}{\metaA{}}
	\have[\ ]{ea1}{\metaA{}\substitutesome{\script{c}}{\script{d}}} \by{=E}{e,a}
	\have[\ ]{ea2}{\metaA{}\substitutesome{\script{d}}{\script{c}}} \by{=E}{e,a}
\end{proof}

\begin{footnotesize}\emph{One constant may replace some or all occurrences of the other.}
\end{footnotesize}





%\newpage

{\large \bf Derived Rules}

\textsc{Dilemma}

\begin{proof}
	\have[m]{ab}{\metaA{}\eor\metaB{}}
	\have[n]{ac}{\metaA{}\eif\metaC{}}
	\have[p]{bc}{\metaB{}\eif\metaC{}}
	\have[\ ]{a}{\metaC{}} \by{DIL}{ab,ac,bc}
\end{proof}

\textsc{Modus Tollens}

\begin{proof}
	\have[m]{ab}{\metaA{}\eif\metaB{}}
	\have[n]{a}{\enot\metaB{}}
	\have[\ ]{b}{\enot\metaA{}} \by{MT}{ab,a}
\end{proof}

\textsc{Hypothetical Syllogism}

\begin{proof}
	\have[m]{ab}{\metaA{}\eif\metaB{}}
	\have[n]{bc}{\metaB{}\eif\metaC{}}
	\have[\ ]{ac}{\metaA{}\eif\metaC{}}\by{HS}{ab,bc}
\end{proof}

\newpage

{\large \bf Replacement Rules}

\begin{footnotesize}\emph{(Replacement rules may be used within complex sentences or on whole lines.)}\end{footnotesize}

\textsc{Commutativity} (Comm)\\
$(\metaA{}\eand\metaB{}) \Longleftrightarrow (\metaB{}\eand\metaA{})$\\
$(\metaA{}\eor\metaB{}) \Longleftrightarrow (\metaB{}\eor\metaA{})$\\
$(\metaA{}\eiff\metaB{}) \Longleftrightarrow (\metaB{}\eiff\metaA{})$

\textsc{DeMorgan} (DeM)\\
$\enot(\metaA{}\eor\metaB{}) \Longleftrightarrow (\enot\metaA{}\eand\enot\metaB{})$\\
$\enot(\metaA{}\eand\metaB{}) \Longleftrightarrow (\enot\metaA{}\eor\enot\metaB{})$

\textsc{Double Negation} (DN)\\
$\enot\enot\metaA{} \Longleftrightarrow \metaA{}$

\textsc{Material Conditional} (MC)\\
$(\metaA{}\eif\metaB{}) \Longleftrightarrow (\enot\metaA{}\eor\metaB{})$\\
$(\metaA{}\eor\metaB{}) \Longleftrightarrow (\enot\metaA{}\eif\metaB{})$

\textsc{Biconditional Exchange} ({\eiff}{ex})\\
$[(\metaA{}\eif\metaB{})\eand(\metaB{}\eif\metaA{})] \Longleftrightarrow (\metaA{}\eiff\metaB{})$

\textsc{Quantifier Negation} (QN)\\
$\enot\qt{\forall}{\script{x}}\metaA{} \Longleftrightarrow \qt{\exists}{\script{x}}\enot\metaA{}$\\
$\enot\qt{\exists}{\script{x}}\metaA{} \Longleftrightarrow \qt{\forall}{\script{x}}\enot\metaA{}$



%%%********************
%Some stuff from CALGARY REMIX QUICK REFERENCE, MUCH MORE EXTENSIVE


