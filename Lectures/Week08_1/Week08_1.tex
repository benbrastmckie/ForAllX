\documentclass[a4paper, 11pt]{article} % Font size (can be 10pt, 11pt or 12pt) and paper size (remove a4paper for US letter paper)
\usepackage[protrusion=true,expansion=true]{microtype} % Better typography
\usepackage{../lecture} %calls local modified style file
\usepackage{graphicx} % Required for including pictures
\usepackage{wrapfig} % Allows in-line images
\usepackage{enumitem} %%Enables control over enumerate and itemize environments
\usepackage{setspace}
\usepackage{amssymb, amsmath, mathrsfs} %%Math packages
\usepackage{stmaryrd}
\usepackage{mathtools}
\usepackage{multicol} 
\usepackage{mathpazo} % Use the Palatino font
\usepackage[T1]{fontenc} % Required for accented characters
\usepackage{array}
\usepackage{bibentry}
\usepackage{prooftrees} 
\usepackage[round]{natbib} %%Or change 'round' to 'square' for square backers
\setcitestyle{aysep=}

\makeatletter
\renewcommand{\maketitle}{
\begin{flushright}
{\LARGE\@title}

\vspace{10pt}

{\@author}
\\ \@date
\end{flushright}

\vspace{-20pt}

}
\makeatother

%----------------------------------------------------------------------------------------
%	TITLE
%----------------------------------------------------------------------------------------

\title{\textbf{The Semantics for $\FOL$}} % Subtitle

\author{\textsc{Logic I}\\ \em Benjamin Brast-McKie} % Institution

\date{\today} % Date

%----------------------------------------------------------------------------------------

\begin{document}

\maketitle % Print the title section

\thispagestyle{empty}

%----------------------------------------------------------------------------------------

\section*{Examples}%
  \label{sec:Examples}
  
\begin{enumerate}
  \item[\it Monadic:] Casey is dancing.
  \item[\it Dyadic:] Al loves Max.
  \item[\it Triadic:] Kim is between Boston and New York.
\end{enumerate}

\section*{Constants and Referents}%
  \label{sec:Extensions and Referents}
  
\begin{enumerate}
  \item[\it Constants:] Constants are interpreted as referring to individuals.
  \item[\it Existence:] Thus we need to know what things there are.
  \item[\it Domain:] A \textit{domain} is any nonempty set $\D$.
  \item[\it Referents:] Interpretations assign constants to elements of $\D$.
  \item[\bf Question 1:] How are we going to interpret predicates?
\end{enumerate}



\section*{Predicates and Extensions}%
  \label{sec:Extensions}
  
\begin{enumerate}
  \item[\it Example:] `Al loves Max' is true \textit{iff} Al bears the loves-relation to Max.
  \item[\it Dyadic Predicates:] Dyadic predicates are interpreted by sets of \textit{ordered pairs} in $\D^2$.
  \item[\bf Question 2:] How are we to interpret $n$-place predicates?
  \item[\it Cartesian Power:] $\D^n=\set{\tuple{\v{x}_1,\ldots,\v{x}_n}:\v{x}_1,\ldots,\v{x}_n\in\D}$.
  \item[\it Extensions:] $n$-place predicates are interpreted by subsets of $\D^n$.
  \item[\it Singletons:] $1$-place predicates are interpreted by subsets of $\D^1=\set{\tuple{\v{x}}:\v{x}\in\D}$.
  \item[\bf Question 3:] How are we to interpret $0$-place predicates? What is $\D^0$?
    \begin{itemize}
      % \item[\it Ordered Pairs:] An ordered pair $\tuple{\v{x}_1,\v{x}_2}=\set{\set{\v{x}_1},\set{\v{x}_1,\v{x}_2}}$.
      \item[\it $n$-Tuples:] Let $\tuple{\v{x}_1,\ldots,\v{x}_n}=\set{\tuple{1,\v{x}_1},\ldots,\tuple{n,\v{x}_n}}$.
      \item[\it $0$-Tuple:] $\tuple{}=\varnothing$.
        % \item\tuple{\tuple{\v{x}_1,\ldots,\v{x}_{n-1}},\v{x}_n}$.
        % \item\tuple{\v{x}_1,\tuple{\v{x}_2,\ldots,\tuple{\v{x}_n,\varnothing}}}$.
      % \item[\it $0$-Tuple:] $\tuple=\varnothing$ is the $0$-tuple.
        % \item $\tuple{\v{x}_1}=\set{1,\v{x}_1}$ is a $1$-tuple.
        % \item $\tuple{\v{x}_1,\ldots,\v{x}_n}=\tuple{\vec{\v{X}}_{n-1},\v{x}_n}$ is an $n+1$-tuple if $X_n$ is a $n$-tuple. 
      \end{itemize}
  \item[\it Truth-Values:] $0$-place predicates are interpreted by subsets of $\D^0=\set{\varnothing}$.
  \item[\it Ordinals:] Note that $0=\varnothing$ and $1=\set{\varnothing}$ are the first two von Neumann ordinals.
\end{enumerate}



\section*{$\FOL$ Models}

\begin{itemize}
  \item[\it Interpretations:] $\I$ is an $\FOL$ interpretation over $\D$ \textit{iff} both: 
    \item $\I(\alpha)\in\D$ for every constant $\alpha$ in $\FOL$. 
    \item $\I(\F^n)\subseteq\D^n$ for every $n$-place predicate $\F^n$.
  \item[\bf Question 4:] What happens if $\D=\varnothing$?
  \item[\it Model:] $\M=\tuple{\D,\I}$ is a model of $\FOL$ \textit{iff} $\I$ is an interpretation over $\D\neq\varnothing$.
  \item[\bf Task 1:] Regiment and interpret the sentences above.
    \item $Dc$, $Lam$, $Bkbn$.
    \item $\D=\set{c,a,m,k,b,n}$.
    \item $\I(D)=\set{\tuple{c}}$.
    \item $\I(L)=\set{\tuple{a,m}}$.
    \item $\I(B)=\set{\tuple{k,b,n}}$.
    \item $\I(c)=c,\ \I(a)=a,\ \ldots$
  \item[\it Lagadonian:] We often take constants to name themselves.
  \item[\bf Question 5:] Do models give us truth-values?
\end{itemize}



\section*{Variable Assignments}

\begin{enumerate}
  \item[\it Assignments:] A variable assignment (v.a.) $\hat{a}(\alpha)\in\D$ for every variable $\alpha$ in $\FOL$.
  \item[\it Referents:] We may define the referent of $\alpha$ in $\M=\tuple{\D,\I}$ as follows:
    \begin{align*}
      \val{\I}{\hat{a}}{(\alpha)}=
        \begin{cases}
          \I(\alpha) & \text{if } \alpha \text{ is a constant} \\
          \hat{a}(\alpha) & \text{if } \alpha \text{ is a variable.}
        \end{cases}
    \end{align*}
  \item[\it Variants:] A v.a. $\hat{c}$ is an $\alpha$-variant of $\hat{a}$ \textit{iff} $\hat{c}(\beta)=\hat{a}(\beta)$ for all $\beta\neq\alpha$.
  \item[\it Example:] Let $\D=\set{1,2,3,4,5}$ where $\hat{a}(x)=1$, $\hat{a}(y)=2$, and $\hat{a}(z)=3$.
    \begin{itemize}
      \item[\bf Task 2:] If $\hat{c}$ is a $y$-variant of $\hat{a}$, what is $\hat{c}(x)$, $\hat{c}(y)$, and $\hat{c}(z)$?
    \end{itemize}
\end{enumerate}



\section*{Example}

\begin{enumerate}
  \item[\it Universal:] Al loves everything, i.e., $\forall xLax$.
  \item[\it Existential:] Someone is dancing, i.e., $\exists x(Px \eand Dx)$.
  \item[\it Mixed:] Everyone loves someone, i.e., $\forall x(Px \eif \exists yLxy)$. 
  \item[\it Complex:] Everything everything loves loves something, i.e., $\forall x(\forall yLyx \eif \exists zLxz)$.
\end{enumerate}



\section*{Semantics for $\FOL$}

\begin{enumerate}
  \item[] $\VV{\I}{\hat{a}}(\F^n\alpha_1,\ldots,\alpha_n)=1$ ~\textit{iff}~ $\tuple{\val{\I}{\hat{a}}{(\alpha_1)},\ldots,\val{\I}{\hat{a}}{(\alpha_n)}}\in\I(\F^n)$.
  \item[] $\VV{\I}{\hat{a}}(\forall\alpha\metaA)=1$ ~\textit{iff}~ $\VV{\I}{\hat{c}}(\metaA)=1$ for every $\alpha$-variant $\hat{c}$ of $\hat{a}$.
  \item[] $\VV{\I}{\hat{a}}(\exists\alpha\metaA)=1$ ~\textit{iff}~ $\VV{\I}{\hat{c}}(\metaA)=1$ for some $\alpha$-variant $\hat{c}$ of $\hat{a}$.
  \item[] $\VV{\I}{\hat{a}}(\enot\metaA)=1$ ~\textit{iff}~ $\VV{\I}{\hat{a}}(\metaA)=0$.
  \item[] $\VV{\I}{\hat{a}}(\metaA \eor \metaB)=1$ ~\textit{iff}~ $\VV{\I}{\hat{a}}(\metaA)=1$ or $\VV{\I}{\hat{a}}(\metaB)=1$ (or both).
  \item[] $\VV{\I}{\hat{a}}(\metaA \eand \metaB)=1$ ~\textit{iff}~ $\VV{\I}{\hat{a}}(\metaA)=1$ and $\VV{\I}{\hat{a}}(\metaB)=1$.
  \item[] $\VV{\I}{\hat{a}}(\metaA \eif \metaB)=1$ ~\textit{iff}~ $\VV{\I}{\hat{a}}(\metaA)=0$ or $\VV{\I}{\hat{a}}(\metaB)=1$ (or both).
  \item[] $\VV{\I}{\hat{a}}(\metaA \eiff \metaB)=1$ ~\textit{iff}~ $\VV{\I}{\hat{a}}(\metaA)=\VV{\I}{\hat{a}}(\metaB)$.
\end{enumerate}



\section*{Truth and Entailment}

\begin{enumerate}
  \item[\it Truth:] $\VV{\I}{}(\metaA)=1$ ~\textit{iff}~ $\VV{\I}{\hat{a}}(\metaA)=1$ for every v.a. $\hat{a}$ where $\metaA$ is a wfs of $\FOL$. 
  % \item[\it Satisfaction:] $\M=\tuple{\D,\I}$ satisfies $\MetaG$ \textit{iff} $\VV{\I}{}(\metaA)=1$ for every $\metaA\in\MetaG$.
  % \item[\it Singletons:] As before $\M$ satisfies $\metaA$ \textit{iff} $\M$ satisfies $\set{\metaA}$.
  \item[\it Logical Consequence:] $\MetaG\vDash\metaA$ just in case $\VV{\I}{}(\metaA) = 1$ whenever $\VV{\I}{}(\metaG) = 1$ for all $\metaG \in \MetaG$.
  \item[\it Tautology:] $\metaA$ is a tautology \textit{iff} $\VV{\I}{}(\metaA) = 1$ for all every model $\M = \tuple{\D, \I}$.
  \item[\it Contradiction:] $\metaA$ is a contradiction \textit{iff} $\VV{\I}{}(\metaA) = 0$ for all every model $\M = \tuple{\D, \I}$.
  \item[\it Contingent:] $\metaA$ is contingent \textit{iff} $\VV{\I}{}(\metaA) \neq \VV{\J}{}(\metaA)$ for some $\I$ and $\J$.
  \item[\it Satisfiable:] $\MetaG$ is satisfiable \textit{iff} there some $\I$ where $\VV{\I}{}(\metaG) = 1$ for all $\metaG \in \MetaG$.
\end{enumerate}







\end{document}

