\documentclass[a4paper, 11pt]{article} % Font size (can be 10pt, 11pt or 12pt) and paper size (remove a4paper for US letter paper)
\usepackage[protrusion=true,expansion=true]{microtype} % Better typography
\usepackage{graphicx} % Required for including pictures
\usepackage{wrapfig} % Allows in-line images
\usepackage{enumitem} %%Enables control over enumerate and itemize environments
\usepackage{setspace}
\usepackage{amssymb, amsmath, mathrsfs, mathabx} %%Math packages
\usepackage{../lecture} %calls local modified style file
\usepackage{stmaryrd}
\usepackage{mathtools}
\usepackage{multicol} 
\usepackage{mathpazo} % Use the Palatino font
\usepackage[T1]{fontenc} % Required for accented characters
\usepackage{array}
\usepackage{bibentry}
\usepackage{prooftrees} 
\usepackage[round]{natbib} %%Or change 'round' to 'square' for square backers
\setcitestyle{aysep=}
% \usepackage{fitchproof} 


\makeatletter
\renewcommand{\maketitle}{
\begin{flushright}
{\LARGE\@title}

\vspace{10pt}

{\@author}
\\ \@date
\end{flushright}

\vspace{30pt}

}
\makeatother

%----------------------------------------------------------------------------------------
%	TITLE
%----------------------------------------------------------------------------------------

\title{\textbf{FOL$^=$ Soundness}} % Subtitle

\author{\textsc{Logic I}\\ \em Benjamin Brast-McKie} % Institution

\date{\today} % Date

%----------------------------------------------------------------------------------------

\begin{document}

\maketitle % Print the title section

\thispagestyle{empty}

%----------------------------------------------------------------------------------------


\section*{Substitution}

\begin{enumerate}
  \item[\it Free For:] $\beta$ is \textsc{free for} $\alpha$ in $\varphi$ just in case there is no free occurrence of $\alpha$ in $\varphi$ in the scope of a quantifier that binds $\beta$. 
  \item[\it Substitution:] If $\beta$ is free for $\alpha$ in $\varphi$, then the \textsc{substitution} $\varphi\unisub{\beta}{\alpha}$ is the result of replacing all free occurrences of $\alpha$ in $\varphi$ with $\beta$.
\end{enumerate}
   






\section*{FOL$^=$ Rules}

\begin{enumerate}
  \item[($\forall$E)] $\forall\alpha\varphi \vdash \varphi\unisub{\beta}{\alpha}$ where $\beta$ is a constant and $\alpha$ is a variable. 
  \item[($\exists$I)] $\varphi\unisub{\beta}{\alpha} \vdash \exists\alpha\varphi$ where $\beta$ is a constant and $\alpha$ is a variable.
  \item[($\forall$I)] $\varphi\unisub{\beta}{\alpha} \vdash \forall\alpha\varphi$ where $\beta$ is a constant, $\alpha$ is a variable, and $\beta$ does not occur in $\forall\alpha\varphi$ or in any undischarged assumption.
  \item[($\exists$E)] If $\exists\alpha\varphi,\varphi\unisub{\beta}{\alpha} \vdash \psi$ where $\beta$ is a constant that does not occur in $\exists\alpha\varphi$, $\psi$, or in any undischarged assumption, then $\exists\alpha\varphi\vdash \psi$.
  \item[($=$I)] $\vdash \alpha = \alpha$ for any constant $\alpha$. 
  \item[($=$E)] $\varphi\unisub{\alpha}{\gamma},\alpha=\beta\vdash\varphi\unisub{\beta}{\gamma}$.
\end{enumerate}






\section*{From Natural to Normative}

\begin{itemize}
  \item[\it Soundness:] If $\Gamma \vdash \varphi$, then $\Gamma \vDash \varphi$.
  \item Soundness shows that we can trust FOL$^=$ to establish validity.
  \item Easier to derive a conclusion that to provide a semantic argument.
  \item Same story as before\ldots
  \item[\it Natural:] FOL$^=$ describes (approximately) how we in fact reason.
  \item The quantifier rules are somewhat natural...
  \item Can we understand the quantifiers only by their rules?
  \item Semantics for the quantifiers is more natural.
  \item Soundness explains why we ought to use FOL$^=$ to reason.
\end{itemize}




\section*{Soundness of FOL$^=$}

\begin{enumerate}
  \item[\it Assume:] $\Gamma \vdash \varphi$, so there is a FOL$^=$ proof $X$ of $\varphi$ from $\Gamma$. 
  \item[\it Lines:] Let $\metaA_i$ be the wfs on line $i$ of $X$.
  \item[\it Dependencies:] Let $\Gamma_i$ be the premises and undischarged assumptions at line $i$. 
  \item[\it Proof:] The proof goes by induction on length of $X$:
    \begin{itemize}
      \item[\bf L11.1:] (Base Step)~~ $\Gamma_1 \vDash \varphi_i$. 
      \item[\bf L11.13:] (Induction Step)~~ If $\Gamma_k \vDash \varphi_k$ for all $k\leq n$, then $\Gamma_{n+1} \vDash \varphi_{n+1}$. 
    \end{itemize}
  \item[\it Finite:] Since $X$ is finite, there is some $m$ where $\varphi_m=\varphi$ and $\Gamma_m=\Gamma$, so $\Gamma \vDash \varphi$.
\end{enumerate}




\section*{Base Step}

\begin{itemize}
  \item[\it Proof:] Every line in a FOL$^=$ proof is either a premise or follows by the rules.
  \item $\varphi_1$ is either a premise or follows by AS or $=$I. 
  \item If $\varphi_1$ is a premise or assumption, then $\Gamma_1=\set{\varphi_1}$, and so $\Gamma_1\vDash\varphi_1$.
  \item If $\varphi_1$ follows by $=$I, then $\varphi_1$ is $\alpha=\alpha$ for some constant $\alpha$. 
  \item Letting $\M=\tuple{\D,\I}$ be any model, $\I(\alpha) \in \D$.
  \item Letting $\va{a}$ be a variable assignment, $\val{\I}{\va{a}}(\alpha)=\val{\I}{\va{a}}(\alpha)$.
  \item So $\VV{\I}{\va{a}}(\alpha=\alpha)=1$, and so $\vDash \alpha=\alpha$, thus $\Gamma_1\vDash \varphi_1$ since $\Gamma_1=\varnothing$.
\end{itemize}



\section*{Induction Step}

\begin{itemize}
  \item[\it Assume:] $\Gamma_k \vDash \varphi_k$ for all $k\leq n$.
  \item If $\varphi_{n+1}$ is a premise, assumption, or $=$I, then the same as above. 
  \item Assume $\varphi_{n+1}$ follows by the FOL$^=$ rules, to show $\Gamma_{n+1}\vDash\varphi_{n+1}$.
  \item There are 11 more rules in PL and an additional 5 from FOL$^=$.
  \item[\bf Question:] Can't we just appeal to the proofs from before for the PL rules?
    \item Models and variable assignments instead of interpretations.
    \item Differences are superficial and provided in the book.
  \item[\it Updated:] Consider how the proofs of the following lemmas must change: 
    \item[\bf L2.1] If $\MetaG \vDash \metaA$, then $\MetaG \cup \MetaS \vDash \metaA$.
    \item[\bf L4.3] For any FOL$^=$ proof $X$, if $\metaA_k$ is live at line $n$ where $k\leq n$, then $\MetaG_k\subseteq \MetaG_{n}$.
    \item[\bf L11.2] If $\MetaG \vDash \metaA$ and $\MetaG \vDash \enot\metaA$, then $\MetaG$ is unsatisfiable.
    \item[\bf L11.3] $\MetaG \cup \set{\metaA}$ is unsatisfiable just in case $\MetaG \vDash \enot\metaA$.
    \item[\bf L11.4] If $\MetaG \cup \set{\metaA} \vDash \metaB$, then $\MetaG \vDash \metaA \eif \metaB$.
  \item[\it Unchanged:] The following lemmas do not need to change:
  \item[\bf L9.1] $\VV{\I}{\va{a}}(\metaA)=\VV{\I}{\va{c}}(\metaA)$ if $\va{a}(\alpha)=\va{c}(\alpha)$ for all free variables $\alpha$ in a wff $\metaA$.
  \item[\bf L9.2] $\VV{\I}{}(\metaA)= 1$ just in case $\VV{\I}{\va{a}}(\metaA)= 1$ for some v.a. $\va{a}$ over $\D$.
\end{itemize}
\vspace{-.25in}





\section*{PL Rules}

\begin{itemize}
  % \item[\checkmark (AS)] Proof is the same as in the base case considered last week.
  % \item[(R)] $\metaA_k=\metaA_{n+1}$ for live $k\leq n$.
  %   \item Thus $\MetaG_k\vDash\metaA_k$ by hypothesis and $\MetaG_k\subseteq\MetaG_{n+1}$ by \textbf{L12.2}.
  %   \item Thus $\MetaG_{n+1}\vDash\metaA_k$ by \textbf{L12.1}, and so $\MetaG_{n+1}\vDash\metaA_{n+1}$.
  % \item[($\enot$I)] There is a proof of $\metaB$ at line $h$ and $\enot\metaB$ at line $j$ from $\metaA$ on line $i$.
  %   %$\MetaG_{n+1} \vDash \metaA_{n+1}$ if $\metaA_{n+1}$ follows from $\MetaG_{n+1}$ by the rule $\enot$I.
  %   \item By hypothesis $\MetaG_h\vDash \metaB$ and $\MetaG_j\vDash\enot\metaB$, where $\MetaG_h,\MetaG_j\subseteq\MetaG_{n+1}\cup\set{\metaA_i}$.
  %   \item By \textbf{L12.1}, $\MetaG_{n+1}\cup\set{\metaA_i}\vDash\metaB$ and $\MetaG_{n+1}\cup\set{\metaA_i}\vDash\enot\metaB$.
  %   \item So $\MetaG_{n+1}\cup\set{\metaA_i}$ is unsatisfiable by \textbf{L12.3}, so $\MetaG_{n+1}\vDash\metaA_{n+1}$ by \textbf{L12.4}.
  \item[($\enot$E)] There is a proof of $\metaB$ at line $h$ and $\enot\metaB$ at line $j$ from $\enot \metaA$ on line $i$.
    %$\MetaG_{n+1} \vDash \metaA_{n+1}$ if $\metaA_{n+1}$ follows from $\MetaG_{n+1}$ by the rule $\enot$I.
    \item By hypothesis $\MetaG_h\vDash \metaB$ and $\MetaG_j\vDash\enot\metaB$, where $\MetaG_h,\MetaG_j\subseteq\MetaG_{n+1}\cup\set{\enot \metaA_i}$.
    \item By \textbf{L2.1}, $\MetaG_{n+1}\cup\set{\enot \metaA_i}\vDash\metaB$ and $\MetaG_{n+1}\cup\set{\enot \metaA_i}\vDash\enot\metaB$.
    \item So $\MetaG_{n+1}\cup\set{\enot \metaA_i}$ is unsatisfiable by \textbf{L11.2}
    \item So $\MetaG_{n+1}\vDash \enot\enot\metaA_{n+1}$ follows by \textbf{L11.3}, and then appeal to the semantics.
    % \item The final step that $\MetaG_{n+1}\vDash \metaA_{n+1}$ must appeal to the semantics for $\FI$. 
  % \item[\checkmark ($\enot$E)] Similar to ($\enot$I).
  % \item[\checkmark ($\eand$I)] Skip.
  % \item[($\eand$E)] $\metaA_{n+1}\eand\metaB$ is live on line $i\leq n$.
  %   \item By hypothesis, $\MetaG_i\vDash\metaA_{n+1}\eand\metaB$ where $\MetaG_i\subseteq\MetaG_{n+1}$ by \textbf{L12.2}
  %   \item Thus $\MetaG_{n+1}\vDash\metaA_{n+1}\eand\metaB$ by \textbf{L12.1}, and so $\MetaG_{n+1}\vDash\metaA_{n+1}$ by semantics.
  % \item[($\eor$I)] 
  % \item[($\eor$E)] 
  % \item[($\eif$I)] There is a proof of $\metaB$ at line $j$ from $\metaA$ on line $i$. 
  %   \item By hypothesis $\MetaG_j\vDash \metaB$, where $\MetaG_j\subseteq\MetaG_{n+1}\cup\set{\metaA}$.
  %   \item So $\MetaG_{n+1}\cup\set{\metaA}\vDash\metaB$, and so $\MetaG_{n+1}\vDash\metaA\eif\metaB$ by \textbf{L12.7}.
  % \item[($\eif$E)] 
  % \item[($\equiv$I)] 
  % \item[($\equiv$E)] 
\end{itemize}
\vspace{-.25in}



\section*{FOL$^=$ Lemmas}

\begin{itemize}
  \item[\bf L11.5] $\VV{\I}{\va{a}}(\metaA)=\VV{\I}{\va{a}}(\metaA\unisub{\beta}{\alpha})$ if $\val{\I}{\va{a}}(\alpha)=\val{\I}{\va{a}}(\beta)$ and $\beta$ is free for $\alpha$ in $\metaA$.
    \item[\it Base:] Assume $\metaA$ is $\F^n\alpha_1,\ldots,\alpha_n$ or $\alpha_1=\alpha_2$ where $\val{\I}{\va{a}}(\alpha)=\val{\I}{\va{a}}(\beta)$. 
    \item Let $\gamma_i=
      \begin{cases}
        \beta     & \text{ if } \alpha_i=\alpha\\
        \alpha_i  & \text{ otherwise.}
      \end{cases}$
    \item $\tuple{\val{\I}{\va{a}}(\alpha_1),\ldots,\val{\I}{\va{a}}(\alpha_n)}\in\I(\F^n) \textit{ ~iff~ } \tuple{\val{\I}{\va{a}}(\gamma_1),\ldots,\val{\I}{\va{a}}(\gamma_n)}\in\I(\F^n)$.
    \item $\val{\I}{\va{a}}(\alpha_1)=\val{\I}{\va{a}}(\alpha_n) \textit{ ~iff~ } \val{\I}{\va{a}}(\gamma_1)=\val{\I}{\va{a}}(\gamma_n)$.
    \item[\it Induction:] If $\comp{\metaA}\leq n$, $\VV{\I}{\va{a}}(\metaA)=\VV{\I}{\va{a}}(\metaA\unisub{\beta}{\alpha})$ whenever $\val{\I}{\va{a}}(\alpha)=\val{\I}{\va{a}}(\beta)$.
    \item[\it Case 2:] Assume $\metaA=\metaB\eand\metaC$ where $\val{\I}{\va{a}}(\alpha)=\val{\I}{\va{a}}(\beta)$.
    \item So $\VV{\I}{\va{a}}(\metaA)=1$ \textit{iff} $\VV{\I}{\va{a}}(\metaB\eand\metaC)=1$ \textit{iff} $\VV{\I}{\va{a}}(\metaB)=\VV{\I}{\va{a}}(\metaC)=1$ \textit{iff} \ldots
    \item[\it Case 6:] Assume $\metaA=\qt{\forall}{\gamma}\metaB$ where $\val{\I}{\va{a}}(\alpha)=\val{\I}{\va{a}}(\beta)$.
    \item If $\gamma=\alpha$, then $\metaA=\metaA\unisub{\beta}{\alpha}$.
    \item If $\gamma\neq\alpha$, $\VV{\I}{\va{a}}(\qt{\forall}{\gamma}\metaB)=1$ \textit{iff} $\VV{\I}{\va{e}}(\metaB)=1$ for all $\gamma$-variants $\va{e}$ of $\va{a}$ \textit{iff}\ldots
    \item Let $\va{e}$ be an arbitrary $\gamma$-variant of $\va{a}$.
    \item Since $\gamma\neq\alpha$, $\va{e}(\alpha)=\va{a}(\alpha)$ if $\alpha$ is a variable, so $\val{\I}{\va{e}}(\alpha)=\val{\I}{\va{a}}(\alpha)$. 
    \item Thus $\val{\I}{\va{e}}(\alpha)=\val{\I}{\va{a}}(\beta)$ follows from the assumption. 
    \item Since $\beta$ is free for $\alpha$ in $\qt{\forall}{\gamma}\metaB$, we know that $\gamma\neq\beta$.
    \item If $\beta$ is a variable, then $\va{e}(\beta)=\va{a}(\beta)$ since $\va{e}$ is a $\gamma$-variant of $\va{a}$.
    \item Thus $\val{\I}{\va{e}}(\beta)=\val{\I}{\va{a}}(\beta)$, and so $\val{\I}{\va{e}}(\alpha)=\val{\I}{\va{e}}(\beta)$.
    \item By hypothesis, $\VV{\I}{\va{e}}(\metaB)=\VV{\I}{\va{e}}(\metaB\unisub{\beta}{\alpha})$, where $\va{e}$ was arbitrary.
    \item \ldots\textit{iff} $\VV{\I}{\va{e}}(\metaB\unisub{\beta}{\alpha})=1$ for all $\gamma$-variants $\va{e}$ of $\va{a}$ \textit{iff} $\VV{\I}{\va{a}}(\metaA\unisub{\beta}{\alpha})=1$.
  \item[\bf L11.6] If $\M=\tuple{\D,\I}$ and $\M'=\tuple{\D,\I'}$ where $\I$ and $\I'$ agree about every constant $\alpha$ and $n$-place predicate $\F^n$ that occurs in $\metaA$, it follows that $\VV{\I}{\va{a}}(\metaA)=\VV{\I'}{\va{a}}(\metaA)$ for any variable assignment $\va{a}$ over $\D$.
    \item[\it Base:] $\tuple{\val{\I}{\va{a}}(\alpha_1),\ldots,\val{\I}{\va{a}}(\alpha_n)}\in\I(\F^n) \textit{ ~iff~ } \tuple{\val{\I'}{\va{a}}(\alpha_1),\ldots,\val{\I'}{\va{a}}(\alpha_n)}\in\I'(\F^n)$.
    \item $\I(\F^n)=\I'(\F^n)$ is immediate from the assumption.
    \item $\val{\I}{\va{a}}(\alpha_i)=\I(\alpha_i)=\I'(\alpha_i)=\val{\I'}{\va{a}}(\alpha_i)$ if $\alpha_i$ is a constant.
    \item $\val{\I}{\va{a}}(\alpha_i)=\va{a}(\alpha_i)=\val{\I'}{\va{a}}(\alpha_i)$ if $\alpha_i$ is a variable.
  \item[\bf L11.7] For a constant $\beta$ not in $\qt{\forall}{\alpha}\metaA$ or $\metaB\in\MetaG$, if $\MetaG \vDash \metaA\unisub{\beta}{\alpha}$, then $\MetaG \vDash \qt{\forall}{\alpha}\metaA$.
      \item Assume $\MetaG \vDash \metaA\unisub{\beta}{\alpha}$ for constant $\beta$ not in $\qt{\forall}{\alpha}\metaA$ or $\MetaG$.
      \item Let $\M = \tuple{\D, \I}$ where $\VV{\I}{}(\metaC) = 1$ for all $\metaC\in\MetaG$, and let $\va{c}$ be arbitrary.
      \item So $\VV{\I}{\va{a}}(\metaC) = 1$ for all $\metaC\in\MetaG$ for all v.a. $\va{a}$, and so for $\va{c}$ in particular.
      \item Let $\M'$ be like $\M$ but for $\I'(\beta)=\va{c}(\alpha)$, and so $\val{\I'}{\va{c}}(\alpha)=\val{\I'}{\va{c}}(\beta)$.
      \item Since $\beta$ does not occur in $\MetaG$, we know $\VV{\I'}{\va{c}}(\metaC) = 1$ for all $\metaC \in \MetaG$ by \textbf{L11.6}.
      \item So $\VV{\I'}{}(\metaC) = 1$ for all $\metaC\in\MetaG$ by \textbf{L9.2}, so $\VV{\I'}{}(\metaA\unisub{\beta}{\alpha}) = 1$ by assumption.
      \item So $\VV{\I'}{\va{a}}(\metaA\unisub{\beta}{\alpha})=1$ for all $\va{a}$, and so $\VV{\I'}{\va{c}}(\metaA\unisub{\beta}{\alpha})=1$ for $\va{c}$ in particular.
      \item So $\VV{\I'}{\va{c}}(\metaA) = 1$ by \textbf{L11.5} given $\val{\I'}{\va{c}}(\alpha) = \val{\I'}{\va{c}}(\beta)$ above. 
      \item Since $\beta$ is not in $\qt{\forall}{\alpha}\metaA$, we know $\beta$ is not in $\metaA$, so $\VV{\I}{\va{c}}(\metaA)=1$ by \textbf{L11.6}.
      \item Since $\va{c}$ was arbitrary, $\VV{\I}{\va{a}}(\qt{\forall}{\alpha}\metaA)=1$ for any $\va{a}$, so $\VV{\I}{}(\qt{\forall}{\alpha}\metaA)=1$.
      \item By generalizing on $\M$, we may conclude that $\MetaG \vDash \qt{\forall}{\alpha}\metaA$.
  % \item[\bf L11.8] $\forall\alpha\metaA \vDash \metaA\unisub{\beta}{\alpha}$ where $\alpha$ is a variable and $\metaA\unisub{\beta}{\alpha}$ is a wfs of $\FI$.
  %   \item Let $\M = \tuple{\D, \I}$ where $\VV{\I}{}(\qt{\forall}{\alpha}\metaA) = 1$, so $\VV{\I}{\va{a}}(\qt{\forall}{\alpha}\metaA)=1$ for arbitrary $\va{a}$. 
  %   \item So $\VV{\I}{\va{c}}(\metaA)=1$ for an $\alpha$-variant $\va{c}$ of $\va{a}$ where $\va{c}(\alpha)=\I(\beta)$.
  %   \item Since $\val{\I}{\va{c}}(\alpha)=\val{\I}{\va{c}}(\beta)$, we know $\VV{\I}{\va{c}}(\metaA\unisub{\beta}{\alpha})=1$ by \textbf{L11.5}.
  %   \item Thus $\VV{\I}{}(\metaA\unisub{\beta}{\alpha})=1$ by \textbf{L9.2}, so $\forall\alpha\metaA \vDash \metaA\unisub{\beta}{\alpha}$ generalizing on $\M$.
  % \item[\bf L11.9] If $\MetaG \vDash \metaA$ and $\Sigma \cup \set{\metaA} \vDash \metaB$, then $\MetaG\cup\Sigma \vDash \metaB$.
  % \item[($\forall$I)] $\metaA_i=\metaA\unisub{\beta}{\alpha}$ for $i\leq n$ live at $n+1$ where $\beta$ is not in $\metaA_{n+1}$ or $\MetaG_{n+1}$.
  %   \item So $\MetaG_i\vDash\metaA_i$ by hypothesis, and $\MetaG_i\subseteq\MetaG_{n+1}$ by \textbf{L4.3}.
  %   \item Thus $\MetaG_{n+1}\vDash\metaA_i$ by \textbf{L2.1}, so $\MetaG_{n+1}\vDash\metaA\unisub{\beta}{\alpha}$.
  %   \item So $\MetaG_{n+1}\vDash \qt{\forall}{\alpha}\metaA$ by \textbf{L11.7} since $\beta$ not in $\qt{\forall}{\alpha}\metaA$ or $\MetaG_{n+1}$.
  %   \item Equivalently, $\MetaG_{n+1}\vDash \metaA_{n+1}$.
  % \item[($\forall$E)] $\metaA_i=\qt{\forall}{\alpha}\metaA$ for $i\leq n$ live at $n+1$ where $\metaA_{n+1}=\metaA\unisub{\beta}{\alpha}$.
  %   \item So $\MetaG_i\vDash\metaA_i$ by hypothesis, and $\MetaG_i\subseteq\MetaG_{n+1}$ by \textbf{L4.3}.
  %   \item Thus $\MetaG_{n+1}\vDash\metaA_i$ by \textbf{L2.1}, so $\MetaG_{n+1}\vDash \qt{\forall}{\alpha}\metaA$.
  %   \item By \textbf{L11.8} $\qt{\forall}{\alpha}\metaA\vDash\metaA\unisub{\beta}{\alpha}$, and so $\MetaG_{n+1}\vDash\metaA\unisub{\beta}{\alpha}$ by \textbf{L11.9}.
  %   \item Equivalently, $\MetaG_{n+1}\vDash \metaA_{n+1}$.
\end{itemize}

\end{document}

