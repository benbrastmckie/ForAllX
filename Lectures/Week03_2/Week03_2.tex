\documentclass[a4paper, 11pt]{article} % Font size (can be 10pt, 11pt or 12pt) and paper size (remove a4paper for US letter paper)
\usepackage[protrusion=true,expansion=true]{microtype} % Better typography
\usepackage{../lecture} %calls local modified style file
\usepackage{graphicx} % Required for including pictures
\usepackage{wrapfig} % Allows in-line images
\usepackage{enumitem} %%Enables control over enumerate and itemize environments
\usepackage{setspace}
\usepackage{amssymb, amsmath, mathrsfs} %%Math packages
\usepackage{stmaryrd}
\usepackage{mathtools}
\usepackage{multicol} 
\usepackage{mathpazo} % Use the Palatino font
\usepackage[T1]{fontenc} % Required for accented characters
\usepackage{array}
\usepackage{bibentry}
\usepackage{prooftrees} 
\usepackage[round]{natbib} %%Or change 'round' to 'square' for square backers
\setcitestyle{aysep={}}
% \usepackage{fitchproof} 

\makeatletter
\renewcommand{\maketitle}{
\begin{flushright}
{\LARGE\@title}

\vspace{10pt}

{\@author}
\\ \@date
\end{flushright}

\vspace{20pt}

}
\makeatother

%----------------------------------------------------------------------------------------
%	TITLE
%----------------------------------------------------------------------------------------

\title{\textbf{Natural Deduction in SL: Part II}} % Subtitle

\author{\textsc{Logic I}\\ \em Benjamin Brast-McKie} % Institution

\date{\today} % Date

%----------------------------------------------------------------------------------------

\begin{document}

\maketitle % Print the title section

\thispagestyle{empty}

%----------------------------------------------------------------------------------------

\section*{Biconditional}

\begin{enumerate}
  \item[\it Elimination:] $A \eiff (B \eif [(A \eand C)\eiff D])\ \vdash (A\eand B) \eif (D \eif C)$. 
  \item[\it Introduction:] $A \eif (B \eand C),\ C \eif (B \eand A) \vdash A \eiff C$.
\end{enumerate}






\section*{Negation}

\begin{enumerate}
  \item[\it Elimination:] $\enot\enot A\ \vdash A$. 
  \item[\it Introduction:] $A \eif (B \eand C),\ C \eif (B \eand A) \vdash A \eiff C$.
\end{enumerate}


\section*{Reiteration}

\begin{enumerate}
  \item[\it Example:] $A,\ B \eif C\ \vdash B \eif (C \eand A)$.
\end{enumerate}





\section*{Proof}

\begin{enumerate}
  \item[\it Proof:] A natural deduction \textsc{proof} (or \textsc{derivation}) of a conclusion $\varphi$ from a set of premises $\Gamma$ in PL is any sequence of lines ending with $\varphi$ on a live line where every line in the sequence is either:
      \begin{itemize}
        \item[(1)] a premise in $\Gamma$; 
        \item[(2)] a discharged assumption; or
        \item[(3)] follows from previous lines by the rules for PL.
      \end{itemize}
  \item[\it Provable:] An wfs $\varphi$ of $\PL$ is \textsc{provable} (or \textsc{derivable}) from $\Gamma$ in PL \textit{iff} there is a natural deduction proof (derivation) of $\varphi$ from $\Gamma$ in PL, i.e., $\Gamma \vdash \varphi$. 
  \item[\it Equivalent:] Two sentences $\varphi$ and $\psi$ are \textsc{provably equivalent} (or \textsc{interderivable}) if and only if both $\varphi\vdash\psi$ and $\psi\vdash\varphi$.
  \item[\it Inconsistent:] A set of sentences $\Gamma$ is \textsc{provably inconsistent} if and only if $\Gamma\vdash\bot$ where $\bot$ is our arbitrarily chosen contradiction, e.g., $A\eand\enot A$.
\end{enumerate}



% \section*{Further Problems}
%
%
% \begin{enumerate}
%   \begin{multicols}{2}
%   \item[\it Law of Excluded Middle:] $A \eor \enot A$. 
%     \begin{itemize}
%       \item $\enot(A \eor \enot A)$ \quad:AS
%       \begin{itemize}
%         \item $A$ \quad:AS 
%         \item $A\eor \enot A$ \quad:$\eor$I 
%         \item $\enot (A\eor \enot A)$ \quad:R 
%       \end{itemize}
%       \item $\enot A$ \quad:$\enot$I 
%       \item $A \eor \enot A$ \quad:$\eor$I 
%       \item $A\eor \enot A$ \quad:$\enot$E 
%     \end{itemize}
%   \item[\it LNC:] $\enot(A \eand \enot A)$. 
%     \item[\it EXQ:] $A,\ \enot A \vdash B$. (\textit{Ex Falso Quodlibet})
%     \begin{itemize}
%       \item $\enot B$ \quad:AS 
%       \begin{itemize}
%         \item $A$ \quad:R 
%         \item $\enot A$ \quad:R 
%       \end{itemize}
%       \item $B$ \quad:$\enot$E 
%       \item[] ~
%       \item[] ~
%     \end{itemize}
%   \end{multicols}
%   \item $L\eiff \enot O,\ L\eor \enot O\ \vdash L$.
%   \item $A\eiff B,\ \vdash \enot A\eiff\enot B$.
%   \item $Z \eif (C \eand \enot N),\ \enot Z \eif (N \eand \enot C)\ \vdash N \eor C$.
% \end{enumerate}

  % \begin{multicols}{2}
  % \end{multicols}



\iffalse

\begin{multicols}{2}


\textit{Conjunction Introduction} (\eand I) \vspace{-1em}
\begin{proof}
	\have[m]{a}{\metaA{}}
	\have[n]{b}{\metaB{}}
	\have[\ ]{c}{\metaA{}\eand\metaB{}} \ai{a, b}
	\have[\ ]{d}{\metaB{}\eand\metaA{}} \ai{a, b}
\end{proof}

\vspace{1em}

\textit{Conditional Introduction} (\eif I) \vspace{-1em}
%\nopagebreak
\begin{proof}
	\open
		\hypo[m]{a}{\metaA{}} \as{for \eif I}{}%\by{want \metaB{}}{}
		\have[n]{b}{\metaB{}}
	\close
	\have[\ ]{ab}{\metaA{}\eif\metaB{}}\ci{a-b}
\end{proof}

\vspace{0.6em}

\textit{Negatation Introduction} (\enot I) \vspace{-1em}
\begin{proof}
\open
	\hypo[m]{na}\metaA{} \as{for \enot I}   %\by{:AS for \enot I}{}
	\have[n]{b}\metaB{}
	\have[o]{nb}{\enot\metaB{}}
\close
\have[\ ]{a}[\ ]{\enot\metaA{}}\ni{na-nb}
\end{proof}

\vspace{0.6em}

\textit{Disjunction Introduction} (\eor I) \vspace{-1em}

\begin{proof}
	\have[m]{a}{\metaA{}}
	\have[\ ]{ab}{\metaA{}\eor\metaB{}}\oi{a}
	\have[\ ]{ba}{\metaB{}\eor\metaA{}}\oi{a}
\end{proof}

%\vspace{1.9em} %3.9 for no extra vspaces
\vspace{0.6em}

\textit{Biconditional Introduction} (\eiff I) \vspace{-1em}

\begin{fitchproof}
	\open
		\hypo[i]{a1}{\metaA{}} \as{for \eiff I}
		\have[j]{b1}{\metaB{}}
	\close
\breakline
	\open
		\hypo[k]{b2}{\metaB{}} \as{for \eiff I}
		\have[l]{a2}{\metaA{}}
	\close
	\have[\ ]{ab}{\metaA{}\eiff\metaB{}}\bi{a1-b1,b2-a2}
\end{fitchproof}


\vfill\null
\columnbreak

%\newpage

\textit{Conjunction Elimination} (\eand E) \vspace{-1em}

\begin{proof}
	\have[m]{ab}{\metaA{}\eand\metaB{}}
	\have[\ ]{a}{\metaA{}} \ae{ab}
	\have[\ ]{b}{\metaB{}} \ae{ab}
\end{proof}

%\vspace{1.9em}
%\vspace{2.9em}
\vspace{0.75em}

\textit{Conditional Elimination} (\eif E)  \vspace{-1em}

\begin{proof}
	\have[m]{ab}{\metaA{}\eif\metaB{}}
	\have[n]{a}{\metaA{}}
	\have[\ ]{b}{\metaB{}} \ce{ab,a}
\end{proof}

%\vspace{1em}
\vspace{0.45em}

\textit{Negation Elimination} (\enot E)  \vspace{-1em}
%%note that I think I'm missing some brackets around the sentences on various liens below! works in proof environment but less robust in nd environment. so i ought to fix these here and in Negation intro 

\begin{proof}
\open
	\hypo[m]{na}{\enot\metaA{}} \as{for \enot E}
	\ellipsesline
	\have[n]{b}\metaB{}
	\have[o]{nb}{\enot\metaB{}}
\close
\have[\ ]{a}[\ ]\metaA{}\ne{na-nb}
\end{proof}

\vspace{1em}

\textit{Disjunction Elimination} (\eor E)  \vspace{-1em}

\begin{proof}
\have[m]{ab}{\metaA{}\eor\metaB{}}
	\open
		\hypo[i]{a}{\metaA{}} \as{for \eor E}
		\have[j]{c1}{\metaC{}}
	\close
\breakline
	\open
		\hypo[k]{b}{\metaB{}} \as{for \eor E}
		\have[l]{c2}{\metaC{}}
	\close
	\have[\ ]{c}{\metaC{}} \oe{ab,a-c1, b-c2}
\end{proof}

\fi



%%% OLD


\section*{Negation}

\begin{enumerate}
  \item[\it Elimination Rule:] $\neg\neg A\ \vdash A$. \quad(\textit{Double Negation Elimination})
  \item $A \vee \neg A$. \quad(\textit{Law of Excluded Middle})
  \item $A,\ \neg A \vdash B$. \quad(\textit{Ex Falso Quodlibet})
  \item[\it Introduction Rule:] $\neg(A \wedge \neg A)$. \quad(\textit{Law of Non-Contradiction})
  \item $A \vdash \neg\neg A$. \quad(\textit{Double Negation Introduction}) 
\end{enumerate}






\section*{Proof}

\begin{enumerate}
  \item[\it Proof:] A natural deduction \textsc{proof} (or \textsc{derivation}) of a conclusion $\varphi$ from a set of premises $\Gamma$ in SD is any finite sequence of lines ending with $\varphi$ on a live line where every line in the sequence is either:
      \begin{itemize}
        \item[(1)] A premise in $\Gamma$; 
        \item[(2)] A discharged assumption; or
        \item[(3)] Follows from previous lines by the rules for SD.
      \end{itemize}
  \item[\it Provable:] An SL sentence $\varphi$ is \textsc{provable} (or \textsc{derivable}) from $\Gamma$ in SD \textit{iff} there is a natural deduction proof (derivation) of $\varphi$ from $\Gamma$ in SD, i.e., $\Gamma \vdash \varphi$. 
  \item[\it Theorems:] An SL sentence $\varphi$ is a \textsc{theorem} of SD \textit{iff}~ $\vdash\varphi$.
  \item[\it Equivalent:] Sentences $\varphi$ and $\psi$ are \textsc{provably equivalent} (or \textsc{interderivable}) if and only if both $\varphi\vdash\psi$ and $\psi\vdash\varphi$, i.e., $\varphi\dashv\vdash\psi$.
  \item[\it Inconsistent:] A set of sentences $\Gamma$ is \textsc{provably inconsistent} \textit{iff} $\Gamma\vdash\bot$ where $\bot$ is the arbitrarily contradiction we chose, i.e., $A\wedge\neg A$.
\end{enumerate}



% \section*{Contradictions}
%
% \begin{enumerate}
%   \item[\it Arbitrary:] Does it really not matter which contradiction we choose?
%   \item[\bf Task 1:] Show that $\varphi\dashv\vdash\psi$ \textit{iff} $\varphi\vdash\bot$ and $\psi\vdash\bot$. 
%     \begin{itemize}
%       \item Assume $\bot = A \wedge \neg A$.
%       \item Show that if $\varphi\vdash\bot$, then $\varphi\vdash\psi$. 
%     \end{itemize}
%   \item[\bf Task 2:] Show that $\varphi\dashv\vdash\psi$ \textit{iff} $\varphi\vDash\bot$ and $\psi\vDash\bot$. 
% \end{enumerate}
%


\section*{Soundness and Completeness}

\begin{enumerate}
  \item[\it Assume:] $\Gamma \vdash \varphi$ \textit{iff} $\Gamma \vDash \varphi$.
  \item[\it Tautologies:] Coextensive with the theorems.
  \item[\it Validity:] The valid SL arguments are derivable in SD, and \textit{vice versa}.
  \item[\bf Task 1:] Can we ever use SD to determine that an argument is invalid?
  \item[\it Uncertainty:] If we haven't found a proof, that doesn't mean one doesn't exist. 
  % \item[\bf Task 2:] What if we can derive the negation of the conclusion from the premises?
  %   \begin{itemize}
  %     \item Does $\Gamma \vdash \neg \varphi$ entail $\Gamma \nvdash \varphi$?
  %   \end{itemize}
  % \item[\bf Task 3:] What can we conclude if both $\Gamma \vdash \neg \varphi$ and $\Gamma \vdash \varphi$?
\end{enumerate}



\section*{Logical Analysis}

\begin{enumerate}
  \item[\bf Task 2:] How can we tell if an argument is valid? 
    \begin{itemize}
      \item Use a semantic argument: true premises and false conclusion. 
      \item Construct a tree proof.
      % \vspace{-.1in}
    \end{itemize}
      \item[\it Pro:] Both methods provide a countermodel if there is one.
      \item[\it Con:] Neither method derives the conclusion from the premises if valid.
    \item[\bf Task 3:] How can we tell if a theorem is valid?
  \begin{multicols}{2}
    \item[\it Tautology?] \quad If YES, prove $\vdash\varphi$.\hfill
    \item[\it Contradiction?] \quad If YES, prove $\vdash\neg\varphi$.\hfill
    \item[\it Contingent?] \quad If YES, provide a models.\hfill
    \item[\it Equivalent?] \quad If YES, prove $\varphi\dashv\vdash\psi$.\hfill
    \item[] If NO, provide a countermodel.
    \item[] If NO, provide a model.
    \item[] If NO, prove $\vdash\varphi$ or $\vdash\neg\varphi$. 
    \item[] If NO, provide a countermodel.
  \end{multicols}
\end{enumerate}



\section*{Schemata}

\begin{enumerate}
  \item[\bf Observe:] Compare rules of inference in SD to SL proofs in SD.
    \begin{itemize}
      \item Whereas the rules are general, SL proofs are particular.
      \item But nothing in our SL proofs depend on the particulars.
    \end{itemize}
  \item[\bf Task 3:] How might we generalise our proofs beyond any instance?
  \item[\it Rule Schemata:] Replace sentence letters in SL proofs with metavariables.
    \begin{itemize}
      \item Premises are replaced with the lines cited by that rule.
      \item New rules require new names if we are to use them.
    \end{itemize}
  \item[\bf Task 4:] Can we also generalise proofs of theorems?
  \item[\it Axiom Schemata:] Amount to lines that can be added without citing lines. 
  \item[\bf Goal:] We want to derive intuitive rule schemata.
\end{enumerate}


\section*{Derivable Schemata}

\begin{enumerate}[leftmargin=1.5in]
  \item[\it Double Negation:] $\neg\neg\varphi\ \dashv\vdash \varphi$.
  \item[\it Ex Falso Quodlibet:] $\varphi,\ \neg\varphi\ \vdash \psi$.
  \item[\it Law of Excluded Middle:] $\vdash \varphi\vee\neg\varphi$.
  \item[\it Law of Non-Contradiction:] $\vdash \neg(\varphi\wedge\neg\varphi)$.
  \item[\it Hypothetical Syllogism:] $\varphi \supset \psi,\ \psi \supset \chi\ \vdash \varphi \supset \chi$.
  \item[\it Modus Tollens:] $\varphi \supset \psi,\ \neg\psi\ \vdash \neg\varphi$.
  \item[\it Contraposition:] $\varphi \supset \psi\ \vdash \neg\psi \supset \neg\varphi$.
  \item[\it Dilemma:] $\varphi \vee \psi,\ \varphi \supset \chi,\ \psi \supset \chi\ \vdash \chi$.
  \item[\it Disjunctive Syllogism:] $\varphi \vee \psi,\ \neg \varphi \vdash \psi$.
  \item[\it $\vee$-Commutativity:] $\varphi \vee \psi\ \vdash \psi \vee \varphi$.
  \item[\it $\wedge$-Commutativity:] $\varphi \wedge \psi\ \vdash \psi \wedge \varphi$.
  \item[\it Biconditional MP:] $\varphi \equiv \psi,\ \neg\varphi\ \vdash \neg\psi$.
  \item[\it $\equiv$-Commutativity:] $\varphi \equiv \psi\ \vdash \psi \equiv \varphi$.
  \item[\it $\wedge$-De Morgan's:] $\neg(\varphi\wedge\psi)\dashv\vdash\neg\varphi\vee\neg\psi$.
  \item[\it $\vee$-De Morgan's:] $\neg(\varphi\vee\psi)\dashv\vdash\neg\varphi\wedge\neg\psi$.
  \item[\it ${\vee}{\wedge}$-Distribution:] $\varphi\vee(\psi\wedge\chi) \dashv\vdash (\varphi\vee\psi)\wedge(\varphi\vee\chi)$.
  \item[\it ${\wedge}{\vee}$-Distribution:] $\varphi\wedge(\psi\vee\chi) \dashv\vdash (\varphi\wedge\psi)\vee(\varphi\wedge\chi)$.
  \item[\it ${\vee}{\wedge}$-Absorption:] $\varphi\vee(\varphi\wedge\psi) \dashv\vdash \varphi$.
  \item[\it ${\wedge}{\vee}$-Absorption:] $\varphi\wedge(\varphi\vee\psi) \dashv\vdash \varphi$.
  \item[\it $\wedge$-Associativity:] $\varphi\wedge(\psi\wedge\chi) \dashv\vdash (\varphi\wedge\psi)\wedge\chi$.
  \item[\it $\vee$-Associativity:] $\varphi\vee(\psi\vee\chi) \dashv\vdash (\varphi\vee\psi)\vee\chi$.
\end{enumerate}




\section*{Axiom System for SL}

\begin{enumerate}
  \item[\it Axiom System:] Consider the axiom and rule schemata, writing `$/$' for deduction.
    \begin{itemize}
      \item $\varphi \supset (\psi \supset \varphi)$.
      \item $(\varphi \supset (\psi \supset \chi)) \supset ((\varphi \supset \psi) \supset (\varphi \supset \chi))$.
      \item $(\neg\varphi \supset \neg\psi) \supset ((\neg\varphi \supset \psi) \supset \varphi)$.
      \item $\varphi \supset \psi,\ \varphi\ /\ \psi$.
    \end{itemize}
  \item[\it PL-Proof:] $\Gamma \vdash_{PL} \varphi$ \textit{iff} there is a finite sequence of SL sentences where every sentence in the sequence is either: (1) a member of $\Gamma$; (2) an axiom schemata; or (3) follows from previous sentences in the sequence by the single rule schemata given above.
  \item[\it Equivalence:] Amazingly, it is possible to show that $\Gamma \vdash_{PL} \varphi$ \textit{iff} $\Gamma \vdash_{SD} \varphi$. 
  \item[\it Definitions:] Given that the axioms and rule schemata only include $\neg$ and $\supset$, we may take these to be the \textit{only} primitive logical connectives, defining all other connectives in their terms. 
    \begin{itemize}
      \item This makes for a very compact description of the same logic.
      \item This logic is much less natural to use, requiring that a lot of derived rules be added to system.
      \item We don't have this problem, though our system is more complex.
    \end{itemize}   
\end{enumerate}

% \section*{Further Problems}
%
% \begin{enumerate}
%   \item $L\equiv \neg O,\ L\vee \neg O\ \vdash L$.
%   \item $A\equiv B\ \vdash \neg A\equiv\neg B$.
%   \item $Z \supset (C \wedge \neg N),\ \neg Z \supset (N \wedge \neg C)\ \vdash N \vee C$.
% \end{enumerate}











\end{document}

