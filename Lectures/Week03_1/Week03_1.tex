\documentclass[a4paper, 11pt]{article} % Font size (can be 10pt, 11pt or 12pt) and paper size (remove a4paper for US letter paper)
\usepackage[protrusion=true,expansion=true]{microtype} % Better typography
\usepackage{../lecture} %calls local modified style file
\usepackage{graphicx} % Required for including pictures
\usepackage{wrapfig} % Allows in-line images
\usepackage{enumitem} %%Enables control over enumerate and itemize environments
\usepackage{setspace}
\usepackage{amssymb, amsmath, mathrsfs} %%Math packages
\usepackage{stmaryrd}
\usepackage{mathtools}
\usepackage{multicol} 
\usepackage{mathpazo} % Use the Palatino font
\usepackage[T1]{fontenc} % Required for accented characters
\usepackage{array}
\usepackage{bibentry}
\usepackage{prooftrees} 
\usepackage[round]{natbib} %%Or change 'round' to 'square' for square backers
\setcitestyle{aysep={}}
% \usepackage{fitchproof} 

\makeatletter
\renewcommand{\maketitle}{
\begin{flushright}
{\LARGE\@title}

\vspace{10pt}

{\@author}
\\ \@date
\end{flushright}

\vspace{20pt}

}
\makeatother

%----------------------------------------------------------------------------------------
%	TITLE
%----------------------------------------------------------------------------------------

\title{\textbf{Natural Deduction in PL}} % Subtitle

\author{\textsc{Logic I}\\ \em Benjamin Brast-McKie} % Institution

\date{\today} % Date

%----------------------------------------------------------------------------------------

\begin{document}

\maketitle % Print the title section

\thispagestyle{empty}

%----------------------------------------------------------------------------------------

\section*{Review from Last Time\ldots}

\begin{enumerate}
  \item Show that $A \eor B,\ B \eif C,\ A \eiff C \vDash C$.
  \item Show that $\{P,\ P \eif Q,\ Q \eif \enot P\}$ is unsatisfiable. 
  \item Show that $\set{P \eif Q,\ \enot P \eor \enot Q,\ Q \eif P}$ is satisfiable.
\end{enumerate}



% TODO other semantic results to talk about?


\section*{Motivation}

\begin{enumerate}
  \item[\it Homophonic:] Prove that $P \eor Q,\ \enot P\ \vDash Q$. 
    \begin{itemize}
      \item The semantic proof makes the same inference.
      \item So why not just draw this inference directly in $\PL$?
      \item What are the basic steps we are allowed to make in a proof?
    \end{itemize}
  \item[\it Semantic Proofs:] Provide a reasonably efficient way to evaluate validity.
    \begin{itemize}
      \item But they can be cumbersome to write.
      \item They explain why a logical consequence holds.
      \item Doesn't say how to reason from some premises to a conclusion.
      \item Thus semantic proofs are not persuasive to the uninitiated.
    \end{itemize}
  \item[\it Natural Deduction:] How would we describe the patterns of natural deduction?
    \begin{itemize}
      \item Want to describe inference itself, starting with the most basic.
      \item Such inferences hold in virtue of the meanings of the operators.
      \item Define a proof to be any composition of basic inferences.
    \end{itemize}
  \item[\it Rules:] Each operator will have an introduction and elimination rule.
    \begin{itemize}
      \item These rules will describe how to reason with the connectives.
      \item Want these rules to be valid.
      \item Also want these rules to be natural.
    \end{itemize}
  \item[\it Metalogic:] 
    \begin{itemize}
      \item This is a completely different approach to formal reasoning.
      \item Nevertheless, these two approaches have the same extension.
      \item Our proof system will help us relate to the logical consequence.
    \end{itemize}
\end{enumerate}





\section*{Basic Anatomy of a Proof}

\begin{enumerate}
  \item[\it Numbers:] Every line is numbered. 
  \item[\it Sentences:] Each line contains exactly one wfs of $\PL$. 
  \item[\it Justification:] Each line includes a justification.
  \item[\it Assumptions:] The justification for a premise is `:PR'.
  \item[\it Bars:] A horizontal bar separates the premises from the steps in the proof.
  \item[\it Conclusion:] The last line is the conclusion. 
\end{enumerate}


\section*{Conditional}

\begin{enumerate}
  \item[\it Elimination:] $A,\ A \eif B,\ B \eif C\ \vdash C$. 
    \begin{itemize}
      \item Easy to derive $C$ using $\eif$E.
      \item What if $A$ was excluded from the premises? 
    \end{itemize}
  \item[\it Introduction:] $A \eif B,\ B \eif C\ \vdash A \eif C$. 
    \begin{itemize}
      \item Need something to work with.
      \item Want to conclude with a conditional claim.
      \item Assumption of $A$ justified by `:AS'.
    \end{itemize}
  \item[\it Subproofs:] Lines in a closed subproof are dead and all else are live.
    \begin{itemize}
      \item $\eif$E can only cite to live lines.
      \item $\eif$I can only cite an appropriate subproof.
    \end{itemize}
\end{enumerate}





\section*{Assumption}

\begin{enumerate}
  \item[\it Example:] $A\ \vdash D \eif [C \eif (B \eif A)]$.
\end{enumerate}




\section*{Conjunction}

\begin{enumerate}
  \item[\it Elimination:] $A \eif (B\eand C),\ B \eif D\ \vdash A \eif D$.
  \item[\it Introduction:] $A \eand B,\ B \eif C\ \vdash A \eand C$.
\end{enumerate}



\section*{Disjunction}

\begin{enumerate}
  \item[\it Introduction:] $A \vdash B \eor ((A \eor C) \eor D)$.
  \item[\it Elimination:] $A\eor(B\eand C)\ \vdash (A \eor B) \eand (A \eor C)$. 
\end{enumerate}







\section*{Biconditional}

\begin{enumerate}
  \item[\it Elimination:] $A \eiff (B \eif [(A \eand C)\eiff D])\ \vdash (A\eand B) \eif (D \eif C)$. 
  \item[\it Introduction:] $A \eif (B \eand C),\ C \eif (B \eand A) \vdash A \eiff C$.
\end{enumerate}






\section*{Negation}

\begin{enumerate}
  \item[\it Elimination:] $\enot\enot A\ \vdash A$. 
  \item[\it Introduction:] $A \eif (B \eand C),\ C \eif (B \eand A) \vdash A \eiff C$.
\end{enumerate}


\section*{Reiteration}

\begin{enumerate}
  \item[\it Example:] $A,\ B \eif C\ \vdash B \eif (C \eand A)$.
\end{enumerate}





\section*{Proof}

\begin{enumerate}
  \item[\it Proof:] A natural deduction \textsc{proof} (or \textsc{derivation}) of a conclusion $\varphi$ from a set of premises $\Gamma$ in PL is any sequence of lines ending with $\varphi$ on a live line where every line in the sequence is either:
      \begin{itemize}
        \item[(1)] a premise in $\Gamma$; 
        \item[(2)] a discharged assumption; or
        \item[(3)] follows from previous lines by the rules for PL.
      \end{itemize}
  \item[\it Provable:] An wfs $\varphi$ of $\PL$ is \textsc{provable} (or \textsc{derivable}) from $\Gamma$ in PL \textit{iff} there is a natural deduction proof (derivation) of $\varphi$ from $\Gamma$ in PL, i.e., $\Gamma \vdash \varphi$. 
  \item[\it Equivalent:] Two sentences $\varphi$ and $\psi$ are \textsc{provably equivalent} (or \textsc{interderivable}) if and only if both $\varphi\vdash\psi$ and $\psi\vdash\varphi$.
  \item[\it Inconsistent:] A set of sentences $\Gamma$ is \textsc{provably inconsistent} if and only if $\Gamma\vdash\bot$ where $\bot$ is our arbitrarily chosen contradiction, e.g., $A\eand\enot A$.
\end{enumerate}



% \section*{Further Problems}
%
%
% \begin{enumerate}
%   \begin{multicols}{2}
%   \item[\it Law of Excluded Middle:] $A \eor \enot A$. 
%     \begin{itemize}
%       \item $\enot(A \eor \enot A)$ \quad:AS
%       \begin{itemize}
%         \item $A$ \quad:AS 
%         \item $A\eor \enot A$ \quad:$\eor$I 
%         \item $\enot (A\eor \enot A)$ \quad:R 
%       \end{itemize}
%       \item $\enot A$ \quad:$\enot$I 
%       \item $A \eor \enot A$ \quad:$\eor$I 
%       \item $A\eor \enot A$ \quad:$\enot$E 
%     \end{itemize}
%   \item[\it LNC:] $\enot(A \eand \enot A)$. 
%     \item[\it EXQ:] $A,\ \enot A \vdash B$. (\textit{Ex Falso Quodlibet})
%     \begin{itemize}
%       \item $\enot B$ \quad:AS 
%       \begin{itemize}
%         \item $A$ \quad:R 
%         \item $\enot A$ \quad:R 
%       \end{itemize}
%       \item $B$ \quad:$\enot$E 
%       \item[] ~
%       \item[] ~
%     \end{itemize}
%   \end{multicols}
%   \item $L\eiff \enot O,\ L\eor \enot O\ \vdash L$.
%   \item $A\eiff B,\ \vdash \enot A\eiff\enot B$.
%   \item $Z \eif (C \eand \enot N),\ \enot Z \eif (N \eand \enot C)\ \vdash N \eor C$.
% \end{enumerate}

  % \begin{multicols}{2}
  % \end{multicols}



\iffalse

\begin{multicols}{2}


\textit{Conjunction Introduction} (\eand I) \vspace{-1em}
\begin{proof}
	\have[m]{a}{\metaA{}}
	\have[n]{b}{\metaB{}}
	\have[\ ]{c}{\metaA{}\eand\metaB{}} \ai{a, b}
	\have[\ ]{d}{\metaB{}\eand\metaA{}} \ai{a, b}
\end{proof}

\vspace{1em}

\textit{Conditional Introduction} (\eif I) \vspace{-1em}
%\nopagebreak
\begin{proof}
	\open
		\hypo[m]{a}{\metaA{}} \as{for \eif I}{}%\by{want \metaB{}}{}
		\have[n]{b}{\metaB{}}
	\close
	\have[\ ]{ab}{\metaA{}\eif\metaB{}}\ci{a-b}
\end{proof}

\vspace{0.6em}

\textit{Negatation Introduction} (\enot I) \vspace{-1em}
\begin{proof}
\open
	\hypo[m]{na}\metaA{} \as{for \enot I}   %\by{:AS for \enot I}{}
	\have[n]{b}\metaB{}
	\have[o]{nb}{\enot\metaB{}}
\close
\have[\ ]{a}[\ ]{\enot\metaA{}}\ni{na-nb}
\end{proof}

\vspace{0.6em}

\textit{Disjunction Introduction} (\eor I) \vspace{-1em}

\begin{proof}
	\have[m]{a}{\metaA{}}
	\have[\ ]{ab}{\metaA{}\eor\metaB{}}\oi{a}
	\have[\ ]{ba}{\metaB{}\eor\metaA{}}\oi{a}
\end{proof}

%\vspace{1.9em} %3.9 for no extra vspaces
\vspace{0.6em}

\textit{Biconditional Introduction} (\eiff I) \vspace{-1em}

\begin{fitchproof}
	\open
		\hypo[i]{a1}{\metaA{}} \as{for \eiff I}
		\have[j]{b1}{\metaB{}}
	\close
\breakline
	\open
		\hypo[k]{b2}{\metaB{}} \as{for \eiff I}
		\have[l]{a2}{\metaA{}}
	\close
	\have[\ ]{ab}{\metaA{}\eiff\metaB{}}\bi{a1-b1,b2-a2}
\end{fitchproof}


\vfill\null
\columnbreak

%\newpage

\textit{Conjunction Elimination} (\eand E) \vspace{-1em}

\begin{proof}
	\have[m]{ab}{\metaA{}\eand\metaB{}}
	\have[\ ]{a}{\metaA{}} \ae{ab}
	\have[\ ]{b}{\metaB{}} \ae{ab}
\end{proof}

%\vspace{1.9em}
%\vspace{2.9em}
\vspace{0.75em}

\textit{Conditional Elimination} (\eif E)  \vspace{-1em}

\begin{proof}
	\have[m]{ab}{\metaA{}\eif\metaB{}}
	\have[n]{a}{\metaA{}}
	\have[\ ]{b}{\metaB{}} \ce{ab,a}
\end{proof}

%\vspace{1em}
\vspace{0.45em}

\textit{Negation Elimination} (\enot E)  \vspace{-1em}
%%note that I think I'm missing some brackets around the sentences on various liens below! works in proof environment but less robust in nd environment. so i ought to fix these here and in Negation intro 

\begin{proof}
\open
	\hypo[m]{na}{\enot\metaA{}} \as{for \enot E}
	\ellipsesline
	\have[n]{b}\metaB{}
	\have[o]{nb}{\enot\metaB{}}
\close
\have[\ ]{a}[\ ]\metaA{}\ne{na-nb}
\end{proof}

\vspace{1em}

\textit{Disjunction Elimination} (\eor E)  \vspace{-1em}

\begin{proof}
\have[m]{ab}{\metaA{}\eor\metaB{}}
	\open
		\hypo[i]{a}{\metaA{}} \as{for \eor E}
		\have[j]{c1}{\metaC{}}
	\close
\breakline
	\open
		\hypo[k]{b}{\metaB{}} \as{for \eor E}
		\have[l]{c2}{\metaC{}}
	\close
	\have[\ ]{c}{\metaC{}} \oe{ab,a-c1, b-c2}
\end{proof}

\fi






\end{document}


