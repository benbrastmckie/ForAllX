\documentclass[a4paper, 11pt]{article} % Font size (can be 10pt, 11pt or 12pt) and paper size (remove a4paper for US letter paper)
\usepackage[protrusion=true,expansion=true]{microtype} % Better typography
\usepackage{../lecture} %calls local modified style file
\usepackage{graphicx} % Required for including pictures
\usepackage{wrapfig} % Allows in-line images
\usepackage{enumitem} %%Enables control over enumerate and itemize environments
\usepackage{setspace}
\usepackage{amssymb, amsmath, mathrsfs} %%Math packages
\usepackage{stmaryrd}
\usepackage{mathtools}
\usepackage{multicol} 
\usepackage{mathpazo} % Use the Palatino font
\usepackage[T1]{fontenc} % Required for accented characters
\usepackage{array}
\usepackage{bibentry}
\usepackage{prooftrees} 
\usepackage[round]{natbib} %%Or change 'round' to 'square' for square backers
\setcitestyle{aysep={}}
% \usepackage{fitchproof} 

\makeatletter
\renewcommand{\maketitle}{
\begin{flushright}
{\LARGE\@title}

\vspace{10pt}

{\@author}
\\ \@date
\end{flushright}

\vspace{-20pt}

}
\makeatother

%----------------------------------------------------------------------------------------
%	TITLE
%----------------------------------------------------------------------------------------

\title{\textbf{Natural Deduction in PL}} % Subtitle

\author{\textsc{Logic I}\\ \em Benjamin Brast-McKie} % Institution

\date{\today} % Date

%----------------------------------------------------------------------------------------

\begin{document}

\maketitle % Print the title section

\thispagestyle{empty}

%----------------------------------------------------------------------------------------

\section*{Review from Last Time\ldots}

\begin{enumerate}
  \item Show that $A \eor B,\ B \eif C,\ A \eiff C \vDash C$.
  \item Show that $\{P,\ P \eif Q,\ Q \eif \enot P\}$ is unsatisfiable. 
  \item Show that $\set{P \eif Q,\ \enot P \eor \enot Q,\ Q \eif P}$ is satisfiable.
\end{enumerate}



% TODO other semantic results to talk about?


\section*{Motivation}

\begin{enumerate}
  \item[\it Homophonic:] Prove that $P \eor Q,\ \enot P\ \vDash Q$. 
    \begin{itemize}
      \item The semantic proof makes the same inference.
      \item So why not just draw this inference directly in $\PL$?
      \item What are the basic steps we are allowed to make in a proof?
    \end{itemize}
  \item[\it Semantic Proofs:] Provide a reasonably efficient way to evaluate validity.
    \begin{itemize}
      \item But they can be cumbersome to write.
      \item They explain why a logical property or relation holds.
      \item Doesn't say how to reason from some premises to a conclusion.
      \item Thus semantic proofs are not persuasive to the uninitiated.
      \item Not so for semantic proofs of invalidity, satisfiability, etc.
    \end{itemize}
  \item[\it Logical Consequence:] How do we describe the extension of $\vDash$? 
  \item[\it Natural Deduction:] How should we describe the patterns of natural deduction?
    \begin{itemize}
      \item What moves can we make in a proof, \textit{viz.} semantic proofs?
      \item Want to describe inference itself, starting with the most basic.
      \item Such inferences hold in virtue of the meanings of the operators.
      \item Define a proof to be any composition of basic inferences.
    \end{itemize}
  \item[\it Rules:] Each operator will have an introduction and elimination rule.
    \begin{itemize}
      \item These rules will describe how to reason with the connectives.
      \item Want these rules to be valid.
      \item Also want these rules to be natural.
    \end{itemize}
  \item[\it Metalogic:] 
    \begin{itemize}
      \item This is a completely different approach to formal reasoning.
      \item Nevertheless, these two approaches have the same extension.
      \item Our proof system will help us relate to logical consequence.
    \end{itemize}
\end{enumerate}





\section*{Basic Anatomy of a Proof}

\begin{enumerate}
  \item[\it List:] Finite list of lines.
  \item[\it Numbers:] Every line is numbered. 
  \item[\it Sentences:] Each line contains exactly one wfs of $\PL$. 
  \item[\it Justification:] Each line includes a justification.
  \item[\it Assumptions:] The justification for a premise is `:PR'.
  \item[\it Bars:] A horizontal bar separates the premises from the steps in the proof.
  \item[\it Conclusion:] The last line is the conclusion. 
\end{enumerate}


\section*{Conditional}

\begin{enumerate}
  \item[\it Elimination:] $A,\ A \eif B,\ B \eif C\ \vdash C$. 
    \begin{itemize}
      \item Easy to derive $C$ using $\eif$E.
      \item What if $A$ was excluded from the premises? 
    \end{itemize}
  \item[\it Introduction:] $A \eif B,\ B \eif C\ \vdash A \eif C$. 
    \begin{itemize}
      \item Need something to work with.
      \item Want to conclude with a conditional claim.
      \item Assumption of $A$ justified by `:AS'.
    \end{itemize}
  \item[\it Subproofs:] Lines in a closed subproof are dead and all else are live.
    \begin{itemize}
      \item $\eif$E can only cite to live lines.
      \item $\eif$I can only cite an appropriate subproof.
    \end{itemize}
\end{enumerate}





\section*{Assumption}

\begin{enumerate}
  \item[\it Example:] $A\ \vdash D \eif [C \eif (B \eif A)]$.
\end{enumerate}




\section*{Conjunction}

\begin{enumerate}
  \item[\it Elimination:] $A \eif (B\eand C),\ B \eif D\ \vdash A \eif D$.
  \item[\it Introduction:] $A \eand B,\ B \eif C\ \vdash A \eand C$.
\end{enumerate}



\section*{Disjunction}

\begin{enumerate}
  \item[\it Introduction:] $A \vdash B \eor ((A \eor C) \eor D)$.
  \item[\it Elimination:] $A\eor(B\eand C)\ \vdash (A \eor B) \eand (A \eor C)$. 
\end{enumerate}












\end{document}


