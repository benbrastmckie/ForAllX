\documentclass[a4paper, 11pt]{article} % Font size (can be 10pt, 11pt or 12pt) and paper size (remove a4paper for US letter paper)
\usepackage[protrusion=true,expansion=true]{microtype} % Better typography
\usepackage{graphicx} % Required for including pictures
\usepackage{wrapfig} % Allows in-line images
\usepackage{enumitem} %%Enables control over enumerate and itemize environments
\usepackage{setspace}
\usepackage{amssymb, amsmath, mathrsfs} %%Math packages
\usepackage{stmaryrd}
\usepackage{mathtools}
\usepackage{mathpazo} % Use the Palatino font
\usepackage[T1]{fontenc} % Required for accented characters
\usepackage{array}
\usepackage{bibentry}
\usepackage[round]{natbib} %%Or change 'round' to 'square' for square backers
\setcitestyle{aysep={}}

% \linespread{1} % Change line spacing here, Palatino benefits from a slight increase by default

\newcommand{\corner}[1]{\ulcorner#1\urcorner} %%Corner quotes
\newcommand{\tuple}[1]{\langle#1\rangle} %%Angle brackets
\newcommand{\set}[1]{\lbrace#1\rbrace} %%Set brackets
\newcommand{\interpret}[1]{\llbracket#1\rrbracket} %%Double brackets
%\DeclarePairedDelimiter\ceil{\lceil}{\rceil}    
\def\therefore{\ensuremath{\ldotp\dot{}\,\ldotp}}
\newcommand{\I}{\mathcal{I}}
\newcommand{\V}[1]{\mathcal{V}_{#1}} %%Corner quotes

\makeatletter
\renewcommand\@biblabel[1]{\textbf{#1.}} % Change the square brackets for each bibliography item from '[1]' to '1.'
\renewcommand{\@listI}{\itemsep=0pt} % Reduce the space between items in the itemize and enumerate environments and the bibliography

\renewcommand{\maketitle}{ % Customize the title - do not edit title and author name here, see the TITLE block below
\begin{flushright} % Right align
{\LARGE\@title} % Increase the font size of the title

\vspace{10pt} % Some vertical space between the title and author name

{\@author} % Author name
\\\@date % Date

\vspace{-10pt} % Some vertical space between the author block and abstract
\end{flushright}
}

%----------------------------------------------------------------------------------------
%	TITLE
%----------------------------------------------------------------------------------------

\title{\textbf{Truth Tables}} % Subtitle

\author{\textsc{Logic I}\\ \em Benjamin Brast-McKie} % Institution

\date{\today} % Date

%----------------------------------------------------------------------------------------

\begin{document}

\maketitle % Print the title section

\thispagestyle{empty}

%----------------------------------------------------------------------------------------

\section*{Truth Functions}

\begin{enumerate}[leftmargin=1.5in,labelsep=.15in] %,label=(\arabic*)]%,label=\roman*]
  \item[\it Previously:] For an interpretation $\I$, a \textsc{valuation} function $\V{\I}$ is the smallest function to assign truth-values to every sentence of SL that satisfies the semantic clauses:
    \item[($A$)] $\V{\I}(\varphi)=\I(\varphi)$ iff $\varphi$ is a sentence letter of SL.
    \item[($\neg$)] $\V{\I}(\neg\varphi)=1$ iff $\V{\I}(\varphi)=0$ (i.e., $\V{\I}(\varphi)\neq 1$).
    \item[($\wedge$)] $\V{\I}(\varphi \wedge \psi)=1$ iff $\V{\I}(\varphi)=1$ and $\V{\I}(\psi)=1$.
    \item[($\vee$)] $\V{\I}(\varphi \vee \psi)=1$ iff $\V{\I}(\varphi)=1$ or $\V{\I}(\psi)=1$ (or both).
    \item[($\supset$)] $\V{\I}(\varphi \supset \psi)=1$ iff $\V{\I}(\varphi)=0$ or $\V{\I}(\psi)=1$ (or both).
    \item[($\equiv$)] $\V{\I}(\varphi \equiv \psi)=1$ iff $\V{\I}(\varphi)=\V{\I}(\psi)$.
  \item[\it Truth Tables:] Use the semantics to fill out the \textsc{characteristic truth tables} given below:
\end{enumerate}


\begin{table}[htb]
\begin{center}
\begin{tabular}{c|c}
$\varphi$ & $\neg\varphi$\\
\hline
1 & 0\\
0 & 1 
\end{tabular}
\ \ \ \ 
\begin{tabular}{c|c|c|c|c|c}
$\varphi$ & $\psi$ & $\varphi\wedge\psi$ & $\varphi\vee\psi$ & $\varphi\supset\psi$ & $\varphi\equiv\psi$\\
\hline
1 & 1 & 1 & 1 & 1 & 1\\
1 & 0 & 0 & 1 & 0 & 0\\
0 & 1 & 0 & 1 & 1 & 0\\
0 & 0 & 0 & 0 & 1 & 1
\end{tabular}
\end{center}
% \caption{The characteristic truth tables for the connectives of SL.}
% \label{table.CharacteristicTTs}
\end{table}


\begin{enumerate}[leftmargin=1.5in,labelsep=.15in] %,label=(\arabic*)]%,label=\roman*]
  \item[\it Sentential Operators:] The connectives are \textsc{sentential operators} which map sentences to sentences.
  \item[\it Truth Functional:] The connectives express truth-functions:\\ 
    $\V{\I}(\neg\varphi)=1-\V{\I}(\varphi)$;\\
    $\V{\I}(\varphi\wedge\psi)=\V{\I}(\varphi)\times\V{\I}(\psi)$.
  \item[\sc Homework:] Given an interpretation $\I$, specify the truth-values of $\varphi\vee\psi$, $\varphi\supset\psi$, and $\varphi\equiv\psi$ as a function of the truth-values of $\varphi$ and $\psi$ in a similar fashion as above.
    % $\V{\I}(\varphi\vee\psi)=1-([1-\V{\I}(\varphi)]\times[1-\V{\I}(\psi)])$;\\
    % $\V{\I}(\varphi\supset\psi)=1-(\V{\I}(\varphi)\times[1-\V{\I}(\psi)])$;\\
    % \mbox{$\V{\I}(\varphi\equiv\psi)=[1-(\V{\I}(\varphi)\times[1-\V{\I}(\psi)])]\times[1-(\V{\I}(\psi)\times[1-\V{\I}(\varphi)])]$.}
  \item[\bf Task 1:] How many unary/binary truth-functions are there?
  \item[\it Adequacy:] Given the expressive limitations of SL, what should we hope to be able to adequately regiment?
\end{enumerate}





\section*{Examples}

\subsection*{\sc Complex Arguments}

\subsubsection*{\it \textbf{Rain}}

\begin{enumerate}
  \item[(1)] If it is raining on a week day, Sam took his car.
  \item[(2)] Kate borrowed Sam's car only if Sam did not take it.
  \item[(3)] Kate borrowed Sam's car just in case she visited her parents.
  \item[(3)] It is raining and Kate visited her parents.
  \item[\therefore] It is not a week day.
\end{enumerate}

\noindent
\textbf{Task 2:} Regiment this argument and construct its truth table.
\vspace{.05in}

\noindent
\textit{Observe:} This argument can be adequately regimented and evaluate in SL.


\subsection*{\sc Conjunction}

\subsubsection*{\it \textbf{Gym}}

\begin{enumerate}
  \item[(1)] Kate took a shower and went to the gym.
  \item[\therefore] Kate went to the gym and took a shower.
\end{enumerate}

\noindent
\textbf{Task 3:} Regiment this argument and construct its truth table.
\vspace{.05in}

\noindent
\textit{Observe:} Conjunction in English can track temporal order.
\vspace{.05in}

\noindent
\textit{Question:} How can we capture the invalidity of this argument in SL?




\subsection*{\sc Disjunction}

\subsection*{\it \textbf{Vault}}

\begin{enumerate}
  \item[(1)] If Kin uses the remote, the trunk will open. 
  \item[(2)] If Adi tries the handle, the trunk will open.
  \item[(3)] If Kin uses the remote and Adi tries the handle, the trunk won't open.
  \item[\therefore] If Kin uses the remote or Adi tries the handle, the trunk will open. 
\end{enumerate}

\noindent
\textbf{Task 4:} Regiment this argument and construct its truth table.
\vspace{.05in}

\noindent
\textit{Observe:} We cannot regiment the conclusion with inclusive-`or'.
\vspace{.05in}

\noindent
\textit{Question:} Can we salvage the validity of this argument?



\subsection*{\sc The Material Conditional}

\subsubsection*{\it \textbf{Roses}}

\begin{enumerate}
  \item[(1)] Sugar is sweet.
  \item[\therefore] The roses are only red if sugar is sweet.
\end{enumerate}


\noindent
\textbf{Task 5:} Regiment this argument and construct its truth table.
\vspace{.05in}

\noindent
\textit{Observe:} The locution `only if' appears to assert something stronger than $\supset$.
\vspace{.05in}




\subsubsection*{\it \textbf{Vacation}}

\begin{enumerate}
  \item[(1)] Casey is not on vacation.
  \item[\therefore] If Casey is on vacation, then he is in Paris.
\end{enumerate}





\subsubsection*{\it \textbf{Crimson}}

\begin{enumerate}
  \item[(1)] Mary doesn't like the ball unless it is crimson.
  \item[(2)] Mary likes the ball.
  \item[\therefore] If the ball is blue, then Mary likes it.
\end{enumerate}






\subsection*{\sc The Biconditional}

\subsubsection*{\it \textbf{Rectangle}}

\begin{enumerate}
  \item[(1)] The room is a square.
  \item[(2)] The room is a rectangle.
  \item[\therefore] The room is a square if and only if it is a rectangle.
\end{enumerate}





\subsubsection*{\it \textbf{Work}}

\begin{enumerate}
  \item[(1)] Kin isn't a professor.
  \item[(2)] Sue isn't a chef.
  \item[\therefore] Kin is a professor just in case Sue is a chef.
\end{enumerate}




\section*{Applications}

\begin{enumerate}[leftmargin=1.5in,labelsep=.15in] %,label=(\arabic*)]%,label=\roman*]
  \item[\it Objection:] The semantics for SL is not good for anything.
  \item[\it Response:] SL is perfect for necessary claims (like in mathematics), as well as sentences where we only care about their truth-value as opposed to their modal profile or subject-matter.
\end{enumerate}




% \section*{Further Notions}
%
% \begin{enumerate}[leftmargin=1.2in,labelsep=.15in] %,label=(\arabic*)]%,label=\roman*]
%   \item[\it Validity:] An argument in SL is \textit{valid} just in case its conclusion is true in any interpretation in which its premises are true.
%   \item[\it Soundness:] An argument is \textit{sound} just in case it is both valid and has true premises.
%   \item[\it Truth:] Soundness reaches beyond the scope of any logic course since truth on an interpretation requires subject-specific knowledge.
%   \item[\it Tautology:] We can talk about sentences of formal languages being \textit{tautologies} just in case they are true on all interpretations.
%   \item[\it Contradiction:] A sentence is a \textit{contradiction} just in case it is false on all interpretations.
%   \item[\it Logical Entailment:] One sentence \textit{logically entails} another just in case every interpretation in which the former is true also makes the latter true.
%   \item[\it Logical Equivalence:] One sentence is \textit{logically equivalent} to another just in case they logically entail each other.
%   \item[\it Consistency:] A set of sentences is \textit{consistent} just in case there is an interpretation which makes every sentence in the set true, and \textit{inconsistent} otherwise.
% \end{enumerate}
%
%
%
% \section*{Examples}
%
% \noindent
% Which sets of sentences are consistent? (e.g., is $\set{(1),(2)}$ consistent?)
%
% \subsection*{\it \textbf{Taller}}
%
% \begin{enumerate}
%   \item[(1)] Liza is taller than Sue.
%   \item[(2)] Sue is taller than Paul.
%   \item[(3)] Paul is taller than Liza.
% \end{enumerate}
%
%
%
%
% \subsection*{\it \textbf{Lost}}
%
% \begin{enumerate}
%   \item[(4)] Kim is either in Somerville or Cambridge.
%   \item[(5)] If Kim is in Somerville, then she is not far from home.
%   \item[(6)] If Kim is not far from home, then she is in Cambridge.
%   \item[(7)] Kim is not in Cambridge.
% \end{enumerate}



% \vfill
%
% \bibliographystyle{Phil_Review} %%bib style found in bst folder, in bibtex folder, in texmf folder.
% \bibliography{Zotero} %%bib database found in bib folder, in bibtex folder


\end{document}

