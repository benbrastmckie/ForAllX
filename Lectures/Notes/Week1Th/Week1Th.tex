\documentclass[a4paper, 11pt]{article} % Font size (can be 10pt, 11pt or 12pt) and paper size (remove a4paper for US letter paper)
\usepackage[protrusion=true,expansion=true]{microtype} % Better typography
\usepackage{graphicx} % Required for including pictures
\usepackage{wrapfig} % Allows in-line images
\usepackage{enumitem} %%Enables control over enumerate and itemize environments
\usepackage{setspace}
\usepackage{amssymb, amsmath, mathrsfs} %%Math packages
\usepackage{stmaryrd}
\usepackage{mathtools}
\usepackage{mathpazo} % Use the Palatino font
\usepackage[T1]{fontenc} % Required for accented characters
\usepackage{array}
\usepackage{bibentry}
\usepackage[round]{natbib} %%Or change 'round' to 'square' for square backers
\setcitestyle{aysep={}}

% \linespread{1} % Change line spacing here, Palatino benefits from a slight increase by default

\newcommand{\corner}[1]{\ulcorner#1\urcorner} %%Corner quotes
\newcommand{\tuple}[1]{\langle#1\rangle} %%Angle brackets
\newcommand{\set}[1]{\lbrace#1\rbrace} %%Set brackets
\newcommand{\interpret}[1]{\llbracket#1\rrbracket} %%Double brackets
%\DeclarePairedDelimiter\ceil{\lceil}{\rceil}    
\def\therefore{\ensuremath{\ldotp\dot{}\,\ldotp}}
\newcommand{\I}{\mathcal{I}}

\makeatletter
\renewcommand\@biblabel[1]{\textbf{#1.}} % Change the square brackets for each bibliography item from '[1]' to '1.'
\renewcommand{\@listI}{\itemsep=0pt} % Reduce the space between items in the itemize and enumerate environments and the bibliography

\renewcommand{\maketitle}{ % Customize the title - do not edit title and author name here, see the TITLE block below
\begin{flushright} % Right align
{\LARGE\@title} % Increase the font size of the title

\vspace{10pt} % Some vertical space between the title and author name

{\@author} % Author name
\\\@date % Date

\vspace{10pt} % Some vertical space between the author block and abstract
\end{flushright}
}

%----------------------------------------------------------------------------------------
%	TITLE
%----------------------------------------------------------------------------------------

\title{\textbf{The Connectives}} % Subtitle

\author{\textsc{Logic I}\\ \em Benjamin Brast-McKie} % Institution

\date{\today} % Date

%----------------------------------------------------------------------------------------

\begin{document}

\maketitle % Print the title section

\thispagestyle{empty}

%----------------------------------------------------------------------------------------

%%% OUTLINE

% Previously we introduced the sentence letters and connectives, referring to the set of all sentences in SL
  % now we will define the sentences of SL a little more carefully
  % to do so, we will:
    % distinguish between object language and metalangauge
    % use quotes to distinguish between use and mention
    % use variables to quantify over the strings of SL
    % use corner quotes to provide a recursive definition of the sentences of SL
  % we will also relate the semantic clauses from before to truth-tables for each of the connectives

\section*{Definitions}

\begin{enumerate}[leftmargin=1.5in,labelsep=.15in] %,label=(\arabic*)]%,label=\roman*]
  \item[\it Previously:] We considered the sentences that could be constructed from the sentence letters with the connectives. We will now seek to specify this construction precisely.
  \item[\it Object Language:] We will be concerned to define the sentences of SL, where the language of SL $\mathcal{L}=\tuple{\mathbb{L},\neg,\vee,\wedge,\supset,\equiv,(,)}$ will be referred to as the \textsc{object language}.
  \item[\it Metalanguage:] We will present this definition with the resources of our \textsc{metalangauge} mathematical English.
  \item[\it Strings:] A \textsc{string} (or \textsc{expression}) of SL is any sequence of symbols from the language of SL.
  \item[\it Quotation:] To talk about strings we will need to name them, where a quoted string is the \textsc{canonical name} for the string quoted.
  \item[\bf Example:] The argument $A, A\supset B \vdash B$ \textit{uses} the strings `$A$', `$A \supset B$', and `$B$'. By contrast, we may say that `$A$' is a sentence letter by using `$A$' to \textit{mention} $A$.
  \item[\it Metalinguistic Variables:] Letting $\varphi,\psi,\chi,\ldots$ be variables whose values are strings, we may quantify over the strings of SL in order to define the sentences of SL.
\end{enumerate}




\section*{The Sentences of SL}

\begin{enumerate}[leftmargin=1.5in,labelsep=.15in] %,label=(\arabic*)]%,label=\roman*]
  \item Every atomic sentence in $\mathbb{L}$ is a wff.
  \item If $\varphi$ and $\psi$ are wffs, then:
    \begin{enumerate}
      \item $\neg\varphi$ is a wff;
      \item $(\varphi\wedge\psi)$ is a wff;
      \item $(\varphi\vee\psi)$ is a wff;
      \item $(\varphi\supset\psi)$ is a wff; and
      \item $(\varphi\equiv\psi)$ is a wff.
    \end{enumerate}
  \item Nothing else is a wff.
\end{enumerate}




\section*{Observations and Conventions}

\begin{enumerate}[leftmargin=1.5in,labelsep=.15in] %,label=(\arabic*)]%,label=\roman*]
  \item[\it Corner Quotes:] Strictly speaking, this definition is non-sense and we need to use corner quotes to fix it.
  \item[\it Conventions:] We will often drop quotes and parentheses for ease: $A\vee B\vee C$ vs $A\vee B\wedge C$.
  \item[\it Sentences:] Since the wffs are good candidates for interpretation (it makes sense to assign them truth-values), we may identify the wffs with the sentences of SL. By contrast, the wffs of QL will not all be sentences of QL.
  \item[\it Therefore:] We will use $\therefore$ or a line to represent arguments, though these are not parts of SL.
\end{enumerate}




\section*{Truth Functionality}

\begin{enumerate}[leftmargin=1.5in,labelsep=.15in] %,label=(\arabic*)]%,label=\roman*]
  \item[\it Sentential Operators:] The connectives are \textsc{sentential operators} which map sentences to sentences.
  \item[\it Truth Functional:] $\I(\neg\varphi)=1-\I(\varphi)$;\\
    $\I(\varphi\wedge\psi)=\I(\varphi)\times\I(\psi)$;\\
    $\I(\varphi\vee\psi)=1-([1-\I(\varphi)]\times[1-\I(\psi)])$;\\
    $\I(\varphi\supset\psi)=1-(\I(\varphi)\times[1-\I(\psi)])$;\\
    \mbox{$\I(\varphi\equiv\psi)=[1-(\I(\varphi)\times[1-\I(\psi)])]\times[1-(\I(\psi)\times[1-\I(\varphi)])]$.}
\end{enumerate}




% \section*{Examples}
%
% \subsection*{\it \textbf{Snow}}
%
% \begin{enumerate}
%   \item[(1)] It's snowing.
%   \item[\therefore] John drove to work.
% \end{enumerate}
%
% \noindent
% \textit{This argument may be compelling, but not certain.}
%
% \subsection*{\it \textbf{Red}}
%
% \begin{enumerate}
%   \item[(1)] The ball is crimson.
%   \item[\therefore] The ball is red.
% \end{enumerate}
%
% \noindent
% \textit{This argument provides certainty, but not on all interpretations.}
%
% \subsection*{\it \textbf{Museum}}
%
% \begin{enumerate}
%   \item[(1)] Kate is either at home or at the Museum.
%   \item[(2)] Kate is not at home.
%   \item[\therefore] Kate is at the Museum.
% \end{enumerate}
%
% \noindent
% \textit{This argument's certainty is independent of the interpretation.}








% \vfill
%
% \bibliographystyle{Phil_Review} %%bib style found in bst folder, in bibtex folder, in texmf folder.
% \bibliography{Zotero} %%bib database found in bib folder, in bibtex folder


\end{document}
