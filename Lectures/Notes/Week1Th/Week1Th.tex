\documentclass[a4paper, 11pt]{article} % Font size (can be 10pt, 11pt or 12pt) and paper size (remove a4paper for US letter paper)
\usepackage[protrusion=true,expansion=true]{microtype} % Better typography
\usepackage{graphicx} % Required for including pictures
\usepackage{wrapfig} % Allows in-line images
\usepackage{enumitem} %%Enables control over enumerate and itemize environments
\usepackage{setspace}
\usepackage{amssymb, amsmath, mathrsfs} %%Math packages
\usepackage{stmaryrd}
\usepackage{mathtools}
\usepackage{mathpazo} % Use the Palatino font
\usepackage[T1]{fontenc} % Required for accented characters
\usepackage{array}
\usepackage{bibentry}
\usepackage[round]{natbib} %%Or change 'round' to 'square' for square backers
\setcitestyle{aysep={}}

% \linespread{1} % Change line spacing here, Palatino benefits from a slight increase by default

\newcommand{\corner}[1]{\ulcorner#1\urcorner} %%Corner quotes
\newcommand{\tuple}[1]{\langle#1\rangle} %%Angle brackets
\newcommand{\set}[1]{\lbrace#1\rbrace} %%Set brackets
\newcommand{\interpret}[1]{\llbracket#1\rrbracket} %%Double brackets
%\DeclarePairedDelimiter\ceil{\lceil}{\rceil}    
\def\therefore{\ensuremath{\ldotp\dot{}\,\ldotp}}
\newcommand{\I}{\mathcal{I}}

\makeatletter
\renewcommand\@biblabel[1]{\textbf{#1.}} % Change the square brackets for each bibliography item from '[1]' to '1.'
\renewcommand{\@listI}{\itemsep=0pt} % Reduce the space between items in the itemize and enumerate environments and the bibliography

\renewcommand{\maketitle}{ % Customize the title - do not edit title and author name here, see the TITLE block below
\begin{flushright} % Right align
{\LARGE\@title} % Increase the font size of the title

\vspace{10pt} % Some vertical space between the title and author name

{\@author} % Author name
\\\@date % Date

\vspace{30pt} % Some vertical space between the author block and abstract
\end{flushright}
}

%----------------------------------------------------------------------------------------
%	TITLE
%----------------------------------------------------------------------------------------

\title{\textbf{The Connectives}} % Subtitle

\author{\textsc{Logic I}\\ \em Benjamin Brast-McKie} % Institution

\date{\today} % Date

%----------------------------------------------------------------------------------------

\begin{document}

\maketitle % Print the title section

\thispagestyle{empty}

%----------------------------------------------------------------------------------------

%%% OUTLINE

% Previously we introduced the sentence letters and connectives, referring to the set of all sentences in SL
  % now we will define the sentences of SL a little more carefully












% \section*{Definitions}
%
% \begin{enumerate}[leftmargin=1.5in,labelsep=.15in] %,label=(\arabic*)]%,label=\roman*]
%   \item[\it Proposition:] A \textsc{proposition} is a way for things to be which is either true or false.
%   \item[\it Declarative Sentence:]  A \textsc{declarative sentence} is a grammatical string of symbols which, on an interpretation, expresses a proposition that is either true or false.
%   \item[\it Argument:] An \textsc{Argument} is a finite sequence of declarative sentences where the final sentence is the \textsc{conclusion} and the preceding sentences are the \textsc{premises}.
% \end{enumerate}
%
%
%
% \section*{Examples}
%
% \subsection*{\it \textbf{Snow}}
%
% \begin{enumerate}
%   \item[(1)] It's snowing.
%   \item[\therefore] John drove to work.
% \end{enumerate}
%
% \noindent
% \textit{This argument may be compelling, but not certain.}
%
% \subsection*{\it \textbf{Red}}
%
% \begin{enumerate}
%   \item[(1)] The ball is crimson.
%   \item[\therefore] The ball is red.
% \end{enumerate}
%
% \noindent
% \textit{This argument provides certainty, but not on all interpretations.}
%
% \subsection*{\it \textbf{Museum}}
%
% \begin{enumerate}
%   \item[(1)] Kate is either at home or at the Museum.
%   \item[(2)] Kate is not at home.
%   \item[\therefore] Kate is at the Museum.
% \end{enumerate}
%
% \noindent
% \textit{This argument's certainty is independent of the interpretation.}








% \vfill
%
% \bibliographystyle{Phil_Review} %%bib style found in bst folder, in bibtex folder, in texmf folder.
% \bibliography{Zotero} %%bib database found in bib folder, in bibtex folder


\end{document}
