\documentclass[a4paper, 11pt]{article} % Font size (can be 10pt, 11pt or 12pt) and paper size (remove a4paper for US letter paper)
\usepackage[protrusion=true,expansion=true]{microtype} % Better typography
\usepackage{graphicx} % Required for including pictures
\usepackage{wrapfig} % Allows in-line images
\usepackage{enumitem} %%Enables control over enumerate and itemize environments
\usepackage{setspace}
\usepackage{amssymb, amsmath, mathrsfs} %%Math packages
\usepackage{stmaryrd}
\usepackage{mathtools}
\usepackage{multicol} 
\usepackage{mathpazo} % Use the Palatino font
\usepackage[T1]{fontenc} % Required for accented characters
\usepackage{array}
\usepackage{bibentry}
\usepackage{prooftrees} 
\usepackage[round]{natbib} %%Or change 'round' to 'square' for square backers
\setcitestyle{aysep={}}

% \linespread{1} % Change line spacing here, Palatino benefits from a slight increase by default

\newcommand{\corner}[1]{\ulcorner#1\urcorner} %%Corner quotes
\newcommand{\tuple}[1]{\langle#1\rangle} %%Angle brackets
\newcommand{\set}[1]{\lbrace#1\rbrace} %%Set brackets
\newcommand{\interpret}[1]{\llbracket#1\rrbracket} %%Double brackets
%\DeclarePairedDelimiter\ceil{\lceil}{\rceil}    
\def\therefore{\ensuremath{\ldotp\dot{}\,\ldotp}}
\newcommand{\I}{\mathcal{I}}
\newcommand{\J}{\mathcal{J}}
\newcommand{\B}{\mathcal{B}}
\newcommand{\even}{\texttt{Even}}
\newcommand{\comp}{\texttt{Comp}}
\newcommand{\res}{\texttt{Res}}
\newcommand{\simp}{\texttt{Simple}}
\newcommand{\leng}{\texttt{Length}}
\newcommand{\V}[1]{\mathcal{V}_{#1}} %%Corner quotes

\makeatletter
\renewcommand\@biblabel[1]{\textbf{#1.}} % Change the square brackets for each bibliography item from '[1]' to '1.'
\renewcommand{\@listI}{\itemsep=0pt} % Reduce the space between items in the itemize and enumerate environments and the bibliography

\renewcommand{\maketitle}{ % Customize the title - do not edit title and author name here, see the TITLE block below
\begin{flushright} % Right align
{\LARGE\@title} % Increase the font size of the title

\vspace{10pt} % Some vertical space between the title and author name

{\@author} % Author name
\\\@date % Date

\vspace{10pt} % Some vertical space between the author block and abstract
\end{flushright}
}

%----------------------------------------------------------------------------------------
%	TITLE
%----------------------------------------------------------------------------------------

\title{\textbf{The Completeness of SL Tree Proofs}} % Subtitle

\author{\textsc{Logic I}\\ \em Benjamin Brast-McKie} % Institution

\date{\today} % Date

%----------------------------------------------------------------------------------------

\begin{document}

\maketitle % Print the title section

\thispagestyle{empty}

%----------------------------------------------------------------------------------------

\section*{The Proof}

\begin{enumerate}
  \item[\it Completeness:] Every unsatisfiable root has a closed tree: $\Gamma \vDash \bot \Rightarrow \Gamma \vdash \bot$.
  \item[\it Contrapositive:] If there is no closed tree with root $\Gamma$, then $\Gamma$ is satisfiable.
  \item[\it Lemma 6:] For any tree $X$ with root $\Gamma$, there is a complete tree $X'$ with root $\Gamma$. 
    \begin{itemize}
      \item Assume there is no closed tree with root $\Gamma$.
      \item Roots are trees, and so $\Gamma$ has a complete tree $X$. 
      \item So $X$ is a complete open tree with a complete open branch $\B$. 
    \end{itemize}
  \item[\bf Note:] This result is purely syntactic.
  \item[\it Lemma 7:] Every complete open branch in an SL tree is satisfiable.
    \begin{itemize}
      \item So $\B$ is satisfiable, and so $\Gamma$ is satisfiable. 
      \item By contraposition, if $\Gamma \vDash \bot$, then $\Gamma \vdash \bot$.
    \end{itemize}
\end{enumerate}




\section*{Resolution}

Let the \textit{resolution} $\res(\varphi)$ provide an upper bound on the number of times that $\varphi$ and its descendants could be resolved in an SL tree. 

\begin{enumerate}
  \item $\res(\varphi)=0$ if $\varphi$ is a literal.
  \item For any SL sentences $\varphi$ and $\psi$:
    \begin{itemize}
      \item $\res(\neg\neg\varphi)=\res(\varphi)+1$.
      \item $\res(\varphi \wedge \psi)=\res(\varphi)+\res(\psi)+1$.
      \item $\res(\neg(\varphi \wedge \psi))=\res(\neg\varphi)+\res(\neg\psi)+1$.
      \item $\res(\varphi \vee \psi)=\res(\varphi)+\res(\psi)+1$.
      \item $\res(\neg(\varphi \vee \psi))=\res(\neg\varphi)+\res(\neg\psi)+1$.
      \item $\res(\varphi \supset \psi)=\res(\neg\varphi)+\res(\psi)+1$.
      \item $\res(\neg(\varphi \supset \psi))=\res(\varphi)+\res(\neg\psi)+1$.
      \item $\res(\varphi \equiv \psi)=\res(\varphi)+\res(\psi) + \res(\neg\varphi)+\res(\neg\psi)+1$.
      \item $\res(\neg(\varphi \equiv \psi))=\res(\varphi) + \res(\neg\psi) + \res(\neg\varphi)+\res(\psi)+1$.
    \end{itemize}
  \item[\it Resolution Set:] Let $[X]$ be the set of SL sentences that are resolvable in a branch of $X$. 
  \item[\it Tree Resolution:] Let $\res(X)=\sum\limits_{\varphi\in [X]}\res(\varphi)$ be an upper bound on resolutions in $X$.
\end{enumerate}




\section*{Supporting Lemmas}

\begin{enumerate}
  \item[\it Lemma 4:] Every SL tree $X$ has a finite number of branches.  
  \item[\it Lemma 5:] For any SL tree $X$ with root $\Gamma$ and $\varphi\in[X]$, there is an SL tree $Y$ with root $\Gamma$ where $\res(Y)<\res(X)$. 
    \begin{itemize}
      \item Let $X$ be an SL tree with root $\Gamma$ where $\varphi\in[X]$. 
      \item By \textit{Lemma 4}, $\varphi$ is resolvable in finitely many branches of $X$. 
      \item So there is a tree $Y$ with root $\Gamma$ that resolves $\varphi$ throughout $X$. 
      \item So $\varphi\notin[Y]$ but the children of $\varphi$ could be in $[Y]$.
      % \item But $\varphi\in\set{\neg\neg\psi,\psi\wedge\chi,\neg(\psi\wedge\chi),\ldots}$
      \item[\it Case 1:] Assume $\varphi=\neg\neg\psi$ where $\psi\in[Y]$ and $\psi\notin[X]$.
      \item So $\res(\psi)<\res(\varphi)$, and so $\res(Y)<\res(X)$.
      \item[\it Case n:] The other cases are similar. 
    \end{itemize}
\end{enumerate}








\section*{Lemma 6}

\begin{enumerate}
  \item[\it Proof:] For any $\Gamma$-tree $X$, there is a complete $\Gamma$-tree $X'$.  
  \item[\it Base:] Assume $X$ is a $\Gamma$-tree where $\res(X)=0$.
    \begin{itemize}
      \item So every $[X]$ is empty, so $X$ is complete. 
    \end{itemize}
  \item[\it Hypothesis:] Every $\Gamma$-tree $X$ where $\res(X)\leq n$ has a complete $\Gamma$-tree $X'$.
  \item[\it Induction:] Let $X$ be a $\Gamma$-tree where $\res(X)=n+1$.
    \begin{itemize}
      \item Since $\res(X)>0$, there is some $\varphi\in[X]$.
      \item By \textit{Lemma 5}, there is some $\Gamma$-tree $Y$ where $\res(Y)<\res(X)$.
      \item By hypothesis, there is a complete $\Gamma$-tree $Y'$.
    \end{itemize}
  \item[\it Conclusion:] By strong induction, QED.
\end{enumerate}



% \section*{Descendants}
%
% \begin{enumerate}
%   \item[\it Tree:] We define the \textit{descendants} of $\varphi$ in an SL tree $X$ recursively:
%     \begin{itemize}
%       \item Every child of $\varphi$ is a descendant of $\varphi$.
%       \item Every child of a descendant of $\varphi$ is a descendant of $\varphi$.
%       \item Nothing else is a descendant of $\varphi$.
%     \end{itemize}
%   \item[\it Branch:] We define the \textit{descendants} of $\varphi$ in a branch $\B$ of an SL tree $X$ to be any descendants of $\varphi$ in $X$ that occur in $\B$. 
% \end{enumerate}
%

\section*{Finite Lemma}

\begin{enumerate}
  \item[\it Proof:] Every branch $\B$ in an SL tree contains finitely many sentences.
  \item[\it Base:] Assume $\B$ belongs to an SL tree $X$ where $\leng(X)=0$, so finite. 
    % \begin{itemize}
    %   \item So $\B$ is identical to the root which is finite. 
    % \end{itemize}
  \item[\it Hypothesis:] Assume that every branch $\B$ of an SL tree $X$ of $\leng(X)=n$ has a finite number of sentences. 
  \item[\it Induction:] Assume that $\B'$ belongs to an SL tree $X'$ of $\leng(X)=n+1$. 
    \begin{itemize}
      \item Let $X$ be a tree where $X'$ is the result of resolving a sentence in $X$. 
      \item So $\leng(X)=n$.
      \item By hypothesis, every branch $\B$ of $X$ has a finite number of branches.
      \item $\B'$ includes at most two more sentences than any branch $\B$ in $X$.
      \item Thus $\B'$ has a finite number of sentences. 
    \end{itemize}
\end{enumerate}



\section*{Lemma 7}

\begin{enumerate}
  \item[\it Proof:] Every complete open branch in an SL tree is satisfiable.
  \item[\it Assume:] Let $\B$ be a complete open branch in an SL tree. 
    \begin{itemize}
      \item Let $\I(\varphi)=1$ \textit{iff} $\varphi$ is a sentence letter in $\B$. 
      \item By the \textit{Finite Lemma}, we may assign sentences in $\B$ a position number where the leaf is 0. 
    \end{itemize}
  \item[\it Base:] Assume $\varphi$ has position 0. 
    \begin{itemize}
      \item Since $\B$ is complete and open, $\varphi$ is a literal. 
      \item[\it Case 1:] If $\varphi$ is a sentence letter, $\V{\I}(\varphi)=\I(\varphi)=1$. 
      \item[\it Case 2:] Assume $\varphi=\neg\psi$ where $\psi$ is a sentence letter. 
      \item Since $\B$ is open, $\psi$ does not occur in $\B$.
      \item So $\V{\I}(\psi)=\I(\psi)=0$, and so $\V{\I}(\varphi)=\V{\I}(\neg\psi)=1$.
    \end{itemize}
  \item[\it Hypothesis:] $\V{\I}(\varphi)=1$ for every $\varphi$ with position $k\leq n$ in $\B$. 
  \item[\it Induction:] Assume $\varphi$ has position $n+1$ in $\B$. 
  \item[\it Case 1:] $\varphi$ is a literal, so $\V{\I}(\varphi)=1$ as above.
  \item[\it Case 2:] $\varphi=\neg\neg\psi$.
    \begin{itemize}
      \item Since $\B$ is complete, $\psi$ occurs in $\B$ in position $k\leq n$. 
      \item By hypothesis, $\V{\I}(\psi)=1$, and so $\V{\I}(\varphi)=\V{\I}(\neg\neg\psi)=1$.
    \end{itemize}
  \item[\it Case 3:] $\varphi=\psi\wedge\chi$.
  \item[\it Case 4:] $\varphi=\neg(\psi\wedge\chi)$.
    \begin{itemize}
      \item Since $\B$ is complete, $\neg\psi,\neg\chi$ occur in $\B$ in positions $j,k\leq n$. 
      \item By hypothesis, $\V{\I}(\neg\psi)=1$ or $\V{\I}(\neg\chi)=1$.
      \item So $\V{\I}(\psi)=0$ or $\V{\I}(\chi)=0$, and so $\V{\I}(\psi\wedge\chi)=0$.
      \item Thus $\V{\I}(\varphi)=\V{\I}(\neg(\psi\wedge\chi))=1$.
    \end{itemize}
  \item[\it Case n:] $\varphi=\neg(\psi\equiv\chi)$.
    \begin{itemize}
      \item Since $\B$ is complete, $\psi$ and $\neg\chi$ occur in $\B$ in positions $j,k\leq n$, or else $\neg\psi$ and $\chi$ occur in $\B$ in positions $j,k\leq n$.
      \item By hypothesis, $\V{\I}(\psi)=\V{\I}(\neg\chi)=1$ or $\V{\I}(\neg\psi)=\V{\I}(\chi)=1$.  
      \item In either case, $\V{\I}(\psi)\neq\V{\I}(\chi)$, and so $\V{\I}(\psi\equiv\chi)=0$.
      \item Thus $\V{\I}(\varphi)=\V{\I}(\neg(\psi\equiv\chi))=1$.
    \end{itemize}
\end{enumerate}










\end{document}

