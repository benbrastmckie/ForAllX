\documentclass[a4paper, 11pt]{article} % Font size (can be 10pt, 11pt or 12pt) and paper size (remove a4paper for US letter paper)
\usepackage[protrusion=true,expansion=true]{microtype} % Better typography
\usepackage{graphicx} % Required for including pictures
\usepackage{wrapfig} % Allows in-line images
\usepackage{enumitem} %%Enables control over enumerate and itemize environments
\usepackage{setspace}
\usepackage{amssymb, amsmath, mathrsfs} %%Math packages
\usepackage{stmaryrd}
\usepackage{mathtools}
\usepackage{multicol} 
\usepackage{mathpazo} % Use the Palatino font
\usepackage[T1]{fontenc} % Required for accented characters
\usepackage{array}
\usepackage{bibentry}
\usepackage{prooftrees} 
\usepackage[round]{natbib} %%Or change 'round' to 'square' for square backers
\setcitestyle{aysep={}}
% \usepackage{fitchproof} 

% \linespread{1} % Change line spacing here, Palatino benefits from a slight increase by default

\newcommand{\corner}[1]{\ulcorner#1\urcorner} %%Corner quotes
\newcommand{\tuple}[1]{\langle#1\rangle} %%Angle brackets
\newcommand{\set}[1]{\lbrace#1\rbrace} %%Set brackets
\newcommand{\interpret}[1]{\llbracket#1\rrbracket} %%Double brackets
%\DeclarePairedDelimiter\ceil{\lceil}{\rceil}    
\def\therefore{\ensuremath{\ldotp\dot{}\,\ldotp}}
\newcommand{\I}{\mathcal{I}}
\newcommand{\J}{\mathcal{J}}
\newcommand{\B}{\mathcal{B}}
\newcommand{\even}{\texttt{Even}}
\newcommand{\comp}{\texttt{Comp}}
\newcommand{\res}{\texttt{Res}}
\newcommand{\simp}{\texttt{Simple}}
\newcommand{\leng}{\texttt{Length}}
\newcommand{\V}[1]{\mathcal{V}_{#1}} %%Corner quotes

\makeatletter
\renewcommand\@biblabel[1]{\textbf{#1.}} % Change the square brackets for each bibliography item from '[1]' to '1.'
\renewcommand{\@listI}{\itemsep=0pt} % Reduce the space between items in the itemize and enumerate environments and the bibliography

\renewcommand{\maketitle}{ % Customize the title - do not edit title and author name here, see the TITLE block below
\begin{flushright} % Right align
{\LARGE\@title} % Increase the font size of the title

\vspace{10pt} % Some vertical space between the title and author name

{\@author} % Author name
\\\@date % Date

\vspace{-10pt} % Some vertical space between the author block and abstract
\end{flushright}
}

%----------------------------------------------------------------------------------------
%	TITLE
%----------------------------------------------------------------------------------------

\title{\textbf{Natural Deduction in SL: Part I}} % Subtitle

\author{\textsc{Logic I}\\ \em Benjamin Brast-McKie} % Institution

\date{\today} % Date

%----------------------------------------------------------------------------------------

\begin{document}

\maketitle % Print the title section

\thispagestyle{empty}

%----------------------------------------------------------------------------------------

\section*{Motivation}

\begin{enumerate}
  \item[\it Proof Trees:] Proof trees provide an efficient way to evaluate validity.
    \begin{itemize}
      \item If an argument is valid, the tree will close.
      \item If an argument is invalid, the tree will give us an interpretation.
    \end{itemize}
  \item[\it Unnatural:] But proof trees do not provide a natural line of reasoning.
    \begin{itemize}
      \item Proof trees go by \textit{reductio} which are not explanatory.
      \item Rules for proof trees are not entirely unnatural.
      \item But trees do not resemble natural reasoning.
    \end{itemize}
  \item[\it Natural Deduction:] How would we describe the patterns of natural deduction?
    \begin{itemize}
      \item Identify a range of intuitively compelling basic inferences in SL.
      \item Such inferences hold in virtue of the meanings of the connectives.
      \item Define a proof to be any composition of basic inferences.
    \end{itemize}
  \item[\it Rules:] Our system will include introduction and elimination rules.
    \begin{itemize}
      \item These rules will describe how to reason with the connectives.
    \end{itemize}
\end{enumerate}




\section*{Conditional}

\begin{enumerate}
  \item[\it Elimination:] $A,\ A \supset B,\ B \supset C\ \vdash C$. 
    \begin{itemize}
      \item Premises justified by `:PR'.
      \item Easy to derive $C$.
      \item What if $A$ was excluded from the premises? 
    \end{itemize}
  \item[\it Introduction:] $A \supset B,\ B \supset C\ \vdash A \supset C$. 
    \begin{itemize}
      \item Need something to work with.
      \item Want to conclude with a conditional claim.
      \item Assumption of $A$ justified by `:AS'.
    \end{itemize}
  \item[\it Subproofs:] Lines in a closed subproof are dead and all else are live.
    \begin{itemize}
      \item $\supset$E can only cite to live lines.
      \item $\supset$I can only cite an appropriate subproof.
    \end{itemize}
\end{enumerate}




\section*{Reiteration}

\begin{enumerate}
  \item[\it Example:] $A\ \vdash D \supset [C \supset (B \supset A)]$.
\end{enumerate}




\section*{Conjunction}

\begin{enumerate}
  \item[\it Elimination:] $A \supset (B\wedge C),\ B \supset D\ \vdash A \supset D$.
  \item[\it Introduction:] $A \wedge B,\ B \supset C\ \vdash A \wedge C$.
\end{enumerate}



\section*{Disjunction}

\begin{enumerate}
  \item[\it Introduction:] $A \vdash B \vee ((A \vee C) \vee D)$.
  \item[\it Elimination:] $A\vee(B\wedge C)\ \vdash (A \vee B) \wedge (A \vee C)$. 
\end{enumerate}



\section*{Biconditional}

\begin{enumerate}
  \item[\it Elimination:] $A \equiv (B \supset [(A \wedge C)\equiv D])\ \vdash (A\wedge B) \supset (D \supset C)$. 
  \item[\it Introduction:] $A \supset (B \wedge C),\ C \supset (B \wedge A) \vdash A \equiv C$.
\end{enumerate}






\section*{Negation}

\begin{enumerate}
  \item[\it Elimination:] $\neg\neg A\ \vdash A$. 
  \item[\it Introduction:] $A \supset (B \wedge C),\ C \supset (B \wedge A) \vdash A \equiv C$.
\end{enumerate}





\section*{Proof}

\begin{enumerate}
  \item[\it Proof:] A natural deduction \textsc{proof} (or \textsc{derivation}) of a conclusion $\varphi$ from a set of premises $\Gamma$ in SD is any sequence of lines ending with $\varphi$ on a live line where every line in the sequence is either:
      \begin{itemize}
        \item[(1)] a premise in $\Gamma$; 
        \item[(2)] a discharged assumption; or
        \item[(3)] follows from previous lines by the rules for SD.
      \end{itemize}
  \item[\it Provable:] An SL sentence $\varphi$ is \textsc{provable} (or \textsc{derivable}) from $\Gamma$ in SD \textit{iff} there is a natural deduction proof (derivation) of $\varphi$ from $\Gamma$ in SD, i.e., $\Gamma \vdash \varphi$. 
  \item[\it Equivalent:] Two sentences $\varphi$ and $\psi$ are \textsc{provably equivalent} (or \textsc{interderivable}) if and only if both $\varphi\vdash\psi$ and $\psi\vdash\varphi$.
  \item[\it Inconsistent:] A set of sentences $\Gamma$ is \textsc{provably inconsistent} if and only if $\Gamma\vdash\bot$ where $\bot$ is our arbitrarily chosen contradiction, e.g., $A\wedge\neg A$.
\end{enumerate}



% \section*{Further Problems}
%
%
% \begin{enumerate}
%   \begin{multicols}{2}
%   \item[\it Law of Excluded Middle:] $A \vee \neg A$. 
%     \begin{itemize}
%       \item $\neg(A \vee \neg A)$ \quad:AS
%       \begin{itemize}
%         \item $A$ \quad:AS 
%         \item $A\vee \neg A$ \quad:$\vee$I 
%         \item $\neg (A\vee \neg A)$ \quad:R 
%       \end{itemize}
%       \item $\neg A$ \quad:$\neg$I 
%       \item $A \vee \neg A$ \quad:$\vee$I 
%       \item $A\vee \neg A$ \quad:$\neg$E 
%     \end{itemize}
%   \item[\it LNC:] $\neg(A \wedge \neg A)$. 
%     \item[\it EXQ:] $A,\ \neg A \vdash B$. (\textit{Ex Falso Quodlibet})
%     \begin{itemize}
%       \item $\neg B$ \quad:AS 
%       \begin{itemize}
%         \item $A$ \quad:R 
%         \item $\neg A$ \quad:R 
%       \end{itemize}
%       \item $B$ \quad:$\neg$E 
%       \item[] ~
%       \item[] ~
%     \end{itemize}
%   \end{multicols}
%   \item $L\equiv \neg O,\ L\vee \neg O\ \vdash L$.
%   \item $A\equiv B,\ \vdash \neg A\equiv\neg B$.
%   \item $Z \supset (C \wedge \neg N),\ \neg Z \supset (N \wedge \neg C)\ \vdash N \vee C$.
% \end{enumerate}

  % \begin{multicols}{2}
  % \end{multicols}



\iffalse

\begin{multicols}{2}


\textit{Conjunction Introduction} (\eand I) \vspace{-1em}
\begin{proof}
	\have[m]{a}{\metaA{}}
	\have[n]{b}{\metaB{}}
	\have[\ ]{c}{\metaA{}\eand\metaB{}} \ai{a, b}
	\have[\ ]{d}{\metaB{}\eand\metaA{}} \ai{a, b}
\end{proof}

\vspace{1em}

\textit{Conditional Introduction} (\eif I) \vspace{-1em}
%\nopagebreak
\begin{proof}
	\open
		\hypo[m]{a}{\metaA{}} \as{for \eif I}{}%\by{want \metaB{}}{}
		\have[n]{b}{\metaB{}}
	\close
	\have[\ ]{ab}{\metaA{}\eif\metaB{}}\ci{a-b}
\end{proof}

\vspace{0.6em}

\textit{Negation Introduction} (\enot I) \vspace{-1em}
\begin{proof}
\open
	\hypo[m]{na}\metaA{} \as{for \enot I}   %\by{:AS for \enot I}{}
	\have[n]{b}\metaB{}
	\have[o]{nb}{\enot\metaB{}}
\close
\have[\ ]{a}[\ ]{\enot\metaA{}}\ni{na-nb}
\end{proof}

\vspace{0.6em}

\textit{Disjunction Introduction} (\eor I) \vspace{-1em}

\begin{proof}
	\have[m]{a}{\metaA{}}
	\have[\ ]{ab}{\metaA{}\eor\metaB{}}\oi{a}
	\have[\ ]{ba}{\metaB{}\eor\metaA{}}\oi{a}
\end{proof}

%\vspace{1.9em} %3.9 for no extra vspaces
\vspace{0.6em}

\textit{Biconditional Introduction} (\eiff I) \vspace{-1em}

\begin{fitchproof}
	\open
		\hypo[i]{a1}{\metaA{}} \as{for \eiff I}
		\have[j]{b1}{\metaB{}}
	\close
\breakline
	\open
		\hypo[k]{b2}{\metaB{}} \as{for \eiff I}
		\have[l]{a2}{\metaA{}}
	\close
	\have[\ ]{ab}{\metaA{}\eiff\metaB{}}\bi{a1-b1,b2-a2}
\end{fitchproof}


\vfill\null
\columnbreak

%\newpage

\textit{Conjunction Elimination} (\eand E) \vspace{-1em}

\begin{proof}
	\have[m]{ab}{\metaA{}\eand\metaB{}}
	\have[\ ]{a}{\metaA{}} \ae{ab}
	\have[\ ]{b}{\metaB{}} \ae{ab}
\end{proof}

%\vspace{1.9em}
%\vspace{2.9em}
\vspace{0.75em}

\textit{Conditional Elimination} (\eif E)  \vspace{-1em}

\begin{proof}
	\have[m]{ab}{\metaA{}\eif\metaB{}}
	\have[n]{a}{\metaA{}}
	\have[\ ]{b}{\metaB{}} \ce{ab,a}
\end{proof}

%\vspace{1em}
\vspace{0.45em}

\textit{Negation Elimination} (\enot E)  \vspace{-1em}
%%note that I think I'm missing some brackets around the sentences on various liens below! works in proof environment but less robust in nd environment. so i ought to fix these here and in negation intro 

\begin{proof}
\open
	\hypo[m]{na}{\enot\metaA{}} \as{for \enot E}
	\ellipsesline
	\have[n]{b}\metaB{}
	\have[o]{nb}{\enot\metaB{}}
\close
\have[\ ]{a}[\ ]\metaA{}\ne{na-nb}
\end{proof}

\vspace{1em}

\textit{Disjunction Elimination} (\eor E)  \vspace{-1em}

\begin{proof}
\have[m]{ab}{\metaA{}\eor\metaB{}}
	\open
		\hypo[i]{a}{\metaA{}} \as{for \eor E}
		\have[j]{c1}{\metaC{}}
	\close
\breakline
	\open
		\hypo[k]{b}{\metaB{}} \as{for \eor E}
		\have[l]{c2}{\metaC{}}
	\close
	\have[\ ]{c}{\metaC{}} \oe{ab,a-c1, b-c2}
\end{proof}

\fi






\end{document}


