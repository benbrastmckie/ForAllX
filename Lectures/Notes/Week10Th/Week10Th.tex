\documentclass[a4paper, 11pt]{article} % Font size (can be 10pt, 11pt or 12pt) and paper size (remove a4paper for US letter paper)
\usepackage[protrusion=true,expansion=true]{microtype} % Better typography
\usepackage{graphicx} % Required for including pictures
\usepackage{wrapfig} % Allows in-line images
\usepackage{enumitem} %%Enables control over enumerate and itemize environments
\usepackage{setspace}
\usepackage{amssymb, amsmath, mathrsfs} %%Math packages
\usepackage{stmaryrd}
\usepackage{mathtools}
\usepackage{multicol} 
\usepackage{mathpazo} % Use the Palatino font
\usepackage[T1]{fontenc} % Required for accented characters
\usepackage{array}
\usepackage{bibentry}
\usepackage{prooftrees} 
\usepackage[round]{natbib} %%Or change 'round' to 'square' for square backers
\setcitestyle{aysep=}
% \usepackage{fitchproof} 

% \linespread{1} % Change line spacing here, Palatino benefits from a slight increase by default

\newcommand{\corner}[1]{\ulcorner#1\urcorner} %%Corner quotes
\newcommand{\tuple}[1]{\langle#1\rangle} %%Angle brackets
\newcommand{\set}[1]{\lbrace#1\rbrace} %%Set brackets
\newcommand{\interpret}[1]{\llbracket#1\rrbracket} %%Double brackets
%\DeclarePairedDelimiter\ceil{\lceil}{\rceil}    
\def\therefore{\ensuremath{\ldotp\dot\,\ldotp}}
\newcommand{\I}{\mathcal{I}}
\newcommand{\J}{\mathcal{J}}
\newcommand{\B}{\mathcal{B}}
\newcommand{\F}{\mathcal{F}}
\newcommand{\M}{\mathcal{M}}
\newcommand{\D}{\mathbb{D}}
\renewcommand{\v}[1]{\mathbf{#1}}
\newcommand{\even}{\texttt{Even}}
\newcommand{\comp}{\texttt{Comp}}
\newcommand{\res}{\texttt{Res}}
\newcommand{\simp}{\texttt{Simple}}
\newcommand{\leng}{\texttt{Length}}
\newcommand{\V}[1]{\mathcal{V}_{#1}} %%Corner quotes
\newcommand{\VV}[2]{\mathcal{V}_{#1}^{#2}} %%
\newcommand{\qt}[2]{#1 #2} % for modifying style of quantifer and bound variable pairs
\newcommand{\unisub}[2]{[#1/#2]}

\makeatletter
\renewcommand\@biblabel[1]{\textbf{#1.}} % Change the square brackets for each bibliography item from '[1]' to '1.'
\renewcommand{\@listI}{\itemsep=0pt} % Reduce the space between items in the itemize and enumerate environments and the bibliography

\renewcommand{\maketitle}{ % Customize the title - do not edit title and author name here, see the TITLE block below
\begin{flushright} % Right align
{\LARGE\@title} % Increase the font size of the title

\vspace{10pt} % Some vertical space between the title and author name

{\@author} % Author name
\\\@date % Date

\vspace{60pt} % Some vertical space between the author block and abstract
\end{flushright}
}

%----------------------------------------------------------------------------------------
%	TITLE
%----------------------------------------------------------------------------------------

\title{\textbf{Quantified Logic with Identity}} % Subtitle

\author{\textsc{Logic I}\\ \em Benjamin Brast-McKie} % Institution

\date{\today} % Date

%----------------------------------------------------------------------------------------

\begin{document}

\maketitle % Print the title section

\thispagestyle{empty}

%----------------------------------------------------------------------------------------




\section*{Uniqueness}

\begin{enumerate}
  \item[\it Uniqueness:] Ingmar trusts Albert, but no one else.
  \item[\it Only:] Regiment the following argument:
    \begin{itemize}
      \item[(1)] Lois Lane only loves Clark Kent.
      \item[(2)] Only Clark Kent is Superman.
      \item[$\therefore$] Lois Lane loves Superman.
    \end{itemize}
\end{enumerate}





\section*{Definite Descriptions}

\begin{enumerate}
  % \item[\it Russell:] Compare the following sentences:
  \item[\bf Question 1:] Regiment the following sentences.
    \begin{itemize}
      \item Socrates is guilty.
      \item Socrates is not guilty.
      \item Socrates is guilty or Socrates is not guilty.
    \end{itemize}
  \item[\bf Question 2:] Regiment the following sentences.
    \begin{itemize}
      \item The king of France is Bald.
      \item The king of France is not bald.
      \item The king of France is bald or the king of France is not bald.
    \end{itemize}
  \item[\bf Question 3:] Which of the sentences above are contingent?
  \item[\it Existence:] If the king of France is Bald, then the king of France exists.
  \item[\it Definite Article:] `The king of France' can't be a name.
  \item[\it Regimentation:] Russell offered the following analysis:
    \begin{itemize}
      \item $\exists x(Kxf \wedge \forall y(Kyf \supset x=y) \wedge Bx)$.
      \item $\exists x(\forall y(Kyf \equiv x=y) \wedge Bx)$.
    \end{itemize}
  \item[\it Negation:] Negation applies to the predicate, not the sentence.
  \item[\bf Task 1:] Regiment the following:
  \item Superman is keeping something from his lover.
  % \item The Queen of spades is a black card.
  \item The man with the axe is not a Jack.
  \item The Ace of diamonds is not the man with the axe.
  \item One-eyed jacks and the man with the axe are wild.
  \item No spy knows the combination to the safe.
  \item The one Ingmar trusts is lying.
  % \item Only Ingmar knows the combination to the safe.
  \item The person who knows the combination to the safe is not a spy.
\end{enumerate}




\section*{At Least:}

\begin{enumerate}
  \item[\bf Task 2:] Regiment the following claims.
  % \item There are no wild cards.
  \item There is at least one wild card.
  \item There are at least two clubs.
  \item There are at least three hearts on the table.
  \item[\bf Question 4:] How can we define these quantifiers in general?
\end{enumerate}





\section*{Substitution}

\begin{enumerate}
  \item[\it Free For:] $\beta$ is \textsc{free for} $\alpha$ in $\varphi$ just in case there is no free occurrence of $\alpha$ in $\varphi$ in the scope of a quantifier that binds $\beta$. 
  \item[\it Constants:] If $\beta$ is a constant, then $\beta$ is free for any $\alpha$ and $\varphi$. 
  \item[\it Substitution:] If $\beta$ is free for $\alpha$ in $\varphi$, then the \textsc{substitution} $\varphi\unisub{\beta}{\alpha}$ is the result of replacing all free occurrences of $\alpha$ in $\varphi$ with $\beta$. 
  \item[\it Examples:] Consider the following cases:
    \begin{enumerate}
      \item $z$ is free for $x$ in $\qt{\forall}{y}(Fxy \supset Fyx)$ 
      \item $y$ is not free for $x$ in $\qt{\forall}{y}(Fxy \supset Fyx)$
    \end{enumerate}
\end{enumerate}
   




\section*{Inequality Quantifiers Defined}

\begin{enumerate}
  \item[\it Definition:] We may define the following abbreviations recursively:
    \begin{itemize}
      \item[\it Base:] $\qt{\exists_{\geq 1}}{\alpha}\varphi \coloneq \qt{\exists}{\alpha}\varphi$.
      \item[\it Recursive:] $\qt{\exists_{\geq n+1}}{\alpha}\varphi \coloneq \qt{\exists}{\alpha}(\varphi \wedge \qt{\exists_{\geq n}}{\beta}(\alpha \neq \beta \wedge \varphi\unisub{\beta}{\alpha})).$
    \end{itemize}
  \item[\it Infinite:] $\Gamma_{\infty} \coloneq \set{\qt{\exists_{\geq n}}{x}(x=x): n\in\mathbb{N}}$.
  \item[\bf Question 5:] What is the smallest model to satisfy $\Gamma_\infty$?
  \item[\it At Most:] Regiment the following claims.
  \item There is at most one wild card.
  \item There are at most two one-eyed jacks.
  \item There are at most three black jacks.
  \item[\it Definition:] $\qt{\exists_{\leq n}}{\alpha}\varphi \coloneq \neg\qt{\exists_{\geq n+1}}{\alpha}\varphi$.
\end{enumerate}
 
 





\section*{Cardinality Quantifiers}
 
\begin{enumerate}
  \item[\bf Task 3:] Regiment the following.
  \item There is one wild card.
  \item There are three hearts on the table.
  \item If the deuce of clubs is wild, then there is exactly one wild card.
  \item[\bf Question 6:] How can we define the cardinality quantifiers in general?
  \item[\it Base:] $\qt{\exists_0}{\alpha}\varphi \coloneq \qt{\forall}{\alpha}\neg\varphi$.
  \item[\it Recursive:] $\qt{\exists_{n+1}}{\alpha}\varphi \coloneq \qt{\exists}{\alpha}(\varphi \wedge \qt{\exists_n}{\beta}(\alpha \neq \beta \wedge \varphi\unisub{\beta}{\alpha}))$.
  \item[\bf Question 7:] How do the cardinality quantifiers relate to the inequality quantifiers?
  \item[\it Between:] $\qt{\exists_{(n,m)}}{\alpha}\varphi \coloneq \qt{\exists_{\geq n}}{\alpha}\varphi \wedge \qt{\exists_{\leq m}}{\alpha}\varphi$~ where $n\leq m$.
  \item[\it Exact:] $\qt{\exists_{n}}{\alpha}\varphi \coloneq \qt{\exists_{(n,n)}}{\alpha}\varphi.$
\end{enumerate}





 % Buffy and Willow were born unto the same generation.
 % There is no more than one slayer born in each generation.
 % A slayer other than Buffy is one of the forces of darkness.
 % Willow will stand against any force of darkness other than a werewolf.
 % Faith will kick everyone except herself.
 % Buffy will kick anyone who stands against a slayer, unless they are also kicking vampires or demons.
 % In every generation a slayer is born.
 % In every generation a slayer is born. She will stand against the vampires, demons, and forces of darkness.
 % In every generation a slayer is born. She alone will stand against the vampires, demons, and forces of darkness.

% \item There are at least three horses in the world.
% \item There are at least three animals in the world.
% \item There is more than one horse in Farmer Brown's field.
% \item There are three horses in Farmer Brown's field.
% \item There is a single winged creature in Farmer Brown's field; any other creatures in the field must be wingless.
% \item The Pegasus is a winged horse.
% \item The animal in Farmer Brown's field is not a horse.
% \item The horse in Farmer Brown's field does not have wings.




\section*{Examples}

\begin{enumerate}
\item Show that $\set{{\neg}Raa, \qt{\forall}{x} (x{=}a \vee Rxa)}$ is satisfiable. 
% \item Show that $\set{\qt{\forall}{x}\qt{\forall}{y}\qt{\forall}{z}(x{=}y \vee y{=}z \vee x{=}z),\qt{\exists}{x}\qt{\exists}{y}\ x{\neq} y}$ is satisfiable.
\item Show that $\qt{\forall}{x}\qt{\forall}{y}\ x{=}y \vdash \neg \qt{\exists}{x}\ x \neq a$.
% \item Show that $\qt{\exists}{x} (x {=} h \wedge x {=} i)$ is contingent.
% \item Show that \{$\qt{\exists}{x}\qt{\exists}{y}(Zx \wedge Zy \wedge x{=}y)$, $\neg Zd$, $d{=}s$\} is satisfiable.
% \item Show that $\qt{\forall}{x}(Dx \supset \qt{\exists}{y} Tyx)\nvdash\qt{\exists}{y} \qt{\exists}{z}\ y{\neq} z$.
\end{enumerate}






\section*{Relations}

\begin{enumerate}
  \item[\bf Task 4:] Is the following argument valid? 
    \begin{itemize}
      \item[-] $\forall x\forall y(Rxy \supset Ryx)$.
      \item[-] $\forall x\forall y\forall z((Rxy \wedge Ryz) \supset Rxz)$.
      \item[$\therefore$] $\forall xRxx$.
    \end{itemize}
  \item[\bf Task 5:] Is the following argument valid?
    \begin{itemize}
      \item[-] $\forall x\forall y\forall z((Rxy \wedge Ryz) \supset Rxz)$.
      \item[-] $\forall x\neg Rxx$.
      \item[$\therefore$] $\forall x\forall y(Rxy \supset \neg Ryx)$.
    \end{itemize}
  % \item[\it Domain:] Let the \textit{domain} $D$ be any set.
  % \item[\it Relation:] A \textit{relation} $R$ on $D$ is any subset of $D^2$.
  % \item[\it Reflexive:] A relation $R$ is \textit{reflexive} on $D$ \textit{iff} $\tuple{x,x}\in R$ for all $x\in D$.
  % \item[\it Non-Reflexive:] A relation $R$ is \textit{non-reflexive} on $D$ \textit{iff} $R$ is not reflexive on $D$.
  % \item[\bf Question 1:] What is it to be \textit{irreflexive}?
  % \item[\it Irreflexive:] A relation $R$ is \textit{irreflexive} on $D$ \textit{iff} $\tuple{x,x}\notin R$ for all $x\in D$.
  % \item[\it Symmetric:] A relation $R$ is \textit{symmetric iff} $\tuple{y,x}\in R$ whenever ${x,y}\in R$.
  % \item[\bf Question 2:] Why don't we need to specify a domain?
  % \item[\bf Question 3:] Why is a relation reflexive or irreflexive with respect to a domain?
  % \item[\it Asymmetric:] A relation $R$ is \textit{asymmetric iff} $\tuple{y,x}\notin R$ whenever $\tuple{x,y}\in R$.
  % \item[\bf Question 4:] What is it to be non-symmetric? How about non-asymmetric?
  % \item[\bf Task 1:] Show that every asymmetric relation is irreflexive.
  % \item[\it Transitive:] A relation $R$ is \textit{transitive iff} $\tuple{x,z}\in R$ whenever $\tuple{x,y},\tuple{y,z}\in R$.
  % \item[\it Intransitive:] A relation $R$ is \textit{intransitive iff} $\tuple{x,z}\notin R$ whenever $\tuple{x,y},\tuple{y,z}\in R$.
  % \item[\bf Question 5:] Is every symmetric transitive relation reflexive? (No: $R=\varnothing$)
  % \item[\bf Task 2:] Show that every transitive irreflexive relation asymmetric?
  % \item[\it Euclidean:] A relation $R$ is \textit{euclidean iff} $\tuple{y,z}\in R$ whenever $\tuple{x,y},\tuple{x,z}\in R$.
  % \item[\bf Task 3:] Show that every transitive symmetric relation is euclidean.
\end{enumerate}



% \section*{Assignment Lemmas}
%
% \begin{enumerate}
%   \item[\it Lemma 1:] If $\hat{a}(\alpha)=\hat{c}(\alpha)$ for all free variables $\alpha$ in a wff $\varphi$, then $\VV{\I}{\hat{a}}(\varphi)=\VV{\I}{\hat{c}}(\varphi)$.
%     \begin{itemize}
%       \item[\it Base:] Assume $\comp(\varphi)=0$, so $\varphi=(\alpha=\beta)$ or $\varphi=\F^n\alpha_1,\ldots,\alpha_n$.
%       \item[($\alpha=\beta$):] \mbox{So $\VV{\I}{\hat{a}}(\varphi)=\VV{\I}{\hat{a}}(\alpha=\beta)=1$ \textit{iff} $\VV{\I}{\hat{a}}(\alpha)=\VV{\I}{\hat{a}}(\beta)$ \textit{iff} $\VV{\I}{\hat{c}}(\alpha)=\VV{\I}{\hat{c}}(\beta)\ldots$}
%       \item[($\F^n\alpha_1,\ldots,\alpha_n$):] \mbox{So $\VV{\I}{\hat{a}}(\varphi)=\VV{\I}{\hat{a}}(\F^n\alpha_1,\ldots,\alpha_n)=1$ \textit{iff} $\tuple{\VV{\I}{\hat{a}}(\alpha_1),\ldots,\VV{\I}{\hat{a}}(\alpha_n)}\in\I(F^n)\ldots$}
%     \end{itemize}
%   \item[\it Lemma 2:] For any sentence $\varphi$: $\VV{\I}{}(\varphi)= 1$ \textit{iff} $\VV{\I}{\hat{a}}(\varphi)= 1$ for every v.a. $\hat{a}$ over $\D$.
%     % \begin{itemize}
%     %   \item[\it LTR:] Assume $\VV{\I}{}(\varphi)= 1$, so $\VV{\I}{\hat{a}}(\varphi)= 1$ for some v.a. $\hat{c}$ over $\D$ .
%     %   \item Let $\hat{a}$ be any v.a. over $\D$.
%     %   \item Since $\varphi$ has no free variables, $\VV{\I}{\hat{a}}(\varphi)=\VV{\I}{\hat{c}}(\varphi)$ by \textit{Lemma 1}.
%     %   \item So $\VV{\I}{\hat{a}}(\varphi)=1$ for all v.a. $\hat{c}$ over $\D$.
%     %   \item[\it RTL:] Assume $\VV{\I}{\hat{a}}(\varphi)=1$ for all v.a. $\hat{a}$ over $\D$.
%     %   \item Since $\D$ is nonempty, there is some v.a. $\hat{a}$, and so $\VV{\I}{}(\varphi)= 1$. 
%     %   % \item So $\VV{\I}{}(\varphi)=1$.
%     % \end{itemize}
%   \item[\it Lemma 3:] For any sentence $\varphi$: $\VV{\I}{}(\varphi)\neq1$ \textit{iff} $\VV{\I}{\hat{a}}(\varphi)\neq 1$ for some v.a. $\hat{a}$ over $\D$.
% \end{enumerate}
%





\end{document}

