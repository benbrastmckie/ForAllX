\documentclass[a4paper, 11pt]{article} % Font size (can be 10pt, 11pt or 12pt) and paper size (remove a4paper for US letter paper)

\usepackage[protrusion=true,expansion=true]{microtype} % Better typography
\usepackage{graphicx} % Required for including pictures
\usepackage{wrapfig} % Allows in-line images
\usepackage{enumitem} %%Enables control over enumerate and itemize environments
\usepackage{setspace}
\usepackage{amssymb, amsmath, mathrsfs} %%Math packages
\usepackage{stmaryrd}
\usepackage{mathtools}
\usepackage{mathpazo} % Use the Palatino font
\usepackage[T1]{fontenc} % Required for accented characters
\usepackage{array}
\usepackage{bibentry}
\usepackage[round]{natbib} %%Or change 'round' to 'square' for square backers
\setcitestyle{aysep={}}

\linespread{1.05} % Change line spacing here, Palatino benefits from a slight increase by default

\newcommand{\corner}[1]{\ulcorner#1\urcorner} %%Corner quotes
\newcommand{\tuple}[1]{\langle#1\rangle} %%Angle brackets
\newcommand{\set}[1]{\lbrace#1\rbrace} %%Set brackets
\newcommand{\interpret}[1]{\llbracket#1\rrbracket} %%Double brackets
%\DeclarePairedDelimiter\ceil{\lceil}{\rceil}    
\def\therefore{\ensuremath{\ldotp\dot{}\,\ldotp}}

\makeatletter
\renewcommand\@biblabel[1]{\textbf{#1.}} % Change the square brackets for each bibliography item from '[1]' to '1.'
\renewcommand{\@listI}{\itemsep=0pt} % Reduce the space between items in the itemize and enumerate environments and the bibliography

\renewcommand{\maketitle}{ % Customize the title - do not edit title and author name here, see the TITLE block below
\begin{flushright} % Right align
{\LARGE\@title} % Increase the font size of the title

\vspace{10pt} % Some vertical space between the title and author name

{\@author} % Author name
\\\@date % Date

\vspace{30pt} % Some vertical space between the author block and abstract
\end{flushright}
}

%----------------------------------------------------------------------------------------
%	TITLE
%----------------------------------------------------------------------------------------

\title{\textbf{What is Logic?}} % Subtitle

\author{\textsc{Logic I}\\ \em Benjamin Brast-McKie} % Institution

\date{\today} % Date

%----------------------------------------------------------------------------------------

\begin{document}

\maketitle % Print the title section

\thispagestyle{empty}

%----------------------------------------------------------------------------------------

\section*{Definitions}

\begin{enumerate}[leftmargin=1.5in,labelsep=.15in] %,label=(\arabic*)]%,label=\roman*]
  \item[\it Proposition:] A \textsc{proposition} is a way for things to be which is either true or false.
  \item[\it Declarative Sentence:]  A \textsc{declarative sentence} is a grammatical string of symbols which, on an interpretation, expresses a proposition that is either true or false.
  \item[\it Argument:] An \textsc{Argument} is a finite sequence of declarative sentences where the final sentence is the \textsc{conclusion} and the preceding sentences are the \textsc{premises}.
\end{enumerate}



\section*{Examples}

\subsection*{\it \textbf{Snow}}

\begin{enumerate}
  \item[(1)] It's snowing.
  \item[\therefore] John drove to work.
\end{enumerate}

\noindent
\textit{This argument may be compelling, but not certain.}

\subsection*{\it \textbf{Red}}

\begin{enumerate}
  \item[(1)] The ball is crimson.
  \item[\therefore] The ball is red.
\end{enumerate}

\noindent
\textit{This argument provides certainty, but not on all interpretations.}

\subsection*{\it \textbf{Museum}}

\begin{enumerate}
  \item[(1)] Kate is either at home or at the Museum.
  \item[(2)] Kate is not at home.
  \item[\therefore] Kate is at the Museum.
\end{enumerate}

\noindent
\textit{This argument's certainty is independent of the interpretation.}


\section*{Informal Validity}

\begin{enumerate}[leftmargin=1.2in,labelsep=.15in] %,label=(\arabic*)]%,label=\roman*]
  \item[\bf Task 1:] Account for the logical strength of \textit{Museum}.
  \item[\it Atomic Sentences:] A declarative sentence is \textsc{atomic} just in case it is not composed of declarative sentences as proper parts.
  \item[\bf Task 2:] Identify atomic sentences in \textit{Museum}.
  \item[\it Informal Interpretation:] Let an \textsc{informal interpretation} assign every atomic sentence of English to exactly one \textsc{truth-value} 1 or 0.
  \item[\it Informal Validity:] An argument is \textsc{informally valid} iff its conclusion is true in every interpretation in which its premises are true.
  \item[\bf Task 3:] Assign truth-values to non-atomic sentences as a function of their parts.
  \item[\it Logical Constants:] Complex sentences are composed from atomic sentences \textit{via} the sentential operators `and', `or', `not', `if', `iff', etc.
  \item[\bf Task 4:] Provide semantic clauses for `not' and `or'.
  \item[\it Atomic:] If $\varphi$ is atomic, then the truth-value of $\varphi$ is given by the interpretation. 
  \item[\it Disjunction:] If $\varphi=\psi\vee\chi$, then $\varphi$ is true iff $\psi$ or $\chi$ is true (or both). 
  \item[\it Negation:] If $\varphi=\neg\psi$, then $\varphi$ is true iff $\psi$ is false. 
  \item[\bf Task 5:] Show that \textit{Museum} is informally valid.
\end{enumerate}





\section*{Formal Languages}

\begin{enumerate}[leftmargin=1.2in,labelsep=.15in] %,label=(\arabic*)]%,label=\roman*]
  \item[\bf Problem 1:] There is no set of all atomic sentences of English, and so no clear notion of an informal interpretation of English.
  \item[\it Suggestion:] Could choose some large set of atomic English sentences, but this would be arbitrary and hard to specify.
  \item[\bf Solution:] We will regiment English arguments in a formal language that is easy to specify precisely.
  \item[\it Sentential Logic:] The \textsc{sentences} of SL are composed of sentence letters $A, B, C, \ldots$ and sentential operators $\neg,\vee,\wedge,\supset,$ and $\equiv$.
  \item[\it Interpretation:] An \textsc{interpretation} of the sentences of SL assigns exactly one truth-value (1 or 0) to each sentence letter.
  \item[\bf Problem 2:] SL cannot regiment all logically valid arguments.
  \item[\it Socrates:] All men are mortal, Socrates is a man $\vdash$ Socrates is mortal. 
  \item[\bf Solution:] This is reason to extend the language, e.g. QL.
\end{enumerate}




\section*{Logical Form}

\subsection*{\it \textbf{Museum}}

\begin{enumerate}
  \item[(1)] The painting is 
  \item[(2)] Kate is not at home.
  \item[\therefore] Kate is at the Museum.
\end{enumerate}

\begin{enumerate}[leftmargin=1.2in,labelsep=.15in] %,label=(\arabic*)]%,label=\roman*]
  \item[\bf Task 1:] Regiment \textit{Socrates} example in SL: $A, B \vdash C$.
  \item[\bf Task 2:] Expand the language
\end{enumerate}



\section*{Further Notions}

\begin{enumerate}[leftmargin=1.2in,labelsep=.15in] %,label=(\arabic*)]%,label=\roman*]
  \item[\it Soundness:] 
  \item[\it Logical Truth:]
  \item[\it Contradiction:]
  \item[\it Logical Entailment:]
  \item[\it Logical Equivalence:]
  \item[\it Consistency:]
  \item[\it Inconsistency:]
\end{enumerate}



% \vfill
%
% \bibliographystyle{Phil_Review} %%bib style found in bst folder, in bibtex folder, in texmf folder.
% \bibliography{Zotero} %%bib database found in bib folder, in bibtex folder


\end{document}
