\documentclass[a4paper, 11pt]{article} % Font size (can be 10pt, 11pt or 12pt) and paper size (remove a4paper for US letter paper)
\usepackage[protrusion=true,expansion=true]{microtype} % Better typography
\usepackage{graphicx} % Required for including pictures
\usepackage{wrapfig} % Allows in-line images
\usepackage{enumitem} %%Enables control over enumerate and itemize environments
\usepackage{setspace}
\usepackage{amssymb, amsmath, mathrsfs} %%Math packages
\usepackage{stmaryrd}
\usepackage{mathtools}
\usepackage{mathpazo} % Use the Palatino font
\usepackage[T1]{fontenc} % Required for accented characters
\usepackage{array}
\usepackage{bibentry}
\usepackage[round]{natbib} %%Or change 'round' to 'square' for square backers
\setcitestyle{aysep={}}

% \linespread{1} % Change line spacing here, Palatino benefits from a slight increase by default

\newcommand{\corner}[1]{\ulcorner#1\urcorner} %%Corner quotes
\newcommand{\tuple}[1]{\langle#1\rangle} %%Angle brackets
\newcommand{\set}[1]{\lbrace#1\rbrace} %%Set brackets
\newcommand{\interpret}[1]{\llbracket#1\rrbracket} %%Double brackets
%\DeclarePairedDelimiter\ceil{\lceil}{\rceil}    
\def\therefore{\ensuremath{\ldotp\dot{}\,\ldotp}}
\newcommand{\I}{\mathcal{I}}

\makeatletter
\renewcommand\@biblabel[1]{\textbf{#1.}} % Change the square brackets for each bibliography item from '[1]' to '1.'
\renewcommand{\@listI}{\itemsep=0pt} % Reduce the space between items in the itemize and enumerate environments and the bibliography

\renewcommand{\maketitle}{ % Customize the title - do not edit title and author name here, see the TITLE block below
\begin{flushright} % Right align
{\LARGE\@title} % Increase the font size of the title

\vspace{10pt} % Some vertical space between the title and author name

{\@author} % Author name
\\\@date % Date

\vspace{30pt} % Some vertical space between the author block and abstract
\end{flushright}
}

%----------------------------------------------------------------------------------------
%	TITLE
%----------------------------------------------------------------------------------------

\title{\textbf{What is Logic?}} % Subtitle

\author{\textsc{Logic I}\\ \em Benjamin Brast-McKie} % Institution

\date{\today} % Date

%----------------------------------------------------------------------------------------

\begin{document}

\maketitle % Print the title section

\thispagestyle{empty}

%----------------------------------------------------------------------------------------

\section*{Definitions}

\begin{enumerate}[leftmargin=1.5in,labelsep=.15in] %,label=(\arabic*)]%,label=\roman*]
  \item[\it Proposition:] A \textsc{proposition} is a way for things to be which is either true or false.
  \item[\it Declarative Sentence:]  A \textsc{declarative sentence} is a grammatical string of symbols which, on an interpretation, expresses a proposition that is either true or false.
  \item[\it Argument:] An \textsc{Argument} is a finite sequence of declarative sentences where the final sentence is the \textsc{conclusion} and the preceding sentences are the \textsc{premises}.
\end{enumerate}



\section*{Examples}

\subsection*{\it \textbf{Snow}}

\begin{enumerate}
  \item[(1)] It's snowing.
  \item[\therefore] John drove to work.
\end{enumerate}

\noindent
\textit{This argument may be compelling, but not certain.}

\subsection*{\it \textbf{Red}}

\begin{enumerate}
  \item[(1)] The ball is crimson.
  \item[\therefore] The ball is red.
\end{enumerate}

\noindent
\textit{This argument provides certainty, but not on all interpretations.}

\subsection*{\it \textbf{Museum}}

\begin{enumerate}
  \item[(1)] Kate is either at home or at the Museum.
  \item[(2)] Kate is not at home.
  \item[\therefore] Kate is at the Museum.
\end{enumerate}

\noindent
\textit{This argument's certainty is independent of the interpretation.}


\section*{Informal Validity}

\begin{enumerate}[leftmargin=1.2in,labelsep=.15in] %,label=(\arabic*)]%,label=\roman*]
  \item[\bf Question 1:] What goes wrong if we assume the premises but deny the conclusion in \textit{Red} and \textit{Museum}?
  \item[\it Answer:] Nature of `crimson' and `red' \textit{vs.} meaning of `or' and `not'.
  % \item[\it Variance:] Allow the interpretation of non-logical terms to vary, holding the meaning of the logical terms constant.
  \item[\it Informal Semantics:] Give informal semantics for disjunction and negation.
  \item[\it Complex Sentences:] Observe that complex sentences in \textit{Museum} are composed from simpler sentences \textit{via} `not' and `or'.
  \item[\it Atomic Sentences:] A declarative sentence is \textsc{atomic} just in case it is not composed of simpler declarative sentences.
  \item[\bf Task 1:] Identify atomic sentences in \textit{Museum}.
  \item[\it Informal Interpretation:] Let an \textsc{informal interpretation} assign every atomic sentence of English to exactly one \textsc{truth-value} 1 or 0.
  \item[\it Informal Validity:] An argument in English is \textsc{informally valid} just in case its conclusion is true in every informal interpretation in which its premises are true.
  \item[\bf Task 2:] Use semantics to show that \textit{Museum} is informally valid.
\end{enumerate}





\section*{Formal Languages}

\begin{enumerate}[leftmargin=1.2in,labelsep=.15in] %,label=(\arabic*)]%,label=\roman*]
  \item[\bf Problem 1:] There is no set of all atomic sentences of English, and so no clear notion of an informal interpretation of English.
  \item[\it Suggestion:] Could choose some large set of atomic English sentences, but this would be arbitrary and hard to specify.
  \item[\bf Solution 1:] We will \textit{regiment} English arguments in a formal language that is both general and easy to specify precisely.
  \item[\it Sentential Logic:] The \textsc{sentences} of SL are composed of sentence letters $A, B, C, \ldots$ and sentential operators $\neg,\vee,\wedge,\supset,$ and $\equiv$.
  \item[\bf Task 3:] Regiment \textit{Museum} in SL: $A\vee B, \neg A \vDash B$.
  \item[\bf Task 4:] Provide semantic clauses for SL.
  % \item[\it Atomic:] If $\varphi$ is atomic, then the truth-value of $\varphi$ is given by the interpretation. 
  \item[\it Interpretation:] An \textsc{interpretation} $\I$ of the sentences of SL assigns exactly one truth-value (1 or 0) to each sentence letter.
  \item[\it Disjunction:] $\I(\varphi\vee\psi)=1$ just in case $\I(\varphi)=1$ or $\I(\psi)=1$ (or both).
  \item[\it Negation:] $\I(\neg\varphi)=1$ just in case $\I(\varphi)=0$.
  \item[\it Logical Validity:] An argument in SL is \textsc{logically valid} just in case its conclusion is true in every interpretation in which its premises are true.
  \item[\bf Task 5:] Show that \textit{Museum} is logically valid.
\end{enumerate}




\section*{Logical Form}

\subsection*{\it \textbf{Picasso}}

\begin{enumerate}
  \item[(1)] The painting is either a Picasso or a counterfeit and illegally traded.
  \item[(2)] The painting is not a Picasso.
  \item[\therefore] The painting is a counterfeit and illegally traded.
\end{enumerate}

\noindent
\textit{This argument is also logically valid.}

\begin{enumerate}[leftmargin=1.2in,labelsep=.15in] %,label=(\arabic*)]%,label=\roman*]
  \item[\bf Question 2:] How does this argument relate to \textit{Museum}? 
  \item[\bf Task 6:] Regiment \textit{Picasso} in SL: $A\vee (B\wedge C), \neg A \vDash B\wedge C$.
  \item[\it Logical Form:] Both arguments are instances of $\varphi \vee \psi, \neg\varphi \vDash \psi$ which is a logically valid argument form. 
  \item[\bf Question 3:] How many logically valid argument forms are there, and how could we hope to describe this space?
  \item[\it Suggestion:] Logical validity in SL describes the space of logically valid arguments, where the logically valid argument forms are patterns in this space.
  \item[\bf Problem 2:] SL cannot regiment all logically valid arguments.
  \item[\it Socrates:] All men are mortal, Socrates is a man $\vDash$ Socrates is mortal. 
  \item[\bf Solution 2:] Rather, logical validity in SL provides a partial answer, where we may extend the language to provide a broader description of logical validity, e.g. QL.
\end{enumerate}




\section*{Proof Theory}

\begin{enumerate}[leftmargin=1.2in,labelsep=.15in] %,label=(\arabic*)]%,label=\roman*]
  \item[\it Model Theory:] We have characterized logical reasoning in terms of truth-preservation across a space of interpretations of the formal language by providing elements of a model theoretic semantics for SL.
  \item[\bf Task 7:] Can we make \textit{Snow} and \textit{Red} logically valid?
  \item[\it Syntactic Account:] Another approach focuses entirely on syntax, using rules to specify which inferences are deductively valid given the meanings of the logical constants.
  \item[\it Metalogic:] Amazingly, these two strategies coincide for both SL and QL, and we will prove these important results later in this course.
  \item[\it Neutrality:] These methods accommodate reasoning about anything whatsoever, though not all logical constants are equally well understood.
\end{enumerate}


% \section*{Further Notions}
%
% \begin{enumerate}[leftmargin=1.2in,labelsep=.15in] %,label=(\arabic*)]%,label=\roman*]
%   \item[\it Soundness:] An argument is \textit{sound} just in case it is both logically valid and has true premises.
%   \item[\it Truth:] Soundness reaches beyond the scope of any logic course since truth on a single interpretation requires subject-specific knowledge.
%   \item[\it Logical Truth:] Instead we can talk about sentences of formal languages being \textit{logically true} just in case they are true on all interpretations.
%   \item[\it Contradiction:] A sentence is a \textit{contradiction} (or \textit{logically false}) just in case it is false on all interpretations.
%   \item[\it Logical Entailment:] One sentence \textit{logically entails} another just in case every interpretation in which the former is true also makes the latter true.
%   \item[\it Logical Equivalence:] One sentence is \textit{logically equivalent} to another just in case they logically entail each other.
%   \item[\it Consistency:] A set of sentences is \textit{consistent} just in case there is an interpretation which makes every sentence in the set true, and \textit{inconsistent} otherwise.
% \end{enumerate}
%
%
%
% \section*{Examples}
%
% \noindent
% Which sets of sentences are consistent? (e.g., is $\set{(1),(2)}$ consistent?)
%
% \subsection*{\it \textbf{Taller}}
%
% \begin{enumerate}
%   \item[(1)] Liza is taller than Sue.
%   \item[(2)] Sue is taller than Paul.
%   \item[(3)] Paul is taller than Liza.
% \end{enumerate}
%
%
%
%
% \subsection*{\it \textbf{Lost}}
%
% \begin{enumerate}
%   \item[(4)] Kim is either in Somerville or Cambridge.
%   \item[(5)] If Kim is in Somerville, then she is not far from home.
%   \item[(6)] If Kim is not far from home, then she is in Cambridge.
%   \item[(7)] Kim is not in Cambridge.
% \end{enumerate}
%






% \vfill
%
% \bibliographystyle{Phil_Review} %%bib style found in bst folder, in bibtex folder, in texmf folder.
% \bibliography{Zotero} %%bib database found in bib folder, in bibtex folder


\end{document}
