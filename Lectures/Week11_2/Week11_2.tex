\documentclass[a4paper, 11pt]{article} % Font size (can be 10pt, 11pt or 12pt) and paper size (remove a4paper for US letter paper)
\usepackage[protrusion=true,expansion=true]{microtype} % Better typography
\usepackage{graphicx} % Required for including pictures
\usepackage{wrapfig} % Allows in-line images
\usepackage{enumitem} %%Enables control over enumerate and itemize environments
\usepackage{setspace}
\usepackage{amssymb, amsmath, mathrsfs, mathabx} %%Math packages
\usepackage{../lecture} %calls local modified style file
\usepackage{stmaryrd}
\usepackage{mathtools}
\usepackage{multicol} 
\usepackage{mathpazo} % Use the Palatino font
\usepackage[T1]{fontenc} % Required for accented characters
\usepackage{array}
\usepackage{bibentry}
\usepackage{prooftrees} 
\usepackage[round]{natbib} %%Or change 'round' to 'square' for square backers
\setcitestyle{aysep=}
% \usepackage{fitchproof} 


\makeatletter
\renewcommand{\maketitle}{
\begin{flushright}
{\LARGE\@title}

\vspace{10pt}

{\@author}
\\ \@date
\end{flushright}

\vspace{-20pt}

}
\makeatother

%----------------------------------------------------------------------------------------
%	TITLE
%----------------------------------------------------------------------------------------

\title{\textbf{Soundness: Part II}} % Subtitle

\author{\textsc{Logic I}\\ \em Benjamin Brast-McKie} % Institution

\date{\today} % Date

%----------------------------------------------------------------------------------------

\begin{document}

\maketitle % Print the title section

\thispagestyle{empty}

%----------------------------------------------------------------------------------------


% \section*{Substitution}
%
% \begin{enumerate}
%   \item[\it Free For:] $\beta$ is \textsc{free for} $\alpha$ in $\varphi$ just in case there is no free occurrence of $\alpha$ in $\varphi$ in the scope of a quantifier that binds $\beta$. 
%   \item[\it Substitution:] If $\beta$ is free for $\alpha$ in $\varphi$, then the \textsc{substitution} $\varphi\unisub{\beta}{\alpha}$ is the result of replacing all free occurrences of $\alpha$ in $\varphi$ with $\beta$.
% \end{enumerate}
   






\section*{FOL$^=$ Rules}

\begin{enumerate}
  \item[($\forall$E)] $\forall\alpha\varphi \vdash \varphi\unisub{\beta}{\alpha}$ where $\beta$ is a constant and $\alpha$ is a variable. 
  \item[($\exists$I)] $\varphi\unisub{\beta}{\alpha} \vdash \exists\alpha\varphi$ where $\beta$ is a constant and $\alpha$ is a variable.
  \item[($\forall$I)] $\varphi\unisub{\beta}{\alpha} \vdash \forall\alpha\varphi$ where $\beta$ is a constant, $\alpha$ is a variable, and $\beta$ does not occur in $\forall\alpha\varphi$ or in any undischarged assumption.
  \item[($\exists$E)] If $\exists\alpha\varphi,\varphi\unisub{\beta}{\alpha} \vdash \psi$ where $\beta$ is a constant that does not occur in $\exists\alpha\varphi$, $\psi$, or in any undischarged assumption, then $\exists\alpha\varphi\vdash \psi$.
  \item[($=$I)] $\vdash \alpha = \alpha$ for any constant $\alpha$. 
  \item[($=$E)] $\varphi\unisub{\alpha}{\gamma},\alpha=\beta\vdash\varphi\unisub{\beta}{\gamma}$.
\end{enumerate}








\section*{Lemmas From Last Time\ldots}%
  \label{sec:Lemmas}
  
\begin{itemize}
  \item[\bf L2.1] If $\MetaG \vDash \metaA$, then $\MetaG \cup \MetaS \vDash \metaA$.
  \item[\bf L4.3] For any FOL$^=$ proof $X$, if $\metaA_k$ is live at line $n$ where $k\leq n$, then $\MetaG_k\subseteq \MetaG_{n}$.
  \item[\bf L9.1] $\VV{\I}{\va{a}}(\metaA)=\VV{\I}{\va{c}}(\metaA)$ if $\va{a}(\alpha)=\va{c}(\alpha)$ for all free variables $\alpha$ in a wff $\metaA$.
  \item[\bf L9.2] $\VV{\I}{}(\metaA)= 1$ just in case $\VV{\I}{\va{a}}(\metaA)= 1$ for some v.a. $\va{a}$ over $\D$.
  \item[\bf L11.2] If $\MetaG \vDash \metaA$ and $\MetaG \vDash \enot\metaA$, then $\MetaG$ is unsatisfiable.
  \item[\bf L11.3] $\MetaG \cup \set{\metaA}$ is unsatisfiable just in case $\MetaG \vDash \enot\metaA$.
  % \item[\bf L11.4] If $\MetaG \cup \set{\metaA} \vDash \metaB$, then $\MetaG \vDash \metaA \eif \metaB$.
  \item[\bf L11.5] $\VV{\I}{\va{a}}(\metaA)=\VV{\I}{\va{a}}(\metaA\unisub{\beta}{\alpha})$ if $\val{\I}{\va{a}}(\alpha)=\val{\I}{\va{a}}(\beta)$ and $\beta$ is free for $\alpha$ in $\metaA$.
  \item[\bf L11.6] If $\M=\tuple{\D,\I}$ and $\M'=\tuple{\D,\I'}$ where $\I$ and $\I'$ agree about every constant $\alpha$ and $n$-place predicate $\F^n$ that occurs in $\metaA$, it follows that $\VV{\I}{\va{a}}(\metaA)=\VV{\I'}{\va{a}}(\metaA)$ for any variable assignment $\va{a}$ over $\D$.
\end{itemize}




\section*{Soundness of FOL$^=$}

\begin{enumerate}
  \item[\it Assume:] $\Gamma \vdash \varphi$, so there is a FOL$^=$ proof $X$ of $\varphi$ from $\Gamma$. 
  \item[\it Lines:] Let $\metaA_i$ be the wfs on line $i$ of $X$ and $\Gamma_i$ be the premises and undischarged assumptions at line $i$. 
  \item[\it Proof:] The proof goes by induction on length of $X$:
    \begin{itemize}
      \item[\bf L11.1:] (Base Step)~~ $\Gamma_1 \vDash \varphi_i$. 
      \item[\bf L11.13:] (Induction Step)~~ If $\Gamma_k \vDash \varphi_k$ for all $k\leq n$, then $\Gamma_{n+1} \vDash \varphi_{n+1}$. 
    \end{itemize}
  \item[\it Finite:] Since $X$ is finite, there is some $m$ where $\varphi_m=\varphi$ and $\Gamma_m=\Gamma$, so $\Gamma \vDash \varphi$.
\end{enumerate}




\section*{Induction Step}

\begin{itemize}
  \item[\it Assume:] $\Gamma_k \vDash \varphi_k$ for all $k\leq n$ where $\metaA_{n+1}$ follows by a FOL$^=$ rule. 
  \item Want to show $\Gamma_{n+1}\vDash\varphi_{n+1}$ by each rule.
\end{itemize}





\section*{Universal Rules}

\begin{itemize}
  \item[\bf L11.7] For a constant $\beta$ not in $\qt{\forall}{\alpha}\metaA$ or $\metaB\in\MetaG$, if $\MetaG \vDash \metaA\unisub{\beta}{\alpha}$, then $\MetaG \vDash \qt{\forall}{\alpha}\metaA$.
      \item Assume $\MetaG \vDash \metaA\unisub{\beta}{\alpha}$ for constant $\beta$ not in $\qt{\forall}{\alpha}\metaA$ or $\MetaG$.
      \item Let $\M = \tuple{\D, \I}$ where $\VV{\I}{}(\metaC) = 1$ for all $\metaC\in\MetaG$, and let $\va{c}$ be arbitrary.
      \item So $\VV{\I}{\va{a}}(\metaC) = 1$ for all $\metaC\in\MetaG$ for all v.a. $\va{a}$, and so for $\va{c}$ in particular.
      \item Let $\M'$ be like $\M$ but for $\I'(\beta)=\va{c}(\alpha)$, and so $\val{\I'}{\va{c}}(\alpha)=\val{\I'}{\va{c}}(\beta)$.
      \item Since $\beta$ does not occur in $\MetaG$, we know $\VV{\I'}{\va{c}}(\metaC) = 1$ for all $\metaC \in \MetaG$ by \textbf{L11.6}.
      \item So $\VV{\I'}{}(\metaC) = 1$ for all $\metaC\in\MetaG$ by \textbf{L9.2}, so $\VV{\I'}{}(\metaA\unisub{\beta}{\alpha}) = 1$ by assumption.
      \item So $\VV{\I'}{\va{a}}(\metaA\unisub{\beta}{\alpha})=1$ for all $\va{a}$, and so $\VV{\I'}{\va{c}}(\metaA\unisub{\beta}{\alpha})=1$ for $\va{c}$ in particular.
      \item So $\VV{\I'}{\va{c}}(\metaA) = 1$ by \textbf{L11.5} given $\val{\I'}{\va{c}}(\alpha) = \val{\I'}{\va{c}}(\beta)$ above. 
      \item Since $\beta$ is not in $\qt{\forall}{\alpha}\metaA$, we know $\beta$ is not in $\metaA$, so $\VV{\I}{\va{c}}(\metaA)=1$ by \textbf{L11.6}.
      \item Since $\va{c}$ was arbitrary, $\VV{\I}{\va{a}}(\qt{\forall}{\alpha}\metaA)=1$ for any $\va{a}$, so $\VV{\I}{}(\qt{\forall}{\alpha}\metaA)=1$.
      % \item By generalizing on $\M$, we may conclude that $\MetaG \vDash \qt{\forall}{\alpha}\metaA$.
  \item[\bf L11.8] $\forall\alpha\metaA \vDash \metaA\unisub{\beta}{\alpha}$ where $\alpha$ is a variable and $\metaA\unisub{\beta}{\alpha}$ is a wfs of $\FI$.
    \item Let $\M = \tuple{\D, \I}$ where $\VV{\I}{}(\qt{\forall}{\alpha}\metaA) = 1$, so $\VV{\I}{\va{a}}(\qt{\forall}{\alpha}\metaA)=1$ for arbitrary $\va{a}$. 
    \item So $\VV{\I}{\va{c}}(\metaA)=1$ for an $\alpha$-variant $\va{c}$ of $\va{a}$ where $\va{c}(\alpha)=\I(\beta)$.
    \item Since $\val{\I}{\va{c}}(\alpha)=\val{\I}{\va{c}}(\beta)$, we know $\VV{\I}{\va{c}}(\metaA\unisub{\beta}{\alpha})=1$ by \textbf{L11.5}.
    \item Thus $\VV{\I}{}(\metaA\unisub{\beta}{\alpha})=1$ by \textbf{L9.2}, so $\forall\alpha\metaA \vDash \metaA\unisub{\beta}{\alpha}$ generalizing on $\M$.
  \item[\bf L11.9] If $\MetaG \vDash \metaA$ and $\Sigma \cup \set{\metaA} \vDash \metaB$, then $\MetaG\cup\Sigma \vDash \metaB$.
  \item[($\forall$I)] $\metaA_i=\metaA\unisub{\beta}{\alpha}$ for $i\leq n$ live at $n+1$ where $\beta$ is not in $\metaA_{n+1}$ or $\MetaG_{n+1}$.
    \item So $\MetaG_i\vDash\metaA_i$ by hypothesis, and $\MetaG_i\subseteq\MetaG_{n+1}$ by \textbf{L4.3}.
    \item Thus $\MetaG_{n+1}\vDash\metaA_i$ by \textbf{L2.1}, so $\MetaG_{n+1}\vDash\metaA\unisub{\beta}{\alpha}$.
    \item So $\MetaG_{n+1}\vDash \qt{\forall}{\alpha}\metaA$ by \textbf{L11.7} since $\beta$ not in $\qt{\forall}{\alpha}\metaA$ or $\MetaG_{n+1}$.
    \item Equivalently, $\MetaG_{n+1}\vDash \metaA_{n+1}$.
  \item[($\forall$E)] $\metaA_i=\qt{\forall}{\alpha}\metaA$ for $i\leq n$ live at $n+1$ where $\metaA_{n+1}=\metaA\unisub{\beta}{\alpha}$.
    \item So $\MetaG_i\vDash\metaA_i$ by hypothesis, and $\MetaG_i\subseteq\MetaG_{n+1}$ by \textbf{L4.3}.
    \item Thus $\MetaG_{n+1}\vDash\metaA_i$ by \textbf{L2.1}, so $\MetaG_{n+1}\vDash \qt{\forall}{\alpha}\metaA$.
    \item By \textbf{L11.8} $\qt{\forall}{\alpha}\metaA\vDash\metaA\unisub{\beta}{\alpha}$, and so $\MetaG_{n+1}\vDash\metaA\unisub{\beta}{\alpha}$ by \textbf{L11.9}.
    \item Equivalently, $\MetaG_{n+1}\vDash \metaA_{n+1}$.
\end{itemize}




\section*{Existential and Identity Lemmas}

\begin{itemize}
  \item[\bf L11.10] $\metaA\unisub{\beta}{\alpha} \vDash \exists\alpha\metaA$ where $\alpha$ is a variable and $\metaA\unisub{\beta}{\alpha}$ is a wfs of $\FI$.
    \item Let $\M=\tuple{\D,\I}$ where $\VV{\I}{}(\metaA\unisub{\beta}{\alpha})=1$, so $\VV{\I}{\va{a}}(\metaA\unisub{\beta}{\alpha})=1$ for all $\va{a}$.
    \item Let $\va{c}$ be s.t. $\va{c}(\alpha)=\I(\beta)$, so $\val{\I}{\va{c}}(\alpha)=\val{\I}{\va{c}}(\beta)$ where $\beta$ is free for $\alpha$ in $\metaA$.
    \item So $\VV{\I}{\va{c}}(\metaA\unisub{\beta}{\alpha})=\VV{\I}{\va{c}}(\metaA)$ by \textbf{L11.5}.
    \item Since $\va{c}$ is an $\alpha$-variant of itself, $\VV{\I}{\va{c}}(\qt{\exists}{\alpha}\metaA)=1$ by semantics.
    \item Since $\metaA\unisub{\beta}{\alpha}$ is a wfs $\FI$, at most $\alpha$ is free in $\metaA$, and so $\qt{\exists}{\alpha}\metaA$ is a wfs. 
    \item Hence $\VV{\I}{}(\qt{\exists}{\alpha}\metaA)=1$ by \textbf{L9.2}, and so $\metaA\unisub{\beta}{\alpha} \vDash \exists\alpha\metaA$.
  \item[\bf L11.11] For any constant $\beta$ that does not occur in $\exists\alpha\metaA$, $\metaB$, or in any sentence $\metaC\in\Gamma$, if $\Gamma \vDash \exists\alpha\metaA$ and $\Gamma \cup \set{\metaA\unisub{\beta}{\alpha}} \vDash \metaB$, then $\Gamma \vDash \metaB$.
    \item Let $\Gamma \vDash \exists\alpha\metaA$ and $\Gamma \cup \set{\metaA\unisub{\beta}{\alpha}} \vDash \metaB$ for a constant $\beta$ not in $\exists\alpha\metaA$, $\metaB$, or $\Gamma$. 
    \item Let $\M=\tuple{\D,\I}$ be a model where $\VV{\I}{}(\metaC) = 1$ for all $\metaC \in \Gamma$.
    \item So $\VV{\I}{}(\qt{\exists}{\alpha}\metaA) = 1$, and so $\VV{\I}{\va{a}}(\qt{\exists}{\alpha}\metaA)=1$ for some v.a. $\va{a}$ by \textbf{L9.2}.
    \item So $\VV{\I}{\va{c}}(\metaA)=1$ for a $\alpha$-variant $\va{c}$ of $\va{a}$ by the semantics.
    \item Let $\M'$ be like $\M$ but for $\I'(\beta)=\va{c}(\alpha)$, and so $\val{\I'}{\va{c}}(\beta)=\val{\I'}{\va{c}}(\alpha)$.
    \item Since $\beta$ is not in $\qt{\exists}{\alpha}\metaA$, it's not in $\metaA$, and so $\VV{\I}{\va{c}}(\metaA)=\VV{\I'}{\va{c}}(\metaA)$ by \textbf{L11.6}.
    \item Since $\val{\I'}{\va{c}}(\beta)=\val{\I'}{\va{c}}(\alpha)$, we know $\VV{\I'}{\va{c}}(\metaA)=\VV{\I'}{\va{c}}(\metaA\unisub{\beta}{\alpha})$ by \textbf{L11.5}. 
    \item Thus $\VV{\I'}{\va{c}}(\metaA\unisub{\beta}{\alpha})=1$, and so $\VV{\I'}{}(\metaA\unisub{\beta}{\alpha})=1$ by \textbf{L9.2}.
    \item Given any $\va{e}$, we know $\VV{\I}{}(\metaC) = 1$ and so $\VV{\I}{\va{e}}(\metaC) = 1$ for any $\metaC \in \MetaG$.
    \item Given the assumptions about $\beta$, by \textbf{L11.6} $\VV{\I'}{\va{e}}(\metaC) = 1$, so $\VV{\I'}{}(\metaC) = 1$.
    \item Generalizing on $\metaC$, it follows that $\VV{\I'}{}(\metaC) = 1$ for all $\metaC \in \Gamma$.
    \item Hence $\VV{\I'}{}(\metaC) = 1$ for all $\metaC \in \Gamma \cup \set{\metaA\unisub{\beta}{\alpha}}$.
    \item So $\VV{\I'}{}(\metaB) = 1$ by assumption, so $\VV{\I'}{\va{g}}(\metaB)=1$ for some $\va{g}$ by \textbf{L9.2}.
    \item Since $\beta$ does not occur in $\metaB$, $\VV{\I}{\va{g}}(\metaB)=1$ for all $\va{g}$ by \textbf{L11.6}.
    \item Thus $\VV{\I}{}(\metaB)=1$ by \textbf{L9.2}, so $\Gamma \vDash \metaB$ by generalizing on $\M$.
  \item[\bf L11.12] If $\alpha$ and $\beta$ are constants, then $\metaA\unisub{\alpha}{\gamma}, \alpha = \beta \vDash \metaA\unisub{\beta}{\gamma}$.
    \item Let $\M=\tuple{\D,\I}$ where $\VV{\I}{}(\metaA\unisub{\alpha}{\gamma}) = \VV{\I}{}(\alpha = \beta) = 1$. 
    \item By \textbf{L9.2}, $\VV{\I}{\va{a}}(\metaA\unisub{\alpha}{\gamma})=1$ for some $\va{a}$, where $\VV{\I}{\va{c}}(\alpha=\beta)=1$ for all $\va{c}$.
    \item So $\VV{\I}{\va{a}}(\alpha=\beta)=1$ in particular, and so $\val{\I}{\va{a}}(\alpha)=\val{\I}{\va{a}}(\beta)$ by semantics.
    \item Since $\beta$ is a constant, $\VV{\I}{\va{a}}(\metaA\unisub{\alpha}{\gamma}) = \VV{\I}{\va{a}}((\metaA\unisub{\alpha}{\gamma})\unisub{\beta}{\alpha})$ by \textbf{L11.5}.
    \item However, $(\metaA\unisub{\alpha}{\gamma})\unisub{\beta}{\alpha}=\metaA\unisub{\beta}{\gamma}$, and so $\VV{\I}{\va{a}}(\metaA\unisub{\beta}{\gamma})=1$.
    \item Since $\metaA\unisub{\beta}{\gamma}$ is a wfs of $\FI$, we know $\VV{\I}{}(\metaA\unisub{\beta}{\gamma})=1$. 
    \item By generalizing on $\M$ we may conclude that $\metaA\unisub{\alpha}{\gamma}, \alpha = \beta \vDash \metaA\unisub{\beta}{\gamma}$. 
\end{itemize}




\section*{Existential and Identity Rules}%
  \label{sec:ExistentialRules}
  
\begin{itemize}
  \item[($\exists$I)] $\Gamma_{n+1} \vDash \metaA_{n+1}$ if $\metaA_{n+1}$ follows from $\Gamma_{n+1}$ by the rule $\exists$I. 
    \item Assume that $\metaA_{n+1}$ follows from $\Gamma_{n+1}$ by existential introduction $\exists$I.
    \item Thus $\metaA_i=\metaA\unisub{\beta}{\alpha}$ is live at $n+1$ for some $i\leq n$ where $\metaA_{n+1}=\qt{\exists}{\alpha}\metaA$.
    \item By \textbf{L4.3}, we know $\Gamma_i\subseteq \Gamma_{n+1}$ where $\Gamma_i\vDash\metaA_i$ by hypotheses.
    \item So $\Gamma_{n+1} \vDash \metaA_i$ by \textbf{L2.1}, and so equivalently, $\Gamma_{n+1} \vDash \metaA\unisub{\beta}{\alpha}$.
    \item Since $\metaA\unisub{\beta}{\alpha} \vDash \qt{\exists}{\alpha}\metaA$ by \textbf{L11.10}, we know $\Gamma_{n+1}\vDash\metaA_{n+1}$ by \textbf{L11.9}.
  \item[($\exists$E)] $\Gamma_{n+1} \vDash \metaA_{n+1}$ if $\metaA_{n+1}$ follows from $\Gamma_{n+1}$ by the rule $\exists$E.
    \item Assume that $\metaA_{n+1}$ follows from $\Gamma_{n+1}$ by existential elimination $\exists$E.
    \item For some $i<j<k\leq n$ where $\metaA_i=\qt{\exists}{\alpha}\metaA$ is live at $n+1$, $\metaA_j=\metaA\unisub{\beta}{\alpha}$ for some constant $\beta$ that does not occur in $\metaA_i$, $\metaA_k$, or any $\metaB\in\Gamma_i$.
      \begin{proof}
        \have[i]{i}{\qt{\exists}{\alpha}\metaA}
        \open	
          \hypo[j]{j}{\metaA\unisub{\beta}{\alpha}} \as{for $\exists$E}
          \have[ ]{x}{\vdots} 
          \have[k]{k}{\metaB}
        \close
        \have[n+1]{n}{\metaB{}} \Ee{i,j-k}
      \end{proof}
    \item By hypothesis, $\Gamma_i\vDash\metaA_i$ and $\Gamma_k\vDash\metaA_k$ where $\Gamma_i\subseteq \Gamma_{n+1}$ by \textbf{L4.3}.
    \item With the exception of $\metaA_j$, every assumption that is undischarged at line $k$ is also undischarged at line $n+1$, and so $\Gamma_k\subseteq\Gamma_{n+1}\cup\set{\metaA_j}$.
    \item By \textbf{L2.1} we know $\Gamma_{n+1} \vDash \metaA_i$ and $\Gamma_{n+1}\cup\set{\metaA_j} \vDash \metaA_k$.
    \item Equivalently, $\Gamma_{n+1} \vDash \qt{\exists}{\alpha}\metaA$ and $\Gamma_{n+1}\cup\set{\metaA\unisub{\beta}{\alpha}} \vDash \metaB$.
    \item Thus $\Gamma_{n+1}\vDash\metaB$ by \textbf{L11.11}, and so $\Gamma_{n+1}\vDash\metaA_{n+1}$.
  \item[($=$E)] $\Gamma_{n+1} \vDash \metaA_{n+1}$ if $\metaA_{n+1}$ follows from $\Gamma_{n+1}$ by the rule $=$E. 
    \item Assume that $\metaA_{n+1}$ follows from $\Gamma_{n+1}$ by existential elimination $=$E.
    \item By parity, $\metaA_i$ is $\alpha=\beta$, $\metaA_j=\metaA\unisub{\alpha}{\gamma}$, and $\metaA_{n+1}=\metaA\unisub{\beta}{\gamma}$.
    \item By \textbf{L4.3}, $\Gamma_i,\Gamma_j\subseteq \Gamma_{n+1}$ where $\Gamma_i\vDash\metaA_i$ and $\Gamma_j\vDash\metaA_j$ by hypotheses.
    \item So $\Gamma_{n+1} \vDash \metaA_i$ and $\Gamma_{n+1} \vDash \metaA_j$ by \textbf{L2.1}.
    \item Equivalently, $\Gamma_{n+1} \vDash \alpha = \beta$ and $\Gamma_{n+1} \vDash \metaA\unisub{\alpha}{\gamma}$.
    \item So $\metaA\unisub{\alpha}{\gamma}, \alpha = \beta \vDash \metaA\unisub{\beta}{\gamma}$ by \textbf{L11.12} since $\alpha$ and $\beta$ are constants.
    \item By two applications of \textbf{L11.9}, $\Gamma_{n+1}\vDash\metaA\unisub{\beta}{\gamma}$, so $\Gamma_{n+1}\vDash\metaA_{n+1}$.
\end{itemize}






% \section*{PL Rules}
%
% \begin{itemize}
%   \item[($\enot$E)] There is a proof of $\metaB$ at line $h$ and $\enot\metaB$ at line $j$ from $\enot \metaA$ on line $i$.
%     %$\MetaG_{n+1} \vDash \metaA_{n+1}$ if $\metaA_{n+1}$ follows from $\MetaG_{n+1}$ by the rule $\enot$I.
%     \item By hypothesis $\MetaG_h\vDash \metaB$ and $\MetaG_j\vDash\enot\metaB$, where $\MetaG_h,\MetaG_j\subseteq\MetaG_{n+1}\cup\set{\enot \metaA_i}$.
%     \item By \textbf{L2.1}, $\MetaG_{n+1}\cup\set{\enot \metaA_i}\vDash\metaB$ and $\MetaG_{n+1}\cup\set{\enot \metaA_i}\vDash\enot\metaB$.
%     \item So $\MetaG_{n+1}\cup\set{\enot \metaA_i}$ is unsatisfiable by \textbf{L11.2}
%     \item So $\MetaG_{n+1}\vDash \enot\enot\metaA_{n+1}$ follows by \textbf{L11.3}, and then appeal to the semantics.
% \end{itemize}
%
%
%
%
%
% \section*{Existential Elimination}
%
% \begin{enumerate}
%   \item[\bf Task 1:] Regiment and derive the following in FOL$^=$.
%   % \item $\qt{\forall}{x}, Rma\vdash \qt{\exists}{x} Rxx$
%   % \item $\qt{\forall}{x}(x{=}n \equiv Mx), \qt{\forall}{x}(Ox \vee \neg Mx)\vdash On$
%   \item The elephant would not obey.\\
%     \underline{Patrick is an elephant.}\\ 
%     Patrick would not obey.
%   \item $\qt{\forall}{x}(Jx \supset Kx)$\\
%     $\qt{\exists}{x}\qt{\forall}{y} Lxy$\\
%     \underline{$\qt{\forall}{x} Jx$}\\
%     $\qt{\exists}{x}(Kx \wedge Lxx)$.
%   \item \underline{$\qt{\exists}{x}(Px \supset \qt{\forall}{x}Qx)$}\\
%     $\qt{\forall}{x}Px \supset \qt{\forall}{x}Qx$.
%   \item \underline{$\qt{\exists}{x}Px \vee \qt{\exists}{x}Qx$}\\
%     $\qt{\exists}{x}(Px \vee Qx)$.
%   % \item $\qt{\exists}{x}(Kx \wedge \qt{\forall}{y}(Ky \supset x{=}y) \wedge Bx), Kd\vdash Bd$
%   % \item $\vdash Pa \supset \qt{\forall}{x}(Px \vee x {\neq} a)$
%   \item Every nonempty asymmetric relation is non-symmetric.
% \end{enumerate}
%
%
%
%
%
%
% \section*{Further Problems: Relations}
%
% \begin{enumerate}
%   \item[\bf Task 1:] Regiment and derive the following in FOL$^=$.
%   \item Every transitive and symmetric relation is quasi-reflexive.
%   \item Only the empty relation is symmetric and asymmetric.
%   \item Every intransitive relation is irreflexive.
%   \item Every intransitive relation is asymmetric.
% \end{enumerate}
%







\end{document}

