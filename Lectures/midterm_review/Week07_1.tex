\documentclass[a4paper, 11pt]{article} % Font size (can be 10pt, 11pt or 12pt) and paper size (remove a4paper for US letter paper)
\usepackage[protrusion=true,expansion=true]{microtype} % Better typography
\usepackage{graphicx} % Required for including pictures
\usepackage{wrapfig} % Allows in-line images
\usepackage{enumitem} %%Enables control over enumerate and itemize environments
\usepackage{setspace}
\usepackage{amssymb, amsmath, mathrsfs} %%Math packages
\usepackage{stmaryrd}
\usepackage{mathtools}
\usepackage{multicol} 
\usepackage{mathpazo} % Use the Palatino font
\usepackage[T1]{fontenc} % Required for accented characters
\usepackage{array}
\usepackage{bibentry}
\usepackage{prooftrees} 
\usepackage[round]{natbib} %%Or change 'round' to 'square' for square backers
\setcitestyle{aysep={}}
% \usepackage{fitchproof} 

% \linespread{1} % Change line spacing here, Palatino benefits from a slight increase by default

\newcommand{\corner}[1]{\ulcorner#1\urcorner} %%Corner quotes
\newcommand{\tuple}[1]{\langle#1\rangle} %%Angle brackets
\newcommand{\set}[1]{\lbrace#1\rbrace} %%Set brackets
\newcommand{\interpret}[1]{\llbracket#1\rrbracket} %%Double brackets
%\DeclarePairedDelimiter\ceil{\lceil}{\rceil}    
\def\therefore{\ensuremath{\ldotp\dot{}\,\ldotp}}
\newcommand{\I}{\mathcal{I}}
\newcommand{\J}{\mathcal{J}}
\newcommand{\B}{\mathcal{B}}
\newcommand{\even}{\texttt{Even}}
\newcommand{\comp}{\texttt{Comp}}
\newcommand{\res}{\texttt{Res}}
\newcommand{\simp}{\texttt{Simple}}
\newcommand{\leng}{\texttt{Length}}
\newcommand{\V}[1]{\mathcal{V}_{#1}} %%Corner quotes

\makeatletter
\renewcommand\@biblabel[1]{\textbf{#1.}} % Change the square brackets for each bibliography item from '[1]' to '1.'
\renewcommand{\@listI}{\itemsep=0pt} % Reduce the space between items in the itemize and enumerate environments and the bibliography

\renewcommand{\maketitle}{ % Customize the title - do not edit title and author name here, see the TITLE block below
\begin{flushright} % Right align
{\LARGE\@title} % Increase the font size of the title

\vspace{10pt} % Some vertical space between the title and author name

{\@author} % Author name
\\\@date % Date

\vspace{30pt} % Some vertical space between the author block and abstract
\end{flushright}
}

%----------------------------------------------------------------------------------------
%	TITLE
%----------------------------------------------------------------------------------------

\title{\textbf{Midterm Review}} % Subtitle

\author{\textsc{Logic I}\\ \em Benjamin Brast-McKie} % Institution

\date{\today} % Date

%----------------------------------------------------------------------------------------

\begin{document}

\maketitle % Print the title section

\thispagestyle{empty}

%----------------------------------------------------------------------------------------

\section*{Derivable Schemata}

\begin{enumerate}[leftmargin=1.5in]
  \item[\it Contraposition:] $\varphi \supset \psi\ \vdash \neg\psi \supset \neg\varphi$.
  \item[\it Hypothetical Syllogism:] $\varphi \supset \psi,\ \psi \supset \chi\ \vdash \varphi \supset \chi$.
  \item[\it Disjunctive Syllogism:] $\varphi \vee \psi,\ \neg \varphi \vdash \psi$.
  \item[\it $\vee$-Conditional:] $\varphi \supset \psi\ \dashv\vdash \neg\varphi \vee \psi$.
  \item[\it $\neg$-Conditional:] $\neg(\varphi \supset \psi)\ \dashv\vdash \varphi \wedge \neg \psi$.
  \item[\it Conditional Weakening:] $\psi\ \vdash \varphi \supset \psi$.
  \item[\it Double Negation:] $\neg\neg\varphi\ \dashv\vdash \varphi$.
  \item[\it $\wedge$-De Morgan's:] $\neg(\varphi\wedge\psi)\dashv\vdash\neg\varphi\vee\neg\psi$.
  \item[\it $\vee$-De Morgan's:] $\neg(\varphi\vee\psi)\dashv\vdash\neg\varphi\wedge\neg\psi$.
  \item[\it Modus Tollens:] $\varphi \supset \psi,\ \neg\psi\ \vdash \neg\varphi$.
  % \item[\it Ex Falso Quodlibet:] $\varphi,\ \neg\varphi\ \vdash \psi$.
  % \item[\it Law of Excluded Middle:] $\vdash \varphi\vee\neg\varphi$.
  % \item[\it Law of Non-Contradiction:] $\vdash \neg(\varphi\wedge\neg\varphi)$.
  % \item[\it Dilemma:] $\varphi \vee \psi,\ \varphi \supset \chi,\ \psi \supset \chi\ \vdash \chi$.
  % \item[\it $\vee$-Commutativity:] $\varphi \vee \psi\ \vdash \psi \vee \varphi$.
  % \item[\it $\wedge$-Commutativity:] $\varphi \wedge \psi\ \vdash \psi \wedge \varphi$.
  % \item[\it Biconditional MP:] $\varphi \equiv \psi,\ \neg\varphi\ \vdash \neg\psi$.
  % \item[\it $\equiv$-Commutativity:] $\varphi \equiv \psi\ \vdash \psi \equiv \varphi$.
  % \item[\it ${\vee}{\wedge}$-Distribution:] $\varphi\vee(\psi\wedge\chi) \dashv\vdash (\varphi\vee\psi)\wedge(\varphi\vee\chi)$.
  % \item[\it ${\wedge}{\vee}$-Distribution:] $\varphi\wedge(\psi\vee\chi) \dashv\vdash (\varphi\wedge\psi)\vee(\varphi\wedge\chi)$.
  % \item[\it ${\vee}{\wedge}$-Absorption:] $\varphi\vee(\varphi\wedge\psi) \dashv\vdash \varphi$.
  % \item[\it ${\wedge}{\vee}$-Absorption:] $\varphi\wedge(\varphi\vee\psi) \dashv\vdash \varphi$.
  % \item[\it $\wedge$-Associativity:] $\varphi\wedge(\psi\wedge\chi) \dashv\vdash (\varphi\wedge\psi)\wedge\chi$.
  % \item[\it $\vee$-Associativity:] $\varphi\vee(\psi\vee\chi) \dashv\vdash (\varphi\vee\psi)\vee\chi$.
\end{enumerate}


\section*{Regimentation}

\noindent
Complete the following tasks for arguments (A) and (B):

\begin{enumerate}
  \item[\bf Task 1:] Write a symbolization key and regiment the argument.
  \item[\bf Task 2:] Determine if the argument is valid.
  \item[\bf Task 3:] Provide a derivation in PL if valid, and a countermodel otherwise.
  \bigskip
  \item[(A)] If Dorothy plays the piano in the morning, then Roger wakes up cranky.
    Dorothy plays piano in the morning unless she is distracted.
    So if Roger does not wake up cranky, then Dorothy must be distracted.
  % \item[(B)] If the fair coin had been flipped, it would have either landed heads or tails.
    % But it's not true that if it had been flipped, it would have landed heads.
    % Therefore if it had been flipped, it would have landed tails.
  \item[(B)] If Cam remembered to do his chores, then things are clean but not neat.
    Cam forgot only if things are neat but not clean.
    Therefore, things are clean just in case they are not neat.
% \item You can fool some people sometimes, but you can't fool all the people all the time.
\end{enumerate}

\noindent


% \section*{Logical Analysis}
%
% \begin{enumerate}
%     \item[\bf Task 1:] How can we tell if a sentence is valid?
%   \begin{multicols}{2}
%     \item[\it Tautology?] \quad If YES, prove $\vdash\varphi$.\hfill
%     \item[\it Contradiction?] \quad If YES, prove $\vdash\neg\varphi$.\hfill
%     \item[\it Contingent?] \quad If YES, provide a models.\hfill
%     \item[\it Equivalent?] \quad If YES, prove $\varphi\dashv\vdash\psi$.\hfill
%     \item[] If NO, provide a countermodel.
%     \item[] If NO, provide a model.
%     \item[] If NO, prove $\vdash\varphi$ or $\vdash\neg\varphi$. 
%     \item[] If NO, provide a countermodel.
%   \end{multicols}
% \end{enumerate}

% \section*{Lemmas}
%
% \begin{enumerate}
%   \item[\tt Lemma 1:] Every satisfiable branch $\B$ in an SL tree $X$ is open. 
%   \item[\tt Lemma 2:] If $X$ is an SL tree with a satisfiable branch $\B$, then any tree $X'$ which is the result of resolving a sentence in $\B$ has a satisfiable branch $\B'$.  
%   \item[\tt Lemma 3:] Every SL tree with a satisfiable root has a satisfiable branch.
%   \item[\tt Lemma 4:] Every SL tree $X$ has a finite number of branches.  
%   \item[\tt Lemma 5:] For any SL tree $X$ with root $\Gamma$ and $\varphi\in[X]$, there is an SL tree $Y$ with root $\Gamma$ where $\res(Y)<\res(X)$. 
%   \item[\tt Lemma 6:] For any tree $X$ with root $\Gamma$, there is a complete tree $X'$ with root $\Gamma$. 
%   \item[\tt Lemma 7:] Every complete open branch in an SL tree is satisfiable.
% \end{enumerate}
%
%
%
%
%
% \section*{Soundness}
%
% \begin{enumerate}
%   \item Assume $\Gamma$ is satisfiable. 
%   \item Let $X$ be an SL tree with root $\Gamma$. 
%   \item So $X$ has a satisfiable branch $\B$ by \textit{Lemma 3}. 
%   \item So $\B$ is open by \textit{Lemma 1}. 
%   \item So $X$ is not closed. 
%   \item More generally, there is no closed SL tree with root $\Gamma$. 
%   \item By contraposition, QED.
% \end{enumerate}
%
%
%
%
% \section*{Completeness}
%
% \begin{enumerate}
%   \item Assume there is no closed tree with root $\Gamma$.
%   \item Roots are trees, and so $\Gamma$ has a complete tree $X$ by \texttt{Lemma 6}.
%   \item So $X$ is a complete open tree with a complete open branch $\B$.
%   \item By \textit{Lemma 7}, $\B$ is satisfiable, and so $\Gamma$ is satisfiable. 
%   \item By contraposition, if $\Gamma \vDash \bot$, then $\Gamma \vdash \bot$.
% \end{enumerate}












\end{document}

