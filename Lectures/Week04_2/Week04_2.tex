\documentclass[a4paper, 11pt]{article} % Font size (can be 10pt, 11pt or 12pt) and paper size (remove a4paper for US letter paper)
\usepackage[protrusion=true,expansion=true]{microtype} % Better typography
\usepackage{../lecture} %calls local modified style file
\usepackage{graphicx} % Required for including pictures
\usepackage{wrapfig} % Allows in-line images
\usepackage{enumitem} %%Enables control over enumerate and itemize environments
\usepackage{setspace}
\usepackage{amssymb, amsmath, mathrsfs} %%Math packages
\usepackage{stmaryrd}
\usepackage{mathtools}
\usepackage{multicol} 
\usepackage{mathpazo} % Use the Palatino font
\usepackage[T1]{fontenc} % Required for accented characters
\usepackage{array}
\usepackage{bibentry}
\usepackage{prooftrees} 
\usepackage[round]{natbib} %%Or change 'round' to 'square' for square backers
\setcitestyle{aysep={}}

\makeatletter
\renewcommand{\maketitle}{
\begin{flushright}
{\LARGE\@title}

\vspace{10pt}

{\@author}
\\ \@date
\end{flushright}

\vspace{0pt}

}
\makeatother

%----------------------------------------------------------------------------------------
%	TITLE
%----------------------------------------------------------------------------------------

\title{\textbf{PL Soundness}} % Subtitle

\author{\textsc{Logic I}\\ \em Benjamin Brast-McKie} % Institution

\date{\today} % Date

%----------------------------------------------------------------------------------------

\begin{document}

\maketitle % Print the title section

\thispagestyle{empty}

%----------------------------------------------------------------------------------------

\section*{Continuing from Last Time...}

\begin{enumerate}
  \item[\it Contradiction:] If $\MetaG \vDash \metaA$ and $\MetaG \vDash \enot\metaA$, then $\MetaG$ is unsatisfiable.
    \item Assume $\MetaG \vDash \metaA$ and $\MetaG \vDash \enot\metaA$.
    \item Assume for contradiction that $\MetaG$ is satisfiable. 
    \item There is some $\PL$ interpretation $\I$ where $\V{\I}(\metaG) = 1$ for all $\metaG \in \MetaG$.
    \item By assumption, $\V{\I}(\metaA) = 1$ and $\V{\I}(\enot \metaA) = 1$.
    \item By the semantics for negation, $\V{\I}(\metaA) \neq 1$, contradicting the above.
    \item Thus $\MetaG$ is unsatisfiable. 
  \item[\it Unsatisfiable:] If $\MetaG \cup \set{\metaA}$ is unsatisfiable, then $\MetaG \vDash \enot\metaA$.
    \item Assume $\MetaG \cup \set{\metaA}$ is unsatisfiable.
    \item Let $\I$ be an arbitrary $\PL$ interpretation where $\V{\I}(\metaG) = 1$ for all $\metaG \in \MetaG$. 
    \item Assume for contradiction that $\V{\I}(\enot \metaA) = 0$.
    \item So $\V{\I}(\metaA) = 1$, and so $\MetaG \cup \set{\metaA}$ is satisfiable contrary to assumption.
    \item Thus for any $\I$, $\V{\I}(\enot \metaA) = 1$ if $\V{\I}(\metaG) = 1$ for all $\metaG \in \MetaG$.
    \item By definition, $\MetaG \vDash \enot\metaA$.
  \item[\bf Task 1:] Let $\I^+(\alpha)=1$ for every sentence letter $\alpha$ in SL. Show that $\V{\I^+}(\varphi)=1$ for every SL sentence $\varphi$ that does not include negation. 
  \item[\bf Task 2:] Every tree has a finite number of branches.
  \item[\bf Task 3:] Show that for every SL sentence $\varphi$, if $\simp(\varphi)$, then there are SL interpretations $\I$ and $\J$ where $\V{\I}(\varphi)=1$ and $\V{\J}(\varphi)=0$. 
  \item[\bf Task 4:] For any SL sentences $\varphi,\psi,\chi$ and SL sentence letter $\alpha$, if $\vDash \varphi \equiv \psi$, then $\vDash \chi_{[\varphi/\alpha]}\equiv\chi_{[\psi/\alpha]}$.
  \item[\bf Task 5:] Every tree can be completed in a finite number of steps.
\end{enumerate}




\section*{Recursive Definitions}

\begin{itemize}
  \item[\it Length:] For any SL tree $X$, we define $\leng(X)$ to be the number of resolution rules that have been applied.
      \item $\leng(X)=0$ for any root $X$.
      \item For any SL tree $X$, if $\leng(X)=n$ and $X'$ is the result of resolving a sentence in exactly one branch in $X$, then $\leng(X')=n+1$.
  \item[\it Constituents:] We define $[\varphi]$ to be the set of sentence letters that occur in $\varphi$. % by taking $[\cdot]:\text{SL}\to\mathcal{P}(\text{SL})$ to be the smallest function to satisfy:
      \item If $\comp(\varphi)=0$, then $[\varphi]=\set{\varphi}$.
      \item For any SL sentences $\varphi$ and $\psi$, and binary connective $\star\in\set{\wedge,\vee,\supset,\equiv}$: 
    \begin{itemize}
      \item[$(\neg)$] $[\neg \varphi]=[\varphi]$; and
      \item[$(\hspace{1pt}\star\hspace{1pt})$] $[\varphi \star \psi]=[\varphi] \cup [\psi]$.
    \end{itemize}
  \item[\it Simplicity:] We define $\simp(\varphi)$ to hold just in case the SL sentence $\varphi$ has at most one occurrence of each sentence letter in SL.
      \item If $\comp(\varphi)=0$, then $\simp(\varphi)$.
      \item For any SL sentences $\varphi$ and $\psi$, and binary connective $\star\in\set{\wedge,\vee,\supset,\equiv}$: 
    \begin{itemize}
      \item[$(\neg)$] $\simp(\neg \varphi)$ if $\simp(\varphi)$; and
      \item[$(\hspace{1pt}\star\hspace{1pt})$] $\simp(\varphi \star \psi)$ if $\simp(\varphi)$, $\simp(\psi)$, and $[\varphi]\cap[\psi]=\varnothing$.
    \end{itemize}
  \item[\it Substitution:] We define $\varphi_{[\chi/\alpha]}$ to be the result of replacing every occurrence of the sentence letter $\alpha$ in $\varphi$ with $\chi$.
      \item If $\comp(\varphi)=0$, then $\varphi_{[\chi/\alpha]}=
        \begin{cases}
          \chi \quad\text{if } \varphi=\alpha,\\
          \varphi \quad\text{otherwise.}
        \end{cases}$
      \item For any SL sentences $\varphi$ and $\psi$, and binary connective $\star\in\set{\wedge,\vee,\supset,\equiv}$: 
    \begin{itemize}
      \item[$(\neg)$] $(\neg\varphi)_{[\chi/\alpha]}=\neg(\varphi_{[\chi/\alpha]})$; and
      \item[$(\hspace{1pt}\star\hspace{1pt})$] $(\varphi\star\psi)_{[\chi/\alpha]}=\varphi_{[\chi/\alpha]}\star\psi_{[\chi/\alpha]}$.
    \end{itemize}
  \item[\it Resolution:] We define $\res(\varphi)$ to be the maximum number of times that we may resolve $\varphi$ and any of its descendants. 
      \item $\res(\varphi)=0$ if $\varphi$ is a literal.
      \item For any SL sentences $\varphi$ and $\psi$, and binary connective $\star\in\set{\wedge,\vee,\supset,\equiv}$: 
        \begin{itemize}
          \item[$(\neg)$] $\res(\neg\neg\varphi)=\res(\varphi)+1$; and
          \item[$(\hspace{1pt}\star\hspace{1pt})$] $\res(\varphi \star \psi)=\res(\varphi)+\res(\psi)+1$.
          \item[$(\hspace{1pt}\star\hspace{1pt})$] $\res(\neg(\varphi \star \psi))=\res(\neg\varphi)+\res(\neg\psi)+1$.
        \end{itemize}
  \item[\it Set Binary:] We extend the definition of $\res$ to sets of sentences as follows:
    $$\res(\Gamma)=\sum\limits_{\varphi\in\Gamma}\res(\varphi).$$
  \item[\it Unresolved:] We define $[X]$ to be the set of unresolved sentences in the SL tree $X$. 
      \item $\varphi\in[X]$ if $X$ is a root and $\varphi$ occurs in $X$.
      \item For any SL tree $X'$ which results from resolving $\varphi\in[X]$ on every open branch in $X$ in which $\varphi$ occurs:
        \begin{itemize}
          \item[$(\neg)$] $[X']=([X]/\set{\varphi})\cup\set{\psi}$ if $\varphi=\neg\neg\psi$;
          \item[$(+)$] $[X']=([X]/\set{\varphi})\cup\set{\psi,\chi}$ if $\varphi\in\set{\psi\wedge\chi,\psi\vee\chi}$;  
          \item[$(-)$] $[X']=([X]/\set{\varphi})\cup\set{\neg\psi,\neg\chi}$ if $\varphi\in\set{\neg(\psi\wedge\chi),\neg(\psi\vee\chi)}$;
          \item[$(\supset)$] $[X']=([X]/\set{\varphi})\cup\set{\neg\psi,\chi}$ if $\varphi=\psi\supset\chi$; 
          \item[$(\not\supset)$] $[X']=([X]/\set{\varphi})\cup\set{\psi,\neg\chi}$ if $\varphi=\neg(\psi\supset\chi)$; 
          \item[$(\equiv)$] $[X']=([X]/\set{\varphi})\cup\set{\psi,\chi, \neg\psi,\neg\chi}$ if $\varphi\in\set{\psi\equiv\chi,\neg(\psi\equiv\chi)}$.
        \end{itemize}
  \item[\it Resolvable:] Letting $\mathbb{L}$ be the set of SL literals, we define $X_U=[X]/\mathbb{L}$ to be the set of SL sentences in $X$ that can be resolved. 
\end{itemize}


\section*{Task 5}

\textit{Proof:}
Given any SL tree $X$, let $X_U$ be the set of resolvable sentences in $X$.
The proof goes by induction on $\res(X_U)$ for any SL tree $X$. 

\textit{Base Case:}
Let $X$ be an SL tree where $\res(X_U)=0$.
By definition, every branch is complete, and so the tree is complete.
Accordingly, $X$ can be completed in a finite number of steps, namely $0$. 

\textit{Hypothesis:}
Assume that for every SL tree $X$, if $\res(X_U)\leq n$, then $X$ can be completed in a finite number of steps. 

\textit{Induction:}
Let $X$ be an SL tree where $\res(X_U)=n+1$.
Thus there is some $\varphi \in X_U$.
Letting $X'$ be the SL tree that results from resolving $\varphi$ on every open branch in $X$, we may observe that $\varphi \notin X'_U$.
Consider the following cases where $\star\in\set{\wedge,\vee,\supset,\equiv}$ is a binary connective:

\textit{Case 1:} 
If $\varphi=\neg\neg\psi$ and $\res(\psi)\neq 0$, then $X'_U=(X_U/\set{\varphi})\cup\set{\psi}$.
If instead $\res(\psi)=0$, then $X'_U=X_U/\set{\varphi}$.
Since $\res(\psi)=\res(\varphi)-1$, it follows either way that $\res(X'_U) \leq \res(X_U)-1=n$.

\textit{Case 2:} 
If $\varphi=\psi\star\chi$ and $\res(\psi)\neq 0$ and $\res(\chi)\neq 0$, then we know that $X'_U=(X_U/\set{\varphi})\cup\set{\psi,\chi}$.
Since $\res(\psi)+\res(\chi)=\res(\varphi)-1$, it follows that $\res(X'_U)=\res(X_U)-1=n$.
If instead $\res(\psi)=0$ or $\res(\chi)=0$, then $\res(X'_U)$ will be even smaller, and so $\res(X'_U)\leq n$.

\textit{Case 3:} 
If $\varphi=\neg(\psi\star\chi)$ and $\res(\neg\psi)\neq 0$ and $\res(\neg\chi)\neq 0$, then $X'_U=(X_U/\set{\varphi})\cup\set{\neg\psi,\neg\chi}$.
Since $\res(\neg\psi)+\res(\neg\chi)=\res(\varphi)-1$, it follows that $\res(X'_U)=\res(X_U)-1=n$.
If instead $\res(\neg\psi)=0$ or $\res(\neg\chi)=0$, then $\res(X'_U)$ will be even smaller, and so $\res(X'_U)\leq n$.

Since in all cases $\res(X'_U)\leq n$, it follows by hypothesis that $X'$ can be completed in a finite number of steps.
We know by \textbf{Task 2} that $X'$ is the result of resolving $\varphi$ in at most a finite number of branches in $X$.
Since the sum of finite numbers is finite, we may conclude that $X$ may be completed in a finite number of steps. 
Thus it follows by induction that every tree $X$ can be completed in a finite number of steps. 












% \section*{Weak Induction}

% \begin{enumerate}
%   \item[\it Even:] If $\varphi$ is a sentence letter of SL, then $\even(\varphi)$.\\
%     For any SL sentences $\varphi$ and $\psi$: 
%     \begin{itemize}
%       \item[$(\neg)$] $\even(\neg \varphi)$ if $\even(\varphi)$;
%       \item[$(\wedge)$] $\even(\varphi \wedge \psi)$ if $\even(\varphi)$ and $\even(\psi)$;
%       \item[] $\vdots$
%     \end{itemize}
% \end{enumerate}







\end{document}

