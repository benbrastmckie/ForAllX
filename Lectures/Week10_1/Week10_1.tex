\documentclass[a4paper, 11pt]{article} % Font size (can be 10pt, 11pt or 12pt) and paper size (remove a4paper for US letter paper)
\usepackage[protrusion=true,expansion=true]{microtype} % Better typography
\usepackage{graphicx} % Required for including pictures
\usepackage{wrapfig} % Allows in-line images
\usepackage{enumitem} %%Enables control over enumerate and itemize environments
\usepackage{setspace}
\usepackage{amssymb, amsmath, mathrsfs, mathabx} %%Math packages
\usepackage{../lecture} %calls local modified style file
% \usepackage{../forallx-mit} %calls local modified style file
\usepackage{stmaryrd}
\usepackage{mathtools}
\usepackage{multicol} 
\usepackage{mathpazo} % Use the Palatino font
\usepackage[T1]{fontenc} % Required for accented characters
\usepackage{array}
\usepackage{bibentry}
\usepackage{prooftrees} 
\usepackage[round]{natbib} %%Or change 'round' to 'square' for square backers
\setcitestyle{aysep=}
% \usepackage{fitchproof} 

\makeatletter
\renewcommand{\maketitle}{
\begin{flushright}
{\LARGE\@title}

\vspace{10pt}

{\@author}
\\ \@date
\end{flushright}

\vspace{-20pt}

}
\makeatother

%----------------------------------------------------------------------------------------
%	TITLE
%----------------------------------------------------------------------------------------

\title{\textbf{Natural Deduction in $\FI$}} % Subtitle

\author{\textsc{Logic I}\\ \em Benjamin Brast-McKie} % Institution

\date{\today} % Date

%----------------------------------------------------------------------------------------

\begin{document}

\maketitle % Print the title section

\thispagestyle{empty}

%----------------------------------------------------------------------------------------

\section*{From Last Time\ldots}

\begin{enumerate}
  \item[\it Free For:] $\beta$ is \textsc{free for} $\alpha$ in $\metaA$ just in case there is no free occurrence of $\alpha$ in $\metaA$ in the scope of a quantifier that binds $\beta$. 
  % \item[\it Constants:] If $\beta$ is a constant, then $\beta$ is free for any $\alpha$ and $\metaA$. 
  \item[\it Substitution:] If $\beta$ is free for $\alpha$ in $\metaA$, then the \textsc{substitution} $\metaA\unisub{\beta}{\alpha}$ is the result of replacing all free occurrences of $\alpha$ in $\metaA$ with $\beta$. 
  \item[\it Instance:] $\metaA\unisub{\beta}{\alpha}$ is a \textsc{substitution instance} of $\qt{\forall}{\alpha}\metaA$ and $\qt{\exists}{\alpha}\metaA$ if $\beta$ is a constant. 
  % \item[\it Pattern:] $\qt{\forall}{x}\metaA \vdash \metaA\unisub{s}{x} \vdash \qt{\exists}{x}\metaA$.
\end{enumerate}
   

  
\section*{Examples}%
  \label{sec:Examples}

\begin{enumerate}
  \begin{multicols}{2}
    \item All humans are mortal.
    % \item Socrates is mortal if human.
    \item \underline{Socrates is human.\quad\quad}
    % \item Socrates is mortal.
    \item Someone is mortal.

    \setcounter{enumi}{0}
    \item $\forall x(Hx \eif Mx)$
    % \item $Hs \eif Ms$ \quad (by $\forall$E) 
    \item \underline{$Hs$\quad\quad} 
    % \item $Ms$
    \item $\exists xMx$ 
  \end{multicols}
\end{enumerate}



\section*{Motivation}

\begin{itemize}
  \item[\it Logical Consequence:] We have defined logical consequence for $\FI$.
    \item We captured logical form by quantifying over all interpretations.
    \item But semantic proofs are cumbersome to write.
  \item[\it Naturalness:] Want a finite and natural description of logical consequence.
  \item[\it Soundness:] Our description should be accurate.
  \item[\it Completeness:] We also want our description to be complete.
  \item[\bf Question:] What rules do we need to derive the following?
    \item Sid loves everything.
    \item Sid loves Bina.
    \item Sid loves something.
\end{itemize}






\section*{Universal Elimination and Existential Introduction}

\begin{enumerate}
  \item[($\forall$E)] $\forall\alpha\metaA \vdash \metaA\unisub{\beta}{\alpha}$ where $\beta$ is a constant and $\alpha$ is a variable. 
  \item[($\exists$I)] $\metaA\unisub{\beta}{\alpha} \vdash \exists\alpha\metaA$ where $\beta$ is a constant and $\alpha$ is a variable. 
  % \item[-] $\forall x(Hx \eif Mx)$
  % \item[-] $Hs$
  % \item[-] $Hs \eif Ms$ \quad (by $\forall$E) 
  % \item[-] $Ms$ \quad (by $\eif$E) 
  % \item[-] $\exists xMx$ \quad (by $\exists$I) 
  \item[\bf Task:] Derive the Socrates argument above.
  \item[\it Universal:] Everyone is rested or beleaguered $\forall x(Rx \eor Bx)$.
  \item[\it Instantial:] Therefore Tom is rested or beleaguered $Rt \eor Bt$.
  \item[\it Existential:] So something is rested or beleaguered $\exists x(Rx \eor Bx)$, $\exists x(Rt \eor Bx)$, \ldots.
    % \begin{itemize}
    %   \begin{multicols}{2}
    %   \item $\exists x(Rx \eor Bx)$.
    %   \item $\exists x(Rx \eor Bt)$.
    %   \item $\exists x(Rt \eor Bt)$.
    %   \item $\exists y\exists x(Ry \eor By)$.
    %   \item $\exists y\exists x(Rx \eor By)$.
    %   \item $\exists x\exists x(Rx \eor Bx)$.
    %   \end{multicols}
    % \end{itemize}
  % \item[\bf Question:] What about universal introduction and existential elimination?
\end{enumerate}






\section*{Universal Introduction}

\begin{enumerate}
  % \item[\it Generalising:] It would seem that we cannot universally generalise from instances.
  \item[\it Invalid:] The following argument is invalid and should not be derivable. 
    \item \underline{Socrates is mortal. \quad ($Ms$)\quad\quad}
    \item Everything is mortal. \quad ($\forall xMx$)
  \item[\it Valid:] Compare the following valid argument which should be derivable:
    \setcounter{enumi}{0}
    \item $\forall x\forall y\forall z((Rxy \eand Ryz) \eif Rxz)$.
    % \item \underline{$\forall x\forall y(Rxy \eif Ryx)$.\quad\quad}
    % \item $\forall x\forall y\forall z((Rxy \eand Rxz) \eif Ryz)$.
    \item \underline{$\forall x\enot Rxx$.\quad\quad}
    \item $\forall x\forall y(Rxy \eif \enot Ryx)$.
  \item[\bf Task:] Use the rules we have to derive as much as we can.  
    \begin{proof}
      \hypo{a}{\qt{\forall}{x}\qt{\forall}{y}\qt{\forall}{z} ((Rxy \eand Ryz) \eif Rxz)}      \pr{}
      \hypo{b}{\qt{\forall}{x} \enot Rxx}      \pr{}
      \have{c}{\qt{\forall}{y}\qt{\forall}{z} ((Ray \eand Ryz) \eif Raz)}      \Ae{a}
      \have{d}{\qt{\forall}{z} ((Rab \eand Rbz) \eif Raz)}      \Ae{c}
      \have{e}{(Rab \eand Rba) \eif Raa}      \Ae{d}
      \have{f}{\enot Raa}      \Ae{b}
        \open
          \hypo{g}{Rab}        \as{for $\eif$I}
            \open
              \hypo{h}{Rba}  \as{for $\enot$I}
              \have{i}{Rab \eand Rba}  \ai{g,h}
              \have{j}{Raa}  \ce{e,i}
              \have{k}{\enot Raa}  \r{f}
            \close
          \have{l}{\enot Rba}  \ni{h-k}
        \close
      \have{m}{Rab \eif Rba}       \ci{g-l}
      \have{n}{\qt{\forall}{y}(Ray \eif Rya)}       \Ai{m}
      \have{o}{\qt{\forall}{x}\qt{\forall}{y}(Rxy \eif Ryx)}       \Ai{n}
    \end{proof}
    \item[\bf Question:] How are we going to introduce universal quantifiers without making the invalid argument above derivable?
  \item[($\forall$I)] $\metaA\unisub{\beta}{\alpha} \vdash \forall\alpha\metaA$ where $\beta$ is a constant, $\alpha$ is a variable, and $\beta$ does not occur in $\forall\alpha\metaA$ or in a premise or any undischarged assumption. 
  % \item[\it Bad Response:] We cannot introduce universal quantifiers under any condition.
  \item[\it Arbitrary:] The constraints on ($\forall$I) require $\beta$ to be arbitrary. 
  \item[\it Review:] Bad inference above is blocked.
  \item[\it In Premise:] Anu loves every dog.\\
    $\forall x(Dx \eif Lax)\ \vdash\ Da \eif Laa\ \nvdash\ \forall x(Dx \eif Lxx)$.
  \item[\it In Conclusion:] All dogs love themselves.\\
    $\forall x(Dx \eif Lxx)\ \vdash\ Da \eif Laa\ \nvdash\ \forall x(Dx \eif Lax)$.
\end{enumerate}






\section*{Existential Elimination}

\begin{enumerate}
  \item[\bf Task:] Compare the following invalid inference.
    \item \underline{Someone is mortal.\quad\quad}
    \item Zeus is mortal.
  \item[\bf Question:] How are we going to eliminate existential quantifiers without making the argument above derivable?
  \item[\it Example:] Consider the following argument:
    \setcounter{enumi}{0}
    \item Everyone who applied found a position $\forall x(Ax \eif \exists y Fxy)$.
    \item \underline{Someone applied $\exists xAx$.\quad\quad}
    \item Someone found a position $\exists x\exists y Fxy$.
  \item[($\exists$E)] If $\exists\alpha\metaA,\metaA\unisub{\beta}{\alpha} \vdash \metaB$ where $\beta$ is a constant that does not occur in $\exists\alpha\metaA$, $\metaB$, or in a premise or undischarged assumption, then $\exists\alpha\metaA\vdash \metaB$.
  \item[\it Derivation:] We can derive the example without deriving the invalid inference.
\end{enumerate}




\section*{Relations}

\begin{enumerate}
  \item[\bf Question:] Is the following argument valid? 
    \item $\forall x\forall y\forall z((Rxy \eand Ryz) \eif Rxz)$.
    \item \underline{$\forall x\forall y(Rxy \eif Ryx)$.\quad\quad}
    \item $\forall xRxx$.
  \item[\bf Question:] Is the following argument valid?
    \setcounter{enumi}{0}
    \item $\forall x\forall y\forall z((Rxy \eand Ryz) \eif Rxz)$.
    \item \underline{$\forall x\enot Rxx$.\quad\quad}
    \item $\forall x\forall y(Rxy \eif \enot Ryx)$.
  % \item[\it Domain:] Let the \textit{domain} $D$ be any set.
  % \item[\it Relation:] A \textit{relation} $R$ on $D$ is any subset of $D^2$.
  % \item[\it Reflexive:] A relation $R$ is \textit{reflexive} on $D$ \textit{iff} $\tuple{x,x}\in R$ for all $x\in D$.
  % \item[\it Non-Reflexive:] A relation $R$ is \textit{non-reflexive} on $D$ \textit{iff} $R$ is not reflexive on $D$.
  % \item[\bf Question 1:] What is it to be \textit{irreflexive}?
  % \item[\it Irreflexive:] A relation $R$ is \textit{irreflexive} on $D$ \textit{iff} $\tuple{x,x}\notin R$ for all $x\in D$.
  % \item[\it Symmetric:] A relation $R$ is \textit{symmetric iff} $\tuple{y,x}\in R$ whenever ${x,y}\in R$.
  % \item[\bf Question 2:] Why don't we need to specify a domain?
  % \item[\bf Question 3:] Why is a relation reflexive or irreflexive with respect to a domain?
  % \item[\it Asymmetric:] A relation $R$ is \textit{asymmetric iff} $\tuple{y,x}\notin R$ whenever $\tuple{x,y}\in R$.
  % \item[\bf Question 4:] What is it to be non-symmetric? How about non-asymmetric?
  % \item[\bf Task 1:] Show that every asymmetric relation is irreflexive.
  % \item[\it Transitive:] A relation $R$ is \textit{transitive iff} $\tuple{x,z}\in R$ whenever $\tuple{x,y},\tuple{y,z}\in R$.
  % \item[\it Intransitive:] A relation $R$ is \textit{intransitive iff} $\tuple{x,z}\notin R$ whenever $\tuple{x,y},\tuple{y,z}\in R$.
  % \item[\bf Question 5:] Is every symmetric transitive relation reflexive? (No: $R=\varnothing$)
  % \item[\bf Task 2:] Show that every transitive irreflexive relation asymmetric?
  % \item[\it Euclidean:] A relation $R$ is \textit{euclidean iff} $\tuple{y,z}\in R$ whenever $\tuple{x,y},\tuple{x,z}\in R$.
  % \item[\bf Task 3:] Show that every transitive symmetric relation is euclidean.
\end{enumerate}




\end{document}

