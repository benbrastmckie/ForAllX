\documentclass[a4paper, 11pt]{article} % Font size (can be 10pt, 11pt or 12pt) and paper size (remove a4paper for US letter paper)
\usepackage[protrusion=true,expansion=true]{microtype} % Better typography
\usepackage{../lecture} %calls local modified style file
\usepackage{graphicx} % Required for including pictures
\usepackage{wrapfig} % Allows in-line images
\usepackage{enumitem} %%Enables control over enumerate and itemize environments
\usepackage{setspace}
\usepackage{amssymb, amsmath, mathrsfs} %%Math packages
\usepackage{stmaryrd}
\usepackage{mathtools}
\usepackage{mathpazo} % Use the Palatino font
\usepackage[T1]{fontenc} % Required for accented characters
\usepackage{array}
\usepackage{bibentry}
\usepackage[round]{natbib} %%Or change 'round' to 'square' for square backers
\setcitestyle{aysep={}}

\makeatletter
\renewcommand{\maketitle}{
\begin{flushright}
{\LARGE\@title}

\vspace{10pt}

{\@author}
\\ \@date
\end{flushright}

\vspace{-20pt}

}
\makeatother

%----------------------------------------------------------------------------------------
%	TITLE
%----------------------------------------------------------------------------------------

\title{\textbf{Logical Consequence}} % Subtitle

\author{\textsc{Logic I}\\ \em Benjamin Brast-McKie} % Institution

\date{\today} % Date

%----------------------------------------------------------------------------------------

\begin{document}

\maketitle % Print the title section

\thispagestyle{empty}

%----------------------------------------------------------------------------------------


\section*{From Last Time\ldots}

\begin{itemize}[leftmargin=1in,labelsep=.15in] %,label=(\arabic*)]%,label=\roman*]
  \item[\it Semantics:] For any interpretation $\I$ of $\PL$, the \textsc{valuation} function $\V{\I}$ from the wfs of $\PL$ to truth-values is defined:
    \item $\V{\I}(\metaA)=\I(\metaA)$ if $\metaA$ is a sentence letter of $\PL$.
    \item $\V{\I}(\neg\metaA)=1$ iff $\V{\I}(\metaA)=0$~~ (i.e., $\V{\I}(\metaA)\neq 1$).
    \item $\V{\I}(\metaA \eand \metaB)=1$ iff $\V{\I}(\metaA)=1$ and $\V{\I}(\metaB)=1$.
    \item $\V{\I}(\metaA \eor \metaB)=1$ iff $\V{\I}(\metaA)=1$ or $\V{\I}(\metaB)=1$ (or both).
    \item $\V{\I}(\metaA \eif \metaB)=1$ iff $\V{\I}(\metaA)=0$ or $\V{\I}(\metaB)=1$ (or both).
    \item $\V{\I}(\metaA \eiff \metaB)=1$ iff $\V{\I}(\metaA)=\V{\I}(\metaB)$.
  \item[\it Characteristic Truth Tables:] As drawn in the textbook\ldots
  % \item[\it Complexity:] $\comp{\metaA}$ is the function from the wfss of $\PL$ to the natural number $\N$ which satisfies the following for any wfss $\metaA$ and $\metaB$ of $\PL$ and $\star \in \set{\eand, \eor, \eif, \eiff}$: 
  %   \item $\comp{\metaA} = 0$ if $\metaA$ is a sentence letter; 
  %   \item $\comp{\enot\metaA} = \comp{\metaA} + 1$; and
  %   \item $\comp{\metaA \star \metaB} = \comp{\metaA} + \comp{\metaB} + 1$.
\end{itemize}







\section*{Complete Truth Tables}

\begin{itemize}[leftmargin=.75in,labelsep=.15in] %,label=(\arabic*)]%,label=\roman*]
  \item[\it Setup:] Write the sentence on the top right, add the constituent sentence letters on the left, and use the characteristic truth tables.
  \item[\it Constituents:] We define $[\metaA]$ to be the set of sentence letters that occur in $\metaA$: % by taking $[\cdot]:\text{SL}\to\mathcal{P}(\text{SL})$ to be the smallest function to satisfy:
      \item $[\metaA]=\set{\metaA}$ if $\metaA$ is a sentence letter of $\PL$.
      \item For any wfss $\metaA$ and $\metaB$ of $\PL$, and $\star \in \set{\eand,\eor,\eif,\eiff}$: 
    \begin{itemize}
      \item[$(\neg)$] $[\neg \metaA]=[\metaA]$;
      \item[$(\hspace{1pt}\star\hspace{1pt})$] $[\metaA \star \metaB]=[\metaA] \cup [\metaB]$;
    \end{itemize}
  \item[\it Rows:] Add $2^n$ rows for $n$ constituent sentence letters.
  \item[\bf Examples:] $[A \eand (B \eor A)] \eif A$, $C \eiff \enot C$, $D$.
  \item[\it Tautology:] Only $1$s under its main connective in its complete truth table.
  \item[\it Contradiction:] Only $0$s under its main connective in its complete truth table.
  % \item[\it Sets of wfss:] Can construct a complete truth table in a similar fashion. 
  \item[\it Logically Contingent:] A $1$ and a $0$ under its main connective in its complete truth table.
  \item[\it Logical Entailment:] On any row of a complete truth table, the consequent has a $1$ under its main connective whenever the antecedent does. 
  \item[\it Logical equivalence:] Identical columns under the main connectives for the sentences.
  \item[\it Satisfiable:] There is a row where all wfss have a $1$ under all main connectives. 
  \item[\it Logical Consequence:] The conclusion has a $1$ under its main connective in every row in which every premise has a $1$ under its main connectives.
\end{itemize}





\section*{Decidability}

\begin{itemize}[leftmargin=.75in,labelsep=.15in] %,label=(\arabic*)]%,label=\roman*]
  \item[\it Effective Procedure:] A finitely describable and (in principle) usable procedure that always finishes and produces a correct answer to the question asked, requiring only that the instructions be followed accurately.
  \item[\bf Question:] How to define the main operators and distribute truth-values?
    \item Recursively, like the formation rules for the wfs of $\PL$.
  \item[\bf Question:] Is it always possible to construct a complete truth table for a wfs?
    \item Sentences have a finite number of constituent sentence letters.
  \item[\it Decidable:] If there is an effective procedure for determining the answer to a question, that question is \textit{decidable}.
    \item It is decidable whether a wfs of $\PL$ is a tautology, etc. 
  \item[\bf Question:] What about a complete truth table for a set of sentences?
    \item Could require infinitely many sentence letters.
    \item We might be able to define an infinite table, but we can't use it.
  \item[\bf Question:] If one procedure is not effective, couldn't there be another one?
    \item It turns out that there is no effective procedure\ldots
    \item There is always an effective procedure for finite sets of sentences.
  \item[\it Validity:] So the validity of finite arguments is decidable.
\end{itemize}






\section*{Partial Truth Tables}

\begin{itemize}[leftmargin=.75in,labelsep=.15in]
  \item[\bf Worry 1:] It is not \textit{that} effective\ldots\ in practice it is daunting for $n > 4$.
  \item[\it Partial Truth Tables:] Sometimes only one or two lines are needed.
    \item $A \eif \enot (A \eor B)$: not a tautology or contradiction, so contingent. 
    \item $B \eiff \enot (A \eor B)$ is a contradiction, so we need a complete table. 
    \item $C \eor (A \eif A)$ is a tautology, so we need a complete table. 
  \item[\it Complete:] To affirm equivalence, entailment, and logical consequence.
  \item[\it Partial:] To affirm that a set is satisfiable.
  \item[\bf Worry 2:] Still daunting sometimes.
  \item[\bf Worry 3:] Definitions all refer to complete truth tables.
    \item Definition of a complete truth table has some minor ambiguities.
    \item These could be fixed, but the result is cumbersome.
  \item[\it Heuristic:] The truth table definitions are best taken to be a heuristic guide for grasping the abstract definitions we may now provide.
\end{itemize}











\end{document}


