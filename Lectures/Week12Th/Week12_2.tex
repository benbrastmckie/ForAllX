\documentclass[a4paper, 11pt]{article} % Font size (can be 10pt, 11pt or 12pt) and paper size (remove a4paper for US letter paper)
\usepackage[protrusion=true,expansion=true]{microtype} % Better typography
\usepackage{../lecture} %calls local modified style file
\usepackage{graphicx} % Required for including pictures
\usepackage{wrapfig} % Allows in-line images
\usepackage{enumitem} %%Enables control over enumerate and itemize environments
\usepackage{setspace}
\usepackage{amssymb, amsmath, mathrsfs} %%Math packages
\usepackage{stmaryrd}
\usepackage{mathtools}
\usepackage{multicol} 
\usepackage{mathpazo} % Use the Palatino font
\usepackage[T1]{fontenc} % Required for accented characters
\usepackage{array}
\usepackage{bibentry}
\usepackage{prooftrees} 
\usepackage[round]{natbib} %%Or change 'round' to 'square' for square backers
\setcitestyle{aysep=}
% \usepackage{fitchproof} 

\newcommand{\qt}[2]{#1 #2} % for modifying style of quantifer and bound variable pairs
\newcommand{\unisub}[2]{[#1/#2]}

\makeatletter
\renewcommand{\maketitle}{
\begin{flushright}
{\LARGE\@title}

\vspace{10pt}

{\@author}
\\ \@date
\end{flushright}

\vspace{0pt}

}
\makeatother

%----------------------------------------------------------------------------------------
%	TITLE
%----------------------------------------------------------------------------------------

\title{\textbf{Existential Elimination and Soundness}} % Subtitle

\author{\textsc{Logic I}\\ \em Benjamin Brast-McKie} % Institution

\date{\today} % Date

%----------------------------------------------------------------------------------------

\begin{document}

\maketitle % Print the title section

\thispagestyle{empty}

%----------------------------------------------------------------------------------------


\section*{Substitution}

\begin{enumerate}
  \item[\it Free For:] $\beta$ is \textsc{free for} $\alpha$ in $\varphi$ just in case there is no free occurrence of $\alpha$ in $\varphi$ in the scope of a quantifier that binds $\beta$. 
  \item[\it Substitution:] If $\beta$ is free for $\alpha$ in $\varphi$, then the \textsc{substitution} $\varphi\unisub{\beta}{\alpha}$ is the result of replacing all free occurrences of $\alpha$ in $\varphi$ with $\beta$.
\end{enumerate}
   






\section*{QD Rules}

\begin{enumerate}
  \item[($\forall$E)] $\forall\alpha\varphi \vdash \varphi\unisub{\beta}{\alpha}$ where $\beta$ is a constant and $\alpha$ is a variable. 
  \item[($\exists$I)] $\varphi\unisub{\beta}{\alpha} \vdash \exists\alpha\varphi$ where $\beta$ is a constant and $\alpha$ is a variable.
  \item[($\forall$I)] $\varphi\unisub{\beta}{\alpha} \vdash \forall\alpha\varphi$ where $\beta$ is a constant, $\alpha$ is a variable, and $\beta$ does not occur in $\forall\alpha\varphi$ or in any undischarged assumption.
  \item[($\exists$E)] If $\exists\alpha\varphi,\varphi\unisub{\beta}{\alpha} \vdash \psi$ where $\beta$ is a constant that does not occur in $\exists\alpha\varphi$, $\psi$, or in any undischarged assumption, then $\exists\alpha\varphi\vdash \psi$.
  \item[($=$I)] $\vdash \alpha = \alpha$ for any constant $\alpha$. 
  \item[($=$E)] $\varphi\unisub{\alpha}{\gamma},\alpha=\beta\vdash\varphi\unisub{\beta}{\gamma}$.
\end{enumerate}






\section*{Existential Elimination}

\begin{enumerate}
  \item[\bf Task 1:] Regiment and derive the following in QD.
  % \item $\qt{\forall}{x}(x = m), Rma\vdash \qt{\exists}{x} Rxx$
  % \item $\qt{\forall}{x}(x{=}n \equiv Mx), \qt{\forall}{x}(Ox \vee \neg Mx)\vdash On$
  \item The elephant would not obey.\\
    \underline{Patrick is an elephant.}\\ 
    Patrick would not obey.
  \item $\qt{\forall}{x}(Jx \supset Kx)$\\
    $\qt{\exists}{x}\qt{\forall}{y} Lxy$\\
    \underline{$\qt{\forall}{x} Jx$}\\
    $\qt{\exists}{x}(Kx \wedge Lxx)$.
  \item \underline{$\qt{\exists}{x}(Px \supset \qt{\forall}{x}Qx)$}\\
    $\qt{\forall}{x}Px \supset \qt{\forall}{x}Qx$.
  \item \underline{$\qt{\exists}{x}Px \vee \qt{\exists}{x}Qx$}\\
    $\qt{\exists}{x}(Px \vee Qx)$.
  % \item $\qt{\exists}{x}(Kx \wedge \qt{\forall}{y}(Ky \supset x{=}y) \wedge Bx), Kd\vdash Bd$
  % \item $\vdash Pa \supset \qt{\forall}{x}(Px \vee x {\neq} a)$
  \item Every nonempty asymmetric relation is non-symmetric.
\end{enumerate}





\section*{Natural to Normative}

% NOTE: this was used before

\begin{enumerate}
  \item[\it Soundness:] If $\Gamma \vdash \varphi$, then $\Gamma \vDash \varphi$.
  \item Shows that we can trust QD to establish validity.
  \item Easier to derive a conclusion that to provide a semantic argument.
  \item The natural rules of deduction preserve validity.
  \item[\it Natural:] QD describes (approximately) how we in fact reason.
  \item[\it Normative:] Soundness explains why we ought to use QD to reason.
\end{enumerate}




\section*{Soundness of QD}

\begin{enumerate}
  \item[\it Assume:] $\Gamma \vdash_{\textsc{qd}} \varphi$, so there is a QD proof $X$ of $\varphi$ from $\Gamma$. 
  \item[\it Lines:] Let $\metaA_i$ be the wfs on line $i$ of $X$.
  \item[\it Dependencies:] Let $\Gamma_i$ be the undischarged assumptions at line $i$. 
  \item[\it Proof:] The proof goes by induction on length of $X$:
    \begin{itemize}
      \item[\it Base:] $\Gamma_1 \vDash \varphi_i$. 
      \item[\it Induction:] If $\Gamma_k \vDash \varphi_k$ for all $k\leq n$, then $\Gamma_{n+1} \vDash \varphi_{n+1}$. 
    \end{itemize}
  \item[\it Finite:] Since $X$ is finite, there is some $m$ where $\Gamma_m=\Gamma$ and $\varphi_m=\varphi$, so $\Gamma \vDash \varphi$.
\end{enumerate}




\section*{Base Case}

\begin{enumerate}
  \item[\it Proof:] Every line in a QD proof is either a premise or follows by the rules.
  \item[\it Assume:] $\varphi_1$ is either a premise or follows by AS or $=$I. 
    \begin{itemize}
      \item[\it Premise:] If $\varphi_1$ is a premise or assumption, then $\Gamma_1=\set{\varphi_1}$, and so $\Gamma_1\vDash\varphi_1$.
      \item[\it Identity:] If $\varphi_1$ follows by $=$I, then $\varphi_1$ is $\alpha=\alpha$ for some constant $\alpha$. 
      \item Letting $\M=\tuple{\D,\I}$ be any model, $\I(\alpha)=\I(\alpha)$.
      \item Letting $\va{a}$ be a variable assignment, $\VV{\I}{\va{a}}(\alpha)=\VV{\I}{\va{a}}(\alpha)$.
      \item So $\VV{\I}{\va{a}}(\alpha=\alpha)=1$, and so $\vDash \alpha=\alpha$.
      \item Thus $\Gamma_1\vDash \varphi_1$ since $\Gamma_1=\varnothing$.
    \end{itemize}

\end{enumerate}



\section*{Induction Case}

\begin{enumerate}
  \item[\it Assume:] $\Gamma_k \vDash \varphi_k$ for all $k\leq n$.
  \item[\it Undischarged:] If $\varphi_{n+1}$ is a premise or assumption, then the argument above applies. 
  \item[\it Rules:] If $\varphi_{n+1}$ follows from $\Gamma_{n+1}$ by the QD rules, then $\Gamma_{n+1}\vDash\varphi_{n+1}$.
  \item[\it Cases:] There are 12 rules in SD and an additional 6 in QD.
\end{enumerate}

\section*{Further Problems: Relations}

\begin{enumerate}
  \item[\bf Task 1:] Regiment and derive the following in QD.
  \item Every transitive and symmetric relation is quasi-reflexive.
  \item Only the empty relation is symmetric and asymmetric.
  \item Every intransitive relation is irreflexive.
  \item Every intransitive relation is asymmetric.
\end{enumerate}










\end{document}

