% !TeX root = ./12-handout.tex

\setcounter{section}{11}
\section{Metalogic for SL}

\begin{frame}
%\large

\scriptsize{\tableofcontents}

\end{frame}

\subsection{A Meta-refresher}

\begin{frame}
\frametitle{SND as a derivation system, provided that...}
%\large

\begin{itemize}[<+->]

\item As we have seen, Sentential Natural Deduction allows us to derive a conclusion from a set of premises:

%trees provide a shortcut for demonstrating that a set of sentences is inconsistent (i.e. unsatisfiable): construct a tree whose root is these sentences s.t. all branches close
%JH: interesting that SND seemingly does NOT let you show that a set of sentences is unsatisfiable; interesting diff in problem-solving support

%\item Underwrites further shortcuts for demonstrating:

\begin{enumerate}[1.)]

\item valid argument: conclusion on last line, in scope of just premises %trees: \\ (its premises and negated conclusion are unsatisfiable)

\item tautology: on last line in scope of NO premises %that a sentence is a tautology (its negation is unsatisfiable)

\item two logically equivalent sentences: (i) their biconditional is a tautology or (ii) derive one from the other and vice versa (which mirrors biconditional introduction!)

\end{enumerate}

\item But our derivations are justified only if system SND is \textit{sound}

\item And guaranteed to have a derivation for every valid argument only if system SND is \textit{complete}

%of a semantically valid argument only if system SND is \textit{complete}



\end{itemize}
\end{frame}

\begin{frame}
\frametitle{A tale of three turnstiles: one semantic; two syntactic}
%\large

\begin{itemize}[<+->]

\item Double Turnstile $\entails$: logical entailment (indexed to our choice of semantics, i.e. the truth-tables for our connectives)

\item Single Turnstile Tree $\vdash_{STD}$: tree-validity in STD \\ (i.e. premises and negated conclusion as root of a tree whose branches all close---recall that this means that $\Gamma \cup \{\enot \Theta\}$ is \emph{tree-inconsistent}) %is interesting how this idea---that unsatisfiability of premises and negated conclusion---is really at heart of our SND completeness proof!
% this means that $\Gamma \cup \{\enot \Theta\}$ is \emph{tree-inconsistent}: \\ There is a tree with this set as the root s.t. \emph{all branches close}

\item Single Turnstile Natural $\vdash_{SND}$: \emph{derivability} in SND

\end{itemize}
\end{frame}

\begin{frame}
\frametitle{A Tale of Three Turnstiles $\entails$ the semantic one}
%\large

\begin{itemize}[<+->]

%\item Recall that the double turnstile `$\entails$' stands for semantic entailment (aka logical consequence) within (classical) sentential logic SL. 

\item ``$\Gamma \entails \Theta$'' means that $\Gamma$ logically entails $\Theta$ \\ Whenever the premises in $\Gamma$ are true, the conclusion $\Theta$ is true 

\item Equivalently: there is no truth-value assignment (TVA) s.t. \\ $\Gamma$ is satisfied while $\Theta$ is false

\item Equivalently, this means that \emph{$\Gamma \cup \{\enot \Theta\}$ is unsatisfiable}: \\ no TVA satisfies the premises and negated conclusion  

\item We'll use this last fact A LOT in our proof that SND is complete! %completeness! 

\end{itemize}
\end{frame}




\begin{frame}
\frametitle{Soundness vs. Completeness}
%\large

% % It's perhaps interesting to think about why we do not have to demonstrate soundness and completeness results for truthtables. It is almost as if the syntax for truthtables is constitutive of the semantics.

% Could also make some verbal remarks about the notion of mathematical rigor, and how this has evolved over time. And how it might still be contested today, and issues of how rigorous physics ought to be remain highly relevant. Different methodological styles in physics and mathematical physics

\begin{itemize}[<+->]

\item By proving that our derivation system is \textit{sound}, we show that SND derivations are `safe' (they never lead us astray)

\medskip 

\bi

\item \emph{Sound}: If $\Gamma \vdash_{SND} \Theta$, then $\Gamma \entails \Theta$
%Single turnstile entails Double Turnstile 

\item (syntactic to semantic: i.e. we chose `good' rules!)

\ei

\bigskip 

\item By proving that SND is \textit{complete}, we show truth tables are not needed to demonstrate validity: SND derivations suffice
%what about showing a set of sentences is unsatisfiable though? can we just not do this w/ SND derivations? would we need to introduce a falsum symbol? 

\medskip 

\bi

\item \emph{Complete}: If $\Gamma \entails \Theta$, then $\Gamma \vdash_{SND} \Theta$
%Double Turnstile entails  Single turnstile  

\item (logical entailment is fully covered by our syntactic rules)

\item (Means: we wrote down \textit{enough} rules!)

\ei

\end{itemize}
\end{frame}



\begin{frame}
\frametitle{Some contrasts with our metalogic proofs for trees (STD)}
%\large

\begin{itemize}[<+->]

%%a similarity: in both cases, we rely on syntactic notions of consistency to construct TVAs that satisfy a set of sentences! e.g. tree-consistency and tree-inconsistency vs. consistency-in-SND 

\item Recall that to prove the soundness and completeness of our tree system STD, we proved the \textit{contrapositive} of these statements

\item[] -- vs. With SND, we'll proceed directly  %is this b/c we're not working w/ the negated conclusion anymore?

\item With trees, our premise set $\Gamma$ was finite

\item[] -- vs. Here, we'll let $\Gamma$ be infinite. \footnotesize{Although of course, whenever we talk about an SND derivation, this derivation must have a FINITE premise set $\Delta \subseteq \Gamma$ (i.e. a finite list of SL wffs justified by `:PR')}

\item Is this finiteness restriction a limitation of trees? 

\item Not in practice: no valid SL argument ever requires infinitely-many premises to entail its conclusion (PS 12 \#4) 

% since SL is \emph{compact}: a set is satisfiable iff every finite subset is satisfiable. By contrapositive of compactness: some finite subset is unsatisfiable iff the set is unsatisfiable. 
%Then recall that an argument is valid iff the premises union negated conclusion is unsatisfiable. So given a valid argument with infinite premises, there must be a finite subset that when unioned with negated conclusion is unsatisfiable. so that gives us a valid argument to the conclusion from finitely many premises. 

% Compactness: a set of SL sentences is satisfiable if and only if each of its finite subsets is satisfiable.

\end{itemize}
\end{frame}

\begin{frame}
\frametitle{Semantic entailment for infinitely-many premises}
%\large

\begin{itemize}[<+->]

\item Let $\Gamma$ be a possibly infinite set of premises; $\Theta$ a conclusion

\item Recall: a TVA assigns `True' or `False' to the (infinitely-many) SL atomic wffs

\item In the case where $\Gamma$ is finite, its premises contain finitely-many atomic wffs, so we can restrict a TVA to a row of a truth table

\item An argument is \emph{semantically invalid} if there is a TVA that makes each wff in $\Gamma$ true but which makes $\Theta$ false

\item In this case we write $\Gamma \nentails \Theta$

\item If there is no such TVA, then $\Gamma \entails \Theta$

\end{itemize}
\end{frame}

\begin{frame}
\frametitle{SND derivability for infinitely-many premises}
%\large

\begin{itemize}[<+->]

\item $\Theta$ is \emph{SND-derivable} from $\Gamma$ provided there is an SND derivation:

\begin{enumerate}[1.)]

\item whose starting premises $\Delta$ are a finite subset of $\Gamma$ 

\item in which $\Theta$ appears on its own in the final line

\item where $\Theta$ is directly next to the main scope line, i.e. only in the scope of the $\Delta$-premises

\end{enumerate}

\item In this case, we write $\Gamma \vdash_{SND} \Theta$ (also: $\Delta \vdash_{SND} \Theta$)

\item If no such derivation exists, then we say that $\Theta$ is NOT SND-derivable from $\Gamma$, and we write $\Gamma \nvdash_{SND} \Theta$

\end{itemize}
\end{frame}

\subsection{Soundness of System SND}

\subsubsection{Righteous Throat-clearing}

\begin{frame}
\frametitle{Soundness: Proof Idea and notation}
%\large

\begin{itemize}[<+->]

\item Subgoal: given any line in an SND derivation, show that the well-formed formula (wff) on that line is entailed by the \\ premises or assumptions accessible from that line


\item Let ``\emph{$P_k$}" be the wff on line $k$, i.e. the $k$-th wff in our derivation

\item Let ``\emph{$\Gamma_k$}" be the set of premises/assumptions accessible on line $k$, i.e. the set of open assumptions/premises in whose scope $P_k$ lies

\item \emph{Subgoal}: given a wff $P_k$ on line $k$, show that $\Gamma_k \entails P_k$

\item (like with soundness for trees, we reason ``from the top down")

\end{itemize}
\end{frame}

\begin{frame}
\frametitle{Soundness: Proof Strategy}
%\large

\begin{itemize}[<+->]

\item Recall that SND derivations are defined recursively: \\ from a (possibly empty) set of premises, we have a finite number of rules to add a line 

\item[] -- These ways include reiteration and an intro and elimination rule for each of our five connectives

\item Hence: do induction on the number of lines in an SND derivation

\item Show that the base case has the property (line \#1)

\item Induction hypothesis: assume the property holds for all lines $\leq k$. 

\item Induction step: show the property holds for line \#k+1 \\ (by considering all possible ways line \#k+1 could arise)

\end{itemize}
\end{frame}

\begin{frame}
\frametitle{Let's get Righteous!}
%\large

\begin{itemize}[<+->]

\item Say that a line $i$ of a derivation is \emph{righteous} just in case $\Gamma_i \entails P_i$, i.e. just in case the set of assumptions/premises accessible from $i$ semantically entail the wff on that line. 

\item Call a derivation \textit{righteous} if every line in it is righteous

\item Our goal is to prove that every derivation in SND is righteous!

%\item If $X \vdash_{SND} P$, then . 

%\item Then, if we can SND-derive a wff $P$ from some set $X$, we'll have $\Gamma \entails P$ and $\Gamma$ will just be a subset of premises in $X$ in whose scope $P$ lies. 



\end{itemize}
\end{frame}

\begin{frame}
\frametitle{Do I sound righteous? (from righteousness to soundness)}
%\large

\begin{itemize}[<+->]

%From Righteousness to Soundness

\item Let $\Gamma$ be any set of SL wffs (possibly infinite)

\item If $\Gamma \vdash_{SND} \metav{P}$, then by definition there is a derivation whose (finitely-many) premises $\Delta$ belong to $\Gamma$, such that \metav{P} occurs on the final line and lies in the scope of $\Delta$ (i.e. $\Delta \vdash_{SND} \metav{P}$)

\item Then by righteousness, $\Delta \entails \metav{P}$

\item[] -- i.e. any TVA that makes $\Delta$ true must make $\metav{P}$ true

%\item i.e. any TVA 

\item So there is no truth-value assignment that makes all the sentences in $\Gamma$ true while making $\metav{P}$ false, so $\Gamma \entails \metav{P}$ as well

\item So we will have shown \emph{Soundness}: If $\Gamma \vdash_{SND} \metav{P}$, then $\Gamma \entails \metav{P}$

\end{itemize}
\end{frame}

\subsubsection{Soundness: the proof itself}

\begin{frame}
\frametitle{Base Case}
%\large

\begin{itemize}[<+->]

\item \emph{Base case}: for any SND derivation, show that $\Gamma_1 \entails \metav{P}_1$.

\item Proof: $\Gamma_1$ is the set of premises accessible at line \#1, which comprises exactly the wff $\metav{P}_1$ 

\item (recall that every premise of a derivation lies in its own scope---
\item[] \qquad i.e. these premises be gettin' high off their own supply)

\item Clearly, $\metav{P}_1 \entails \metav{P}_1 $, so $\{ \metav{P}_1 \} \entails \metav{P}_1$

\item So line \#1 is righteous (i.e. $\Gamma_1 \entails \metav{P}_1$)

\end{itemize}
\end{frame}

\begin{frame}
\frametitle{Stating the Induction Step}
%\large

\begin{itemize}[<+->]

\item  \emph{Induction Hypothesis}: Assume that every line $i$ for $1 < i \leq k$ is righteous (i.e. that $\Gamma_i \entails \metav{P}_i $)

\item Induction step: Consider line \#k+1; show that $\Gamma_{k+1} \entails \metav{P}_{k+1}$

\item We have 12 cases to consider! 11 of these arise from our 11 SND-sanctioned rules for extending a derivation. 

\item What is the 12th case?? (We could say 13, but that is BAD LUCK)

\end{itemize}
\end{frame}

\begin{frame}
\frametitle{Case 1: Premise or Assumption}
%\large

\begin{itemize}[<+->]

\item \emph{Case 1}: $\metav{P}_{k +1}$ is a premise (:$PR$) or a subproof assumption (:$AS$). \\ Show that $\Gamma_{k+1} \entails \metav{P}_{k+1}$

\item Either way, $\metav{P}_{k +1} \in \Gamma_{k+1}$ (since every premise and assumption lies within its own scope)

\item So given a TVA that makes every sentence in $\Gamma_{k+1}$ true, \\ this TVA must make $\metav{P}_{k +1} $ true 

\item So $\Gamma_{k+1} \entails \metav{P}_{k+1}$; so this case be righteous! 

\end{itemize}
\end{frame}

\begin{frame}
\frametitle{Case 2: Reiteration}
%\large

\begin{itemize}[<+->]

\item \emph{Case 2}: $\metav{P}_{k +1}$ arises from an application of rule $R$, reiteration

\item Then wff $\metav{P}_{k +1}$ appears on an earlier line \#i as the wff $\metav{P}_{i}$

\item By the induction hypothesis, line \#i is righteous, so $\Gamma_{i} \entails \metav{P}_{i}$. 

\item[] --Hence, we also have $\Gamma_{i} \entails \metav{P}_{k+1}$ (since $\metav{P}_{i} = \metav{P}_{k+1}$)

\item To apply rule $R$, $\metav{P}_{k +1}$ must lie to the right of   line \#i's rightmost scope line $\Rightarrow$ $\Gamma_{i} \subseteq \Gamma_{k+1}$ (i.e., all of the premises/assumptions accessible at line \#i must also be accessible at line \#k+1). 

\item Since $\Gamma_{i} \entails \metav{P}_{k+1}$ and $\Gamma_{i} \subseteq \Gamma_{k+1}$, we have  $\Gamma_{k+1} \entails \metav{P}_{k+1}$

\item Draw a schematic derivation to better understand $\Gamma_{i} \subseteq \Gamma_{k+1}$!

%the scope line condition!

%the PR/AS accessible at line 

\end{itemize}
\end{frame}

\begin{frame}
\frametitle{Case 3: Conjunction Introduction \footnotesize{(Things be heating up---finally!)}}
%\large

\begin{itemize}[<+->]

\item \emph{Case 3}: $\metav{P}_{k +1} := (\metav{Q} \eand \metav{R})$ arises from an application of rule $\eand$I

%\makebox[\textwidth]{\footnotesize{Things be heating up (finally!)}}

\item Then on two earlier lines \#h and \#j, \metav{Q} and \metav{R} appear, respectively

\item By the IH, both of these lines are righteous, so $\Gamma_{h} \entails \metav{Q}$ and $\Gamma_{j} \entails \metav{R}$

\item By rule $\eand$I, both these lines must be accessible on line \#k+1

\item So $\Gamma_{h} \cup \Gamma_{j} \subseteq \Gamma_{k+1}$ (i.e. both $\Gamma_{h}$ and $\Gamma_{j}$ are subsets of $\Gamma_{k+1}$)

\item Hence, any TVA that satisfies $\Gamma_{k+1}$ must satisfy both  $\Gamma_{h}$ and $\Gamma_{j}$, and hence satisfy \metav{Q} and also satisfy \metav{R}

\item Thus, any TVA that satisfies $\Gamma_{k+1}$ satisfies $(\metav{Q} \eand \metav{R})$

\item So $\Gamma_{k+1} \entails \metav{P}_{k+1}$



\end{itemize}
\end{frame}

\begin{frame}
\frametitle{Case 4: Conjunction Elimination}
%\large

\begin{itemize}[<+->]

\item \emph{Case 4}: $\metav{P}_{k +1}$ arises from an application of rule $\eand${}E

\makebox[\textwidth]{\footnotesize{I'm about to eliminate this proof, son!}} 

\item Then there is an earlier line \#h of the form $\metav{P}_{k +1} \eand \metav{Q} $ or $\metav{Q} \eand \metav{P}_{k +1}$

\item By the IH, line \#h is righteous, so $\Gamma_{h} \entails \metav{P}_{h}$

\item Since line \#h is accessible at line \#k+1, $\Gamma_{h} \subseteq \Gamma_{k+1}$

\item So any TVA that satisfies $\Gamma_{k+1}$ also satisfies $\Gamma_{h}$ and thereby makes true $\metav{P}_{h}$

\item By the truth conditions for conjunctions, any TVA that satisfies $\metav{P}_{h}$ satisfies both conjuncts, in particular $\metav{P}_{k +1}$

\item So $\Gamma_{k+1} \entails \metav{P}_{k+1}$ and line \#k+1 is righteous

\end{itemize}
\end{frame}

\begin{frame}
\frametitle{Case 8: Conditional Introduction}
%\large

\begin{itemize}[<+->]

\item \emph{Case 8}: $\metav{P}_{k +1}$ arises from rule $\eif${}I, which involves a subproof! 

\item $\metav{P}_{k +1}$ must be of the form $\metav{Q} \eif \metav{R}$ (\textbf{draw derivation} to define terms)

\item NTS: $\Gamma_{k+1} \entails \metav{Q} \eif \metav{R}$ given that $\Gamma_h \entails \metav{Q}$ and \emph{$\Gamma_j \entails \metav{R}$}, by Ind. Hyp.

\item Proceed by cases: either $\Gamma_{k+1}$ satisfies \metav{Q} or it doesn't:
%really just relying on this latter entailment! 

\item If $\Gamma_{k+1}$ does not satisfy \metav{Q}, then it trivially satisfies $\metav{Q} \eif \metav{R}$

\item Otherwise, $\Gamma_{k+1}$ satisfies \metav{Q}. Since $\Gamma_j \subseteq \Gamma_{k+1} \cup \{ \metav{Q} \}$, this means that $\Gamma_j$ is satisfied in this case. Then since line \#j is righteous, we have $\Gamma_{k+1} \cup \{ \metav{Q} \} \entails \metav{R}$. So in this case, $\Gamma_{k+1}$ satisfies $\metav{Q} \eif \metav{R}$ as well. 

\item So in either case, $\Gamma_{k+1} \entails \metav{P}_{k+1}$

%So in this case, we have $\Gamma_{k+1} \entails 

%$\Gamma_j \entails \metav{R}$. And 

\end{itemize}
\end{frame}

\begin{frame}
\frametitle{Case 9: Negation Introduction}
%\large

\begin{itemize}[<+->]

\item \emph{Case 8}: $\metav{P}_{k +1}$ arises from rule $\enot${}I, using a subproof! 

\item $\metav{P}_{k +1}$ must be of form $\enot \metav{Q}$; \textbf{draw derivation to define lines}

\item NTS: $\Gamma_{k+1} \entails \enot \metav{Q}$ given that $\Gamma_h \entails \metav{Q}$, $\Gamma_j \entails \metav{R}$ \emph{and} $\Gamma_m \entails \enot \metav{R}$ (by IH)

\item Notice that $\Gamma_j$ and $\Gamma_m$ are both subsets of $\Gamma_{k+1} \cup \{ \metav{Q}\}$ 

\item[] Hence, $\Gamma_{k+1} \cup \{ \metav{Q}\}$ entails both $\metav{R}$ and $\enot \metav{R}$ as well. 

\item[] Thus, any TVA that satisfies $\Gamma_{k+1} \cup \{ \metav{Q}\}$ must make both $\metav{R}$ and $\enot \metav{R}$ true, which is impossible (i.e. there can be no such TVA). 
\item[] $\Rightarrow$ $\Gamma_{k+1} \cup \{ \metav{Q}\}$ is unsatisfiable. Hence, $\Gamma_{k+1} \entails \enot \metav{Q}$

%i.e. whenever a TVA makes every wff in $\Gamma_{k+1}$ true, it must make Q false

\end{itemize}
\end{frame}


\iffalse 
\begin{frame}
\frametitle{Reminder for Josh!}
%\large

\begin{itemize}[<+->]

\item If we actually make it this far, give hints on PS12 soundness question!

\item If the people don't want these hints, then clearly they'd rather be complete!
%hence, then let's move on to completenes

\item ``The customer is always right!"

\item (Schematize this sentence in quantifier logic)
%The Corporation says: 
\end{itemize}
\end{frame}
\fi 

\subsection{Completeness of System SND}

\subsubsection{Completing our terminology}

\begin{frame}
\frametitle{Semantic vs. Syntactic Consistency}
%\large

\begin{itemize}[<+->]

\item We will appeal to two distinct notions of consistency throughout

\item One is \emph{semantic}: this is the notion we are already familiar with:

\item[] there is a TVA that \emph{satisfies} every sentence in the set

\item We introduce a new \textbf{syntactic} notion of consistency relative to our SND derivation system: 

\item[] -- a set of SL wffs is \textbf{SND-consistent} provided that you can't derive contradictory sentences from it in SND

\item Core proof idea: we'll show that if a set of sentences is \textbf{consistent-in-SND}, then it is also semantically consistent (i.e. \emph{satisfiable}). So by the contrapositive: if a set is \textbf{\textcolor{OGlyallpink}{un}}satisfiable, then it is \textbf{\textcolor{OGlyallpink}{in}}consistent-in-SND. 

\end{itemize}
\end{frame}


\begin{frame}
\frametitle{Semantic: Satisfiable (truth-functionally consistent)}
%\large

\begin{itemize}[<+->]

\item Recall: a set of SL sentences is \emph{satisfiable} provided there is a TVA that makes all of them true

%\begin{itemize}

\item This is a \textit{semantic} notion of consistency

\item i.e. \emph{truth-functionally consistent} 
%TF-consistent, (jointly) \emph{satisfiable}  

%\end{itemize}

\item Contrast this with the syntactic notion of \textbf{consistency in SND}:

\end{itemize}
\end{frame}




\begin{frame}
\frametitle{Syntactic: (In)consistent-in-SND (derivationally consistent)}
%\large

\begin{itemize}[<+->]

\item Let $\Gamma$ be a (possibly infinite) set of SL wffs 

\item \textbf{\textcolor{OGlyallpink}{Inconsistent-in-SND}}: from premises in $\Gamma$, we can derive contradictory formulas $R$ and $\enot R$ in the scope of the main scope line (i.e. these premises)

\item \emph{Consistent-in-SND}: $\Gamma$ is not SND-inconsistent, i.e. there is no derivation from premises in $\Gamma$ resulting in contradictory formulas within the main scope

\item Other words we might use for these concepts: SND-inconsistent, derivationally-inconsistent, SND-consistent, etc.

\item Just remember: this syntactic notion has nothing to do with truth value assignments!

\end{itemize}
\end{frame}



\subsubsection{Proof Sketch}

\begin{frame}
\frametitle{Proof Sketch}
%\large

\begin{itemize}[<+->]

\item Goal: prove the completeness of SL: for every SL wff $\metav{P}$ and every set $\Gamma$ of SL sentences, if $\Gamma \entails \metav{P}$ then $\Gamma \vdash_{SND} \metav{P}$

\item So assume that $\Gamma \entails \metav{P}$. 

\item Recall from week 5: this means that $\Gamma \cup \{\enot \metav{P}\}$ is semantically inconsistent (i.e. \textbf{\textcolor{OGlyallpink}{unsatisfiable}}): \\ no TVA satisfies the premises and negated conclusion  

\item We now appeal to a \emph{Consistency lemma} that is the heart of the enterprise: any SND-consistent set of SL sentences is satisfiable (i.e. semantically consistent)

\end{itemize}
\end{frame}

\begin{frame}
\frametitle{Proof Sketch: Using the consistency lemma}
%\large

\begin{itemize}[<+->]

\item \emph{Consistency lemma}: any SND-consistent set of SL sentences is satisfiable

\item \textbf{\textcolor{OGlyallpink}{Contrapositive}} of CL: any set of SL sentences that is \textcolor{OGlyallpink}{Un}satisfiable is SND-\textcolor{OGlyallpink}{In}consistent

\item From  $\Gamma \entails \metav{P}$ we know that $\Gamma \cup \{\enot \metav{P}\}$ is unsatisfiable

\item So by the contrapositive of CL, we see that $\Gamma \cup \{\enot \metav{P}\}$ is SND-inconsistent

\item This means that we can derive a pair of contradictory sentences $R$ and $\enot R$ from $\Gamma \cup \{\enot \metav{P}\}$! So using the power of negation elimination, we can derive $\metav{P}$ from $\Gamma$, i.e. $\Gamma \vdash \metav{P}$. So we are `done'! 

\end{itemize}
\end{frame}

\begin{frame}
\frametitle{Negation Elimination Refresher (book's claim 6.4.4)}
%\large

\begin{itemize}[<+->]

\item Claim: if $\Gamma \cup \{\enot \metav{P}\}$ is \textbf{\textcolor{OGlyallpink}{SND-inconsistent}}, then $\Gamma \vdash \metav{P}$

\item Proof: starting with (finitely-many) premises $\Delta$ from $\Gamma$, introduce $\enot \metav{P}$ as a subproof assumption for negation elimination

\item Since $\Gamma \cup \{\enot \metav{P}\}$ is SND-inconsistent, we can derive a contradictory pair $R$ and $\enot R$ within the scope of wffs in $\Delta \cup \{\enot \metav{P}\}$

\item Then discharge this assumption $\enot \metav{P}$ by negation elimination, writing $\metav{P}$, now in the scope of $\Delta$. So $\Delta \vdash \metav{P}$

\item Since $\Delta \subseteq \Gamma$, we have $\Gamma \vdash \metav{P}$

%\item Recall that from a contradictory pair, we can derive anything! 



\end{itemize}
\end{frame}

\subsubsection{The completely straightforward part}

\begin{frame}
\frametitle{Core subgoal: Prove consistency lemma (book's 6.4.2)}
%\large

\begin{itemize}[<+->]

\item So all we have to do is prove the \emph{consistency lemma}: any SND-consistent set of SL sentences is satisfiable

\item We'll prove this lemma in three `stages':

\item The first two are straightforward: given an SND-consistent set $\Gamma$, we construct a \textbf{\textcolor{blue}{superset $\Gamma^{\ast}$}} that is \textit{\textcolor{blue}{maximally} SND-consistent}

\item In the third stage, we show that any maximally SND-consistent set is \alert{satisfiable}: we use maximal consistency to construct a TVA that satisfies every sentence in $\Gamma^{\ast}$

\item Since by construction $\Gamma \subseteq \Gamma^{\ast}$, this TVA satisfies $\Gamma$ as well. 

\item \footnotesize{(The idea in the third stage is similar to what we did with trees: use a syntactic consistency property to construct a TVA that satisfies a set of wffs: with trees we had `\textcolor{blue}{complete} \alert{open} branches'; here we have \textcolor{blue}{maximal}-\alert{SND-consistency})} 
\item The third stage comprises a tedious lemma and induction! \\ PS12 problems 2 and 3 provide practice with this tedium! 

%this is just like constructing a TVA that satisfies all of the sentences in 


\end{itemize}
\end{frame}

\begin{frame}
\frametitle{Maximally SND-consistent}
%\large

\begin{itemize}[<+->]

\item A set $\Gamma^{\ast}$ of SL wffs is \emph{maximally SND-consistent} provided that:

\begin{enumerate}[1.)]

\item $\Gamma^{\ast}$ is SND-consistent (i.e. can't derive contradictory sentences)

\item adding \textbf{any} additional wff to $\Gamma^{\ast}$ would result in an SND-\textcolor{OGlyallpink}{inconsistent} set

\end{enumerate} 

\item i.e. for any $P \notin \Gamma^{\ast}$, $\{P\} \cup \Gamma^{\ast}$ is SND-\textcolor{OGlyallpink}{inconsistent}

\item Motivation: it is straightforward (but tedious) to show that a maximally SND-consistent set is semantically consistent
% % the idea here is very similar to what we did in the completeness proof for the tree system: we rely on an `complete open' derivation (i.e. one w/ no contradictions in main scope) to construct a TVA that satisfies every sentence in the appropriate scope of the starting premises

\item[] -- Moreover, every SND-consistent set is a subset of a maximally SND-consistent set. \item[] -- So we piggyback on an appropriate $\Gamma^{\ast}$ to show that any SND-consistent set $\Gamma$ is also \alert{satisfiable} %semantically consistent

\end{itemize}
\end{frame}

\begin{frame}
\frametitle{Stage 1: Constructing $\Gamma^{\ast}$}
%\large

\begin{itemize}[<+->]

\item Let $\Gamma$ be an SND-consistent set of SL wffs (possibly infinite)

\item To construct $\Gamma^{\ast}$, we first \emph{enumerate} the SL wffs, so that every SL wff is associated with a unique positive integer $\{1, 2, 3, \dots \}$

\item Then consider the first wff `$A$' in our enumeration. \\ If $A$ can be added to $\Gamma$ without the resulting set being SND-inconsistent, then let  $\Gamma_1 := \Gamma \cup \{A\}$. 

\item Otherwise, let  $\Gamma_1 := \Gamma$ (so that $\Gamma_1$ stays SND-consistent)

% \item In general, if $P_k$ is the $k$-th sentence in our enumeration, then $\Gamma_{k+1}$ is $\Gamma_k \cup \{P_k\}$ provided $\Gamma_k \cup \{P_k\}$ is SND-consistent; \\ otherwise, $\Gamma_{k+1}$ equals $\Gamma_k$

\item Then, proceed to the second wff in our enumeration. \\ If it can be added to $\Gamma_1$ without the new set being SND-inconsistent, let $\Gamma_2$ be the result. Otherwise, let $\Gamma_2 := \Gamma_1$

\item $\Gamma^{\ast}$ is the result of `doing' this procedure for every SL wff

\item More precisely, $\Gamma^{\ast} := \bigcup_{k=1}^{\infty} \Gamma_k$


\end{itemize}
\end{frame}

\begin{frame}
\frametitle{Enumeration (lexical ordering)}
%\large

\begin{itemize}[<+->]

\item Analogy: we can enumerate words by length, using their alphabetical order to break ties %(so that 5-letter words beginning w/ `a' come before those w/ `b')

\item Can do the same for SL wffs by stipulating an `alphabetical order':

\item $\enot, \eor, \eand, \eif, \eiff, (, ), 0, 1, \dots, 9, A, B, \dots, Z$

\item Each symbol is assigned an \textbf{index} between `10' and `55' %so 43 indices total: 55-10+1-3, since we actually skip 17, 18, and 19, starting 0 at `20' so that A starts at `30'

\item Then each SL wff corresponds to a unique positive integer, constructed by replacing each symbol in the wff with its index, from left to right. 

\item So with our ordering, `$A$' is the first wff; `$B$' the second \dots up to $Z$, and then we hit $\enot A$ ($\mapsto 1030$), then $\enot B$ ($\mapsto 1031$), etc. 

\end{itemize}
\end{frame}

\begin{frame}
\frametitle{Stage 2: $\Gamma^{\ast}$ is maximally SND-consistent}
%\large

\begin{itemize}[<+->]

\item This requires proving two claims (from definition of M-SND-C):

\bigskip

\begin{enumerate}[1.)]

\item $\Gamma^{\ast}$ is consistent in SND

\item Adding any additional wff to $\Gamma^{\ast}$ would result in an \textbf{\textcolor{OGlyallpink}{SND-inconsistent}} set

\end{enumerate}

\bigskip

\item We prove these in turn

\end{itemize}
\end{frame}

\begin{frame}
\frametitle{Stage 2 (i): $\Gamma^{\ast}$ is SND-consistent}
%\large

\begin{itemize}[<+->]

\item Assume for \textit{reductio} that $\Gamma^{\ast}$ is inconsistent in SND

\item Then there would be an SND derivation with finite premise set $\Delta \subset \Gamma^{\ast}$ that derives a contradictory pair $R$ and $\enot R$

\item Since $\Delta$ is finite, there exists some $k \in \mathbb{N}$ s.t. $\Delta \subset \Gamma_k$. \\ So then this $\Gamma_k$ would be \textcolor{OGlyallpink}{SND-inconsistent}. 

\item Yet, we constructed each $\Gamma_k$ such that each is \alert{SND-consistent}: 

\bi

\item In general, if $P_k$ is the $k$-th sentence in our enumeration, then $\Gamma_{k+1}$ is $\Gamma_k \cup \{P_k\}$ provided that $\Gamma_k \cup \{P_k\}$ is SND-consistent; \\ otherwise, $\Gamma_{k+1}$ equals $\Gamma_k$ (so SND-consistent either way)

\ei

\item Hence, $\Gamma^{\ast}$ must be SND-consistent, on pain of \textit{reductio} 

\item \footnotesize{(note: the book's proof, p. 256, is way more complicated than necessary\dots)}
% does a lot of work to show this that seems unnecessary)

\end{itemize}
\end{frame}


\begin{frame}
\frametitle{Stage 2 (ii): $\Gamma^{\ast}$ is \textcolor{blue}{maximally} SND-consistent}
%\large

\begin{itemize}[<+->]

\item Assume for \textit{reductio} that $\Gamma^{\ast}$ weren't maximally SND-consistent, despite being SND-consistent

\item i.e. assume \textit{it is \textcolor{red}{not the case that}} for all additional wff, adding it to $\Gamma^{\ast}$ would result in an \textcolor{OGlyallpink}{SND-inconsistent} set

% % Relevant claim: for any additional wff not already in $\Gamma^{\ast}$, adding Q to $\Gamma^{\ast}$ results in an SND-inconsistent set. So we negate this: there exists some Q such that when added to $\Gamma^{\ast}$, $\Gamma^{\ast}$  remains consistent. 

\item[] $\Rightarrow$ there exists a wff $\metav{Q}$ that we could add to $\Gamma^{\ast}$ while preserving \alert{SND-consistency} (i.e. there would be some wff that we neglected that could make  $\Gamma^{\ast}$ an even `bigger' SND-consistent set)
%relevant notion of `size' here is given by subset relation, rather than cardinality

\item Yet, $\metav{Q}$ would appear in our enumeration as some wff $P_k$, `considered' at the $k$-th stage of our construction of $\Gamma^{\ast}$.

\item So if $\metav{Q}$ isn't in $\Gamma^{\ast}$, then this is because adding it `would have' made $\Gamma_k \subset \Gamma^{\ast}$ \textcolor{OGlyallpink}{SND-inconsistent}. \\ So $\{\metav{Q}\} \cup \Gamma^{\ast}$ must be SND-inconsistent (\textit{reductio}!)

%So $\{\metav{Q}\} \cup \Gamma_k$ and hence $\{\metav{Q}\} \cup \Gamma^{\ast}$ must be SND-inconsistent (\textit{reductio}!)




%So adding $\metav{Q}$ would result in $\Gamma^{\ast}$ being SND-inconsistent

\item So we can't add any $\metav{Q}$ to $\Gamma^{\ast}$ while preserving SND-consistency 

%So there can't be a wff $\metav{Q}$ that we could add to $\Gamma^{\ast}$ while preserving SND-consistency 

\end{itemize}
\end{frame}

\subsubsection{Stage 3: The completely tedious part}

\begin{frame}
\frametitle{Stage 3: The Maximal Consistency Lemma (book's 6.4.8)}
%\large

\begin{itemize}[<+->]

\item \textbf{\textcolor{blue}{Maximal} \alert{Consistency Lemma}}: any set that is maximally-SND-consistent is satisfiable

\item So there exists a TVA that satisfies every sentence in $\Gamma^{\ast}$. \\ We construct this TVA, calling it ``$\metav{I}$" (the book calls it $\textbf{A}^{\ast}$)
%$\mathbf{A}^{\ast}$

\item Proof idea: since $\Gamma^{\ast}$ is M-SND-C, for any wff $\metav{Q}$, either $\metav{Q} \in \Gamma^{\ast}$ or $\textcolor{red}{\enot \metav{Q}} \in \Gamma^{\ast}$ (you're either in the club or your `\textcolor{red}{nemesis}' is!)
%you're out of the club!)

\item[] This holds in particular for each atomic wff

\item Define the TVA $\metav{I}$ such that $\metav{I}(B) = True$ iff atomic $B \in \Gamma^{\ast}$

\item Then by the recursive structure of SL wffs, $\metav{I}(\metav{Q}) = True$ iff $\metav{Q} \in \Gamma^{\ast}$

\end{itemize}
\end{frame}

\begin{frame}
\frametitle{Stage 3 (i): the Membership Lemma (book's 6.4.11)}
%\large

\begin{itemize}[<+->]

\item To induct on SL, we first show some constraints on $\Gamma^{\ast}$ membership

\item Basically, $\Gamma^{\ast}$ is like a club with a bouncer who enforces maximal consistency. Before the bouncer lets a wff into $\Gamma^{\ast}$, he checks who else is in the club 
%(the smaller fish, relative to our lexical ordering)

\item \emph{Membership Lemma} for club $\Gamma^{\ast}$: if \metav{P} and \metav{Q} are SL wffs, then:

\begin{enumerate}[a.)]

\item $\enot \metav{P} \in \Gamma^{\ast}$ if and only if $\metav{P} \notin \Gamma^{\ast}$

\item $\metav{P} \eand \metav{Q} \in \Gamma^{\ast}$ if and only if both $\metav{P}\in \Gamma^{\ast}$ and $\metav{Q}\in \Gamma^{\ast}$

\item $\metav{P} \eor \metav{Q} \in \Gamma^{\ast}$ if and only if either $\metav{P}\in \Gamma^{\ast}$ or $\metav{Q}\in \Gamma^{\ast}$

\item $\metav{P} \eif \metav{Q} \in \Gamma^{\ast}$ if and only if either $\metav{P}\notin \Gamma^{\ast}$ or $\metav{Q}\in \Gamma^{\ast}$

\item $\metav{P} \eiff \metav{Q} \in \Gamma^{\ast}$ iff either (i) $\metav{P}\in \Gamma^{\ast}$ and $\metav{Q}\in \Gamma^{\ast}$ or (ii) $\metav{P}\notin \Gamma^{\ast}$ and $\metav{Q}\notin \Gamma^{\ast}$

\end{enumerate}

\item Notice how these syntactic constraints mirror truth-conditions!

\item \footnotesize{Moral: We all want to belong, but sometimes our enemies get in the way!}

%people get in the way!}


\end{itemize}
\end{frame}


\begin{frame}
\frametitle{Stage 3 (i): Key Fact aka \emph{The Door} lemma (book's 6.4.9)}
%\large

\begin{itemize}[<+->]

\item To prove the membership lemma's cases (a)--(e), we'll use another lemma (hint: it's lemmas all the way down):
%we'll use a lemma for a lemma

\item \emph{The Door}: if $\Gamma \vdash P$, and $\Gamma^{\ast}$ is a maximally SND-consistent superset of $\Gamma$, then $P \in \Gamma^{\ast}$ \\ (mnemonic: ``$\Gamma\vdash P$" pushes $P$ through the door!) %of our fictional club!
%Bouncer says yesssss)

\item Proof: first, assume that $\Gamma \vdash P$ (we'll use this fact below)

\item Next, assume for \textit{reductio} that $P \notin \Gamma^{\ast}$. Then since $\Gamma^{\ast}$ is maximally SND-consistent, $\Gamma^{\ast} \cup \{ P\}$ must be \textcolor{OGlyallpink}{inconsistent in SND}. 

\item Hence, by negation introduction, $\Gamma^{\ast} \vdash \enot P$

\item By assumption, $\Gamma \vdash P$, so also $\Gamma^{\ast} \vdash P$, since $\Gamma \subseteq \Gamma^{\ast}$

\item So $\Gamma^{\ast}$ derives both $P$ and $\enot P$. \textit{Reductio}! (since $\Gamma^{\ast}$ is M-SND-C)

\item Hence, if $\Gamma \vdash P$ and $\Gamma \subseteq \Gamma^{\ast}$, then $P$ must belong to $\Gamma^{\ast}$ 

\end{itemize}
\end{frame}




\begin{frame}
\frametitle{Membership Lemma: Case (a)}
%\large

\begin{itemize}[<+->]


\item \emph{Case (a)}: $\enot \metav{P} \in \Gamma^{\ast}$ if and only if $\metav{P} \notin \Gamma^{\ast}$

\item Two directions to prove:

\item[] $\Rightarrow$: Assume $\enot \metav{P} \in \Gamma^{\ast}$. Then if  $\metav{P}$ were in $\Gamma^{\ast}$, we could derive contradictory sentences. 

\item[] So since $\Gamma^{\ast}$ is SND-consistent, we must have $\metav{P} \notin \Gamma^{\ast}$

\item[] $\Leftarrow$: Assume $\metav{P} \notin \Gamma^{\ast}$. Then adding $\metav{P}$ to $\Gamma^{\ast}$ results in an SND-inconsistent set. Hence, there is some finite subset $\Delta \subset \Gamma^{\ast}$ s.t. $\Delta \cup \{ \metav{P}\}$ is SND-inconsistent (i.e. derives contradictory sentence pair). 

\item So by negation introduction, $\Delta \vdash \enot \metav{P}$

\item So by The Door lemma, $\enot \metav{P} \in \Gamma^{\ast}$


\end{itemize}
\end{frame}

\begin{frame}
\frametitle{Membership Lemma: Cases (b)-(e)}
%\large

\begin{itemize}[<+->]

\item See the book for cases (b) ($\metav{P} \eand \metav{Q}$) and (d) ($\metav{P} \eif \metav{Q}$)

\item Case (c) is PS12 \#2: $\metav{P} \eor \metav{Q} \in \Gamma^{\ast}$ if and only if either $\metav{P}\in \Gamma^{\ast}$ or $\metav{Q}\in \Gamma^{\ast}$

\item We skip case (e) ($\metav{P} \eiff \metav{Q}$) because \dots \pause \emph{YOLO}



\end{itemize}
\end{frame}

\begin{frame}
\frametitle{Stage 3 (ii): Induction on SL (i.e. we be clubbin')}
%\large

\begin{itemize}[<+->]

\item Goal: construct a TVA \metav{I} that satisfies the M-SND-C set $\Gamma^{\ast}$

\item[] Suffices to construct \metav{I} s.t. $\metav{I}(\metav{Q}) = True$ iff $\metav{Q} \in \Gamma^{\ast}$, $\forall  \metav{Q} \in$ SL. \\ Say that a wff is ``\emph{clubbin'} " whenever it meets this property

\item Define $\metav{I}$ such that $\metav{I}(B) = True$ iff atomic $B \in \Gamma^{\ast}$

\item \emph{Base case}: each atomic wff is true on $\metav{I}$ iff it belongs to $\Gamma^{\ast}$ \\ (i.e. the atomics be clubbin')

\item (Strong) \emph{Induction hypothesis}: assume every SL wff with $1$ to $k$-many connectives is clubbin' 

\item Induction step: show that an arbitrary SL wff with k+1-many connectives is clubbin' 

%, i.e. a wff is true on $\metav{I}$ iff it belongs to $\Gamma^{\ast}$

\end{itemize}
\end{frame}

\begin{frame}
\frametitle{Base Case}
%\large

\begin{itemize}[<+->]

\item Need to show \emph{TWO} directions!: 

\item \emph{Base case}: each atomic wff is true on $\metav{I}$ \emph{iff} it belongs to $\Gamma^{\ast}$

\item Recall that we defined $\metav{I}$ such that $\metav{I}(B) = True$ \emph{iff} atomic $B \in \Gamma^{\ast}$

\item So both directions are met by construction 

\item We proceed to do induction using our SL induction schema: \\ an arbitrary sentence \metav{P} with k+1-many connectives has one of five forms, coming from our five connectives. 

\end{itemize}
\end{frame}

\begin{frame}
\frametitle{Induction on SL: Case 1}
%\large

\begin{itemize}[<+->]

\item \emph{Case 1}: \metav{P} has the form $\enot \metav{Q}$, where since $\metav{Q}$ has $k$-connectives, it is clubbin by the IH (i.e. $\metav{I}(\metav{Q}) = 1$ if and only if $\metav{Q} \in \Gamma^{\ast}$)

%it is clubbin

\item NTS: (i) (the $\Rightarrow$direction) if $\metav{I}(\metav{P}) = True$ then $\metav{P} \in \Gamma^{\ast}$ and \\ 
(ii) (the $\Leftarrow$direction) if $\metav{P} \in \Gamma^{\ast}$, then $\metav{I}(\metav{P}) = True$
\item[] (\textit{Alternative (ii)}: show contrapositive: if $\metav{I}(\metav{P}) = 0$, then $\metav{P} \notin \Gamma^{\ast}$)

\item[$\Rightarrow$] if $\metav{I}(\metav{P}) = 1$, then $\metav{I}(\metav{Q}) = 0$. Since \metav{Q} is clubbin', we have $\metav{Q} \notin \Gamma^{\ast}$. 
\item[] By Membership lemma (a), $\textcolor{red}{\enot \metav{Q}} \in \Gamma^{\ast}$, so $\metav{P} \in \Gamma^{\ast}$

%\item[] Since $\metav{Q}$ has $k$-connectives, by the IH it is clubbin. So 

\item[$\Leftarrow$] if $\metav{P} \in \Gamma^{\ast}$, then $\enot \metav{Q} \in \Gamma^{\ast}$. So by Membership lemma (a), $\metav{Q} \notin \Gamma^{\ast}$. 
\item[] Since \metav{Q} is clubbin', we have $\metav{I}(\metav{Q}) = 0$. \item[] So by the truth conditions for negation, $\metav{I}(\metav{P}) = 1$


\end{itemize}
\end{frame}

\begin{frame}
\frametitle{Induction on SL: Cases 2--5}
%\large

\begin{itemize}[<+->]

\item Need to show: \metav{P} be clubbin', i.e. $\metav{I}(\metav{P}) = True$ iff $\metav{P} \in \Gamma^{\ast}$, \\ where \metav{P} is arbitrary SL wff with k+1-many connectives

\item \alert{Induction hypothesis}: assume every SL wff with $1$ to $k$-many connectives is clubbin' 

\item \emph{Case 2}: \metav{P} has the form $\metav{Q} \eand \metav{R}$

%\item See the book for cases (b) ($\metav{P} \eand \metav{Q}$) and (d) ($\metav{P} \eif \metav{Q}$)

\item \emph{Case 3} is PS12 \#3: \metav{P} has the form $\metav{Q} \eor \metav{R}$

\item Case 4: \metav{P} has the form $\metav{Q} \eif \metav{R}$ (see book p.260!)

\item Case 5: \metav{P} has the form $\metav{Q} \eiff \metav{R}$ (we'll do this case if and only if we accomplish all other goals in our lives)
% (whereof we cannot speak, we must be silent)

%\item We skip case (e) ($\metav{P} \eiff \metav{Q}$) because \dots \pause \emph{YOLO}


\end{itemize}
\end{frame}


\begin{frame}
\frametitle{Reminder for Josh!}
%\large

\begin{itemize}[<+->]

\item If we actually make it this far, give hints on PS12 completeness question ($P \eor Q$)! or do Case (d), which is most analogous 

\item If the people don't want these hints, then clearly they're already complete!
% rather be complete!
%hence, then let's move on to completenes

\item ``The customer is always right!"

\item (Schematize this sentence in quantifier logic)
%The Corporation says: 


\end{itemize}
\end{frame}




\iffalse 

\begin{frame}
\frametitle{Recall SND:}

  \begin{itemize}[<+->]
    \item d
    \emph{d} ($\enot$, $\eor$, $\eand$, $\eif$, $\eiff$)
  
  \begin{block}{blah}
    \begin{itemize}[<+->]
      \item[] d

  \item[] d

  \item[] d
\end{itemize} 
\end{block}

  \begin{definition}
  d
  \end{definition}


\end{itemize}
\end{frame}

\fi 