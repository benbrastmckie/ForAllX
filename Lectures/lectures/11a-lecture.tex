% !TeX root = ./11a-handout.tex

\setcounter{section}{10}

\section{Multiple quantifiers}

\begin{frame}
%\large

\scriptsize{\tableofcontents}

\end{frame}

\subsection{Two quantifiers}

\begin{frame}
  \frametitle{Formulas expressing relations}

\begin{itemize}[<+->]
  \item A formula $\metav{A}\qv{x}$ with one free variable expresses a \emph{property}
  \item A formula $\metav{B}\qr{x}{y}$ with two free variables expresses a \emph{relation}
  \item $\qt{\forall}{x}\qt{\forall}{y}\, \metav{B}\qr{x}{y}$ is a sentence: 
  \item It's true iff 
  \emph{every pair} of objects $\alpha$, $\beta$ stand in the relation expressed by $\metav{B}\qr{x}{y}$.
  \item $\qt{\exists}{x}\qt{\exists}{y}\, \metav{B}\qr{x}{y}$ is a sentence:
  \item It's true iff
  \emph{at least one pair} of objects $\alpha$, $\beta$ stand in the relation expressed by $\metav{B}\qr{x}{y}$.
\end{itemize}
\end{frame}

\begin{frame}
  \frametitle{Multiple uses of a single quantifier: $\forall$}

\begin{itemize}[<+->]
  \item $A\qr{x}{y}$: $x$ admires $y$.
  \item $\qt{\forall}{x}\qt{\forall}{y}\, A\qr{x}{y}$: for every pair $\langle
  \alpha,\beta\rangle$, $\alpha$ admires $\beta$.
  \item In other words: everyone admires everyone.
  \item Note: ``every pair'' includes pairs $\langle\alpha, \alpha\rangle$, i.e.,
  \item $\qt{\forall}{x}\qt{\forall}{y}\, A\qr{x}{y}$ is true only if all pairs $\langle \alert{\alpha, \alpha}\rangle$ satisfy $A\qr{x}{y}$.
  \item That means, everyone admires themselves, in addition to everyone else.
  \item So: $\qt{\forall}{x}\qt{\forall}{y}\, A\qr{x}{y}$ does \emph{not} symbolize ``everyone admires everyone \emph{else}.'' (To handle that, we'll need identity!)
\end{itemize}

\end{frame}

\begin{frame}
  \frametitle{Multiple uses of single quantifier: $\exists$}

\begin{itemize}[<+->]
  \item $\qt{\exists}{x}\qt{\exists}{y}\, A\qr{x}{y}$: for at least one pair $\langle \alpha,\beta\rangle$, $\alpha$ admires $\beta$.
  \item In other words: at least one person admires at least one person.
  \item Note: includes pairs $\langle\alpha, \alpha\rangle$, i.e.,
  \item $\qt{\exists}{x}\qt{\exists}{y}\, A\qr{x}{y}$ is already true if a single pair $\langle \alpha, \alpha\rangle$ satisfies $ A\qr{x}{y}$.
  \item That means, we could just have one person admiring themselves.
  \item So: $\qt{\exists}{x}\qt{\exists}{y}\, A\qr{x}{y}$ does \emph{not} symbolize ``someone admires someone \emph{else}.'' (again, for that, we'll need the identity predicate)
\end{itemize}
\end{frame}

\begin{frame}
    \frametitle{Alternating quantifiers}

\begin{enumerate}[<+->]
  \item $\qt{\forall}{x} \qt{\exists}{y} \, A\qr{x}{y}$
  \item<2->[] Everyone admires someone\\
    (possibly themselves)
  \item $\qt{\forall}{y} \qt{\exists}{x} \, A\qr{x}{y}$
  \item<3->[] Everyone is admired by someone\\
  (not necessarily the same person)
  \item $\qt{\exists}{x} \qt{\forall}{y} \, A\qr{x}{y}$
  \item<4->[] Someone admires everyone\\
  (including themselves)
  \item $\qt{\exists}{y} \qt{\forall}{x} \, A\qr{x}{y}$
  \item<5->[] Someone is admired by everyone\\
    (including themselves)
\end{enumerate}

\end{frame}

\begin{frame}
    \frametitle{Convergence vs. uniform convergence}

\begin{itemize}[<+->]
\item A function $f$ is \emph{point-wise continuous} if
\[
\qt{\forall}{\epsilon}\qt{\forall}{x}\qt{\forall}{y}\qt{\exists}{\delta}(\left|x - y\right| < \delta \to \left|f\qvp{x} - f\qvp{y}\right| < \epsilon)
\]
\item A function $f$ is \emph{uniformly continuous} if
\[
\qt{\forall}{\epsilon} \qt{\exists}{\delta} \qt{\forall}{x}\qt{\forall}{y}(\left|x - y\right| < \delta \to \left|f\qvp{x} - f\qvp{y}\right| < \epsilon)
\]
\end{itemize}

\end{frame}

\subsection{Using quantifiers to express properties}

\begin{frame}
\frametitle{Our symbolization key}

    \begin{ekey}
    \item[$Domain$] people alive in \year{} and items of clothing
    \item[a] Autumn
    \item[g] Greta
    \item[P\qv{x}] \gap{x} is a person
    \item[L\qv{x}] \gap{x} is an item of clothing.
    \item[E\qv{x}] \gap{x} is a cape
    \item[R\qr{x}{y}] \gap{x} wears \gap{y}
    \item[H\qv{x}] \gap{x} is a hero
    \item[I\qv{x}] \gap{x} inspires
    \item[Y\qr{x}{y}] \gap{x} is younger than \gap{y}
    \item[A\qr{x}{y}] \gap{x} admires \gap{y}
    \item[O\qr{x}{y}] \gap{x} owns \gap{y}
    \end{ekey}
\end{frame}


\begin{frame}
  \frametitle{Expressing properties, revisited}
    \begin{itemize}[<+->]
      \item One-place predicates express properties, e.g.,
      \item[] $H\qv{x}$ expresses property ``being a hero''
      \item Combinations of predicates (with connectives, names) can
      express derived properties, e.g.,
      \begin{itemize}[<+->]
        \item[] $A\qr{x}{g}$ expresses ``$x$ admires Greta''
        \item[] $H\qv{x} \eand C\qv{x}$ expresses ``$x$ is a hero who wears a cape''
      \end{itemize}
    \item Using quantifiers, we can express even more complex
    properties, e.g.,
    \item[] $\qt{\exists}{y}(P\qv{y} \eand A\qr{x}{y})$ expresses ``$x$ admires someone''
    \end{itemize}
  \end{frame}
  
  \begin{frame}
    \frametitle{Finding, using properties expressed}
  
  \begin{itemize}[<+->]
    \item If you can say it for Greta, you can say it for $x$.
    \begin{itemize}[<+->]
      \item Greta admires a hero.
      \item[] \alert{$\qt{\exists}{y}(H\qv{y} \eand A\qr{g}{y})$}
      \item $x$ admires a hero.
      \item[] \alert{$\qt{\exists}{y}(H\qv{y} \eand A\qr{x}{y})$}
    \end{itemize}
    \item If you can say it for $x$, you can say it for Greta.
    \begin{itemize}[<+->]
      \item $x$ wears a cape.
      \item[] \alert{$\qt{\exists}{y}(E\qv{y} \eand R\qr{x}{y})$}
      \item Greta wears a cape.
      \item[] \alert{$\qt{\exists}{y}(E\qv{y} \eand R\qr{g}{y})$}
    \end{itemize}
  \end{itemize}
  \begin{ekey}\scriptsize
    \item[E\qv{x}] \gap{x} is a cape
    \item[R\qr{x}{y}] \gap{x} wears \gap{y}
  \end{ekey}
  \end{frame}
  
  \begin{frame}
  \frametitle{Examples}
  
  \begin{itemize}[<+->]
    \item $x$ wears a cape.
    \item[] \alert{$\qt{\exists}{y}(E\qv{y} \eand R\qr{x}{y})$}
    \item $x$ is admired by everyone.
    \item[] \alert{$\qt{\forall}{y}(P\qv{y} \eif A\qr{y}{x})$}
    \item $x$ admires a hero.
    \item[] \alert{$\qt{\exists}{y}(H\qv{y} \eand A\qr{x}{y})$}
    \item $x$ admires only heroes.
    \item[] \alert{$\qt{\forall}{y}(A\qr{x}{y} \eif H\qv{y})$}
    \item $x$ is unclothed (i.e. naked).
    \item[] \alert{$\enot\qt{\exists}{y}(L\qv{y} \eand R\qr{x}{y})$}\\
     \alert{$\qt{\forall}{y}(L\qv{y} \eif \enot R\qr{x}{y})$}
  \end{itemize}
  
  {\scriptsize
  \begin{tabular}{llll}
    $P\qv{x}$ & \gap{x} is a person &
    $L\qv{x}$ & \gap{x} is an item of clothing\\
    $E\qv{x}$ & \gap{x} is a cape &
    $R\qr{x}{y}$ & \gap{x} wears \gap{y}
  \end{tabular}}
  \end{frame}

\subsection{Multiple determiners}

\begin{frame}
\frametitle{``Determiner phrases'' say what?}
%\large

\begin{itemize}[<+->]

\item Determiners: quantifiers and indefinite or definite articles \\ (also possessives and demonstratives)

\item e.g. many, some, a, the, his, their, this, that

\item Determiner phrases: combine a determiner with a (possibily modified) noun:

\item `all heroes'; `a cape'

\item `some woman'; `the donkey'

\end{itemize}
\end{frame}

\begin{frame}
  \frametitle{Symbolizing multiple determiners}

\begin{itemize}[<+->]
\item What if your sentence contains more than one determiner phrase?
\item Deal with each determiner separately!
\item Think of determiner phrase as replaced with name or variable---result has one less determiner.
\item When you're down to one determiner, apply known methods for single quantifiers.
\item This results in formulas that express properties or relations, but themselves contain quantifiers.
\end{itemize}
\end{frame}

\begin{frame}
    \frametitle{Two separate determiner phrases}

\begin{itemize}[<+->]
\item \textcolor{highlightB}{All heroes} wear \textcolor{highlightA}{a cape}
\item \textcolor{highlightB}{All heroes} satisfy ``$x$ wears \textcolor{highlightA}{a cape}'' \[
\textcolor{highlightB}{\qt{\forall}{x}(H\qv{x} \eif {}} \text{``$x$ wears \textcolor{highlightA}{a cape}''})
\]
\item $x$ wears \textcolor{highlightA}{a cape}\[
 \textcolor{highlightA}{\qt{\exists}{y}(E\qv{y} \land{}} R\qr{x}{y})
\]
\item Together:
\[
\textcolor{highlightB}{\qt{\forall}{x}(H\qv{x} \eif {}} \textcolor{highlightA}{\qt{\exists}{y}(E\qv{y} \land{}} R\qr{x}{y}))
\]
\end{itemize}
\end{frame}

\begin{frame}
    \frametitle{Determiner within determiner phrase}

\begin{itemize}[<+->]
\item \textcolor{highlightB}{All heroes who wear \textcolor{highlightA}{ a cape}} admire Greta.
\item All things that satisfy ``$x$ is a hero who wears \textcolor{highlightA}{a cape}'' admire Greta.
\[
\qt{\forall}{x} (\text{``x is a hero who wears a cape''} \eif A\qr{x}{g})
\]
\item $x$ is a hero who wears \textcolor{highlightA}{a cape}
\[
 H\qv{x} \land \qt{\exists}{y}(E\qv{y} \land R\qr{x}{y})
\]
\item Together:
\[
\qt{\forall}{x}((H\qv{x} \land \qt{\exists}{y}(E\qv{y} \land R\qr{x}{y})) \eif A\qr{x}{g})
\]
\end{itemize}
\end{frame}

\begin{frame}
    \frametitle{Mary Astell, 1666--1731}

\begin{columns}
\begin{column}{3cm}
\pgfimage[height=4cm]{../assets/astell}
\end{column}
\begin{column}{7cm}
\begin{itemize}
\item British political philosopher
\item \textit{Some Reflections upon Marriage} (1700)
\item In preface to 3rd ed. 1706 reacts to William Nicholls' claim (in \textit{The Duty of Inferiors
towards their Superiors, in Five Practical Discourses} (London 1701), Discourse IV: The Duty of Wives to their
Husbands), that women are naturally inferior to men.
\end{itemize}
\end{column}
\end{columns}
\end{frame}



\begin{frame}
    \frametitle{Astell TL;DR}

\begin{itemize}
  \item What can Nicholls possibly mean by ``women are naturally inferior to men''?
  \item It can't be that some woman is inferior to some man, since
  that's ``no great discovery.''
  \item After all, surely some men are inferior to some women.
  \item The obviously intended meaning must be: \emph{all} women are
  inferior to \emph{all} men.
  \item But that can't be right, for then ``the greatest Queen ought
  not to command but to obey her Footman.''
  \item It can't even be just: \emph{all} women are inferior to
  \emph{some} men.
  \item Since ``had they been pleased to remember their Oaths of
  Allegiance and Supremacy, they might have known that \textit{One}
  Woman is superior to \textit{All} the Men in these Nations.''
\end{itemize}

\end{frame}

\begin{frame}
    \frametitle{Symbolizing Astell}

\begin{itemize}[<+->]
\item \textcolor{highlightB}{ Some woman} is superior to \textcolor{highlightA}{ every man}
\item \textcolor{highlightB}{ Some woman} satisfies ``$x$ is superior to
\textcolor{highlightA}{ every man}''
\[\textcolor{highlightB}{\qt{\exists}{x}(W\qv{x} \land {}}\text{``$x$ is superior to \textcolor{highlightA}{every man}''})\]
\item $x$ is superior to \textcolor{highlightA}{ every man}
\[
\textcolor{highlightA}{\qt{\forall}{y}(M\qv{y} \eif {}}S\qr{x}{y})
\]
\item Together:
\[
\textcolor{highlightB}{\qt{\exists}{x}(W\qv{x} \land{}} \textcolor{highlightA}{\qt{\forall}{y}(M\qv{y} \to {}}S\qr{x}{y})\textcolor{highlightB}{)}
\]
\end{itemize}
\end{frame}

\begin{frame}
    \frametitle{Formalizing Astell}

\begin{itemize}[<+->]
\item Some woman is superior to some man.
\item[] \alert{$\qt{\exists}{x}(W\qv{x} \land \qt{\exists}{y}(M\qv{y} \land S\qr{x}{y}))$}
\item Every woman is superior to every man.
\item[] \alert{$\qt{\forall}{x}(W\qv{x} \to \qt{\forall}{y}(M\qv{y} \to S\qr{x}{y}))$}
\item Every woman is superior to some man.
\item[]\alert{$\qt{\forall}{x}(W\qv{x} \to \qt{\exists}{y}(M\qv{y} \land S\qr{x}{y}))$}
\item Some woman is superior to every man.
\item[] \alert{$\qt{\exists}{x}(W\qv{x} \land \qt{\forall}{y}(M\qv{y} \to S\qr{x}{y}))$}
\end{itemize}
\end{frame}



\begin{frame}
    \frametitle{``Any'': sometimes existential}

\begin{itemize}[<+->]
\item Any (every) cape is worn by a hero:
\[
\uncover<2->{\qt{\forall}{x}(E\qv{x} \eif \qt{\exists}{y}(H\qv{y} \land R\qr{y}{x}))}
\]\pause
\item No hero wears any cape:
\begin{align*}
\uncover<3->{\qt{\forall}{x}(H\qv{x} & \eif \enot\qt{\exists}{y}(E\qv{y} \land R\qr{x}{y}))} {}\\
\uncover<4->{\enot \qt{\exists}{x}(H\qv{x} & \eand \qt{\exists}{y}(E\qv{y} \land R\qr{x}{y}))}
\end{align*}
\item No hero wears every cape:
\begin{align*}
\uncover<5->{\qt{\forall}{x}(H\qv{x} & \eif \enot\qt{\forall}{y}(E\qv{y} \eif R\qr{x}{y}))}{}\\
\uncover<6->{\enot\qt{\exists}{x}(H\qv{x} & \eand \qt{\forall}{y}(E\qv{y} \eif R\qr{x}{y}))}
\end{align*}
\end{itemize}
\end{frame}

\subsection{Quantifier scope ambiguity}

\begin{frame}
  \frametitle{More scope ambiguity}

\begin{itemize}[<+->]
\item ``Autumn and Greta admire Isra or Luisa.''
\item Two logically distinct, natural readings:
\item[1)] Autumn admires Isra or Luisa, \emph{and} so does Greta.
\begin{align*}
(A\qr{a}{i} \lor {} & A\qr{a}{l}) \emph{\land} {}\\
(A\qr{g}{i} \lor {} & A\qr{g}{l})
\end{align*}
\item[2)] Autumn and Greta both admire Isra, \textcolor{highlightB}{or} they both admire Luisa.
\begin{align*}
(A\qr{a}{i} \land {} & A\qr{g}{i}) \textcolor{highlightB}{\lor} {}\\
(A\qr{a}{l} \land {} & A\qr{g}{l})
\end{align*}
\end{itemize}

\end{frame}


\begin{frame}
    \frametitle{Negation and the quantifiers}

\begin{itemize}[<+->]
\item ``All heroes don't inspire''
\begin{itemize}[<+->]
\item Denial of ``all heroes inspire''. Ask: ``Do all heroes inspire\\
(Answer: No, \textit{it's not the case that} all heroes inspire '')\pauses %No, all heroes don't inspire
\begin{align*}
\lnot\qt{\forall}{x}(H\qv{x} & {} \to I\qv{x}) \\
\qt{\exists}{x}(H\qv{x} & {} \land\lnot I\qv{x})
\end{align*}
\item All heroes are not inspiring, i.e.,\\
No heroes inspire\pauses
\begin{align*}
\qt{\forall}{x}(H\qv{x} & {} \to \lnot I\qv{x}) \\
\lnot\qt{\exists}{x}(H\qv{x} & {}\land I\qv{x})
\end{align*}
\end{itemize}
\end{itemize}
\end{frame}

\begin{frame}
    \frametitle{Multiple quantifiers and ambiguity}

\begin{itemize}[<+->]
\item ``All heroes wear a cape''
\begin{itemize}[<+->]
\item ``A cape'' in the scope of ``all heroes'', i.e.,\\
``For every hero, there is a cape they wear''\pauses
\[
\qt{\forall}{x}(H\qv{x} \to \qt{\exists}{y}(E\qv{y} \land R\qr{x}{y}))
\]
\[
\qt{\forall}{x}\qt{\exists}{y}(H\qv{x} \to (E\qv{y} \land R\qr{x}{y}))
\]
\item ``All heroes'' in scope of ``a cape'', i.e.,\\
``There is a cape which every hero wears''\pauses
\[
\qt{\exists}{y}(E\qv{y} \land \qt{\forall}{x}(H\qv{x} \to R\qr{x}{y}))
\]
\[
\qt{\exists}{y}\qt{\forall}{x}(E\qv{y} \land (H\qv{x} \to R\qr{x}{y}))
\]
\end{itemize}

\item A (probably bad) joke: ````Every day, a tourist is mugged on the streets of New York. He's going through a lot of wallets.'' %how does he keep getting the funds to buy a new wallet? kind people are donating their wallets to him. 

%\item Compare the joke: ``Every day, a tourist is mugged on the streets of New York. We will interview him tonight.'' %%JH: no idea how this is a joke??? is it a riff on `every day' vs. `tonight'?
%or is the riff on scope ambiguity: joke interpt is that there is a particular tourist who is mugged every day. so i tried to improve the joke to make it possibly funny? 
\end{itemize}
\end{frame}


\subsection{Donkey sentences}

\begin{frame}
    \frametitle{Happy farmers}

``Every farmer who owns a donkey is happy''

\begin{itemize}[<+->]
\item Step-by-step symbolization: ``All $A$s are $B$s''
\item $x$ is a farmer who owns a donkey \dots\[
F\qv{x} \land \qt{\exists}{y}(D\qv{y} \land O\qr{x}{y})
\]
\item \textcolor{highlightA}{Every} farmer who owns a donkey \textcolor{highlightA}{is happy}
\[
\textcolor{highlightA}{\qt{\forall}{x}(}\textcolor{OGlyallpink}{(}F\qv{x} \land \qt{\exists}{y}(D\qv{y} \land O\qr{x}{y})\textcolor{OGlyallpink}{)} \textcolor{highlightA}{\eif H\qv{x})}
\]
\item Notice how `a donkey' is bound by an existential here
\end{itemize}
\end{frame}

\begin{frame}
    \frametitle{Unhappy donkeys :( }

``Every farmer who owns a donkey beats it''

\begin{itemize}[<+->]
\item Step-by-step symbolization: ``All As are Bs''
\item $x$ is a farmer who owns a donkey \dots\[
F\qv{x} \land \qt{\exists}{y}(D\qv{y} \land O\qr{x}{y})
\]
\item \textcolor{highlightA}{Every} farmer who owns a donkey \textcolor{highlightA}{beats} \textcolor{highlightB}{it}:
\[
\textcolor{highlightA}{\qt{\forall}{x}(}\textcolor{OGlyallpink}{(}F\qv{x} \land \qt{\exists}{y}(D\qv{y} \land O\qr{x}{y})\textcolor{OGlyallpink}{)} \textcolor{highlightA}{\to B\qr{x}{\textcolor{highlightB}{y}})}
\]
\item PROBLEM: `\textcolor{highlightB}{y}' is unbound! So this is not a QL sentence. Gasp! 
\end{itemize}
\end{frame}

\begin{frame}
\frametitle{Save the donkeys: a failed attempt}
%\large

\begin{itemize}[<+->]

\item This was our problem: a donkey lay beaten and \textcolor{highlightB}{unbound}:
\[
\textcolor{highlightA}{\qt{\forall}{x}(}\textcolor{OGlyallpink}{(}F\qv{x} \land \qt{\exists}{y}(D\qv{y} \land O\qr{x}{y})\textcolor{OGlyallpink}{)} \textcolor{highlightA}{\to B\qr{x}{\textcolor{highlightB}{y}})}
\]

\item Can we simply extend the scope of the existential? 
\[
\textcolor{highlightA}{\qt{\forall}{x}}\textcolor{highlightB}{\qt{\exists}{y}(}\textcolor{OGlyallpink}{(}F\qv{x} \land (D\qv{y} \land O\qr{x}{y})\textcolor{OGlyallpink}{)} \textcolor{highlightA}{\to B\qr{x}{\textcolor{highlightB}{y)}}}
\]

\item `y' is now bound, but alas, this sentence is trivially true:

\item Provided at least one thing in our UD is not a donkey, that thing makes the antecedent of the conditional false, making the conditional trivially true, for any $x$. 

\item[] In particular, our farmer is not a donkey.
\item[] But he still sounds like kind of a jack@\$\$! 





\end{itemize}
\end{frame}

\begin{frame}
    \frametitle{Symbolizing donkey sentences}

``Every farmer who owns a donkey beats it''\pauses

\begin{itemize}[<+->]
\item When is it false that every farmer who owns a donkey beats it? \pauses
If there's a farmer who owns a donkey but doesn't beat it. Deny that! \pauses
\[
\alert{\lnot\qt{\exists}{x}(F\qv{x} \land \qt{\exists}{y}(D\qv{y} \land O\qr{x}{y} \land \lnot B\qr{x}{y}))}
\]
\item For every farmer and every donkey they own: the farmer beats the donkey.
\[
\alert{\qt{\forall}{x}\qt{\forall}{y}((F\qv{x} \land (D\qv{y} \eand O\qr{x}{y})) \to B\qr{x}{y})}
\]
\item Every farmer beats every donkey they own.
\[
\alert{\qt{\forall}{x}(F\qv{x} \to \qt{\forall}{y}((D\qv{y} \land O\qr{x}{y}) \to B\qr{x}{y}))}
\]
%%JH: but still seems like we're limited: I can here the reading where the sentence means the farmer beats only one of his donkies!
\item But what about the case where at least one farmer with a donkey beats only one of his donkeys? $\#$Quitting
\end{itemize}
\end{frame}
