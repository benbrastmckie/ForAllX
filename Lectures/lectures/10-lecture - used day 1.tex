% !TeX root = ./10-handout-old.tex

\setcounter{section}{9}

% %note to josh: think about using zach tutorial 7 as part of week 10 lecture material, along w/ stuff from GB on natural deduction! and also stuff i wrote for 303 discussion section 

\section{Proofs in QL}

\begin{frame}
%\large

\scriptsize{\tableofcontents}

\end{frame}

\subsection{Some Equivalences in QL}

\begin{frame}
\frametitle{Motivation: Proving Equivalences}
%\large

\begin{itemize}[<+->]

\item We'd like to have a method for proving that two sentences are equivalent, or that an argument is valid 
%along with a method for showing valid

\item Models only provide counterexamples to equivalence or validity

\item By extending our natural deduction system to cover QL, we'll be able to prove two sentences are equivalent! 

\bi

\item Derive one from the other and vice versa

\item Basically: show that the biconditional of the two sentences is a tautology 

\ei


\end{itemize}
\end{frame}

\begin{frame}
\frametitle{Some Equivalences: Quantifiers commuting over connectives}
%\large

\begin{enumerate}[<+->]

\item $F\qv{a} \eand \qt{\exists}{x}G\qv{x}$ is equivalent to $\qt{\exists}{x}(F\qv{a} \eand G\qv{x})$

\item $F\qv{a} \eand \qt{\forall}{x}G\qv{x}$ is equivalent to $\qt{\forall}{x}(F\qv{a} \eand G\qv{x})$

\item $F\qv{a} \eor \qt{\exists}{x}G\qv{x}$ is equivalent to $\qt{\exists}{x}(F\qv{a} \eor G\qv{x})$

\item $F\qv{a} \eor \qt{\forall}{x}G\qv{x}$ is equivalent to $\qt{\forall}{x}(F\qv{a} \eor G\qv{x})$

\item $F\qv{a} \eif \qt{\exists}{x}G\qv{x}$ is equivalent to $\qt{\exists}{x}(F\qv{a} \eif G\qv{x})$

\item $F\qv{a} \eif \qt{\forall}{x}G\qv{x}$ is equivalent to $\qt{\forall}{x}(F\qv{a} \eif G\qv{x})$

\item[] -- But note that `$\eif$' is not symmetric, so we have to examine the converse: e.g. $\qt{\exists}{x}G\qv{x} \eif F\qv{a}$!

\item[] -- And there are no analogous rules for biconditionals 


\end{enumerate}
%ask what happens if we put $F\qv{a}$ in the consequent, 
\end{frame}

\begin{frame}
\frametitle{Quantifiers NOT commuting over some conditionals}
%\large

`$\eif$' is not symmetric, so we have to examine the converse %: e.g. $\qt{\exists}{x}G\qv{x}$ \eif F\qv{a}$!

\begin{enumerate}[<+->]

\item[bad:] $\qt{\exists}{x}G\qv{x} \eif F\qv{a}$ is NOT equivalent to $\qt{\exists}{x}(G\qv{x} \eif F\qv{a})$

\item[] Instead, the existential converts to a universal when we scope over the consequent:

\item[7.] $\qt{\exists}{x}G\qv{x} \eif F\qv{a}$ is equivalent to  $\qt{\forall}{x}(G\qv{x} \eif F\qv{a})$

\item[bad:] $\qt{\forall}{x}G\qv{x} \eif F\qv{a}$ is NOT equivalent to $\qt{\forall}{x}(G\qv{x} \eif F\qv{a})$

\item[] Instead, the universal converts to an existential when we scope over the consequent:

\item[8.] $\qt{\forall}{x}G\qv{x} \eif F\qv{a}$ is equivalent to  $\qt{\exists}{x}(G\qv{x} \eif F\qv{a})$


\end{enumerate}

%ask what happens if we put $F\qv{a}$ in the consequent, 
\end{frame}


\begin{frame}
\frametitle{Informal Argument: If anyone G's, then c F's}
%\large

\begin{itemize}[<+->]

\item[7.] $\qt{\exists}{x}G\qv{x} \eif F\qv{c}$ is equivalent to $\qt{\forall}{x}(G\qv{x} \eif F\qv{c})$

\item Recall: ``Any'' in antecedents but \emph{without} pronouns referring back to
  them are \emph{existential}:
  \begin{itemize}[<+->]
    \item[] If \emph{anyone} is a hero, Greta is.
    \item[] Roughly: if there are heroes (at all), Greta is a hero.
    \item[] \alert{$\qt{\exists}{x}\, H\qv{x} \eif H\qv{g}$}
    %need to check restrictions on parentheses here! recall some tricky stuff...
  \end{itemize}

%\item Gloss: If anyone G's, then c F's

\item Intuitively: for everybody, if they are a hero, then Greta is a hero

\bi
 \item[] \alert{$\qt{\forall}{x}(H\qv{x} \eif H\qv{g})$}
\ei

\item We would like to show that if we can derive one of these sentences, then we can derive the other. 

\item[] Our proof system will show this \textit{naturally}!


\end{itemize}
\end{frame}

\begin{frame}
\frametitle{Informal Argument: If everyone G's, then c F's}
%\large

%note to Josh: this is related to hw problem 8.6:

%likewise, explain why the following solN to 8.6 does NOT work:

%(Ax)((Gx& Jc) > Pa)
%If everyone graduates and Carol gets a job, then Alice will become a physician. : correct answer: (((\forall x)Gx & Jc) > Pa)

\begin{itemize}[<+->]

\item[8.] $\qt{\forall}{x}G\qv{x} \eif F\qv{c}$ is equivalent to  $\qt{\exists}{x}(G\qv{x} \eif F\qv{c})$

\item By truth conditions for `$\eif$', first sentence is equivalent to:

\item[] $\enot \qt{\forall}{x}G\qv{x} \eor F\qv{c}$

\item[] Apply de Morgan's: $\qt{\exists}{x}\enot G\qv{x} \eor F\qv{c}$

\item Now apply our rule 3: existential commutes over disjunction (provided the disjunct doesn't contain the bound variable $x$)

\item[] $\qt{\exists}{x}(\enot G\qv{x} \eor F\qv{c})$, and apply `$\eif$' truth conditions again:

\item Yields $\qt{\exists}{x}(G\qv{x} \eif F\qv{c})$

\end{itemize}
\end{frame}




\subsection{Rules for $\forall$}

\begin{frame}
\frametitle{Rules for formal proofs}

\begin{itemize}[<+->]
\item Need rules for $\forall$ and $\exists$ for formal proofs
\item Formal proofs now more important, because no alternative
  (truth-table method)
\item Intro and Elim rules should be
\begin{itemize}[<+->]
\item simple
\item elegant (not involve other connectives or quantifiers)
\item yield only valid arguments
\end{itemize}
\end{itemize}

\end{frame}

\begin{frame}
\frametitle{Candidates for rules}

\begin{itemize}[<+->]
\item Only simple sentence close to $\qt{\forall}{x}\, \metav{A}\qv{\script{x}}$ is
$\metav A\qv{\script{c}}$
\item Gives simple, elegant $\forall$E rule:

\begin{fitchproof}
\have[k]{a}{\qt{\forall}{\script{x}}\, \metav A\qv{\script{x}}}
\have[ ]{b}{\metav A\qv{\script{c}}}\Ae{a}
	%\have[m]{a}{\qt{\forall}{\script{x}}\metaA{}(\dots \script{x} \dots \script{x} \dots)}
	%\have [\vdots] {n} {\hspace{2em} \vdots}
	%\have[s]{c}{\metaA{}(\dots \script{c} \dots \script{c} \dots)} \Ae{a}
	%\have[\ ]{c}{\metaA{}\substitute{\script{x}}{\script{c}}} \Ae{a}
\end{fitchproof}

\bigskip

\item This is a good rule: $\qt{\forall}{\script{x}}\,\metav A\qv{\script{x}}
\models \metav A\qv{\script{c}}$.
\end{itemize}
\end{frame}

\begin{frame}
  \frametitle{Candidates for rules}

\begin{itemize}[<+->]
\item Problem: corresponding ``intro rule'' isn't valid:
\begin{fitchproof}
\have[k]{a}{\metav A\qv{\script{c}}}
\have[ ]{b}{\qt{\forall}{\script{x}}\,\metav A\qv{\script{x}}}\by{\emph{(doesn't follow)}}{a}
\end{fitchproof}
\item Diagnosis: the $\script{c}$ in $\metav A\qv{\script{c}}$ is a name for
a \emph{specific object}.
\item We need a name for an \emph{arbitrary,
unspecified object}.
\item If $\metav A\qv{\script{c}}$ is true for whatever $\script{c}$
\emph{could} name, then $\metav A\qv{\script{x}}$ is satisfied by \emph{every} object.
\end{itemize}
\end{frame}

\begin{frame}
\frametitle{Names for arbitrary objects}

\begin{itemize}[<+->]
\item When we give proofs of general claims, we often do use names for
arbitrary objects (well, mathematicians do at least).

\begin{earg}
\item[] All heroes admire Greta.
\item[] Only people who wear capes admire Greta.
\item[\therefore] All heroes wear capes.
\end{earg}
\pause
Proof: Let Carl be any hero.  \\ Since all heroes admire Greta, Carl
admires Greta. \\ Since only people who wear capes admire Greta, Carl wears a cape. But ``Carl'' stands for \emph{any} hero. \\ So all heroes
wear capes.
\end{itemize}
\end{frame}

\begin{frame}
  \frametitle{Universal generalization}

\begin{fitchproof}
  \have[k]{a}{\metav{A}\qv{\script{c}}}
  \have[ ]{b}{\qt{\forall}{\script{x}}\,\metav{A}\qv{\script{x}}}\Ai{a}
\end{fitchproof}
  \begin{itemize}[<+->]
  \item $\script{c}$ is special: $\script{c}$ must not appear in any
  premise or assumption of a subproof not already ended
  \item $\metav{A}\qv{\script{x}}$ is obtained from $\metav{A}\qv{\script{c}}$ by replacing \emph{all} occurrences of $\script{c}$ by $\script{x}$.
  \item In other words, $\script{c}$ must also not occur in $\forall
  \script{x}\,\metav{A}\qv{\script{x}}$.
 \end{itemize}
\end{frame}

\begin{frame}
\frametitle{General conditional proof}

Proving ``All $A$s are $B$s''

\begin{fitchproof}
  \open
  \hypo[k]{a}{A\qv{c}} \as{for $\eif$I}
  \have[l]{b}{B\qv{c}}
  \close
  \have{c}{A\qv{c} \eif B\qv{c}}\ci{a-b}
  \have{d}{\qt{\forall}{x}(A\qv{x} \eif B\qv{x})}\Ai{c}
\end{fitchproof}
\end{frame}

\begin{frame}
\frametitle{Example}

\begin{earg}
  \item[] All heroes admire Greta.
  \item[] Only people who wear capes admire Greta.
  \item[\therefore] All heroes wear capes.
\end{earg}

\bigskip

\begin{earg}
  \item[] $\qt{\forall}{x}(H\qv{x} \eif A\qr{x}{g})$
  \item[] $\qt{\forall}{x}(A\qr{x}{g} \eif C\qv{x})$
  \item[\therefore] $\qt{\forall}{x}(H\qv{x} \eif C\qv{x})$
\end{earg}

Let's do it \href{https://tinyurl.com/4dst762n}{on Carnap (PP10.2)}!

%\href{https://carnap.io/shared/rzach@ucalgary.ca/Practice\%20Problems\%20VII.md}{carnap.io}
\end{frame}

\begin{frame}
\frametitle{Example}
\small
\begin{fitchproof}
  \hypo{a}{\qt{\forall}{x}(H\qv{x} \eif A\qr{x}{g})} \pr{}
  \hypo{b}{\qt{\forall}{x}(A\qr{x}{g} \eif C\qv{x})} \pr{}
  \open
  \hypo{c}{H\qv{c}} \as{for $\eif$I}
  \have{d}{H\qv{c} \eif A\qr{c}{g}}\Ae{a}
  \have{e}{A\qr{c}{g}}\ce{d,c}
  \have{f}{A\qr{c}{g} \eif C\qv{c}}\Ae{b}
  \have{g}{C\qv{c}}\ce{f,e}
  \close
  \have{h}{H\qv{c} \eif C\qv{c}}\ci{c-g}
  \have{i}{\qt{\forall}{x}(H\qv{x} \eif C\qv{x})}\Ai{h}
\end{fitchproof}
\end{frame}

\begin{frame}
  \frametitle{Example}
  \small
  \begin{fitchproof}
    \hypo{1}{\qt{\forall}{x}\,A\qv{x} \eor \qt{\forall}{x}\, B\qv{x}} \pr{}
    \open
    \hypo{2}{\qt{\forall}{x}\,A\qv{x}} \as{for $\lor$E}
    \have{3}{A\qv{c}}\Ae{2}
    \have{4}{A\qv{c} \eor B\qv{c}}\oi{3}
\close
\open
\hypo{5}{\qt{\forall}{x}\,B\qv{x}} \as{for $\lor$E}
\have{6}{B\qv{c}}\Ae{5}
\have{7}{A\qv{c} \eor B\qv{c}}\oi{6}
\close
  \have{8}{A\qv{c} \eor B\qv{c}}\oe{1,2-4,5-7}
  \have{9}{\qt{\forall}{x}(A\qv{x} \eor B\qv{x})}\Ai{8}
  \end{fitchproof}
  \end{frame}

\subsection{Rules for $\exists$}

\begin{frame}
\frametitle{Rules for $\exists$}

\begin{itemize}[<+->]
\item If we know of a specific object that it satisfies $\metav A\qv{x}$, we know that at least one object satisfies $\metav A\qv{x}$.
\item So this rule is valid:
\begin{fitchproof}
  \have[k]{a}{\metav A\qv{\script{c}}}
  \have[ ]{b}{\qt{\exists}{\script{x}}\, \metav A\qv{\script{x}}}\Ei{a}
\end{fitchproof}
\end{itemize}
\end{frame}

\begin{frame}
\frametitle{Arbitrary objects again}

\begin{itemize}
  \item Problem: corresponding ``elimination rule'' isn't valid:
  \bigskip
  \begin{fitchproof}
    \have[k]{a}{\qt{\exists}{\script{x}}\, \metav A\qv{\script{x}}}
    \have[ ]{b}{\metav A\qv{\script{c}}}\by{\emph{doesn't follow from}}{a}
  \end{fitchproof}
  \bigskip
  \item If we know that $\qt{\exists}{\script{x}}\, \metav A\qv{\script{x}}$ is
  true, we know that \emph{at least one} object satisfies $\metav A\qv{\script{x}}$, but not which one(s).
\item To use this information, we have to introduce a temporary name
that stands for any one of the objects that satisfy $\metav A(\script{x})$.
\end{itemize}
\end{frame}

\begin{frame}
  \frametitle{Reasoning from existential information}

  \begin{itemize}
    \item To use $\qt{\exists}{\script{x}}\,\metav A\qv{\script{x}}$, pretend the
    $\script{x}$ has a name~$\script{c}$, and reason from $\metav A(\script{c})$.
    \item This is what we do to reason informally from existential information, e.g.,
  \begin{earg}
  \item[] There are heroes who wear capes.
  \item[] Anyone who wears a cape admires Greta.
  \item[\therefore] Some heroes admire Greta.
  \end{earg}

\pause
Proof: We know there are heroes who wear capes. \\ Let Cate be an
arbitrary one of them. \\ So Cate wears a cape. Since anyone who wears a
cape admires Greta, Cate admires Greta. Since Cate is a hero who
admires Greta, some heroes admire Greta.
\end{itemize}

\end{frame}

\begin{frame}
\frametitle{Existential elimination (mind the \emph{restriction}!)}

\begin{itemize}[<+->]
  \item If
  \begin{itemize}[<+->]
    \item we know that some object satisfies $\metav A\qv{\script{x}}$,
    \item we hypothetically assume that $c$ is one of them (i.e.,
  assume $\metav A\qv{\script{c}}$), 
    \item and we can prove that $\metav B$ follows from this assumption,
\end{itemize}
\item[] then $\metav B$ follows already from $\qt{\exists}{\script{x}}\, \metav A\qv{\script{x}}$.
\item Rule for existential elimination:

\begin{fitchproof}
  \have[k]{a}{\qt{\exists}{\script{x}}\, \metav A\qv{\script{x}}}
  \open
  \hypo[m]{b}{\metav A\qv{\script{c}}} \as{for $\exists$E}
  \have[n]{c}{\metav B}
\close
\have[ ]{d}{\metav B}\Ee{a,b-c}
\end{fitchproof}
\medskip
\item $\script{c}$ is special: $\script{c}$ \emph{must NOT appear outside subproof}
\end{itemize}
\end{frame}

\begin{frame}
\frametitle{Example}

\begin{earg}
  \item[] There are heroes who wear capes.
  \item[] Anyone who wears a cape admires Greta.
  \item[\therefore] Some heroes admire Greta.
\end{earg}
\bigskip
\begin{earg}
  \item[] $\qt{\exists}{x}(H\qv{x} \eand C\qv{x})$
  \item[] $\qt{\forall}{x}(C\qv{x} \eif A\qr{x}{g})$
  \item[\therefore] $\qt{\exists}{x}(H\qv{x} \eand A\qr{x}{g})$
\end{earg}
\end{frame}

\begin{frame}
\frametitle{Example (PP10.5)}
\footnotesize
\begin{fitchproof}
  \hypo{a}{\qt{\exists}{x}(H\qv{x} \eand C\qv{x})} \pr{}
  \hypo{b}{\qt{\forall}{x}(C\qv{x} \eif A\qr{x}{g}} \pr{}
  \open
  \hypo{c}{H\qv{c} \eand C\qv{c}} \as{for $\exists$E}
  \have{d}{C\qv{c}}\ae{c}
  \have{e}{C\qv{c} \eif A\qr{c}{g}}\Ae{b}
  \have{f}{A\qr{c}{g}}\ce{d,e}
  \have{g}{H\qv{c}}\ae{c}
  \have{h}{H\qv{c} \eand A\qr{c}{g}}\ai{d,g}
  \have{i}{\qt{\exists}{x}(H\qv{x} \eand A\qr{x}{g})}\Ei{h}
  \close
  \have{j}{\qt{\exists}{x}(H\qv{x} \eand A\qr{x}{g})}\Ei{a,c-i}
\end{fitchproof}
\end{frame}

