% !TeX root = ./slides-04.tex

\setcounter{section}{3}

\section{Proofs in SL}

\subsection{Why proofs?}

\begin{frame}
  \frametitle{Showing that an argument is valid}

  \begin{itemize}[<+->]
    \item Construct a truth table; verify there is no valuation where
    premises are true and conclusion is false.
    \item Truth tables can get very large very quickly.
    \item E.g., the example argument
      \[
      C \eor E, A \eor M, A \eif \enot C, \enot S \eand \enot M \therefore E
      \]
      requires 32 lines and 608 individual truth values.
    \item Is there a better way?
  \end{itemize}
\end{frame}

\begin{frame}
  \frametitle{Proofs}

  \begin{itemize}[<+->]
    \item Idea: work our way from premises to conclusion using steps
    we know are entailed by the premises.
    \item For instance:
      \begin{itemize}[<+->]
        \item From ``Neither Sarah nor Amir enjoys hiking'' we can
        conclude ``Amir doesn't enjoy hiking.''
        \item From ``Either Amir lives in Chicago or he enjoys hiking''
        and ``Amir doesn't enjoy hiking'' we can conclude ``Amir
        lives in Chicago'' (Disjunctive syllogism DS).
        \item etc.
      \end{itemize}
    \item If we manage to work from the premises to the conclusion in
    this way, we know that the argument must be valid.
  \end{itemize}

\end{frame}

\begin{frame}
  \frametitle{An informal proof}

  \begin{block}{Our argument}
  1. Sarah lives in Chicago or Erie.\\
  2. Amir lives in Chicago unless he enjoys hiking.\\
  3. If Amir lives in Chicago, Sarah doesn't.\\
  4. Neither Sarah nor Amir enjoy hiking.\\
  $\therefore$ Sarah lives in Erie.
  \end{block}

  \begin{enumerate}[<+->]
    \item[5.] Amir doesn't enjoy hiking (from 4).
    \item[6.] Amir lives in Chicago (from 2 and 5).
    \item[7.] Sarah doesn't live in Chicago (from 3 and 6).
    \item[8.] Sarah lives in Erie (from 1 and 7).
  \end{enumerate}

\end{frame}

\begin{frame}
  \frametitle{A more formal proof}

  \begin{block}{Our argument}
    1. \alert<4>{$C \eor E$}\\
    2. \alert<2>{$A \eor M$}\\
    3. \alert<3>{$A \eif \enot C$}\\
    4. \alert<1>{$\enot S \eand \enot M$} \\
  $\therefore$ $E$
  \end{block}

  \begin{enumerate}[<+->]
    \item[5.] \alert<1,2>{$\enot M$} (from 4, since $\metav{P} \eand \metav{Q}
    \entails \metav{Q}$)
    \item[6.] \alert<2,3>{$A$} (from 2 and 5, since $\metav{P} \eor \metav{Q}, \enot \metav{Q} \entails \metav{P}$)
    \item[7.] \alert<3,4>{$\enot C$} (from 3 and 6, since $\metav{P} \eif \metav{Q}, \metav{P}
    \entails \metav{Q}$)
    \item[8.] \alert<4>{$E$} (from 1 and 7, since $\metav{P} \eor \metav{Q}, \enot \metav{P}
    \entails \metav{Q}$)
  \end{enumerate}

\end{frame}

\begin{frame}
  \frametitle{Formal proofs}

  \begin{itemize}[<+->]
    \item Numbered lines containing sentences of SL.
    \item A line may be a \emph{premise}.
    \item If it's not a premise, it must be \emph{justified}.
    \item Justification involves:
      \begin{itemize}
        \item a \emph{rule}, and
        \item prior lines (referred to by line number).
      \end{itemize}
    \item But: what's a rule?
  \end{itemize}
\end{frame}

\subsection{Rules for \eand}

\begin{frame}
  \frametitle{Rules of natural deduction}

  \begin{itemize}[<+->]
    \item Rules should be \dots
      \begin{itemize}[<+->]
        \item \emph{Simple}: cite just a few lines as justification.
        \item \emph{Obvious}: justified line should clearly be entailed by justifications.
        \item \emph{Schematic}: can be described just by \emph{forms} of
        sentences involved.
        \item \emph{Few in number}: want to make do with just a handful.
      \end{itemize}
    \item We'll have two rules per connective: an \emph{introduction}
    and an \emph{elimination} rule.
    \item They'll be used to either:
    \begin{itemize}[<+->]
      \item justify (say) $\metav{P} \eand \metav{Q}$
    (introduce \eand), or 
    \item justify something \emph{using} $\metav{P}
    \eand \metav{Q}$ (eliminate \eand).
    \end{itemize}
  \end{itemize}
\end{frame}

\begin{frame}
  \frametitle{Eliminating \eand}

  \begin{itemize}[<+->]
    \item What can we \emph{justify using} $\metav{P} \eand \metav{Q}$?
    \item A conjunction entails each conjunct:
    \begin{align*}
      & \metav{P} \eand \metav{Q} \entails \metav{P}\\
      & \metav{P} \eand \metav{Q} \entails \metav{Q}
    \end{align*}
    \item Already used this above to get $\enot M$ from $\enot S \eand
    \enot M$, i.e., from ``Neither
    Sarah nor Amir enjoys hiking'' we concluded\\
 ``Amir doesn't enjoy hiking''.
    \item (Role of $\metav{P}$ played by $\enot S$ and that of $\metav{Q}$
    played by $\enot M$.)
  \end{itemize}
\end{frame}

\begin{frame}
  \frametitle{Introducing \eand}

  \begin{itemize}[<+->]
    \item What do we \emph{need to justify} $\metav{P} \eand \metav{Q}$?
    \item We need both \metav{P} and \metav{Q}:
    \begin{align*}
      & \metav{P}, \metav{Q} \entails \metav{P}\eand \metav{Q}
    \end{align*}
    \item For instance, if we have ``Sarah doesn't enjoy hiking'' and
    also ``Amir doesn't enjoy hiking'', we can conclude\\
    ``Neither
    Sarah nor Amir enjoys hiking''.
    \item (Role of $\metav{P}$ played by $\enot S$ and $\metav{Q}$
    played by $\enot M$: $\enot S, \enot M \entails \enot S \eand
    \enot M$.)
  \end{itemize}
\end{frame}

\begin{frame}
  \frametitle{Rules for \eand}
  \begin{columns}
    \begin{column}{3cm}
  \begin{fitchproof}
    \have[m]{a}{\metav{P}}
    \have[n]{b}{\metav{Q}}
    \have[\ ]{c}{\metav{P}\eand\metav{Q}} \ai{a, b}
  \end{fitchproof}
\end{column}
\begin{column}{3cm}
  \begin{fitchproof}
    \have[m]{ab}{\metav{P}\eand\metav{Q}}
    \have[\ ]{a}{\metav{P}} \ae{ab}
  \end{fitchproof}
  \begin{fitchproof}
    \have[m]{ab}{\metav{P}\eand\metav{Q}}
    \have[\ ]{b}{\metav{Q}} \ae{ab}
  \end{fitchproof}
\end{column}
\end{columns}
We'll illustrate using the exercises in \href{https://carnap.io/shared/rzach@ucalgary.ca/Practice\%20Problems\%20IV.md}{Carnap}.
\end{frame}

\begin{frame}
  \begin{fitchproof}
    \hypo{1}{A \eand B}
    \have{2}{A}\ae{1}
    \have{3}{B}\ae{1}
    \have{4}{B \eand A}\ai{2,3}
  \end{fitchproof}
\end{frame}

\begin{frame}
  \begin{fitchproof}
    \hypo{1}{A \eand (B \eand C)}
    \have{2}{A}\ae{1}
    \have{3}{B \eand C}\ae{1}
    \have{4}{C}\ae{3}
    \have{5}{A \eand C}\ai{2,4}
  \end{fitchproof}
\end{frame}

\subsection{Rules for \eif}

\begin{frame}
  \frametitle{Eliminating \eif}

  \begin{itemize}[<+->]
    \item What can we \emph{justify using} $\metav{P} \eif \metav{Q}$?
    \item We used the conditional ``If Amir lives in Chicago, Sarah
    isn't'' to justify ``Sarah doesn't live in
    Chicago''.
    \item What is the general rule? What can we justify using
    $\metav{P} \eif \metav{Q}$? What do we need in addition to $\metav{P} \eif \metav{Q}$?
    \item The principle is \emph{modus ponens}:
    \[ \metav{P} \eif \metav{Q}, \metav{P} \entails \metav{Q}\]
    \item (In inference from $A \eif \enot C$ and $A$ to $\enot C$, role
    of \metav{P} is played by $A$ and role of \metav{Q} by $\enot C$.)
  \end{itemize}
\end{frame}

\begin{frame}
  \frametitle{Elimination rule for \eif}
  \begin{fitchproof}
    \have[m]{a}{\metav{P} \eif \metav{Q}}
    \have[n]{b}{\metav{P}}
    \have[\ ]{c}{\metav{Q}} \ce{a, b}
  \end{fitchproof}

  We'll illustrate using the exercise
  \href{https://carnap.io/shared/rzach@ucalgary.ca/Practice\%20Problems\%20IV.md}{in
  Carnap}: we
  show that $A \eand B, A \eif C, B \eif D \models C \eand D$.
\end{frame}

\begin{frame}
  \begin{fitchproof}
    \hypo{1}{A \eand B}
    \hypo{2}{A \eif C}
    \hypo{3}{B \eif D}
    \have{4}{A}\ae{1}
    \have{5}{C}\ce{2,4}
    \have{6}{B}\ae{1}
    \have{7}{D}\ce{3,6}
    \have{8}{C \eand D}\ai{5,7}
  \end{fitchproof}
\end{frame}

\begin{frame}
  \frametitle{Introducing \eif}

  \begin{itemize}[<+->]
  \item How do we justify a conditional? What should we require for a proof
  of $\metav{P} \eif \metav{Q}$ (say, from some premise $\metav{R}$)?

  \item We need a proof that shows that $\metav{R} \entails \metav{P} \eif
  \metav{Q}$.

  \item Idea: show instead that $\metav{R}, \metav{P} \entails \metav{Q}$.

  \item The conditional $\eif$ no longer appears, so this seems easier.

  \item It's a good move, because if $\metav{R}, \metav{P} \entails
  \metav{Q}$
  then $\metav{R} \entails \metav{P} \eif
  \metav{Q}$.
  \end{itemize}
\end{frame}

\begin{frame}
  \frametitle{Justifying $\eif$I}
  \begin{block}{Fact}
  If $\metav{R}, \metav{P} \entails
  \metav{Q}$
  then
  $\metav{R} \entails \metav{P} \eif
  \metav{Q}$.
  \end{block}

  \begin{itemize}[<+->]
    \item If $\metav{R}, \metav{P} \entails
    \metav{Q}$ then every valuation makes one of $\metav{R},
    \metav{P}$ false or it makes $\metav{Q}$ true.
    \item Let's show that no valuation is a counterexample to $\metav{R} \entails \metav{P} \eif
    \metav{Q}$:
    \begin{enumerate}
      \item A valuation that makes $\metav{R}$ and
    $\metav{P}$ true, and $\metav{Q}$ false, is impossible if  $\metav{R}, \metav{P} \entails
    \metav{Q}$.
    \item So any valuation must make $\metav{R}$ false, $\metav{P}$
    false, or $\metav{Q}$ true.
    \item If it makes $\metav{R}$ false, it's not a counterexample to $\metav{R} \entails \metav{P} \eif
    \metav{Q}$.
    \item If it makes $\metav{P}$ false, it makes $\metav{P} \eif
    \metav{Q}$ true, so it's not a counterexample.
    \item If it makes $\metav{Q}$ true, it also makes $\metav{P} \eif
    \metav{Q}$ true, so it's not a counterexample.
    \end{enumerate}
    \item So, there are no counterexamples to $\metav{R} \entails \metav{P} \eif
    \metav{Q}$.
  \end{itemize}
\end{frame}

\begin{frame}
  \frametitle{Subproofs}

  \begin{itemize}[<+->]
    \item We want to justify $\metav{P} \eif \metav{Q}$ by giving a
    proof of $\metav{Q}$ from $\metav{P}$ as a premise.
    \item How to do this in a proof? We can't just add something as a
    premise and then remove it later!
    \item Solution: add $\metav{P}$ as a premise in the middle, and keep track
    of what depends on that premise (say, by a indenting and vertical
    line).
    \item Once we're done (have proved \metav{Q}), close this ``subproof''.
    \item Justification of $\metav{P} \eif \metav{Q}$ is the \emph{entire} subproof.
    \item \emph{Important}: nothing \emph{inside} a subproof is available
    outside as a justification (it depends on the assumption)
  \end{itemize}
\end{frame}

\begin{frame}
  \frametitle{Introduction rule for \eif}
  \begin{fitchproof}
    \open
    \hypo[m]{a}{\metav{P}}
    \ellipsesline
    \have[n]{b}{\metav{Q}}
    \close
    \have[\ ]{c}{\metav{P} \eif \metav{Q}} \ci{a-b}
  \end{fitchproof}
  We'll illustrate using the exercises
  \href{https://carnap.io/shared/rzach@ucalgary.ca/Practice\%20Problems\%20IV.md}{here}
  \begin{itemize}
  \item Show: $A \eif B, B \eif C \models A \eif C$.
  \item Show: $A \eif (B \eif C) \models (A \eand B) \eif (A \eand C)$
  \end{itemize}
\end{frame}

\begin{frame}
  \begin{fitchproof}
    \hypo{1}{A \eif B}
    \hypo{2}{B \eif C}
    \open
    \hypo{3}{A}
    \have{4}{B}\ce{1,3}
    \have{5}{C}\ce{2,4}
    \close
    \have{6}{A \eif C}\ci{3-5}
  \end{fitchproof}
\end{frame}

\begin{frame}
  \begin{fitchproof}
    \hypo{1}{A \eif (B \eif C)}
    \open
    \hypo{2}{A \eand B}
    \have{3}{A}\ae{2}
    \have{4}{B \eif C}\ce{1,3}
    \have{5}{B}\ae{2}
    \have{6}{C}\ce{4,5}
    \have{7}{A \eand C}\ai{3,6}
    \close
    \have{8}{(A \eand B) \eif (A \eand C)}\ci{2-7}
  \end{fitchproof}
\end{frame}

\subsection{Use of subproofs}

\begin{frame}
  \frametitle{Reiteration}

  $\metav{P} \models \metav{P}$, so ``Reiteration'' R is a good rule:

  \begin{fitchproof}
    \have[m]{ab}{\metav{P}}
    \have[k]{a}{\metav{P}} \by{R}{ab}
  \end{fitchproof}

  Uses of reiteration:
  \begin{itemize}[<+->]
    \item Proof of $A \models A$.
    \item Proof that $A \eif (B \eif A)$ is a tautology.
  \end{itemize}
\end{frame}

\begin{frame}
  \begin{fitchproof}
    \open
    \hypo{1}{A}
    \have{2}{A}\by{R}{1}
    \close
    \have{3}{A \eif A}\ci{1-2}
  \end{fitchproof}
\end{frame}

\begin{frame}
  \begin{fitchproof}
    \open
    \hypo{1}{A}
    \open
    \hypo{2}{B}
    \have{3}{A}\by{R}{1}
    \close
    \have{4}{B \eif A}\ci{2-3}
    \close
    \have{5}{A \eif (B \eif A)}\ci{1-4}
  \end{fitchproof}
\end{frame}

\begin{frame}
  \frametitle{Rules for justifications and subproofs}

  \begin{itemize}[<+->]
    \item When a rule calls for a subproof, we cite it as $n$--$m$,
    with first and last line numbers of the subproof.
    \item Assumption line \emph{and} last line have to match rule.
    \item After a subproof is done, you can only cite the whole thing,
    and not any individual line in it.
    \item Subproofs can be nested.
    \item When that happens, you also can't cite any subproof entirely
    contained inside another subproof, once the surrounding subproof
    is done.
  \end{itemize}
\end{frame}

\begin{frame}
  \frametitle{Reiteration}
Which are correct applications of R?
  \begin{fitchproof}
    \open
    \hypo{ab}{A}
    \open
    \hypo{b}{A}
    \have{a}{A} \by{\uncover<2->{\alert{\checkmark}} R}{ab}
    \close
    \have{d}{A} \by{\uncover<3->{\alert{\checkmark}} R}{ab}
    \have{c}{A} \by{\uncover<4->{\alert{\text{\ding{55}}}} R}{b}
    \close
    \have{e}{A} \by{\uncover<5->{\alert{\text{\ding{55}}}} R}{b}
    \have{f}{A} \by{\uncover<6>{\alert{\text{\ding{55}}}} R}{ab}
  \end{fitchproof}
\end{frame}

\subsection{Rules for \eor}

\begin{frame}
  \frametitle{Introduction rule for \eor}
  We have $\metav{P} \entails \metav{P} \eor \metav{Q}$. So:
  \begin{fitchproof}
    \have[m]{ab}{\metav{P}}
    \have[\ ]{a}{\metav{P}\eor\metav{Q}} \oi{ab}
  \end{fitchproof}
  \begin{fitchproof}
    \have[m]{ab}{\metav{Q}}
    \have[\ ]{b}{\metav{P}\eor\metav{Q}} \oi{ab}
  \end{fitchproof}
\end{frame}

\begin{frame}
  \begin{fitchproof}
    \open
    \hypo{1}{A}
    \have{2}{B \eor A}\oi{1}
    \close
    \have{3}{A \eif (B \eor A)}\ci{1-2}
  \end{fitchproof}
\end{frame}

\begin{frame}
  \frametitle{Eliminating \eor}

  \begin{itemize}[<+->]
  \item What can we justify with disjunction $\metav{P} \lor \metav{Q}$?

  \item Not $\metav{P}$ and also not $\metav{Q}$: neither is entailed by
  $\metav{P} \eor \metav{Q}$.

  \item But: if both $\metav{P}$ and $\metav{Q}$ separately entail some
  third sentence $\metav{R}$, then we know that $\metav{R}$ follows!

  \item To show this, we need \emph{two} proofs that show $\metav{R}$, but
  in each proof we are allowed to use only one of $\metav{P}$, $\metav{Q}$.
  \end{itemize}
\end{frame}

\begin{frame}
  \frametitle{Elimination rule for \eor}
  \begin{fitchproof}
    \have[m]{o}{\metav{P} \eor \metav{Q}}
    \open
    \hypo[i]{a}{\metav{P}}
    \ellipsesline
    \have[j]{b}{\metav{R}}
    \close
    \open
    \hypo[k]{aa}{\metav{Q}}
    \ellipsesline
    \have[l]{bb}{\metav{R}}
    \close
    \have[\ ]{c}{\metav{R}} \oe{o, a-b, aa-bb}
  \end{fitchproof}
\end{frame}


\begin{frame}
  \begin{fitchproof}
    \hypo{1}{A \eor B}
    \open
    \hypo{2}{A}
    \have{3}{B \eor A}\oi{2}
    \close
    \open
    \hypo{4}{B}
    \have{5}{B \eor A}\oi{4}
    \close
    \have{6}{B \eor A}\oe{1,2-3,4-5}
  \end{fitchproof}
\end{frame}

\begin{frame}
  \begin{fitchproof}
    \hypo{1}{A \eor B}
    \hypo{2}{A \eif B}
    \open
    \hypo{3}{A}
    \have{4}{B}\ce{2,3}
    \close
    \open
    \hypo{5}{B}
    \have{6}{B}\by{R}{5}
    \close
    \have{7}{B}\oe{1,3-4,5-6}
  \end{fitchproof}
\end{frame}

\subsection{Contradictions}

\begin{frame}
  \frametitle{Contradictions}

  \begin{itemize}[<+->]
  \item In proofs, we don't just use the premises of the argument, but also
  sentences we've proved, and sentences we've assumed (for $\eif$I,
  $\eor$E).

  \item Sometimes it happens that assumptions we must make for correct
  applications of these rules are incompatible with the premises.

  \item For instance, to prove the disjunctive syllogism $A \eor B,
  \enot B \models A$ using \eor E, the assumption $B$ for the second
  case conflicts with the premise $\enot B$.
  \end{itemize}
\end{frame}

\begin{frame}
  \frametitle{Disjunctive syllogism}
  \begin{fitchproof}
    \have{o}{A \lor B}
    \hypo{oa}{\alert{\enot B}}
    \open
    \hypo{a}{A}
    \have{b}{A} \by{R}{a}
    \close
    \open
    \hypo{aa}{\alert{B}}
    \ellipsesline
    \have[k]{bb}{A}
    \close
    \have{c}{A} \oe{o, a-b, aa-bb}
  \end{fitchproof}
\end{frame}

\begin{frame}
  \frametitle{Contradictions: eliminating $\enot$}

  We highlight the situation where inside a subproof we have run into
  a contradictory situation by the symbol \[\alert{\bot}\]

  \begin{fitchproof}
    \have[m]{a}{\enot\metav{P}}
    \have[n]{b}{\metav{P}}
    \have[\ ]{c}{\bot} \ne{a, b}
  \end{fitchproof}

  Since this also eliminates a $\enot$, we'll call it $\enot$E.
\end{frame}

\begin{frame}
  \frametitle{Explosion}

  \begin{itemize}[<+->]
  \item Any argument with jointly unsatisfiable premises is valid.
  \item So whenever we can justify $\bot$ in a proof, we should be able to
  justify \emph{anything}.
  \item ``From a contradiction, anything follows.''

  \begin{fitchproof}
    \have[m]{ab}{\bot}
    \have[k]{a}{\metav{P}} \by{X}{ab}
  \end{fitchproof}
  \end{itemize}
\end{frame}

\begin{frame}
  \frametitle{Disjunctive syllogism}
  \begin{fitchproof}
    \have{o}{A \lor B}
    \hypo{oa}{\alert{\enot B}}
    \open
    \hypo{a}{A}
    \have{b}{A} \by{R}{a}
    \close
    \open
    \hypo{aa}{\alert{B}}
    \have{bot}{\alert{\bot}}\by{\alert{\enot E}}{oa,aa}
    \have{bb}{\alert{A}}\by{\alert{X}}{bot}
    \close
    \have{c}{A} \oe{o, a-b, aa-bb}
  \end{fitchproof}
\end{frame}

\subsection{Introducing \enot}

\begin{frame}
  \frametitle{Introducing $\enot$}

  \begin{itemize}[<+->]
    \item An argument is valid iff the premises
    together with the negation of the conclusion are jointly
    unsatisfiable.
    \item For instance:
    \begin{itemize}[<+->]
      \item $\metav{Q} \entails \metav{P}$ iff
      $\metav{Q}$ and $\enot\metav{P}$ are jointly unsatisfiable.
      \item $\metav{Q} \entails \enot\metav{P}$ iff
      $\metav{Q}$ and $\metav{P}$ are jointly unsatisfiable.
    \end{itemize}
    \item This last one gives us idea for $\enot$I rule: To justify
    $\enot\metav{P}$, show that $\metav{P}$ (together with all other
    premises) is unsatisfiable.
    \item Unsatisfiable means: a contradiction ($\bot$) follows!
  \end{itemize}
\end{frame}

\begin{frame}
  \frametitle{Introduction rule for \enot}
  \begin{fitchproof}
    \open
    \hypo[m]{a}{\metav{P}}
    \ellipsesline
    \have[n]{b}{\bot}
    \close
    \have[\ ]{c}{\enot\metav{P}} \ni{a-b}
  \end{fitchproof}
\end{frame}

\begin{frame}
  \begin{fitchproof}
    \open
    \hypo{1}{A \eif B}
    \open
    \hypo{2}{\enot B}
    \open
    \hypo{3}{A}
    \have{4}{B}\ce{1,3}
    \have{5}{\ered}\ne{2,4}
    \close
    \have{6}{\enot A}\ni{3-5}
    \close
    \have{7}{\enot B \eif \enot A}\ci{2-6}
    \close
    \have{8}{(A \eif B) \eif (\enot B \eif \enot A)}\ci{1-7}
  \end{fitchproof}
\end{frame}

\begin{frame}
  \frametitle{Indirect proof rule}
  \begin{fitchproof}
    \open
    \hypo[m]{a}{\enot\metav{P}}
    \ellipsesline
    \have[n]{b}{\bot}
    \close
    \have[\ ]{c}{\metav{P}} \by{IP}{a-b}
  \end{fitchproof}
\end{frame}

\begin{frame}
  \begin{fitchproof}
    \open
    \hypo{1}{\enot A \eif \enot B}
    \open
    \hypo{2}{B}
    \open
    \hypo{3}{\enot A}
    \have{4}{\enot B}\ce{1,3}
    \have{5}{\ered}\ne{2,4}
    \close
    \have{6}{A}\by{IP}{3-5}
    \close
    \have{7}{B \eif A}\ci{2-6}
    \close
    \have{8}{(\enot A \eif \enot B) \eif (B \eif A)}\ci{1-7}
  \end{fitchproof}
\end{frame}

\subsection{Strategies and examples}

\begin{frame}
  \frametitle{The rules, one more time: \eand}
\begin{columns}
  \begin{column}{3cm}
  \begin{fitchproof}
    \have[m]{a}{\metav{P}}
    \have[n]{b}{\metav{Q}}
    \have[\ ]{c}{\metav{P}\eand\metav{Q}} \ai{a, b}
  \end{fitchproof}
\end{column}
\begin{column}{3cm}
  \begin{fitchproof}
    \have[m]{ab}{\metav{P}\eand\metav{Q}}
    \have[\ ]{a}{\metav{P}} \ae{ab}
  \end{fitchproof}
  \begin{fitchproof}
    \have[m]{ab}{\metav{P}\eand\metav{Q}}
    \have[\ ]{b}{\metav{Q}} \ae{ab}
  \end{fitchproof}
\end{column}
\end{columns}
\end{frame}

\begin{frame}
  \frametitle{The rules, one more time: \eif}
\begin{columns}
  \begin{column}{3cm}
    \begin{fitchproof}
      \open
      \hypo[m]{a}{\metav{P}}
      \ellipsesline
      \have[n]{b}{\metav{Q}}
      \close
      \have[\ ]{c}{\metav{P} \eif \metav{Q}} \ci{a-b}
    \end{fitchproof}
  \end{column}
  \begin{column}{3cm}
    \begin{fitchproof}
      \have[m]{a}{\metav{P} \eif \metav{Q}}
      \have[n]{b}{\metav{P}}
      \have[\ ]{c}{\metav{Q}} \ce{a, b}
    \end{fitchproof}
  \end{column}
\end{columns}
\end{frame}

\begin{frame}
  \frametitle{The rules, one more time: \eor}
\begin{columns}

  \begin{column}{3cm}
    \begin{fitchproof}
      \have[m]{o}{\metav{P} \eor \metav{Q}}
      \open
      \hypo[i]{a}{\metav{P}}
      \ellipsesline
      \have[j]{b}{\metav{R}}
      \close
      \open
      \hypo[k]{aa}{\metav{Q}}
      \ellipsesline
      \have[l]{bb}{\metav{R}}
      \close
      \have[\ ]{c}{\metav{R}} \oe{o, a-b, aa-bb}
    \end{fitchproof}
  \end{column}
  \begin{column}{2.5cm}
      \begin{fitchproof}
        \have[m]{ab}{\metav{P}}
        \have[\ ]{a}{\metav{P}\eor\metav{Q}} \oi{ab}
      \end{fitchproof}
      \begin{fitchproof}
        \have[m]{ab}{\metav{Q}}
        \have[\ ]{b}{\metav{P}\eor\metav{Q}} \oi{ab}
      \end{fitchproof}
  \end{column}
\end{columns}
\end{frame}


\begin{frame}
  \frametitle{The rules, one more time: \enot}
\begin{columns}
  \begin{column}{3cm}
    \begin{fitchproof}
      \have[m]{a}{\enot\metav{P}}
      \have[n]{b}{\metav{P}}
      \have[\ ]{c}{\bot} \ne{a, b}
    \end{fitchproof}
    \end{column}
  \begin{column}{3cm}
    \begin{fitchproof}
      \open
      \hypo[m]{a}{\metav{P}}
      \ellipsesline
      \have[n]{b}{\bot}
      \close
      \have[\ ]{c}{\enot\metav{P}} \ni{a-b}
    \end{fitchproof}
  \end{column}
\end{columns}
\end{frame}

\begin{frame}
  \frametitle{The rules, one more time: R, X, and IP}
\begin{columns}
  \begin{column}{3cm}
    \begin{fitchproof}
      \have[m]{ab}{\metav{P}}
      \have[k]{a}{\metav{P}} \by{R}{ab}
    \end{fitchproof}
    \begin{fitchproof}
      \have[m]{ab}{\bot}
      \have[k]{a}{\metav{P}} \by{X}{ab}
    \end{fitchproof}
  \end{column}
  \begin{column}{3cm}
  \begin{fitchproof}
    \open
    \hypo[m]{a}{\enot\metav{P}}
    \ellipsesline
    \have[n]{b}{\bot}
    \close
    \have[\ ]{c}{\metav{P}} \by{IP}{a-b}
  \end{fitchproof}
    \end{column}
\end{columns}
\end{frame}

\begin{frame}
  \frametitle{Working forward and backward}

  \begin{itemize}[<+->]
    \item \emph{Working backward} from a conclusion (goal) means:
    \begin{itemize}[<+->]
      \item Find main connective of goal sentence
      \item Match with conclusion of corresponding I rule
      \item Write out (above the goal!) what you'd need to apply that rule
    \end{itemize}
    \item \emph{Working forward} from a premise, assumption, or
    already justified sentence means:
    \begin{itemize}[<+->]
      \item Find main connective of premise, assumption, or sentence
      \item Match with top premise of corresponding E rule
      \item Write out what else you need to apply the E rule (new goals)
      \item If necessary, write out conclusion of the rule
    \end{itemize}
  \end{itemize}
\end{frame}

\begin{frame}
  \frametitle{Constructing a proof}

  \begin{itemize}[<+->]
    \item Write out premises at the top (if there are any)
    \item Write conclusion at bottom
    \item Work backward \& forward from goals and premises/assumptions
    in this order:
    \begin{itemize}[<+->]
      \item Work backward using \eand I, \eif I, \eiff I, \enot I, or
      forward using \eor E
      \item Work forward using \eand E
      \item Work forward from \enot E if your goal is \ered
      \item Work forward using \eif E, \eiff E
      \item Work backward from \eor I
      \item Try IP
    \end{itemize}
    \item Repeat for each new goal from top
  \end{itemize}
\end{frame}

\begin{frame}\small
  \begin{fitchproof}
    \hypo{1}{\enot(A \eor B)}
    \open
    \hypo{2}{A}
    \have{3}{A \eor B}\oi{2}
    \have{4}{\ered}\ne{1,3}
    \close
    \have{5}{\enot A}\ni{2-4}
    \open
    \hypo{2a}{B}
    \have{3a}{A \eor B}\oi{2a}
    \have{4a}{\ered}\ne{1,3a}
    \close
    \have{5a}{\enot B}\ni{2a-4a}
    \have{8}{\enot A \eand \enot B}\ai{5,5a}
  \end{fitchproof}

  \note[itemize]{    \begin{itemize}

    \item Construct proof strategically
    \item Conclusion is \eand: needs two subgoals above, $\enot A$ and
    $\enot B$.
    \item Tackle $\enot A$ first: main connective is \enot, use \enot I
    \item \enot I needs subproof with assumption $A$ and last line \ered
    \item To prove \ered, work forward from line 1: use \enot E
    \item We have $\enot(A \eor B)$, we also need $A \eor B$
    \item Work backward, ie try to apply \eor I. We'd need $A$ or $B$.
    \item $A$ already there (assumption of subproof)
    \item Fill in all justifications for first handful
    \item Second half the same
  \end{itemize}
  }
\end{frame}

\begin{frame}\scriptsize
  \begin{fitchproof}
    \open
    \hypo{1}{\enot A \eand \enot B}
    \open
    \hypo{2}{A \eor B}
    \open
    \hypo{3}{A}
    \have{4}{\enot A}\ae{1}
    \have{5}{\ered}\ne{4,3}
    \close
    \open
    \hypo{6}{B}
    \have{7}{\enot B}\ae{1}
    \have{8}{\ered}\ne{7,6}
    \close
    \have{9}{\ered}\oe{2,3-5,6-8}
    \close
    \have{10}{\enot(A \eor B)}\ni{2-9}
    \close
    \have{11}{(\enot A \eand \enot B) \eif \enot(A \eor B)}\ci{1-10}
  \end{fitchproof}

  \note[itemize]{    \begin{itemize}
    \item Conclusion is \eif: needs subproof with $\enot A \eand \enot
    B$ as assumption, $\enot(A \eor B)$ as last line
    \item New goal is $\enot(A \eor B)$, use \enot I: needs a subproof with $A \eor B$ as
    assumption and \ered as last line
    \item Want \ered: could try \enot E, but we don't have any sentence $\enot \metav{P}$.
    \item Better to first work forward from $A \eor B$: apply \eor E.
    \item Needs two subproofs, one stating with $A$ the other with $B$
    \item Last line of each must match goal we want to prove (\ered)
    \item Now in subproof, trying to prove \ered from $A$
    \item Now's the time to work forward from $\enot A \eand \enot B$
    to get $\enot A$
    \item That lets us justify \ered using \enot E
    \item Second subproof the same
    \end{itemize}
  }
\end{frame}


\begin{frame}
  \begin{fitchproof}
    \open
    \hypo{1}{\enot(A \eor \enot A)}
    \open
    \hypo{2}{A}
    \have{3}{A \eor \enot A}\oi{2}
    \have{4}{\ered}\ne{1,3}
    \close
    \have{5}{\enot A}\ni{2-4}
    \have{6}{A \eor \enot A}\oi{5}
    \have{7}{\ered}\ne{1, 6}
    \close
    \have{8}{A \eor \enot A}\by{IP}{1-7}
  \end{fitchproof}

    \note[itemize]{\scriptsize    \begin{itemize}
      \item Conclusion is $A \eor \enot A$. Working backward using
      \eor I doesn't work, since neither $A$ nor $\enot A$ is
      tautology, so neither has a proof with no premises
      \item No other strategy applies. Last resort: use IP
      \item Needs subproof, with $\enot(A \eor \enot A)$ as assumption, \ered{} as last line
      \item To prove \ered, work forward using \enot E if possible (ie
      if some $\enot \metav{P}$ is available).  There is: $\enot(A \eor \enot A)$ on line 1
      \item For \enot E we also need $\metav{P}$, i.e., $A \eor \enot
      A$: enter that as new goal
      \item Now we should try working backward using \eor I again! Let's try $\enot A$ as new goal
      \item Work backward using \enot I: needs subproof with
      assumption $A$ and last line \ered
      \item Once agian: when trying to prove \ered, work forward using
      E when possible, ie, again look for $A \eor \enot A$ as goal
      \item That can be justified using the new assumption $A$
    \end{itemize}}
\end{frame}