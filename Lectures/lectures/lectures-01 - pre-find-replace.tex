% !TeX root = ./slides-01.tex

\setcounter{section}{0}
\section{What is logic?}
\subsection{Arguments and validity}

\begin{frame}
  \frametitle{An easy puzzle}

  \begin{block}{Where does Sanjeev live?}
  Sanjeev lives in Calgary or in Edmonton.\\
  Sanjeev doesn't live in Edmonton.
  \end{block}

  \begin{itemize}
    \item<2>[A:] Obviously, in Calgary.
  \end{itemize}

\end{frame}

\begin{frame}
  \frametitle{Arguments and sentences}

  \begin{block}{Argument 1}
  Sanjeev lives in Calgary or in Edmonton.\\
  Sanjeev doesn't live in Edmonton.\\
  Therefore, Sanjeev lives in Calgary.
  \end{block}

  \begin{itemize}[<+->]
  \item Such an argument consists of \emph{sentences}.
  \item Individual sentences are the kinds that can be \emph{true} or \emph{false}.
  \item ``Therefore'' ($\therefore$) indicates that the last sentence (supposedly)
  \emph{follows from} the first two.
  \item The last sentence is called the \emph{conclusion}.
  \item The others are called the \emph{premises}.
  \end{itemize}

\end{frame}

\begin{frame}
  \frametitle{Valid and invalid arguments}

  \begin{block}{Argument 2}
  Mandy enjoys skiing or hiking (or both).\\
  Mandy doesn't enjoy hiking.\\
  $\therefore$ Mandy enjoys skiing.
  \end{block}

  \begin{block}{Argument 3}
  Mandy enjoys skiing or hiking (or both).\\
  Mandy enjoys skiing.\\
  $\therefore$ Mandy doesn't enjoy hiking.
  \end{block}

  What's the difference?

  \note[itemize]{
    \begin{itemize}
    \item Arg 2 is like argument 1
    \item Think-pair-share argument 2
    \item Depending on result of experiment/agreement talk about
    inclusive/exclusive
    \item Argument 2 is valid if the ``or'' is exclusive.
    \end{itemize}
  }
\end{frame}


\begin{frame}
  \frametitle{(Deductive) Validity}

  \begin{definition}<1->
  An argument is (deductively) \emph{valid} if there is no case where all its
  premises are true and the conclusion is false.
  \end{definition}

  \begin{definition}<2->
  An argument is \emph{invalid} if there is at least one case where
  all its premises are true and the conclusion is false (i.e., if it
  is not valid).
  \end{definition}

  \begin{definition}<3->
  A \emph{case} is some hypothetical scenario that makes each sentence
  in an argument either true or false.
  \end{definition}
\end{frame}

\begin{frame}
  \frametitle{Argument 2 is valid}

  \begin{block}{Argument 2}
  Mandy enjoys skiing or hiking.\\
  Mandy doesn't enjoy hiking.\\
  $\therefore$ Mandy enjoys skiing.
  \end{block}

Argument 2 is \emph{valid}: whenever the premises are
true, the conclusion is also true.

\end{frame}

\begin{frame}
  \frametitle{Argument 3 is not valid}

  \begin{block}{Argument 3}
  Mandy enjoys skiing or hiking.\\
  Mandy enjoys skiing.\\
  $\therefore$ Mandy doesn't enjoy hiking.
  \end{block}

  Argument 3 is \emph{invalid}: there is a possible case where the
  premises are true and the conclusion isn't (Mandy enjoys both skiing
  and hiking).

\end{frame}

\begin{frame}
  \frametitle{A harder puzzle}

  \begin{block}{Where does Sarah live?}
  Sarah lives in Calgary or Edmonton.\\
  Amir lives in Calgary unless he enjoys hiking.\\
  If Amir lives in Calgary, Sarah doesn't.\\
  Neither Sarah nor Amir enjoy hiking.
  \end{block}

\end{frame}

\subsection{Cases and determining validity}

\begin{frame}
  \frametitle{Validity}

  \begin{definition}
  An argument is \emph{valid} if there is no case where all its
  premises are true and the conclusion is false.
  \end{definition}

  \begin{definition}
  An argument is \emph{invalid} if there is at least one case where
  all its premises are true and the conclusion is false (i.e., if it
  is not valid).
  \end{definition}
\end{frame}

\begin{frame}{Cases}
  
\begin{definition}
  A \emph{case} is some hypothetical scenario that makes each sentence
  in an argument either true or false.
\end{definition}

\begin{itemize}[<+->]
  \item E.g., imagine you have a friend, her name is Mandy, she loves
  hiking but hates skiing.
  \item That's a case where ``Mandy enjoys hiking or skiing'' is true.
  \item Some cases can be imagined even though they never happen IRL, e.g,
  ``It is raining and the skies are clear.''
  \item Some things you can't imagine, e.g.,
  ``There is a blizzard but there is no wind.''
\end{itemize}

\end{frame}

\begin{frame}
  \frametitle{Determining validity}

  \begin{itemize}[<+->]
    \item Imagine a case where the conclusion is false.
    \item Are the premises true? You're done: invalid.
    \item Otherwise, change or expand the case to make them true
    (without making the conclusion also true).
    \item Can't? (Probably) valid.
  \end{itemize}

  \uncover<5->{OR}

  \begin{itemize}[<+->]
    \item Imagine a case where all premises are true.
    \item Is the conclusion false? You're done: invalid.
    \item Otherwise, change or expand the case to make it false
    (without making the premises false).
    \item Can't? (Probably) valid.
  \end{itemize}
\end{frame}

\begin{frame}{Deductively Valid?}
  \begin{earg}
    \item[] Some rodents have bushy tails.
    \item[] All squirrels are rodents.
    \item[\therefore] Some squirrels have bushy tails.
  \end{earg}
\pause
  \begin{itemize}[<+->]
    \item Imagine squirrels evolving so that they no longer have bushy
    tails. Then conclusion is false.
    \item But premises still true:
    \begin{itemize}[<+->]
      \item Imagine chinchillas still have bushy tails.
      \item Imagine also that squirrels have not evolved too
      much---they are still rodents.
    \end{itemize}
  \end{itemize}
\end{frame}

\begin{frame}{Valid?}
  \begin{earg}
    \item[] All rodents have bushy tails.
    \item[] All squirrels are rodents.
    \item[\therefore] All squirrels have bushy tails.
  \end{earg}
\pause
  \begin{itemize}[<+->]
    \item If it were invalid, you'd have a case that makes the
    conclusion false: some squirrels without bushy tails.
    \item They would have to be rodents still (otherwise premise 2 false).
    \item And that would require that they have bushy tails (otherwise premise 1 false).
  \end{itemize}
\end{frame}

\subsection{Other logical notions}

\begin{frame}
  \frametitle{Logical Consistency}

  \begin{definition}
  Sentences are (logically) \emph{consistent} if there is a case where they
  are all true. \\ $\bullet$ also called `jointly possible' or `satisfiable' 
  \end{definition}

  \begin{definition}
    Sentences are (logically) \emph{inconsistent} if there is no case where they
    are all true. \\ $\bullet$ also called `jointly impossible' or `unsatisfiable' 
    \end{definition}
\end{frame}


\begin{frame}
  \frametitle{Consistent?}
  \begin{earg}
    \item[] Some carnivores have bushy tails.
    \item[] All carnivores are mammals. 
    \item[] No mammals have bushy tails.
  \end{earg}

  \begin{itemize}
    \item<2> No case makes them all true at the same time, so
    \emph{jointly impossible}.
  \end{itemize}
\end{frame}

\begin{frame}
  \frametitle{Valid?}
  \begin{earg}
    \item[] Some carnivores have bushy tails.
    \item[] All carnivores are mammals. 
    \item[] No mammals have bushy tails.
    \item[\therefore] All birds are carnivores. 
  \end{earg}

  \pause

  \begin{itemize}[<+->]
    \item The premises cannot all be true in the same case, so jointly
    impossible.
    \item So: no case makes all the premises true.
    \item So also: no case makes the premises true and the conclusion false.
    \item \emph{Arguments with jointly impossible premises are automatically
    valid}, regardless of what the conclusion is.
  \end{itemize}
\end{frame}

\begin{frame}{Tautology (logically necessary truth)}

  \begin{definition}
    A sentence is a \emph{necessary truth} if there is no case where
    it is false. \\ also called a `necessary truth' or `truth-functionally true'
    \end{definition}
\pause
    \begin{itemize}[<+->]
      \item If it's snowing, it's snowing.
      \item Every fawn is a deer.
      \item The number 5 is prime.
      \item Physical objects are extended.
    \end{itemize}
\end{frame}

\begin{frame}{Tautology}
    What can you say about an argument where the conclusion is a
    necessary truth? 
\pause
\begin{itemize}[<+->]
  \item If the conclusion is a necessary truth, there is no case where
  it is false.
  \item So there is no case where it is false, and the premises of the
  argument are true.
  \item \emph{Arguments with necessary truths as conclusions are
  automatically valid}, regardless of what the premises are.
\end{itemize}
  \end{frame}

\begin{frame}
  \frametitle{Logical equivalence}

  \begin{definition}
  Two sentences are (logically) \emph{equivalent} if there is no case where one is
  true and the other is false.
  \end{definition}

  \begin{itemize}[<+->]
    \item What can you say about an argument where one of the premises
    is equivalent to the conclusion? Is it automatically valid?
    \item Can you have two equivalent sentences that are jointly impossible?
  \end{itemize}

\end{frame}

\subsection{Symbolization and TFL}

\begin{frame}
  \frametitle{Validity in virtue of form}

  \begin{block}<1->{Argument 1}
  Sanjeev lives in Calgary or Edmonton.\\
  Sanjeev doesn't live in Edmonton.\\
  $\therefore$ Sanjeev lives in Calgary.
  \end{block}

  \begin{block}<1->{Argument 2}
  Mandy enjoys skiing or hiking.\\
  Mandy doesn't enjoy hiking.\\
  $\therefore$ Mandy enjoys skiing.
  \end{block}

  \begin{block}<2>{Form of arguments 1 \& 2}
  $X$ or $Y$.\\
  Not $Y$.\\
  $\therefore$ $X$.
  \end{block}

\end{frame}

\begin{frame}
  \frametitle{Some valid argument forms}

  \begin{block}{Disjunctive syllogism}
  $X$ or $Y$.\\
  Not $Y$.\\
  $\therefore\ X.$
  \end{block}

  \begin{block}{Modus ponens}
  If $X$ then $Y$.\\
  $X$.\\
  $\therefore\ Y.$
  \end{block}

  \begin{block}{Hypothetical syllogism}
  If $X$ then $Y$.\\
  If $Y$ then $Z$.\\
  $\therefore$ If $X$ then $Z$.
  \end{block}
\end{frame}

\begin{frame}
  \frametitle{Symbolizing arguments}

  \begin{block}{Symbolization key}
  $S$: Mandy enjoys skiing\\
  $H$: Mandy enjoys hiking
  \end{block}

  \begin{block}{Argument 2}
    \begin{tabular}{@{}l@{}l@{}}
      Mandy enjoys skiing or Mandy enjoys hiking.  & \ \emph{$(S \lor H)$}\\
      Not: Mandy enjoy hiking. & \emph{$\lnot H$}\\
      $\therefore$ Mandy enjoys skiing. & $\therefore$ \emph{$S$}
    \end{tabular}
  \end{block}
\end{frame}

\begin{frame}
  \frametitle{The language of TFL}

  \begin{itemize}[<+->]
  \item \emph{Sentence letters}, such as `$H$' and `$S$', to symbolize basic sentences (`Mandy likes hiking')
  \item \emph{Connectives}, to indicate how basic sentences are connected
  \begin{description}
    \item[$\lor$] either \dots or \dots
    \item[$\land$] both \dots and \dots
    \item[$\to$] if \dots then \dots
    \item[$\lnot$] not \dots
  \end{description}

  \item[] This can get complicated, e.g.:

  ``Mandy enjoys skiing or hiking, and if she lives in Edmonton, she
  doesn't enjoy both.''
  \[
  ((S \lor H) \land (E \to \lnot(S \land H)))
  \]
  \end{itemize}
\end{frame}

\subsection{What are we going to learn, and why?}

\begin{frame}
  \frametitle{What is logic?}

  \begin{itemize}[<+->]
  \item \emph{Logic is the science of what follows from what.}
  \item Sometimes a conclusion follows from the premises, sometimes it
  does not:
  \begin{itemize}[<+->]
    \item Mandy lives in Calgary.\\ Everyone who lives in Calgary likes hiking.\\
    $\therefore$ Mandy likes hiking.
    \item Mandy lives in Calgary.\\ Everyone who likes hiking lives in Calgary.\\
    $\therefore$ Mandy likes hiking.
    \end{itemize}
  \item Logic investigates what makes the first argument \emph{valid}
    and the second \emph{invalid}.
  \end{itemize}
\end{frame}

\begin{frame}
  \frametitle{What is formal logic?}

  \begin{itemize}[<+->]
  \item Studies logical properties of \emph{formal languages} (TFL and
  FOL, not English).
    \begin{itemize}[<+->]
    \item Logical consequence (what follows from what?)
    \item Logical consistency (when do sentences contradict one another?)
    \end{itemize}
  \item Expressive power (what can be expressed in a given formal
  language, and how?)
  \item Formal models (mathematical structures described by formal language)
  \item Inference and proof systems (how can it be proved that something
  follows from something else?)
  \item (Meta-logical properties of logical systems)
  \end{itemize}
\end{frame}

\begin{frame}
  \frametitle{Plan for the course}

  \begin{itemize}[<+->]
  \item Truth-functional logic (TFL)
    \begin{itemize}[<+->]
    \item Symbolization in the formal language of TFL ($H, \lor,
    \land, \to, \lnot$)
    \item Testing for validity: truth-tables
    \item Proofs in natural deduction
    \end{itemize}
  \item First-order logic (FOL)
    \begin{itemize}[<+->]
    \item More fine-grained symbolization ($E(m,h)$, $\forall$
    `every', $\exists$ `some', $=$)
    \item Semantics: interpretations
    \item Proofs in natural deduction
    \end{itemize}
  \item Some advanced topics: expressive completeness, normal forms
  \end{itemize}
\end{frame}

\begin{frame}
  \frametitle{What is logic good for? (Philosophy)}

  \begin{itemize}[<+->]
  \item Logic originates in philosophy (Aristotle), traditionally considered a sub-discipline of philosophy.
  \item Valid arguments are critical in philosophical research.
  \item Formal tools of logic are useful to make intuitive philosophical
  notions precise, e.g.,
    \begin{itemize}[<+->]
    \item Possibility and necessity
    \item Time
    \item Composition and parthood (mereology)
    \item Moral obligation and permissibility
    \item Belief and knowledge
    \end{itemize}
  \item Logic applies to semantics of natural language (philosophy of
  language, linguistics).
  \end{itemize}
\end{frame}

\begin{frame}
  \frametitle{What is logic good for? (Mathematics)}

  \begin{itemize}[<+->]
  \item Formal logic was developed in the quest for a foundations of
  mathematics (19th C.).
  \item Logical systems provide precise foundational framework for
  mathematics:
    \begin{itemize}[<+->]
    \item Axiomatic systems (e.g, geometry)
    \item Algebraic structures (e.g., groups)
    \item Set theory (e.g, Zermelo-Fraenkel with Choice)
    \end{itemize}
  \item Precision
    \begin{itemize}[<+->]
    \item Formal language makes claims more precise.
    \item Formal structures can point to alternatives, unveil gaps in proofs.
    \item Formal proof systems make proofs rigorous.
    \item Formal proofs make mechanical \emph{proof checking} and \emph{proof search} possible.
    \end{itemize}
  \end{itemize}

\end{frame}


\begin{frame}
  \frametitle{What is logic good for? (Computer Science)}

  \begin{itemize}[<+->]
    \item Computer science deals with lots of formal languages.
    \item Logic is a good example of how to set up and use formal languages.
    \item Logic : Computer Science $=$ Calculus : Natural Science
    \item Applications of logical systems in CS are numerous:
  \begin{itemize}[<+->]
  \item Combinational logic circuits
  \item Database query languages
  \item Logic programming
  \item Knowledge representation
  \item Automated reasoning
  \item Formal specification and verification (of programs, of hardware designs)
  \item Theoretical computer science (theory of computational
  complexity, semantics of programming languages)
 \end{itemize}
\end{itemize}

\end{frame}
