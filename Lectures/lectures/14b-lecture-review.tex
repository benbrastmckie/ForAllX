% !TeX root = ./14b-handout-review.tex

\setcounter{section}{13}

\section{Final Review!}

\begin{frame}
%\large

\scriptsize{\tableofcontents}

\end{frame}

\subsection{Checking Soundness for Alt. Rules}

\begin{frame}
\frametitle{Alternative Natural Deduction Rules}
%\large

\begin{itemize}[<+->]

\item As we did with trees (system STD), we can consider whether modifying SND with a new rule preserves soundness

\item Method for generating new cases: take a case in the book and add a negation symbol(s) somewhere; 
\item[] then figure out what a sound rule would give you. 

%Below, I have included an additional case as a practice problem. 

\end{itemize}
\end{frame}

\begin{frame}
\frametitle{Negated Conjunction Introduction}
%\large

\begin{itemize}[<+->]

\item Consider a system SND$^*$ just like SND except that we add the following rule:

\item \emph{Negated Conjunction Introduction}: from $\enot \metav{Q}$ derive $\enot (\metav{Q} \eand \metav{R})$

\item Does this rule preserve soundness? If so, extend our proof by adding a case to the induction (showing that the new line is righteous); If not, provide a concrete counterexample to soundness of SND$^*$

\item Strategy: first do a heuristic: do the earlier accessible sentences semantically entail the final sentence? 
\bi

\item If yes, then the new rule preserves soundness (proceed to formally extend the proof!)

\item If no, then you should be able to construct a concrete counterexample to soundness (i.e. case where $\Gamma \vdash_{SND^*} P$ but $\Gamma \nentails P$ for a concrete set of SL sentences $\Gamma$ 

\ei


\end{itemize}
\end{frame}

\begin{frame}
\frametitle{Notation for Soundness Cases}
%\large

\begin{itemize}

\item $\Gamma_i$ stands for the set of assumptions that are open at the $i$-th line, i.e. these are the accessible premises/assumptions at line $i$. They are every premise/assumption (sentence sitting on a horizontal line) such that its scope line (vertical line) travels all the way down to line $i$, and line $i$ is to the right of this vertical line.  

\item $\BM{P_i}$ stands for the sentence that is on the $i $-th line.

\item $\Delta \subseteq \Gamma$ means that the set $\Delta$ is a subset of $\Gamma $.

\item $\Gamma\cup\{\BM{Q}\}$ means that we have added the sentence $\BM{Q}$ to the set of sentences $\Gamma $ (we have taken their union).

\end{itemize} 

\end{frame}

\begin{frame}
\frametitle{Induction Hypothesis and Key Fact}
%\large

\begin{itemize}

\item {\it Induction hypothesis} for Soundness: assume that the soundness/righteousness property holds for all lines $i$ less than the $k +1$-st line, i.e. if $i \leq k$ and if $\Gamma_i \vdash\BM{P_i} $, then $\Gamma_i \vDash\BM{P_i} $. 

\item In words: we are assuming that if we can derive a sentence $\BM{P_i} $ from a set of assumptions $\Gamma_i$, then those assumptions semantically entail that sentence.

\item Lemma 6.3.2 (a.k.a. Useful Fact 1): if $\Gamma \vDash \BM{P}$ and $\Gamma $ is a subset of a larger set $\Gamma '$, then the larger set semantically entails the sentence $\BM{P}$ as well, i.e. $\Gamma ' \vDash \BM{P} $.


\end{itemize}
\end{frame}

\begin{frame}
\frametitle{Negated Conjunction Introduction}
%\large

\begin{itemize}[<+->]

\item \emph{Negated Conjunction Introduction}: from $\enot \metav{Q}$ derive $\enot (\metav{Q} \eand \metav{R})$

\item Draw the deduction with the final sentence on line \#k+1; \\ label everything schematically so that you can refer to earlier line numbers and their open premise sets $\Gamma_m$ 

\item Extend the proof of soundness by showing that a line generated by this rule is righteous, i.e. $\Gamma_{k+1} \entails \metav{P}_{k+1}$

\item Rely on the relevant subset relations between the various $\Gamma$ premise sets

\item Reason about relevant semantic entailment claims by using the truthtables for the connectives

\end{itemize}
\end{frame}

\begin{frame}
\frametitle{Schematic Solution Steps (if you're totally lost)}
%\large

\begin{enumerate}[<+->]

\small

\item Label the lines in your diagram with lowercase letters (e.g. j, $\ell$, m, n, etc.) so that you can refer to them. Label the LAST LINE as $k+1$. 

\item  Reexpress the derivation diagram in terms of single turnstiles, i.e. if you have $P$ on line j, then $\Gamma_j \vdash P$ (i.e. the set of open assumptions at line j provides a derivation for $P$).

\item Apply the induction hypothesis to any lines that are less than the $k +1 $-th line. This lets you convert these single turnstiles into double turnstiles, e.g. $\Gamma_j \vDash P$, provided that $j < (k+1)$.

\item Relate the set of assumptions open at various lines (your $\Gamma$'s) to the set of assumptions open at the last line, $\Gamma_{k +1} $. This will involve the subset relation $\subseteq$, e.g. $\Gamma_j \subseteq \Gamma_{k +1} $. 
\begin{itemize}

\item If the sentence $P_j$ at line j is an additional open assumption that is not open at line $k+1$, then you need to tack this on, using the union operation: $\Gamma_j \subseteq (\Gamma_{k +1} \cup P_j) $.

\end{itemize}



\end{enumerate}

\end{frame}

\begin{frame}
\frametitle{Schematic Solution Steps continued}
%\large

\begin{enumerate}[<+->]

\small

\item[5.] Apply useful fact 1 (i.e. lemma 6.3.2), using the relation(s) in the previous step. E.g., if you have $\Gamma_j \vDash P$ (from step 3) and $\Gamma_j \subseteq \Gamma_{k +1} $ (from step 4), then useful fact 1 entails that $\Gamma_{k +1} \vDash P$.

\item[6.] Next, consider an arbitrary truth value assignment that makes all of the sentences in $\Gamma_{k +1} $ true. Use whatever double turnstiles are at your disposal to infer that some other sentence(s) is true. 

\item[6.] Then, use a truth table to argue why the sentence on the last line (line $k +1$) must be true as well under this truth value assignment.

\item[8.] Pat yourself on the back (soundly)!

\end{enumerate}

\end{frame}

\begin{frame}
\frametitle{For additional guidance on Soundness, see\dots}
%\large

\begin{itemize}

\item Section 6.3 of \textit{The Logic Book} (reading for Week 12)

\item pages 246-250 contain most of the cases for our system SND

\item PS12 \#1 handles negation elimination (case 10)

\item \S6.3 Exercises on page 250-251, problem \#4 parts a thru d

\item Think of your own cases by throwing in negation symbols, thinking about de Morgan's or other semantically equivalent sentences, etc.! 

\end{itemize}
\end{frame}

\subsection{Applications of soundness \& completeness}

\begin{frame}
\frametitle{Applying soundness and/or completeness theorems}
%\large

\begin{itemize}[<+->]

\item PS12, problems \#4--7 illustrate simple applications of the soundness and/or completeness theorems

\item The final might contain problems of a similar flavor

\end{itemize}
\end{frame}

\begin{frame}
\frametitle{A Key Fact to Remember, Understand, Retain}
%\large

\begin{itemize}[<+->]

\item  If $\Gamma \cup \{ \enot \metav{P} \}$ is unsatisfiable, what else can we say? 

\item Answer: $\Gamma \entails \metav{P}$ (and vice-versa)

\item If $\Gamma \entails \enot \metav{Q}$, what else can we say? 

\item Answer: $\Gamma \cup \{\metav{Q} \}$ is unsatisfiable (and vice-versa)

\item See p. 245 if you don't believe this; but should be able to give valid arguments for these claims verbally! 

%\item What if $\Gamma \cup \{\metav{Q} \}$ is unsatisfiable?

%\item Answer, then $\Gamma \entails \enot \metav{Q}$

\end{itemize}
\end{frame}

\begin{frame}
\frametitle{Practice w/ Applying Soundness \& Completeness}
%\large

To avoid ambiguity, let the sentences and sets of sentences be from QL, and let `$\vdash$' denote $\vdash_{QND}$

\begin{enumerate}[<+->]

\item Prove or provide a counterexample to the following statement: \\ If $\Gamma \entails \metav{P}$ and $\Delta \vdash \metav{Q}$, then  $\Gamma \cup \Delta \vdash \metav{P} \eand \metav{Q}$

%Following is too similar to the preceding one! 
%\item If $\Gamma \vdash \BM{P}$ and $\Delta \vdash \BM{Q}$, prove or provide a counterexample that $(\Gamma \cup\Delta) \vDash \BM{P}\& \BM{Q} $

\item If $\Gamma \vdash (S \lor R)$ and $\Gamma \vdash \sim (S \lor R)$, prove or provide a counterexample that $\Gamma \vDash S$
%(Hint: remember that if a set of sentences is truth-functionally inconsistent, it vacuously entails any sentence).

%%perhaps save following for the exam! 
%\item If $\Gamma \vdash \BM{P}$ and $\Delta \cup\{\BM{H}\}$ is truth-functionally inconsistent, prove or provide a counterexample that $(\Gamma \cup\Delta) \vDash (\BM{P} \lor \BM{R}) $ (note that I really do mean ``$\BM{H}$" and then ``$\BM{R}$.")

%good application of counterexample method! just let gamma be P and let delta be R haha. 
\item If $\Gamma \cup \{ \enot \BM{P} \}$ is unsatisfiable and $\Delta \vdash \BM{R}$, prove or provide a counterexample to $(\Gamma \cup\Delta) \vdash (\sim\BM{P} \equiv \BM{R}) $.

%%should i turn the following into an instance of compactness as well? 
\item Prove or give a counterexample to the following statement: \\ If $\Gamma$ is satisfiable, then $\{\sim S \mid S \in \Gamma\} $ is satisfiable.
%%answer: false! let \Gamma be a set of tautologies! Then every ~S is a contradiction, and so the set of these is unsatisfiable 

\end{enumerate}
\end{frame}

\begin{frame}
\frametitle{Concept Review (if totally lost)}
%\large

\begin{itemize}

\item Soundness theorem for SND: if you have a single turnstile (in SND), then you have a double turnstile. In words: if a set of assumptions gives you a derivation (in SND) for a sentence $S$, then those assumptions semantically entail that sentence $S $. In symbols: if $\Gamma \vdash_{SND} \BM{S} $, then $\Gamma \vDash \BM{S} $. 
%Mnemonic: think `Soundness Singles'. 

\item Completeness theorem for SND: if you have a double turnstile, then you have a single turnstile (in SND). In words: if a set of assumptions semantically entails a sentence $S$, then those assumptions gives you a derivation (in SND) for that sentence $S $. In symbols: if $\Gamma \vDash \BM{S} $, then $\Gamma \vdash_{SND} \BM{S} $.

\item Likewise for QL and QND  
%Mnemonic: think `dolla dolla doubles make Washington complete.'

%\item Truth-functional inconsistency (6.3.5): if $\Gamma \cup \{\BM{Q}\}$ is truth-functionally inconsistent, then $\Gamma \vDash (\sim \BM{Q}) $. See page 245 for short proof. The backwards direction is also true. 


\end{itemize}
\end{frame}

\begin{frame}
\frametitle{Solution Tips for Logically Complete Students}
%\large

\begin{enumerate}

\item Use the soundness theorem to convert any single turnstiles you have (from system SND) into double turnstiles.

\item Convert claims about unsatisfiability into double turnstile relations
%Use theorem 6.3.5 (about truth-functional inconsistency) to get a double turnstile.

\item Use the completeness theorem to convert any double turnstiles you have into single turnstiles.

\item If you get stuck, write out the definitions of any key terms involved. These will guide you on your path to victory.

\item If you have to provide a counterexample, think about the simplest counterexample that gets the job done. Your counterexample must involve ACTUAL sentences; not metavariables
% CONCRETE, i.e. you must give a concrete instance for any set(s) referenced in the problem, showing how these concrete instances satisfy the problem assumptions but violate the problem conclusion.

\item Pray for a stroke of insight! (Jk! Try reasoning backwards to figure out what you need!)

\end{enumerate}
\end{frame}



\subsection{Alt. Cases of Membership Lemma}

\begin{frame}
\frametitle{Alternative Cases of Membership Lemma}
%\large

\begin{itemize}[<+->]

\item In the completeness proofs, recall that the five SND membership lemma cases are motivated by truth-functional considerations

\item We can prove variants of these cases, e.g. the following: modified version of case (e): $\sim\BM{P}\equiv\BM{Q}\in\Gamma^{\ast}$ if and only if either i) both $\BM{P}\notin\Gamma^{\ast}$ and $\BM{Q}\in\Gamma^{\ast}$ or ii) both $\BM{P}\in\Gamma^{\ast}$ or $\BM{Q}\notin\Gamma^{\ast}$.

\item Alternately, one can be given an alternative SND rule (replacing one of our 11 sanctioned rules) from which to reprove a given case of the membership lemma (using the Door lemma)

\end{itemize}
\end{frame}

\begin{frame}
\frametitle{Mega Reminder: TWO directions to show!}
%\large

\begin{itemize}[<+->]

\item Note that all of these cases have TWO directions, and you need to prove BOTH directions to complete the problem. 
\item[] First, you want to assume the thing on the left and derive the thing on the right (forward direction). 
\item[] Second, you want to assume the thing on the right and derive the thing on the left (backwards direction).

\item Sometimes a case involves subcases, each of which can require its own non-trivial SND deduction (e.g. cases (c) and (d) for disjunction and conditional)

\item Finally, remember that the membership lemma is purely syntactic! No mention of truth-value assignments here!

\end{itemize}
\end{frame}

\begin{frame}
\frametitle{Membership Lemma (not that I'm a bouncer!)}
%\large

\begin{itemize}[<+->]

\item \emph{Membership Lemma} for club $\Gamma^{\ast}$: if \metav{P} and \metav{Q} are SL wffs, then:

\begin{enumerate}[a.)]

\item $\enot \metav{P} \in \Gamma^{\ast}$ if and only if $\metav{P} \notin \Gamma^{\ast}$

\item $\metav{P} \eand \metav{Q} \in \Gamma^{\ast}$ if and only if both $\metav{P}\in \Gamma^{\ast}$ and $\metav{Q}\in \Gamma^{\ast}$

\item $\metav{P} \eor \metav{Q} \in \Gamma^{\ast}$ if and only if either $\metav{P}\in \Gamma^{\ast}$ or $\metav{Q}\in \Gamma^{\ast}$

\item $\metav{P} \eif \metav{Q} \in \Gamma^{\ast}$ if and only if either $\metav{P}\notin \Gamma^{\ast}$ or $\metav{Q}\in \Gamma^{\ast}$

\item $\metav{P} \eiff \metav{Q} \in \Gamma^{\ast}$ iff either (i) $\metav{P}\in \Gamma^{\ast}$ and $\metav{Q}\in \Gamma^{\ast}$ or (ii) $\metav{P}\notin \Gamma^{\ast}$ and $\metav{Q}\notin \Gamma^{\ast}$

\end{enumerate}

\item Notice how these syntactic constraints mirror truth-conditions!

\end{itemize}
\end{frame}

\begin{frame}
\frametitle{Two Examples of Membership Lemma}
%\large

\begin{itemize}[<+->]

\item Case (a) is very useful: $\enot \metav{P} \in \Gamma^{\ast}$ if and only if $\metav{P} \notin \Gamma^{\ast}$

\item (modified version of case b): \\ $\BM{P}\&\sim\BM{Q}\in\Gamma^{\ast}$ if and only if $\BM{P}\in\Gamma^{\ast}$ and $\BM{Q}\notin\Gamma^{\ast}$

\item Additional practice problem (modified version of case c): \\ prove that $\BM{P}\lor\sim\BM{Q}\in\Gamma^{\ast}$ if and only if either $\BM{P}\in\Gamma^{\ast}$ or $\BM{Q}\notin\Gamma^{\ast}$. 

\item Study case (d) for the conditional (bottom of p. 258)! 

\item Note that the 7-line derivation for case d) has a serious typo on line 2: the justification should be ``:A / $\supset$ I", i.e.  :AS for conditional intro.

\end{itemize}
\end{frame}

\begin{frame}
\frametitle{Maximally Consistent-in-SND}
%\large

\begin{itemize}

\item $\Gamma^{\ast}$ is maximally-SND-consistent provided that both (i) $\Gamma^{\ast}$ is consistent in SND (i.e. can't derive any contradictions) and \\ (ii) if $\BM{P}$ is not in $\Gamma^{\ast}$, then $\Gamma^{\ast}\cup\{\BM{P}\}$ is inconsistent in SND.

\item In other words: you can't derive a contradiction from assumptions in $\Gamma^{\ast}$. And if $\BM{P}\notin\Gamma^{\ast}$, then $\Gamma^{\ast}\cup\{\BM{P}\}$ lets you derive a contradictory pair (i.e. you can derive both $\BM{R}$ and $\sim\BM{R}$). 

\item Assuming that you are not asked to prove a variant of case (a), you can help yourself to this result. Hence, if a sentence $\BM{P}\notin\Gamma^{\ast}$, then case (a) lets you conclude that $\sim\BM{P}\in\Gamma^{\ast}$, and vice versa: if $\sim\BM{P}\in\Gamma^{\ast}$, then you can conclude that $\BM{P}\notin\Gamma^{\ast}$. 

\end{itemize}
\end{frame}

\begin{frame}
\frametitle{The Door Lemma}
%\large

\begin{itemize}[<+->]

\item Lemma 6.4.9 (a.k.a. `The Door'): this lemma helps you show that a sentence $S$ is a member of a maximally SND-consistent set $\Gamma^{\ast}$: 
\begin{itemize}

\item if you can derive $S$ from a subset $\Gamma$ of a maximally SND-consistent set $\Gamma^{\ast}$, then $S$ is a member of $\Gamma^{\ast}$.

\item In symbols: if $\Gamma \vdash S$ and $\Gamma \subseteq \Gamma^{\ast} $, then $S\in\Gamma^{\ast} $. In particular, if $\Gamma^{\ast} \vdash S$, then $S\in\Gamma^{\ast} $.

\item Hence the strategy: if you are trying to show that  $S\in\Gamma^{\ast} $, figure out how to derive $S$ in SND from sentences you have assumed are in $\Gamma^{\ast} $. Then, apply The Door. 

\end{itemize}



\end{itemize}
\end{frame}




\subsection{Translations in QL, with Identity}

\begin{frame}
\frametitle{Some Structures to remember from SL}
%\large

\begin{itemize}[<+->]

\item \emph{P only if Q}: $P \eif Q$ (order preserved) (equiv: $\enot Q \eif \enot P$)

\item \emph{Unless B, C} or \emph{C unless B}: use OR: $B \eor C$

\item \emph{J just in case K}: $J \eiff K$ 

\item Q if P; Q provided that P; Q given that P; if P, then Q: $P \eif Q$



\end{itemize}
\end{frame}

\begin{frame}
\frametitle{Some Simple Examples not involving identity}
%\large

% % Following two questions come from JTapp winter 2019 PHIL 303 practice final

Domain: all people; 

Predicates: \emph{Dx}: x went to Disneyland; \emph{Kxy}: x knows y; 

Constants/Names: j for John; m for Mary

\begin{itemize}[<+->]

\item Schematize ``Everyone who went to Disneyland knows someone who didn't go there'':

\item[] Answer: $\qt{\forall}{x} ( Dx \eif \qt{\exists}{y} ( Kxy \eand \enot Dy))$

\item ``There is someone who knows both Mary and John but doesn't know themself'':

\item[] Answer: $\qt{\exists}{x} (Kxm \eand Kxj \eand \enot Kxx)$

\item ``Everyone who knows John also knows Mary'':

\item[] Answer: $\qt{\forall}{x} (Kxj \eif Kxm)$



\end{itemize}
\end{frame}

\subsubsection{Only, Neither, Counting}

\begin{frame}
  \frametitle{Singular ``only''}

  \begin{itemize}[<+->]
   % \item ``\emph{No-one other than Greta} is a hero'':
    %\item[] \emph{$\enot\qt{\exists}{x}(H\qv{x} \eand \enot x\!\!=\!\!g)$}
    %\item[] \emph{$\qt{\forall}{x}(H\qv{x} \eif x\!\!=\!\!g)$}
    \item ``\emph{Only Greta} is a hero'':
    \item Content: No-one other than Greta is a hero, \emph{AND} Greta is a hero:
    \item[] \emph{$\qt{\forall}{x}(H\qv{x} \eif x\!\!=\!\!g) \eand H\qv{g}$}
    \item[] \emph{$\qt{\forall}{x}(H\qv{x} \eiff x\!\!=\!\!g)$}
    %JH: are these two symbolizations quantificationally equivalent? i.e. can i derive one from the other? think about---could be good exercise. alternativelY: is there a countermodel to equivalence
    %maybe they disagree in case nothing is a hero, latter is trivially true? whereas former requires g to be in extension of H. 
  \end{itemize}
  \end{frame}

\begin{frame}
    \frametitle{Schematizing `Neither'}

\begin{itemize}[<+->]
\item ``Neither hero inspires'': this means that 
\item[] There are \emph{exactly 2} heroes, and neither of them inspires:
\begin{align*}
\qt{\exists}{x}\qt{\exists}{y}\Big(\textcolor{OGlyallpink}{(}(\lnot x\!\!=\!\!y \land (H\qv{x} \land H\qv{y})) & {}\land {}\\
\qt{\forall}{z}(H\qv{z} \to (z = x \lor z = y))\textcolor{OGlyallpink}{)} &{} \land {}\\
(\enot I\qv{x} \eand \enot I\qv{y})\Big)&
\end{align*}
\end{itemize}
\end{frame}


\begin{frame}
    \frametitle{At least $n$}

\begin{itemize}[<+->]

\item Remember: we interpret ``three heros are inspiring'' to mean ``\textbf{at least} three heros are inspiring''
\item At least 1 hero is inspiring:
\[
\alert{\qt{\exists}{x}(H\qv{x} \land I\qv{x})}
\]
\item At least 2 heroes are inspiring:
\[
\alert{\qt{\exists}{x}\qt{\exists}{y}}(\alert{\enot x\!\!=\!\!y \land ((H\qv{x} \land I\qv{x}) \land (H\qv{y} \land I\qv{y}))})
\]
\item At least 3 heroes are inspiring:
\begin{align*}
& \alert{\qt{\exists}{x}\qt{\exists}{y}\qt{\exists}{z}}\Big(\alert{(\enot x\!\!=\!\!y \land (\enot y = z \land \enot x = z)) \eand {}} \\
& \qquad \alert{\big((H\qv{x} \land I\qv{x}) \land ((H\qv{y} \land I\qv{y}) \eand (H\qv{z} \land I\qv{z}))\big)}\Big)
\end{align*}
\end{itemize}
\end{frame}

\begin{frame}
  \frametitle{At least $n$}

\begin{itemize}[<+->]
\item Note: must state that \emph{every pair} of variables is different, e.g.,
\begin{align*}
\qt{\exists}{x_1}\qt{\exists}{x_2}\qt{\exists}{x_3}(&(\enot x_1 = x_2 \land \enot x_2 = x_3) \land {}\\
& (H\qv{x_1} \land (H\qv{x_2} \land H\qv{x_3})))
\end{align*}
only says ``There are at least two heroes''!
\begin{itemize}[<+->]
  \item Take extension of $H\qv{x}$ to be: $1,2$
  \item Then $1$ can play role of $x_1$ and $x_3$, $2$ role of $x_2$.
  \item Both ``$\enot 1=2$'' and ``$\enot 2=3$'' are true.
\end{itemize}

\item UD: People. Predicates: \emph{Dx}: x went to Disneyland; \emph{Kxy}: x knows y 

\item ``There is somebody who went to Disneyland and knows at least two people who didn't go there''

\item[] Answer: $\qt{\exists}{x} (Dx \eand \qt{\exists}{y} \qt{\exists}{z} (\enot y=z \eand Kxy \eand Kxz \eand \enot Dy \eand \enot Dz) )$

\end{itemize}
\end{frame}


\begin{frame}
    \frametitle{At most $n$}

\begin{itemize}
\item<1-> There are \emph{at most $n$} $A$s $\Leftrightarrow$ There are
\emph{not at least $n+1$} $A$s
\[
\qt{\exists^{\alert{\le n}}}{x}\, A\qv{x} \Leftrightarrow \alert{\lnot} \qt{\exists^{\alert{\ge(n+1)}}}{x} \, A\qv{x}
\]
\item<2-> For instance: There are at most two heroes:
\begin{align*}
\uncover<2->{\lnot \qt{\exists}{x}\qt{\exists}{y}\qt{\exists}{z}((H\qv{x} \land (H\qv{y} \land H\qv{z}))}
& \uncover<2->{\land (\lnot x =y \land (\lnot x = z \land \lnot y = z)) )}\\
\uncover<3->{\qt{\forall}{x}\qt{\forall}{y}\qt{\forall}{z}((H\qv{x} \land (H\qv{y} \land H\qv{z}))
} &\uncover<3->{\eif (x =y \lor ( x = z \lor y = z)) )}
\end{align*}

%JTapp Winter 2019, 303, actual final; part of question 4b) 

\item ``At most one person who knows Mary doesn't know John''

\item[] Answer: $\enot \qt{\exists}{x}\qt{\exists}{y}(\enot x = y \eand Kxm \eand Kym \eand \enot Kxj \eand \enot Kyj)$


\end{itemize}
\end{frame}



\subsubsection{`The' Definite Description}

\begin{frame}
    \frametitle{Definite descriptions}

\begin{itemize}[<+->]

\item Reminder: singular possessives like ``Earth's moon" can be interpreted like the definite description ``the moon of Earth." But plural possessives like ``Mars's moons" aren't definite descriptions.

\item Definite description: \emph{the so-and-so}
\item Russell's analysis of definite description: to say\\[1ex]
\centerline{``The $A$ is B''}
is to say:
\item There is something, which:
\begin{itemize}[<+->]
\item is $A$,
\item is the \emph{only} $A$ (i.e. the unique thing that is $A$),
\item is $B$.
\end{itemize}
\item In QL:
\[
\qt{\exists}{x}(A\qv{x} \land \qt{\forall}{y}(A\qv{y} \to x\!\!=\!\!y) \land B\qv{x})
\]
\item or more succinctly:
\[
\qt{\exists}{x}\qt{\forall}{y}\big((A\qv{y} \eiff x\!\!=\!\!y) \land B\qv{x}\big)
\]
\end{itemize}
\end{frame}



\begin{frame}
\frametitle{Example: `The author of Waverley is blah'}
%\large

\begin{itemize}[<+->]

\item Schematize ``The author of \textit{Waverley} is Scottish'':

\item Use the following symbolization key:

\item $Ax$: x is an author; $Wxz$: x wrote z; $Sx$: x is Scottish; $\ell$: \textit{Waverley}
\item[]
\[
\qt{\exists}{x}(A\qv{x} \land Wx\ell \land \qt{\forall}{y}( (A\qv{y} \land Wy\ell) \to x\!\!=\!\!y) \land S\qv{x})
\]

%Perhaps why Russell used `the author of Waverley' (i.e. scottish poet/novelist Sir Walter Scott) as his running example: Scott published Waverley anonymously and then subsequent novels as being "by the author of Waverley".

\end{itemize}
\end{frame}


\begin{frame}
  \frametitle{Singular possessive (a definite description)}

  \begin{itemize}[<+->]
    \item Singular possessives form noun phrases, e.g., ``Joe's cape''
    \item They work like definite descriptions: \\ Joe's cape is the cape Joe owns.
    E.g.:
    \begin{itemize}
      \item ``Autumn wears \alert{Joe's cape}'' symbolizes the same as:
      \item[] ``Autumn wears \alert{the cape that Joe owns}'':
      \item[]
      \begin{align*}
        \qt{\exists}{x}\Big(& \uncover<6->{\textcolor{OGlyallpink}{(}(E\qv{x} \land O\qr{j}{x}) \land {}}\\
        & \uncover<7->{\qt{\forall}{y}((E\qv{y} \land O\qr{j}{y}) \eif x\!\!=\!\!y)\textcolor{OGlyallpink}{)} \land {}}\\
        & \uncover<8->{W\qr{a}{x}\Big)}
      \end{align*}
    \end{itemize}
  \end{itemize}
\end{frame}

\begin{frame}
  \frametitle{Singular vs. plural possessive}

  \begin{itemize}[<+->]
    \item Compare \emph{plural} possessives: those are `$\forall$'s':
    \bigskip
    \begin{itemize}[<+->]
      \item ``Autumn wears \alert{Joe's cape\textbf{s}}'' symbolizes the same
      as:
      \item[] ``Autumn wears every cape that Joe owns'':
      \[\qt{\forall}{x}\big((E\qv{x} \land O\qr{j}{x}) \eif W\qr{a}{x}\big)\]
    \end{itemize}
    \item So plural possessives are NOT definite descriptions. 
  \end{itemize}
\end{frame}



\subsection{Interpretations/Models for QL}

\begin{frame}
\frametitle{Interpretations/Models for QL}
%\large

\begin{itemize}[<+->]

\item Study PS9, especially problems like 9, 10, 13, 15, 17, and 19

\item Understand how to input this stuff into \textit{Carnap}! 

\item Understand what it takes to make an existential statement true (at least one object in the domain must satisfy the statement)

\item Understand what it takes to make a universal statement true (every object in the domain must satisfy the statement)

\item Watch out for conditionals, which are trivially satisfied if the antecedent is false

\end{itemize}
\end{frame}

\begin{frame}
  \frametitle{Truth of sentences of QL}

  \begin{itemize}[<+->]
    \item Given an interpretation $I$ \dots
    \item An \emph{atomic sentence} is true iff the referents of the constants are in the extension of the predicate:
    \begin{itemize}
    \item $P\qv{a}$ is true iff referent `$r$' of $a$ is in extension of $P$
    \item $R\qr{a}{b}$ is true iff $\langle r,p\rangle$ is in extension of $R$\\
    (where $r$ is referent of $a$, and $p$ is referent of $b$)
    \end{itemize}
    \item $\enot\metav{A}$ is true iff $\metav{A}$ is false
    \item $\metav{A} \eor \metav{B}$ is true iff at least one of $\metav{A}$, $\metav{B}$ is true
    \item $\metav{A} \eand \metav{B}$ is true iff both $\metav{A}$, $\metav{B}$ are true
    \item $\metav{A} \eif \metav{B}$ is true iff $\metav{A}$ is false or $\metav{B}$ is true
  \end{itemize}
\end{frame}

\begin{frame}
  \frametitle{Truth of quantified sentences}

  \begin{itemize}[<+->]
    \item $\qt{\exists}{x}\,\metav{A}\qv{x}$ is true iff $\metav{A}\qv{x}$ is \emph{satisfied} by \emph{at least one} object in the domain
    \begin{itemize}
    \item $r$ satisfies $\metav{A}\qv{x}$ in $I$ iff $\metav{A}\qv{r}$ is true in the interpretation 
     %following perhaps has a typo: \item $o$ satisfies $\metav{A}\qv{x}$ iff $\metav{A}\qv{c}$ is true in interpretation just like $I$, but with $o$ as referent of $c$
    \end{itemize}
    \bigskip
    
    \item $\qt{\forall}{x}\,\metav{A}\qv{x}$ is true iff $\metav{A}\qv{x}$ is \emph{satisfied} by \emph{every} object in the domain
  \end{itemize}
\end{frame}

\begin{frame}
  \frametitle{Truth of quantified sentences}
\large 
  \begin{itemize}[<+->]
    \item $\qt{\exists}{x}\,(\metav{A}\qv{x} \eand \metav{B}\qv{x})$ is true iff \textcolor{OGlyallpink}{some} object satisfies `$\metav{A}\qv{x} \eand \metav{B}\qv{x}$'
    \begin{itemize}
      \item $o$ satisfies `$\metav{A}\qv{x} \eand \metav{B}\qv{x}$' iff it satisfies both $\metav{A}\qv{x}$ and $\metav{B}\qv{x}$
    \end{itemize}
    
\bigskip

    \item $\qt{\forall}{x}\,(\metav{A}\qv{x} \eif \metav{B}\qv{x})$ is true iff \emph{every} object satisfies `$\metav{A}\qv{x} \eif \metav{B}\qv{x}$'
    \medskip
    \begin{itemize} 
    \large 
      \item $o$ satisfies `$\metav{A}\qv{x} \eif \metav{B}\qv{x}$' iff
      either
      \bigskip
      \begin{itemize}  
      \normalsize
        \item $o$ does not satisfy $\metav{A}\qv{x}$ (vacuously true conditional)
        \item[] \makebox[\textwidth]{or}
        \item $o$ does satisfy $\metav{B}\qv{x}$
      \end{itemize}
       \bigskip
    \end{itemize}
  \end{itemize}
\end{frame}


\subsection{Derivations in QND}

\begin{frame}
\frametitle{Quick Tips and a Practice Problem}
%\large

\begin{itemize}[<+->]

\item If you can do the derivations on PS10, you are probably in great shape! 

\item Focus in particular on the rules/syntax surrounding Existential Elimination, conditional introduction, and Universal Instantiation 

\item If you are going to do $\exists$E, it's typically best to start your proof with that and work within the $\exists$E subproof until you get what you need (which may not be what you want) 

\item Construct a deduction showing the following: \\ $\qt{\exists}{x} Qx$, $\qt{\forall}{y} (Qy \eif Py) \vdash_{QND} \qt{\exists}{z} Pz$

\item Another practice problem!: \\ $\qt{\exists}{x} Gx \eif  Fa \; \;$  $ \vdash_{QND} \qt{\forall}{x} (Gx \eif Fa)  $ 

\item The other direction is MUCH trickier (but can be done in 10 lines)! \\ $\qt{\forall}{x} (Gx \eif Fa)   \vdash_{QND}  \qt{\exists}{x} Gx \eif Fa$

%(Ex) Qx, (Ax) (Qx \eif Px) :|-: (Ex) Px



\end{itemize}
\end{frame}


\begin{frame}
\frametitle{Some General Advice (locate the MAIN quantifier)}
%\large

\begin{itemize}[<+->]

%\item Study problems like those in PS10 and week 10 practice

%Josh: try to select analogs from the practice problems???

\item Make sure you have a firm grip on the four rules of QND, and the rules of SND (see PS6), and the rule sheet

\item Make sure especially that you know how to correctly apply existential elimination (and check those three conditions) and universal introduction (and check those two conditions).

\item There are no special conditions for universal elimination, and the one for existential introduction is technically enforced by our recursive defN of QL wffs

\item Note that you CANNOT apply any of the SND rules {\it within} the scope of a quantifier! You must first eliminate the quantifiers to apply any SND rules to the stuff inside. 

\item In general, each rule applies only to the WHOLE sentence, not a part. So you CANNOT apply a rule to just part of a sentence. 

\end{itemize}
\end{frame}

\begin{frame}
\frametitle{Some more Specific Advice}
%\large

\begin{itemize}[<+->]

\item Make sure you understand how to build up a conditional by using conditional introduction! 
\bi
\item Assume the conditional's antecedent, justified by :AS for $>I$

\item Derive the consequent in the scope of this assumption (possibly starting further subproofs to get to the consequent, like negation elimination)

\item Then discharge the assumption and write the conditional!
\ei

\item You may then apply a quantifier rule to the conditional, e.g. Existential Introduction, to which you could then finish an existential elimination if you were in the scope of an EE subproof

\item If you get stuck, try to work from the bottom up. Think about what you would first have to derive to build your ultimate goal.

\item If you get stuck on a subgoal, assume the opposite of your subgoal to try using either negation introduction or negation elimination to keep going.

\end{itemize}
\end{frame}









