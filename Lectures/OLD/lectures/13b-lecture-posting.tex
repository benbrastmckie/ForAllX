% !TeX root = ./13b-handout-completeness-posting.tex

% %version that takes out some explicit content haha 

% I suppose that in many ways, the style of proof that I provide here (similar to Gordon's) is better than how the logic book remains entirely within QL. The proof that I provide obviously generalizes to any first-order language. You simply consider an extension of that language that adds a new symbol, allowing you to consider infinitely-more constants.

% Perhaps the reason why there cannot be a constructive proof of completeness for predicate logic: if we knew how to construct a (finite) natural deduction for any valid argument, then we would have a decision procedure for determining validity. Yet, we know that predicate logic is undecidable. So perhaps this is what gives us knowledge of the impossibility of having a constructive proof (since otherwise, you might just think that people haven't tried hard enough, but that such a proof could be found)

% At the end of my modified proof, I can presumably appeal to one of the isomorphism results shown in Belot logic notes. Basically, the model M and the model M prime are isomorphic, so they should Entail isomorphic statements


% Easier proof idea: construct $\Gamma'$ such that the starting $\Gamma$ remains a subset. So rather than mapping un-primed subscripted constants to primed ones, just allow us in constructing $\Gamma^{\ast}$ to appeal to a new batch of constants in the language QL', namely coming from a new batch of number indices, allowed only for constants. 
%, I'll be able to avoid the appeals to 11.1.13 and 11.4.8. Instead, I'll just be able to note immediately, as Gordon does, that since each sentence in $\Gamma$ belongs to $\Gamma^{\ast}$ , satisfying the latter satisfies the former. So then everything is completely parallel to what we were doing for SL

% I think I will need some kind of lemma like the following: if a set of QL-sentences is satisfied in a QL' model, then there exists a QL-model that satisfies it. 
% Simple idea that at least should work based on the way that we have constructed the model for $\Gamma^{\ast}$ : construct the relevant model M for the starting QL-sentences $\Gamma$  by simply defining it over the subset domain that excludes the primed-constants. And define it's interpreation function as a restriction of the one for M*. Then this is a well-defined QL-model that satisfies $\Gamma$ .
% Is there something that I really ought to proof your? E.g. involving a kind of isomorphism or a locality principle? I technically could not use locality as stated since that relies on having the same domain.

% % But maybe there is a much easier way of proceeding, using locality. Construct the QL-model such that it does have the same domain, and construct the QL-interpretation functions such that it is a restriction of the one for QL'. Then considering any QL-sentence from $\Gamma$ , it seems like locality straightforwardly applies
% problem: locality as stated applies only to two QL-models, and here I am comparing a QL' model to a QL-model....

% Maybe I should just proceed directly from the truth-conditions for QL sentences: given a variable assignment in the primed model, restrict this to the unprimed model and argue that it satisfies the QL sentence. i.e. use a d_I' to define a d_I that meets the truth conditions! 

% % % Easiest way out of this muddle:, which perhaps embarrassingly I did not see before. Just note that the model M* we construct is not just a QL' model for $\Gamma^{\ast}$  BUT ALSO a QL-model for  $\Gamma$ . This is simply because it meets the definition of a QL-model: we have a domain (which includes constants from QL' but now functioning as `objects') and a map I that maps all the stuff in QL to stuff in the domain. namely our I is the restriction of the I* in M*, restricted to symbols in QL, all of which appear in QL'. so we're done!!!! 

% One remaining possible worry: if the domain is the same in these two models, then how can the QL sentence (Ax)Px be true if much of what is in the domain is the primed-subscripted constants? By the satisfaction conditions for universal quantification, we would need the variant d_I[r/x] to satisfy P FOR EVERY object r in the domain, including the primed-subscripted constants. but then what are the names for these things??? do we necessarily need names for these things?

% Regardless though, it still seems like the definition of a QL-model is automatically satisfied. so maybe we just keep the same extensions for everything and it's just the case that not every object in the domain has a name in QL and that's fine (?) 

% Is this a unique worry to the way that I/Gordon am going about this? Or does the same kind of worry arise for what the logic book does with the even-subscripted set?

% % Completely satisfying resolution to my worry: note that to satisfy an existential, we don't require that our model has a name for every object in the domain. E.g. on Carnap problem set 9, we just needed something in the domain to be in the extension of the predicate. We often didn't have any constants, or we had way fewer constants. e.g. one constant b which names 2. but (Ex)Px for there is a prime could still be true provided we put 3 in the extension of P. 
% so even though language QL won't be able to name a lot of stuff in the domain (namely, all of the subscripted primed constants), that stuff in the domain can still make a lot of the relevant universals and existentials in $\Gamma$  true in our QL-model. 

\setcounter{section}{12}
\section{Completeness of QND}

\begin{frame}
%\large

\scriptsize{\tableofcontents}

\end{frame}

\begin{frame}
\frametitle{Completeness of QND}
%\large

\begin{itemize}[<+->]

\item \emph{QND is Complete}: For any set $\Gamma$  of QL-sentences and any QL-sentence $\metav{P}$, if $\Gamma$ semantically entails $\metav{P}$, then there exists a derivation of $\metav{P}$ from $\Gamma$ in our natural deduction system QND

\bi

\item In symbols: If $\Gamma \entails \Theta$, then $\Gamma \vdash_{QND} \Theta$
%Double Turnstile entails  Single turnstile  
\item Note that $\Gamma$ can be countably infinite 

\ei

%\item (logical entailment is fully covered by our syntactic rules; Means: we wrote down \textit{enough} rules!)

\bigskip

\item Completeness guarantees that for any valid QL-argument, there is at least one corresponding deduction in QND. 

\item So we need not reason about arbitrary models to determine if a QL-argument is valid; reasoning in QND suffices! WOW COOL

\end{itemize}
\end{frame}

\begin{frame}
\frametitle{``$\entails$'': our Semantic Double Turnstile}
%\large

\begin{itemize}[<+->]

%\item Recall that the double turnstile `$\entails$' stands for semantic entailment (aka logical consequence) within (classical) sentential logic QL. 

\item ``$\Gamma \entails \metav{P}$'' means that $\Gamma$ logically entails $\metav{P}$ \\ In any QL-model $\mathfrak{M}$ where the premises in $\Gamma$ are true, the conclusion $\metav{P}$ is true 

\item Equivalently: there is no QL-model $\mathfrak{M}$ such that \\ $\Gamma$ is satisfied while $\metav{P}$ is false

\item Equivalently, this means that \emph{$\Gamma \cup \{\enot \metav{P}\}$ is unsatisfiable}: \\ no QL-model makes-true the premises and negated conclusion  

\item We'll use this last fact A LOT in our proof that QND is complete! %completeness! 

\end{itemize}
\end{frame}


\iffalse
\begin{frame}
\frametitle{Soundness vs. Completeness}
%\large

% % It's perhaps interesting to think about why we do not have to demonstrate soundness and completeness results for truthtables. It is almost as if the syntax for truthtables is constitutive of the semantics.

% Could also make some verbal remarks about the notion of mathematical rigor, and how this has evolved over time. And how it might still be contested today, and issues of how rigorous physics ought to be remain highly relevant. Different methodological styles in physics and mathematical physics

\begin{itemize}[<+->]

\item Let $\Gamma$ be any set of \textit{sentences} of QL and $\Theta$ any sentence of QL. 

\item By proving that our derivation system is \textit{sound}, we show that QND derivations are `safe' (they preserve truth)

\medskip 

\bi

\item \emph{Sound}: If $\Gamma \vdash_{QND} \Theta$, then $\Gamma \entails \Theta$
%Single turnstile entails Double Turnstile 

\item (syntactic to semantic: i.e. we chose `good' rules!)

\ei

\bigskip 

\item By proving that QND is \textit{complete}, we show that reasoning about arbitrary models is not needed to demonstrate validity: \\ QND derivations suffice
%what about showing a set of sentences is unsatisfiable though? can we just not do this w/ QND derivations? would we need to introduce a falsum symbol? 

\medskip 

\bi

\item \emph{Complete}: If $\Gamma \entails \Theta$, then $\Gamma \vdash_{QND} \Theta$
%Double Turnstile entails  Single turnstile  

\item (logical entailment is fully covered by our syntactic rules)

\item (Means: we wrote down \textit{enough} rules!)

\ei

\end{itemize}
\end{frame}

\fi 


\iffalse %***********************************************************
\subsection{A Meta-refresher}

\begin{frame}
\frametitle{QND as a derivation system, provided that...}
%\large

\begin{itemize}[<+->]

\item As we have seen, Sentential Natural Deduction allows us to derive a conclusion from a set of premises:

%trees provide a shortcut for demonstrating that a set of sentences is inconsistent (i.e. unsatisfiable): construct a tree whose root is these sentences s.t. all branches close
%JH: interesting that QND seemingly does NOT let you show that a set of sentences is unsatisfiable; interesting diff in problem-solving support

%\item Underwrites further shortcuts for demonstrating:

\begin{enumerate}[1.)]

\item valid argument: conclusion on last line, in scope of just premises %trees: \\ (its premises and negated conclusion are unsatisfiable)

\item tautology: on last line in scope of NO premises %that a sentence is a tautology (its negation is unsatisfiable)

\item two logically equivalent sentences: (i) their biconditional is a tautology or (ii) derive one from the other and vice versa (which mirrors biconditional introduction!)

\end{enumerate}

\item But our derivations are justified only if system QND is \textit{sound}

\item And guaranteed to have a derivation for every valid argument only if system QND is \textit{complete}

%of a semantically valid argument only if system QND is \textit{complete}



\end{itemize}
\end{frame}

\begin{frame}
\frametitle{A tale of three turnstiles: one semantic; two syntactic}
%\large

\begin{itemize}[<+->]

\item Double Turnstile $\entails$: logical entailment (indexed to our choice of semantics, i.e. the truth-tables for our connectives)

\item Single Turnstile Tree $\vdash_{STD}$: tree-validity in STD \\ (i.e. premises and negated conclusion as root of a tree whose branches all close---recall that this means that $\Gamma \cup \{\enot \Theta\}$ is \emph{tree-inconsistent}) %is interesting how this idea---that unsatisfiability of premises and negated conclusion---is really at heart of our QND completeness proof!
% this means that $\Gamma \cup \{\enot \Theta\}$ is \emph{tree-inconsistent}: \\ There is a tree with this set as the root s.t. \emph{all branches close}

\item Single Turnstile Natural $\vdash_{QND}$: \emph{derivability} in QND

\end{itemize}
\end{frame}

\begin{frame}
\frametitle{A Tale of Three Turnstiles $\entails$ the semantic one}
%\large

\begin{itemize}[<+->]

%\item Recall that the double turnstile `$\entails$' stands for semantic entailment (aka logical consequence) within (classical) sentential logic QL. 

\item ``$\Gamma \entails \Theta$'' means that $\Gamma$ logically entails $\Theta$ \\ Whenever the premises in $\Gamma$ are true, the conclusion $\Theta$ is true 

\item Equivalently: there is no truth-value assignment (TVA) s.t. \\ $\Gamma$ is satisfied while $\Theta$ is false

\item Equivalently, this means that \emph{$\Gamma \cup \{\enot \Theta\}$ is unsatisfiable}: \\ no TVA satisfies the premises and negated conclusion  

\item We'll use this last fact A LOT in our proof that QND is complete! %completeness! 

\end{itemize}
\end{frame}


\fi %***********************************************************















%\iffalse %*************************************************************************

%\subsection{Completeness of System QND}

\subsection{Semantic vs. Syntactic Consistency}

\begin{frame}
\frametitle{Semantic vs. Syntactic Consistency}
%\large

\begin{itemize}[<+->]

\item As with SND, we appeal to two distinct notions of consistency

\item One is \emph{semantic}: %this is the notion we are already familiar with:

\item[] there is a QL-model that \emph{satisfies} every sentence in the set

\item We introduce a new \textbf{syntactic} notion of consistency relative to QND:

\item[] -- a set of QL sentences is \textbf{QND-consistent} provided that you can't derive contradictory sentences from it in QND

\item Core proof idea: we'll show that if a set of sentences is \textbf{QND-consistent}, then it is also semantically consistent (i.e. \emph{satisfiable}). So by the contrapositive: if a set is \textbf{\textcolor{OGlyallpink}{un}}satisfiable, then it is \textbf{\textcolor{OGlyallpink}{in}}consistent-in-QND. 

\end{itemize}
\end{frame}


\begin{frame}
\frametitle{Semantic: \emph{Satisfiable} (quantificationally consistent)}
%\large

\begin{itemize}[<+->]

\item Recall: a set of QL sentences is \emph{satisfiable} provided there is \\at least one QL-model $\mathfrak{M}$ that makes all of them true

%\begin{itemize}

\item This is a \textit{semantic} notion of consistency \\ (aka ``quantificational consistency'')  
%TF-consistent, (jointly) \emph{satisfiable}  

%\end{itemize}

\item Contrast this with the syntactic notion of \textbf{consistency in QND}:

\end{itemize}
\end{frame}




\begin{frame}
\frametitle{Syntactic: (In)consistent-in-QND (derivationally consistent)}
%\large

\begin{itemize}[<+->]

\item Let $\Gamma$ be a (possibly infinite) set of QL sentences 

\item \textbf{\textcolor{OGlyallpink}{Inconsistent-in-QND}}: from premises in $\Gamma$, we can derive contradictory formulas $R$ and $\enot R$ in the scope of the main scope line (i.e. in the scope of these premises)

\item \emph{Consistent-in-QND}: $\Gamma$ is not QND-inconsistent, i.e. there is no derivation from premises in $\Gamma$ resulting in contradictory formulas within the main scope

\item Other words we might use for these concepts: QND-inconsistent, derivationally-inconsistent, QND-consistent, etc.

\item Just remember: this syntactic notion has nothing to do with models or interpretations!

\end{itemize}
\end{frame}



\subsection{Proof Sketch}

\begin{frame}
\frametitle{Proof Sketch: Just like what we did for SL!}
%\large

\begin{itemize}[<+->]

\item Goal: prove the completeness of QL: for every QL sentence $\metav{P}$ and every set $\Gamma$ of QL sentences, if $\Gamma \entails \metav{P}$ then $\Gamma \vdash_{QND} \metav{P}$

\item So assume that $\Gamma \entails \metav{P}$. 

\item This means that $\Gamma \cup \{\enot \metav{P}\}$ is \textbf{\textcolor{OGlyallpink}{unsatisfiable}}:\\ no QL-model satisfies the premises and negated conclusion \\ (i.e. $\Gamma \cup \{\enot \metav{P}\}$ is \textit{semantically} inconsistent)

\item We now appeal to a \emph{Consistency lemma} that remains the heart of the enterprise: any QND-consistent set of QL sentences is satisfiable (i.e. semantically consistent)

\end{itemize}
\end{frame}

\begin{frame}
\frametitle{Proof Sketch: Using the consistency lemma}
%\large

\begin{itemize}[<+->]

\item \emph{Consistency lemma} (CL): any QND-consistent set of QL sentences is satisfiable, i.e. true in some QL-model $\mathfrak{M}$

\item \textbf{\textcolor{OGlyallpink}{Contrapositive}} of CL: any set of QL sentences that is \textcolor{OGlyallpink}{Un}satisfiable is QND-\textcolor{OGlyallpink}{In}consistent

\item From  $\Gamma \entails \metav{P}$ we know that $\Gamma \cup \{\enot \metav{P}\}$ is unsatisfiable

\item So by the contrapositive of CL, we see that $\Gamma \cup \{\enot \metav{P}\}$ is QND-inconsistent

\item This means that we can derive a pair of contradictory sentences $R$ and $\enot R$ from $\Gamma \cup \{\enot \metav{P}\}$! So using the power of negation elimination, we can derive $\metav{P}$ from $\Gamma$, i.e. $\Gamma \vdash_{QND} \metav{P}$. So we are `done'! 

\end{itemize}
\end{frame}

\begin{frame}
\frametitle{Negation Elimination Refresher (book's Exercise 11.4.2)}
%\large

\begin{itemize}

\item Claim: if $\Gamma \cup \{\enot \metav{P}\}$ is \textbf{\textcolor{OGlyallpink}{QND-inconsistent}}, then $\Gamma \vdash_{QND} \metav{P}$

\item Proof: starting with (finitely-many) premises $\Delta$ from $\Gamma$, introduce $\enot \metav{P}$ as a subproof assumption for negation elimination

\item Since $\Gamma \cup \{\enot \metav{P}\}$ is QND-inconsistent, we can derive a contradictory pair $R$ and $\enot R$ within the scope of sentences in $\Delta \cup \{\enot \metav{P}\}$

\item Then discharge this assumption $\enot \metav{P}$ by negation elimination, writing $\metav{P}$, now in the scope of $\Delta$. So $\Delta \vdash_{QND} \metav{P}$

\item Since $\Delta \subseteq \Gamma$, we have $\Gamma \vdash_{QND} \metav{P}$

%\item Recall that from a contradictory pair, we can derive anything! 



\end{itemize}
\end{frame}

\begin{frame}
\frametitle{Core subgoal: Prove consistency lemma (book's 11.4.2)}
%\large

\begin{itemize}[<+->]

\item So all we have to do is prove the \emph{consistency lemma}: any QND-consistent set of QL sentences is satisfiable

\item As with SL, we'll prove this lemma in several `stages':

\item The first two are straightforward: given a QND-consistent set $\Gamma$, we construct a \textbf{\textcolor{blue}{superset $\Gamma^{\ast}$}} that is \textit{\textcolor{blue}{maximally} QND-consistent} and \textit{\textcolor{purple}{existentially complete}} ($\exists$-complete)

\item In the third stage, we show that any $\exists$-complete, maximally QND-consistent set is \alert{satisfiable}: we use maximal consistency and $\exists$-completeness to construct a model that satisfies every sentence in $\Gamma^{\ast}$. \textit{Wrinkle}: we work in an extended language QL$'$!

\item Since by construction $\Gamma \subseteq \Gamma^{\ast}$, this QL$'$-model satisfies $\Gamma$ (in QL$'$)

\item From our QL$'$-model, we generate a QL-model that satisfies $\Gamma$  

%\item \footnotesize{(The idea in the third stage is similar to what we did with trees: use a syntactic consistency property to construct a TVA that satisfies a set of sentences: with trees we had `\textcolor{blue}{complete} \alert{open} branches'; here we have \textcolor{blue}{maximal}-\alert{QND-consistency})} 

%\item The final stage extends our membership lemma and induction! 

%\\ PS12 problems 2 and 3 provide practice with this tedium! 

%this is just like constructing a TVA that satisfies all of the sentences in 


\end{itemize}
\end{frame}

%\subsection{The straightforward part}

\subsection{Stage 0: $\exists$-Completeness and QL$'$}

\begin{frame}
\frametitle{Maximally QND-consistent (no longer enough!)}
%\large

\begin{itemize}[<+->]

\item A set $\Gamma^{\ast}$ of QL or QL$'$ sentences is \emph{maximally QND-consistent} provided that:

\begin{enumerate}[1.)]

\item $\Gamma^{\ast}$ is QND-consistent (i.e. can't derive contradictory sentences)

\item adding \textbf{any} additional sentence to $\Gamma^{\ast}$ would result in an QND-\textcolor{OGlyallpink}{inconsistent} set

\end{enumerate} 

\item i.e. for any $P \notin \Gamma^{\ast}$, $\{P\} \cup \Gamma^{\ast}$ is QND-\textcolor{OGlyallpink}{inconsistent}

\item Unlike with SL, maximal derivational consistency is no longer enough to ensure satisfiability

\item \small{Recall that our purely syntactic membership lemma is motivated by the truth-conditions for QL sentences: sentences belong to $\Gamma^{\ast}$ iff the relevant ``truth-condition pieces" belong to $\Gamma^{\ast}$ as well} 

\item  \small{To extend our membership lemma to quantified sentences, \\ we require that every existential sentence in $\Gamma^{\ast}$ has a substitution instance also in $\Gamma^{\ast}$. So we introduce a new property}:

%\item Motivation: it is straightforward (but tedious) to show that a maximally QND-consistent set is semantically consistent
% % the idea here is very similar to what we did in the completeness proof for the tree system: we rely on an `complete open' derivation (i.e. one w/ no contradictions in main scope) to construct a TVA that satisfies every sentence in the appropriate scope of the starting premises

%\item[] -- Moreover, every QND-consistent set is a subset of a maximally QND-consistent set. \item[] -- So we piggyback on an appropriate $\Gamma^{\ast}$ to show that any QND-consistent set $\Gamma$ is also \alert{satisfiable} %semantically consistent

\end{itemize}
\end{frame}

\begin{frame}
\frametitle{Existential-completeness: definition and motivation}
%\large

\begin{itemize}[<+->]

\item \emph{$\exists$-completeness}: a set $\Gamma$ of QL or QL$'$ sentences is \textit{existentially-complete} just in case for every sentence in $\Gamma$ of the form $\qt{\exists}{\script{x}} \metav{P}$, at least one substitution instance $\metav{P}[\script{c}/\script{x}]$ is in $\Gamma$ 

\bigskip 

\item Motivation: $\qt{\exists}{\script{x}} \metav{P}$ is true in a model iff some object $r \in D$ is a $\metav{P}$ 

%\item Ultimately, we'll use the recursive structure of QL$'$ and a membership lemma for $\Gamma^{\ast}$ that relies on ``truth-condition pieces"

\item To construct an $\exists$-complete set $\Gamma^{\ast}$, we need recourse to a countable infinity of unused constants. 

\item Otherwise, new substitution instances that we add could ``contradict" sentences already in $\Gamma$, spoiling QND-consistency 

\item \textit{Problem}: our starting $\Gamma$ might be infinite and so already use infinitely-many constants from QL. What are we to do? 
%syntactic--to--semantic strategy 


%We will ultimately be arguing that each sentence of $\Gamma^{\ast}$ is true in a model if and only if that sentence belongs to $\Gamma^{\ast}$ 

%
\end{itemize}
\end{frame}



\begin{frame}
\frametitle{It's a bird! It's a plane! It's \dots Language QL$'$???}
%\large

\begin{itemize}[<+->]

\item QL$'$ is exactly like QL except that we allow subscripted \textit{constants} to have \textbf{primed-indices}

\item e.g. $c_{11'}$, $b_{234'}$, $g_{2'}$ (`$\prime$'-symbol always at the end)

\item Unsubscripted constants remain the same: $a$ thru $v$ 

\item So QL$'$ just adds one new symbol `$\prime$', allowed to occur only at the end of indices for constants

\item The recursive structure of truth-in-QL$'$ is defined exactly the same as for QL (using our good friend, satisfaction semantics!)

\item Note that we do NOT allow primed indices on Predicates

\item Moral: reach \textit{for the stars}, \textbf{\textcolor{OGlyallpink}{not}} drugs

%\item So all semantic and syntactic notions for QL are defined for QL$'$

\end{itemize}
\end{frame}

\subsection{Stage 1: Constructing $\Gamma^{\ast}$}

\begin{frame}
\frametitle{Stage 1(i): first enumerate the sentences of QL$'$!}
%\large

\begin{itemize}[<+->]

\item Let $\Gamma$ be a QND-consistent set of QL sentences (possibly infinite)

\item To construct $\Gamma^{\ast}$, we first \emph{enumerate} the QL$'$ sentences, so that every QL$'$ sentence is associated with a unique positive integer $\{1, 2, 3, \dots \}$

%\item Analogy: we can enumerate words by length, using their alphabetical order to break ties %(so that 5-letter words beginning w/ `a' come before those w/ `b')

\item As with SL, stipulate an `alphabetical order' for QL$'$ symbols

\item $\enot, \eor, \eand, \eif, \eiff, (, ), 0, 1, \dots, 9, A, B, \dots, Z$, \emph{$a$}, $\dots, v,$ \emph{$w$}, $x, y, z,$ \emph{$\forall$}, \emph{$\exists$}, \emph{$^{\prime}$}

\item Assign each symbol an \textbf{index} between `10' and `84' (skip 17--19) %so 43 indices total: 55-10+1-3, since we actually skip 17, 18, and 19, starting 0 at `20' so that A starts at `30'

\item Then each QL$'$ sentence corresponds to a unique positive integer, constructed by replacing each symbol in the sentence with its index, from left to right. 

\item So with our ordering, `$A$' is the first sentence; `$B$' the second \dots up to $Z$, and then we hit $\enot A$ ($\mapsto 1030$), then $\enot B$ ($\mapsto 1031$), etc. 


\end{itemize}
\end{frame}

\begin{frame}
\frametitle{Recall what we did in SL to form $\Gamma^{\ast}$ Max.-SND-Consist.}
%\large

\begin{itemize}[<+->]

\item We considered the first sentence `$A$' in our enumeration. \\ If $A$ could be added to $\Gamma$ without the resulting set being SND-inconsistent, then we let  $\Gamma_1 := \Gamma \cup \{A\}$. 

\item Otherwise, let $\Gamma_1 := \Gamma$ (so that $\Gamma_1$ stays SND-consistent)

% \item In general, if $P_k$ is the $k$-th sentence in our enumeration, then $\Gamma_{k+1}$ is $\Gamma_k \cup \{P_k\}$ provided $\Gamma_k \cup \{P_k\}$ is SND-consistent; \\ otherwise, $\Gamma_{k+1}$ equals $\Gamma_k$

\item We proceeded to the 2nd sentence in our enumeration. \\ If it could be added to $\Gamma_1$ without the new set being SND-inconsistent, let $\Gamma_2$ be the result. Otherwise, let $\Gamma_2 := \Gamma_1$

\item $\Gamma^{\ast}$ was the result of `doing' this procedure for every SL sentence

\item Now we need to complicate matters a bit, to handle sentences of the form $\qt{\exists}{\script{x}} \metav{P}$ and ensure we add a suitable substitution instance whenever we can add $\qt{\exists}{\script{x}} \metav{P}$ while preserving QND-consistency

%\item More precisely, $\Gamma^{\ast} := \bigcup_{k=1}^{\infty} \Gamma_k$
\end{itemize}
\end{frame}

\begin{frame}
\frametitle{Building up $\Gamma^{\ast}$}
%\large

\begin{itemize}[<+->]

\item Given a QND-consistent set of QL sentences $\Gamma$, let $\Gamma_0 := \Gamma $ 

\item Consider the $k$-th sentence $P_k$ in our enumeration of QL\emph{$'$}

\item Define $\Gamma_{k+1} $ as follows:

\begin{enumerate}[i.)]

\item $\Gamma_k$ if the set $\Gamma_k \cup \{P_k \}$ is QND-\textbf{\textcolor{OGlyallpink}{INconsistent}}

\item $\Gamma_k \cup \{P_k \}$ if $P_k$ does NOT have the form $\qt{\exists}{\script{x}} \metav{Q}$, and $\Gamma_k \cup \{P_k \}$ is QND-consistent

\item $\Gamma_k \cup \{P_k, P^{\dagger}_k \}$ if $\Gamma_k \cup \{P_k \}$ is QND-consistent AND $P_k$ DOES have the form $\qt{\exists}{\script{x}} \metav{Q}$, where $P^{\dagger}_k$ is a substitution instance $\metav{Q}\substitute{\script{x}}{\script{c}}$, and $\script{c}$ is the \alert{alphabetically earliest constant} not in $P_k$ or any sentence in $\Gamma_k$

\item[] -- Such a $\script{c}$ is guaranteed to exist because  $\Gamma_0$ belongs to QL. 
\item[] -- So the countable-infinity of primed subscripted constants from QL$'$ are available at each stage if needed.   

\end{enumerate}

\item Then $\Gamma^{\ast} := \bigcup_{k=0}^{\infty} \Gamma_k$

\end{itemize}
\end{frame}

\subsection{Stage 2: $\Gamma^{\ast}$ is M-QND-C \& $\exists$-complete}

\begin{frame}
\frametitle{Stage 2: $\Gamma^{\ast}$ is maximally QND-consistent \& $\exists$-complete}
%\large

\begin{itemize}[<+->]

\item This requires proving three claims (from the definitions):

\bigskip

\begin{enumerate}[1.)]

\item $\Gamma^{\ast}$ is consistent in QND

\item Adding any additional sentence to $\Gamma^{\ast}$ would result in a \textbf{\textcolor{OGlyallpink}{QND-inconsistent}} set

\item For every QL$'$ sentence of the form $\qt{\exists}{\script{x}} \metav{Q}$ in $\Gamma^{\ast}$, at least one substitution instance $\metav{Q}\substitute{\script{x}}{\script{c}}$ belongs to $\Gamma^{\ast}$

\end{enumerate}

\bigskip

\item We prove these in turn

\end{itemize}
\end{frame}

\begin{frame}
\frametitle{Stage 2 (i): $\Gamma^{\ast}$ is QND-consistent}
%\large

\begin{itemize}[<+->]

\item Assume for \textit{reductio} that $\Gamma^{\ast}$ is inconsistent in QND

\item Then there would be a QND derivation with finite premise set $\Delta \subset \Gamma^{\ast}$ that derives a contradictory pair $R$ and $\enot R$

\item Since $\Delta$ is finite, there exists some $k+1 \in \mathbb{N}$ s.t. $\Delta \subset \Gamma_{k+1}$. \\ So then this $\Gamma_{k+1}$ would be \textcolor{OGlyallpink}{QND-inconsistent}. 

\item Yet, each $\Gamma_{k+1}$ is necessarily \alert{QND-consistent}: 

\bi

%\item Case where we don't add $P_k$: $\Gamma_{k+1} = $\Gamma_{k}$ which we assume is QND-consistent

\item If $P_k$ is not existential, it joins $\Gamma_{k+1}$ only if $\Gamma_k \cup \{P_k \}$ is QND-consistent---by condition (ii)

\item If $P_k$ is of the form $\qt{\exists}{\script{x}} \metav{Q}$, it joins only if $\Gamma_k \cup \{P_k \}$ is QND-consistent. 
\item[] -- It remains to show that $\Gamma_k \cup \{\qt{\exists}{\script{x}} \metav{Q}, \metav{Q}\substitute{\script{x}}{\script{c}} \}$ is QND-consistent

\item \textbf{\textcolor{highlightB}{Lemma}}: if $\script{c}$ does not occur in a QND-C set $\Gamma_k \cup \{ \qt{\exists}{\script{x}} \metav{Q} \}$, then $\Gamma_k \cup \{\qt{\exists}{\script{x}} \metav{Q}, \metav{Q}\substitute{\script{x}}{\script{c}} \}$ is QND-consistent

%where $P^{\dagger}_k$ is a substitution instance $\metav{Q}\substitute{\script{x}}{\script{c}}$

%\item For each sentence $P_k$, we ``added" $P_k$ (and possibly a substitution instance) to $\Gamma_{k+1}$ only if the resulting set was QND-consistent

%\item In general, if $P_k$ is the $k$-th sentence in our enumeration, then $\Gamma_{k+1}$ is $\Gamma_k \cup \{P_k\}$ provided that $\Gamma_k \cup \{P_k\}$ is QND-consistent; \\ otherwise, $\Gamma_{k+1}$ equals $\Gamma_k$ (so QND-consistent either way)

\ei

\item Hence, $\Gamma^{\ast}$ must be QND-consistent, on pain of \textit{reductio} 

% % Question: why do we need to manually add a substitution instance? Since the substitution instance will itself have a number $k$ in the enumeration, couldn't we just wait to consider it later? Perhaps the worry is that certain ways of building up would maintain QND-consistency but then systematically exclude every substitution instance of an existential in the set. Think about a minimal example where this occurs. We would presumably need a sentence of the form `everything is not a Q", but then how could we have added ``something is a Q". maybe this could work for existentials that are conditionals??? can't get it to easily work out as a minimal counterexample...

% Proof idea for the Lemma: since $\Gamma_k \cup \{ \qt{\exists}{\script{x}} \metav{Q} \}$ is QND-consistent, we cannot derive a contradictory pair R and ~R from any sentences in it. Now when we add in the substitution instance, clearly this could not itself be the R, since by assumption the instantiating constant does not appear in the previous set, so we could not have ~Q[c/x] in the set already. 
% So the contradictory pair, if one were to arise, would have to be the result of derivations that both cite Q[c/x]. So we would have some finite Delta \cup Q[c/x] proves R and some possibly different finite Delta' \cup Q[c/x] proves ~R. But we could just enlarge one of the Deltas to have a single finite set Delta \cup Q[c/x] that proves both R and ~R. then by soundness, Delta \cup Q[c/x] would have to semantically entail a contradiction
% so at this point, it would suffice to prove that Delta \cup Q[c/x] is satisfiable , i.e. semantically consistent.
%That would show that there is some model that makes Delta \cup Q[c/x] true and then would also have to make R and ~R true, which is impossible. 

%(which I guess is what the book tries to do with their lemma 11.1.10, which presumably uses locality!), but we still don't seem to know that $\Gamma_k \cup \{ \qt{\exists}{\script{x}} \metav{Q} \}$ is satisfiable, which is needed to apply 11.1.10
% Perhaps at this point we can appeal to the soundness of our system

%other proof idea: go through our QND rules and argue that in no case could adding Q[c/x] lead to a contradictory sentence pair, assuming that the sentences w/ (Ex)Q doesn't lead to a contradictory sentence pair! 

% % Perhaps an even better proof idea! Assuming that \gamma \cup (Ex)Q is QND-consistent, this means that whenever we assume the given substitution instance Q[c/x] to start a sub proof for existential elimination, we must not be able to derive contradictory R and ~R. Since if we could, we could derive any Psi from \gamma \cup (Ex)Q, including a Psi that contradicts another sentence in \gamma \cup (Ex)Q. So that would amount to showing that \gamma \cup (Ex)Q is QND-inconsistent, which contradicts our assumption that it is not.
% I think this proof idea totally works and does not rely on the semantic 11.1.10 at all! (Which I'm still not really sure how we could get that to work in this case, since we have not shown that \gamma \cup (Ex)Q is quantificationally/semantically consistent. 

%wrote this out on back of section 11.1 and it totally works!!! then i realized the the book student soLN manual actually includes EXACTLY the same proof in problem 6 of section 11.4. Perhaps they simply lost track of what they show in problem 6. So they should really cite exercise 11.4.6 rather than the lemma 11.1.10, since they don't actually use that lemma here. It also seems like they, in this edition, forgot to note that problem 6 actually has two parts. They just list part b, leaving out part a, on page 583.

\end{itemize}
\end{frame}


\begin{frame}
\frametitle{Stage 2 (ii): $\Gamma^{\ast}$ is \textcolor{orange}{maximally} QND-consistent}
%\large

\begin{itemize}

% %Note to Josh: Logic book p. 572 actually gives a direct proof of this claim that is probably better and more straightforward! although the proof below clarifies the defN of maximally QND-consistent! 

\item Assume for \textit{reductio} that $\Gamma^{\ast}$ weren't maximally QND-consistent, despite being QND-consistent

\item i.e. assume \textit{it is \textcolor{red}{not the case that}} for all other sentences, adding it to $\Gamma^{\ast}$ would result in a \textcolor{OGlyallpink}{QND-inconsistent} set

% % Relevant claim: for any additional sentence not already in $\Gamma^{\ast}$, adding Q to $\Gamma^{\ast}$ results in an QND-inconsistent set. So we negate this: there exists some Q such that when added to $\Gamma^{\ast}$, $\Gamma^{\ast}$  remains consistent. 

\item[] $\Rightarrow$ there exists a sentence $\metav{Q}$ that we could add to $\Gamma^{\ast}$ while preserving \alert{QND-consistency} (i.e. there is some sentence we neglected that could make $\Gamma^{\ast}$ a `bigger' QND-consistent set)
%relevant notion of `size' here is given by subset relation, rather than cardinality

\item Yet, $\metav{Q}$ would appear in our enumeration as some sentence $P_k$, `considered' at the $k$-th stage of our construction of $\Gamma^{\ast}$.

\item So if $\metav{Q}$ isn't in $\Gamma^{\ast}$, then this is because adding it `would have' made $\Gamma_k \subset \Gamma^{\ast}$ \textcolor{OGlyallpink}{QND-inconsistent}. \\ So $\{\metav{Q}\} \cup \Gamma^{\ast}$ must be QND-inconsistent (\textit{reductio}!)

%So $\{\metav{Q}\} \cup \Gamma_k$ and hence $\{\metav{Q}\} \cup \Gamma^{\ast}$ must be QND-inconsistent (\textit{reductio}!)




%So adding $\metav{Q}$ would result in $\Gamma^{\ast}$ being QND-inconsistent

\item So we can't add any $\metav{Q}$ to $\Gamma^{\ast}$ while preserving QND-consistency 

%So there can't be a sentence $\metav{Q}$ that we could add to $\Gamma^{\ast}$ while preserving QND-consistency 

\end{itemize}
\end{frame}

\begin{frame}
\frametitle{Stage 2 (iii): $\Gamma^{\ast}$ is \textcolor{purple}{$\exists$-complete}}
%\large

\begin{itemize}[<+->]

\item We simply need to show that for each sentence of the form $ \qt{\exists}{\script{x}} \metav{Q} \in \Gamma^{\ast}$, a substitution instance $\metav{Q}\substitute{\script{x}}{\script{c}}$ also belongs to $\Gamma^{\ast}$ 

\item Note that this is true by construction: each sentence of the form $ \qt{\exists}{\script{x}} \metav{Q}$ occurs in our QL$'$-enumeration: 

\item If we could have ``added" $ \qt{\exists}{\script{x}} \metav{Q}$ at the $k$-th stage while preserving QND-consistency, then we also added a substitution instance. 

\item This is so even if $ \qt{\exists}{\script{x}} \metav{Q}$ is already in $\Gamma_0 := \Gamma$, since by condition (iii) $\Gamma_{k+1} := \Gamma_k \cup \{\qt{\exists}{\script{x}} \metav{Q}, \metav{Q}\substitute{\script{x}}{\script{c}} \}$ which in this case would equal $\Gamma_k \cup \{\metav{Q}\substitute{\script{x}}{\script{c}} \}$ (since in this case, $\qt{\exists}{\script{x}} \metav{Q} \in \Gamma_k$)

%\item So if $\Gamma$ 

\end{itemize}
\end{frame}

\subsection{Stage 3: Model Construction}

% With my `primed-proof idea', where do numbers like 11 go? to 1'1'? perhaps alpha_11 would be better? , Although then I really am introducing infinitely many new constants. But I guess I'm doing the same thing here?
% I guess for my desired isomorphism, I need 11 to go to 11'. But I guess I can say this while still only introducing the 10 new symbols 0', 1', 2', 3'. Just put some wff clause on that primes come at most at the end of an index! This isn't really different in principle than saying that when we index constants, we can't write something like B_{1C0}, i.e. we disallow putting atomics in the index. so clearly, we have control over the syntax of the index for constants! 
% I could note that this is no different than introducing a new symbol alpha, that we can then index in the usual way, and use the index alpha itself as an index for constants.

\begin{frame}
\frametitle{Stage 3: The Maximal Consistency Lemma ($\approx$ book's 11.4.7)}
%\large

\begin{itemize}[<+->]

\item \textbf{\textcolor{purple}{$\exists$-C}} \textbf{\textcolor{blue}{Maximal} \alert{Consistency Lemma}}: every QL$'$ set that is maximally-QND-consistent and $\exists$-complete is satisfiable

\item (there exists a QL$'$-model that makes-true every sentence in $\Gamma^{\ast}$) \\ We construct this model, calling it ``$\mathfrak{M}^{\ast}$" ($\approx$book's ``$\textbf{I}^{\ast}$'')
%$\mathbf{A}^{\ast}$

\item Proof idea: since $\Gamma^{\ast}$ is M-QND-C, for any sentence $\metav{P}$, either $\metav{P} \in \Gamma^{\ast}$ or $\textcolor{red}{\enot \metav{P}} \in \Gamma^{\ast}$ (you're either in the club or your `\textcolor{red}{nemesis}' is!)
%you're out of the club!)

\item[] This holds in particular for each QL$'$-atomic sentence

\item Construct a QL$'$-model $\mathfrak{M}^{\ast}$ such that for each atomic QL$'$-sentence $\metav{A}$, $\mathfrak{M}^{\ast} \entails \metav{A}$ iff $\metav{A} \in \Gamma^{\ast}$

\item Then by the recursive structure of QL$'$ sentences, $\mathfrak{M}^{\ast} \entails \metav{P}$ iff $\metav{P} \in \Gamma^{\ast}$

\end{itemize}
\end{frame}

\subsubsection{The Membership Lemma}

\begin{frame}
\frametitle{Stage 3 (i): the Membership Lemma (book's 11.4.6)}
%\large

\begin{itemize}[<+->]

\item To induct on QL$'$, we first constrain $\Gamma^{\ast}$ membership

\item Basically, $\Gamma^{\ast}$ is \textit{THE} club with the MOST ANGELIC bouncer you've eva seen, who enforces maximal consistency. \item[] Before this *angel* lets a sentence into $\Gamma^{\ast}$, 
\item[] he checks who else is GOOD. You hear? 
%(the smaller fish, relative to our lexical ordering)

\item \emph{Membership Lemma} for club: if \metav{P} and \metav{Q} are QL$'$ sentences, then:

\begin{enumerate}[a.)]

\item $\enot \metav{P} \in \Gamma^{\ast}$ if and only if $\metav{P} \notin \Gamma^{\ast}$

\item $\metav{P} \eand \metav{Q} \in \Gamma^{\ast}$ if and only if both $\metav{P}\in \Gamma^{\ast}$ and $\metav{Q}\in \Gamma^{\ast}$

\item $\metav{P} \eor \metav{Q} \in \Gamma^{\ast}$ if and only if either $\metav{P}\in \Gamma^{\ast}$ or $\metav{Q}\in \Gamma^{\ast}$

\item $\metav{P} \eif \metav{Q} \in \Gamma^{\ast}$ if and only if either $\metav{P}\notin \Gamma^{\ast}$ or $\metav{Q}\in \Gamma^{\ast}$

\item $\metav{P} \eiff \metav{Q} \in \Gamma^{\ast}$ iff either (i) $\metav{P}\in \Gamma^{\ast}$ and $\metav{Q}\in \Gamma^{\ast}$ or (ii) $\metav{P}\notin \Gamma^{\ast}$ and $\metav{Q}\notin \Gamma^{\ast}$

\item \emph{$\qt{\forall}{\script{x}} \metav{P} \in \Gamma^{\ast}$} iff for each constant $\script{c}$, $\metav{P}\substitute{\script{x}}{\script{c}} \in \Gamma^{\ast}$

\item  \emph{$\qt{\exists}{\script{x}} \metav{P} \in \Gamma^{\ast}$} iff for at least one constant $\script{c}$, $\metav{P}\substitute{\script{x}}{\script{c}} \in \Gamma^{\ast}$

% In keeping with the forall X notation, I could presumably change this last condition to the following, using partial substitution instances:

% % \item[] alternatively: $\qt{\exists}{\script{x}}\metav{P}\substitutesome{\script{c}}{\script{x}} \in \Gamma^{\ast}$ iff $\metav{P}[\script{c}] \in \Gamma^{\ast}$ 
% but maybe this then comes into conflict with the condition a? Perhaps I need to restrict to those sentences P where c appears! 

\end{enumerate}

%\item Notice how these syntactic constraints mirror truth-conditions!

%\item \footnotesize{Moral: We all want to belong, but sometimes our enemies get in the way!}

%people get in the way!}


\end{itemize}
\end{frame}


\begin{frame}
\frametitle{Stage 3 (i): \emph{The Stairway} to heaven (book's 11.4.5)}
%\large

\begin{itemize}[<+->]

\item To prove the membership lemma's cases (a)--(g), we'll use another lemma (\textit{NB}: and she's buying, a lemma, to 
\item[]heavennnnnnnnn!):
%we'll use a lemma for a lemma

\item \emph{The Stairway}: if $\Gamma \vdash P$, and $\Gamma^{\ast}$ is a maximally QND-consistent superset of $\Gamma$, then $P \in \Gamma^{\ast}$ \\ (mnemonic: ``$\Gamma\vdash P$" pushes $P$ up to QL$'$-heaven!) %of our fictional club!
%Bouncer says yesssss)

\item Proof: first, assume that $\Gamma \vdash P$ (we'll use this fact below)
\bi

\item Next, assume for \textit{reductio} that $P \notin \Gamma^{\ast}$. Then since $\Gamma^{\ast}$ is maximally QND-consistent, $\Gamma^{\ast} \cup \{ P\}$ must be \textcolor{OGlyallpink}{inconsistent in QND}. 

\item Hence, by negation introduction, $\Gamma^{\ast} \vdash \enot P$

\item By assumption, $\Gamma \vdash P$, so also $\Gamma^{\ast} \vdash P$, since $\Gamma \subseteq \Gamma^{\ast}$

\item So $\Gamma^{\ast}$ derives both $P$ and $\enot P$. \textit{Reductio}! (since $\Gamma^{\ast}$ is M-QND-C)

\item Hence, if $\Gamma \vdash P$ and $\Gamma \subseteq \Gamma^{\ast}$, then $P$ must belong to $\Gamma^{\ast}$ 
\ei

\end{itemize}
\end{frame}


\begin{frame}
\frametitle{Membership Lemma: Cases (a)-(e)}
%\large

\begin{itemize}[<+->]

\item I have a feeling that\dots

\item WE'VE SEEN THIS incredible content BEFORE! (for SL)

\item see the next slide for a refresher 

\item \textit{Long story short}: There's a feeling I get
\item[] When I look to the west
\item[] And my spirit is crying for leaving

\end{itemize}
\end{frame}

\begin{frame}
\frametitle{Membership Lemma: Case (a)}
%\large

\begin{itemize}

\item \emph{Case (a)}: $\enot \metav{P} \in \Gamma^{\ast}$ if and only if $\metav{P} \notin \Gamma^{\ast}$

\item Two directions to prove:

\item[] $\Rightarrow$: Assume $\enot \metav{P} \in \Gamma^{\ast}$. Then if  $\metav{P}$ were in $\Gamma^{\ast}$, we could derive contradictory sentences. 

\item[] So since $\Gamma^{\ast}$ is QND-consistent, we must have $\metav{P} \notin \Gamma^{\ast}$

\item[] $\Leftarrow$: Assume $\metav{P} \notin \Gamma^{\ast}$. Then adding $\metav{P}$ to $\Gamma^{\ast}$ results in an QND-inconsistent set. Hence, there is some finite subset $\Delta \subset \Gamma^{\ast}$ s.t. $\Delta \cup \{ \metav{P}\}$ is QND-inconsistent (i.e. derives contradictory sentence pair). 

\item So by negation introduction, $\Delta \vdash \enot \metav{P}$

\item So by \emph{The Stairway}, $\enot \metav{P} \in \Gamma^{\ast}$


\end{itemize}
\end{frame}

\begin{frame}
\frametitle{Membership Lemma: Case (f) (something Universally new)}
%\large

\begin{itemize}[<+->]

\item \emph{Case (f)}: $\qt{\forall}{\script{x}} \metav{P} \in \Gamma^{\ast}$ iff for each constant $\script{c}$, $\metav{P}\substitute{\script{x}}{\script{c}} \in \Gamma^{\ast}$

\item Two directions to prove: 

\item[] $\Rightarrow$: Assume $\qt{\forall}{\script{x}} \metav{P} \in \Gamma^{\ast}$
\item[] -- Then \alert{for any} substitution instance $\metav{P}\substitute{\script{x}}{\script{c}}$, we note that $\qt{\forall}{\script{x}} \metav{P} \vdash_{QND} \metav{P}\substitute{\script{x}}{\script{c}}$ by $\forall E$. So by the Stairway, $\metav{P}[\script{c}/\script{x}] \in \Gamma^{\ast}$

% % think about whether we could proceed directly in the backwards case, or whether we really need to rely on the contrapositive. Proceeding directly, we would assume that for each constant, the relevant substitution instance belongs to the club. We would then aim to show that we could derive the universally quantified sentence. Maybe it is not clear to me why we couldn't just use one instance of universal introduction? I guess the worry is that this proof wouldn't be relying on the fact that we have all of the substitution instances, but seemingly that is really relevant! Otherwise, we're basically doing something just like existential introduction.

\item[] $\Leftarrow$: Assume $\qt{\forall}{\script{x}} \metav{P} \notin \Gamma^{\ast}$. Show that for some constant $\script{c}$, $\metav{P}\substitute{\script{x}}{\script{c}} \notin \Gamma^{\ast}$

\item[] -- Then $\enot \qt{\forall}{\script{x}} \metav{P} \in \Gamma^{\ast}$ by membership clause (a)

\item[] -- Then the derivation on p. 573 or---if I have no life---the derivation on the next slide, shows by the Stairway that $\qt{\exists}{\script{x}}\enot \metav{P} \in \Gamma^{\ast}$, i.e. $\enot \qt{\forall}{\script{x}} \metav{P} \vdash_{QND} \qt{\exists}{\script{x}}\enot \metav{P}$

% % Note that this derivation is very similar to the bonus/optional problem on problem set 10, where we show that ~(∀x)~Bx ⊃ (∃x)Bx is a tautology. so basically the same proof just replacing ``~B" w/ ``P"

% % Note that technically, with the syntax coming from for-allX, I need to modify the assumption/premises in the derivation to have the substitution and partial substitution instances. This perhaps has the benefit of showing that we don't necessarily need a full substitution instance. A partial substitution instance of the existential at the end would suffice. And that seems perhaps more in keeping with the spirit of being existentially-complete anyway.
% E.g. existential completeness does not require full substitution instances; presumably a partial substitution instance would be fine (?) [but maybe not if we need to get rid of all the free variables, to have a sentence???]

\item[] -- Then since $\Gamma^{\ast}$ is $\exists$-complete, there is at least one substitution instance $\enot \metav{P}\substitute{\script{x}}{b} \in \Gamma^{\ast}$. So by (a), $\metav{P}\substitute{\script{x}}{b} \notin \Gamma^{\ast}$, which is what we needed to show.


\end{itemize}
\end{frame}

\begin{frame}
\frametitle{Membership Lemma: Case (g) (it's getting existential) }
%\large

\begin{itemize}[<+->]

\item \emph{Case (g)}: $\qt{\exists}{\script{x}} \metav{P} \in \Gamma^{\ast}$ iff for at least one constant $\script{c}$, $\metav{P}\substitute{\script{x}}{\script{c}} \in \Gamma^{\ast}$

%\item Two directions to prove: 

\item $\Rightarrow$: Assume $\qt{\exists}{\script{x}} \metav{P} \in \Gamma^{\ast}$

\item[] Then since $\Gamma^{\ast}$ is $\exists$-complete, there is at least one substitution instance $\metav{P}\substitute{\script{x}}{\script{c}} \in \Gamma^{\ast}$

\item $\Leftarrow$: assume that $\metav{P}\substitute{\script{x}}{\script{c}} \in \Gamma^{\ast}$. 

\item[] Note that $\metav{P}\substitute{\script{x}}{\script{c}} \vdash_{QND} \qt{\exists}{\script{x}} \metav{P}$ by Existential introduction
% % note that here, I am relying on the syntax of the logic book rule for existential introduction. So perhaps I really should modify these membership Lemma's to match the syntax that we have been using on our rule sheet. But I do like the parallelism between the logic book's membership Lemma's cases f and g. 

\item[] So by the Stairway, $\qt{\exists}{\script{x}} \metav{P} \in \Gamma^{\ast}$

% % note that there's really no way, seemingly, for us to rely on a derivation rule like existential elimination to show that we can derive an instance! 
% It seems like we really do need to appeal to existential-completeness at this point! So existential completeness is needed for the membership Lemma, which ultimately builds in the right truth conditions into the syntactic structure of the Henkin model 

\item This completes the Membership Lemma, so we proceed to construct a model that satisfies $\Gamma^{\ast}$ (in virtue of being maximally-QND-consistent and $\exists $-complete)!


\end{itemize}
\end{frame}


\iffalse

\begin{frame}
\frametitle{Membership Lemma: Cases (b)-(e)}
%\large

\begin{itemize}[<+->]

\item See the book for cases (b) ($\metav{P} \eand \metav{Q}$) and (d) ($\metav{P} \eif \metav{Q}$)

\item Case (c) is PS12 \#2: $\metav{P} \eor \metav{Q} \in \Gamma^{\ast}$ if and only if either $\metav{P}\in \Gamma^{\ast}$ or $\metav{Q}\in \Gamma^{\ast}$

\item We skip case (e) ($\metav{P} \eiff \metav{Q}$) because \dots \pause \emph{YOLO}

\end{itemize}
\end{frame}

\fi 


\subsubsection{Model Construction}

\begin{frame}
\frametitle{Stage 3 (ii): Model construction (smart choices$=$lazy choices)}
%\large

\begin{itemize}[<+->]

\item A model's domain can be \textit{any} set of objects. Note that, conveniently, symbols \textit{are} objects (``words are labels on boxes'')

\item We define $\mathfrak{M}^{\ast} := (D, I^{\ast})$ as follows:

\begin{enumerate}

\item Let $D =$ the set of constant symbols in QL$'$, which includes all QL-constants (e.g. unprimed subscripted constants like $j_{22}$)
%the book uses positive integers, but to what end!

\item For the $0$-th place predicates, i.e. the sentence letters $B$, $I^{\ast}(B) = true$ iff $B \in \Gamma^{\ast}$

\item For each QL$'$-constant $\script{c}$, define $I^{\ast}(\script{c}) = \script{c}$ (each names itself)

\item For each $k$-place predicate $P$, $I^{\ast}(P):= Ext(P)$ includes all and only those $k$-tuples $\langle \script{c}_1, \dots, \script{c}_k \rangle $ such that $P\script{c}_1\dots\script{c}_k \in \Gamma^{\ast}$

% % to make the following entailment claim, perhaps I need to see how Gordon handles the SL atomic sentences within predicate logic. He must have some kind of convention regarding the extension of SL atomic sentences.
%\item[] (Note that Condition 4 entails Condition 2, but we make Cond. 2 explicit for clarity)
\end{enumerate}

\end{itemize}
\end{frame}

\begin{frame}
\frametitle{Some important properties of our Model $\mathfrak{M}^{\ast}$}
%\large

\begin{itemize}[<+->]

\item By condition 3, each individual constant refers to a \textit{unique} member of the domain, namely `itself' (now `objectified' in $D$!)

\item For each atomic sentence $\metav{A}$ of QL$'$, $\mathfrak{M}^{\ast} \entails \metav{A}$ iff $\metav{A} \in \Gamma^{\ast}$ \\ (follows from conditions 2--4)

\item By condition 3, every member of the domain is named by a constant, namely itself

%\item Note that we need condition 3 to set up this correspondence between truth of atomic's in the model and membership in $\Gamma^{\ast}$ . Otherwise, if two constants refer to the same object, then condition 4 would require that a corresponding tuple both be and not be in the extension of a predicate. 
% (See bottom of page 574)

\item We will occasionally rely on these properties in our induction






\end{itemize}
\end{frame}

\subsubsection{Induction on QL$'$ (we be clubbin')}

\begin{frame}
\frametitle{Stage 3 (iii): Induction on QL$'$ (i.e. we still be clubbin')}
%\large

%so since doing induction on QL$'$, I guess I need to formally define recursive structure of its wff, which is just like QL only allowing more constants, namely by allowing more index symbols JUST for the constants. i guess I ought to make that clear! 

\begin{itemize}[<+->]

\item Goal: construct a QL$'$-model $\mathfrak{M}^{\ast}$ that satisfies the $\exists$-C M-QND-C set $\Gamma^{\ast}$, i.e. that makes true everything in $\Gamma^{\ast}$ ($\mathfrak{M}^{\ast} \entails \Gamma^{\ast}$)

\item[] -- Suffices to construct $\mathfrak{M}^{\ast}$ s.t.  $\mathfrak{M}^{\ast} \entails \metav{P}$ iff $\metav{P} \in \Gamma^{\ast}$ 

\item[] Say that a sentence is ``\emph{clubbin'} " whenever it meets this property

\item We induct on the number of logical operators in a QL$'$ sentence: \\ i.e. the five connectives and two quantifiers (``conquans'')

%$\metav{I}(\metav{Q}) = True$ iff $\metav{Q} \in \Gamma^{\ast}$, $\forall  \metav{Q} \in$ QL. \\ Say that a sentence is ``\emph{clubbin'} " whenever it meets this property

%\item Define $\metav{I}$ such that $\metav{I}(B) = True$ iff atomic $B \in \Gamma^{\ast}$

\item \emph{Base case}: show that each QL$'$-sentence with \alert{zero} logical operators is clubbin' (i.e. the QL$'$-atomics be clubbin')

%atomic sentence is true on $\metav{I}$ iff it belongs to $\Gamma^{\ast}$ \\ (i.e. the atomics be clubbin')

\item (Strong) \emph{Induction hypothesis}: assume every QL$'$ sentence with $1$ to $k$-many operators is clubbin' 

\item Induction step: show that an arbitrary QL$'$ sentence with k+1-many operators is clubbin' 

%, i.e. a sentence is true on $\metav{I}$ iff it belongs to $\Gamma^{\ast}$

\end{itemize}
\end{frame}

\begin{frame}
\frametitle{Base Case (true by construction)}
%\large

\begin{itemize}[<+->]

\item Consider an arbitrary QL$'$-sentence $\metav{A}$ that has zero logical operators. (Two directions to show! ``iff")

\item Then $\metav{A}$ is either an atomic sentence letter $B$ or of the form $P\script{c}_1 \dots \script{c}_n$ for $n$-place predicate $P$. 

\item If a sentence letter, then by part 2 of our definition of $\mathfrak{M}^{\ast}$, $I^{\ast}(B) = true$ iff $B \in \Gamma^{\ast}$ (i.e. $\mathfrak{M}^{\ast} \entails B$ iff $B \in \Gamma^{\ast}$)

\item If $\metav{A}$ is of form $P\script{c}_1 \dots \script{c}_n$, then by definition $\mathfrak{M}^{\ast} \entails P\script{c}_1 \dots \script{c}_n$ iff $\langle \script{c}^D_1, \dots, \script{c}^D_n \rangle \in Ext(P)$. 

\item[] By part 4, $\langle \script{c}^D_1, \dots, \script{c}^D_n \rangle = \langle \script{c}_1, \dots, \script{c}_n \rangle \in Ext(P)$ iff $P\script{c}_1 \dots \script{c}_n \in \Gamma^{\ast}$

%\item Need to show \emph{TWO} directions!: 

%\item So both directions are met by construction 

\item We proceed to do induction using our QL$'$ induction schema: \\ an arbitrary sentence \metav{P} with k+1-many connectives has one of seven forms, coming from our seven operators 

\end{itemize}
\end{frame}

\begin{frame}
\frametitle{Induction on QL$'$: Cases 1--5}
%\large

\begin{itemize}

\item Cases 1--5 are just like what did to prove the completeness of SND

\item See the next slide for a refresher (\textit{mutatis mutandis})!

\item Need to show: \metav{P} be clubbin', i.e. $\metav{P}$ is true on $\mathfrak{M}^{\ast}$ iff $\metav{P} \in \Gamma^{\ast}$, \\ where \metav{P} is arbitrary QL$'$ sentence with k+1-many operators

\item \alert{Induction hypothesis}: assume every QL sentence with $1$ to $k$-many operators is clubbin' 

\item Case 1: \metav{P} has the form $\enot \metav{Q}$

\item Case 2: \metav{P} has the form $\metav{Q} \eand \metav{R}$

%\item Cases (b) ($\metav{P} \eand \metav{Q}$) and (d) ($\metav{P} \eif \metav{Q}$)

\item Case 3:  \metav{P} has the form $\metav{Q} \eor \metav{R}$

\item Case 4: \metav{P} has the form $\metav{Q} \eif \metav{R}$ 

\item Case 5: \metav{P} has the form $\metav{Q} \eiff \metav{R}$ 

%\item We skip case (e) ($\metav{P} \eiff \metav{Q}$) because \dots \pause \emph{YOLO}


\end{itemize}
\end{frame}

\begin{frame}
\frametitle{Induction on QL$'$: Case 1}
%\large

\begin{itemize}[<+->]

\item \emph{Case 1}: \metav{P} has the form $\enot \metav{Q}$, where since $\metav{Q}$ has $k$-operators, it is clubbin by the IH (i.e. $\mathfrak{M}^{\ast} \entails \metav{Q}$ if and only if $\metav{Q} \in \Gamma^{\ast}$)

%it is clubbin

\item NTS: (i) (the $\Rightarrow$direction) if $\mathfrak{M}^{\ast} \entails \metav{P}$ then $\metav{P} \in \Gamma^{\ast}$ and \\ 
(ii) (the $\Leftarrow$direction) if $\metav{P} \in \Gamma^{\ast}$, then $\mathfrak{M}^{\ast} \entails \metav{P}$
\item[] (\textit{Alternative (ii)}: show contrapositive: if $\mathfrak{M}^{\ast} \nentails \metav{P}$, then $\metav{P} \notin \Gamma^{\ast}$)

\item[$\Rightarrow$] if $\mathfrak{M}^{\ast} \entails \metav{P}$, then $\mathfrak{M}^{\ast} \nentails\metav{Q}$. Since \metav{Q} is clubbin', we have $\metav{Q} \notin \Gamma^{\ast}$ 
\item[] By Membership lemma (a), $\textcolor{red}{\enot \metav{Q}} \in \Gamma^{\ast}$, so $\metav{P} \in \Gamma^{\ast}$

%\item[] Since $\metav{Q}$ has $k$-connectives, by the IH it is clubbin. So 

\item[$\Leftarrow$] if $\metav{P} \in \Gamma^{\ast}$, then $\enot \metav{Q} \in \Gamma^{\ast}$. So by Membership lemma (a), $\metav{Q} \notin \Gamma^{\ast}$. 
\item[] Since \metav{Q} is clubbin', we have $\mathfrak{M}^{\ast} \nentails\metav{Q}$. (i.e.  \metav{Q} is false in $\mathfrak{M}^{\ast}$)
\item[] So by the truth conditions for negation, $\metav{P}$ is true in $\mathfrak{M}^{\ast}$, i.e.  $\mathfrak{M}^{\ast} \entails \metav{P}$


\end{itemize}
\end{frame}


\begin{frame}
\frametitle{Induction on QL$'$: Case 7 (existential quantifier)}
%\large

\begin{itemize}[<+->]

\item \emph{Case 7}: \metav{P} has the form $\qt{\exists}{\script{x}} \metav{Q}$ \\ (warning: ``\metav{Q}" is not a sentence, so sadly it can't be clubbin')

\item We will use Membership Lemma \emph{Case (g)}: \\ $\qt{\exists}{\script{x}} \metav{Q} \in \Gamma^{\ast}$ iff for at least one constant $\script{c}$, $\metav{Q}[\script{c}/\script{x}] \in \Gamma^{\ast}$

%where since $\metav{Q}$ has $k$-operators, it is clubbin by the IH (i.e. $\mathfrak{M}^{\ast} \entails \metav{Q}$ if and only if $\metav{Q} \in \Gamma^{\ast}$)

\bigskip

\item[$\Rightarrow$] Assume $\mathfrak{M}^{\ast} \entails \qt{\exists}{\script{x}} \metav{Q}$. (need to show that $\qt{\exists}{\script{x}} \metav{Q} \in \Gamma^{\ast}$)
\item[] -- Then by the truth-conditions for existential, there is some object $r \in D$ that satisfies $\metav{Q}$. 
\item[] -- `$r$' names object $r$, so substitution instance $\metav{Q}\substitute{\script{x}}{r}$ is true in $\mathfrak{M}^{\ast}$
\item[] -- This substitution instance has less than $k+1$-operators, so it is clubbin'. Hence, by the IH, $\metav{Q}\substitute{\script{x}}{r} \in \Gamma^{\ast}$ (since $\mathfrak{M}^{\ast} \entails \metav{Q}\substitute{\script{x}}{r}$)

\medskip

\item[] -- So by membership case (g), $\qt{\exists}{\script{x}} \metav{Q} \in \Gamma^{\ast}$ 


\end{itemize}
\end{frame}

\begin{frame}
\frametitle{Induction on QL$'$: Case 7 backwards direction}
%\large

\begin{itemize}[<+->]

\item \emph{Case 7}: \metav{P} has the form $\qt{\exists}{\script{x}} \metav{Q}$ %\\ (warning: ``\metav{Q}" is not a sentence, so it can't be clubbin')

\item Use Membership Lemma \emph{Case (g)}: \\ $\qt{\exists}{\script{x}} \metav{Q} \in \Gamma^{\ast}$ iff for at least one constant $\script{c}$, $\metav{Q}[\script{c}/\script{x}] \in \Gamma^{\ast}$

\bigskip

%where since $\metav{Q}$ has $k$-operators, it is clubbin by the IH (i.e. $\mathfrak{M}^{\ast} \entails \metav{Q}$ if and only if $\metav{Q} \in \Gamma^{\ast}$)

\item[$\Leftarrow$] Assume $\qt{\exists}{\script{x}} \metav{Q} \in \Gamma^{\ast}$. Show that $\mathfrak{M}^{\ast} \entails \qt{\exists}{\script{x}} \metav{Q}$

\item[] -- Then by membership case (g), there is at least one substitution instance $\metav{Q}\substitute{\script{x}}{\script{c}} \in \Gamma^{\ast}$, for some constant $\script{c}$

\item[] -- Since $\metav{Q}\substitute{\script{x}}{\script{c}}$ has fewer than $k+1$-operators, it is clubbin'. 
\item[] -- So by the Induction Hypothesis, $\mathfrak{M}^{\ast} \entails \metav{Q}\substitute{\script{x}}{\script{c}}$. 

\item[] -- Since `$\script{c}$' names object $\script{c}$, we see that $\script{c}$ satisfies $\metav{Q}$ in $\mathfrak{M}^{\ast}$

\item[] -- So by the truth-conditions for existentials, $\qt{\exists}{\script{x}} \metav{Q}$ is true in $\mathfrak{M}^{\ast}$

\end{itemize}
\end{frame}

\begin{frame}
\frametitle{Induction on QL$'$: Case 6 (universal quantifier)}
%\large

\begin{itemize}[<+->]

\item \emph{Case 6}: \metav{P} has the form $\qt{\forall}{\script{x}} \metav{Q}$ \\ (warning: ``\metav{Q}" is not a sentence, so it can't be clubbin')

\item We will use Membership Lemma \emph{Case (f)}: \\ \emph{$\qt{\forall}{\script{x}} \metav{Q} \in \Gamma^{\ast}$} iff for each constant $\script{c}$, $\metav{Q}\substitute{\script{x}}{\script{c}} \in \Gamma^{\ast}$


%$\qt{\exists}{\script{x}} \metav{Q} \in \Gamma^{\ast}$ iff for at least one constant $\script{c}$, $\metav{Q}[\script{c}/\script{x}] \in \Gamma^{\ast}$

%where since $\metav{Q}$ has $k$-operators, it is clubbin by the IH (i.e. $\mathfrak{M}^{\ast} \entails \metav{Q}$ if and only if $\metav{Q} \in \Gamma^{\ast}$)

\medskip

\item[$\Rightarrow$] Assume $\mathfrak{M}^{\ast} \entails \qt{\forall}{\script{x}} \metav{Q}$. Show that $\qt{\forall}{\script{x}} \metav{Q} \in \Gamma^{\ast}$

\bi

\item Then every object satisfies $ \metav{Q}$, so every substitution instance for every constant is true in $\mathfrak{M}^{\ast}$ (since each object is named by itself)

\item These $\metav{Q}\substitute{\script{x}}{\script{c}}$ are clubbin' by the IH, so they all belong to $\Gamma^{\ast}$. \\ So then by Membership Lemma case (f), $\qt{\forall}{\script{x}} \metav{Q} \in \Gamma^{\ast}$

\ei

\iffalse
\item[] -- Then by the truth-conditions for existential, there is some object $r \in D$ that satisfies $\metav{Q}$. 
\item[] -- `$r$' names object $r$, so substitution instance $\metav{Q}\substitute{\script{x}}{r}$ is true in $\mathfrak{M}^{\ast}$
\item[] -- This substitution instance has less than $k+1$-operators, so it is clubbin'. Hence, by the IH, $\metav{Q}\substitute{\script{x}}{r} \in \Gamma^{\ast}$ (since $\mathfrak{M}^{\ast} \entails \metav{Q}\substitute{\script{x}}{r}$)

\item[] -- So by membership case (g), $\qt{\exists}{\script{x}} \metav{Q} \in \Gamma^{\ast}$ 

\fi

\item[$\Leftarrow$] Assume $\qt{\forall}{\script{x}} \metav{Q} \in \Gamma^{\ast}$. Show that $\mathfrak{M}^{\ast} \entails \qt{\forall}{\script{x}} \metav{Q}$

\bi
\item Practice this yourself! 
\ei


\end{itemize}
\end{frame}

\begin{frame}
\frametitle{Upshots of our Induction}
%\large

\begin{itemize}[<+->]

\item Having handled every case (in spirit), we conclude that every sentence of QL$'$ is clubbin': 

\item For all QL$'$-sentences $\metav{P}$, $\mathfrak{M}^{\ast} \entails \metav{P}$ iff $\metav{P} \in \Gamma^{\ast}$ 

\item Hence, the QL$'$-model $\mathfrak{M}^{\ast}$ makes-true every sentence in  $\Gamma^{\ast}$, showing that this set is satisfiable

\item Hence, we have proven the \textbf{\textcolor{purple}{$\exists$-C}} \textbf{\textcolor{blue}{Maximal} \alert{Consistency Lemma}}: \\ every QL$'$ set that is maximally-QND-consistent and $\exists$-complete is satisfiable in QL\emph{$'$}

\item It remains to prove the \emph{Consistency Lemma}, i.e. that any QND-consistent QL-set (like our O.G. $\Gamma$) is satisfiable \emph{in QL}!

\end{itemize}
\end{frame}

\subsubsection{Stage 4? Salvation}

\begin{frame}
\frametitle{From satisfiability of $\Gamma^{\ast}$ to satisfiability of $\Gamma$}
%\large

\begin{itemize}[<+->]

\item We have shown that the maximally-QND-consistent and existentially complete $\Gamma^{\ast}$ is satisfiable in QL$'$ 

\item It remains to show that QND-consistent $\Gamma$ is satisfiable \emph{in QL}

\item i.e. we need a QL-model $\mathfrak{M}$ s.t. $\mathfrak{M} \entails \Gamma$

\item \textbf{\textcolor{OGlyallpink}{Hopes and dreams}}: by construction $\Gamma \subset \Gamma^{\ast}$, so $\mathfrak{M}^{\ast} \entails \Gamma$ in QL$'$. \\ But how are we to get a QL-model for $\Gamma$ from this??????

\item \emph{Salvation}: note that the model $\mathfrak{M}^{\ast}$ we constructed is \textit{not only} a QL$'$ model for $\Gamma^{\ast}$  \textit{BUT ALSO} a QL-model for  $\Gamma$!

\item Since the language of QL is contained in QL$'$, $\mathfrak{M}^{\ast} := (D, I^{\ast})$ maps all symbols of QL to objects in $D$

\item If you like, you can define a QL-model $\mathfrak{M} := (D, I)$ s.t. $I$ is the restriction of $I^{\ast}$ to unprimed constants in QL. Then $\mathfrak{M} \entails \Gamma$. 

\item $\Box$ Q.E.D. MOST BLESSED STUDENTS!!! (i.e. \textit{quod erat demonstrandum})

%\item This is simply because it meets the definition of a QL-model: we have a domain (which includes constants from QL$'$ but now functioning as `objects') and a map I that maps all the stuff in QL to stuff in the domain. 

%\item namely let $I$ be the restriction of the $I^{\ast}$ in $\mathfrak{M}^{\ast}$, restricted to symbols in QL, all of which appear in QL$'$. so we're done

\iffalse 

\item Given a QL$'$-model $\mathfrak{M}'=(D, I')$ that satisfies $\Gamma'$, we construct a QL-model $\mathfrak{M}= (D, I)$ with the same domain as follows: 
\item[] for each subscripted QL-constant $\script{c}_i$, $I(\script{c}_i) :=I'(\script{c}'_i)$, and for the non-subscripted constants $I$ and $I'$ agree.
\item[] For each $k$-place predicate $P$, $I(P) := I'(P)$ 

\item Note that $I$ is well-defined because by construction the unprimed and primed constants stand in one-to-one correspondence, \\ and $\Gamma$ and $\Gamma'$ have exactly the same predicates %(they differ at most in the constants they use). 


% this amounts to simpling `relabeling' any primed-constant with an unprimed constant. 

\item Claim: Then the QL-model $\mathfrak{M}$ satisfies $\Gamma$ 

\fi 

\end{itemize}
\end{frame}

\begin{frame}
\frametitle{Did we need to manually enforce $\exists $-completeness?}
%\large

\begin{itemize}[<+->]

\item In our condition (iii) for building up $\Gamma^{\ast}$, we manually enforced adding a substitution instance to our growing $\Gamma_{k+1}$ whenever we add an existential sentence. 

\item Some have wondered: shouldn't condition (ii) take care of this? Substitution instances are QL$'$ sentences, so they arise at some $k$ in our enumeration as well

\item Really the issue is the following: are there maximally QND-consistent sets that are NOT existentially-complete? If so, then our condition (iii) is not idle

\item So to show the necessity of our condition (iii) (or something like it), it suffices to construct a maximally QND-consistent set that has an existential sentence but no substitution instances for it. 

% To show the necessity of our condition, it suffices to construct 

\end{itemize}
\end{frame}

\begin{frame}
\frametitle{A maximally QND-consistent but existentially INCOMPLETE set}
%\large

\begin{itemize}[<+->]

\item Let $\Gamma_0$ be the set $\{ \qt{\exists}{x}\enot Fx \}$

\item Its substitution instances have the form $\enot F[c/x]$, e.g. $\enot Fc$. 

\item Notice that the `enemies' of these substitution instances always occur earlier in our enumeration, e.g. $Fc$ occurs before $\enot Fc$, $Fj'_{22}$ occurs before $\enot Fj'_{22}$ (the enemies always have one less symbol, so their index has two fewer digits)

\item So imagine that we dropped condition (iii) and built up $\Gamma^{\ast}$ using only conditions (i) and (ii). 

\item then $\Gamma^{\ast}$ would contain $\qt{\exists}{x}\enot Fx$ and every instance of an `enemy' substitution instance: $F\script{c}$ for all constants $\script{c}$

\item  $\Gamma^{\ast}$ would NOT contain a single substitution instance of $\qt{\exists}{x}\enot Fx$ because every time we hit a $\enot F\script{c}$ at its stage, $\Gamma_k$ would already contain its enemy $F\script{c}$, so that adding $\enot F\script{c}$ would result in a QND-inconsistent set. 





\end{itemize}
\end{frame}

\begin{frame}
\frametitle{Some remaining concerns about this construction}
%\large

\begin{itemize}[<+->]

\item Intuitively, you might think that a set containing $\qt{\exists}{x}\enot Fx$ and all these enemies $F\script{c}$ would be QND-inconsistent.

\item But it is not! Note that from existential elimination, we cannot start our subproof with a constant occuring in a premise, and these `enemies' would be premises. So there is no way to derive a contradiction
%we can derive a particular substitution instance of an existential 

\item Similarly, we can NOT go from an enemy to a contradictory universal $\qt{\forall}{x}Fx$ because the constant can't occur in a premise

\item Notice as well that the membership lemma would fail. We would have every instance of $F\script{c}$ but $\Gamma^{\ast}$ would NOT contain $\qt{\forall}{x}Fx$ because this sentence is QND-inconsistent with $\qt{\exists}{x}\enot Fx$ \\ (as an 8-line deduction shows)

\end{itemize}
\end{frame}

\iffalse

\begin{frame}
\frametitle{Reminder for Josh!}
%\large

\begin{itemize}[<+->]

\item If we actually make it this far, give hints on PS12 completeness question ($P \eor Q$)! or do Case (d), which is most analogous 

\item If the people don't want these hints, then clearly they're already complete!
% rather be complete!
%hence, then let's move on to completenes

\item ``The customer is always right!"

\item (Schematize this sentence in quantifier logic)
%The Corporation says: 


\end{itemize}
\end{frame}

\fi 


%\fi %*****************************************************************************


\iffalse 

\begin{frame}
\frametitle{Recall QND:}

  \begin{itemize}[<+->]
    \item d
    \emph{d} ($\enot$, $\eor$, $\eand$, $\eif$, $\eiff$)
  
  \begin{block}{blah}
    \begin{itemize}[<+->]
      \item[] d

  \item[] d

  \item[] d
\end{itemize} 
\end{block}

  \begin{definition}
  d
  \end{definition}


\end{itemize}
\end{frame}

\fi 