% !TeX root = ./8-handout.tex

\setcounter{section}{7}

\section{Intro. to Quantifier Logic}

\begin{frame}
%\large

\scriptsize{\tableofcontents}

\end{frame}

\subsection{The goals of QL}

\begin{frame}
  \frametitle{Limits of symbolization in SL}

\begin{itemize}[<+->]
  \item Consider the argument:
  \begin{earg}
    \item[] Greta is a hero.
    \item[\therefore] There is a hero.
  \end{earg}
  \item It's clearly valid: in any case in which Greta is a hero,
  someone (or something, at least) is a hero, so there must be a hero.
  \item But its symbolization in SL is invalid in SL:
  \begin{earg}
    \item[] $G$
    \item[\therefore] $H$
  \end{earg}
\end{itemize}
\end{frame}

\begin{frame}
  \frametitle{The problem}

  \begin{itemize}[<+->]
    \item Symbolization in SL allows us to break down sentences
    containing ``and,'' ``or,'' ``if--then'' and determine validity in
    virtue of these \emph{connectives}.
    \item Anything that can't be further broken down must be
    symbolized by sentence letters.
    \item That includes basic sentences like ``Greta is a hero,'' but also:
    \begin{itemize}
      \item Everyone is a hero.
      \item No one is a hero.
      \item All heroes wear capes.
    \end{itemize}
  \end{itemize}
\end{frame}

\begin{frame}
  \frametitle{The goals of \emph{Quantifier Logic} (QL)}
  \begin{itemize}[<+->]
    \item Finer-grained symbolization
    \item Augments SL (all of SL and \textit{more}!)
    \item Allows for precise semantics (like truth tables for SL)
    \item Works with natural deduction (add new rules!)
    \item Be simple \& expressive (only a few new symbols!)
    %\item New language: \emph{quantifier logic QL}
  \end{itemize}
\end{frame}

\begin{frame}
  \frametitle{The goals of QL}

  \begin{itemize}[<+->]
    \item Consider the valid argument:
      \begin{earg}
        \item[] Greta is a hero.
        \item[] Greta does not wear a cape.
        \item[\therefore] Not all heroes wear capes.
      \end{earg}
    \item We'll need to connect the occurrences of the name
    ``Greta'' in the premises
    \item We'll need to connect ``hero'' in the premise and conclusion
    \item We want to retain using the symbol `\enot{}' for ``not''
    \item Ultimately, we'll want our argument-symbolization to have a proof
  \end{itemize}
\end{frame}

\subsection{Beginning symbolization in QL}

\begin{frame}
  \frametitle{First steps: names (a.k.a. `constants')}

  \begin{itemize}[<+->]
    \item Purpose of a \emph{proper name}: to pick out a single, specific thing.
    \item (Contrast with common nouns like ``hero'' or ``rock'' which pick out collections of things)
    \item For simplicity, we'll only consider names that pick out a \textbf{specific object} (often within a hypothetical case we're considering)
    \item Later on, we'll be able to deal with other expressions that play a similar role to names, e.g., ``the president of the USA''
    \item In QL, names are symbolized by lowercase letters $a$--$v$ \\ (allowing natural number sub-scripts, e.g. $m_1$, $t_{2022}$)
  \end{itemize}
\end{frame}

\begin{frame}
  \frametitle{First steps: predicates (including properties and relations)}

  \begin{itemize}[<+->]
    \item Remove a name from a sentence. What's left over is a \emph{predicate}:\\
      \begin{earg}
        \item[] Greta is a hero
        \item[$\Rightarrow$] \gap{x} is a hero. (i.e. property of being a hero)
      \end{earg}
      \begin{earg}
        \item[] Greta admires Autumn
        \item[$\Rightarrow$] \gap{x} admires \gap{y}. (i.e. relation of x admiring y)
      \end{earg}
    \item In QL, predicates are symbolized using uppercase letters
    $A$--$Z$ plus a number of argument slots (marked with variables), \\
    e.g., $H\qv{x}$ or $A\qr{x}{y}$.
    \item Argument slots correspond to blanks.
  \end{itemize}
\end{frame}

\begin{frame}
  \frametitle{Symbolization keys}

  \begin{itemize}[<+->]
    \item \emph{Names}/\emph{constants}: lowercase letters for proper names
    \item \emph{Predicates}: uppercase letters with variables marking
    blanks
    \begin{ekey}
    \item[a] Autumn
    \item[g] Greta
    \item[H\qv{x}] \gap{x} is a hero
    \item[V\qv{x}] \gap{x} is a villain 
    \item[I\qv{x}] \gap{x} inspires
    \item[C\qv{x}] \gap{x} wears a cape
    \item[W\qr{x}{y}] \gap{x} welcomed \gap{y}
    \item[A\qr{x}{y}] \gap{x} admires \gap{y}
    \item[Y\qr{x}{y}] \gap{x} is younger than \gap{y} 
    \end{ekey}
    \item \emph{Domain}: the non-predicate objects we're talking about in a context---also called grandly the `\emph{Universe of Discourse}' (UD)\\
  \hspace{5em}  e.g., people alive in \year{} %somewhat sad that this will auto-update every year i teach logic...
  \end{itemize}
\end{frame}

\begin{frame}
\frametitle{Symbolization of Sentences without Quantifiers}

\begin{itemize}[<+->]
\item Basic sentences: predicates with names replacing variables.

\bigskip

\begin{itemize}[<+->]
  \item Greta is a hero: \alert{$H\qv{g}$}
  \item Greta admires Autum: \alert{$A\qr{g}{a}$}
\end{itemize}

\bigskip

\item Combinations using connectives:

\bigskip

\begin{itemize}[<+->]
  \item Greta and Autumn are heroes: \alert{$H\qv{g} \eand H\qv{a}$}
  \item If Autumn admires Greta, then Autumn is a hero: \alert{$A\qr{a}{g} \eif H\qv{a}$}
\end{itemize}
  \end{itemize}
\end{frame}

\begin{frame}
\frametitle{Symbolization of Pronouns}

\begin{itemize}[<+->]
\item Replacing pronouns by antecedents:
\begin{itemize}[<+->]
  \item If Autumn is a hero, Greta admires \textbf{her}: \alert{$H\qv{a} \eif A\qr{g}{a}$}
  \item Greta doesn't admire \textbf{herself}: \alert{$\enot A\qr{g}{g}$} (but she should!)
  \item Greta and Autumn welcomed \textbf{each other}: \alert{$W\qr{g}{a} \eand W\qr{a}{g}$}
\end{itemize}

\bigskip

\item Modifiers:
\begin{itemize}[<+->]
  \item Autumn is an \textbf{inspiring hero}:\\
  i.e. Autumn inspires, and she is a hero: \alert{$I\qv{a} \eand H\qv{a}$}
  \item Greta is a \textbf{hero who doesn't wear a cape}:\\
   Greta is a hero, and it's not the case that Greta wears a cape:
   \alert{$H\qv{g} \eand \enot C\qv{g}$}
\end{itemize}
\end{itemize}
\end{frame}

\begin{frame}
\frametitle{Mind the Modifiers!}

\begin{itemize}[<+->]
\item `Greta is an international hero':
\begin{itemize}
  \item Can't be paraphrased as\\
  ``Greta is international and a hero.''
  \item So ``\gap{} is an international hero'' needs its own
  predicate
\end{itemize}
\item `The \href{https://en.wikipedia.org/wiki/Piltdown_Man}{Piltdown Man} is a fake fossil'
\begin{itemize}[<+->]
  \item Can't be paraphrased as\\
  ``The Piltdown Man is fake and a fossil.''
  \item Since ``fake'' and other privative adjectives (``pretend,''
  ``fictitious'') deny the property that they modify!  (`fake news' isn't news!)
  %(a `fake nose' is not a nose!)
  %`imply the opposite', but not sure what this really means 
  %more precisely: imply an absence of a quality 
\end{itemize}
\end{itemize}
\end{frame}

\begin{frame}
\frametitle{Examples}

\begin{itemize}[<+->]
  \item Autumn and Greta are inspiring heroes.\\
  \item[] \alert{$(I\qv{a} \eand H\qv{a}) \eand (I\qv{g} \eand H\qv{g})$}
  \item Greta admires Autumn but not herself.\\
  \item[] \alert{$A\qr{g}{a} \eand \enot A\qr{g}{g}$}
  \item Greta inspires only if Autumn does.
  \item[] \alert{$I\qv{g} \eif I\qv{a}$}
  \item Greta and Autumn welcomed each other.
  \item[] \alert{$W\qr{g}{a} \eand W\qr{a}{g}$} 
  \item Greta is older than Autumn.
  \item[] \alert{$Y\qr{a}{g}$}  (i.e. Autumn is \textit{younger than} Greta)
  \item One of Greta and Autumn welcomed the other.
  \item[] At least one: \hspace{6em} Exactly one:\\
  \alert{$W\qr{g}{a} \eor W\qr{a}{g}$} \hspace{6em} \alert{$(W\qr{g}{a} \eor W\qr{a}{g}) \eand \enot(W\qr{g}{a} \eand W\qr{a}{g})$}
\end{itemize}
\end{frame}

\subsection{The existential quantifier}

\begin{frame}
\frametitle{Existential quantifier (something or other)}

\begin{itemize}[<+->]
  \item In English: ``something,'' ``someone,'' ``there is \dots''
  \item For instance:
  \begin{itemize}[<+->]
    \item \emph{Someone} wears a cape.
    \item \emph{There is} a hero.
    \item \emph{Something} inspires.
  \end{itemize}
  
  \bigskip
  
  \item Note: often goes where names and pronouns are placed
  \item But works differently from names (``something'' doesn't pick
  out a unique, specific object).
\end{itemize}
\end{frame}

\begin{frame}
  \frametitle{How (not) to symbolize ``something''}

  \begin{itemize}[<+->]
    \item Idea(?): introduce a special term `$\mathit{sg}$' for `a something'?
    \item Problem: now we can't distinguish between
    
    \medskip
    
    \begin{itemize}[<+->]
      \item Someone is a hero and wears a cape.
      \item Someone is a hero and someone wears a cape.
    \end{itemize}
       \medskip
    as both would be symbolized by `$H(\mathit{sg}) \eand
    C(\mathit{sg})$'.
    \item Better idea: symbolize (complex) \emph{properties} and introduce a
    notation for expressing that properties are \emph{instantiated}
  \end{itemize}
\end{frame}

\begin{frame}
\frametitle{Expressing properties (and relations)}

  \begin{itemize}[<+->]
    \item One-place predicates \emph{express} properties, e.g.,
    \begin{itemize}[<+->]
      \item $H\qv{x}$ expresses property ``being a hero'' 
      \item $I\qv{x}$ expresses ``is inspiring'' (`x' is a variable)
    \end{itemize}
    \bigskip
    \item Combinations of predicates (with connectives, names) can
    express \emph{derived} properties, e.g.,
    \begin{itemize}[<+->]
      \item $A\qr{x}{g}$ expresses ``admires Greta''
      \item $W\qr{a}{x}$ expresses ``is welcomed by Autumn''
      \item $H\qv{x} \eand C\qv{x}$ expresses ``is a hero who wears a cape''
    \end{itemize}
        \bigskip
    \item Note: all contain a \emph{single} variable $x$
  \end{itemize}
\end{frame}

\begin{frame}
  \frametitle{The existential quantifier $\exists$}

  \begin{itemize}[<+->]
    \item Symbol for ``there is'': \emph{$\exists$}
    \item Combine `$\exists$' with an expression for a property (e.g., $(H\qv{x} \eand
    C\qv{x})$) to say ``something (or someone) has that property''
    \item Put the variable that serves as a marker for the gap
    after $\exists$. E.g.,
    \[ \alert{\qt{\exists}{x}\,(H\qv{x} \eand C\qv{x})} \]
    says ``Someone is a hero and wears a cape''
    \item MUST always wrap a quantifier and the variable it `binds' within \emph{parentheses}: $\qt{\exists}{y}$; \textit{Carnap} will require this!!!
  \end{itemize}
\end{frame}

\begin{frame}
\frametitle{Quantifiers and variables}
  %\begin{itemize}[<+->]
   % \item 
     \vspace{-1em}
    \begin{align*}
   \text{Compare:      } \qt{\exists}{x}\,(H\qv{x} & \eand C\qv{x}) \text{ to }\\
    \qt{\exists}{x} \, H\qv{x} &\eand \qt{\exists}{x}\, C\qv{x}
    \end{align*}
    \vspace{-1em}
\begin{itemize}[<+->]
    \item In first case, \textit{the same person} must be a hero and wear a cape.
    \item In second case, one person can be the hero and another (possibly different) person wears a cape.
    \item Instances of `$\qt{\exists}{x}$' separated by other connectives are independent, even if they bind the same variable~$x$. \\ e.g. there's no difference in meaning between the following: 
    \begin{align*}
     & \qt{\exists}{x} \, H\qv{x} \eand \qt{\exists}{x}\, C\qv{x} \; \; \text{ \textbf{vs.} }\\
     & \qt{\exists}{x} \, H\qv{x} \eand \qt{\exists}{y}\, C\qv{y}
    \end{align*}
    \item But we'll \textcolor{red}{never write} `$\qt{\exists}{x}\qt{\exists}{x} (Hx \eand Cx)$' 
  \end{itemize}
\end{frame}

\begin{frame}
\frametitle{The domain (UD) and quantifiers}

\begin{itemize}[<+->]
  \item Symbolization key gives a domain of objects being talked about.
  \item Quantifier \emph{ranges over} this `universe of discourse' (UD).
  \item That means: $\qt{\exists}{x}\, \dots x \dots$ is true iff some object
  \emph{in the domain} has the property expressed by $\dots x \dots$.
  \item Domain makes a difference: Consider $\qt{\exists}{x}\, W\qr{x}{g}$.
  \begin{itemize}
    \item True if someone welcomed Greta (say, Autumn did).
    \item Now take the domain to include only Greta.
    \item Relative to that domain, $\qt{\exists}{x}\, W\qr{x}{g}$ is true iff
    Greta welcomed herself (e.g. to the left-over chocolate fondue!) 
  \end{itemize}
\end{itemize}
\end{frame}

\begin{frame}
\frametitle{Quantifier restriction in English}

\begin{itemize}[<+->]
  \item ``something'' and ``someone'' work grammatically like singular
  terms (they can go where names can also go).
  \item ``some'' (on its own) does not: it is a \emph{determiner} and
  needs a \emph{complement}, e.g.,
    \medskip
  \begin{itemize}
    \item a common noun (``some hero''), or
    \item a noun phrase (``some admirer of Greta'').
  \end{itemize}
  
  \medskip
  \item ``some'' + complement works grammatically like ``someone'',
  e.g., ``\emph{Some hero} wears a cape''
  \item General form: ``Some $F$ is $G$.''
\end{itemize}
\end{frame}

\begin{frame}
\frametitle{Quantifier restriction in QL}

\begin{itemize}[<+->]
  \item ``Some $F$ is $G$'' \emph{restricts} the ``something''
  quantifier to $F$s.
  \item We could (and linguists often do) mark restrictions in the
  quantifier, e.g., $(\qt{\exists}{x}\colon F\qv{x}) G\qv{x}$
  \item We won't because we can do without this additional notation
  \item ``Some $F$ is $G$'' is true iff there is something which
  is both $F$ and also $G$, so:
  \item ``Some $F$ is $G$'' can be symbolized as
  \[ \qt{\exists}{x}(F\qv{x} \eand G\qv{x}) \]
  \item We'll also symbolize the plural form this way\\ (``Some $F$s are
  $G$s'').
  \item And more generally (most) sentences of the form:\\
  ``$G$(some $F$)'' or ``$G$(something that $F$s)''.
\end{itemize}
\end{frame}

\begin{frame}
\frametitle{Examples}

\begin{itemize}[<+->]
  \item \alert{Some hero} wears a cape.\\
  \alert{Some heroes} wear capes.\\
  \item[] \alert{$\qt{\exists}{x}(H\qv{x} \eand C\qv{x})$}
  \item \alert{Someone who wears a cape} welcomed Greta.\\
  \item[] \alert{$\qt{\exists}{x}(C\qv{x} \eand W\qr{x}{g})$}
  \item Greta admires \alert{some hero who wears a cape}.
  \item[] \alert{$\qt{\exists}{x}((H\qv{x} \eand C\qv{x}) \eand A\qr{g}{x})$}
  \item Autumn welcomed \alert{someone who welcomed Greta}.
  \item[] \alert{$\qt{\exists}{x}(W\qr{x}{g} \eand W\qr{a}{x})$}
\end{itemize}
\end{frame}

\subsection{The universal quantifier}

\begin{frame}
\frametitle{Universal quantifier}

\begin{itemize}[<+->]
  \item ``\emph{Something} is $F$'' is true iff \emph{at least one} element of domain
  is~$F$.
  \item ``\emph{Everything} is $F$'' is true iff \emph{every element} of the domain
  is~$F$.
  \item In QL: $\qt{\forall}{x}\, F\qv{x}$.
  \item E.g.:
    \begin{itemize}
      \item ``Everyone wears a cape'': \alert{$\qt{\forall}{x}\,C\qv{x}$}
      \item ``Everyone welcomed Greta or Autumn'': \alert{$\qt{\forall}{
      x}(W\qr{x}{g} \eor W\qr{x}{a})$}
    \end{itemize}
\end{itemize}
\end{frame}

\begin{frame}
\frametitle{Universal determiners: all, every, any}

\begin{itemize}[<+->]
  \item Determiners with universal meaning: \emph{all, every, any}.
  \item Take complements (just like ``some'' does), e.g.,
      \begin{itemize}[<+->]
        \item \alert{Every hero} inspires.
        \item \alert{All heroes} inspire.
        \item \alert{Any hero} inspires.
      \end{itemize}

\bigskip

  \item These are true in the same cases (i.e. they are synonymous).
  \item ``Every $F$ is $G$'' is true iff everything \alert{which is an
  $F$} is $G$.
  \item Watch out for ``any'': not always universal.
  %josh: try to get an example! to have off hand; maybe `it's going to happen any day now' meaning `there is some day in the future when it's going to happen' 
\end{itemize}
\end{frame}

\begin{frame}
\frametitle{Restricted $\forall$ in QL}

  \begin{itemize}[<+->]
    \item Suppose we can symbolize two properties `$F$' and `$G$'.
    \item How do we symbolize ``Every $F$ is $G$''?
    \item Initial Ideas (only one of which is correct):
      \begin{itemize}[<+->]
        \item \alert{$\qt{\forall}{x}(F\qv{x} \eand G\qv{x})$}
        \\\uncover<7->{If true, everything must be $F$.\\
        So can be false when ``Every $F$ is $G$'' is true.}
        \item \alert{$\qt{\forall}{x}(F\qv{x} \eor G\qv{x})$}
        \\ \uncover<8->{True if everything is $F$ (without being $G$).\\
        So can be true when ``Every $F$ is $G$'' is false.}
        \item \alert{$\qt{\forall}{x}(F\qv{x} \eif G\qv{x})$}
        \\ \uncover<9>{If $x$ is $F$, $x$ must also be $G$.\\
        (If $x$ is not $F$, doesn't matter if it's $G$ or not.)}
      \end{itemize}
  \end{itemize}
\end{frame}

\begin{frame}
\frametitle{Symbolizing ``all $F$s are $G$s'' (memorize this!)}

Symbolize the following as \emph{\[\qt{\forall}{x}(F\qv{x} \to G\qv{x})\]}

\begin{itemize}
  \item All $F$s are $G$s.
  \item Every $F$ is $G$.
  \item Any $F$ is $G$.
\end{itemize}
\end{frame}

\begin{frame}
\frametitle{Examples}

\begin{itemize}[<+->]
  \item \emph{Every hero} wears a cape.\\
  \emph{All heroes} wear capes.\\
  \item[] \alert{$\qt{\forall}{x}(H\qv{x} \eif C\qv{x})$}
  \item \emph{Every hero who wears a cape} welcomed Greta.\\
  \item[] \alert{$\qt{\forall}{x}((H\qv{x} \eand C\qv{x}) \eif W\qr{x}{g})$}
  \item Greta and Autumn admire \emph{anyone who wears a cape}.
  \item[] \alert{$\qt{\forall}{x}(C\qv{x} \eif (A\qr{g}{x} \eand A\qr{a}{x}))$}
  \item Autumn welcomed \emph{everyone who welcomed Greta}.
  \item[] \alert{$\qt{\forall}{x}(W\qr{x}{g} \eif W\qr{a}{x})$}
  \item All heroes and villains welcomed Greta (a tricky one!). 
  \item[] \alert{$\qt{\forall}{x}((H\qv{x}
  \mathbin{\colorbox{highlightbg}{\eor}} V\qv{x}) \eif W\qr{x}{g})$}
  \item[] equivalent to \alert{$\qt{\forall}{x}( (H\qv{x} \eif W\qr{x}{g}) \eand (V\qv{x} \eif W\qr{x}{g}))$}
  %perhaps i should think about how `or' distributes over conditional. e.g. is (PvQ)>R equivalent to (P>R) & (Q>R)? 
  %former is false whenever P or Q true and R false. Latter is also false exactly when P or Q is true and R is false.  verified equivalence with an online truth table 
    %so could presumably split this into two separate universal quantified expressions with an ampersand connecting them!
\end{itemize}
\end{frame}

\subsection{`No', `only', `a', `some', and `any'}

\begin{frame}
\frametitle{No $F$ is $G$}

\begin{itemize}[<+->]
  \item ``\emph{No $F$s are $G$s}'' can be paraphrased as 
\begin{itemize}[<+->]
  \item ``\emph{Every} $F$ is \emph{not}-$G$,'' or as
  \item ``\emph{Not: some} $F$ is $G$.''
  \end{itemize}
\bigskip
  \item So symbolize it using:
\begin{itemize}
  \item \emph{$\qt{\forall}{x}(F\qv{x} \to \enot G\qv{x})$} or
  \item \emph{$\enot\qt{\exists}{x}(F\qv{x} \eand G\qv{x})$} (i.e. `it is not the case that there is something that is both an F and a G')
\end{itemize}
\end{itemize}

\end{frame}

\begin{frame}
\frametitle{Examples}

\begin{itemize}[<+->]
  \item \emph{No hero} wears a cape.\\
  \emph{No heroes} wear capes.\\
  \item[] \alert{$\qt{\forall}{x}(H\qv{x} \eif \enot C\qv{x})$}
  \item \emph{No hero who wears a cape} welcomed Greta.\\
  \item[] \alert{$\qt{\forall}{x}((H\qv{x} \eand C\qv{x}) \eif \enot W\qr{x}{g})$}
  \item Greta admires \emph{no one who wears a cape}.
  \item[] \alert{$\enot\qt{\exists}{x}(C\qv{x} \eand A\qr{g}{x})$}
  \item Autumn welcomed \emph{no one who welcomed Greta}.
  \item[] \alert{$\enot\qt{\exists}{x}(W\qr{x}{g} \eand W\qr{a}{x})$}
\end{itemize}
\end{frame}


\begin{frame}
\frametitle{Only $F$s are $G$}

\begin{itemize}[<+->]
\item When is ``Only $F$s are $G$s'' false?
\item When there is a \emph{non}-$F$ that is a $G$.
\item So symbolize it as
\[
  \alert{\enot\qt{\exists}{x}(\enot F\qv{x} \eand G\qv{x})}
  \]
\item Or, paraphrase it as: ``Any $x$ is $G$ \emph{only if} it is $F$''
\item So another symbolization is:
\[
  \alert{\qt{\forall}{x}(G\qv{x} \to F\qv{x})}
\]
\item[] i.e. being an F is a necessary condition for being a G: \\ if it's not an F, then it can't be a G 
\end{itemize}
\end{frame}

\begin{frame}
\frametitle{Examples}

\begin{itemize}[<+->]
  \item Only heroes wear capes: \\
  \item[] \alert{$\qt{\forall}{x}(C\qv{x} \eif H\qv{x})$} (being a hero is necessary for cape-wearing)
  \item Only heroes who wear capes welcomed Greta.\\
  \item[] \alert{$\qt{\forall}{x}(W\qr{x}{g} \eif (H\qv{x} \eand C\qv{x}))$}
  \item Greta admires only people who wear capes.
  \item[] \alert{$\enot\qt{\exists}{x}(\enot  C\qv{x} \eand A\qr{g}{x})$}
  \item Autumn welcomed only heroes and villains.
  \item[] \alert{$\enot\qt{\exists}{x}(\enot (H\qv{x} \eor V\qv{x}) \eand W\qr{a}{x})$}
  \item[] \alert{$\qt{\forall}{x}(W\qr{a}{x} \eif (H\qv{x} \eor V\qv{x}))$}
\end{itemize}
\end{frame}

\begin{frame}
\frametitle{The indefinite article}

\begin{itemize}[<+->]
  \item We use ``is a'' to indicate predication, e.g., ``Greta is a hero.''
  \item Often ``a'' is used to claim existence, e.g.,
  \begin{itemize}[<+->]
    \item[] Greta admires a hero.
    \item[] \alert{$\qt{\exists}{x} (H\qv{x} \eand A\qr{g}{x})$}
  \end{itemize}
  \item But a \emph{generic} indefinite is closer to a
  universal quantifier:
  \begin{itemize}[<+->]
    \item[] A hero is someone who inspires.
    \item[] \alert{$\qt{\forall}{x}(H\qv{x} \eif I\qv{x})$}
  \end{itemize}
  \item Be careful if the indefinite article is in the antecedent of a conditional:
  \begin{itemize}[<+->]
    \item[] If \emph{a hero} wears a cape, \emph{they} inspire.
    \item[] That means: all heroes who wear capes inspire.
    \item[] \alert{$\qt{\forall}{x}((H\qv{x} \eand C\qv{x}) \eif I\qv{x})$}
  \end{itemize}
\end{itemize}
\end{frame}

\begin{frame}
\frametitle{Universal ``some''; existential ``any''}

\begin{itemize}[<+->]
  \item ``Someone,'' ``something'' can require a \emph{universal} quantifier:\\
 if it's in the antecedent of a conditional, with a pronoun in the consequent referring back to it, e.g.,
  \begin{itemize}[<+->]
    \item[] If \emph{someone} is a hero, Autum admires \emph{them}.
    \item[] Roughly: Autumn admires all heroes.
    \item[] \alert{$\qt{\forall}{x}(H\qv{x} \eif A\qr{a}{x})$}
  \end{itemize}
  \item ``Any'' in antecedents but \emph{without} pronouns referring back to
  them are \emph{existential}:
  \begin{itemize}[<+->]
    \item[] If \emph{anyone} is a hero, Greta is.
    \item[] Roughly: if there are heroes (at all), Greta is a hero.
    \item[] \alert{$\qt{\exists}{x}\, H\qv{x} \eif H\qv{g}$}
    %need to check restrictions on parentheses here! recall some tricky stuff...
  \end{itemize}
\end{itemize}
\end{frame}

\subsection{Mixed domains}

\begin{frame}
\frametitle{Mixed domains}

\begin{itemize}[<+->]
  \item Sometimes you want to talk about more than one kind of thing.
  \item The domain can include any mix of things (e.g., people,
  animals, items of clothing, feelings)
  \item Proper symbolization then needs predicates for these kinds, e.g.:
    \begin{ekey}
    \item[$Domain$] people alive in \year{} and items of clothing
    \item[P\qv{x}] \gap{x} is a person
    \item[L\qv{x}] \gap{x} is an item of clothing.
    \item[E\qv{x}] \gap{x} is a cape (recall: `Cx' is `wears a cape')
    \item[R\qr{x}{y}] \gap{x} wears \gap{y}
    \end{ekey}
\end{itemize}
\end{frame}


\begin{frame}
\frametitle{Quantification in mixed domains}

\begin{itemize}[<+->]
\item Not everyone is wearing a cape.
\begin{itemize}
  \item In domain of people only:
  \item[] \alert{$\enot \qt{\forall}{x}\, C\qv{x}$}
  \item In mixed domain:
  \item[] \alert{$\enot \qt{\forall}{x}\, (P\qv{x} \eif C\qv{x})$}
\end{itemize}
\item Some people inspire.
\begin{itemize}[<+->]
  \item In domain of people only:
  \item[] \alert{$\qt{\exists}{x}\,I\qv{x}$}
  \item In mixed domain:
  \item[] \alert{$\qt{\exists}{x}(P\qv{x} \eand I\qv{x})$}
\end{itemize}
\item Greta wears something.
\item[] \alert{$\qt{\exists}{x}(L\qv{x} \eand R\qr{g}{x})$}
\end{itemize}
\end{frame}

\subsection{Captain Morgan's (tele)scope!}

\begin{frame}
\frametitle{(Spiced) de Morgan's for Quantifiers}

\begin{itemize}[<+->]

\item We can push negations through quantified expressions, flipping the quantifiers and negating what lies in their scope:

\item $\enot \qt{\forall}{\script{x}}\metaA{}(\script{x})$: ``it's not the case that for everything, Phi'' \\ \makebox[\textwidth]{is equivalent to} \\
\item [] $\qt{\exists}{\script{x}}\enot \metaA{}(\script{x})$: ``there exists something such that not-Phi''

\item $\enot \qt{\exists}{\script{x}}\metaA{}(\script{x})$: ``it's not the case that for something, Phi'' \\ \makebox[\textwidth]{is equivalent to} \\
\item [] $\qt{\forall}{\script{x}}\enot \metaA{}(\script{x})$: ``for everything, not-Phi'', i.e. Phi for-nothing! 

\item[] A concrete example:

\item \textit{It's not the case that some hero wears a cape}:

\item[]  \alert{$\enot \qt{\exists}{x}(H\qv{x} \eand C\qv{x})$}

\item[] \alert{$\qt{\forall}{x}\enot (H\qv{x} \eand C\qv{x})$} (now apply regular de Morgan's!) 

\item[]\alert{$\qt{\forall}{x} (\enot H\qv{x} \eor \enot C\qv{x})$} 

\iffalse 
  \item \alert{Someone who wears a cape} welcomed Greta.\\
  \item[] \alert{$\qt{\exists}{x}(C\qv{x} \eand W\qr{x}{g})$}
  \item Greta admires \alert{some hero who wears a cape}.
  \item[] \alert{$\qt{\exists}{x}((H\qv{x} \eand C\qv{x}) \eand A\qr{g}{x})$}
  \item Autumn welcomed \alert{someone who welcomed Greta}.
  \item[] \alert{$\qt{\exists}{x}(W\qr{x}{g} \eand W\qr{a}{x})$}
\fi 

\end{itemize}
\end{frame}

\begin{frame}
\frametitle{Quantifier Scope}
%\large

\begin{itemize}[<+->]

\item \emph{Scope of a Quantifer}: the \textbf{sub}formula for which the quantifier is \\ \phantom{\emph{Scope of a Quantifer} : } the main logical operator 

\item[] in `$\qt{\exists}{x}(L\qv{x} \eand R\qr{g}{x})$', the scope of the existential is `\alert{$(L\qv{x} \eand R\qr{g}{x})$}'

\item[] in `$\qt{\exists}{x}L\qv{x} \eand \qt{\forall}{y}R\qr{g}{y}$', the scope of the existential is `\alert{$L\qv{x}$}'

\item A variable is \emph{bound} if it lies within the scope of a quantifer

\item By the recursive definition that follows, each variable is bound by at most one quantifier!

\end{itemize}
\end{frame}

\begin{frame}
\frametitle{Recall our Recursive definition of SL wffs}
%\large

\begin{enumerate}[<+->]

\item Every atomic formula is a wff.
\item If \metaA{} is a wff, then $\enot\metaA{}$ is a wff.
\item If \metaA{} and \metaB{} are wffs, then $(\metaA{}\eand\metaB{})$ is a wff.
\item If \metaA{} and \metaB{} are wffs, $(\metaA{}\eor\metaB{})$ is a wff.
\item If \metaA{} and \metaB{} are wffs, then $(\metaA{}\eif\metaB{})$ is a wff.
\item If \metaA{} and \metaB{} are wffs, then $(\metaA{}\eiff\metaB{})$ is a wff.

\end{enumerate}

 \uncover<7->{Nothing else is a wff of SL! (But some new things are wffs of QL!)}

\end{frame}

\begin{frame}
\frametitle{Extending our Recursive definition to QL wffs}
%\large

\begin{enumerate}[<+->]

\item[1.$^{\ast}$] \emph{Atomic formula of QL}: an $n$-place predicate followed by $n$ terms (i.e. constants or variables), where $n \in \mathbb{N}$

\item[] This includes the SL atomic wff, which are 0-place predicates

\item[7.] If \metaA{} is a wff, \script{x} is a variable, and \alert{\metaA{} contains no \script{x}-quantifiers}, then \emph{$\qt{\forall}{\script{x}}\metaA{}$} is a wff.

\item[8.] If \metaA{} is a wff, \script{x} is a variable, and \alert{\metaA{} contains no \script{x}-quantifiers}, then \emph{$\qt{\exists}{\script{x}}\metaA{}$} is a wff.

\item[9.] All and only wffs of QL come from the prior 8 rules.

\end{enumerate}



\end{frame}

\begin{frame}
\frametitle{QL Sentences: proper subset of QL wffs}
%\large

\begin{itemize}[<+->]

\item Recall: the wffs of SL \textit{just are} the sentences of SL, i.e. the statements that are true or false under a truth-value assignment to atomic sentences 

\item Not all wffs of QL are sentences! Watch out! 

\item Let $L$ be a two-place predicate. Then $L\qr{x}{x}$ is an atomic wff of QL, but NOT a sentence (`$L\qr{x}{x}$' is neither true nor false)

\item the $x$'s in `$L\qr{x}{x}$' are \emph{free variables}, i.e. unbound variables 

\item \emph{Sentence of QL}: a wff that has no free variables: i.e. any variable that occurs is bound by a quantifier 

%\item The sentences of QL are a proper subset of the QL wffs

\end{itemize}
\end{frame}





