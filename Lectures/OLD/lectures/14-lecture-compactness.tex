% !TeX root = ./14-handout.tex

%JH: add in stuff on expressive adequacy at the end, in case need to kill time
% alt option to work through total/linear ordering from partial ordering result! 
%could also work through normal forms and notion of making properties manifest!!!! e.g. research application. could also save this for review session if we're killin time! 

\setcounter{section}{13}
\section{Compactness of SL \& QL}

\begin{frame}
%\large

\scriptsize{\tableofcontents}

\end{frame}

\begin{frame}
\frametitle{Soundness and Completeness}
%\large

\begin{itemize}[<+->]

\item Let $\Gamma$ be any set of \textit{sentences} of QL and $\Theta$ any sentence of QL. 

\item For our two natural deduction systems SND and QND, we have proven the following (where QND extends SND):

\medskip 

\item \emph{Soundness}: If $\Gamma \vdash_{QND} \Theta$, then $\Gamma \entails \Theta$
%Single turnstile entails Double Turnstile 

\bi 

\item QND derivations are `safe' (they preserve truth)

\item (syntactic to semantic: i.e. we chose `good' rules!)

\ei

\bigskip 

\item \emph{Completeness}: If $\Gamma \entails \Theta$, then $\Gamma \vdash_{QND} \Theta$
%Double Turnstile entails  Single turnstile  

%\item By proving that QND is \textit{complete}, we show that 
%what about showing a set of sentences is unsatisfiable though? can we just not do this w/ SND derivations? would we need to introduce a falsum symbol? 

\medskip 

\bi

\item reasoning about arbitrary models is not needed to demonstrate validity: QND derivations suffice

\item (logical entailment is fully covered by our syntactic rules)


\ei

\end{itemize}
\end{frame}

\subsection{Compactness of SL}

\begin{frame}
\frametitle{Compactness of SL}
%\large

\begin{itemize}[<+->]

\item \emph{Compactness of SL}: for any set $\Gamma$ of SL-sentences (possibly infinite), $\Gamma$ is satisfiable \emph{if and only if} every finite subset $\Delta \subseteq \Gamma$ is satisfiable (i.e. for each $\Delta$, there is a truth-value assignment that makes all sentences in $\Delta$ true). 

\item Relying on our valiant labors in proving the soundness and completeness of SND, we gain an elementary proof of compactness

\item This proof is ``impure" because it relies on syntactic notions, whereas the statement of compactness is purely semantic.

\end{itemize}
\end{frame}

\begin{frame}
\frametitle{An ``impure" proof of Compactness}
%\large

\begin{itemize}[<+->]

\item \emph{Compactness of SL}: for any set $\Gamma$ of SL-sentences, $\Gamma$ is satisfiable \emph{if and only if} every finite subset $\Delta \subseteq \Gamma$ is satisfiable

\item[$\Rightarrow$] (trivial direction): Assume that $\Gamma$ is satisfiable. Then there is a TVA that makes true every sentence in $\Gamma$. 
\item[] -- This TVA satisfies every finite subset $\Delta \subseteq \Gamma$.

\item[$\Leftarrow$] (nontrivial direction): Assume that every finite subset $\Delta \subseteq \Gamma$ is satisfiable. 
\item[] -- Assume for \textit{reductio} that $\Gamma$ is unsatisfiable. \\ Then there is no TVA that makes true every sentence in $\Gamma$. 

\item[] -- Hence, for any contradiction \metav{C} (e.g. $P \eand \enot P$), we have $\Gamma \entails \metav{C}$



\end{itemize}
\end{frame}

\begin{frame}
\frametitle{Impure proof: non-trivial direction continued}
%\large

\begin{itemize}[<+->]

\item (From above: $\Gamma$ unsatisfiable $\Rightarrow$ $\Gamma \entails \metav{C}$, for contradiction \metav{C})

%From above, we have $\Gamma \entails \metav{C}$, for contradiction \metav{C}

\item Hence, by completeness of SND, we can derive \metav{C} from $\Gamma$: $\Gamma \vdash_{SND} \metav{C}$. 

\item Since derivations are finite, there exists a finite $\Delta \subseteq \Gamma$ such that $\Delta \vdash_{SND} \metav{C}$

\item Then, by soundness of SND, $\Delta \entails \metav{C}$. Since \metav{C} is unsatisfiable, this means that $\Delta $ must be unsatisfiable as well. 

\item But that contradicts our starting assumption that every finite subset $\Delta \subseteq \Gamma$ is satisfiable. 

\item So $\Gamma$ must be satisfiable (proving compactness)

\end{itemize}
\end{frame}

\begin{frame}
\frametitle{What does compactness of SL tell us?}
%\large

\begin{itemize}[<+->]

\item \textit{Question}: Are there any arguments of SL that have infinitely-many premises, where no premise is redundant?

\item Assume that $\Gamma \entails \metav{P}$. Then what can we say about $\Gamma \cup \{\enot \metav{P} \}$?

%\item[] \makebox[\textwidth]{$\Gamma \cup \{\enot \metav{P} \}$ is \textbf{\textcolor{OGlyallpink}{unsatisfiable}}!}

\item[] \qquad \qquad \qquad \qquad $\Gamma \cup \{\enot \metav{P} \}$ is \textbf{\textcolor{OGlyallpink}{unsatisfiable}}!


\item[] -- So by one \textcolor{OGlyallpink}{Contrapositive} of Compactness, there exists a finite \\  \qquad subset $\Delta \subset \Gamma \cup \{\enot \metav{P} \}$ that is \textcolor{OGlyallpink}{unsatisfiable}. 

\item[] -- Easy to show that there is a finite $\Gamma_f \subset \Gamma$ s.t. $\Gamma_f \cup \{\enot \metav{P} \}$ is \\ \qquad \textcolor{OGlyallpink}{unsatisfiable} as well. So $\Gamma_f \entails \metav{P}$

% % Proof is by cases: if \enot P is not in \Delta, then \Delta is already an unsatisfiable finite subset of $\Gamma$ s.t. \Delta \cup \enot P is unsatisfiable (since adding a sentence to an unsatisfiable set maintain unsatisfiability
% otherwise, \enot P IS in \Delta. Then, define a finite subset of $\Gamma$ by removing \enot P from \Delta. Call this \Gamma_f := \Delta \ \{ \enot P \}. Then \Gamma_f union \enot P is unsatisfiable (by assumption), and so \Gamma_f entails P. 

%\item[] -- Since an unsatisfiable set entails every SL sentence, $\Delta \entails \metav{P}$, \\ \qquad where $\Delta$ is finite. %So the answer is ``no"! 

%\item So by completeness, there exists an SND derivation $\Delta \vdash_{SND} \metav{P}$

\item \textit{Upshot}: every valid argument relies on finitely-many premises

\item Contrast proof here with PS12 \#4, which shows same result using completeness \textit{and soundness}, relying on syntactic $\vdash_{SND}$

\item Whereas our argument above proceeds entirely semantically, using compactness and semantic entailment $\entails$ 

% Although at the moment, the syntactic elements of soundness are bundled into our proof of compactness. So we haven't yet given a purely semantic argument

\item If only we could prove compactness purely semantically?!

\end{itemize}
\end{frame}

\subsection{A `Pure' proof of SL compactness}

\begin{frame}
\frametitle{A `Pure' proof of the Compactness of SL}
%\large

\begin{itemize}[<+->]

\item Using a very similar idea to our construction of the maximally-SND-consistent set $\Gamma^{\ast}$, we can provide a purely semantic and yet still elementary proof of SL compactness

\item Proof sketch: assuming that every finite subset of $\Gamma$ is satisfiable, we will construct a superset $\Gamma^{\ast} \supset \Gamma$ for which it is easy to define a truth-value assignment that satisfies every sentence in $\Gamma^{\ast}$, and hence in $\Gamma$.

\item As with our earlier completeness proof, $\Gamma^{\ast}$ comes along with a membership lemma, which we use for our induction over SL. 

\end{itemize}
\end{frame}

\begin{frame}
\frametitle{Beginning the Proof}
%\large

\begin{itemize}[<+->]

\item[$\Rightarrow$] (easy direction): assume that the (possibly infinite) set of SL-wffs $\Gamma$ is satisfiable. Then there is a TVA that makes true every sentence in $\Gamma$, and this TVA satisfies every finite subset of $\Gamma$.

\item[$\Leftarrow$] (harder direction): Assume that every finite subset $\Delta \subset \Gamma$ is satisfiable. Show that $\Gamma$ is satisfiable (nontrivial if $\Gamma$ is infinite).

\item Notice that it suffices to construct a superset $\Gamma^{\ast}$ of $\Gamma$ that is satisfiable. Then the TVA that makes true everything in $\Gamma^{\ast}$ will make true everything in $\Gamma$.

\item To proceed, we introduce an idea very similar to the notion of a maximally-consistent-in-SND set. But now using only \textit{semantic} notions (so avoiding our proof system). 

\end{itemize}
\end{frame}

\begin{frame}
\frametitle{Maximally finitely satisfiable sets}
%\large

\begin{itemize}[<+->]

\item A set $\Gamma^{\ast}$ of SL wffs is \emph{maximally finitely satisfiable} (MFS) provided that:

\begin{enumerate}[1.)]

\item Every finite subset of $\Gamma^{\ast}$ is satisfiable ($\Gamma^{\ast}$ is ``\textbf{finitely satisfiable}")

\item For each SL wff \metav{P}, if $\Gamma^{\ast} \cup \{\metav{P} \}$ is FS, then $\metav{P} \in \Gamma^{\ast}$ (\textit{``semantic Door"}) \\ Otherwise, adding any additional \metav{P} to $\Gamma^{\ast}$ breaks finite-satisfiability

\item[] i.e. $\metav{P} \notin \Gamma^{\ast}$ iff $\Gamma^{\ast} \cup \{\metav{P} \}$ has an unsatisfiable finite subset 
%%this is our semantic analog of The Door lemma! 

%i.e. for any $\metav{P} \notin \Gamma^{\ast}$, $\Gamma^{\ast} \cup \{\metav{P} \}$ has an unsatisfiable finite subset 

%exactly one of \metav{P} or \enot\metav{P} belongs to $\Gamma^{\ast}$. 

%\item $\Gamma^{\ast}$ is SND-consistent (i.e. can't derive contradictory sentences)

%\item adding \textbf{any} additional wff to $\Gamma^{\ast}$ would result in an SND-\textcolor{OGlyallpink}{inconsistent} set

\end{enumerate} 

\bigskip

\item Next we'll show that any MFS set is satisfiable (this mirrors our ``maximal consistency lemma'' from our completeness proof)

\item To do this, we'll prove a membership lemma that facilitates an induction over SL! 

\item Finally, we'll show how to construct an MFS $\Gamma^{\ast}$ from any finitely-satisfiable $\Gamma$ \\ (i.e. what we assume at the start of the nontrivial-direction)

\end{itemize}
\end{frame}

\begin{frame}
\frametitle{Membership Lemma for MFS sets (``complete clubs")}
%\large

\begin{itemize}[<+->]

\item To induct on SL, we first show some constraints on $\Gamma^{\ast}$ membership

\item Basically, $\Gamma^{\ast}$ has a a bouncer who enforces maximal finite satisfiability. 
%(the smaller fish, relative to our lexical ordering)

\item \emph{Membership Lemma} for club $\Gamma^{\ast}$: if \metav{P} and \metav{Q} are SL wffs, then:

\begin{enumerate}[a.)]

\item $\enot \metav{P} \in \Gamma^{\ast}$ if and only if $\metav{P} \notin \Gamma^{\ast}$

\item $\metav{P} \eand \metav{Q} \in \Gamma^{\ast}$ if and only if both $\metav{P}\in \Gamma^{\ast}$ and $\metav{Q}\in \Gamma^{\ast}$

\item $\metav{P} \eor \metav{Q} \in \Gamma^{\ast}$ if and only if either $\metav{P}\in \Gamma^{\ast}$ or $\metav{Q}\in \Gamma^{\ast}$

\item $\metav{P} \eif \metav{Q} \in \Gamma^{\ast}$ if and only if either $\metav{P}\notin \Gamma^{\ast}$ or $\metav{Q}\in \Gamma^{\ast}$

\item $\metav{P} \eiff \metav{Q} \in \Gamma^{\ast}$ iff either (i) $\metav{P}\in \Gamma^{\ast}$ and $\metav{Q}\in \Gamma^{\ast}$ or (ii) $\metav{P}\notin \Gamma^{\ast}$ and $\metav{Q}\notin \Gamma^{\ast}$

\end{enumerate}

\item These syntactic constraints mirror truth-conditions, but we will now NOT rely on our proof system to prove this lemma 

\item (We built an analog of ``the Door" into the definition of MFS sets)

\end{itemize}
\end{frame}

\begin{frame}
\frametitle{Proof of Membership Lemma for MFS Sets}
%\large

\begin{itemize}[<+->]

\item \textbf{\textcolor{highlightB}{Case (a)}}: $\enot \metav{P} \in \Gamma^{\ast}$ iff $\metav{P} \notin \Gamma^{\ast}$: use condition 2) (``semantic Door") of MFS sets: $\metav{P} \notin \Gamma^{\ast}$ iff $\Gamma^{\ast} \cup \{\metav{P} \}$ has an unsatisfiable finite subset 

\item For the other cases, we rely on Case (a), the truth tables for the connectives, and the fact that $\Gamma^{\ast}$ is finitely-satisfiable, i.e. every finite subset is satisfiable. 

\item[] (So we do lots of \textit{reductio} proofs: assume that a membership case fails, apply Case (a), and then show this would result in an unsatisfiable finite subset---contradicting condition (1), i.e. that all finite subsets are satisfiable).  

\item So \textcolor{highlightB}{imagine we've proven the membership lemma!} 

\item Then define a TVA $\metav{I}^{\ast}$ that makes true every atomic sentence in $\Gamma^{\ast}$; 
\item[] -- show by induction that this TVA satisfies every sentence in $\Gamma^{\ast}$ (just as in our proof of completeness of SND!)

\end{itemize}
\end{frame}










\iffalse %%%%VAN MCGEE FRAMING/DEFNnS************************************

\begin{frame}
\frametitle{Maximally finitely satisfiable set a.k.a. `a Complete Club'}
%\large

\begin{itemize}[<+->]

\item A set $\Gamma^{\ast}$ of SL wffs is \emph{maximally finitely satisfiable} (MFS) provided that:

\begin{enumerate}[1.)]

\item Every finite subset of $\Gamma^{\ast}$ is satisfiable ($\Gamma^{\ast}$ is ``finitely satisfiable")

\item For each SL wff \metav{P}, exactly one of \metav{P} or \enot\metav{P} belongs to $\Gamma^{\ast}$. 

%\item $\Gamma^{\ast}$ is SND-consistent (i.e. can't derive contradictory sentences)

%\item adding \textbf{any} additional wff to $\Gamma^{\ast}$ would result in an SND-\textcolor{OGlyallpink}{inconsistent} set

\end{enumerate} 

\bigskip

\item Next we'll show that any MFS set is satisfiable (this mirrors our ``maximal consistency lemma'' from our completeness proof)

\item To do this, we'll prove a membership lemma that facilitates an induction over SL! 

\item Finally, we'll show how to construct an MFS $\Gamma^{\ast}$ from any finitely-satisfiable $\Gamma$ (i.e. what we assume at the start of $\Leftarrow$-direction)

\end{itemize}
\end{frame}

\begin{frame}
\frametitle{Proof of Membership Lemma for MFS Sets}
%\large

\begin{itemize}[<+->]

\item \emph{Case (a)}: $\enot \metav{P} \in \Gamma^{\ast}$ iff $\metav{P} \notin \Gamma^{\ast}$: we defined MFS sets to have this property

\item For the other cases, we rely on Case (a), the truth tables for the connectives, and the fact that $\Gamma^{\ast}$ is finitely-satisfiable, i.e. every finite subset is satisfiable. 

\item[] (So lots of reductio proofs, where you assume that a membership case fails, apply Case (a), and then show this would result in an unsatisfiable finite subset---contradicting our assumption that all finite subsets are satisfiable).  

\item \textcolor{highlightB}{So imagine we've proven the membership lemma!} 

\item Then define a TVA $\metav{I}^{\ast}$ that makes true every atomic sentence in $\Gamma^{\ast}$; 
\item[] -- show by induction that this TVA satisfies every sentence in $\Gamma^{\ast}$ \\ (just as in our proof of completeness of SND!)

\end{itemize}
\end{frame}

\fi %%%END VAN MCGEE FRAMING/DEFNS**************************












\begin{frame}
\frametitle{Building an MFS $\Gamma^{\ast}$ from a finitely-satisfiable $\Gamma$}
%\large

\begin{itemize}[<+->]

\item It remains to construct a maximally finitely-satisfiable superset $\Gamma^{\ast}$ of a finitely-satisfiable $\Gamma$ 

%\item Let $\Gamma$ be a finitely-satisfiable set of SL wffs (possibly infinite)

\item We first \emph{enumerate} the SL wffs, so that every SL wff is associated with a unique positive integer $\{1, 2, 3, \dots \}$

\item Consider the first wff `$A$' in our enumeration. \\ If $A$ can be added to $\Gamma$ while preserving finite satisfiability, then let  $\Gamma_1 := \Gamma \cup \{A\}$. 

\item Otherwise, let  $\Gamma_1 := \Gamma$ (so that $\Gamma_1$ stays FS)

% \item In general, if $P_k$ is the $k$-th sentence in our enumeration, then $\Gamma_{k+1}$ is $\Gamma_k \cup \{P_k\}$ provided $\Gamma_k \cup \{P_k\}$ is SND-consistent; \\ otherwise, $\Gamma_{k+1}$ equals $\Gamma_k$

\item Then, proceed to the second wff in our enumeration. \\ If it can be added to $\Gamma_1$ without the new set breaking FS, let $\Gamma_2$ be the result. Otherwise, let $\Gamma_2 := \Gamma_1$

\item $\Gamma^{\ast}$ is the result of `doing' this procedure for every SL wff

\item More precisely, $\Gamma^{\ast} := \bigcup_{k=1}^{\infty} \Gamma_k$

\end{itemize}
\end{frame}

\begin{frame}
\frametitle{Claim: $\Gamma^{\ast}$ is maximally finitely satisfiable (MFS)}
%\large

\begin{itemize}[<+->]

\item At this point, it suffices to prove that $\Gamma^{\ast}$ is MFS

\item[1.)] Clearly, $\Gamma^{\ast}$ is finitely satisfiable. If it were not, then some $\Gamma_k \subset \Gamma^{\ast}$ would be finitely unsatisfiable, but that contradicts our construction conditions. 

\item[2.)] Moreover, $\Gamma^{\ast}$ is maximal: if there were a wff \metav{Q} that could be added to $\Gamma^{\ast}$ while preserving finite satisfiability, we would have added \metav{Q} at its enumeration stage. 

\item[] -- So if $\metav{Q} \notin \Gamma^{\ast}$, it must be that $\Gamma^{\ast} \cup \{\metav{Q} \}$ is \textit{not} finitely satisfiable. 

\item So we're done! Any finitely satisfiable $\Gamma$ is a subset of an MFS $\Gamma^{\ast}$, which we've shown is satisfiable! So $\Gamma$ is satisfiable!

\end{itemize}
\end{frame}



\subsection{Compactness of First-order Languages}

\begin{frame}
\frametitle{Compactness of QL}
%\large

\begin{itemize}[<+->]

\item \emph{Compactness of QL}: for any set $\Gamma$ of QL-sentences, $\Gamma$ is satisfiable if and only if every finite subset $\Delta \subseteq \Gamma$ is satisfiable \\ (i.e. $\qt{\forall}{\Delta}$ $\exists$ a QL-model $\mathfrak{M}_{\Delta}$ that makes true every sentence in $\Delta$). 

\item \textit{Mutatis mutandis}, we can provide an analogous impure  proof, relying on the soundness and completeness of system QND

\item[] And also a `pure' proof, constructing a maximally finitely satisfiable and \textit{existentially complete} superset $\Gamma^{\ast}$. 

\item To widen the interest of our results, let's generalize compactness to any first-order language $\metav{L}$

\end{itemize}
\end{frame}

\begin{frame}
\frametitle{First-order Languages (FOLs)}
%\large

\begin{itemize}[<+->]

\item \emph{First-order language \metav{L}}: a set of well-formed formulae specified by a recursion clause like the one we gave for QL, where the symbols of \metav{L} include: 

\bi 

%\item variables $w, x, y, z$ (possibly with subscripts $n \in \mathbb{N}$) 

\item Variables: $w, x, y, z$ (allowing subscripts $n \in \mathbb{N}$) 
\item Operators: our five sentential connectives and two quantifiers
\item Punctuation: left and right parentheses
\item \textcolor{OGlyallpink}{Names}: a set of constants (allowing subscripts $n \in\mathbb{N}$)
\item \textcolor{OGlyallpink}{Predicates}: a non-empty set of capital letters (allowing subscripts), each with ``an invisible label" giving its arity \\ (e.g. 0-place, 1-place, 2-place, etc.) 
\item a set of \textcolor{OGlyallpink}{function} symbols $f(c)$ (syntax: $f$ maps terms to terms) 
% notice that we won't get into trouble syntactically, since we otherwise have never placed parentheses after a lowercase roman letter! So we can add this syntactic stipulation to our recursion clause
\ei 

\item \textcolor{OGlyallpink}{Different} FOLs differ in their names, predicates, and functions

\end{itemize}
\end{frame}

\begin{frame}
\frametitle{\metav{L}-models and interpretations}
%\large

\begin{itemize}[<+->]

\item Let \metav{L} be a first-order language, containing constants and $k$-place predicates (e.g. the language of QL)
\bi
\item recall that the atomic sentences of SL are 0th-place predicates
\ei
\item An \metav{L}-model $\mathfrak{M} := (D, I)$ consists of

\begin{enumerate}

\item A non-empty set $D$ of objects, called the domain of   $\mathfrak{M}$

\item A map I (the \textit{interpretation} of $\mathfrak{M}$), which maps the vocabulary of \metav{L} to objects and ordered pairs from $D$ as follows:

\begin{itemize}

\small

\item For each constant $c \in \mathfrak{L}$, $I(c)$ is an element of $D$, called the \textit{referent} or denotation of $c$

\item For each k-place predicate $P$ of $\mathfrak{L}$, $I(P)$ is a set of ordered $k$-tuples of objects in $D$, called the \textit{extension} of $P$
\item $I$ maps SL atomics to ``true" or ``false" (i.e. `1' or `0')
% $k$-place relation defined on $D$, 
% GB: you can think of a k-place relation as a subset of $D^k$, i.e. the space of k-tuples of objects in D. 

\end{itemize}

\end{enumerate}

%\item Our text uses `models' and `interpretations' interchangeably, but the above disambiguation is convenient

\end{itemize}
\end{frame}

\begin{frame}
\frametitle{FOL with identity and functions}
%\large

\begin{itemize}[<+->]

% % Information on this slide comes from lecture 13 of Gordon 414, compactness of predicate logic

\item With some minor modifications, we could extend our soundness and completeness proofs for QND to FOLs and deduction systems that include (1) a privileged identity predicate ``$=$" and \\ (2) functions that syntactically map terms to terms \\ (interpreted as mapping the domain $D$ to itself)

\item Like our symbol ``$\prime$", we add in some new symbol ``$\alpha$" that doesn't occur in our FOL, to give a countable infinity of unused constants

\item To construct our maximally-syntactically-consistent, existentially complete superset $\Gamma^{\ast}$, we focus on equivalence classes of co-referential constants, since now some constants might name the same object in the domain (e.g. $c=d$) 

% Adding function symbols does not really affect the soundness or completeness proofs

\item Using the axiom of choice, we could even handle FOLs that have  \textit{uncountably many} predicates or constants!
% % Nothing really changes in the derivation system or the soundness proof. We use the axiom of choice to construct the maximally-syntactically-consistent set

\end{itemize}
\end{frame}

\begin{frame}
\frametitle{Compactness for a first-order language}
%\large

\begin{itemize}[<+->]

\item \emph{Compactness of a FOL \metav{L}}: for any set $\Gamma$ of \metav{L}-sentences (possibly infinite), $\Gamma$ is satisfiable if and only if every finite subset $\Delta \subseteq \Gamma$ is satisfiable. 

\item We could prove this either by (i) using a soundness and completeness result for an \metav{L}-deduction system; \\ (ii) generalizing our `pure' proof for SL; or \\ (iii) generalizing the topological proof of SL compactness (relying on results from topology, e.g. Tychonoff's theorem)


\end{itemize}
\end{frame}

\subsection{The L{\"o}wenheim--Skolem theorems}

\begin{frame}
\frametitle{Downwards!}
%\large

\begin{itemize}[<+->]

\item Terminology: we'll say that a model $\mathfrak{M}$ is \textit{infinite} if its domain $D$ is infinite in size. Likewise for saying that a model is finite, or countably infinite.

\item \emph{Downward L{\"o}wenheim--Skolem}: let $\Gamma$ be a set of \metav{L}-sentences. If $\Gamma$ is satisfiable in an infinite model, then it is satisfiable in a countably infinite model.

\item \textit{Gloss}: we can always descend from an infinite model to a countably infinite model

\item Proof(s): (1) be impure and piggyback on completeness proof \\ or (2) use compactness and satisfiability lemma for MFS sets
% in our completeness proof, we constructed a countably infinite model

% But do we have an issue here? In our completeness proof, we started simply by assuming that $\Gamma$ is syntactically consistent (?)

%\item note that for an uncountable FOL \metav{L}, the consequent of DLS becomes: there is a model whose domain is the same size as the set of \metav{L}-sentences

\end{itemize}
\end{frame}

\begin{frame}
\frametitle{Down to be Impure}
%\large

\begin{itemize}[<+->]

\item  \emph{Converse consistency lemma}: if \metav{L}-set $\Gamma$ is satisfiable, then $\Gamma$ is syntactically-consistent (for a given deduction system \metav{L}ND that we've shown is sound)

\item \textcolor{highlightB}{Proof: good exercise}!!! Assume for \textit{reductio} that $\Gamma$ is syntactically-inconsistent and then apply soundness

\item So since $\Gamma$ is satisfiable, it is syntactically consistent.
\item[] Then, appeal to our consistency lemma shown in the course of proving completeness: for any syntactically-consistent set, there is a maximally-consistent (and $\exists$-complete) set that is satisfiable, where we showed this by constructing a countably infinite model
% technically, we added a countable infinity of constants to \metav{L}, forming a language $\metav{L}^+$. We then constructed a countably infinite $\metav{L}^+$-model for $\Gamma^{\ast}$, which, when restricted, gave a countably infinite $\metav{L}$-model for $\Gamma$ 

\item So $\Gamma$ has a countably infinite model



\end{itemize}
\end{frame}

\begin{frame}
\frametitle{Down with impurity: apply compactness}
%\large

\begin{itemize}[<+->]

\item \textit{Pure proof of Downward LS}: assume that $\Gamma$ is satisfied in an infinite model. Then it is satisfiable, and so by compactness theorem for FOL,  $\Gamma$ is finitely-satisfiable

\item Modify our construction to form a maximally finitely-satisfiable and $\exists$-complete superset $\Gamma^{\ast}$ of $\Gamma$ 

\item Prove a satisfiability lemma: any such $\Gamma^{\ast}$ is satisfiable, where we show this by constructing a countably infinite $\metav{L}^+$-model 
\item[] ($\metav{L}^+$ arises from \metav{L} by adding a countable-infinity of new constants)

%\item[] (we add a countable-infinity of new constants to form $\metav{L}^+$)

\item Then, $\Gamma$ has a countably infinite \metav{L}-model



\end{itemize}
\end{frame}

\begin{frame}
\frametitle{Onwards and Upwards!}
%\large

\begin{itemize}[<+->]

\item \emph{Upward L{\"o}wenheim--Skolem}:  let $\Gamma$ be a set of \metav{L}-sentences. \\ If $\Gamma$ is satisfiable in an infinite model $\mathfrak{M} := (D, I)$, then it is satisfiable in models of arbitrary size larger than $|D|$

\item \textit{Proof Sketch}: extend the set of constants \metav{C} of \metav{L} with an uncountable set \metav{E} that contains \metav{C}. 
\item[] Extend the FOL \metav{L} to $\metav{L}^+$ with \metav{E} as its set of constants and with identity predicate $=$. 

\item Construct an $\metav{L}^+$-set $\Gamma^{+}$ by adjoining to $\Gamma$  every sentence of the form $\enot c = d$ for every distinct $c, d \in \metav{E}$.

\item Show that $\Gamma^{+}$ is finitely-satisfiable and hence by compactness satisfiable. Then note that any $\metav{L}^+$-model satisfying $\Gamma^{+}$ must have a domain as large \metav{E}. Restrict the interpretation function to construct an $\metav{L}$-model for $\Gamma$ with domain $|D| = |\metav{E}|$
% the same size domain 
% % to show that $\Gamma^{+}$ is finitely-satisfiable, we can rely on our assumption that $\Gamma$ is satisfiable in an infinite model, but presumably we would need to modify this infinite model to accommodate the sentences of the form $\enot c = d$ (?) 
%GB notes don't explain how to do this...

\end{itemize}
\end{frame}


\iffalse 

% % If time, incorporate proof construction from Belot 414, lecture 13, proof after `a confession'

\begin{frame}
\frametitle{An elementary proof for FOL w/out identity}
%\large

\begin{itemize}[<+->]

\item \emph{Upward LS without identity}:

\end{itemize}
\end{frame}

\fi 



\subsection{Skolem's `Paradox'}

\begin{frame}
\frametitle{ZFC as a first-order language}
%\large

\begin{itemize}[<+->]

\item \emph{Zermelo--Fraenkel set theory with choice} (ZFC): \\ a FOL \metav{ZFC} that has identity and a 2-place predicate for set-membership `$\in$', written between (rather than before) terms when forming atomic wffs

\item In standard models, we interpret the objects as sets

\item A list of axioms or axiom schemas, e.g.\\  \textit{Null set axiom}: $\qt{\exists}{x}\qt{\forall}{y} y \notin x$ (i.e. there is an empty set $\emptyset$)

% % Note that there are technically infinitely-many axioms, since the axioms for separation and replacement are axiom schemas

\item[] \textit{Axiom of Extensionality}: $\qt{\forall}{x}\qt{\forall}{y}\qt{\forall}{z}( (z \in x \eiff z \in y) \eif x=y)$ \\ (i.e. two sets are identical iff they have the same members)

\item \emph{Axiom of Choice}: if $x$ is a set whose members are non-empty sets and no two members of $x$ share a member, then there is a set $y$ that contains exactly one element of each set in $x$ 


\end{itemize}
\end{frame}

\begin{frame}
\frametitle{Skolem's `Paradox'}
%\large

\begin{itemize}[<+->]

%%incorporate wisdom from van McGee Function Signs lecture p 6, in predicate calculus folder! 

% % Initial assumption below is basically the assumption that ZFC is satisfiable (i.e. consistent), which Godel's second incompleteness theorem tells that we cannot prove w/in ZFC if it is in fact consistent 

\item If ZFC has any models, then it has a \emph{countable model} \\ (since by downward LS, an infinite model entails a countably infinite model. Any finite model is already countable---and can be extended to a countably infinite model as well)
% % other, more precisely, I think probably VFC doesn't have any finite countable models, since we could presumably construct within ZFC the infinite set of sentences L_k that say there are at least k things. then this set has no finite models

\item Yet, we can prove within ZFC that there are \textcolor{OGlyallpink}{uncountable sets}, e.g. the power set of $\mathbb{N}$ has cardinality of $\mathbb{R}$

\item `Paradox': how can a \alert{countably-infinite model} make true the claim that there are \textcolor{OGlyallpink}{uncountable sets}? 

\end{itemize}
\end{frame}

\begin{frame}
\frametitle{Paradox Assuaged! (paradise regained?)}
%\large

\begin{itemize}[<+->]

\item Suppose that ZFC is satisfiable and so has a countable model $\mathfrak{M}$

\item $\mathfrak{M}$ makes true all the axioms of ZFC and hence all the consequences of these axioms, including the claim $U$ that says ``the powerset of $\mathbb{N}$ is uncountable". Denote this set as `$2^{\mathbb{N}}$' 

\item $U$: there is an injection but no bijection from $\mathbb{N}$ to $2^{\mathbb{N}}$; $\mathfrak{M} \entails U$

\item Since $\mathfrak{M}$ is countable, the sets $\mathbb{N}$ and $2^{\mathbb{N}}$ in $\mathfrak{M}$ are definitely countable ($\mathfrak{M}$ has only countably many objects in its domain to serve as members of objects in that domain)

\item So clearly, there \textcolor{OGlyallpink}{IS} a bijection between the sets that correspond to $\mathbb{N}$ and $2^{\mathbb{N}}$ in $\mathfrak{M}$ (we can prove this bijection in a metalanguage)

\item BUT (\alert{resolution}), this bijection is not itself an object in $\mathfrak{M}$. \\ So $\mathfrak{M}$ itself represents $2^{\mathbb{N}}$ as uncountable
% the countably infinite model does not semantically entail this bijection

\end{itemize}
\end{frame}

% % If possible, I could note/discuss the additional stuff that van McGee says about Skolem's paradox, e.g. how Skolem himself interpreted this result as showing that the size of sets is model relative. 

\subsection{Problems for finitism}

\begin{frame}
\frametitle{Saying that there are finitely-many things}
%\large

\begin{itemize}[<+->]

\item As shown on PS13 problems \#2, 5, and 6, we have some ISSUES when it comes to saying that there are finitely-many things in quantifier logic

\item It seems like we definitely cannot accomplish this putatively possible task through  \textit{sentences}

\item Is there any other way we might go about enforcing there being finitely-many things (e.g. if we think there probably are only finitely-many things and want a FOL to reflect that)?

\end{itemize}
\end{frame}

\begin{frame}
\frametitle{Adding a finitely-many Quantifier}
%\large

\begin{itemize}[<+->]

\item If not through sentences, perhaps through operators, e.g. quantifiers!

\item \textit{Idea}: add a `finitely-many' quantifier, \reflectbox{F}, to FOL

\item Syntactically, we define \reflectbox{F} just like a quantifier: if \metav{P} is a formula where $x$ does not appear bound, then (\reflectbox{F}x)\metav{P} is a wff

\item Semantically, we extend satisfiability semantics (oh boy---not that sh** again) so that (\reflectbox{F}x)\metav{P} is true in a model if and only if there are finitely-many \metav{P}-objects in the model's $D$, i.e. $|D|$ is finite

\item \emph{Question}: what would it take to modify our derivation system QND to make it sound and complete for quantifier logic with a finitely-many quantifier (QL-\reflectbox{F})?

\item \emph{Answer}: no derivation system can be sound 
\& complete for QL-\reflectbox{F}!

% % Perhaps this is one place where it comes in handy that we have proven compactness through soundness and completeness!

\item[] -- F***!!! INFINITE F***!!!

\end{itemize}
\end{frame}

\begin{frame}
\frametitle{Finite Hopes \& Finite Dreams: dashed upon $\infty$-many rocks}
%\large

\begin{itemize}[<+->]

% % Following could be an interesting example of a counter-mathematical. E.g. good example for a paper on this stuff.

\item Suppose for \textit{reductio} that we had a sound and complete derivation system for QL-\reflectbox{F}

\item Then, we could prove that QL-\reflectbox{F} is compact (see slides 3-4)
%a compactness theorem for 

\item Yet, the entailment relation $\entails_{QL-\reflectbox{F}}$ for this logic is NOT compact:

\item Consider the sentence $F := \qt{\reflectbox{F}}{x}x=x$, which says ``there are finitely-many things that equal themselves." This is just a fancy way of saying that there are finitely-many things in the domain (since everything is identical to itself and nothing else).

\item Then consider the set $X := \{F, L_1, L_2, \dots \}$, containing $F$ and each $L_k$ for $k \in \mathbb{N}$, where $L_k$ says ``there are at least $k$-things"

\item Set $X$ is finitely-satisfiable, but it is not satisfiable (violating compactness). Any way of making true the infinitely-many $L_n$'s requires an infinite model, which then can't make true sentence $F$ 

\end{itemize}
\end{frame}

\subsection{A topological proof of SL compactness}

%%note that we could also proceed much more simpler (w/ less explicit connection to topology) by doing a complete story/membership lemma proof like what McGee does in his SC compactness notes. so in a sense we're really well set up for this from our proof of completeness. very similar idea, just dropping consistent-in-SND for satisfiability and making the Henkin model 

\begin{frame}
\frametitle{What does ``compactness" normally mean?}
%\large

\begin{itemize}[<+->]

\item \emph{Topological space} $(X, \tau)$: a topology on a set $X$ is a collection of \textbf{open sets} $\tau$ s.t. the following sets are open: (i) $\emptyset$ and $X$; (ii) arbitrary unions of open sets; (iii) finite intersections of open sets


\item A set is closed in $(X, \tau)$ if its complement is open \\ (NB: sets can be `clopen', i.e. both open AND closed)

\medskip

\item Compactness in topology: a topological space is \emph{compact} iff every open cover has a finite subcover

\item Equivalently: every collection of closed subsets obeying the \textit{finite intersection property} has non-empty intersection

\medskip

\item Finite intersection property (FIP): a set of subsets $\{ F_{\beta} \}_{\beta \in B}$ of a topological space has the FIP if for every finite subset $B_0$ of our index set $B$, the intersection of all the sets $F_{\beta}$ for $\beta \in B_0$ is non-empty, i.e. provided that $\bigcap_{\beta \in B} F_{\beta}$ is non-empty 



\end{itemize}
\end{frame}

\begin{frame}
\frametitle{Why call the logical property ``compactness"?}
%\large

\begin{itemize}[<+->]

\item The compactness of SL is equivalent to the compactness of a particular topological space, namely a topology on the set of truth-value assignments (TVAs)

\item Let \metav{A} be the set of atomic wffs and let \metav{E} be the set of TVAs

\item for each atomic wff $A$, let \emph{$U^0_A$} be the set of TVAs that assign $A$ false, and let \emph{$U^1_A$} be the set of TVAs that assign $A$ true

\item Endow the set \metav{E} with a topology by stipulating that (i) for each atomic wff $A$, $U^0_A$ and $U^1_A$ are open and (ii) every non-empty open set arises as a union of these $U^0$s and $U^1$s 
% % so presumably this provides a basis for the topological space



\item \emph{Claim}: the compactness of SL is equivalent to the compactness of this topological space \metav{E}

\item Note that if we prove that 1) compactness of \metav{E} entails compactness of SL and that 2) \metav{E} is compact, then 
\item[] we will have proven compactness without detour through syntax!

%\item[] we will have given a purely semantic proof of compactness, independent of our conventional, syntactic derivation systems! 

\end{itemize}
\end{frame}

\begin{frame}
\frametitle{Step 1: \metav{E} compact entails SL is compact}
%\large

\begin{itemize}[<+->]

\item Assume that $(\metav{E}, \tau)$ is compact. Consider an arbitrary set $\Gamma$ of SL sentences that is finitely satisfiable. 
\item[] NTS:  $\Gamma$ is satisfiable (the other direction is trivial)

\item Consider an arbitrary wff \metav{P}. \textit{Lemma}: the set $U_{\metav{P}} \subset \metav{E}$ of TVAs that make \metav{P} true is open (proof: use disjunctive normal form and take a matching union of finite intersections of the $U^0_A$s and $U^1_B$s for atomics that compose \metav{P}!)
%or their negations on the truth-table rows where \metav{P} is true!)

\item So $U_{\enot \metav{P}}$ is also open. Since the complement of  $U_{\metav{P}}$ is $U_{\enot \metav{P}}$, $U_{\metav{P}}$ is both closed and open 
% of the sets of evaluations that make any wff true are clopen in this topology. This seems to be a general feature of Stone spaces, namely that we work with clopen sets. 

\end{itemize}
\end{frame}

\begin{frame}
\frametitle{Step 1 continued: applying topological compactness}
%\large

\begin{itemize}[<+->]

\item So, for each wff \metav{P} in $\Gamma$, the set of TVAs $U_{\metav{P}}$ that make \metav{P} true is a closed subset of \metav{E}

\item So to say that each finite subset $\Gamma_0$ of $\Gamma$ is satisfiable is equivalent to saying that the family $\{  U_{\metav{P}}:  \metav{P} \in \Gamma \}$ is a family of closed subsets of  \metav{E} with the \textit{finite intersection property} (i.e. for any finite subset of this family, the intersection of its members $U_{\metav{Q}}$ is non-empty)

\item Since we are assuming that \metav{E} is compact, the intersection of ALL members of this family $\{  U_{\metav{P}}:  \metav{P} \in \Gamma \}$ is non-empty

\item i.e. this intersection must contain at least one TVA in \metav{E}

\item Hence, there is a TVA that makes true all of the members of $\Gamma$ 

\end{itemize}
\end{frame}

\begin{frame}
\frametitle{Step 2: show that $(\metav{E}, \tau)$ is compact}
%\large

\begin{itemize}[<+->]

\item We can think about \metav{E} as equalling $2^\metav{A}$, i.e. the set of maps from the SL atomics $\metav{A}$ to the set $\{0, 1\}$
% not clear to me if this definition in terms of mapping elements to  0, 1 is equivalent to the power set. Maybe it's equivalent to mapping an element to a given set or not? i.e. as the powerset of the SL atomics \metav{A}, i.e. 

\item Equip the set $\{0, 1\}$ with the discrete topology (i.e. every subset is open). Then the product topology on $2^\metav{A}$ equals the topology $(\metav{E}, \tau)$ defined earlier. 

\item Since there are countably many SL atomics, $2^\metav{A}$ is homeomorphic to the Cantor set (comprises $\infty$-binary sequences of 0s and 1s)

% %this video, around 6:30minutes, basically illustrates this homeomorphism. 0 is like the TVA that assigns each atomic sentence false. 1 is the TVA that assigns each atomic true. then for each element of the cantor set, that is a TVA
% https://www.youtube.com/watch?v=eSgogjYj_uw
%note that Cantor set is powerset of a countably infinite set, so cantor set is uncountable!
% so for countably many atomics, there are uncountably many TVAs (by a diagonal argument!)

% elements of cantor set are infinite list of 0's and 1's

\item Note that the Cantor set is compact, since it is a closed subset of a compact set (namely the closed unit interval [0, 1])

\item If we allow \metav{A} to have arbitrarily many SL atomics, then we could use Tychonoff's theorem (equivalent to the axiom of choice) to show that $2^\metav{A}$ is compact 

\item \textit{Tychonoff}: a product of compact spaces is compact in the product topology 

\end{itemize}
\end{frame}




























