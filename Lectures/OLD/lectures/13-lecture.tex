% !TeX root = ./13-handout.tex

% %JH stuff to build in: review of semantic concepts in QL. 
%review of substitution instances and notation! see lecture 10 slides
%might want to review the rules as well, and note that we'll see this week why we need these restrictions on three of them!
%note that I can probably just copy over my nice fitch proofs from lecture 10 in terms of substitution instances! 
%look out for any place my slight discrepancy w/ logic book could matter! 

\setcounter{section}{12}
\section{Metalogic for QL}

\begin{frame}
%\large

\scriptsize{\tableofcontents}

\end{frame}

\begin{frame}
\frametitle{Soundness vs. Completeness}
%\large

% % It's perhaps interesting to think about why we do not have to demonstrate soundness and completeness results for truthtables. It is almost as if the syntax for truthtables is constitutive of the semantics.

% Could also make some verbal remarks about the notion of mathematical rigor, and how this has evolved over time. And how it might still be contested today, and issues of how rigorous physics ought to be remain highly relevant. Different methodological styles in physics and mathematical physics

\begin{itemize}[<+->]

\item Let $\Gamma$ be any set of \textit{sentences} of QL and $\Theta$ any sentence of QL. 

\item By proving that our derivation system is \textit{sound}, we show that QND derivations are `safe' (they preserve truth)

\medskip 

\bi

\item \emph{Sound}: If $\Gamma \vdash_{QND} \Theta$, then $\Gamma \entails \Theta$
%Single turnstile entails Double Turnstile 

\item (syntactic to semantic: i.e. we chose `good' rules!)

\ei

\bigskip 

\item By proving that QND is \textit{complete}, we show that reasoning about arbitrary models is not needed to demonstrate validity: \\ QND derivations suffice
%what about showing a set of sentences is unsatisfiable though? can we just not do this w/ SND derivations? would we need to introduce a falsum symbol? 

\medskip 

\bi

\item \emph{Complete}: If $\Gamma \entails \Theta$, then $\Gamma \vdash_{QND} \Theta$
%Double Turnstile entails  Single turnstile  

\item (logical entailment is fully covered by our syntactic rules)

\item (Means: we wrote down \textit{enough} rules!)

\ei

\end{itemize}
\end{frame}

\subsection{Truth and Satisfaction in QL}

\begin{frame}
\frametitle{Recap: models and interpretations}
%\large

\begin{itemize}[<+->]

\item Let $\mathfrak{L}$ be a first-order language, containing constants and $k$-place predicates (e.g. the language of QL)
\bi
\item recall that the atomic sentences of SL are 0th-place predicates
\ei
\item An $\mathfrak{L}$-model $\mathfrak{M} := (D, I)$ consists of

\begin{enumerate}

\item A non-empty set $D$ of objects, called the domain of   $\mathfrak{M}$

\item A map I (the \textit{interpretation} of $\mathfrak{M}$), which maps the vocabulary of $\mathfrak{L}$ to objects and ordered pairs from $D$ as follows:

\begin{itemize}

\item For each constant $c \in \mathfrak{L}$, $I(c)$ is an element of $D$, called the \textit{referent} or denotation of $c$

\item For each k-place predicate $P$ of $\mathfrak{L}$, $I(P)$ is a set of ordered $k$-tuples of objects in $D$, called the \textit{extension} of $P$
% $k$-place relation defined on $D$, 
% GB: you can think of a k-place relation as a subset of $D^k$, i.e. the space of k-tuples of objects in D. 

\end{itemize}

\end{enumerate}

\item Our text uses `models' and `interpretations' interchangeably, but the above disambiguation is convenient

\end{itemize}
\end{frame}

\begin{frame}
\frametitle{Truth in a Model: simple examples}
%\large

\begin{itemize}[<+->]

\item Consider a simple language $\mathfrak{L}$ comprising a one-place predicate $P$, a two-place predicate $R$, and constants $a$ and $b$. 

\item Fix an $\mathfrak{L}$-model $\mathfrak{M} := (D, I)$, e.g. our $D$ could be $\mathbb{N}$. \\ $I$ is what we would punch into \textit{Carnap} on PS9

\item $Pa$ is true in $\mathfrak{M}$ provided that the object $I(a)$ has the property $I(P)$, i.e. $I(a)$ lies in the extension of $P$. Then we'll write $\mathfrak{M} \entails Pa$

\item $Rab$ is true in $\mathfrak{M}$ provided that the objects $I(a)$ and $I(b)$ stand in relation $R$. In this case, we'll write $\mathfrak{M} \entails Rab$

\end{itemize}
\end{frame}

\begin{frame}
\frametitle{Satisfaction in a model: simple example}
%\large

\begin{itemize}[<+->]

\item What to say about something with free variables, such as $Px$?

\item This is a wff of QL but not a sentence (it is neither true nor false in a model)

\item Idea: for each object $r$ in $D$, we know whether $Px$ \textit{would be true} if we replaced $r$ for $x$, i.e. if $x$ stood for $r$ \\ (b/c we know the extension of $P$, i.e. all the objects that are $P$)

\item \textit{Shorthand}:if $Pc$ is true in $\mathfrak{M}$, then $I(c)$ \alert{satisfies} $Px$ in $\mathfrak{M}$
% if $Pr$ is true in $\mathfrak{M}$, then $r$ \alert{satisfies} $Px$ in $\mathfrak{M}$

\item \textit{Longhand}: define a variable assignment $\mathbf{d}_I$ that maps variables to objects. Then $\mathbf{d}_I$ \emph{satisfies} $Px$ provided that $\mathbf{d}_I (x)$ has property I(P). We can write $\mathfrak{M}_{\mathbf{d}_I} \entails Px$

\end{itemize}
\end{frame}

\begin{frame}
\frametitle{Satisfaction for Atomic Wffs}
%\large

\begin{itemize}[<+->]

\item \emph{Atomic wffs}: Suppose $\metav{Q}$ is atomic. Then $\metav{Q}$ is of the form $\metav{P}t_1\dots t_k$ where each $t_i$ is a term, i.e. a constant or a variable. 

\item $I$ assigns constants to objects $t_i^D$; $d_I$ maps variables to objects $t_i^D$ \\ (the text calls these denotations under $I$ or $d_I$ ``den$_{I, d_I} (t_i)$'')

\item $d_I$ satisfies $\metav{Q}$ provided that the k-tuple of these objects $\langle t_1^D, \dots , t_k^D \rangle$ lies in the extension of $\metav{Q}$, i.e. in $I(\metav{Q})$


\end{itemize}
\end{frame}

\begin{frame}
\frametitle{Satisfaction for Quantified Wffs}
%\large

\begin{itemize}[<+->]

\item For $r \in D$ we write $d_I [r/x]$ for the assignment that agrees with $d_I$ except necessarily assigning $r$ for $x$

\item \emph{Existentially Quantified}: Suppose we have a wff of the form $\qt{\exists}{x} \metav{Q}$. Then $d_I$ satisfies $\qt{\exists}{x} \metav{Q}$ provided there is SOME object $r \in D$ such that $d_I [r/x]$ satisfies $\metav{Q}$
\bi
\item Intuition: provided there's at least one thing you can plug in for $x$ such that $\metav{Q}$ comes out true
\ei

\item \emph{Universally Quantified}: Suppose we have a wff of the form $\qt{\forall}{x} \metav{Q}$. Then $d_I$ satisfies $\qt{\forall}{x} \metav{Q}$ provided $d_I [r/x]$ satisfies $\metav{Q}$ for EACH object $r \in D$ 
\bi
\item Intuition: no matter what you plug in for $x$, $\metav{Q}$ comes out true
\ei

%For $r \in D$ we write $d_I [r/x]$ for the assignment that agrees with $d_I$ except necessarily assigning $r$ for $x$. 

\end{itemize}
\end{frame}

\begin{frame}
\frametitle{From Satisfaction to Truth}
%\large

\begin{itemize}[<+->]

\item Focus on the sentences of QL, which have no free variables

\item \alert{Lemma}: given a model $\mathfrak{M} = (D, I)$ and a QL \textit{sentence} $\metav{P}$, either all variable assignments $d_I$ satisfy $\metav{P}$ or none do. 

\item Hence, we can define truth in a QL-model as follows: 

\item  A sentence $\metav{P}$ of QL is \emph{true} on model $\mathfrak{M}$ iff some variable assignment $d_I$ satisfies $\metav{P}$ in $\mathfrak{M}$

\item A sentence $\metav{P}$ of QL is \emph{false} on model $\mathfrak{M}$ otherwise, i.e. if no variable assignment $d_I$ satisfies $\metav{P}$ in $\mathfrak{M}$

%\item A sentence $\metav{P}$ of QL is true on model $\mathfrak{M}$ iff every variable assignment $d_I$ satisfies $\metav{P}$ in $\mathfrak{M}$

%\item A sentence $\metav{P}$ of QL is false on model $\mathfrak{M}$ iff NO variable assignment $d_I$ satisfies $\metav{P}$ in $\mathfrak{M}$

%on an interpretation $I$ (i.e. true in model $\mathfrak{M}_I$

% It seems like perhaps if I followed GB in focusing on the variable assignments that only make assignments to free variables, then I could define truth in a model more simply, namely as being satisfied by the trivial variable assignment $d_0$, which assigns no variables to any objects. This trivial variable assignment is the only one that applies when dealing with sentences, since there are no free variables in sentences.
%\item Now, consider the \textit{trivial variable assignment} $d_{I,0}$, which assigns no variables to any objects.



\end{itemize}
\end{frame}

\begin{frame}
  \frametitle{Shorthand: Truth of quantified sentences}
%from week 9 slides

  \begin{itemize}[<+->]
    \item $\qt{\exists}{x}\,\metav{A}\qv{x}$ is true iff $\metav{A}\qv{x}$ is \emph{satisfied} by \emph{at least one} object in $D$
    \begin{itemize}
\item $r = I(c)$ satisfies $\metav{A}\qv{x}$ in $\mathfrak{M}$ iff $\metav{A}\qv{c}$ is true in $\mathfrak{M}$ 
%in interpretation just like $I$, but with $o$ as referent of $c$
      %I think the following has typo: \item $o$ satisfies $\metav{A}\qv{x}$ iff $\metav{A}\qv{c}$ is true in interpretation just like $I$, but with $o$ as referent of $c$

\item e.g. there is at least one object in the domain that is an \metav{A}

\item Formally, there is a variable assignment $d_I$ with at least one variant $d_I [r/x]$ s.t. $d_I [r/x]$ satisfies $\metav{A}\qv{x}$
    \end{itemize}

\bigskip

    \item $\qt{\forall}{x}\,\metav{A}\qv{x}$ is true iff $\metav{A}\qv{x}$ is \emph{satisfied} by \emph{every} object in the domain
    \bi

\item e.g. everything in $D$ is an  \metav{A}
\item Formally, there is a variable assignment $d_I$ such that for each $r \in D$, each variant $d_I [r/x]$ satisfies $\metav{A}\qv{x}$
\ei
  \end{itemize}
\end{frame}

\begin{frame}
  \frametitle{Examples of Shorthand}
%from week 9 slides 
\large 
  \begin{itemize}
    \item $\qt{\exists}{x}\,(\metav{A}\qv{x} \eand \metav{B}\qv{x})$ is true iff \textcolor{OGlyallpink}{some} object satisfies `$\metav{A}\qv{x} \eand \metav{B}\qv{x}$'
    \begin{itemize}
      \item $o$ satisfies `$\metav{A}\qv{x} \eand \metav{B}\qv{x}$' iff it satisfies both $\metav{A}\qv{x}$ and $\metav{B}\qv{x}$
    \end{itemize}
    
\bigskip

    \item $\qt{\forall}{x}\,(\metav{A}\qv{x} \eif \metav{B}\qv{x})$ is true iff \emph{every} object satisfies `$\metav{A}\qv{x} \eif \metav{B}\qv{x}$'
    \medskip
    \begin{itemize} 
    \large 
      \item $o$ satisfies `$\metav{A}\qv{x} \eif \metav{B}\qv{x}$' iff
      either
      \bigskip
      \begin{itemize}  
      \normalsize
        \item $o$ does not satisfy $\metav{A}\qv{x}$ (vacuously true conditional)
        \item[] \makebox[\textwidth]{or}
        \item $o$ does satisfy $\metav{B}\qv{x}$
      \end{itemize}
       \bigskip
    \end{itemize}
  \end{itemize}
\end{frame}


\begin{frame}
\frametitle{Semantic Notions in QL}
%\large

\begin{itemize}[<+->]

\item Given a premise-set $\Gamma$ of QL-\textit{sentences} and a conclusion \textit{sentence} $\metav{Q}$, we have the following semantic notions:

\item \emph{Entailment}: $\Gamma$ QL-entails $\metav{Q}$ provided that there is no QL-model $\mathfrak{M}$ where $\Gamma$ is true but $\metav{Q}$ is false. We write $\Gamma \entails \metav{Q}$ \\ -- we say that the argument from $\Gamma$ to $\metav{Q}$ is \emph{QL-valid}

\item  \emph{Satisfiability}: we say that a set of sentences $\Gamma$ is jointly \emph{satisfiable} (aka QL-consistent) provided that there exists at least one QL-model $\mathfrak{M}$ where each sentence in $\Gamma$ is true 



\end{itemize}
\end{frame}

\subsection{Recap: Substitution Instances} %from week 10 lecture!

\begin{frame}
\frametitle{(Full) Substitution Instances}
%\large

\begin{itemize}

\item ``$\metav{Q}\substitute{\script{x}}{\script{c}}$'' is the sentence you get from $\qt{\forall}{\script{x}}\metav{Q}$ or $\qt{\exists}{\script{x}}\metav{Q}$ by dropping the quantifier and putting $\script{c}$ in place of \emph{every} $\script{x}$ in $\metav{Q}$

%\item $\metav{Q}\substitute{\script{x}}{\script{c}}$ is the sentence you get from $\qt{\forall}{\script{x}}\metav{Q}$ by dropping the $\qt{\forall}{\script{x}}$ quantifier and putting $\script{c}$ in place of \emph{every} $\script{x}$ in $\metav{Q}$. 

\item The other variables are untouched! 

\item Read ``$\substitute{x}{c}$'' as saying ``substitute $c$ for every $x$'', i.e. all the $x$'s are replaced by $c$'s! 

%\item $\metav{Q}\substitute{\script{x}}{\script{c}}$ can also arise from $\qt{\exists}{\script{x}}\metav{Q}$ by dropping the $\qt{\exists}{\script{x}}$ and putting $\script{c}$ in place of \emph{every} $\script{x}$ in $\metav{Q}$. 

%\item Equivalent notation: \metav{Q}\hspace{.15em}\raisebox{.3ex}{\fbox{$\script{x}\Rightarrow\script{c}$}}

\end{itemize}
\end{frame}

\begin{frame}
\frametitle{Some Examples of Substitution Instances}
%\large

\begin{itemize}

\item Instances of $\qt{\forall}{y}$Hy: 

\begin{itemize}

\item Ha, Hb, Hm$_{11}$  

\end{itemize}

\item Instances of $\qt{\exists}{z}$Haz: 

\begin{itemize}

\item Haa, Hab, Haj$_3$  

\end{itemize}

\item Instances of $\qt{\exists}{z}$$(Hz \eand Fzz)$:

\begin{itemize}

\item Remember to replace \emph{EVERY} occurance of $z$ with the chosen constant:

\item $(Ha \eand Faa)$, $(Hc \eand Fcc)$

\item The following are \textcolor{red}{NOT} substitution instances:

\item $(Ha \eand Faz)$, $(Hy \eand Faa)$, $(Ha \eand Fab)$

\end{itemize}

\end{itemize}
\end{frame}

\begin{frame}
\frametitle{Partial Substitution Instances}
%\large

\begin{itemize}

\item For Existential Introduction, we can use a partial substitution instance of the wff $\metav{Q}$:

\item ``$\metav{Q}\substitutesome{c}{\script{x}}$'' indicates that  the variable \script{x} replaces some but not necessarily all occurrences of the constant \script{c} in $\metav{Q}$. 

\item You can decide which occurrences of $\script{c}$ to replace and which to leave in place

\end{itemize}
\end{frame}

\begin{frame}
\frametitle{Examples of Partial Substitution Instances!}
%\large

\begin{itemize}

\item `$\metav{Q}\substitutesome{c}{\script{x}}$' indicates that  the variable \script{x} does not need to replace all occurrences of the constant \script{c} in $\metav{Q}$

\end{itemize}

\begin{multicols}{2}

\begin{fitchproof}
	\hypo{a}{Rdd}
	\have{d}{\qt{\exists}{x}Rxx} \Ei{a}
	\have{b}{\qt{\exists}{x}Rxd} \Ei{a}
	\have{c}{\qt{\exists}{z}Rdz} \Ei{a}
	\have{e}{\qt{\exists}{y}\qt{\exists}{z}Ryz} \Ei{c}
\end{fitchproof}
\columnbreak

\begin{center}
\textit{Existential Introduction} ($\exists$I) \vspace{-0.5em}
\begin{fitchproof}
\have[m]{a}{\metav{Q}}
	\have [\vdots] {n} {\hspace{2em} \vdots}
\have[s]{c}{\qt{\exists}{\script{x}}\metav{Q}\substitutesome{\script{c}}{\script{x}}} \Ei{a}
	%\have[s]{c}{\qt{\exists}{\script{x}}\metav{Q}(\dots \script{x} \dots \script{x} \dots)} \Ei{a}
\end{fitchproof}
%\vspace{-4em} \textbf{Provided that} $\script{x}$ does not occur already in $\metav{Q}(\script{c})$. \\ Note that $\script{x}$ may replace some or all occurrences of $\script{c}$.
    \end{center}
 -- Note: since $\metav{Q}$ is a sentence, and by our recursion clause for wff,  $\script{x}$ cannot occur in $\metav{Q}$. %\\[1em] -- As indicated by $\substitutesome{c}{\script{x}}$, $\script{x}$ \emph{may} replace \emph{just some} occurrences of $\script{c}$
 \vspace{5em}

\end{multicols}
\end{frame}

\begin{frame}
\frametitle{Substitution Lemma (Logic Book 11.1.1)}
%\large

\begin{itemize}[<+->]

\item Consider $\metav{Q} := Fxx$. Then $\metav{Q}\substitute{x}{c} = Fcc$

\item  ``variant $d_I [I(c) /x]=d_I [r/x]$ satisfies $Fxx$'' means that when we assign $x$ to $r=I(c)$, Fcc is true ($\langle r, r \rangle \in Extension(F)$) %Fxx is satisfied, i.e. 
%(so $r$ Fs itself)

\item ``$d_I$ satisfies $\metav{Q}\substitute{x}{c}$'' means roughly that whatever objects $d_I$ assigns variables, the result lies in the Extension of $\metav{Q}$

\item Note that since $x$ doesn't appear in $Fcc$, $d_I$ treats $Fcc$ just like $d_I [I(c) /x]$ treats $Fxx$

%\item means that $d_I$ assigns $x$ to an object $r=I(c)$ such that object $r$ bears relation F to itself ($Fcc \in Extension(F)$) 



\item ``$d_I$ satisfies $\metav{Q}\substitute{x}{c}$'' is equivalent to ``$d_I [I(c) /x]$ satisfies $\metav{Q}$''

\item \emph{Substitution Lemma}: let $\metav{Q}$ be a wff of QL. The variable assignment $d_I$ satisfies $\metav{Q}\substitute{\script{x}}{\script{c}}$ if and only if $d_I [I(\script{c})/\script{x}]$ satisfies $\metav{Q}$



\end{itemize}
\end{frame}



\subsection{QL rules recap}

\begin{frame}
\frametitle{Rules for the Universal Quantifier}

\begin{multicols}{2}

\textit{Universal Elimination} ($\forall$E) \vspace{0em}

\begin{fitchproof}
	\have[m]{a}{\qt{\forall}{\script{x}}\metav{Q}}
	\have [\vdots] {n} {\hspace{2em} \vdots}
	\have[s]{c}{\metav{Q}\substitute{\script{x}}{\script{c}}} \Ae{a}

\end{fitchproof}
-- Note that you replace \emph{EVERY} instance of $\script{x}$ with $\script{c}$ \\[1em] -- Notation: $\metav{Q}\substitute{\script{x}}{\script{c}}$ \\[1em] -- read ``$\script{c}$ for $\script{x}$''
\columnbreak

\textit{Universal Introduction} ($\forall$I) \vspace{0em}

\begin{fitchproof}
	\have[m]{a}{\metav{Q}}
	\have [\vdots] {n} {\hspace{2em} \vdots}
	\have[s]{c}{\qt{\forall}{\script{x}}\metav{Q}\substitute{\script{c}}{\script{x}}} \Ai{a}
\end{fitchproof}

\textbf{Provided that both} \\
\textbf{(i)} $\script{c}$ does not occur in any other undischarged assumptions that \metav{Q} is in the scope of. \\
\textbf{(ii)} $\script{x}$ does not occur already in $\metav{Q}$.
\vspace{2em}
\end{multicols} 

\end{frame}

%\iffalse
\begin{frame}
\frametitle{Rules for the Existential Quantifier}
\footnotesize


\begin{multicols}{2}
\begin{center}
\textit{Existential Introduction} ($\exists$I) \vspace{-0.5em}
\begin{fitchproof}
\have[m]{a}{\metav{Q}}
	\have [\vdots] {n} {\hspace{2em} \vdots}
\have[s]{c}{\qt{\exists}{\script{x}}\metav{Q}\substitutesome{\script{c}}{\script{x}}} \Ei{a}
	%\have[s]{c}{\qt{\exists}{\script{x}}\metav{Q}(\dots \script{x} \dots \script{x} \dots)} \Ei{a}
\end{fitchproof}
%\vspace{-4em} \textbf{Provided that} $\script{x}$ does not occur already in $\metav{Q}(\script{c})$. \\ Note that $\script{x}$ may replace some or all occurrences of $\script{c}$.
    \end{center}
 -- \textbf{Provided that} $\script{x}$ does not occur already in $\metav{Q}$. \\[1em] -- As indicated by $\substitutesome{\script{c}}{\script{x}}$, $\script{x}$ \emph{may} replace \emph{just some} occurrences of $\script{c}$
 \vspace{5em}
\columnbreak

\begin{center}
\textit{Existential Elimination} ($\exists$E) \vspace{-1em}
\begin{fitchproof}
	\have[m]{a}{\qt{\exists}{\script{x}} \metav{Q}}
	\have [\ ] {n} {\hspace{2em} \vdots}
	\open	
		\hypo[n]{b}{\metav{Q}\substitute{\script{x}}{\script{c}}} \as{for $\exists$E}
		\have [\ ] {n2} {\hspace{2em} \vdots}
		\have[s]{c}{\metaB{}}
	\close
	%\have [\ ] {n3} {\hspace{2em} \vdots}
	\have[s+1]{d}{\metaB{}} \Ee{a,b-c}
\end{fitchproof}
    \end{center}
    \textit{Simplified}: \textbf{provided that} $\script{c}$ doesn't occur \textbf{anywhere else outside} the subproof
     \vspace{1em}
\end{multicols} 
\end{frame}
%\fi

\begin{frame}
\frametitle{Motivating these restrictions on various rules!}
%\large

\begin{itemize}[<+->]

\item We'll now see why the rules $\forall I$ and $\exists E$ require us to follow the stated, non-trivial restrictions

\item Without these restrictions, earlier sentences in the derivation would not semantically entail later sentences

\item For QND to be sound, we need $\Gamma \vdash_{QND} \metav{P}$ to be sufficient for $\Gamma \entails \metav{P}$. 

\item As with SND, we will prove this by showing that the set of open assumptions $\Gamma_k$ on line \#$k$ semantically entail the sentence $\metav{P}_k$ on that line, for all lines $k$ in any QND derivation 



\end{itemize}
\end{frame}

\iffalse %***********************************************************
\subsection{A Meta-refresher}

\begin{frame}
\frametitle{SND as a derivation system, provided that...}
%\large

\begin{itemize}[<+->]

\item As we have seen, Sentential Natural Deduction allows us to derive a conclusion from a set of premises:

%trees provide a shortcut for demonstrating that a set of sentences is inconsistent (i.e. unsatisfiable): construct a tree whose root is these sentences s.t. all branches close
%JH: interesting that SND seemingly does NOT let you show that a set of sentences is unsatisfiable; interesting diff in problem-solving support

%\item Underwrites further shortcuts for demonstrating:

\begin{enumerate}[1.)]

\item valid argument: conclusion on last line, in scope of just premises %trees: \\ (its premises and negated conclusion are unsatisfiable)

\item tautology: on last line in scope of NO premises %that a sentence is a tautology (its negation is unsatisfiable)

\item two logically equivalent sentences: (i) their biconditional is a tautology or (ii) derive one from the other and vice versa (which mirrors biconditional introduction!)

\end{enumerate}

\item But our derivations are justified only if system SND is \textit{sound}

\item And guaranteed to have a derivation for every valid argument only if system SND is \textit{complete}

%of a semantically valid argument only if system SND is \textit{complete}



\end{itemize}
\end{frame}

\begin{frame}
\frametitle{A tale of three turnstiles: one semantic; two syntactic}
%\large

\begin{itemize}[<+->]

\item Double Turnstile $\entails$: logical entailment (indexed to our choice of semantics, i.e. the truth-tables for our connectives)

\item Single Turnstile Tree $\vdash_{STD}$: tree-validity in STD \\ (i.e. premises and negated conclusion as root of a tree whose branches all close---recall that this means that $\Gamma \cup \{\enot \Theta\}$ is \emph{tree-inconsistent}) %is interesting how this idea---that unsatisfiability of premises and negated conclusion---is really at heart of our SND completeness proof!
% this means that $\Gamma \cup \{\enot \Theta\}$ is \emph{tree-inconsistent}: \\ There is a tree with this set as the root s.t. \emph{all branches close}

\item Single Turnstile Natural $\vdash_{SND}$: \emph{derivability} in SND

\end{itemize}
\end{frame}

\begin{frame}
\frametitle{A Tale of Three Turnstiles $\entails$ the semantic one}
%\large

\begin{itemize}[<+->]

%\item Recall that the double turnstile `$\entails$' stands for semantic entailment (aka logical consequence) within (classical) sentential logic SL. 

\item ``$\Gamma \entails \Theta$'' means that $\Gamma$ logically entails $\Theta$ \\ Whenever the premises in $\Gamma$ are true, the conclusion $\Theta$ is true 

\item Equivalently: there is no truth-value assignment (TVA) s.t. \\ $\Gamma$ is satisfied while $\Theta$ is false

\item Equivalently, this means that \emph{$\Gamma \cup \{\enot \Theta\}$ is unsatisfiable}: \\ no TVA satisfies the premises and negated conclusion  

\item We'll use this last fact A LOT in our proof that SND is complete! %completeness! 

\end{itemize}
\end{frame}


\fi %***********************************************************









\subsection{Soundness of System QND}

\begin{frame}
\frametitle{Semantic entailment for infinitely-many premises}
%\large

\begin{itemize}[<+->]

\item Let $\Gamma$ be a possibly infinite set of QL-\emph{sentences}; $\Theta$ a conclusion

%\item Recall: a TVA assigns `True' or `False' to the (infinitely-many) SL atomic wffs

%\item In the case where $\Gamma$ is finite, its premises contain finitely-many atomic wffs, so we can restrict a TVA to a row of a truth table

\item An argument is \emph{semantically invalid} if there is a model $\mathfrak{M}$ that makes true each sentence in $\Gamma$ but which makes $\Theta$ false

\item In this case we write $\Gamma \nentails \Theta$

\item If there is no such QL-model, then $\Gamma \entails \Theta$, i.e. if whenever we have $\mathfrak{M} \entails \Gamma$ we also have $\mathfrak{M} \entails \Theta$

\end{itemize}
\end{frame}

\begin{frame}
\frametitle{QND derivability for infinitely-many premises}
%\large

\begin{itemize}

\item $\Theta$ is \emph{QND-derivable} from $\Gamma$ provided there is an QND derivation:

\begin{enumerate}[1.)]

\item whose starting premises $\Delta$ are a finite subset of $\Gamma$ 

\item in which $\Theta$ appears on its own in the final line

\item where $\Theta$ is directly next to the main scope line, i.e. only in the scope of the $\Delta$-premises

\end{enumerate}

\item In this case, we write $\Gamma \vdash_{QND} \Theta$ (also: $\Delta \vdash_{QND} \Theta$)

\item If no such derivation exists, then we say that $\Theta$ is NOT QND-derivable from $\Gamma$, and we write $\Gamma \nvdash_{QND} \Theta$

\end{itemize}
\end{frame}

\subsubsection{More Righteousness?!}

\begin{frame}
\frametitle{Soundness: Proof Idea and notation}
%\large

\begin{itemize}[<+->]

\item Subgoal: given any line in a QND derivation, show that the QL-\textit{sentence} on that line is entailed by the \\ premises or assumptions accessible from that line


\item Let ``\emph{$P_k$}" be the sentence on line $k$ of our derivation%, i.e. the $k$-th sentence %in our derivation

\item Let ``\emph{$\Gamma_k$}" be the set of premises/assumptions accessible on line $k$, i.e. the set of open assumptions/premises in whose scope $P_k$ lies

\item \emph{Subgoal}: given a sentence $P_k$ on line $k$, show that $\Gamma_k \entails P_k$

%\item (like with soundness for trees, we reason ``from the top down")

\end{itemize}
\end{frame}

\begin{frame}
\frametitle{Soundness: Proof Strategy}
%\large

\begin{itemize}

\item Recall that QND derivations are defined recursively: \\ from a (possibly empty) set of premises, we have a finite number of rules to add a line 

\item[] -- These ways include all our SND rules plus an intro and elimination rule for our quantifiers $\forall$ and $\exists$

\item Hence: do induction on the number of lines in an QND derivation

\item Show that the base case has the property (line \#1)

\item Induction hypothesis: assume the property holds for all lines $\leq k$. 

\item Induction step: show the property holds for line \#k+1 \\ (by considering all possible ways line \#k+1 could arise)

\end{itemize}
\end{frame}

\begin{frame}
\frametitle{Let's remain Righteous!}
%\large

\begin{itemize}[<+->]

\item Recall: a line $i$ of a derivation is \emph{righteous} just in case $\Gamma_i \entails P_i$, i.e. just in case \textbf{the set of assumptions/premises accessible from $i$} semantically entail the sentence on that line. 

\item Call a derivation \textit{righteous} if every line in it is righteous

\item Our goal is to prove that every derivation in QND is righteous!

\item We will extend our induction for SND to cover our four new rules! 

%\item If $X \vdash_{SND} P$, then . 

%\item Then, if we can SND-derive a wff $P$ from some set $X$, we'll have $\Gamma \entails P$ and $\Gamma$ will just be a subset of premises in $X$ in whose scope $P$ lies. 



\end{itemize}
\end{frame}

\begin{frame}
\frametitle{From righteousness to soundness:}
%\large

\begin{itemize}[<+->]

%From Righteousness to Soundness

\item Let $\Gamma$ be any set of QL sentences (possibly infinite)

\item If $\Gamma \vdash_{QND} \metav{P}$, then by definition there is a derivation from finitely-many premises $\Delta \subseteq \Gamma$, such that \metav{P} occurs on the final line and lies in the scope of $\Delta$ (i.e. $\Delta \vdash_{QND} \metav{P}$)

\item Then by righteousness, $\Delta \entails \metav{P}$

\item[] -- i.e. any model $\mathfrak{M}$ that makes $\Delta$ true must make $\metav{P}$ true
%any TVA that makes $\Delta$ true must make $\metav{P}$ true

%\item i.e. any TVA 

\item So there is no QL-model that makes all the sentences in $\Gamma$ true while making $\metav{P}$ false, so $\Gamma \entails \metav{P}$ as well

\item So we will have shown \emph{Soundness}: If $\Gamma \vdash_{QND} \metav{P}$, then $\Gamma \entails \metav{P}$

\end{itemize}
\end{frame}

\subsubsection{Soundness: the proof itself}

\begin{frame}
\frametitle{Base Case}
%\large

\begin{itemize}

\item \emph{Base case}: for any QND derivation, show that $\Gamma_1 \entails \metav{P}_1$.

\item Proof: $\Gamma_1$ is the set of premises accessible at line \#1, which comprises exactly the QL-sentence $\metav{P}_1$ 

\item (recall that every premise of a derivation lies in its own scope)
%\item[] \qquad i.e. these premises be gettin' high off their own supply)

\item Clearly, $\metav{P}_1 \entails \metav{P}_1 $, so $\{ \metav{P}_1 \} \entails \metav{P}_1$

\item So line \#1 is righteous (i.e. $\Gamma_1 \entails \metav{P}_1$)

\end{itemize}
\end{frame}

\begin{frame}
\frametitle{Stating the Induction Step}
%\large

\begin{itemize}

\item  \emph{Induction Hypothesis}: Assume that every line $i$ for $1 < i \leq k$ is righteous (i.e. that $\Gamma_i \entails \metav{P}_i $)

\item Induction step: Consider line \#k+1; show that $\Gamma_{k+1} \entails \metav{P}_{k+1}$

\item We have 16 cases to consider! We have essentially already considered 12 of these from our soundness proof for SND
%11 of these arise from our 11 SND-sanctioned rules for extending a derivation. 

\item We have four new cases: our intro. and elimin. rules for $\forall$ and $\exists$

%\item What is the 12th case?? (We could say 13, but that is BAD LUCK)

\end{itemize}
\end{frame}

\begin{frame}
\frametitle{Cases 1--12: modifying our soundness proof for SND}
%\large

\begin{itemize}[<+->]

\item For each of these 12 cases, we simply replace ``truth-value assignments'' with ``QL-models'' (or interpretations), along with replacing truth-functional semantic notions with ones defined for quantifier logic

\item e.g. quantificational entailment, quantificational consistency/satisfiability 

\item e.g. ``$\Gamma_{k+1} \entails \metav{P}_{k+1}$'' now means ``sentence $\metav{P}_{k+1}$ is true in all models that make-true the premise set $\Gamma_{k+1}$''. 

\item Equivalently: $\Gamma_{k+1} \cup \{ \enot \metav{P}_{k+1} \}$ is unsatisfiable in QL


\end{itemize}
\end{frame}


\begin{frame}
\frametitle{Case 13 (gasp): For-all Elimination}
%\large

\begin{itemize}[<+->]

\item \emph{Case 13}: $\metav{P}_{k +1}$ is derived by Universal Elimination (:$\forall E$) \\ Show that $\Gamma_{k+1} \entails \metav{P}_{k+1}$

\item $\metav{P}_{k+1}$ must have the form $\metav{Q}\substitute{\script{x}}{\script{c}}$ (read ``$\script{c}$ for $\script{x}$'')

\item By the IH,  line \#h is righteous, so $\Gamma_{h} \entails \qt{\forall}{\script{x}}\metav{Q}$

\item Since every assumption that is accessible at line \#h is also accessible at line \#k+1, we have $\Gamma_h \subseteq \Gamma_{k+1}$

\item Hence, $\Gamma_{k+1} \entails \qt{\forall}{\script{x}}\metav{Q}$

\item \textbf{Lemma}: a universally quantified sentence entails each of its substitution instances. \\ $\Rightarrow$ any model that makes-true $\qt{\forall}{\script{x}}\metav{Q}$ also makes-true $\metav{Q}\substitute{\script{x}}{\script{c}}$
% %JH: include a proof of this fact 11.1.4 on next slide!

\item Hence, $\Gamma_{k+1} \entails \metav{Q}\substitute{\script{x}}{\script{c}}$

\end{itemize}
\end{frame}

\begin{frame}
\frametitle{Lemma for Case 13}
%\large

\begin{itemize}[<+->]

\item \emph{Lemma}: a universally quantified sentence entails each of its substitution instances: $\qt{\forall}{\script{x}}\metav{Q} \entails \metav{Q}\substitute{\script{x}}{\script{c}}$ for each constant $\script{c}$ %note that the constants are names; not in the Domain themselves. objects are in the domain 

\item Consider an arbitrary model $\mathfrak{M}$ that makes true $\qt{\forall}{\script{x}}\metav{Q}$
\item Then by defN of true-in-QL, there is some variable assignment $d_I$ that satisfies $\qt{\forall}{\script{x}}\metav{Q}$ in $\mathfrak{M}$
\item By the satisfaction-conditions for univ. quant. sentences, this means that for each $r \in D$, the variant $d_I [r/x]$ satisfies $\metav{Q}$

\item So for each $\script{c}$, $I$ must assign $\script{c}$ an object $r \in D$ s.t. $d_I [I(\script{c})/x]$ satisfies $\metav{Q}$. %So this holds for any particular $\script{c}$ %This holds in particular for the 

\item Hence, variable assignment $d_I$ satisfies $\metav{Q}\substitute{\script{x}}{\script{c}}$ (Lemma 11.1.1)

\item (Intuition: no matter which $r$ in $D$ is assigned to $\script{c}$, $\metav{Q}\substitute{\script{x}}{\script{c}}$ is true)
 
\item So, for each constant  $\script{c}$, $\mathfrak{M} \entails \metav{Q}\substitute{\script{x}}{\script{c}}$, i.e. is true in the model
%i.e. the substitution instance is true in model M

\end{itemize}
\end{frame}


\iffalse

\begin{frame}
\frametitle{OLD Lemma for Case 13}
%\large

\begin{itemize}[<+->]

\item \textit{Lemma}: a universally quantified sentence entails each of its substitution instances, \\ $\qt{\forall}{\script{x}}\metav{Q} \entails \metav{Q}\substitute{\script{x}}{\script{c}}$ for each $\script{c} \in Domain$ \textit{of Discourse}

\item Consider an arbitrary model that satisfies $\qt{\forall}{\script{x}}\metav{Q}$. 
\item Then by the satisfaction-conditions for universally quantified sentences, this means that whatever constant in the UD we replace $\script{x}$ with, the resulting substitution instance is satisfied

\item So, $\metav{Q}\substitute{\script{x}}{\script{c}}$ is satisfied for each $\script{c} \in UD$

\end{itemize}
\end{frame}

\fi 

\begin{frame}
\frametitle{Case 14: Existential Introduction}
%\large

\begin{itemize}[<+->]

\item \emph{Case 14}: $\metav{P}_{k +1}$ is derived by Existential Introduction (:$\exists I$) %\\ Show that $\Gamma_{k+1} \entails \metav{P}_{k+1}$

\item $\metav{P}_{k+1}$ must have the form $\qt{\exists}{\script{x}}\metav{Q}\substitutesome{\script{c}}{\script{x}}$ (read ``$\script{x}$ for some $\script{c}$'')

\item By the IH,  line \#h is righteous, so $\Gamma_{h} \entails \metav{Q}$ (where $\script{c}$ appears)

\item Since every assumption that is accessible at line \#h is also accessible at line \#k+1, we have $\Gamma_h \subseteq \Gamma_{k+1}$

\item Hence, $\Gamma_{k+1} \entails \metav{Q}$

\item \textbf{Lemma}: a sentence $\metav{Q}$ entails any existentially quantified (possibly partial) substitution instance $\qt{\exists}{\script{x}}\metav{Q}\substitutesome{\script{c}}{\script{x}}$. 

%note that by our recursive defN of QL wffs, we auto-enforce the one restriction on rule EI, namely that $\script{x}$ does not occur already in \metav{Q}(\script{c})

%a universally quantified sentence entails each of its substitution instances, \\ i.e. any model that satisfies $\qt{\forall}{\script{x}}\metav{Q}$ also satisfies $\metav{Q}\substitute{\script{x}}{\script{c}}$
% %JH: include a proof of this fact 11.1.4 on next slide!

\item Hence, $\Gamma_{k+1} \entails \qt{\exists}{\script{x}}\metav{Q}\substitutesome{\script{c}}{\script{x}}$

\end{itemize}
\end{frame}


\begin{frame}
\frametitle{Lemma for Case 14}
%\large

\begin{itemize}[<+->]

\item \emph{Lemma}: a sentence $\metav{Q}$ entails any existentially quantified (possibly partial) substitution instance $\qt{\exists}{\script{x}}\metav{Q}\substitutesome{\script{c}}{\script{x}}$

\item Consider an arbitrary model $\mathfrak{M}$ that makes true $\metav{Q}[\script{c}]$
%where we introduce some metalanguage notation to show that constant c appears in Q

\item Then by defN of true-in-QL, there is some variable assignment $d_I$ that satisfies $\metav{Q}[\script{c}]$ in $\mathfrak{M}$. Let $r$ be the object in $D$ that $\script{c}$ stands for.

\item Then $d_I [r/\script{x}]$ satisfies $\metav{Q}\substitutesome{\script{c}}{\script{x}}$ (i.e. the open sentence we get by replacing some \script{c}'s with $\script{x}$ is satisfied by object $r$)


\item Recall: $d_I$ satisfies $\qt{\exists}{\script{x}}\metav{Q}\substitutesome{\script{c}}{\script{x}}$ provided there is some object $r \in D$ s.t. $d_I [r/\script{x}]$ satisfies $\metav{Q}\substitutesome{\script{c}}{\script{\script{x}}}$

\item Hence, by the defN of true-in-QL, $\qt{\exists}{\script{x}}\metav{Q}\substitutesome{\script{c}}{\script{x}}$ is true in $\mathfrak{M}$


\end{itemize}
\end{frame}





\begin{frame}
\frametitle{Case 16: Existential Elimination}
%\large

\begin{itemize}[<+->]

\item \emph{Case 16}: $\metav{P}_{k +1}$ is derived by Existential Elimination (:$\exists E$) %\\ Show that $\Gamma_{k+1} \entails \metav{P}_{k+1}$

%\item Denote``$\metav{P}_{k+1}$'' by ``$\Psi$'' for convenience

\item We require that $\script{c}$ not occur in $\qt{\exists}{\script{x}} \metav{Q}$, $\metav{P}_{k +1}$, or $\Gamma_{k+1}$
% I suppose that technically, we require the constant c not to occur in the premise set for line h. More specifically, the premise set for line j minus sentence P_j 

%\item So $\Gamma_m  \subseteq \Gamma_{k+1} \cup \{ \metav{Q}\substitute{\script{x}}{\script{c}} \}$


\item By the IH,  lines \#h and \#m are righteous, so $\Gamma_{h} \entails \qt{\exists}{\script{x}} \metav{Q}$ and $\Gamma_{m} \entails \metav{P}_{k +1}$

\item Since every assumption/premise that is accessible at line \#h is also accessible at line \#k+1, we have $\Gamma_h \subseteq \Gamma_{k+1}$. So \alert{$\Gamma_{k+1} \entails  \qt{\exists}{\script{x}} \metav{Q}$}

\item Note that every member of $\Gamma_m$ is accessible at \#k+1 except assumption $\metav{Q}\substitute{\script{x}}{\script{c}}$ on line \#j. So $\Gamma_m  \subseteq \Gamma_{k+1} \cup \{ \metav{Q}\substitute{\script{x}}{\script{c}} \}$

\item So since $\Gamma_{m} \entails \metav{P}_{k +1}$, we have $\Gamma_{k+1} \cup \{ \metav{Q}\substitute{\script{x}}{\script{c}} \} \entails \metav{P}_{k +1}$

\item \textbf{Lemma}: if (i) constant $\script{c}$ does not occur in $\qt{\exists}{\script{x}} \metav{Q}$, $\metav{P}$, or set $\Gamma$, \\ (ii) $\Gamma \entails \qt{\exists}{\script{x}} \metav{Q}$ and (iii) $\Gamma \cup \{ \metav{Q}\substitute{\script{x}}{\script{c}} \} \entails \metav{P}$, then $\Gamma \entails \metav{P}$

\item Hence, $\Gamma_{k+1} \entails \metav{P}_{k +1}$, so line \#k+1 is righteous! 


\end{itemize}
\end{frame}

\begin{frame}
\frametitle{Locality Lemma (Book's 11.1.7)}
%\large

\begin{itemize}[<+->]

\item \emph{Locality Lemma}: `local agreement' on interpretation between models arises iff there is `agreement' on entailment relations. 

\item Set-up (for a given sentence $\metav{P}$): Consider two QL-models $\mathfrak{M}^1 := (D, I_1)$ and $\mathfrak{M}^2 := (D, I_2)$ with the same domain $D$, whose interpretation functions $I_1$ and $I_2$ give the same interpretations for any constants or predicates appearing in QL-sentence $\metav{P}$ 
\item[] (so any differences between $\mathfrak{M}^1$ and $\mathfrak{M}^2$ arise from how they interpret QL-symbols NOT appearing in $\metav{P}$). 

\item Then  $\mathfrak{M}^1 \entails \metav{P}$ if and only if $\mathfrak{M}^2 \entails \metav{P}$

\item We will use this lemma for the cases of $\forall I$ and $\exists E$


% Note that the logic book explicitly considers sentence letters as well, which I considering to be 0th place predicates

%agree on how to interpret in $D$



\end{itemize}
\end{frame}

\begin{frame}
\frametitle{Lemma for Case 16}
%\large

\begin{itemize}[<+->]

\item \emph{Lemma}: if (i) constant $\script{c}$ does not occur in $\qt{\exists}{\script{x}} \metav{Q}$, $\metav{P}$, or set $\Gamma$, \\ (ii) $\Gamma \entails \qt{\exists}{\script{x}} \metav{Q}$ and (iii) $\Gamma \cup \{ \metav{Q}\substitute{\script{x}}{\script{c}} \} \entails \metav{P}$, then $\Gamma \entails \metav{P}$

\item NTS: In a model $\mathfrak{M}:=(D, I)$ that makes all members of $\Gamma$ true (i.e. $\mathfrak{M} \entails \Gamma$), $\metav{P}$ is true (i.e. show that $\mathfrak{M} \entails \metav{P}$)

%\item NTS: there exists a variable assignment $d_I$ that satisfies $\metav{P}$ in a model $\mathfrak{M}$ that makes all members of $\Gamma$ true, i.e. where $\mathfrak{M} \entails \Gamma$

\item Since $\Gamma \entails \qt{\exists}{\script{x}} \metav{Q}$, there exists some object $r \in D$ that satisfies $\metav{Q}$ (i.e. there exists a $d_I$ s.t. $d_I [r/ \script{x}]$ satisfies $\metav{Q}$) 

\item Since $\script{c}$ does not occur in $\qt{\exists}{\script{x}} \metav{Q}$, $\metav{P}$, or set $\Gamma$, we can define a new model $\mathfrak{M}'$ that is just like $\mathfrak{M}$ except that $I'(\script{c}) = r$. 

\item Then since $d_I [r/ \script{x}]$ satisfies $\metav{Q}$, we have $d_{I'} [r/ \script{x}] = d_{I'} [I'(\script{c})/ \script{x}]$ satisfies $\metav{Q}$ as well. 

\item So by Lemma 11.1.1, $d_{I'}$ satisfies $\metav{Q}\substitute{\script{x}}{\script{c}}$, so $\mathfrak{M}' \entails \metav{Q}\substitute{\script{x}}{\script{c}}$

%%i think i need to fix the following to use 11.1.1
%\item Then since $d_I [r/ \script{x}]$ satisfies $\metav{Q}$, $d_{I'}[r/ \script{x}]$ satisfies $\metav{Q}\substitute{\script{x}}{\script{c}}$, so $\mathfrak{M}' \entails \metav{Q}\substitute{\script{x}}{\script{c}}$. 

\item By Locality, since $\mathfrak{M} \entails \Gamma$, we have $\mathfrak{M}' \entails \Gamma$ too 

\item So $\mathfrak{M}' \entails \Gamma \cup \{ \metav{Q}\substitute{\script{x}}{\script{c}} \}$ which by assumption $ \entails \metav{P}$. So $\mathfrak{M}' \entails \metav{P}$

\item Hence, by Locality, $\mathfrak{M} \entails \metav{P}$, so $\Gamma \entails \metav{P}$

%\item Since $\Gamma \cup \{ \metav{Q}\substitute{\script{x}}{\script{c}} \} \entails \metav{P}$, 

%Letting $I(\script{c}) = r$, $d_I[r/ \script{x}]$ satisfies $\metav{Q}\substitute{\script{x}}{\script{c}}$

%\item Hence, $d_I[r/ \script{x}]$


\end{itemize}
\end{frame}





\begin{frame}
\frametitle{Case 15: For-all Introduction}
%\large

\begin{itemize}[<+->]

\item \emph{Case 15}: $\metav{P}_{k +1}$ is derived by Universal Introduction (:$\forall I$) %\\ Show that $\Gamma_{k+1} \entails \metav{P}_{k+1}$

\item $\metav{P}_{k+1}$ must have the form $\qt{\forall}{\script{x}}\metav{Q}\substitute{\script{c}}{\script{x}}$, with $\script{c}$ not appearing in $\Gamma_h$ %(read ``$\script{c}$ for $\script{x}$'') %logic book has scope of \Gamma_{k+1} here. think about whether this difference matters. did i screw up on my own rule sheet?

% %JH: we do seem to have a major problem perhaps lol. my rule sheet says that c does not occur in the scope of any assumptions that Q is in the scope of, but Q is in its own scope and c occurs in Q obviously. So that seems to screw up any nice statement here! 
%solution! line h is not a premise or assumption! so line h is not itself in \Gamma_h! so I think i'm good after all. whew. 

\item By the IH,  line \#h is righteous, so $\Gamma_{h} \entails \metav{Q}$ (where $\script{c}$ appears)

\item Since every assumption/premise that is accessible at line \#h is also accessible at line \#k+1, we have $\Gamma_h \subseteq \Gamma_{k+1}$

\item Hence, $\Gamma_{k+1} \entails \metav{Q}$

\item \textbf{Lemma}: if $\script{c}$ does not appear in any member of set $\Gamma$, then if $\Gamma \entails \metav{Q}$, we have $\Gamma \entails \qt{\forall}{\script{x}}\metav{Q}\substitute{\script{c}}{\script{x}}$

%logic book: if $\script{c}$ does not appear in $\qt{\forall}{\script{x}}\metav{Q}\substitute{\script{c}}{\script{x}}$ or any member of a set $\Gamma$, then if $\Gamma \entails \metav{Q}$, we have $\Gamma \entails \qt{\forall}{\script{x}}\metav{Q}\substitute{\script{c}}{\script{x}}$

% Note that based on our notation, it is trivial that the constant c does not occur in the universally-quantify conclusion, since our notation tells us to replace each instance of c with x. 
% Instead, what we need to assume is that the variable x does not occur already in Q, but note that if it did, then the conclusion for-allx Q wouldn't even be a wff! so again this one seems auto-enforced! 

% a universally quantified sentence entails each of its substitution instances, \\ i.e. any model that satisfies $\qt{\forall}{\script{x}}\metav{Q}$ also satisfies $\metav{Q}\substitute{\script{x}}{\script{c}}$
% %JH: include a proof of this fact 11.1.4 on next slide!

\item Our rule $\forall I$ requires that $\script{c}$ does not appear in $\Gamma_{k+1}$, so by the lemma, $\Gamma_{k+1} \entails \qt{\forall}{\script{x}}\metav{Q}\substitute{\script{c}}{\script{x}}$

\end{itemize}
\end{frame}




\begin{frame}
\frametitle{Lemma for Case 15}
%\large

\begin{itemize}[<+->]

\item \emph{Lemma}: if $\script{c}$ does not appear in any member of set $\Gamma$, then if $\Gamma \entails \metav{Q}$, we have $\Gamma \entails \qt{\forall}{\script{x}}\metav{Q}\substitute{\script{c}}{\script{x}}$

\item Consider an arbitrary model $\mathfrak{M}$ that makes true all of the sentences in $\Gamma$. Then $\mathfrak{M} \entails \metav{Q}$ (i.e. $\metav{Q}$ is true in $\mathfrak{M}$) %$\metav{Q}[c]$
%where we introduce some metalanguage notation to show that constant c appears in Q

\item So there exists a variable assignment $d_I$ that satisfies $\metav{Q}$ in $\mathfrak{M}$

%\item By 11.1.1, $d_I$ satisfies $\metav{Q}\substitute{\script{c}}{\script{x}}$ iff $d_I[I(\script{c})/\script{x}]$ satisfies $\metav{Q}$

\item Goal: show that there exists a variable assignment $d'_I$ that satisfies $\qt{\forall}{\script{x}}\metav{Q}\substitute{\script{c}}{\script{x}}$
\bi

\item i.e. a $d'_I$ s.t. $d'_I [r/\script{x}]$ satisfies $\metav{Q}\substitute{\script{c}}{\script{x}}$ for each object $r \in D$
\item We will actually show that the given $d_I$ does the trick!
%i hope this is what i'm actually showing...
\ei

\end{itemize}
\end{frame}

\begin{frame}
\frametitle{Lemma for Case 15 continued}
%\large

\begin{itemize}[<+->]

\item Notice that for $\qt{\forall}{\script{x}}\metav{Q}\substitute{\script{c}}{\script{x}}$ to be a wff, $\script{x}$ must not already occur in sentence $\metav{Q}$, so $\metav{Q}$ can't already have a $\script{x}$-quantifier. 

\item So $ \script{x}$ occurs freely in $\metav{Q}\substitute{\script{c}}{\script{x}}$ as the \textbf{only} free variable \\ (since $\qt{\forall}{\script{x}}\metav{Q}\substitute{\script{c}}{\script{x}}$ is, by assumption, a sentence)

\item Hence, a variable assignment $d_I$ of free variables in $\metav{Q}\substitute{\script{c}}{\script{x}}$ to objects in $D$ amounts to a choice of object $r \in D$ to assign $\script{x}$

\item So $d_I$ must make some choice $r := I(c)$ of object to assign $\script{x}$
% %JH: am i cheating by assuming I(c)=r? or am i allowed to do this? or am I just defining r as the denotation of c? r could of course be denoted by other constants as well though...
%i think this is allowed b/c I'm saying 

\item Hence $d_I$ simply equals $d_I[r/ \script{x}]$. %\\ Assume also that $I(c) = r$

%Given this choice for $r$, $d_I$ simply equals $d_I[r/ \script{x}]$. Assume also that $I(c) = r$ %i.e. they agree everywhere, even with respect to assigning r for x, since d_I has to assign x to something and the variant makes the same choice 

\item By 11.1.1, $d_I$ satisfies $\metav{Q}\substitute{\script{c}}{\script{x}}$ iff $d_I[I(\script{c})/\script{x}]$ satisfies $\metav{Q}$

\item So $d_I$ and hence $d_I[r/ \script{x}]$ satisfies $\metav{Q}\substitute{\script{c}}{\script{x}}$

%\item By defN of true-in-QL, $d_I$ must satisfy $\metav{Q}$, so $d_I [r/x]$ satisfies $\metav{Q}\substitute{\script{c}}{\script{x}}$ in $\mathfrak{M}$

\end{itemize}
\end{frame}


\begin{frame}
\frametitle{Lemma for Case 15 continued \textit{more}}
%\large

\begin{itemize}[<+->]

\item For each $r \in D$, we define a new model $\mathfrak{M}_r$ whose interpretation function $I_r$ is just like $I$ except that it assigns the constant $\script{c}$ to $r$ 

\item Now, $\mathfrak{M} \entails \Gamma$ and $\mathfrak{M}_r$ differs by $\mathfrak{M}$ only in how it interprets a symbol that does not occur in $\Gamma$. 

\item So by Locality, we have $\mathfrak{M}_r \entails \Gamma$ and hence $\mathfrak{M}_r \entails \metav{Q}$

\item So for each object $r \in D$, we have a $d_{I_r} [r/\script{x}]$ that satisfies $\metav{Q}\substitute{\script{c}}{\script{x}}$. 

%\item Define variable assignment $d'_I$ as  

\item This is equivalent to saying that for each $r \in D$, $d_I [r/\script{x}]$ satisfies $\metav{Q}\substitute{\script{c}}{\script{x}}$, since $d_I$ and each $d_{I_r}$ agree on what to assign every other variable besides possibly $\script{x}$

\item And by defN, this means that $d_I$ satisfies $\qt{\forall}{\script{x}}\metav{Q}\substitute{\script{c}}{\script{x}}$, and hence this sentence is true in $\mathfrak{M}$

%\item So for each $r \in D$ there is a variable assignment 



%\item So, we know there is some variable assignment $d_I$ that satisfies $\metav{Q}$. We want to use this to show there is some assignment that satisfies $\qt{\forall}{\script{x}}\metav{Q}\substitute{\script{c}}{\script{x}}$, which requires it to satisfy Q for each r standing in for x. 



%\item Then by defN of true-in-QL, there is some variable assignment $d_I$ that satisfies $\metav{Q}[c]$ in $\mathfrak{M}$. Let $r$ be the object in $D$ that $c$ stands for.




\end{itemize}
\end{frame}








\iffalse 

\begin{frame}
\frametitle{Case 1: Premise or Assumption}
%\large

\begin{itemize}[<+->]

\item \emph{Case 1}: $\metav{P}_{k +1}$ is a premise (:$PR$) or a subproof assumption (:$AS$). \\ Show that $\Gamma_{k+1} \entails \metav{P}_{k+1}$

\item Either way, $\metav{P}_{k +1} \in \Gamma_{k+1}$ (since every premise and assumption lies within its own scope)

\item So given a TVA that makes every sentence in $\Gamma_{k+1}$ true, \\ this TVA must make $\metav{P}_{k +1} $ true 

\item So $\Gamma_{k+1} \entails \metav{P}_{k+1}$; so this case be righteous! 

\end{itemize}
\end{frame}

\begin{frame}
\frametitle{Case 2: Reiteration}
%\large

\begin{itemize}[<+->]

\item \emph{Case 2}: $\metav{P}_{k +1}$ arises from an application of rule $R$, reiteration

\item Then wff $\metav{P}_{k +1}$ appears on an earlier line \#i as the wff $\metav{P}_{i}$

\item By the induction hypothesis, line \#i is righteous, so $\Gamma_{i} \entails \metav{P}_{i}$. 

\item[] --Hence, we also have $\Gamma_{i} \entails \metav{P}_{k+1}$ (since $\metav{P}_{i} = \metav{P}_{k+1}$)

\item To apply rule $R$, $\metav{P}_{k +1}$ must lie to the right of   line \#i's rightmost scope line $\Rightarrow$ $\Gamma_{i} \subseteq \Gamma_{k+1}$ (i.e., all of the premises/assumptions accessible at line \#i must also be accessible at line \#k+1). 

\item Since $\Gamma_{i} \entails \metav{P}_{k+1}$ and $\Gamma_{i} \subseteq \Gamma_{k+1}$, we have  $\Gamma_{k+1} \entails \metav{P}_{k+1}$

\item Draw a schematic derivation to better understand $\Gamma_{i} \subseteq \Gamma_{k+1}$!

%the scope line condition!

%the PR/AS accessible at line 

\end{itemize}
\end{frame}

\begin{frame}
\frametitle{Case 3: Conjunction Introduction \footnotesize{(Things be heating up---finally!)}}
%\large

\begin{itemize}[<+->]

\item \emph{Case 3}: $\metav{P}_{k +1} := (\metav{Q} \eand \metav{R})$ arises from an application of rule $\eand$I

%\makebox[\textwidth]{\footnotesize{Things be heating up (finally!)}}

\item Then on two earlier lines \#h and \#j, \metav{Q} and \metav{R} appear, respectively

\item By the IH, both of these lines are righteous, so $\Gamma_{h} \entails \metav{Q}$ and $\Gamma_{j} \entails \metav{R}$

\item By rule $\eand$I, both these lines must be accessible on line \#k+1

\item So $\Gamma_{h} \cup \Gamma_{j} \subseteq \Gamma_{k+1}$ (i.e. both $\Gamma_{h}$ and $\Gamma_{j}$ are subsets of $\Gamma_{k+1}$)

\item Hence, any TVA that satisfies $\Gamma_{k+1}$ must satisfy both  $\Gamma_{h}$ and $\Gamma_{j}$, and hence satisfy \metav{Q} and also satisfy \metav{R}

\item Thus, any TVA that satisfies $\Gamma_{k+1}$ satisfies $(\metav{Q} \eand \metav{R})$

\item So $\Gamma_{k+1} \entails \metav{P}_{k+1}$



\end{itemize}
\end{frame}

\begin{frame}
\frametitle{Case 4: Conjunction Elimination}
%\large

\begin{itemize}[<+->]

\item \emph{Case 4}: $\metav{P}_{k +1}$ arises from an application of rule $\eand${}E

\makebox[\textwidth]{\footnotesize{I'm about to eliminate this proof, son!}} 

\item Then there is an earlier line \#h of the form $\metav{P}_{k +1} \eand \metav{Q} $ or $\metav{Q} \eand \metav{P}_{k +1}$

\item By the IH, line \#h is righteous, so $\Gamma_{h} \entails \metav{P}_{h}$

\item Since line \#h is accessible at line \#k+1, $\Gamma_{h} \subseteq \Gamma_{k+1}$

\item So any TVA that satisfies $\Gamma_{k+1}$ also satisfies $\Gamma_{h}$ and thereby makes true $\metav{P}_{h}$

\item By the truth conditions for conjunctions, any TVA that satisfies $\metav{P}_{h}$ satisfies both conjuncts, in particular $\metav{P}_{k +1}$

\item So $\Gamma_{k+1} \entails \metav{P}_{k+1}$ and line \#k+1 is righteous

\end{itemize}
\end{frame}

\begin{frame}
\frametitle{Case 8: Conditional Introduction}
%\large

\begin{itemize}[<+->]

\item \emph{Case 8}: $\metav{P}_{k +1}$ arises from rule $\eif${}I, which involves a subproof! 

\item $\metav{P}_{k +1}$ must be of the form $\metav{Q} \eif \metav{R}$ (\textbf{draw derivation} to define terms)

\item NTS: $\Gamma_{k+1} \entails \metav{Q} \eif \metav{R}$ given that $\Gamma_h \entails \metav{Q}$ and \emph{$\Gamma_j \entails \metav{R}$}, by Ind. Hyp.

\item Proceed by cases: either $\Gamma_{k+1}$ satisfies \metav{Q} or it doesn't:
%really just relying on this latter entailment! 

\item If $\Gamma_{k+1}$ does not satisfy \metav{Q}, then it trivially satisfies $\metav{Q} \eif \metav{R}$

\item Otherwise, $\Gamma_{k+1}$ satisfies \metav{Q}. Since $\Gamma_j \subseteq \Gamma_{k+1} \cup \{ \metav{Q} \}$, this means that $\Gamma_j$ is satisfied in this case. Then since line \#j is righteous, we have $\Gamma_{k+1} \cup \{ \metav{Q} \} \entails \metav{R}$. So in this case, $\Gamma_{k+1}$ satisfies $\metav{Q} \eif \metav{R}$ as well. 

\item So in either case, $\Gamma_{k+1} \entails \metav{P}_{k+1}$

%So in this case, we have $\Gamma_{k+1} \entails 

%$\Gamma_j \entails \metav{R}$. And 

\end{itemize}
\end{frame}

\begin{frame}
\frametitle{Case 9: Negation Introduction}
%\large

\begin{itemize}[<+->]

\item \emph{Case 8}: $\metav{P}_{k +1}$ arises from rule $\enot${}I, using a subproof! 

\item $\metav{P}_{k +1}$ must be of form $\enot \metav{Q}$; \textbf{draw derivation to define lines}

\item NTS: $\Gamma_{k+1} \entails \enot \metav{Q}$ given that $\Gamma_h \entails \metav{Q}$, $\Gamma_j \entails \metav{R}$ \emph{and} $\Gamma_m \entails \enot \metav{R}$ (by IH)

\item Notice that $\Gamma_j$ and $\Gamma_m$ are both subsets of $\Gamma_{k+1} \cup \{ \metav{Q}\}$ 

\item[] Hence, $\Gamma_{k+1} \cup \{ \metav{Q}\}$ entails both $\metav{R}$ and $\enot \metav{R}$ as well. 

\item[] Thus, any TVA that satisfies $\Gamma_{k+1} \cup \{ \metav{Q}\}$ must make both $\metav{R}$ and $\enot \metav{R}$ true, which is impossible (i.e. there can be no such TVA). 
\item[] $\Rightarrow$ $\Gamma_{k+1} \cup \{ \metav{Q}\}$ is unsatisfiable. Hence, $\Gamma_{k+1} \entails \enot \metav{Q}$

%i.e. whenever a TVA makes every wff in $\Gamma_{k+1}$ true, it must make Q false

\end{itemize}
\end{frame}

\fi 

\iffalse %*************************************************************************

\subsection{Completeness of System QND}

\subsubsection{Completing our terminology}

\begin{frame}
\frametitle{Semantic vs. Syntactic Consistency}
%\large

\begin{itemize}[<+->]

\item We will appeal to two distinct notions of consistency throughout

\item One is \emph{semantic}: this is the notion we are already familiar with:

\item[] there is a TVA that \emph{satisfies} every sentence in the set

\item We introduce a new \textbf{syntactic} notion of consistency relative to our SND derivation system: 

\item[] -- a set of SL wffs is \textbf{SND-consistent} provided that you can't derive contradictory sentences from it in SND

\item Core proof idea: we'll show that if a set of sentences is \textbf{consistent-in-SND}, then it is also semantically consistent (i.e. \emph{satisfiable}). So by the contrapositive: if a set is \textbf{\textcolor{OGlyallpink}{un}}satisfiable, then it is \textbf{\textcolor{OGlyallpink}{in}}consistent-in-SND. 

\end{itemize}
\end{frame}


\begin{frame}
\frametitle{Semantic: Satisfiable (truth-functionally consistent)}
%\large

\begin{itemize}[<+->]

\item Recall: a set of SL sentences is \emph{satisfiable} provided there is a TVA that makes all of them true

%\begin{itemize}

\item This is a \textit{semantic} notion of consistency

\item i.e. \emph{truth-functionally consistent} 
%TF-consistent, (jointly) \emph{satisfiable}  

%\end{itemize}

\item Contrast this with the syntactic notion of \textbf{consistency in SND}:

\end{itemize}
\end{frame}




\begin{frame}
\frametitle{Syntactic: (In)consistent-in-SND (derivationally consistent)}
%\large

\begin{itemize}[<+->]

\item Let $\Gamma$ be a (possibly infinite) set of SL wffs 

\item \textbf{\textcolor{OGlyallpink}{Inconsistent-in-SND}}: from premises in $\Gamma$, we can derive contradictory formulas $R$ and $\enot R$ in the scope of the main scope line (i.e. these premises)

\item \emph{Consistent-in-SND}: $\Gamma$ is not SND-inconsistent, i.e. there is no derivation from premises in $\Gamma$ resulting in contradictory formulas within the main scope

\item Other words we might use for these concepts: SND-inconsistent, derivationally-inconsistent, SND-consistent, etc.

\item Just remember: this syntactic notion has nothing to do with truth value assignments!

\end{itemize}
\end{frame}



\subsubsection{Proof Sketch}

\begin{frame}
\frametitle{Proof Sketch}
%\large

\begin{itemize}[<+->]

\item Goal: prove the completeness of SL: for every SL wff $\metav{P}$ and every set $\Gamma$ of SL sentences, if $\Gamma \entails \metav{P}$ then $\Gamma \vdash \metav{P}$

\item So assume that $\Gamma \entails \metav{P}$. 

\item Recall from week 5: this means that $\Gamma \cup \{\enot \metav{P}\}$ is semantically inconsistent (i.e. \textbf{\textcolor{OGlyallpink}{unsatisfiable}}): \\ no TVA satisfies the premises and negated conclusion  

\item We now appeal to a \emph{Consistency lemma} that is the heart of the enterprise: any SND-consistent set of SL sentences is satisfiable (i.e. semantically consistent)

\end{itemize}
\end{frame}

\begin{frame}
\frametitle{Proof Sketch: Using the consistency lemma}
%\large

\begin{itemize}[<+->]

\item \emph{Consistency lemma}: any SND-consistent set of SL sentences is satisfiable

\item \textbf{\textcolor{OGlyallpink}{Contrapositive}} of CL: any set of SL sentences that is \textcolor{OGlyallpink}{Un}satisfiable is SND-\textcolor{OGlyallpink}{In}consistent

\item From  $\Gamma \entails \metav{P}$ we know that $\Gamma \cup \{\enot \metav{P}\}$ is unsatisfiable

\item So by the contrapositive of CL, we see that $\Gamma \cup \{\enot \metav{P}\}$ is SND-inconsistent

\item This means that we can derive a pair of contradictory sentences $R$ and $\enot R$ from $\Gamma \cup \{\enot \metav{P}\}$! So using the power of negation elimination, we can derive $\metav{P}$ from $\Gamma$, i.e. $\Gamma \vdash \metav{P}$. So we are `done'! 

\end{itemize}
\end{frame}

\begin{frame}
\frametitle{Negation Elimination Refresher (book's claim 6.4.4)}
%\large

\begin{itemize}[<+->]

\item Claim: if $\Gamma \cup \{\enot \metav{P}\}$ is \textbf{\textcolor{OGlyallpink}{SND-inconsistent}}, then $\Gamma \vdash \metav{P}$

\item Proof: starting with (finitely-many) premises $\Delta$ from $\Gamma$, introduce $\enot \metav{P}$ as a subproof assumption for negation elimination

\item Since $\Gamma \cup \{\enot \metav{P}\}$ is SND-inconsistent, we can derive a contradictory pair $R$ and $\enot R$ within the scope of wffs in $\Delta \cup \{\enot \metav{P}\}$

\item Then discharge this assumption $\enot \metav{P}$ by negation elimination, writing $\metav{P}$, now in the scope of $\Delta$. So $\Delta \vdash \metav{P}$

\item Since $\Delta \subseteq \Gamma$, we have $\Gamma \vdash \metav{P}$

%\item Recall that from a contradictory pair, we can derive anything! 



\end{itemize}
\end{frame}

\subsubsection{The completely straightforward part}

\begin{frame}
\frametitle{Core subgoal: Prove consistency lemma (book's 6.4.2)}
%\large

\begin{itemize}[<+->]

\item So all we have to do is prove the \emph{consistency lemma}: any SND-consistent set of SL sentences is satisfiable

\item We'll prove this lemma in three `stages':

\item The first two are straightforward: given an SND-consistent set $\Gamma$, we construct a \textbf{\textcolor{blue}{superset $\Gamma^{\ast}$}} that is \textit{\textcolor{blue}{maximally} SND-consistent}

\item In the third stage, we show that any maximally SND-consistent set is \alert{satisfiable}: we use maximal consistency to construct a TVA that satisfies every sentence in $\Gamma^{\ast}$

\item Since by construction $\Gamma \subseteq \Gamma^{\ast}$, this TVA satisfies $\Gamma$ as well. 

\item \footnotesize{(The idea in the third stage is similar to what we did with trees: use a syntactic consistency property to construct a TVA that satisfies a set of wffs: with trees we had `\textcolor{blue}{complete} \alert{open} branches'; here we have \textcolor{blue}{maximal}-\alert{SND-consistency})} 
\item The third stage comprises a tedious lemma and induction! \\ PS12 problems 2 and 3 provide practice with this tedium! 

%this is just like constructing a TVA that satisfies all of the sentences in 


\end{itemize}
\end{frame}

\begin{frame}
\frametitle{Maximally SND-consistent}
%\large

\begin{itemize}[<+->]

\item A set $\Gamma^{\ast}$ of SL wffs is \emph{maximally SND-consistent} provided that:

\begin{enumerate}[1.)]

\item $\Gamma^{\ast}$ is SND-consistent (i.e. can't derive contradictory sentences)

\item adding \textbf{any} additional wff to $\Gamma^{\ast}$ would result in an SND-\textcolor{OGlyallpink}{inconsistent} set

\end{enumerate} 

\item i.e. for any $P \notin \Gamma^{\ast}$, $\{P\} \cup \Gamma^{\ast}$ is SND-\textcolor{OGlyallpink}{inconsistent}

\item Motivation: it is straightforward (but tedious) to show that a maximally SND-consistent set is semantically consistent
% % the idea here is very similar to what we did in the completeness proof for the tree system: we rely on an `complete open' derivation (i.e. one w/ no contradictions in main scope) to construct a TVA that satisfies every sentence in the appropriate scope of the starting premises

\item[] -- Moreover, every SND-consistent set is a subset of a maximally SND-consistent set. \item[] -- So we piggyback on an appropriate $\Gamma^{\ast}$ to show that any SND-consistent set $\Gamma$ is also \alert{satisfiable} %semantically consistent

\end{itemize}
\end{frame}

\begin{frame}
\frametitle{Stage 1: Constructing $\Gamma^{\ast}$}
%\large

\begin{itemize}[<+->]

\item Let $\Gamma$ be an SND-consistent set of SL wffs (possibly infinite)

\item To construct $\Gamma^{\ast}$, we first \emph{enumerate} the SL wffs, so that every SL wff is associated with a unique positive integer $\{1, 2, 3, \dots \}$

\item Then consider the first wff `$A$' in our enumeration. \\ If $A$ can be added to $\Gamma$ without the resulting set being SND-inconsistent, then let  $\Gamma_1 := \Gamma \cup \{A\}$. 

\item Otherwise, let  $\Gamma_1 := \Gamma$ (so that $\Gamma_1$ stays SND-consistent)

% \item In general, if $P_k$ is the $k$-th sentence in our enumeration, then $\Gamma_{k+1}$ is $\Gamma_k \cup \{P_k\}$ provided $\Gamma_k \cup \{P_k\}$ is SND-consistent; \\ otherwise, $\Gamma_{k+1}$ equals $\Gamma_k$

\item Then, proceed to the second wff in our enumeration. \\ If it can be added to $\Gamma_1$ without the new set being SND-inconsistent, let $\Gamma_2$ be the result. Otherwise, let $\Gamma_2 := \Gamma_1$

\item $\Gamma^{\ast}$ is the result of `doing' this procedure for every SL wff

\item More precisely, $\Gamma^{\ast} := \bigcup_{k=1}^{\infty} \Gamma_k$


\end{itemize}
\end{frame}

\begin{frame}
\frametitle{Enumeration (lexical ordering)}
%\large

\begin{itemize}[<+->]

\item Analogy: we can enumerate words by length, using their alphabetical order to break ties %(so that 5-letter words beginning w/ `a' come before those w/ `b')

\item Can do the same for SL wffs by stipulating an `alphabetical order':

\item $\enot, \eor, \eand, \eif, \eiff, (, ), 0, 1, \dots, 9, A, B, \dots, Z$

\item Each symbol is assigned an \textbf{index} between `10' and `55' %so 43 indices total: 55-10+1-3, since we actually skip 17, 18, and 19, starting 0 at `20' so that A starts at `30'

\item Then each SL wff corresponds to a unique positive integer, constructed by replacing each symbol in the wff with its index, from left to right. 

\item So with our ordering, `$A$' is the first wff; `$B$' the second \dots up to $Z$, and then we hit $\enot A$ ($\mapsto 1030$), then $\enot B$ ($\mapsto 1031$), etc. 

\end{itemize}
\end{frame}

\begin{frame}
\frametitle{Stage 2: $\Gamma^{\ast}$ is maximally SND-consistent}
%\large

\begin{itemize}[<+->]

\item This requires proving two claims (from definition of M-SND-C):

\bigskip

\begin{enumerate}[1.)]

\item $\Gamma^{\ast}$ is consistent in SND

\item Adding any additional wff to $\Gamma^{\ast}$ would result in an \textbf{\textcolor{OGlyallpink}{SND-inconsistent}} set

\end{enumerate}

\bigskip

\item We prove these in turn

\end{itemize}
\end{frame}

\begin{frame}
\frametitle{Stage 2 (i): $\Gamma^{\ast}$ is SND-consistent}
%\large

\begin{itemize}[<+->]

\item Assume for \textit{reductio} that $\Gamma^{\ast}$ is inconsistent in SND

\item Then there would be an SND derivation with finite premise set $\Delta \subset \Gamma^{\ast}$ that derives a contradictory pair $R$ and $\enot R$

\item Since $\Delta$ is finite, there exists some $k \in \mathbb{N}$ s.t. $\Delta \subset \Gamma_k$. \\ So then this $\Gamma_k$ would be \textcolor{OGlyallpink}{SND-inconsistent}. 

\item Yet, we constructed each $\Gamma_k$ such that each is \alert{SND-consistent}: 

\bi

\item In general, if $P_k$ is the $k$-th sentence in our enumeration, then $\Gamma_{k+1}$ is $\Gamma_k \cup \{P_k\}$ provided that $\Gamma_k \cup \{P_k\}$ is SND-consistent; \\ otherwise, $\Gamma_{k+1}$ equals $\Gamma_k$ (so SND-consistent either way)

\ei

\item Hence, $\Gamma^{\ast}$ must be SND-consistent, on pain of \textit{reductio} 

\item \footnotesize{(note: the book's proof, p. 256, is way more complicated than necessary\dots)}
% does a lot of work to show this that seems unnecessary)

\end{itemize}
\end{frame}


\begin{frame}
\frametitle{Stage 2 (ii): $\Gamma^{\ast}$ is \textcolor{blue}{maximally} SND-consistent}
%\large

\begin{itemize}[<+->]

\item Assume for \textit{reductio} that $\Gamma^{\ast}$ weren't maximally SND-consistent, despite being SND-consistent

\item i.e. assume \textit{it is \textcolor{red}{not the case that}} for all additional wff, adding it to $\Gamma^{\ast}$ would result in an \textcolor{OGlyallpink}{SND-inconsistent} set

% % Relevant claim: for any additional wff not already in $\Gamma^{\ast}$, adding Q to $\Gamma^{\ast}$ results in an SND-inconsistent set. So we negate this: there exists some Q such that when added to $\Gamma^{\ast}$, $\Gamma^{\ast}$  remains consistent. 

\item[] $\Rightarrow$ there exists a wff $\metav{Q}$ that we could add to $\Gamma^{\ast}$ while preserving \alert{SND-consistency} (i.e. there would be some wff that we neglected that could make  $\Gamma^{\ast}$ an even `bigger' SND-consistent set)
%relevant notion of `size' here is given by subset relation, rather than cardinality

\item Yet, $\metav{Q}$ would appear in our enumeration as some wff $P_k$, `considered' at the $k$-th stage of our construction of $\Gamma^{\ast}$.

\item So if $\metav{Q}$ isn't in $\Gamma^{\ast}$, then this is because adding it `would have' made $\Gamma_k \subset \Gamma^{\ast}$ \textcolor{OGlyallpink}{SND-inconsistent}. \\ So $\{\metav{Q}\} \cup \Gamma^{\ast}$ must be SND-inconsistent (\textit{reductio}!)

%So $\{\metav{Q}\} \cup \Gamma_k$ and hence $\{\metav{Q}\} \cup \Gamma^{\ast}$ must be SND-inconsistent (\textit{reductio}!)




%So adding $\metav{Q}$ would result in $\Gamma^{\ast}$ being SND-inconsistent

\item So we can't add any $\metav{Q}$ to $\Gamma^{\ast}$ while preserving SND-consistency 

%So there can't be a wff $\metav{Q}$ that we could add to $\Gamma^{\ast}$ while preserving SND-consistency 

\end{itemize}
\end{frame}

\subsubsection{Stage 3: The completely tedious part}

\begin{frame}
\frametitle{Stage 3: The Maximal Consistency Lemma (book's 6.4.8)}
%\large

\begin{itemize}[<+->]

\item \textbf{\textcolor{blue}{Maximal} \alert{Consistency Lemma}}: any set that is maximally-SND-consistent is satisfiable

\item So there exists a TVA that satisfies every sentence in $\Gamma^{\ast}$. \\ We construct this TVA, calling it ``$\metav{I}$" (the book calls it $\textbf{A}^{\ast}$)
%$\mathbf{A}^{\ast}$

\item Proof idea: since $\Gamma^{\ast}$ is M-SND-C, for any wff $\metav{Q}$, either $\metav{Q} \in \Gamma^{\ast}$ or $\textcolor{red}{\enot \metav{Q}} \in \Gamma^{\ast}$ (you're either in the club or your `\textcolor{red}{nemesis}' is!)
%you're out of the club!)

\item[] This holds in particular for each atomic wff

\item Define the TVA $\metav{I}$ such that $\metav{I}(B) = True$ iff atomic $B \in \Gamma^{\ast}$

\item Then by the recursive structure of SL wffs, $\metav{I}(\metav{Q}) = True$ iff $\metav{Q} \in \Gamma^{\ast}$

\end{itemize}
\end{frame}

\begin{frame}
\frametitle{Stage 3 (i): the Membership Lemma (book's 6.4.11)}
%\large

\begin{itemize}[<+->]

\item To induct on SL, we first show some constraints on $\Gamma^{\ast}$ membership

\item Basically, $\Gamma^{\ast}$ is like a club with a bouncer who enforces maximal consistency. Before the bouncer lets a wff into $\Gamma^{\ast}$, he checks who else is in the club 
%(the smaller fish, relative to our lexical ordering)

\item \emph{Membership Lemma} for club $\Gamma^{\ast}$: if \metav{P} and \metav{Q} are SL wffs, then:

\begin{enumerate}[a.)]

\item $\enot \metav{P} \in \Gamma^{\ast}$ if and only if $\metav{P} \notin \Gamma^{\ast}$

\item $\metav{P} \eand \metav{Q} \in \Gamma^{\ast}$ if and only if both $\metav{P}\in \Gamma^{\ast}$ and $\metav{Q}\in \Gamma^{\ast}$

\item $\metav{P} \eor \metav{Q} \in \Gamma^{\ast}$ if and only if either $\metav{P}\in \Gamma^{\ast}$ or $\metav{Q}\in \Gamma^{\ast}$

\item $\metav{P} \eif \metav{Q} \in \Gamma^{\ast}$ if and only if either $\metav{P}\notin \Gamma^{\ast}$ or $\metav{Q}\in \Gamma^{\ast}$

\item $\metav{P} \eiff \metav{Q} \in \Gamma^{\ast}$ iff either (i) $\metav{P}\in \Gamma^{\ast}$ and $\metav{Q}\in \Gamma^{\ast}$ or (ii) $\metav{P}\notin \Gamma^{\ast}$ and $\metav{Q}\notin \Gamma^{\ast}$

\end{enumerate}

\item Notice how these syntactic constraints mirror truth-conditions!

\item \footnotesize{Moral: We all want to belong, but sometimes our enemies get in the way!}

%people get in the way!}


\end{itemize}
\end{frame}


\begin{frame}
\frametitle{Stage 3 (i): Key Fact aka \emph{The Door} lemma (book's 6.4.9)}
%\large

\begin{itemize}[<+->]

\item To prove the membership lemma's cases (a)--(e), we'll use another lemma (hint: it's lemmas all the way down):
%we'll use a lemma for a lemma

\item \emph{The Door}: if $\Gamma \vdash P$, and $\Gamma^{\ast}$ is a maximally SND-consistent superset of $\Gamma$, then $P \in \Gamma^{\ast}$ \\ (mnemonic: ``$\Gamma\vdash P$" pushes $P$ through the door!) %of our fictional club!
%Bouncer says yesssss)

\item Proof: first, assume that $\Gamma \vdash P$ (we'll use this fact below)

\item Next, assume for \textit{reductio} that $P \notin \Gamma^{\ast}$. Then since $\Gamma^{\ast}$ is maximally SND-consistent, $\Gamma^{\ast} \cup \{ P\}$ must be \textcolor{OGlyallpink}{inconsistent in SND}. 

\item Hence, by negation introduction, $\Gamma^{\ast} \vdash \enot P$

\item By assumption, $\Gamma \vdash P$, so also $\Gamma^{\ast} \vdash P$, since $\Gamma \subseteq \Gamma^{\ast}$

\item So $\Gamma^{\ast}$ derives both $P$ and $\enot P$. \textit{Reductio}! (since $\Gamma^{\ast}$ is M-SND-C)

\item Hence, if $\Gamma \vdash P$ and $\Gamma \subseteq \Gamma^{\ast}$, then $P$ must belong to $\Gamma^{\ast}$ 

\end{itemize}
\end{frame}




\begin{frame}
\frametitle{Membership Lemma: Case (a)}
%\large

\begin{itemize}[<+->]


\item \emph{Case (a)}: $\enot \metav{P} \in \Gamma^{\ast}$ if and only if $\metav{P} \notin \Gamma^{\ast}$

\item Two directions to prove:

\item[] $\Rightarrow$: Assume $\enot \metav{P} \in \Gamma^{\ast}$. Then if  $\metav{P}$ were in $\Gamma^{\ast}$, we could derive contradictory sentences. 

\item[] So since $\Gamma^{\ast}$ is SND-consistent, we must have $\metav{P} \notin \Gamma^{\ast}$

\item[] $\Leftarrow$: Assume $\metav{P} \notin \Gamma^{\ast}$. Then adding $\metav{P}$ to $\Gamma^{\ast}$ results in an SND-inconsistent set. Hence, there is some finite subset $\Delta \subset \Gamma^{\ast}$ s.t. $\Delta \cup \{ \metav{P}\}$ is SND-inconsistent (i.e. derives contradictory sentence pair). 

\item So by negation introduction, $\Delta \vdash \enot \metav{P}$

\item So by The Door lemma, $\enot \metav{P} \in \Gamma^{\ast}$


\end{itemize}
\end{frame}

\begin{frame}
\frametitle{Membership Lemma: Cases (b)-(e)}
%\large

\begin{itemize}[<+->]

\item See the book for cases (b) ($\metav{P} \eand \metav{Q}$) and (d) ($\metav{P} \eif \metav{Q}$)

\item Case (c) is PS12 \#2: $\metav{P} \eor \metav{Q} \in \Gamma^{\ast}$ if and only if either $\metav{P}\in \Gamma^{\ast}$ or $\metav{Q}\in \Gamma^{\ast}$

\item We skip case (e) ($\metav{P} \eiff \metav{Q}$) because \dots \pause \emph{YOLO}



\end{itemize}
\end{frame}

\begin{frame}
\frametitle{Stage 3 (ii): Induction on SL (i.e. we be clubbin')}
%\large

\begin{itemize}[<+->]

\item Goal: construct a TVA \metav{I} that satisfies the M-SND-C set $\Gamma^{\ast}$

\item[] Suffices to construct \metav{I} s.t. $\metav{I}(\metav{Q}) = True$ iff $\metav{Q} \in \Gamma^{\ast}$, $\forall  \metav{Q} \in$ SL. \\ Say that a wff is ``\emph{clubbin'} " whenever it meets this property

\item Define $\metav{I}$ such that $\metav{I}(B) = True$ iff atomic $B \in \Gamma^{\ast}$

\item \emph{Base case}: each atomic wff is true on $\metav{I}$ iff it belongs to $\Gamma^{\ast}$ \\ (i.e. the atomics be clubbin')

\item (Strong) \emph{Induction hypothesis}: assume every SL wff with $1$ to $k$-many connectives is clubbin' 

\item Induction step: show that an arbitrary SL wff with k+1-many connectives is clubbin' 

%, i.e. a wff is true on $\metav{I}$ iff it belongs to $\Gamma^{\ast}$

\end{itemize}
\end{frame}

\begin{frame}
\frametitle{Base Case}
%\large

\begin{itemize}[<+->]

\item Need to show \emph{TWO} directions!: 

\item \emph{Base case}: each atomic wff is true on $\metav{I}$ \emph{iff} it belongs to $\Gamma^{\ast}$

\item Recall that we defined $\metav{I}$ such that $\metav{I}(B) = True$ \emph{iff} atomic $B \in \Gamma^{\ast}$

\item So both directions are met by construction 

\item We proceed to do induction using our SL induction schema: \\ an arbitrary sentence \metav{P} with k+1-many connectives has one of five forms, coming from our five connectives. 

\end{itemize}
\end{frame}

\begin{frame}
\frametitle{Induction on SL: Case 1}
%\large

\begin{itemize}[<+->]

\item \emph{Case 1}: \metav{P} has the form $\enot \metav{Q}$, where since $\metav{Q}$ has $k$-connectives, it is clubbin by the IH (i.e. $\metav{I}(\metav{Q}) = 1$ if and only if $\metav{Q} \in \Gamma^{\ast}$)

%it is clubbin

\item NTS: (i) (the $\Rightarrow$direction) if $\metav{I}(\metav{P}) = True$ then $\metav{P} \in \Gamma^{\ast}$ and \\ 
(ii) (the $\Leftarrow$direction) if $\metav{P} \in \Gamma^{\ast}$, then $\metav{I}(\metav{P}) = True$
\item[] (\textit{Alternative (ii)}: show contrapositive: if $\metav{I}(\metav{P}) = 0$, then $\metav{P} \notin \Gamma^{\ast}$)

\item[$\Rightarrow$] if $\metav{I}(\metav{P}) = 1$, then $\metav{I}(\metav{Q}) = 0$. Since \metav{Q} is clubbin', we have $\metav{Q} \notin \Gamma^{\ast}$. 
\item[] By Membership lemma (a), $\textcolor{red}{\enot \metav{Q}} \in \Gamma^{\ast}$, so $\metav{P} \in \Gamma^{\ast}$

%\item[] Since $\metav{Q}$ has $k$-connectives, by the IH it is clubbin. So 

\item[$\Leftarrow$] if $\metav{P} \in \Gamma^{\ast}$, then $\enot \metav{Q} \in \Gamma^{\ast}$. So by Membership lemma (a), $\metav{Q} \notin \Gamma^{\ast}$. 
\item[] Since \metav{Q} is clubbin', we have $\metav{I}(\metav{Q}) = 0$. \item[] So by the truth conditions for negation, $\metav{I}(\metav{P}) = 1$


\end{itemize}
\end{frame}

\begin{frame}
\frametitle{Induction on SL: Cases 2--5}
%\large

\begin{itemize}[<+->]

\item Need to show: \metav{P} be clubbin', i.e. $\metav{I}(\metav{P}) = True$ iff $\metav{P} \in \Gamma^{\ast}$, \\ where \metav{P} is arbitrary SL wff with k+1-many connectives

\item \alert{Induction hypothesis}: assume every SL wff with $1$ to $k$-many connectives is clubbin' 

\item \emph{Case 2}: \metav{P} has the form $\metav{Q} \eand \metav{R}$

%\item See the book for cases (b) ($\metav{P} \eand \metav{Q}$) and (d) ($\metav{P} \eif \metav{Q}$)

\item \emph{Case 3} is PS12 \#3: \metav{P} has the form $\metav{Q} \eor \metav{R}$

\item Case 4: \metav{P} has the form $\metav{Q} \eif \metav{R}$ (see book p.260!)

\item Case 5: \metav{P} has the form $\metav{Q} \eiff \metav{R}$ (we'll do this case if and only if we accomplish all other goals in our lives)
% (whereof we cannot speak, we must be silent)

%\item We skip case (e) ($\metav{P} \eiff \metav{Q}$) because \dots \pause \emph{YOLO}


\end{itemize}
\end{frame}


\begin{frame}
\frametitle{Reminder for Josh!}
%\large

\begin{itemize}[<+->]

\item If we actually make it this far, give hints on PS12 completeness question ($P \eor Q$)! or do Case (d), which is most analogous 

\item If the people don't want these hints, then clearly they're already complete!
% rather be complete!
%hence, then let's move on to completenes

\item ``The customer is always right!"

\item (Schematize this sentence in quantifier logic)
%The Corporation says: 


\end{itemize}
\end{frame}


\fi %*****************************************************************************


\iffalse 

\begin{frame}
\frametitle{Recall SND:}

  \begin{itemize}[<+->]
    \item d
    \emph{d} ($\enot$, $\eor$, $\eand$, $\eif$, $\eiff$)
  
  \begin{block}{blah}
    \begin{itemize}[<+->]
      \item[] d

  \item[] d

  \item[] d
\end{itemize} 
\end{block}

  \begin{definition}
  d
  \end{definition}


\end{itemize}
\end{frame}

\fi 