% !TeX root = ./6-handout.tex

\setcounter{section}{5}

\section{Proofs in SL}

\begin{frame}
%\large

\scriptsize{\tableofcontents}

\end{frame}

\subsection{Who ordered \textit{that}???}

\begin{frame}
\frametitle{The Power of a Proof System}
%\large

\begin{itemize}[<+->]

\item We have already seen how powerful a proof system can be compared to truth tables:

\item Consider our running example argument: 
      \[
      C \eor E, A \eor M, A \eif \enot C, \enot S \eand \enot M \therefore E
      \]
      requires 32 lines and 512 truth values

\item With trees, we showed validity in 10 lines (13 sentences)

\item Why not just tree always and everywhere? 

\end{itemize}
\end{frame}

\begin{frame}
\frametitle{For the Love of Trees}
%\large

\begin{itemize}[<+->]

\item The advantages of trees create limitations:

\begin{itemize}

\item Mechanical $\Rightarrow$ not `normal' pattern of inference

\item Mirror partial truth-tables $\Rightarrow$ implicitly referencing truth-values

\item Always checking satisfiability of the root $\Rightarrow$ we don't really `derive' anything at the end of the proof, unlike in normal inference 

\item (this is why so many of us keep forgetting that it is the NEGATION of the conclusion that goes in the root)

\end{itemize}

\item We would like to form a better model of human inference patterns
% % Key point: we do not ordinarily reason in terms of surveying truth value assignments, nor in talking about satisfiability, or open or closed branches.

\item e.g., common rules such as Modus Ponens, disjunctive syllogism

\end{itemize}
\end{frame}

\begin{frame}
\frametitle{Idiosyncrasies of Table or Tree Reasoning}
%\large

\begin{itemize}[<+->]

\item Tables:  

\bi 
\item Construct a truth table

\item verify there is no TVA where premises are true but conclusion is false

\ei 

\item This is NOT how we typically reason through an argument

\item Trees: ask whether a set of sentences is satisfiable:

\bi
\item Put premises and NEGATION of conclusion in root

\item If tree closes, then unsatisfiable root (valid argument)

\item If tree remains open, then satisfiable root (invalid argument)
\ei

\item Again, this is NOT how we typically reason through an argument 

\end{itemize}
\end{frame}

\begin{frame}
\frametitle{The Very Idea of `Natural Deduction'}
%\large

\begin{itemize}[<+->]

\item We commonly reason according to certain inference rules

\item And we often make assumptions in the \textit{middle} of our reasoning, derive an intermediate conclusion, and `discharge' the assumption (e.g. in proof by contradiction)

\item We would like to see if we can \textit{vindicate} these patterns:

\begin{itemize}

\item Show that these rules never get us into trouble: soundness

\item Show that we have enough rules to handle any valid argument (including additional rules we might want to add): completeness   

\end{itemize}

\item Perhaps our natural deduction system \textit{explains} the success of our ordinary inference patterns

\end{itemize}
\end{frame}


\begin{frame}
  \frametitle{Ordinary Reasoning by ``Proofs"}

  \begin{itemize}[<+->]
    \item Idea: work our way from premises to conclusion using steps
    we know are entailed by the premises.
    \item For instance:
      \begin{itemize}[<+->]
        \item From ``Neither Sarah nor Amir enjoys hiking'' we can
        conclude ``Amir doesn't enjoy hiking.''
        \item From ``Either Amir lives in Chicago or he enjoys hiking''
        and ``Amir doesn't enjoy hiking'' we can conclude ``Amir
        lives in Chicago'' (Disjunctive syllogism DS).
        \item etc.
      \end{itemize}
    \item If we manage to work from the premises to the conclusion in
    this way, we know that the argument must be valid.
  \end{itemize}

\end{frame}

\begin{frame}
  \frametitle{An informal proof}

  \begin{block}{Our argument}
  1. Sarah lives in Chicago or Erie.\\
  2. Amir lives in Chicago unless he enjoys hiking.\\
  3. If Amir lives in Chicago, Sarah doesn't.\\
  4. Neither Sarah nor Amir enjoy hiking.\\
  $\therefore$ Sarah lives in Erie.
  \end{block}

  \begin{enumerate}[<+->]
    \item[5.] Amir doesn't enjoy hiking (from 4).
    \item[6.] Amir lives in Chicago (from 2 and 5).
    \item[7.] Sarah doesn't live in Chicago (from 3 and 6).
    \item[8.] Sarah lives in Erie (from 1 and 7).
  \end{enumerate}

\end{frame}

\begin{frame}
  \frametitle{A more formal proof}

  \begin{block}{Our argument}
    1. \alert<4>{$C \eor E$}\\
    2. \alert<2>{$A \eor M$}\\
    3. \alert<3>{$A \eif \enot C$}\\
    4. \alert<1>{$\enot S \eand \enot M$} \\
  $\therefore$ $E$
  \end{block}

  \begin{enumerate}[<+->]
    \item[5.] \alert<1,2>{$\enot M$} (from 4, since $\metav{P} \eand \metav{Q}
    \entails \metav{Q}$)
    \item[6.] \alert<2,3>{$A$} (from 2 and 5, since $\{ \metav{P} \eor \metav{Q}, \enot \metav{Q} \} \entails \metav{P}$)
    \item[7.] \alert<3,4>{$\enot C$} (from 3 and 6, since $\{ \metav{P} \eif \metav{Q}, \metav{P} \} \entails \metav{Q}$, i.e. via `modus ponens')
    \item[8.] \alert<4>{$E$} (from 1 and 7, since $\{ \metav{P} \eor \metav{Q}, \enot \metav{P} \}    \entails \metav{Q}$)
  \end{enumerate}

\end{frame}

\begin{frame}
  \frametitle{Some aspects of our Formal Deductions}

  \begin{itemize}[<+->]
    \item Numbered lines contain sentences of SL
    \item A line may be a \emph{premise} (:PR).
\item A line may be an \emph{assumption} (:AS)
    \item If neither a premise nor assumption, it must be \emph{justified} %\textit{by a rule}
    \item Justification requires:
\medskip
      \begin{itemize}
        \item a \emph{rule} (e.g. `\&E'), and
        \item prior line(s) invoked by the rule---referenced by line number(s)
       \item starting with a colon: e.g. `: 2 \&E'
      \end{itemize}
\medskip
    \item But: what are the rules? (very different from `what IS a rule'?)
  \end{itemize}
\end{frame}


\begin{frame}
  \frametitle{Aspects of our Rules for Natural Deduction}

  \begin{itemize}[<+->]
    \item Our Rules will (mostly) be \dots
      \begin{itemize}[<+->]
        \item \emph{Simple}: cite just a few lines as justification
        \item \emph{Obvious}: new line should clearly be entailed by justifications
        \item \emph{Schematic}: can be described just by \emph{forms} of
        sentences involved
        \item \emph{Few in number}: want to make do with just a handful
      \end{itemize}
    \item We'll have two rules per connective: \\ an \emph{introduction}
    and an \emph{elimination} rule
    \item They'll be used to either:
    \begin{itemize}[<+->]
      \item justify (say) $\metav{P} \eand \metav{Q}$ (i.e. to `introduce' \eand), or 
    \item justify something \emph{using} $\metav{P}
    \eand \metav{Q}$ (i.e. to `eliminate' \eand).
    \end{itemize}
  \end{itemize}
\end{frame}

\subsection{Conjunction Intro and Elimination (Rules for \eand)}


\begin{frame}
  \frametitle{Eliminating \eand}

  \begin{itemize}[<+->]
    \item What can we \emph{justify using} $\metav{P} \eand \metav{Q}$?
    \item A conjunction entails each conjunct:
    \begin{align*}
      & \metav{P} \eand \metav{Q} \entails \metav{P}\\
      & \metav{P} \eand \metav{Q} \entails \metav{Q}
    \end{align*}
    \item Already used this above to get $\enot M$ from $\enot S \eand
    \enot M$, i.e., from ``Neither
    Sarah nor Amir enjoys hiking'' we concluded\\
 ``Amir doesn't enjoy hiking''.
    \item (Role of $\metav{P}$ played by $\enot S$ and that of $\metav{Q}$
    played by $\enot M$)
  \end{itemize}
\end{frame}

\begin{frame}
  \frametitle{Introducing \eand}

  \begin{itemize}[<+->]
    \item What do we \emph{need to justify} $\metav{P} \eand \metav{Q}$?
    \item We need both \metav{P} and \metav{Q}:
    \begin{align*}
      & \{ \metav{P}, \metav{Q} \} \entails \metav{P}\eand \metav{Q}
    \end{align*}
    \item For instance, if we have ``Sarah doesn't enjoy hiking'' and
    also ``Amir doesn't enjoy hiking'', we can conclude\\
    ``Neither
    Sarah nor Amir enjoys hiking''
    \item (Role of $\metav{P}$ played by $\enot S$ and $\metav{Q}$ by $\enot M$: \{ $\enot S, \enot M \} \entails \enot S \eand \enot M$)
  \end{itemize}
\end{frame}

\begin{frame}
  \frametitle{Rules for \eand}
  \begin{columns}
    \begin{column}{3cm}
  \begin{fitchproof}
    \have[m]{a}{\metav{P}}
    \have[n]{b}{\metav{Q}}
    \have[\ ]{c}{\metav{P}\eand\metav{Q}} \ai{a, b}
  \end{fitchproof}
\end{column}
\begin{column}{3cm}
  \begin{fitchproof}
    \have[m]{ab}{\metav{P}\eand\metav{Q}}
    \have[\ ]{a}{\metav{P}} \ae{ab}
  \end{fitchproof}
  \begin{fitchproof}
    \have[m]{ab}{\metav{P}\eand\metav{Q}}
    \have[\ ]{b}{\metav{Q}} \ae{ab}
  \end{fitchproof}
\end{column}
\end{columns}
We'll illustrate using exercises in our \href{https://tinyurl.com/2p82rpv5}{Week 6 Practice Problems on Carnap}.
\end{frame}

\begin{frame}
  \begin{fitchproof}
    \hypo{1}{A \eand B} \pr{}
    \have{2}{A}\ae{1}
    \have{3}{B}\ae{1}
    \have{4}{B \eand A}\ai{2,3}
  \end{fitchproof}
\end{frame}

\begin{frame}
  \begin{fitchproof}
    \hypo{1}{A \eand (B \eand C)} \pr{}
    \have{2}{A}\ae{1}
    \have{3}{B \eand C}\ae{1}
    \have{4}{C}\ae{3}
    \have{5}{A \eand C}\ai{2,4}
  \end{fitchproof}
\end{frame}

\subsection{Conditional Intro and Elim. (Rules for \eif)}

\begin{frame}
  \frametitle{Eliminating \eif}

  \begin{itemize}[<+->]
    \item What can we \emph{justify using} $\metav{P} \eif \metav{Q}$?
    \item We used the conditional ``If Amir lives in Chicago, Sarah
    doesn't'' to justify ``Sarah doesn't live in
    Chicago''.
    \item What is the general rule? What can we justify using
    $\metav{P} \eif \metav{Q}$? What do we need in addition to $\metav{P} \eif \metav{Q}$?
    \item The principle is \emph{modus ponens} (affirming the antecedent):
    \[ \{ \metav{P} \eif \metav{Q}, \metav{P} \} \entails \metav{Q}\]
    \item (When inferring from $A \eif \enot C$ and $A$ to $\enot C$, the role
    of \metav{P} is played by $A$ and role of \metav{Q} by $\enot C$.)
  \end{itemize}
\end{frame}

\begin{frame}
  \frametitle{Elimination rule for \eif}
  \begin{fitchproof}
    \have[m]{a}{\metav{P} \eif \metav{Q}}
    \have[n]{b}{\metav{P}}
    \have[\ ]{c}{\metav{Q}} \ce{a, b}
  \end{fitchproof}

  Let's illustrate this rule using an exercise
 \href{https://tinyurl.com/2p82rpv5}{in Carnap}: \\ we
  show that $\{A \eand B, A \eif C, B \eif D \} \entails C \eand D$.
\end{frame}

\begin{frame}
  \begin{fitchproof}
    \hypo{1}{A \eand B} \pr{}
    \hypo{2}{A \eif C} \pr{}
    \hypo{3}{B \eif D} \pr{}
    \have{4}{A}\ae{1}
    \have{5}{C}\ce{2,4}
    \have{6}{B}\ae{1}
    \have{7}{D}\ce{3,6}
    \have{8}{C \eand D}\ai{5,7}
  \end{fitchproof}
\end{frame}

\begin{frame}
  \frametitle{Introducing \eif}

  \begin{itemize}[<+->]
  \item How do we justify a conditional? What should we require for a proof
  of $\metav{P} \eif \metav{Q}$ (say, from some premise $\metav{R}$)?

  \item We need a proof that shows that $\metav{R} \entails \metav{P} \eif
  \metav{Q}$.

  \item Idea: show instead that $\metav{R}, \metav{P} \entails \metav{Q}$.

  \item The conditional $\eif$ no longer appears, so this seems easier.

  \item It's a good move, because if $\metav{R}, \metav{P} \entails
  \metav{Q}$
  then $\metav{R} \entails \metav{P} \eif
  \metav{Q}$.
  \end{itemize}
\end{frame}

\begin{frame}
  \frametitle{Justifying $\eif$I}
  \begin{block}{Fact}
  If $\metav{R}, \metav{P} \entails
  \metav{Q}$
  then
  $\metav{R} \entails \metav{P} \eif
  \metav{Q}$.
  \end{block}

  \begin{itemize}[<+->]
    \item If $\metav{R}, \metav{P} \entails
    \metav{Q}$ then every TVA makes $\metav{R}$ or
   $\metav{P}$ false or it makes $\metav{Q}$ true
    \item Let's show that no valuation is a counterexample to $\metav{R} \entails \metav{P} \eif
    \metav{Q}$:
    \begin{enumerate}
      \item A valuation that makes $\metav{R}$ and
    $\metav{P}$ true, but $\metav{Q}$ false, is impossible if  $\metav{R}, \metav{P} \entails
    \metav{Q}$.
    \item So any valuation must make $\metav{R}$ false, $\metav{P}$
    false, or $\metav{Q}$ true.
    \item If it makes $\metav{R}$ false, it's not a counterexample to $\metav{R} \entails \metav{P} \eif
    \metav{Q}$.
    \item If it makes $\metav{P}$ false, it makes $\metav{P} \eif
    \metav{Q}$ true, so it's not a counterexample.
    \item If it makes $\metav{Q}$ true, it also makes $\metav{P} \eif
    \metav{Q}$ true, so it's not a counterexample.
    \end{enumerate}
    \item So, there are no counterexamples to $\metav{R} \entails \metav{P} \eif
    \metav{Q}$.
\item If $\metav{R} = \varnothing$, then from $\metav{P} \entails \metav{Q}$ we can infer $ \entails \metav{P} \eif \metav{Q}$
  \end{itemize}
\end{frame}

\begin{frame}
  \frametitle{Subproofs (CRUCIAL CONCEPT)}

  \begin{itemize}[<+->]
    \item We want to justify $\metav{P} \eif \metav{Q}$ by giving a
    proof of $\metav{Q}$ from \textit{assumption} $\metav{P}$ (and possibly other premises $\Gamma$, e.g. $\metav{R}$)
    \item How to do this in a proof? We can add something as a
    premise and \textit{discharge} it later!
    \item Solution: add $\metav{P}$ as an assumption (:AS), and keep track
    of what depends on that assumption (say, by indenting and a vertical
    line)
    \item Once we're done (have proved \metav{Q}), close this ``subproof''.
    \item Justification of $\metav{P} \eif \metav{Q}$ is the \emph{entire} subproof (use a HYPHEN)
    \item \emph{Important}: nothing \emph{inside} a subproof is available
    outside as a justification (since inner lines might depend on the assumption)
  \end{itemize}
\end{frame}

\begin{frame}
  \frametitle{Introduction rule for \eif}
  \begin{fitchproof}
    \open
    \hypo[m]{a}{\metav{P}} \as{for \eif I}
    \ellipsesline
    \have[n]{b}{\metav{Q}}
    \close
    \have[\ ]{c}{\metav{P} \eif \metav{Q}} \ci{a-b}
  \end{fitchproof}

NOTE THE \emph{HYPHEN} IN THE JUSTIFICATION LINE!!!

  We'll illustrate using more exercises from \href{https://tinyurl.com/2p82rpv5}{Week 6 Practice Problems}
  \begin{itemize}
  \item Show: $\{ A \eif B, B \eif C \} \entails A \eif C$.
  \item Show: $A \eif (B \eif C) \entails (A \eand B) \eif (A \eand C)$
  \end{itemize}
\end{frame}

\begin{frame}
  \begin{fitchproof}
    \hypo{1}{A \eif B} \pr{}
    \hypo{2}{B \eif C} \pr{}
    \open
    \hypo{3}{A} \as{for \eif I}
    \have{4}{B}\ce{1,3}
    \have{5}{C}\ce{2,4}
    \close
    \have{6}{A \eif C}\ci{3-5}
  \end{fitchproof}
\end{frame}

\begin{frame}
  \begin{fitchproof}
    \hypo{1}{A \eif (B \eif C)} \pr{}
    \open
    \hypo{2}{A \eand B} \as{for \eif I}
    \have{3}{A}\ae{2}
    \have{4}{B \eif C}\ce{1,3}
    \have{5}{B}\ae{2}
    \have{6}{C}\ce{4,5}
    \have{7}{A \eand C}\ai{3,6}
    \close
    \have{8}{(A \eand B) \eif (A \eand C)}\ci{2-7}
  \end{fitchproof}
\end{frame}

\subsection{Use of subproofs}

\begin{frame}
  \frametitle{Reiteration (for the 11th hour!)}

  $\metav{P} \entails \metav{P}$, so ``Reiteration'' R is a good rule:

  \begin{fitchproof}
    \have[m]{ab}{\metav{P}}
    \have[k]{a}{\metav{P}} \by{R}{ab}
  \end{fitchproof}

  Uses of reiteration (to the \href{https://tinyurl.com/2p82rpv5}{Carnap!}):

  \begin{itemize}[<+->]
    \item Proof of $A \entails A$.
    \item Proof that $A \eif (B \eif A)$ is a tautology.
  \end{itemize}
\end{frame}

\begin{frame}
  \begin{fitchproof}
    \open
    \hypo{1}{A} \as{for \eif I}
    \have{2}{A}\by{R}{1}
    \close
    \have{3}{A \eif A}\ci{1-2}
  \end{fitchproof}

Again, note the \emph{HYPHEN}! Even though our subproof is only two lines, we still write `:1\textbf{-}2' and NOT `:1, 2'. 
\end{frame}

\begin{frame}
  \begin{fitchproof}
    \open
    \hypo{1}{A} \as{for \eif I}
    \open
    \hypo{2}{B} \as{for \eif I}
    \have{3}{A}\by{R}{1}
    \close
    \have{4}{B \eif A}\ci{2-3}
    \close
    \have{5}{A \eif (B \eif A)}\ci{1-4}
  \end{fitchproof}
\end{frame}

\begin{frame}
  \frametitle{Rules for justifications and subproofs}

  \begin{itemize}[<+->]
    \item When a rule calls for a subproof, we cite it as ``: $m$--$n$",
   hyphenating the first and last line numbers of the subproof.
    \item Sentences on the Assumption line \emph{and} last line MUST match rule
    \item After a subproof is done, you can only cite the whole thing,
    NOT any line in it (you are `outside the scope' of these lines)
    \item Subproofs \footnotesize{(subproofs can be nested)} \normalsize{can be nested}
    \item You also can't cite any subproof entirely
    contained inside another subproof, once the surrounding subproof
    is completed (since again you'd be `outside the scope' of those lines)
  \end{itemize}
\end{frame}

\begin{frame}
  \frametitle{Reiteration (do's a don'ts) DOs and DON'Ts!}
Which are correct applications of R?
  \begin{fitchproof}
    \open
    \hypo{ab}{A} \as{}
    \open
    \hypo{b}{A} \as{}
    \have{a}{A} \by{\uncover<2->{\alert{\checkmark}} R}{ab}
    \close
    \have{d}{A} \by{\uncover<3->{\alert{\checkmark}} R}{ab}
    \have{c}{A} \by{\uncover<4->{\alert{\text{\ding{55}}}} R}{b}
    \close
    \have{e}{A} \by{\uncover<5->{\alert{\text{\ding{55}}}} R}{b}
    \have{f}{A} \by{\uncover<6>{\alert{\text{\ding{55}}}} R}{ab}
  \end{fitchproof}
\end{frame}

\subsection{Disjunction Intro and Elim. (Rules for \eor)}

\begin{frame}
  \frametitle{Introduction rule for \eor}
  We have $\metav{P} \entails \metav{P} \eor \metav{Q}$. So:

\begin{multicols}{2}

  \begin{fitchproof}
    \have[m]{ab}{\metav{P}}
    \have[\ ]{a}{\metav{P}\eor\metav{Q}} \oi{ab}
  \end{fitchproof}
\columnbreak

  \begin{fitchproof}
    \have[m]{ab}{\metav{Q}}
    \have[\ ]{b}{\metav{P}\eor\metav{Q}} \oi{ab}
  \end{fitchproof}

\end{multicols}

\begin{itemize}[<+->]

\item Note that the introduced disjunct can be ANYTHING!

\item And you can introduce on the left OR right side!

\item Let's do practice problem 6.10 on \href{https://tinyurl.com/2p82rpv5}{Carnap!}

\end{itemize}

\end{frame}

\begin{frame}
  \begin{fitchproof}
    \open
    \hypo{1}{A} \as{for \eif I}
    \have{2}{B \eor A}\oi{1}
    \close
    \have{3}{A \eif (B \eor A)}\ci{1-2}
  \end{fitchproof}
\end{frame}

\begin{frame}
  \frametitle{Eliminating $\eor$ (Proof by Cases)}

  \begin{itemize}[<+->]
  \item What can we justify with disjunction $\metav{P} \lor \metav{Q}$?

  \item Not $\metav{P}$ and also not $\metav{Q}$: neither is entailed by
  $\metav{P} \eor \metav{Q}$.

  \item But: if both $\metav{P}$ and $\metav{Q}$ separately entail some
  third sentence $\metav{R}$, then we know that $\metav{R}$ follows from the disjunction!

  \item To show this, we need \emph{two} subproofs that show $\metav{R}$, but
  in each proof we are allowed to use only one of $\metav{P}$, $\metav{Q}$.
  \end{itemize}
\end{frame}

\begin{frame}\footnotesize
  \frametitle{Elimination rule for $\eor$ (Proof by Cases)}

\begin{multicols}{2}

\begin{center}
  \begin{fitchproof}
    \have[m]{o}{\metav{P} \eor \metav{Q}}
    \open
    \hypo[i]{a}{\metav{P}} \as{for \eor E}
    \ellipsesline
    \have[j]{b}{\metav{R}}
    \close
\breakline
    \open
    \hypo[k]{aa}{\metav{Q}} \as{for \eor E}
    \ellipsesline
    \have[\ell]{bb}{\metav{R}}
    \close
    \have[\ ]{c}{\metav{R}} \oe{o, a-b, aa-bb}
  \end{fitchproof}
  \end{center}

   \columnbreak
    
\begin{center}
 \begin{itemize}[<+->]

\item From \metav{P} we derive \metav{R}

\item Start a subproof for each disjunct

\item The subproofs need not be adjacent, but if they are, \emph{separate with -{}-}

\item From \metav{Q} we derive \metav{R}

\item You can swap the order of the subproofs

\item Remember to cite BOTH subproofs (hyphens!), AND the line with the disjunction

\item Remember to pop out of subproof level at the end!

\end{itemize}

% \ellipsesline
  \end{center}
  
\end{multicols}
  
\end{frame}


\begin{frame}
  \begin{fitchproof}
    \hypo{1}{A \eor B} \pr{}
    \open
    \hypo{2}{A} \as{for \eor E}
    \have{3}{B \eor A}\oi{2}
    \close
%\breakline%makes the line numbers go super wonky 
    \open
    \hypo{4}{B} \as{for \eor E}
    \have{5}{B \eor A}\oi{4}
    \close
    \have{6}{B \eor A}\oe{1,2-3,4-5}
  \end{fitchproof}

  \begin{itemize}[<+->]
  
  \item In Carnap: Need -{}- between the subproofs

\item Note: need the \emph{SAME sentence} as the last line of each subproof

\item Note the complex justification structure: (a) line with disjunction, (b) first subproof, (c) second subproof, (d) the rule itself 

\item Proceed to Carnap PP6.15!

  \end{itemize}
\end{frame}

\begin{frame}
  \begin{fitchproof}
    \hypo{1}{A \eor B} \pr{}
    \hypo{2}{A \eif B} \pr{}
    \open
    \hypo{3}{A} \as{for \eor E}
    \have{4}{B}\ce{2,3}
    \close
%\breakline%makes the line numbers go super wonky 
    \open
    \hypo{5}{B} \as{for \eor E}
    \have{6}{B}\by{R}{5}
    \close
    \have{7}{B}\oe{1,3-4,5-6}
  \end{fitchproof}
\end{frame}


\subsection{Negation Intro and Elimination} 

\begin{frame}
  \frametitle{Introducing $\enot$}

  \begin{itemize}[<+->]
    \item An argument is valid iff the premises
    together with the negation of the conclusion are jointly
    unsatisfiable.
    \item For instance:
    \begin{itemize}[<+->]
      \item $\metav{Q} \entails \metav{P}$ iff
      $\metav{Q}$ and $\enot\metav{P}$ are jointly unsatisfiable.
      \item $\metav{Q} \entails \enot\metav{P}$ iff
      $\metav{Q}$ and $\metav{P}$ are jointly unsatisfiable.
    \end{itemize}
    \item This last one gives us idea for $\enot$I rule: To justify
    $\enot\metav{P}$, show that $\metav{P}$ (together with all other
    premises) is unsatisfiable.
    \item Unsatisfiable means: a contradiction follows!
  \end{itemize}
\end{frame}

\begin{frame}
  \frametitle{Negation Introduction (\enot I)}
  
%\begin{columns}
  %\begin{column}{3cm}
\begin{multicols}{2}

\begin{center}
%\textit{Negation Intro} (\enot I) \vspace{0.25em}
    \begin{fitchproof}
    \open
	\hypo[m]{na}\metaA{} \as{for \enot I}
	 \ellipsesline
	\have[n]{b}\metaB{}
	 \ellipsesline
	\have[o]{nb}{\enot\metaB{}}
\close
\have[\ ]{a}[\ ]{\enot\metaA{}}\ni{na-nb}
    \end{fitchproof}
    \end{center}
   % \end{column}
   
   \columnbreak
    
%  \begin{column}{3cm}
\begin{center}

 \begin{itemize}[<+->]

\item Assume the \emph{non}-negated wff!

\item Derive a sentence and its negation (could be \metaA{}!)

\item (\metaB{} and \enot\metaB{} can appear in opposite order)

\item Pop out of the subproof and \emph{introduce} that negativity! 

\item Remember to cite the WHOLE subproof (hyphen!)

\item Let's try \href{https://tinyurl.com/2p82rpv5}{exercise PP6.21}:

\end{itemize}


  \end{center}
%\end{columns}
\end{multicols}

%Note that you can swap the order of \metaB{} and \enot\metaB{} in the subproofs! 
\end{frame}

\begin{frame}
  \begin{fitchproof}
    \open
    \hypo{1}{A \eif B} \as{for \eif I}
    \open
    \hypo{2}{\enot B} \as{for \eif I}
    \open
    \hypo{3}{A} \as{for \enot I}
    \have{4}{B}\ce{1,3}
    \have{5}{\enot B}\by{R}{2}
    \close
    \have{6}{\enot A}\ni{3-5}
    \close
    \have{7}{\enot B \eif \enot A}\ci{2-6}
    \close
    \have{8}{(A \eif B) \eif (\enot B \eif \enot A)}\ci{1-7}
  \end{fitchproof}
\end{frame}

\begin{frame}
  \frametitle{Negation Elimination (\enot E)}
  
\begin{multicols}{2}

\begin{center}
    \begin{fitchproof}
\open
	\hypo[m]{na}{\enot\metaA{}}\as{for \enot E}{}
	 \ellipsesline
	\have[n]{b}\metaB{}
	 \ellipsesline
	\have[o]{nb}{\enot\metaB{}}
\close
\have[\ ]{a}[\ ]\metaA{}\ne{na-nb}
\end{fitchproof}
    \end{center}
   % \end{column}
   
   \columnbreak
    
\begin{center}

 \begin{itemize}[<+->]

\item Assume the \emph{negated} wff!

\item Derive a sentence and its negation (could be \metaA{}!)

\item (\metaB{} and \enot\metaB{} can appear in opposite order)

\item Pop out of the subproof and \emph{eliminate} that negativity! 

\item Put a smile on!

\item Remember to cite the WHOLE subproof (hyphen!)

\item Let's try \href{https://tinyurl.com/2p82rpv5}{exercise PP6.22}:

\end{itemize}
  \end{center}
\end{multicols}

%Note that you can swap the order of \metaB{} and \enot\metaB{} in the subproofs! 
\end{frame}

\begin{frame}
  \begin{fitchproof}
    \open
    \hypo{1}{\enot A \eif \enot B} \as{for \eif I}
    \open
    \hypo{2}{B} \as{for \eif I}
    \open
    \hypo{3}{\enot A} \as{for \enot E}
    \have{4}{\enot B}\ce{1,3}
    \have{5}{B}\by{R}{2}
    \close
    \have{6}{A}\by{\enot E}{3-5}
    \close
    \have{7}{B \eif A}\ci{2-6}
    \close
    \have{8}{(\enot A \eif \enot B) \eif (B \eif A)}\ci{1-7}
  \end{fitchproof}
\end{frame}



%******************START BICONDITIONAL RULES**************************



\subsection{Biconditional Intro and Elimination ($<$-$>$)} 

\begin{frame}
  \frametitle{Biconditional  Introduction (\eiff I)    (Type $<$-$>$ !!!)}
  
\begin{multicols}{2}

\begin{center}
%\textit{Biconditional  Intro} (\eiff I) \vspace{0.25em}
    \begin{fitchproof}
\open
		\hypo[i]{a1}{\metav{A}} \as{for \eiff I}
		\ellipsesline
		\have[j]{b1}{\metav{B}}
	\close
\breakline
	\open
		\hypo[k]{b2}{\metav{B}} \as{for \eiff I}
		\ellipsesline
		\have[l]{a2}{\metav{A}}
	\close
	\have[\ ]{ab}{\metav{A}\eiff\metav{B}}\bi{a1-b1,b2-a2}
    \end{fitchproof}
    \end{center}
   
   \columnbreak
    
\begin{center}
 \begin{itemize}[<+->]

\item Like doing conditional intro twice, from both directions

\item You can swap the order of the subproofs

\item The subproofs need not be adjacent, but if they are, \emph{separate with -{}-}

\item Remember to cite BOTH subproofs (hyphens!)

\item Remember to pop out of subproof line!

\end{itemize}

% \ellipsesline
  \end{center}
\end{multicols}

%Note that you can swap the order of \metaB{} and \enot\metaB{} in the subproofs! 
\end{frame}

\begin{frame}
  \frametitle{Biconditional  Elimination (\eiff E)     (Type $<$-$>$ !!!)}
  
\begin{multicols}{2}

\begin{center}
%\textit{Biconditional Elimination} (\enot E) %\vspace{1em}
    \begin{fitchproof}
	\have[m]{ab}{\metav{A}\eiff\metav{B}}
	\have[n]{a}{\metav{A}}
	\have[\ ]{b}{\metav{B}} \be{ab,a}
\end{fitchproof}

\begin{fitchproof}
	\have[m]{ab}{\metav{A}\eiff\metav{B}}
	\have[n]{a}{\metav{B}}
	\have[\ ]{b}{\metav{A}} \be{ab,a}
\end{fitchproof}

    \end{center}
  
   
   \columnbreak
    
\begin{center}

 \begin{itemize}[<+->]

\item Just like conditional elimination! 

\item Only now you can eliminate from either side! (power!)

\item There can be lines between lines m and n

\item Remember to cite the lines of both (i) the biconditional and (ii) the side you have already

\item Carnap issue: must type $<$-$>$E 

\end{itemize}
  \end{center}
\end{multicols}

%Note that you can swap the order of \metaB{} and \enot\metaB{} in the subproofs! 
\end{frame}

\begin{frame}
\frametitle{Issue with Typing $\eiff$ in Carnap}
%\large

\begin{itemize}[<+->]

\item For Carnap to recognize \eiff I or \eiff E in the justification column, you sadly must type $<$-$>$ I or $<$-$>$ E  % vs.  $<${}-{}$>$ E

\item This is a bummer; I hope we can have it fixed (eventually)

\item It is still fine to type $<$$>$ for the biconditional symbol in the sentences

\item You can also copy/paste the \eiff symbol from elsewhere on the page! 


\end{itemize}
\end{frame}


\subsection{Strategies and examples}

\begin{frame}
  \frametitle{Working forward and backward}

  \begin{itemize}[<+->]
    \item \emph{Working backward} from a conclusion (goal) means:
    \begin{itemize}[<+->]
      \item Find main connective of goal sentence
      \item Match with conclusion of corresponding \textbf{I}ntro rule
      \item Write out (above the goal!) what you'd need to apply that rule
    \end{itemize}
    \bigskip
    \item \emph{Working forward} from a premise, assumption, or
    already justified sentence means:
    \begin{itemize}[<+->]
      \item Find main connective of premise, assumption, or sentence
      \item Match with top premise of corresponding E rule
      \item Write out what else you need to apply the E rule (new goals)
      \item If necessary, write out conclusion of the rule
    \end{itemize}
  \end{itemize}
\end{frame}

\begin{frame}
  \frametitle{Constructing a proof}

  \begin{itemize}[<+->]
    \item Write out premises at the top (if there are any)
    \item Write conclusion at bottom
    \item Work backward \& forward from goals and premises/assumptions
    in this order:
    \begin{itemize}[<+->]
      \item Work backward using \eand I, \eif I, \eiff I, \enot I/E, or
      forward using \eor E
      \item Work forward using \eand E
      %\item Work forward from \enot I/E to a contradiction 
      \item Work forward using \eif E, \eiff E
      \item Work backward from \eor I
      \item Try Negation Intro or Elimination, working toward a contradiction
    \end{itemize}
    \item Repeat for each new goal from top
  \end{itemize}
\end{frame}



\subsection{The Rules, Reiterated}

\begin{frame}
  \frametitle{The rules, one more time: Reiteration}
  
\begin{multicols}{2}

    \begin{fitchproof}
      \have[m]{ab}{\metav{P}}
     \ellipsesline
      \have[k]{a}{\metav{P}} \by{R}{ab}
    \end{fitchproof}
  
\columnbreak
  
  
  \bi
  
  \item Remember that you must be in the scope of the line you're reiterating
  
  \item e.g. if you're outside a subproof, you can't reiterate anything wholly within the subproof
  
  \ei
  
\end{multicols}
\end{frame}

\begin{frame}
  \frametitle{The rules: Conjunction Intro (\eand I) and Elimination (\eand E)}
\begin{columns}
  \begin{column}{3cm}
  \begin{fitchproof}
    \have[m]{a}{\metav{P}}
    \have[n]{b}{\metav{Q}}
    \have[\ ]{c}{\metav{P}\eand\metav{Q}} \ai{a, b}
  \end{fitchproof}
\end{column}
\begin{column}{3cm}
  \begin{fitchproof}
    \have[m]{ab}{\metav{P}\eand\metav{Q}}
    \have[\ ]{a}{\metav{P}} \ae{ab}
  \end{fitchproof}
  \begin{fitchproof}
    \have[m]{ab}{\metav{P}\eand\metav{Q}}
    \have[\ ]{b}{\metav{Q}} \ae{ab}
  \end{fitchproof}
\end{column}
\end{columns}
\end{frame}

\begin{frame}
  \frametitle{The rules: Conditional Intro (\eif I) and Elim (\eif E)}
\begin{columns}
  \begin{column}{3cm}
    \begin{fitchproof}
      \open
      \hypo[m]{a}{\metav{P}} \as{for \eif I}
      \ellipsesline
      \have[n]{b}{\metav{Q}}
      \close
      \have[\ ]{c}{\metav{P} \eif \metav{Q}} \ci{a-b}
    \end{fitchproof}
  \end{column}
  \begin{column}{3cm}
    \begin{fitchproof}
      \have[m]{a}{\metav{P} \eif \metav{Q}}
      \have[n]{b}{\metav{P}}
      \have[\ ]{c}{\metav{Q}} \ce{a, b}
    \end{fitchproof}
  \end{column}
\end{columns}
\end{frame}

\begin{frame}\footnotesize
  \frametitle{The rules: Disjunction Intro (\eor I) and Elimination (\eor E)}
\begin{multicols}{2}

  \begin{center}
    \begin{fitchproof}
      \have[m]{o}{\metav{P} \eor \metav{Q}}
      \open
      \hypo[i]{a}{\metav{P}} \as{for \eor E}
      \ellipsesline
      \have[j]{b}{\metav{R}}
      \close
\breakline %adds the -- but makes the proof too long, so add \footnotesize after begin frame 
      \open
      \hypo[k]{aa}{\metav{Q}} \as{for \eor E}
      \ellipsesline
      \have[\ell]{bb}{\metav{R}}
      \close
      \have[\ ]{c}{\metav{R}} \oe{o, a-b, aa-bb}
    \end{fitchproof} 
    \end{center}

\columnbreak
  
  \begin{center}
      \begin{fitchproof}
        \have[m]{ab}{\metav{P}}
        \have[\ ]{a}{\metav{P}\eor\metav{Q}} \oi{ab}
      \end{fitchproof}
      \begin{fitchproof}
        \have[m]{ab}{\metav{Q}}
        \have[\ ]{b}{\metav{P}\eor\metav{Q}} \oi{ab}
      \end{fitchproof}
       \end{center}
       
       Remember that \metav{P} can be the same wff as \metav{Q}
       
       so can introduce $\metav{P} \eor \metav{P}$ from \metav{P}
\end{multicols}
\end{frame}

\begin{frame}
  \frametitle{The rules: Negation Intro and Elimination}
  
%\begin{columns}
  %\begin{column}{3cm}
\begin{multicols}{2}

\begin{center}
\textit{Negation Intro} (\enot I) \vspace{0.25em}
    \begin{fitchproof}
    \open
	\hypo[m]{na}\metaA{} \as{for \enot I}
	 \ellipsesline
	\have[n]{b}\metaB{}
	 \ellipsesline
	\have[o]{nb}{\enot\metaB{}}
\close
\have[\ ]{a}[\ ]{\enot\metaA{}}\ni{na-nb}
    \end{fitchproof}
    \end{center}
   % \end{column}
   
   \columnbreak
    
%  \begin{column}{3cm}
\begin{center}
\textit{Neg. Elimination} (\enot E) %\vspace{1em}
    \begin{fitchproof}
\open
	\hypo[m]{na}{\enot\metaA{}}\as{for \enot E}{}
	 \ellipsesline
	\have[n]{b}\metaB{}
	 \ellipsesline
	\have[o]{nb}{\enot\metaB{}}
\close
\have[\ ]{a}[\ ]\metaA{}\ne{na-nb}
    \end{fitchproof}
    \end{center}
 % \end{column}
%\end{columns}
\end{multicols}

Note that you can swap the order of \metaB{} and \enot\metaB{} in the subproofs! 
\end{frame}


\begin{frame}
  \frametitle{The rules: Biconditional  Intro and Elimination ($<$-$>$)}
  
%\begin{columns}
  %\begin{column}{3cm}
\begin{multicols}{2}

\begin{center}
%\textit{Biconditional  Intro} (\eiff I) \vspace{0.25em}
    \begin{fitchproof}
\open
		\hypo[i]{a1}{\metav{A}} \as{for \eiff I}
		\ellipsesline
		\have[j]{b1}{\metav{B}}
	\close
\breakline
	\open
		\hypo[k]{b2}{\metav{B}} \as{for \eiff I}
		\ellipsesline
		\have[l]{a2}{\metav{A}}
	\close
	\have[\ ]{ab}{\metav{A}\eiff\metav{B}}\bi{a1-b1,b2-a2}
    \end{fitchproof}
    \end{center}
   % \end{column}
   
   \columnbreak
    
%  \begin{column}{3cm}
\begin{center}
\textit{Biconditional Elimination} (\eiff E) %\vspace{1em}
    \begin{fitchproof}
	\have[m]{ab}{\metav{A}\eiff\metav{B}}
	\have[n]{a}{\metav{A}}
	\have[\ ]{b}{\metav{B}} \be{ab,a}
\end{fitchproof}

\begin{fitchproof}
	\have[m]{ab}{\metav{A}\eiff\metav{B}}
	\have[n]{a}{\metav{B}}
	\have[\ ]{b}{\metav{A}} \be{ab,a}
\end{fitchproof}
    \end{center}
 % \end{column}
%\end{columns}
\end{multicols}

Note that you can swap the order of \metaB{} and \enot\metaB{} in the subproofs! 
\end{frame}















%*************************Stuff specific to Calgary Edition****************************
\iffalse

\subsection{Contradictions}

\begin{frame}
  \frametitle{Contradictions}

  \begin{itemize}[<+->]
  \item In proofs, we don't just use the premises of the argument, but also
  sentences we've proved, and sentences we've assumed (for $\eif$I,
  $\eor$E).

  \item Sometimes it happens that assumptions we must make for correct
  applications of these rules are incompatible with the premises.

  \item For instance, to prove the disjunctive syllogism $A \eor B,
  \enot B \entails A$ using \eor E, the assumption $B$ for the second
  case conflicts with the premise $\enot B$.
  \end{itemize}
\end{frame}

\begin{frame}
  \frametitle{Disjunctive syllogism}
  \begin{fitchproof}
    \have{o}{A \lor B}
    \hypo{oa}{\alert{\enot B}}
    \open
    \hypo{a}{A}
    \have{b}{A} \by{R}{a}
    \close
    \open
    \hypo{aa}{\alert{B}}
    \ellipsesline
    \have[k]{bb}{A}
    \close
    \have{c}{A} \oe{o, a-b, aa-bb}
  \end{fitchproof}
\end{frame}

\begin{frame}
  \frametitle{Contradictions: eliminating $\enot$}

  We highlight the situation where inside a subproof we have run into
  a contradictory situation by the symbol \[\alert{\bot}\]

  \begin{fitchproof}
    \have[m]{a}{\enot\metav{P}}
    \have[n]{b}{\metav{P}}
    \have[\ ]{c}{\bot} \ne{a, b}
  \end{fitchproof}

  Since this also eliminates a $\enot$, we'll call it $\enot$E.
\end{frame}

\begin{frame}
  \frametitle{Explosion}

  \begin{itemize}[<+->]
  \item Any argument with jointly unsatisfiable premises is valid.
  \item So whenever we can justify $\bot$ in a proof, we should be able to
  justify \emph{anything}.
  \item ``From a contradiction, anything follows.''

  \begin{fitchproof}
    \have[m]{ab}{\bot}
    \have[k]{a}{\metav{P}} \by{X}{ab}
  \end{fitchproof}
  \end{itemize}
\end{frame}

\begin{frame}
  \frametitle{Disjunctive syllogism}
  \begin{fitchproof}
    \have{o}{A \lor B}
    \hypo{oa}{\alert{\enot B}}
    \open
    \hypo{a}{A}
    \have{b}{A} \by{R}{a}
    \close
    \open
    \hypo{aa}{\alert{B}}
    \have{bot}{\alert{\bot}}\by{\alert{\enot E}}{oa,aa}
    \have{bb}{\alert{A}}\by{\alert{X}}{bot}
    \close
    \have{c}{A} \oe{o, a-b, aa-bb}
  \end{fitchproof}
\end{frame}

\subsection{Introducing \enot}



\begin{frame}
  \frametitle{Introduction rule for \enot}
  \begin{fitchproof}
    \open
    \hypo[m]{a}{\metav{P}}
    \ellipsesline
    \have[n]{b}{\bot}
    \close
    \have[\ ]{c}{\enot\metav{P}} \ni{a-b}
  \end{fitchproof}
\end{frame}



\begin{frame}
  \frametitle{Indirect proof rule}
  \begin{fitchproof}
    \open
    \hypo[m]{a}{\enot\metav{P}}
    \ellipsesline
    \have[n]{b}{\bot}
    \close
    \have[\ ]{c}{\metav{P}} \by{IP}{a-b}
  \end{fitchproof}
\end{frame}

\begin{frame}
  \begin{fitchproof}
    \open
    \hypo{1}{\enot A \eif \enot B}
    \open
    \hypo{2}{B}
    \open
    \hypo{3}{\enot A}
    \have{4}{\enot B}\ce{1,3}
    \have{5}{\ered}\ne{2,4}
    \close
    \have{6}{A}\by{IP}{3-5}
    \close
    \have{7}{B \eif A}\ci{2-6}
    \close
    \have{8}{(\enot A \eif \enot B) \eif (B \eif A)}\ci{1-7}
  \end{fitchproof}
\end{frame}

\begin{frame}
  \frametitle{The rules, one more time: Reiteration}
\begin{columns}
  \begin{column}{3cm}
    \begin{fitchproof}
      \have[m]{ab}{\metav{P}}
      \have[k]{a}{\metav{P}} \by{R}{ab}
    \end{fitchproof}
    \begin{fitchproof}
      \have[m]{ab}{\bot}
      \have[k]{a}{\metav{P}} \by{X}{ab}
    \end{fitchproof}
  \end{column}
  \begin{column}{3cm}
  \begin{fitchproof}
    \open
    \hypo[m]{a}{\enot\metav{P}}
    \ellipsesline
    \have[n]{b}{\bot}
    \close
    \have[\ ]{c}{\metav{P}} \by{IP}{a-b}
  \end{fitchproof}
    \end{column}
\end{columns}
\end{frame}

\begin{frame}\small
  \begin{fitchproof}
    \hypo{1}{\enot(A \eor B)}
    \open
    \hypo{2}{A}
    \have{3}{A \eor B}\oi{2}
    \have{4}{\ered}\ne{1,3}
    \close
    \have{5}{\enot A}\ni{2-4}
    \open
    \hypo{2a}{B}
    \have{3a}{A \eor B}\oi{2a}
    \have{4a}{\ered}\ne{1,3a}
    \close
    \have{5a}{\enot B}\ni{2a-4a}
    \have{8}{\enot A \eand \enot B}\ai{5,5a}
  \end{fitchproof}

  \note[itemize]{    \begin{itemize}

    \item Construct proof strategically
    \item Conclusion is \eand: needs two subgoals above, $\enot A$ and
    $\enot B$.
    \item Tackle $\enot A$ first: main connective is \enot, use \enot I
    \item \enot I needs subproof with assumption $A$ and last line \ered
    \item To prove \ered, work forward from line 1: use \enot E
    \item We have $\enot(A \eor B)$, we also need $A \eor B$
    \item Work backward, ie try to apply \eor I. We'd need $A$ or $B$.
    \item $A$ already there (assumption of subproof)
    \item Fill in all justifications for first handful
    \item Second half the same
  \end{itemize}
  }
\end{frame}

\begin{frame}\scriptsize
  \begin{fitchproof}
    \open
    \hypo{1}{\enot A \eand \enot B}
    \open
    \hypo{2}{A \eor B}
    \open
    \hypo{3}{A}
    \have{4}{\enot A}\ae{1}
    \have{5}{\ered}\ne{4,3}
    \close
    \open
    \hypo{6}{B}
    \have{7}{\enot B}\ae{1}
    \have{8}{\ered}\ne{7,6}
    \close
    \have{9}{\ered}\oe{2,3-5,6-8}
    \close
    \have{10}{\enot(A \eor B)}\ni{2-9}
    \close
    \have{11}{(\enot A \eand \enot B) \eif \enot(A \eor B)}\ci{1-10}
  \end{fitchproof}

  \note[itemize]{    \begin{itemize}
    \item Conclusion is \eif: needs subproof with $\enot A \eand \enot
    B$ as assumption, $\enot(A \eor B)$ as last line
    \item New goal is $\enot(A \eor B)$, use \enot I: needs a subproof with $A \eor B$ as
    assumption and \ered as last line
    \item Want \ered: could try \enot E, but we don't have any sentence $\enot \metav{P}$.
    \item Better to first work forward from $A \eor B$: apply \eor E.
    \item Needs two subproofs, one stating with $A$ the other with $B$
    \item Last line of each must match goal we want to prove (\ered)
    \item Now in subproof, trying to prove \ered from $A$
    \item Now's the time to work forward from $\enot A \eand \enot B$
    to get $\enot A$
    \item That lets us justify \ered using \enot E
    \item Second subproof the same
    \end{itemize}
  }
\end{frame}


\begin{frame}
  \begin{fitchproof}
    \open
    \hypo{1}{\enot(A \eor \enot A)}
    \open
    \hypo{2}{A}
    \have{3}{A \eor \enot A}\oi{2}
    \have{4}{\ered}\ne{1,3}
    \close
    \have{5}{\enot A}\ni{2-4}
    \have{6}{A \eor \enot A}\oi{5}
    \have{7}{\ered}\ne{1, 6}
    \close
    \have{8}{A \eor \enot A}\by{IP}{1-7}
  \end{fitchproof}

    \note[itemize]{\scriptsize    \begin{itemize}
      \item Conclusion is $A \eor \enot A$. Working backward using
      \eor I doesn't work, since neither $A$ nor $\enot A$ is
      tautology, so neither has a proof with no premises
      \item No other strategy applies. Last resort: use IP
      \item Needs subproof, with $\enot(A \eor \enot A)$ as assumption, \ered{} as last line
      \item To prove \ered, work forward using \enot E if possible (ie
      if some $\enot \metav{P}$ is available).  There is: $\enot(A \eor \enot A)$ on line 1
      \item For \enot E we also need $\metav{P}$, i.e., $A \eor \enot
      A$: enter that as new goal
      \item Now we should try working backward using \eor I again! Let's try $\enot A$ as new goal
      \item Work backward using \enot I: needs subproof with
      assumption $A$ and last line \ered
      \item Once agian: when trying to prove \ered, work forward using
      E when possible, ie, again look for $A \eor \enot A$ as goal
      \item That can be justified using the new assumption $A$
    \end{itemize}}
\end{frame}

\fi 
%**********************END OF CALGARY SPECIFIC STUFF********************************










