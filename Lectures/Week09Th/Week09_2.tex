\documentclass[a4paper, 11pt]{article} % Font size (can be 10pt, 11pt or 12pt) and paper size (remove a4paper for US letter paper)
\usepackage[protrusion=true,expansion=true]{microtype} % Better typography
\usepackage{graphicx} % Required for including pictures
\usepackage{wrapfig} % Allows in-line images
\usepackage{enumitem} %%Enables control over enumerate and itemize environments
\usepackage{setspace}
\usepackage{amssymb, amsmath, mathrsfs} %%Math packages
\usepackage{stmaryrd}
\usepackage{mathtools}
\usepackage{multicol} 
\usepackage{mathpazo} % Use the Palatino font
\usepackage[T1]{fontenc} % Required for accented characters
\usepackage{array}
\usepackage{bibentry}
\usepackage{prooftrees} 
\usepackage[round]{natbib} %%Or change 'round' to 'square' for square backers
\setcitestyle{aysep=}
% \usepackage{fitchproof} 

% \linespread{1} % Change line spacing here, Palatino benefits from a slight increase by default

\newcommand{\corner}[1]{\ulcorner#1\urcorner} %%Corner quotes
\newcommand{\tuple}[1]{\langle#1\rangle} %%Angle brackets
\newcommand{\set}[1]{\lbrace#1\rbrace} %%Set brackets
\newcommand{\interpret}[1]{\llbracket#1\rrbracket} %%Double brackets
%\DeclarePairedDelimiter\ceil{\lceil}{\rceil}    
\def\therefore{\ensuremath{\ldotp\dot\,\ldotp}}
\newcommand{\I}{\mathcal{I}}
\newcommand{\J}{\mathcal{J}}
\newcommand{\B}{\mathcal{B}}
\newcommand{\F}{\mathcal{F}}
\newcommand{\M}{\mathcal{M}}
\newcommand{\D}{\mathbb{D}}
\renewcommand{\v}[1]{\mathbf{#1}}
\newcommand{\even}{\texttt{Even}}
\newcommand{\comp}{\texttt{Comp}}
\newcommand{\res}{\texttt{Res}}
\newcommand{\simp}{\texttt{Simple}}
\newcommand{\leng}{\texttt{Length}}
\newcommand{\V}[1]{\mathcal{V}_{#1}} %%Corner quotes
\newcommand{\VV}[2]{\mathcal{V}_{#1}^{#2}} %%

\makeatletter
\renewcommand\@biblabel[1]{\textbf{#1.}} % Change the square brackets for each bibliography item from '[1]' to '1.'
\renewcommand{\@listI}{\itemsep=0pt} % Reduce the space between items in the itemize and enumerate environments and the bibliography

\renewcommand{\maketitle}{ % Customize the title - do not edit title and author name here, see the TITLE block below
\begin{flushright} % Right align
{\LARGE\@title} % Increase the font size of the title

\vspace{10pt} % Some vertical space between the title and author name

{\@author} % Author name
\\\@date % Date

\vspace{30pt} % Some vertical space between the author block and abstract
\end{flushright}
}

%----------------------------------------------------------------------------------------
%	TITLE
%----------------------------------------------------------------------------------------

\title{\textbf{Minimal Models and Variable Assignments}} % Subtitle

\author{\textsc{Logic I}\\ \em Benjamin Brast-McKie} % Institution

\date{\today} % Date

%----------------------------------------------------------------------------------------

\begin{document}

\maketitle % Print the title section

\thispagestyle{empty}

%----------------------------------------------------------------------------------------





\section*{QL Models}

\begin{enumerate}
  \item[\it Interpretations:] $\I$ is an QL interpretation over $\D$ \textit{iff} both: 
    \begin{itemize}
      \item $\I(\alpha)\in\D$ for every constant $\alpha$ in QL. 
      \item $\I(\F^n)\subseteq\D^n$ for every $n$-place predicate $\F^n$.
    \end{itemize}
  \item[\it Model:] $\M=\tuple{\D,\I}$ is a model of QL \textit{iff} $\I$ is a QL interpretation over $\D\neq\varnothing$.
\end{enumerate}



\section*{Variable Assignments}

\begin{enumerate}
  \item[\it Assignments:] A variable assignment $\hat{a}(\alpha)\in\D$ for every variable $\alpha$ in QL.
  \item[\it Singular Terms:] We may define the referent of $\alpha$ in $\M=\tuple{\D,\I}$ as follows:\\
    \begin{align*}
      \VV{\I}{\hat{a}}{(\alpha)}=
        \begin{cases}
          \I(\alpha) & \text{if } \alpha \text{ is a constant} \\
          \hat{a}(\alpha) & \text{if } \alpha \text{ is a variable.}
        \end{cases}
    \end{align*}
  \item[\it Variants:] A $\hat{c}$ is an $\alpha$-variant of $\hat{a}$ \textit{iff} $\hat{c}(\beta)=\hat{a}(\beta)$ for all $\beta\neq\alpha$.
\end{enumerate}



\section*{Semantics for QL}

\begin{enumerate}
  \item[($A$)] $\VV{\I}{\hat{a}}(\F^n\alpha_1,\ldots,\alpha_n)=1$ ~\textit{iff}~ $\tuple{\VV{\I}{\hat{a}}{(\alpha_1)},\ldots,\VV{\I}{\hat{a}}{(\alpha_n)}}\in\I(\F^n)$.
  \item[(\hspace{1pt}$\forall$\hspace{1pt})] $\VV{\I}{\hat{a}}(\forall\alpha\varphi)=1$ ~\textit{iff}~ $\VV{\I}{\hat{c}}(\varphi)=1$ for every $\alpha$-variant $\hat{c}$ of $\hat{a}$.
  \item[(\hspace{1pt}$\exists$\hspace{1pt})] $\VV{\I}{\hat{a}}(\exists\alpha\varphi)=1$ ~\textit{iff}~ $\VV{\I}{\hat{c}}(\varphi)=1$ for some $\alpha$-variant $\hat{c}$ of $\hat{a}$.
  \item[($\neg$)] $\VV{\I}{\hat{a}}(\neg\varphi)=1$ ~\textit{iff}~ $\VV{\I}{\hat{a}}(\varphi)\neq 1$.
  \item[($\vee$)] $\VV{\I}{\hat{a}}(\varphi \vee \psi)=1$ ~\textit{iff}~ $\VV{\I}{\hat{a}}(\varphi)=1$ or $\VV{\I}{\hat{a}}(\psi)=1$ (or both).
  \item[($\wedge$)] $\VV{\I}{\hat{a}}(\varphi \wedge \psi)=1$ ~\textit{iff}~ $\VV{\I}{\hat{a}}(\varphi)=1$ and $\VV{\I}{\hat{a}}(\psi)=1$.
  \item[($\supset$)] $\VV{\I}{\hat{a}}(\varphi \supset \psi)=1$ ~\textit{iff}~ $\VV{\I}{\hat{a}}(\varphi)=0$ or $\VV{\I}{\hat{a}}(\psi)=1$ (or both).
  \item[($\equiv$)] $\VV{\I}{\hat{a}}(\varphi \equiv \psi)=1$ ~\textit{iff}~ $\VV{\I}{\hat{a}}(\varphi)=\VV{\I}{\hat{a}}(\psi)$.
    \vspace{.1in}
  \item[\it Truth:] $\VV{\I}{}(\varphi)=1$ ~\textit{iff}~ $\VV{\I}{\hat{a}}(\varphi)=1$ for some $\hat{a}$ where $\varphi$ is a sentence of QL. 
\end{enumerate}






\section*{Assignment Lemmas}

\begin{enumerate}
  \item[\it Lemma 1:] If $\hat{a}(\alpha)=\hat{c}(\alpha)$ for all free variables $\alpha$ in a wff $\varphi$, then $\VV{\I}{\hat{a}}(\varphi)=\VV{\I}{\hat{c}}(\varphi)$.
    \begin{itemize}
      \item Goes by routine induction on complexity.
    \end{itemize}
  \item[\it Lemma 2:] For any sentence $\varphi$: $\VV{\I}{}(\varphi)= 1$ \textit{iff} $\VV{\I}{\hat{a}}(\varphi)= 1$ for every v.a. $\hat{a}$ over $\D$.
    \begin{itemize}
      \item[\it LTR:] Assume $\VV{\I}{}(\varphi)= 1$, so $\VV{\I}{\hat{a}}(\varphi)= 1$ for some v.a. $\hat{c}$ over $\D$ .
      \item Let $\hat{a}$ be any v.a. over $\D$.
      \item Since $\varphi$ has no free variables, $\VV{\I}{\hat{a}}(\varphi)=\VV{\I}{\hat{c}}(\varphi)$ by \textit{Lemma 1}.
      \item So $\VV{\I}{\hat{a}}(\varphi)=1$ for all v.a. $\hat{c}$ over $\D$.
      \item[\it RTL:] Assume $\VV{\I}{\hat{a}}(\varphi)=1$ for all v.a. $\hat{a}$ over $\D$.
      \item Since $\D$ is nonempty, there is some v.a. $\hat{a}$, and so $\VV{\I}{}(\varphi)= 1$. 
      % \item So $\VV{\I}{}(\varphi)=1$.
    \end{itemize}
  \item[\it Lemma 3:] For any sentence $\varphi$: $\VV{\I}{}(\varphi)\neq1$ \textit{iff} $\VV{\I}{\hat{a}}(\varphi)\neq 1$ for some v.a. $\hat{a}$ over $\D$.
\end{enumerate}



\section*{Minimal Models}

\begin{itemize}
  \item[\bf Task 1:] Provide minimal models in which the following are true/false.
  \item Al loves everything, i.e., $\forall xLax$.
    \begin{itemize}
      \item[\it True:] Let $\hat{a}$ be a v.a. over $\D=\set{a}$.
      \item Let $\hat{c}$ be any $x$-variant of $\hat{a}$.
      \item So $\hat{c}(x)=a$ and $\I(a)=a$.
      \item Since $\I(L)=\set{\tuple{a,a}}$, we know $\tuple{\VV{\I}{\hat{c}}(a),\VV{\I}{\hat{c}}(x)}\in\I(L)$.
      \item So $\VV{\I}{\hat{c}}(Lax)=1$, and so $\VV{\I}{\hat{a}}(\forall xLax)=1$.
      \item[\it False:] Let $\D=\set{a}$ and $\I(L)=\varnothing$.
      \item Assume $\VV{\I}{}(\forall xLax)=1$ for contradiction. 
      \item So $\VV{\I}{\hat{a}}(\forall xLax)=1$ for some v.a. $\hat{a}$.
      \item So $\VV{\I}{\hat{a}}(Lax)=1$ since $\hat{a}$ is an $x$-variant of itself.
      \item So $\tuple{\VV{\I}{\hat{a}}(a),\VV{\I}{\hat{a}}(x)}\in\I(L)$, and so $\I(L)\neq\varnothing$.
      % \item[\it False:] Let $\hat{a}$ be a v.a. over $\D=\set{a}$ and $\I(L)=\varnothing$.
      % \item So $\tuple{\VV{\I}{\hat{a}}(a),\VV{\I}{\hat{a}}(x)}\notin\I(L)$.
      % \item So $\VV{\I}{\hat{a}}(Lax)\neq 1$.
      % \item Since $\hat{a}$ is a $x$-variant of itself, $\VV{\I}{\hat{a}}(\forall xLax)\neq 1$.
      % \item So $\VV{\I}{}(\forall xLax)\neq 1$ by \textit{Lemma 2}.
    \end{itemize}
  \item Someone is dancing, i.e., $\exists x(Px \wedge Dx)$.
    \begin{itemize}
      \item[\it True:] Let $\hat{a}$ be a v.a. over $\D=\set{a}$ where $\va{a}(x)=a$.
      \item Since $\I(P)=\I(D)=\set{\tuple{a}}$, we know $\tuple{\VV{\I}{\hat{a}}(x)}\in\I(P)=\I(D)$.
      \item So $\VV{\I}{\hat{a}}(Px)=\VV{\I}{\hat{a}}(Dx)=1$, and so $\VV{\I}{\hat{a}}(Px \wedge Dx)=1$.
      \item Since $\hat{a}$ is a $x$-variant of itself, $\VV{\I}{\hat{a}}(\exists x(Px \wedge Dx))=1$.
      \item Thus $\VV{\I}{}(\exists x(Px \wedge Dx))=1$.
      \item[\it False:] Let $\D=\set{a}$ and $\I(P)=\varnothing$.
      \item Assume $\VV{\I}{}(\exists x(Px \wedge Dx))=1$ for contradiction.
      \item So $\VV{\I}{\hat{a}}(\exists x(Px \wedge Dx))=1$ for some v.a. $\hat{a}$.
      \item So $\VV{\I}{\hat{c}}(Px \wedge Dx)= 1$ for some $x$-variant $\hat{c}$ of $\hat{a}$.
      \item So $\VV{\I}{\hat{c}}(Px)= 1$, and so $\tuple{\VV{\I}{\hat{c}}(x)}\in\I(P)$.
      \item Thus $\I(P)\neq\varnothing$.
    \end{itemize}
  \item No set is a member of itself. \quad \texttt{[contingent]}\\
    $\neg\exists x(Sx \wedge x\in x)$
  \item There is a set with no members. \quad \texttt{[contingent]}\\
    $\exists x(Sx \wedge \forall y(y\notin x))$
  \item Everyone loves someone. \quad \texttt{[contingent]}\\ 
    $\forall x(Px \supset \exists yLxy)$.
  \item The guests that remained were pleased with the party. \quad \texttt{[contingent]}\\  
    $\forall x(Rxp \supset Px)$.
  \item I haven't met a cat that likes Merra. \quad \texttt{[contingent]}\\
    $\neg\exists x(Mbx \wedge Cx \wedge Lmx)$
  \item Kate found a job that she loved. \quad \texttt{[contingent]}\\
    $\exists x(Fkx \wedge Jx \wedge Lkx)$
  \item Everything everything loves loves something. \quad \texttt{[contingent]}\\  
    $\forall x(\forall yLyx \supset \exists zLxz)$.
\end{itemize}





\section*{Quantifier Exchange}

\begin{enumerate}
  \item[$({\neg}{\forall})$] $\neg\forall x \varphi \Dashv\vDash \exists x\neg \varphi$.
    \begin{itemize}
      \item[\it LTR:] Let $\M=\tuple{\D,\I}$ satisfy $\neg\forall x \varphi$.
      \item So $\VV{\I}{\hat{a}}(\neg\forall x \varphi)=1$ for some v.a. $\hat{a}$.
      \item So $\VV{\I}{\hat{a}}(\forall x \varphi)\neq 1$.
      \item So $\VV{\I}{\hat{c}}(\varphi)\neq 1$ for some $x$-variants $\hat{c}$ of $\hat{a}$.
      \item So $\VV{\I}{\hat{c}}(\neg\varphi)=1$ for some $x$-variants $\hat{c}$ of $\hat{a}$.
      \item So $\VV{\I}{\hat{a}}(\exists x\neg\varphi)=1$, and so $\VV{\I}{}(\forall x\neg\varphi)=1$.
    \end{itemize}
  \item[$({\neg}{\exists})$] $\neg\exists x \varphi \Dashv\vDash \forall x\neg \varphi$.
    \begin{itemize}
      \item[\it LTR:] Let $\M=\tuple{\D,\I}$ satisfy $\neg\exists x \varphi$.
      \item So $\VV{\I}{\hat{a}}(\neg\exists x \varphi)=1$ for some v.a. $\hat{a}$.
      \item So $\VV{\I}{\hat{a}}(\exists x \varphi)\neq 1$.
      \item So $\VV{\I}{\hat{c}}(\varphi)\neq 1$ for all $x$-variants $\hat{c}$ of $\hat{a}$.
      \item So $\VV{\I}{\hat{c}}(\neg\varphi)=1$ for all $x$-variants $\hat{c}$ of $\hat{a}$.
      \item So $\VV{\I}{\hat{a}}(\forall x\neg\varphi)=1$, and so $\VV{\I}{}(\forall x\neg\varphi)=1$.
    \end{itemize}
  % \item[$({\forall}{\neg})$] $\forall x\neg \varphi \vDash \neg\exists x \varphi$.
    % \begin{itemize}
    %   \item[\it Proof:] Let $\M=\tuple{\D,\I}$ satisfy $\neg\exists x \varphi$.
    % \end{itemize}
  % \item[$({\exists}{\neg})$] $\exists x\neg \varphi \vDash \neg\forall x \varphi$.
    % \begin{itemize}
    %   \item[\it Proof:] Let $\M=\tuple{\D,\I}$ satisfy $\neg\forall x \varphi$.
    % \end{itemize}
\end{enumerate}





\section*{Arguments}

\begin{enumerate}
  \item[\it Bigger:] Regiment the following argument:
    \begin{itemize}
      \item Whenever something is bigger than another, the latter is not bigger than the former.\\
        $\forall x\forall y(Bxy \supset \neg Byx)$.
      \item[\therefore] Nothing is bigger than itself.\\
        $\neg\exists x Bxx$.
    \end{itemize}
  \item[\it Proof:] Let $\M=\tuple{\D,\I}$ be any model which satisfies the premise. 
    \begin{itemize}
      \item So $\VV{\I}{\hat{a}}(\forall x\forall y(Bxy \supset \neg Byx))=1$ for some v.a. $\hat{a}$.
      \item Assume $\VV{\I}{}(\neg\exists x Bxx)\neq 1$ for contradiction. 
      \item So $\VV{\I}{\hat{a}}(\neg\exists x Bxx)\neq 1$ in particular.
      \item So $\VV{\I}{\hat{a}}(\exists x Bxx)=1$.
      \item So $\VV{\I}{\hat{c}}(Bxx)=1$ for some $x$-variant $\hat{c}$ of $\hat{a}$.
      \item So $\tuple{\VV{\I}{\hat{c}}(x),\VV{\I}{\hat{c}}(x)}\in\I(B)$, and so $\tuple{\hat{c}(x),\hat{c}(x)}\in\I(B)$.
      \item So $\VV{\I}{\hat{c}}(\forall y(Bxy \supset \neg Byx))=1$.
      \item So $\VV{\I}{\hat{e}}(Bxy \supset \neg Byx)=1$ for $y$-variant $\hat{e}$ where $\hat{e}(y)=\hat{c}(x)$.
      \item So $\VV{\I}{\hat{e}}(Bxy)\neq 1$ or $\VV{\I}{\hat{e}}(\neg Byx)=1$.
      \item So $\VV{\I}{\hat{e}}(Bxy)\neq 1$ or $\VV{\I}{\hat{e}}(Byx)\neq 1$.
      \item So $\tuple{\hat{e}(x),\hat{e}(y)}\notin\I(B)$ or $\tuple{\hat{e}(y),\hat{e}(x)}\notin\I(B)$.
      \item So $\tuple{\hat{c}(x),\hat{c}(x)}\notin\I(B)$ or $\tuple{\hat{c}(x),\hat{c}(x)}\notin\I(B)$ since $\hat{e}(x)=\hat{c}(x)$.
      \item So $\tuple{\hat{c}(x),\hat{c}(x)}\notin\I(B)$, contradicting the above.
    \end{itemize}
  \item[\it Love:] Regiment the following argument:
    \begin{itemize}
      \item Cam doesn't love anyone who loves him back.\\
        $\forall x(Lxc \supset \neg Lcx)$.
      \item May loves everyone who loves themselves.\\
        $\forall y(Lyy \supset Lmy)$.
      \item[\therefore] If Cam loves himself, he doesn't love May.\\
        $Lcc \supset \neg Lcm$.
    \end{itemize}
  % \item[\it Valid:] Let $\M=\tuple{\D,\I}$ be any model which satisfies the premises.
  %   \begin{itemize}
  %     \item So $\VV{\I}{\hat{a}}(\forall x(Lxc \supset \neg Lcx))=\VV{\I}{\hat{c}}(\forall y(Lyy \supset Lmy))=1$.
  %     \item Let $\hat{b}$ be a $x$-variant of $\hat{a}$ where $\hat{b}(x)=\I(m)$.
  %     \item Let $\hat{d}$ be a $y$-variant of $\hat{c}$ where $\hat{d}(y)=\I(c)$.
  %     \item So $\VV{\I}{\hat{b}}(Lxc \supset \neg Lcx)=\VV{\I}{\hat{d}}(Lyy \supset Lmy)=1$.
  %     % \item Assume $\VV{\I}{}(Lcc \supset \neg Lcm)\neq 1$ for contradiction.
  %     % \item So $\VV{\I}{\hat{e}}(Lcc \supset \neg Lcm)\neq 1$ for some v.a. $\hat{e}$.
  %     % \item Thus $\VV{\I}{\hat{e}}(Lcc)= 1$ and $\VV{\I}{\hat{e}}(\neg Lcm)\neq 1$, so $\VV{\I}{\hat{e}}(Lcm)=1$.
  %     % \item So $\tuple{\VV{\I}{\hat{e}}(c),\VV{\I}{\hat{e}}(c)}\in\I(L)$ and $\tuple{\VV{\I}{\hat{e}}(c),\VV{\I}{\hat{e}}(m)}\in\I(L)$. 
  %     % \item So $\tuple{\I(c),\I(c)}\in\I(L)$ and $\tuple{\I(c),\I(m)}\in\I(L)$. 
  %     \item From the first, $\VV{\I}{\hat{b}}(Lxc)\neq 1$ or $\VV{\I}{\hat{b}}(Lcx)\neq 1$.
  %     % \item So $\tuple{\VV{\I}{\hat{b}}(x),\VV{\I}{\hat{b}}(c)}\notin\I(L)$ or $\tuple{\VV{\I}{\hat{b}}(c),\VV{\I}{\hat{b}}(x)}\notin\I(L)$. 
  %     \item So $\tuple{\hat{b}(x),\I(c)}\notin\I(L)$ or $\tuple{\I(c),\hat{b}(x)}\notin\I(L)$. 
  %     \item So $\tuple{\I(m),\I(c)}\notin\I(L)$ or $\tuple{\I(c),\I(m)}\notin\I(L)$. 
  %     \item From the latter, $\VV{\I}{\hat{d}}(Lyy)\neq 1$ or $\VV{\I}{\hat{d}}(Lmy)=1$.
  %     \item So $\tuple{\hat{d}(y),\hat{d}(y)}\notin\I(L)$ or $\tuple{\I(m),\hat{d}(y)}\in\I(L)$. 
  %     \item So $\tuple{\I(c),\I(c)}\notin\I(L)$ or $\tuple{\I(m),\I(c)}\in\I(L)$. 
  %     \item Assume $\tuple{\I(c),\I(c)}\in\I(L)$ for discharge. 
  %     \item So $\tuple{\I(m),\I(c)}\in\I(L)$, and so $\tuple{\I(c),\I(m)}\notin\I(L)$. 
  %     \item Thus 
  %   \end{itemize}
  % \item[\it Gunk] Regiment the following argument:
  %   \item Nothing is a part of itself.
  %   \item Whenever one thing is bigger than a second thing, and the second thing is bigger than a third thing, then the first thing is bigger than the third thing. 
  %   \item[\therefore] Whenever something is bigger than a second thing, the second thing is not bigger than the first.
  %   % \item[\therefore] Nothing is bigger 
  \item[\it Taller:] Regiment the following argument:
    \begin{itemize}
      \item If a first is taller than a second who is taller than a third, then the first is taller than the third.\\
        $\forall x\forall y\forall z((Txy \wedge Tyz) \supset Txz)$.
      \item Nothing is taller than itself.\\
        $\neg\exists xTxx$.
      \item[\therefore] \mbox{If a first is taller than a second, the second isn't taller than the first.}\\
        $\forall x\forall y(Txy \supset \neg Tyx)$.
    \end{itemize}
\end{enumerate}







% \section*{Relations}

% \begin{enumerate}
%   % \item[\it Domain:] Let the \textit{domain} $D$ be any set.
%   \item[\it Relation:] A \textit{relation} $R$ on $D$ is any subset of $D^2$.
%   \item[\it Reflexive:] A relation $R$ is \textit{reflexive} on $D$ \textit{iff} $\tuple{x,x}\in R$ for all $x\in D$.
%   \item[\it Non-Reflexive:] A relation $R$ is \textit{non-reflexive} on $D$ \textit{iff} $R$ is not reflexive on $D$.
%   \item[\bf Question 1:] What is it to be \textit{irreflexive}?
%   \item[\it Irreflexive:] A relation $R$ is \textit{irreflexive} on $D$ \textit{iff} $\tuple{x,x}\notin R$ for all $x\in D$.
%   \item[\it Symmetric:] A relation $R$ is \textit{symmetric iff} $\tuple{y,x}\in R$ whenever ${x,y}\in R$.
%   \item[\bf Question 2:] Why don't we need to specify a domain?
%   \item[\bf Question 3:] Why is a relation reflexive or irreflexive with respect to a domain?
%   \item[\it Asymmetric:] A relation $R$ is \textit{asymmetric iff} $\tuple{y,x}\notin R$ whenever $\tuple{x,y}\in R$.
%   \item[\bf Question 4:] What is it to be non-symmetric? How about non-asymmetric?
%   \item[\bf Task 1:] Show that every asymmetric relation is irreflexive.
%   \item[\it Transitive:] A relation $R$ is \textit{transitive iff} $\tuple{x,z}\in R$ whenever $\tuple{x,y},\tuple{y,z}\in R$.
%   \item[\it Intransitive:] A relation $R$ is \textit{intransitive iff} $\tuple{x,z}\notin R$ whenever $\tuple{x,y},\tuple{y,z}\in R$.
%   \item[\bf Question 5:] Is every symmetric transitive relation reflexive? (No: $R=\varnothing$)
%   \item[\bf Task 2:] Show that every transitive irreflexive relation asymmetric?
%   \item[\it Euclidean:] A relation $R$ is \textit{euclidean iff} $\tuple{y,z}\in R$ whenever $\tuple{x,y},\tuple{x,z}\in R$.
%   \item[\bf Task 3:] Show that every transitive symmetric relation is euclidean.
% \end{enumerate}


  % \item Diamonds last forever.




\end{document}

