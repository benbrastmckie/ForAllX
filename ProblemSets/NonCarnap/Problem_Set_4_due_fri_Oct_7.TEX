 \documentclass[12pt]{article}
 
 \usepackage{geometry}
\geometry{verbose,tmargin=1in,bmargin=1in,lmargin=1in,rmargin=1in}
 
 \iffalse
 \textheight     11truein
 \vsize     10.0truein
 \topmargin      .15truein
 \textwidth 6.5truein \columnwidth \textwidth 
 \setlength{\oddsidemargin}{0truein} 
 %\footheight     0.0truein
 \footskip       0.75truein
 \headheight     .25truein
 \headsep        0.25truein
 \fi 

\usepackage{amsmath} %for align* environment and gather*
\usepackage{xref}
 %    \textheight     10.0truein
 \usepackage{graphics}
 \usepackage{pstricks}
% \usepackage{pst-tree}
% \usepackage{pst-node,pst-tree}
 \usepackage{makeidx}
 
 \def\therefore{\ensuremath{\ldotp\dot{}\,\ldotp}}
% disjunction
\def\eor{\ensuremath{\vee}}
% conjunction: 
% {\,^{_{_{_{_{\mbox{\footnotesize\textbullet}}}}}}} gives the dot
\def\eand{\ensuremath{\,\&\,}}
% conditional: \rightarrow gives the right arrow
\def\eif{\ensuremath{\supset}}
% biconditional: \leftrightarrow gives the left and right arrow
\def\eiff{\ensuremath{\equiv}}
% negation: {\sim} gives the swung dash 
%\def\enot{\ensuremath{\neg}}
%\def\enot{\ensuremath{\sim}} %note that \sim is defined as a relation, which leads to spacing issues. adding a \! leads to more spacing issues (piled up double negations).  

\def\enot{\ensuremath{{\sim}}} %redefining as {\sim} treats the tilde as a unary operator, rather than a relation, solving a lot of the spacing issues. 

\let\oldsim\sim %renames any \sim commands as \oldsim. 
\renewcommand{\sim}{{\oldsim}} %redefines \sim as unary operator version of \sim, in case there are any straggling \sim commands in the wild

 \pagestyle{empty}

\begin{document}

\input macs
%\input fitch
\newcommand{\detritus}[1]{}


\thispagestyle{empty}

%**************CREDIT TO GORDON BELOT FOR THESE PROBLEMS (as far as I know)******************
%*********from his 303 problem sets 3 and 4 *************************************

\iffalse
\parindent = 0pt
\hspace*{0.0in}\parbox[t]{2.5in}{
Philosophy 24.241\\[3pt]
Symbolic Logic\\[3pt]
Fall, 2022
}
\fi 

%\bigskip %\bigskip

\iffalse 
\begin{center}
\Large\bf Problem Set 4\\[1ex] 
 Due Fri. {\bf{October 7th}} by 5pm Eastern\\[3ex]
\end{center}
\fi

\begin{center}
\Large Problem Set 4 \large{(24.241 Symbolic Logic)}\\[1ex] 
 Due Fri. \textbf{October 7th} by \textbf{5pm} Eastern\\ \normalsize{\textbf{Please scan and upload to Canvas as a pdf}; feel free to \textit{also} turn in a paper copy to Philosophy Dept on 8th floor Stata Center, Dreyfoos-wing} \\[3ex] 
\end{center}

Question 0: if you worked with up to two classmates, please list their names! 
%Some of these problems draw from the posted Induction and Recursion notes.\\

%For questions 1 and 2, provide good translations of the following arguments into the language of sentential logic. Then, investigate their validity using the tree method (STD)

\begin{enumerate}

\item (i) Schematize the following argument into the language of sentential logic. \\ (ii) Then, investigate its validity using the tree method (STD): 

``If the lawyer did it, then the doctor did not. Therefore, if the doctor did it, then the lawyer did not.''

\begin{itemize}

\item Symbolization Key: B = the lawyer did it; G = the doctor did it  

\end{itemize}


\item (i) Schematize the following argument into the language of sentential logic. \\ (ii) Then, investigate its validity using the tree method (STD): 

``If na\"ive realism is true, then na\"ive realism is false. Therefore, na\"ive realism is false.''

\medskip

\item Show via the tree method that the following is a tautology: 

\makebox[\textwidth]{$\big( ( P \eor Q) \eand (P \eor R) \big) \eif \big (P \eor (Q \eand R) \big ) $}

\medskip

\item Test the following argument for validity using the tree method (STD):
\begin{align*}
 A \eand (B \eor C) \\ 
(\enot C \eor H) \eand  (H \eif \enot H)\\
- - - - - - - - - -\\
\therefore \; \enot B 
\end{align*}

\item Test the following argument for validity using the tree method (STD):
\begin{align*}
 A \eand (B \eif C) \\ 
- - - - - - - - - -\\
\therefore \; (A \eand C) \eor (A \eand \enot B)  
\end{align*}

\medskip

\item Use a tree to check whether the following formula is a tautology. State your conclusion. If the formula is \textit{not} a tautology, then use the tree to find a truth value assignment that makes the formula false: 
 
 \makebox[\textwidth]{$\big(P \eif (Q \eif R ) \big) \eif \big( ( P \eif Q) \eif (P \eif R) \big) $}
 
 
 \iffalse

\item (i) Translate the following argument into the language of sentential logic. (ii) Check its validity using a tree, and state your conclusion. If the argument is invalid, use the tree to find a truth value assignment that makes its premises true and conclusion false.

\begin{quote}
If logic monkeys are hirsute, then logic monkeys are orgulous. And if space dogs are splenetic, then space dogs are bilious. So both if logic monkeys are hirsute then space dogs are bilious, and if space dogs are splenetic then logic monkeys are orgulous. 
\end{quote}

Symbolization Key: H = logic monkeys are hirsute; O = logic monkeys are orgulous; S = space dogs are splenetic; B = space dogs are bilious

\fi 






























\end{enumerate}


\end{document}