\documentclass[12pt]{article}
 
\usepackage{geometry}
\geometry{verbose,tmargin=1in,bmargin=1in,lmargin=1in,rmargin=1in}
\usepackage{multicol}
 
 \iffalse
 \textheight     11truein
 \vsize     10.0truein
 \topmargin      .15truein
 \textwidth 6.5truein \columnwidth \textwidth 
 \setlength{\oddsidemargin}{0truein} 
 %\footheight     0.0truein
 \footskip       0.75truein
 \headheight     .25truein
 \headsep        0.25truein
 \fi 

\usepackage{amsmath} %for align* environment and gather*
\usepackage{xref}
 %    \textheight     10.0truein
 \usepackage{graphics}
 \usepackage{pstricks}
% \usepackage{pst-tree}
% \usepackage{pst-node,pst-tree}
 \usepackage{makeidx}
 

 
 %\usepackage{forallx-ubc-Hunt} %calls local modified style file, but could lead to conflicts. i ought to make my own `common.sty' file for the problem sets. maybe using the same `common' file as the lecture notes? i really ought to just have a single consistent set of style files for the book, problem sets, and lecture notes! 
 
 %\def\therefore{\ensuremath{\ldotp\dot{}\,\ldotp}}
% disjunction
\def\eor{\ensuremath{\vee}}
% conjunction: 
% {\,^{_{_{_{_{\mbox{\footnotesize\textbullet}}}}}}} gives the dot
\def\eand{\ensuremath{\,\&\,}}
% conditional: \rightarrow gives the right arrow
\def\eif{\ensuremath{\supset}}
% biconditional: \leftrightarrow gives the left and right arrow
\def\eiff{\ensuremath{\equiv}}
% negation: {\sim} gives the swung dash 
%\def\enot{\ensuremath{\neg}}
%\def\enot{\ensuremath{\sim}} %note that \sim is defined as a relation, which leads to spacing issues. adding a \! leads to more spacing issues (piled up double negations).  

\def\enot{\ensuremath{{\sim}}} %redefining as {\sim} treats the tilde as a unary operator, rather than a relation, solving a lot of the spacing issues. 

\let\oldsim\sim %renames any \sim commands as \oldsim. 
\renewcommand{\sim}{{\oldsim}} %redefines \sim as unary operator version of \sim, in case there are any straggling \sim commands in the wild

% metalanguage variables: change greek to A and B if you prefer
\def\metaA{\ensuremath{\varPhi}}
\def\metaB{\ensuremath{\varPsi}}
\def\metaC{\ensuremath{\varOmega}}
\def\metaD{\ensuremath{\varDelta}}
\def\metaSetX{\ensuremath{\mathcal{X}}}
\def\metaSetY{\ensuremath{\mathcal{Y}}}
\def\metaSetZ{\ensuremath{\mathcal{Z}}}

%Calgary script and metav commands: 
\newcommand*{\script}[1]{\ensuremath{\mathcal{#1}}}
\newcommand*{\metav}[1]{\ensuremath{\mathcal{#1}}}

% \pagestyle{empty}
 
 % Tree stuff
 
  \usepackage{prooftrees} %i copied over prooftrees file from Ichikawa source files, which I think is pre-2019 version
  
 %Note that I probably ought to just update my prooftrees package, since I downloaded the zip file and can just paste over the older version! (but who knows what else this could change...)
%JRH: adding in definition of line no override, local option included in 2019 revision. since my prooftrees package is not up to date! 
% see code here: https://tex.stackexchange.com/questions/415976/manually-set-line-numbers-if-prooftrees-sty
% see p. 24 of prooftrees manual for directions on using this. works w/ {}, e.g. line no override={n+1}

%the command `vdotsline' lets you put anything in number column, without a period appearing afterwards. so it's like `line no override' without \linenumberstyle
%e.g. for vertical dots vertically aligned, use: vdotsline={\\[-0.55em] \vdots}

\forestset{
  line no override/.style={
    before drawing tree={
      for name/.process={Ow}{proof tree proof line no}{line no ##1}{
        content=\linenumberstyle{#1},
        typeset node,
      },
    },
  },
  no line no/.style={
    before drawing tree={
      for name/.process={Ow}{proof tree proof line no}{line no ##1}{
        content=,
        typeset node,
      },
    },
  },
  vdotsline/.style={
    before drawing tree={
      for name/.process={Ow}{proof tree proof line no}{line no ##1}{
        content=#1,
        typeset node,
      },
    },
  },
  default preamble={
	single branches,
	close with=\ensuremath{\times},
	just sep=1.75em,
	line no sep=1.75em
	}
}

\begin{document}

\input macs
%\input fitch
\newcommand{\detritus}[1]{}


\thispagestyle{empty}



\iffalse
\parindent = 0pt
\hspace*{0.0in}\parbox[t]{2.5in}{
Philosophy 24.241\\[3pt]
Symbolic Logic\\[3pt]
Fall, 2022
}
\fi 

%\bigskip %\bigskip

\iffalse 
\begin{center}
\Large\bf Problem Set 5 \large{(24.241 Symbolic Logic)}\\[1ex] 
 Due Fri. {\bf{October 14th}} by 5pm Eastern\\[3ex]
\end{center}
\fi

\begin{center}
\Large Problem Set 5\\[1ex] 
 Due Fri. October 13 by 5pm Eastern\\ 
  \vspace{.1in}
  \normalsize{(Please scan and upload to Canvas as a pdf)} \\[3ex] 
\end{center}


Question 0: if you worked with up to two classmates, please list their names! 
%Some of these problems draw from the posted Induction and Recursion notes.\\

%For questions 1 and 2, provide good translations of the following arguments into the language of sentential logic. Then, investigate their validity using the tree method (STD)

\begin{enumerate}

%%idea: two mandatory problems, then two sets where they choose one! 

\item 
  Consider an argument with premises $\varphi$ and $\psi$ and conclusion $\chi$.
  Suppose that you are able to construct a closed tree for this argument even though you make no use of $\varphi$ and $\psi$, i.e., you do not resolve these sentences, and when you close a branch, it is never because it contains $\varphi$ or because it contains $\psi$.
  Assuming soundness and completeness, what is the most informative semantic property you can ascribe to $\chi$? 

%What is the most informative thing you can say about $\Theta$? \\ (feel free to use symbols `P', `Q', and conclusion `C' if you don't feel like writing Greek letters or worry about your handwriting!)
%Let $\Phi$, \Psi , and \Theta be wffs 
%From GB 303, HW 4

%solution: C must be a tautology. Whatever we did in the given tree with A, B, and „ C at its root to make all the branches close, we could have done in a tree with just „ C at its root—since by assumption A and B played no role in the further work we did. So there is a tree with just „ C at its root in which all branches close.

\item Assuming soundness and completeness, show that any closed SL tree has an unsatisfiable root.
  
% \item Prove that in a sound and complete tree system, no argument has both a complete open tree and a closed tree.
% (\textit{Hints}: Show that if an argument is tree-invalid, then it is not tree-valid. To do this, apply \textbf{a key fact} coming from our proof of completeness, and then apply the soundness result. Alternatively, you could do a proof by contradiction, using the same results.) 





\item
What follows are two alternative resolution rules.
Consider the SL tree proof systems that result from replacing the corresponding resolution rule in our original system with one of the rules below.
(Thus there are two alternative systems to consider, one for each rule.)


\begin{multicols}{2}

\textit{Conditional Reduction} (R\eif) \vspace{1em}

%\begin{center}
\begin{prooftree}
{line numbering, single branches}
[\varphi\eif\psi{}, line no override={m}
[\vdots, vdotsline={\\[-0.55em] \vdots}, grouped
	[\enot\varphi{} \eand \psi{}, line no override={j}]
]
]
\end{prooftree}
%\end{center}

\columnbreak

\textit{Exhaustive Negated Conditional} (E\enot\eif) \vspace{0.42em}

%\begin{center}
\begin{prooftree}
{line numbering, single branches}
[\enot(\varphi\eif\psi), line no override={m}
[\vdots, vdotsline={\\[-0.55em] \vdots}, grouped
	[\varphi, line no override={j}]
	[\enot\psi]
	[\varphi \eand \enot \psi] 
]
]
\end{prooftree}
%\end{center}

\end{multicols}



\textbf{Question:} 
Are either of the modified tree systems sound?
If so, explain how to extend our soundness proof to a system which includes the rule in question.
Note that you do not need to rewrite the soundness proof in total, but do spell out the key changes to any lemmas used to prove soundness, using those lemmas to outline the soundness proof.
If either of the modified tree systems fail to be sound, give a tree that is a counterexample to the soundness of such a system.

% \textbf{(b) Would the modified tree system be complete?} 
% If so, explain how to extend our inductive completeness proof to a system with this rule.
% If not, give a tree that is a counterexample to the completeness of  STD$^{\ast}$. 

%\end{enumerate}





 
\iffalse

\item (i) Translate the following argument into the language of sentential logic. (ii) Check its validity using a tree, and state your conclusion. If the argument is invalid, use the tree to find a truth value assignment that makes its premises true and conclusion false.

\begin{quote}
If logic monkeys are hirsute, then logic monkeys are orgulous. And if space dogs are splenetic, then space dogs are bilious. So both if logic monkeys are hirsute then space dogs are bilious, and if space dogs are splenetic then logic monkeys are orgulous. 
\end{quote}

Symbolization Key: H = logic monkeys are hirsute; O = logic monkeys are orgulous; S = space dogs are splenetic; B = space dogs are bilious

\fi 






























\end{enumerate}


\end{document}
