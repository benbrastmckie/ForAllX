 \documentclass[12pt]{article}
 
 \usepackage{geometry}
\geometry{verbose,tmargin=1in,bmargin=.8in,lmargin=1in,rmargin=1in}

\usepackage{multicol}
 
 \iffalse
 \textheight     11truein
 \vsize     10.0truein
 \topmargin      .15truein
 \textwidth 6.5truein \columnwidth \textwidth 
 \setlength{\oddsidemargin}{0truein} 
 %\footheight     0.0truein
 \footskip       0.75truein
 \headheight     .25truein
 \headsep        0.25truein
 \fi 

\usepackage{amsmath} %for align* environment and gather*
\usepackage{xref}
 %    \textheight     10.0truein
 \usepackage{graphics}
 \usepackage{pstricks}
% \usepackage{pst-tree}
% \usepackage{pst-node,pst-tree}
 \usepackage{makeidx}
 

 
 %\usepackage{forallx-ubc-Hunt} %calls local modified style file, but could lead to conflicts. i ought to make my own `common.sty' file for the problem sets. maybe using the same `common' file as the lecture notes? i really ought to just have a single consistent set of style files for the book, problem sets, and lecture notes! 
 
 %\def\therefore{\ensuremath{\ldotp\dot{}\,\ldotp}}
% disjunction
\def\eor{\ensuremath{\vee}}
% conjunction: 
% {\,^{_{_{_{_{\mbox{\footnotesize\textbullet}}}}}}} gives the dot
\def\eand{\ensuremath{\,\&\,}}
% conditional: \rightarrow gives the right arrow
\def\eif{\ensuremath{\supset}}
% biconditional: \leftrightarrow gives the left and right arrow
\def\eiff{\ensuremath{\equiv}}
% negation: {\sim} gives the swung dash 
%\def\enot{\ensuremath{\neg}}
%\def\enot{\ensuremath{\sim}} %note that \sim is defined as a relation, which leads to spacing issues. adding a \! leads to more spacing issues (piled up double negations).  

\def\enot{\ensuremath{{\sim}}} %redefining as {\sim} treats the tilde as a unary operator, rather than a relation, solving a lot of the spacing issues. 

\let\oldsim\sim %renames any \sim commands as \oldsim. 
\renewcommand{\sim}{{\oldsim}} %redefines \sim as unary operator version of \sim, in case there are any straggling \sim commands in the wild

% metalanguage variables: change greek to A and B if you prefer
\def\metaA{\ensuremath{\varPhi}}
\def\metaB{\ensuremath{\varPsi}}
\def\metaC{\ensuremath{\varOmega}}
\def\metaD{\ensuremath{\varDelta}}
\def\metaSetX{\ensuremath{\mathcal{X}}}
\def\metaSetY{\ensuremath{\mathcal{Y}}}
\def\metaSetZ{\ensuremath{\mathcal{Z}}}

%Calgary script and metav commands: 
\newcommand*{\script}[1]{\ensuremath{\mathcal{#1}}}
\newcommand*{\metav}[1]{\ensuremath{\mathcal{#1}}}

% \pagestyle{empty}
 
 % Tree stuff
 
  \usepackage{prooftrees} %i copied over prooftrees file from Ichikawa source files, which I think is pre-2019 version
  
 %Note that I probably ought to just update my prooftrees package, since I downloaded the zip file and can just paste over the older version! (but who knows what else this could change...)
%JRH: adding in definition of line no override, local option included in 2019 revision. since my prooftrees package is not up to date! 
% see code here: https://tex.stackexchange.com/questions/415976/manually-set-line-numbers-if-prooftrees-sty
% see p. 24 of prooftrees manual for directions on using this. works w/ {}, e.g. line no override={n+1}

%the command `vdotsline' lets you put anything in number column, without a period appearing afterwards. so it's like `line no override' without \linenumberstyle
%e.g. for vertical dots vertically aligned, use: vdotsline={\\[-0.55em] \vdots}

\forestset{
  line no override/.style={
    before drawing tree={
      for name/.process={Ow}{proof tree proof line no}{line no ##1}{
        content=\linenumberstyle{#1},
        typeset node,
      },
    },
  },
  no line no/.style={
    before drawing tree={
      for name/.process={Ow}{proof tree proof line no}{line no ##1}{
        content=,
        typeset node,
      },
    },
  },
  vdotsline/.style={
    before drawing tree={
      for name/.process={Ow}{proof tree proof line no}{line no ##1}{
        content=#1,
        typeset node,
      },
    },
  },
  default preamble={
	single branches,
	close with=\ensuremath{\times},
	just sep=1.75em,
	line no sep=1.75em
	}
}

\begin{document}

\input macs
%\input fitch
\newcommand{\detritus}[1]{}


\thispagestyle{empty}



\iffalse
\parindent = 0pt
\hspace*{0.0in}\parbox[t]{2.5in}{
Philosophy 24.241\\[3pt]
Symbolic Logic\\[3pt]
Fall, 2022
}
\fi 

%\bigskip %\bigskip

\iffalse 
\begin{center}
\Large\bf Problem Set 12 \large{(24.241 Symbolic Logic)}\\[1ex] 
 Due Saturday {\bf{December 3rd}} by noon Eastern\\[3ex]
\end{center}
\fi

\begin{center}
\Large Problem Set 12 \large{(24.241 Symbolic Logic)} \\[1ex] 
 Due Saturday \textbf{Dec. 3rd} by \textbf{Noon} Eastern\\ \normalsize{\textbf{Please scan and upload to Canvas as a pdf}} %; feel free to \textit{also} turn in a paper copy to Philosophy Dept on 8th floor Stata Center, Dreyfoos-wing} \\[3ex] 
% \textbf{Answer FOUR questions total: 1 and 2, 3 Xor 4, 5 Xor 6 (exclusive or's!)}
\end{center}

Note: the last four questions are really straightforward and can be answered succinctly. \\ The first two require more work.  Unless noted otherwise, let `$\vdash$' stand for `$\vdash_{SND}$' \\%[1ex]

Question 0: if you worked with up to two classmates, please list their names!

\medskip
%Some of these problems draw from the posted Induction and Recursion notes.\\

%For questions 1 and 2, provide good translations of the following arguments into the language of sentential logic. Then, investigate their validity using the tree method (STD)

\begin{enumerate}

%%idea: two mandatory problems, then two sets where they choose one! 

\item In our inductive proof of the soundness of SND, prove the case where the sentence $P_{k+1}$ is justified by Negation Elimination. $[14pts]$ \\ (this is the missing `Case 10' in the \textit{Logic Book}'s proof of soundness, p. 249). \\

%(feel free to use symbols `P', `Q', and conclusion `C' if you don't feel like writing Greek letters or you worry about your Greek handwriting!)

\item Prove missing case (c) of Theorem 6.4.11 on p. 258 of the \textit{Logic Book}, Chapter 6. i.e. If $\Gamma^{\ast}$ is a maximally consistent set in SND and P and Q are two arbitrary wffs of SL, prove that $(P \lor Q) \in \Gamma^{\ast}$ if and only if either $P \in \Gamma^{\ast}$ or $Q \in \Gamma^{\ast}$. $[28pts]$ \\ -- Note that you have to prove \textbf{BOTH directions} of this if-and-only-if statement!  \\ -- In your proof, you will probably appeal to a schematic SND derivation, and you \textbf{MUST PROVIDE} this derivation (see bottom of p. 258 for examples) \\

\item Prove missing case (3) of Theorem 6.4.8 on p. 259-260 of the \textit{Logic Book}. $[20pts]$ i.e. \\ Show that a sentence of the form $Q \lor R$ with k+1-many connectives is true on the truth value assignment $\mathbf{A}^{\ast}$ if and only if the sentence $Q \lor R$ belongs to a given maximally consistent set of sentences $\Gamma^{\ast}$. Where here ``$\mathbf{A}^{\ast}$'' is defined as the TVA that assigns True to every atomic sentence in $\Gamma^{\ast}$ and False to every atomic sentence not in $\Gamma^{\ast}$. \\ (Note that you will ultimately appeal to what you have just shown in the prior problem) \\

\item Let $S$ be a sentence of SL and $\Gamma$ an \textit{infinite} set of SL sentences. Using the completeness and soundness theorems (but \textbf{NOT} compactness), prove the following: \\ if $\Gamma \vDash S$, then there is some {\it{finite}} set $\Delta \subset \Gamma$ such that $\Delta \vDash S$. $[14pts]$\\

%(Hint: Derivations are only finitely many lines long.)

%Say that $S$ is a sentence of {\it{SL}} and $\Gamma$ is a set of sentences of {\it{SL}}, and $\Gamma$ is infinite.\\

%This follows from the {\it{compactness theorem}} 6.4.12, but do not use the compactness theorem in your proof. I want you to show how the statement follows directly from the completeness and soundness theorems. The argument takes just four or five lines. (Hint: Derivations are only finitely many lines long.)\\

\item  Using the soundness theorem, show that you cannot SND-derive two contradictory sentences $R$ and $\enot R$ from just the atomic sentence $B$. \\ (i.e. prove that the set $\{B\}$ is ``consistent in SND''.) $[14pts]$ \\



\item Say that we add a new rule $R^*$ to the rules of SND, to form a larger system SND$^*$, equipped with its own single turnstile $\Gamma \vdash_{SND^*} S$  for ``$S$ is derivable from the set $\Gamma$ using the rules of SND$^*$". 


%For ease of expression, write $\Gamma \vdash_{SD} S$ for ``$S$ is derivable from $\Gamma$ using the rules of $SD$" and $\Gamma \vdash_{SD^*} S$  for ``$S$ is derivable from $\Gamma$ using the rules of $SD^*$".\\

Say that there is a set $\Gamma$ of sentences of SL and a sentence $S$ of SL such that $\Gamma \vdash_{SND^*} S$ but $\Gamma \nvdash_{SND} S$ (i.e. we can derive $S$ from $\Gamma$ in SND$^*$ but not in SND). \\ %$\Gamma \not\vdash_{SND} S$.\\

Prove that SND$^*$ is an unsound system of rules. $[10pts]$ 



% %TO include in a future PS or even PS13 I suppose \item Let $SD_\lor$ be the system that is just like $SD$ except that it lacks the rule of $\lor E$. Indicate why and where the proof of completeness for SD would break down if you were using $SD_\lor$.

%Could also include something like the following on PS 13 i suppose:

\detritus{ OMIT COUNTING BIT  \item Say that you have the language of all the strings in the infinite alphabet $\{a_1, a_2, a_3, \ldots a_n \ldots\}$ (with no empty string). State a correlation that associates a distinct natural number with each string in this language. 

Remark: Here is an obvious first try, modelled on the strategy for the finite alphabet of {\it{PL}}, as used on p. 254 - 255 of the textbook. {\bf{It doesn't work, but it is instructive to see why not.}} The simplest try would be to just let each letter be associated with its subscript, and then string together the associated numbers. That is, associate $a_1 \leftrightarrow 1, a_2 \leftrightarrow 2, a_3 \leftrightarrow 3, a_4, \leftrightarrow 4 \ldots, a_i \leftrightarrow i \ldots$ then string together the associated numbers to correspond to an associated string: $a_{i_1}a_{i_2}a_{i_3}\ldots a_{i_n} \leftrightarrow i_1i_2i_3\ldots i_n$.\\

The reason this strategy doesn't work is that it doesn't associate a unique number to each string of symbols. For example, both $a_1a_1$ and $a_{11}$ get assigned 11 by this scheme. Both $a_3a_2a_6a_{83}$ and $a_{32}a_{683}$ get assigned 32683. There are lots of other examples of different strings getting the same number. Your answer has to avoid this problem. There are a bunch of different ways to do this.\\

Potentially useful fact:  Remember that $n$ is a prime number if it is a natural number greater than 1, and the only numbers that evenly divide it are 1 and $n$ itself. Every natural number has a {\it{unique}} prime power decomposition. That is, $n$ can be written in {\it{exactly one way}} as a product $q_1^{\alpha_1}\cdot q_2^{\alpha_2} \cdot q_3^{\alpha_3} \ldots q_m^{\alpha_m}$ where each $q_i$ is a prime number and each $\alpha_i$ a natural number greater than 0. So for example: $12 = 2^2 \cdot 3$, $1960 = 2^3\cdot 5 \cdot 7^2$ and these are the only ways to decompose 12 and 1960 into products of powers of primes.    END DETRITUS OMIT COUNTING BIT.}


%\newpage


%\item[] \begin{center} Pick \textbf{one} of questions 5 or 6 to answer; do parts (a) AND (b): \end{center}

%\bigskip









 
\iffalse

d

\fi 






























\end{enumerate}


\end{document}